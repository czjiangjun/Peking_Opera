\documentclass[12pt, oneside, AutoFakeBold, a4paper]{article}      % Specifies the document class
%\documentclass[10pt, twoside, a4paper]{article}      % Specifies the document class

%%%%%%%%%%%%%%%%% CJK 中文版面控制  %%%%%%%%%%%%%%%%%%%%%%%%%%%%%%
%\usepackage{CJK} % CTEX-CJK 中文支持                            %
\usepackage{xeCJK} % seperate the english and chinese		 %
\usepackage{CJKutf8} % Texlive 中文支持                         %
\usepackage{CJKnumb} %中文序号                                   %
\usepackage{indentfirst} % 中文段落首行缩进                      %
%\setlength\parindent{22pt}       % 段落起始缩进量               %
\renewcommand{\baselinestretch}{1.2} % 中文行间距调整            %
\setlength{\textwidth}{16cm}                                     %
\setlength{\textheight}{24cm}                                    %
\setlength{\topmargin}{-1cm}                                     %
\setlength{\oddsidemargin}{0.1cm}                                %
\setlength{\evensidemargin}{\oddsidemargin}                      %
%%%%%%%%%%%%%%%%%%%%%%%%%%%%%%%%%%%%%%%%%%%%%%%%%%%%%%%%%%%%%%%%%%

\usepackage{authblk}					 %作者地址和E-mail
\usepackage{amsmath,amsthm,amsfonts,amssymb,bm}          %数学公式
\usepackage{mathrsfs}                                    %英文花体
\usepackage{tikz}					 %绘制平面图形
%\usepackage[dvipdfmx]{movie15_dvipdfmx} %插入视频
\usepackage{xcolor}                                        %使用默认允许使用颜色
%\usepackage{hyperref} 
\usepackage{graphicx}
\usepackage{subfigure}           %图片跨页
\usepackage{animate}		 %插入动画
\usepackage{caption}
\captionsetup{font=footnotesize}

%\usepackage[version=3]{mhchem}		%化学公式
\usepackage{chemformula}
\usepackage{chemfig}		%化学公式

\usepackage{fontspec} % use to set font
\setCJKmainfont{SimSun}
\XeTeXlinebreaklocale "zh"  % Auto linebreak for chinese
\XeTeXlinebreakskip = 0pt plus 1pt % Auto linebreak for chinese

\usepackage{longtable}                                   %使用长表格
\usepackage{multirow}
\usepackage{makecell}		%允许单元格内换行

\usepackage{arydshln}
\newcommand{\adots}{\mathinner{\mkern2mu%
\raisebox{0.1em}{.}\mkern2mu\raisebox{0.4em}{.}%
\mkern2mu\raisebox{0.7em}{.}\mkern1mu}}
%%%%%%%%%%%%%%%%%%%%%%%%%  参考文献引用 %%%%%%%%%%%%%%%%%%%%%%%%%%%
%%尽量使用 BibTeX(含有超链接,数据库的条目URL即可)                %
%%%%%%%%%%%%%%%%%%%%%%%%%%%%%%%%%%%%%%%%%%%%%%%%%%%%%%%%%%%%%%%%%%%

\usepackage[numbers,sort&compress]{natbib} %紧密排列             %
%\usepackage[sectionbib]{chapterbib}        %每章节单独参考文献   %
%\usepackage{footbib}			   %脚注列出参考文献    %
\usepackage{hypernat}                                                                         %
%\usepackage[dvipdfm,bookmarksopen=true,pdfstartview=FitH,CJKbookmarks]{hyperref}              %
\usepackage[bookmarksopen=true,pdfstartview=FitH,CJKbookmarks]{hyperref}              %
\hypersetup{bookmarksnumbered,colorlinks,linkcolor=green,citecolor=blue,urlcolor=red}         %
%参考文献含有超链接引用时需要下列宏包,注意与natbib有冲突        %
%\usepackage[dvipdfm]{hyperref}                                  %
%\usepackage{hypernat}                                           %
\newcommand{\upcite}[1]{\hspace{0ex}\textsuperscript{\cite{#1}}} %
%%%%%%%%%%%%%%%%%%%%%%%%%%%%%%%%%%%%%%%%%%%%%%%%%%%%%%%%%%%%%%%%%%%%%%%%%%%%%%%%%%%%%%%%%%%%%%%
%\AtBeginDvi{\special{pdf:tounicode GBK-EUC-UCS2}} %CTEX用dvipdfmx的话,用该命令可以解决      %
%						   %pdf书签的中文乱码问题		      %
%%%%%%%%%%%%%%%%%%%%%%%%%%%%%%%%%%%%%%%%%%%%%%%%%%%%%%%%%%%%%%%%%%%%%%%%%%%%%%%%%%%%%%%%%%%%%%%

%---------------------------------xeCJK下设置中文字体-----------------------------------------%  
\setCJKfamilyfont{song}{SimSun}                             %宋体 song  
\newcommand{\song}{\CJKfamily{song}}                        % 宋体   (Windows自带simsun.ttf)  
\setCJKfamilyfont{xs}{NSimSun}                              %新宋体 xs  
\newcommand{\xs}{\CJKfamily{xs}}  
\setCJKfamilyfont{fs}{FangSong_GB2312}                      %仿宋2312 fs  
\newcommand{\fs}{\CJKfamily{fs}}                            %仿宋体 (Windows自带simfs.ttf)  
\setCJKfamilyfont{kai}{KaiTi_GB2312}                        %楷体2312  kai  
\newcommand{\kai}{\CJKfamily{kai}}                            
\setCJKfamilyfont{yh}{Microsoft YaHei}                    %微软雅黑 yh  
\newcommand{\yh}{\CJKfamily{yh}}  
\setCJKfamilyfont{hei}{SimHei}                                    %黑体  hei  
\newcommand{\hei}{\CJKfamily{hei}}                          % 黑体   (Windows自带simhei.ttf)  
\setCJKfamilyfont{msunicode}{Arial Unicode MS}            %Arial Unicode MS: msunicode  
\newcommand{\msunicode}{\CJKfamily{msunicode}}  
\setCJKfamilyfont{li}{LiSu}                                            %隶书  li  
\newcommand{\li}{\CJKfamily{li}}  
\setCJKfamilyfont{yy}{YouYuan}                             %幼圆  yy  
\newcommand{\yy}{\CJKfamily{yy}}  
\setCJKfamilyfont{xm}{MingLiU}                                        %细明体  xm  
\newcommand{\xm}{\CJKfamily{xm}}  
\setCJKfamilyfont{xxm}{PMingLiU}                             %新细明体  xxm  
\newcommand{\xxm}{\CJKfamily{xxm}}  
\setCJKfamilyfont{hwsong}{STSong}                            %华文宋体  hwsong  
\newcommand{\hwsong}{\CJKfamily{hwsong}}  
\setCJKfamilyfont{hwzs}{STZhongsong}                        %华文中宋  hwzs  
\newcommand{\hwzs}{\CJKfamily{hwzs}}  
\setCJKfamilyfont{hwfs}{STFangsong}                            %华文仿宋  hwfs  
\newcommand{\hwfs}{\CJKfamily{hwfs}}  
\setCJKfamilyfont{hwxh}{STXihei}                                %华文细黑  hwxh  
\newcommand{\hwxh}{\CJKfamily{hwxh}}  
\setCJKfamilyfont{hwl}{STLiti}                                        %华文隶书  hwl  
\newcommand{\hwl}{\CJKfamily{hwl}}  
\setCJKfamilyfont{hwxw}{STXinwei}                                %华文新魏  hwxw  
\newcommand{\hwxw}{\CJKfamily{hwxw}}  
\setCJKfamilyfont{hwk}{STKaiti}                                    %华文楷体  hwk  
\newcommand{\hwk}{\CJKfamily{hwk}}  
\setCJKfamilyfont{hwxk}{STXingkai}                            %华文行楷  hwxk  
\newcommand{\hwxk}{\CJKfamily{hwxk}}  
\setCJKfamilyfont{hwcy}{STCaiyun}                                 %华文彩云 hwcy  
\newcommand{\hwcy}{\CJKfamily{hwcy}}  
\setCJKfamilyfont{hwhp}{STHupo}                                 %华文琥珀   hwhp  
\newcommand{\hwhp}{\CJKfamily{hwhp}}  
\setCJKfamilyfont{fzsong}{Simsun (Founder Extended)}     %方正宋体超大字符集   fzsong  
\newcommand{\fzsong}{\CJKfamily{fzsong}}  
\setCJKfamilyfont{fzyao}{FZYaoTi}                                    %方正姚体  fzy  
\newcommand{\fzyao}{\CJKfamily{fzyao}}  
\setCJKfamilyfont{fzshu}{FZShuTi}                                    %方正舒体 fzshu  
\newcommand{\fzshu}{\CJKfamily{fzshu}}  
\setCJKfamilyfont{asong}{Adobe Song Std}                        %Adobe 宋体  asong  
\newcommand{\asong}{\CJKfamily{asong}}  
\setCJKfamilyfont{ahei}{Adobe Heiti Std}                            %Adobe 黑体  ahei  
\newcommand{\ahei}{\CJKfamily{ahei}}  
\setCJKfamilyfont{akai}{Adobe Kaiti Std}                            %Adobe 楷体  akai  
\newcommand{\akai}{\CJKfamily{akai}}  
%------------------------------设置字体大小------------------------%  
\newcommand{\chuhao}{\fontsize{42pt}{\baselineskip}\selectfont}     %初号  
\newcommand{\xiaochuhao}{\fontsize{36pt}{\baselineskip}\selectfont} %小初号  
\newcommand{\yihao}{\fontsize{28pt}{\baselineskip}\selectfont}      %一号  
\newcommand{\erhao}{\fontsize{21pt}{\baselineskip}\selectfont}      %二号  
\newcommand{\xiaoerhao}{\fontsize{18pt}{\baselineskip}\selectfont}  %小二号  
\newcommand{\sanhao}{\fontsize{15.75pt}{\baselineskip}\selectfont}  %三号  
\newcommand{\sihao}{\fontsize{14pt}{\baselineskip}\selectfont}%     四号  
\newcommand{\xiaosihao}{\fontsize{12pt}{\baselineskip}\selectfont}  %小四号  
\newcommand{\wuhao}{\fontsize{10.5pt}{\baselineskip}\selectfont}    %五号  
\newcommand{\xiaowuhao}{\fontsize{9pt}{\baselineskip}\selectfont}   %小五号  
\newcommand{\liuhao}{\fontsize{7.875pt}{\baselineskip}\selectfont}  %六号  
\newcommand{\qihao}{\fontsize{5.25pt}{\baselineskip}\selectfont}    %七号  

%%%%%%%%%%%%%%%%%%%%%  % 插图使用位置  %%%%%%%%%%%%%%%%%%%%%%%%%%%
\graphicspath{{Figures/}}                            %
%%%%%%%%%%%%%%%%%%%%%%%%%%%%%%%%%%%%%%%%%%%%%%%%%%%%%%%%%%%%%%%%%%

\usepackage{verbatim}			%Verbatim 宏包重新实现了 Verbatim 环境,并且提供一个命令可以导入一个 ASCII 文件到文档中
%\verbatiminput{filename}

%在beamer里面使用verbatim环境,可以通过在frame的参数里面添加 containsverbatim / fragile来解决,不过 containsverbatim 会导致pause失效
%\begin{frame}[containsverbatim] %也可以用 \begin{frame}[fragile]
%	\begin{verbatim}
%	\usepackage{xcolor}
%	TEST
%	\end{verbatim}
%\end{frame}

%%%%%%%%%%%%%%%%%%%  说戏剧本集特殊设置   %%%%%%%%%%%%%%%%%%%%%%
\setcounter{secnumdepth}{0} % 此处设置
\usepackage[center]{titlesec}
\setcounter{tocdepth}{2}
\newcommand{\spacept}[2]{#1\hspace{#2}~} %
%%%%%%%%%%%%%%%%%%%%%%%%%%%%%%%%%%%%%%%%%%%%%%%%%%%%%%%%%%%%%%%%

%%%%%%%%%%%%%%%%%%%%%%%%%%%%% 用 authblk 包 支持作者和E-mail %%%%%%%%%%%%%%%%%%%%%%%%%%%%%%%%%
%\title{More than one Author with different Affiliations}				     %
\title{\hei\Huge{刘曾复教授说戏剧本集文稿}}
%\author[a]{Author A}									     %
\author[]{}   %
%\author[a]{Author B}									     %
%\author[a]{Author C \thanks{Corresponding author: email@mail.com}}			     %
%\author[a]{Author/通讯作者 C \thanks{Corresponding author: cores-email@mail.com}}     %
%\author[b]{Author D}									     %
%\author[b]{Author/作者 D}									     %
%\author[b]{Author E}									     %
%\affil[a]{Department of Computer Science, \LaTeX\ University}				     %
%\affil[a]{作者单位-1 \authorcr 地址}    %\authorcr表示换行
%\affil[b]{Department of Mechanical Engineering, \LaTeX\ University}			     %
%\affil[b]{作者单位-2}			     %
											     %
%%% 使用 \thanks 定义通讯作者								     %
%%\affil命令后的{}中的内容,如果觉得需要换行的话,换行命令是\authorcr(不是\\)。
%%Email中可以吧相同邮箱的人@前面的内容写在一个{}里,用逗号隔开。注意{和}前面要加\。例如:
%%\affil[*]{单位1, \authorcr Email: \{zuozhe1, zuozhe2\}@yahoo.com, zuozhe3@sina.com}
											     %
\renewcommand*{\Authfont}{\small\rm} % 修改作者的字体与大小				     %
\renewcommand*{\Affilfont}{\small\it} % 修改机构名称的字体与大小			     %
\renewcommand\Authands{ and } % 去掉 and 前的逗号					     %
\renewcommand\Authands{ , } % 将 and 换成逗号					     %
\date{} % 去掉日期									     %
%\date{2020-12-30}									     %
%%%%%%%%%%%%%%%%%%%%%%%%%%%%%%%%%%%%%%%%%%%%%%%%%%%%%%%%%%%%%%%%%%%%%%%%%%%%%%%%%%%%%%%%%%%%%%

%%%%%%%%%%%%%%%%%%%%%  % 页眉-页脚设计  %%%%%%%%%%%%%%%%%%%%%%%%%%%
\usepackage{fancyhdr}           %使用页眉-页脚                   %
%%\renewcommand{\headrulewidth}{3pt} %页眉(单)线宽(默认黑色),设为0可以去页眉线
%\makeatletter % 双线页眉
%\def\headrule{\color{blue}{\if@fancyplain\let\headrulewidth\plainheadrulewidth\fi%
%\hrule\@height 0.5pt \@width\headwidth\vskip1pt %上面线为0.5pt粗
%\hrule\@height 3.0pt\@width\headwidth  %下面3pt粗
%\vskip-2\headrulewidth\vskip-1pt}      %两条线的距离1pt
%  \vspace{6mm}}     %双线与下面正文之间的垂直间距
%\makeatother

%%\renewcommand{\footrulewidth}{3pt} %页脚线宽(默认黑色),设为0可以去页脚线
%\makeatletter % 双线页眉
%\def\footrule{{\color{blue}{\if@fancyplain\let\footrulewidth\plainfootrulewidth\fi%
%\hrule\@height 3.0pt \@width\headwidth}}
%  \vspace{2mm}}
%\makeatother

%\pagestyle{fancy}
\lhead{} %页眉左边位置内容,并加粗 
%\lhead{\bfseries Result} %页眉左边位置内容,并加粗 
%\chead{} % 页眉中间位置内容
%\rhead{\includegraphics[scale=0.20]{Figures/BCC_logo-1.png}}%在此处插入logo.pdf图片 图片靠右
\rhead{} % 页眉中间位置内容
%\lfoot{}  %页脚
%\rule{\temptablewidth}{1pt}
%\cfoot{}
%\rfoot{}
%\fancyfoot[C]{} %去掉页码
%%%%%%%%%%%%%%%%%  % pagestyleR常用格式  %%%%%%%%%%%%%%%%%%%%%%%%%
%% empty 无页眉页脚
%% plain 无页眉,页脚为居中页码
%% headings 页眉为章节标题,无页脚
%% myheadings 页眉内容可自定义,无页脚
%%%%%%%%%%%%%%%%%%%%%%%%%%%%%%%%%%%%%%%%%%%%%%%%%%%%%%%%%%%%%%%%%%

\begin{document}

%\begin{CJK}{UTF8}{gbsn} %针对文字编码为unix %CJK自带的utf-8简体字体有gbsn(宋体)和gkai(楷体)
%\begin{CJK}{GBK}{hei}	%针对文字编码为doc
%\begin{CJK}{GBK}{hei}	 %针对文字编码为doc
%\CJKindent     %在CJK环境中,中文段落起始缩进2个中文字符
%\indent
%
\renewcommand{\abstractname}{\small{\CJKfamily{hei} 摘\quad 要}} %\CJKfamily{hei} 设置中文字体,字号用\big \small来设
\renewcommand{\contentsname}{\centering\CJKfamily{hei} 目~~~录}
\renewcommand{\refname}{\centering\CJKfamily{hei} 主~要~参~考~资~料}
%\renewcommand{\figurename}{\CJKfamily{hei} 图.}
%\renewcommand{\figurename}{{\bf Fig}.}
\renewcommand{\figurename}{}
%\renewcommand{\tablename}{\CJKfamily{hei} 表.}
\renewcommand{\tablename}{{\bf Tab}.}
%\renewcommand{\thesubfigure}{\roman{subfigure}}  \makeatletter %子图标记罗马字母
%\renewcommand{\thesubfigure}{\tiny(\alph{subfigure})}  \makeatletter %子图标记英文字母
%\renewcommand{\thesubfigure}{}  \makeatletter %子图无标记

%将图表的Caption写成 图(表) Num. 格式
%\makeatletter
%\long\def\@makecaption#1#2{%
%  \vskip\abovecaptionskip
%  \sbox\@tempboxa{#1. #2}%
%  \ifdim \wd\@tempboxa >\hsize
%    #1. #2\par
%  \else
%    \global \@minipagefalse
%    \hb@xt@\hsize{\hfil\box\@tempboxa\hfil}%
%  \fi
%  \vskip\belowcaptionskip}
%\makeatother

\newcommand{\keywords}[1]{{\hspace{0pt}\small{\CJKfamily{hei} 关键词:}{\hspace{2ex}{#1}}\bigskip}}

%%%%%%%%%%%%%%%%%%中文字体设置%%%%%%%%%%%%%%%%%%%%%%%%%%%
%默认字体 defalut fonts \TeX 是一种排版工具 \\		%
%{\bfseries 粗体 bold \TeX 是一种排版工具} \\		%
%{\CJKfamily{song}宋体 songti \TeX 是一种排版工具} \\	%
%{\CJKfamily{hei} 黑体 heiti \TeX 是一种排版工具} \\	%
%{\CJKfamily{kai} 楷书 kaishu \TeX 是一种排版工具} \\	%
%{\CJKfamily{fs} 仿宋 fangsong \TeX 是一种排版工具} \\	%
%%%%%%%%%%%%%%%%%%%%%%%%%%%%%%%%%%%%%%%%%%%%%%%%%%%%%%%%%

%\addcontentsline{toc}{section}{Bibliography}

%-------------------------------The Title of The Paper-----------------------------------------%
%\title{标题}
%----------------------------------------------------------------------------------------------%

%----------------------The Authors and the address of The Paper--------------------------------%
%\author{
%作者:
%\small
%Author1, Author2, Author3\footnote{Communication author's E-mail} \\    %Authors' Names	       %
%\small
%(The Address,City Post code)						%Address	       %
%}
%\affil[$\dagger$]{清华大学~材料加工研究所~A213}
%\affil{清华大学~材料加工研究所~A213}
%\date{}					%if necessary					       %
%----------------------------------------------------------------------------------------------%
%%%%%%%%%%%%%%%%%%%%%%%%%%%%%%%%%%%%%%%%%%%%%%%%%%%%%%%%%%%%%%%%%%%%%%%%%%%%%%%%%%%%%%%%%%%%%%%%%%%%%%%%%%%%%%%%%%%%%
\maketitle
\thispagestyle{empty}    % 清空页码                                      %
%\thispagestyle{fancy}   % 首页插入页眉页脚 

%-------------------------------------------------------------------------------The Abstract and the keywords of The Paper----------------------------------------------------------------------------%
%\begin{abstract}
%The content of the abstract
%\end{abstract}

\newpage
\pagestyle{empty}    % 清空页码                                      %
\begin{figure}[h!]
\centering
\vspace{+0.2in}
\includegraphics[height=1.20\textwidth,width=0.82\textwidth,viewport=0 0 360 520,clip]{Liu_Zengfu.jpg}
\caption*{\hei 刘曾复~教授~~(1914.11.9-2012.6.27)}
\label{Liu_Zengfu}
\end{figure}

\newpage
\begin{figure}[h!]
\centering
%\vspace{+0.2in}
\includegraphics[height=1.38\textwidth,width=1.0\textwidth,viewport=0 0 1050 1550,clip]{Liu-Wu.png}
\caption*{\hei 刘曾复~先生~和~吴小如~先生}
\label{Collect_Liu_Wu}
\end{figure}

\newpage
\begin{figure}[h!]
\centering
%\vspace{-10.5pt}
\includegraphics[height=0.60\textwidth,width=1.0\textwidth,viewport=0 0 500 300,clip]{Zhu-Liu.jpg}
\caption*{\hei 朱家溍~先生~和~刘曾复~先生}
\label{Collect_Zhu_Wu}
\end{figure}

\begin{figure}[h!]
\centering
%\vspace{-10.5pt}
\includegraphics[height=0.60\textwidth,width=1.0\textwidth,viewport=0 0 500 300,clip]{Collect_Zhu-Liu-Wu-Wang.jpg}
\caption*{\hei 左起:~刘曾复~先生、朱家溍~先生、吴小如~先生、王金璐~先生~等~合影}
\label{Collect_Liy_Zhu_Wu_Wang}
\end{figure}
%\keywords{Keyword1; Keyword2; Keyword3}

\newpage
\begin{figure}[h!]
\centering
\vspace{-0.6in}
\includegraphics[height=0.68\textwidth,width=0.50\textwidth,viewport=0 0 950 1300,clip]{PekOpe_Liu-1.jpg}
\caption*{\hei 刘曾复~先生~抄录的《取金陵》曹良臣的单词}
\label{Script}
\end{figure}
\vspace{30pt}
\begin{figure}[hbtp!]
\hspace*{-0.5in}
\begin{minipage}[t]{0.53\textwidth}
	\centering
	\includegraphics[height=1.00\textwidth,width=1.20\textwidth,viewport=0 0 1750 1300,clip]{PekOpe_Liu-2.jpg}
	\caption*{\hei \fontsize{8.5pt}{4.0pt}\selectfont{左:~刘曾复~先生~保存的部分说戏录音磁带}}
\end{minipage}
\hspace{0.6in}
\begin{minipage}[t]{0.43\textwidth}
	\centering
	\vspace{-3.7in}
	\includegraphics[height=1.10\textwidth,width=1.70\textwidth,angle=270, viewport=0 0 2750 1950,clip]{PekOpe_Wu-5.jpg}
%	\caption*{\hei \fontsize{8.5pt}{4.0pt}\selectfont{右:吴小如~先生~保存的刘曾复先生的说戏录音磁带}}
	\caption*{\hei \fontsize{8.5pt}{4.0pt}\selectfont{右:吴小如~先生~保存的各类说戏录音磁带}}
\end{minipage}
\label{Records}
\end{figure}



\include{Data/chap-illustration}
%-------------------------------------------------------------------------------The Content of The Paper----------------------------------------------------------------------------------------------%
\newpage
\pagestyle{plain}   % 删除页眉                                        %
\addcontentsline{toc}{subsection}{\CJKfamily{hei} 目~录}
\tableofcontents %% 制作目录(目录是根据标题自动生成的)
%-----------------------------------------------------------------------------------------------------------------------------------------------------------------------------------------------------%

%----------------------------------------------------------------------------------------The Body Of The Paper----------------------------------------------------------------------------------------%
%Introduction

%\section{Introduction}
%导言
\include{Data/chap-introduction}

%\section{正文章节}
%参考文献的引用方式1\upcite{QCQC_2014}
\pagenumbering{arabic}
%\include{Data/chap-01}
%\include{Data/chap-02}
%\addcontentsline{toc}{section}{\hfill[\hei 三国·两晋]\hfill}
\newpage
\chead{三国·两晋} % 页眉中间位置内容
\hypertarget{ux6349ux653eux66f9-ux4e4b-ux9648ux5bab}{%
\subsection{捉放曹 之
陈宫}\label{ux6349ux653eux66f9-ux4e4b-ux9648ux5bab}}

{[}第一场{]}

{[}引子{]}官居县令,与黎民,判断冤情。

(念)头戴乌纱奉行先\protect\hyperlink{fn136}{\textsuperscript{136}},四乡开可\protect\hyperlink{fn137}{\textsuperscript{137}}万民欢。家邑有语呼循吏\protect\hyperlink{fn138}{\textsuperscript{138}},德配汪洋水地天\protect\hyperlink{fn139}{\textsuperscript{139}}。

下官姓陈名宫,字公台。蒙圣恩,身授中牟县令(或:职授中牟县正堂)。(自到任以来,地方宁靖。)前者董太师有钧谕到来,命各府州县,画影图形,捉拿刺客曹操。也曾命班头王申\protect\hyperlink{fn140}{\textsuperscript{140}}(、李顺)等四门严拿(或:严查),未见交差回报(或:未见交签)。今当三、六、九日放告之期(或:今当三、六、九日升堂理事)。

左右,伺候了。\\
罢了。

(捉拿刺客之事如何?)

喜从何来?

有何为证?

呈上来。

左右,看赏。(或:来,看赏。)

愿者何来?

呵呵哈哈哈\ldots{}\ldots{}(笑介)

官升吏赏,理所当然。国家法度自无不行。

吩咐下去,将刺客带上堂来。

【西皮摇板】曹孟德进衙来齐声威吓,胥吏们列两旁虎立山坡。观此人面貌上带定凶恶,见本县不下跪却是为何。

下站可是曹操?

见了本县大胆不跪(或:为何不跪)。

你可知(或:岂不知)王子犯法,与民同罪。

刺杀太师,还说无罪。

虽非亲眼得见,现有董太师钧谕到来,(你)还敢强辩?

帘外之官(或:我在帘外为官),不问朝阁之事。

住口。

【西皮二六】曹孟德休得要谤毁朝阁,董太师他自有治国韬略。扶灵帝虽无功却也无过(或:扶灵帝为都尉并无过错),十常侍乱宫闱扫荡妖魔。收下了吕奉先威震海角,传一令好一似地动山挪。我将你解进京是我份责(或:是我所责),千金赏万户侯加官进爵(或:并非为千金赏加官晋爵)。你好比扑灯蛾自来投火,又好比抢食鱼自投在网罗。你好比平阳虎把路走错,擒虎易放虎难自己揣摩。

【西皮快板】听他言不由我双眉皱锁,这件事好教我无计奈何。既擒住若放他罪归于我,若不放又恐怕惹出风波。左思量右辗转(或:前思量后辗转)无计定夺\protect\hyperlink{fn141}{\textsuperscript{141}},

有了!

【西皮快板】学苏秦仿张仪计上心窝。既擒住放不放全凭于我,就是放也说个情理相合。

【西皮摇板】曹孟德说此话如梦方觉,七品官焉能得辅相朝阁。(倒不如弃县令随他入伙,走天涯奔海角重整山河。)下位去与明公忙松绳索,

【西皮摇板】胥吏们且回避爷有发落。

{[}第二场{]}

【西皮摇板】手挽手与明公二堂里坐,驾光临少奉迎望乞恕过。

适才关前、堂上多有冒犯,明公海涵(或:望乞恕罪。)。

明公今欲(或:明公意欲)何往?

我意欲随同明公,奔走(或:海走)天涯,(会合诸侯,)共图大事(,幸勿见却)。

不妨,老母妻室,现在原郡,料然无事。

明公请至书房(待茶),容我安排(或:待我吩咐他们)。

(曹操 暂时别。)

少刻奉(陪)。来,

吩咐下去,老爷下乡查旱,多则半月,少则十天(或:多则十日,少则七天)。将信印付与佐堂执掌(或:掌管),不可(或:休得)罗唣。

附耳上来。

记下了。

小心把守。

{[}第三场{]}

【西皮散板】路上行人马蹄忙。

【西皮散板】见一老丈坐道旁。

明公,(你我)还是趱路要紧呐。

明公,去得的?

这就不敢。

嗯哼。

冒造宝庄,老丈海涵。

有座。

老丈,

【西皮快板】多蒙老丈美言奖,释放皇家一栋梁。七品的郎官何足讲,同奔原为汉家邦。

前途用过,不必费心(或:不用费心呐)。

(啊)明公,适才(或:方才)老丈提起令尊大人,明公双目落泪,真乃忠孝双全。

明公啊。

【西皮快板】休流泪来免悲伤,忠孝二字天下扬。同心协力除奸党,凌烟阁上把名扬。

这般时候,哪道而去?

家常随便,万勿劳心。(或:前途用过,万勿费心呐。)

呵呵哈哈哈\ldots{}\ldots{}(笑介)

【西皮摇板】老丈亲自沽佳酿,待人礼仪似孟尝。

听见什么?

诶,老丈与令尊大人(有)八拜之交,焉有此事,你何必多疑呀?(或:焉有此意,你不要多疑呀。)

呃,这倒使得。

【西皮散板】言语恍惚实难详。

又听见什么?

老丈临行言道,家中菜蔬俱有,只是缺少美酒(或:缺少好酒)。亲自去往前村沽酒回来,还要把敬你我。你不要见差了。

明白何来?

我看老丈面带厚道,断非贪赏之辈。

(依你之见?)

\textless{}\textbf{叫头}\textgreater{}明公!

待等老丈沽酒回来,问上他三言两语(或:三言五语),他若无言,那时节再动手也还,不,不,不迟呀。

依你之见?

使不得。

唉!

【西皮散板】未必他有此心肠。

【西皮散板】求赏焉有此风光。

(使不得!)

(唉!)

【西皮散板】他一家大小要遭祸殃。

【西皮散板】吓得我三魂七魄茫啊。

【西皮散板】陈宫上前扯衣裳。

明公(,将老丈一家杀死,)你意欲何往?

杀人放火不是你我所为。

【西皮散板】杀人还要火焚房。

哎呀!

【西皮散板】见一捆猪在厨房。

明公,你到底将他一家错杀了。

老丈一片好心,杀猪宰羊,款待你我,岂不是错杀了?

你去看呐。

嘿嘿。

\textless{}\textbf{叫头}\textgreater{}明公!

你将老丈一家杀死,待等老丈沽酒回来,问起情由,你有(或:你是)何言答对?

事到如今,也只好是一走哇。

走哇,

走哇,

走走走!

{[}第四场{]}

啊?!

【西皮快板】背转身来自参详。指望他是定国安邦将,却原来贼是个人面兽心肠。

啊,老丈,不必强留,回家自然明白。

你我后会有期,就此谢谢了。

【反西皮散板】陈宫心中似刀扎。

老丈!

【反西皮散板】多蒙老丈你的美意大,好意反成恶冤家。急忙里难说你我的知心话,

老丈!

【反西皮散板】休怨我陈宫你怨他。

【西皮摇板】他人不走事有差。

有什么言语,(你)饶他一条老命吧。

不要回来。

哎呀!

【西皮散板】陈宫一见咽喉哑,白发老丈染黄沙。一家大小丧剑\textless{}\textbf{哭头}\textgreater{}下,老丈啊,

呀呸!

【西皮散板】再与孟德把话答。

\textless{}\textbf{叫头}\textgreater{}明公!

你将老丈一家杀死,尚且追悔不及,出庄又将老丈剑劈道旁,是何理也?

似你这等(或:这般)行事,岂不怕天下人咒骂于你?

哦!

【西皮慢板】听他言吓得我心惊胆怕,背转身自埋怨我自己做差。我先前指望他宽宏量大,却原来贼是个无义的冤家。马行在夹道内我难以回马,这才是花随水水不能恋花。这时候我只得暂且忍耐在心下,既同行共大事必须要劝解于他。

【西皮二六】休道我言语多必有奸诈,你本是大义人把事作差。吕伯奢与你父相交不假,为什么起疑心杀他的全家。一家人被你杀也就该罢,出庄来你为何把老丈来杀,是何根芽。

【西皮摇板】好言语劝不醒蠢牛木马,把此贼好一比井底之蛙。

但凭于你。

{[}第五场\protect\hyperlink{fn142}{\textsuperscript{142}}{]}

马不要下鞍。

明灯一盏。

鞍马劳顿,吞吃不下呀。

既已同行,何言(或:说什么)不服?

你的疑心呐,也忒大了。(或:你那疑心,也忒大了。)

明公。

睡着了。

我陈宫好悔也!

【二黄慢板】一轮明月照窗下,陈宫心中乱如麻。悔不该心猿并意马,悔不该随他人到吕家。吕伯奢可算得义气大,杀猪沽酒款待于他。又谁知此贼的疑心太大,拔出剑将他的满门杀。一家人俱丧在宝剑之下,年迈老丈命染黄沙。屈死的冤鬼魂休来怨咱,自有那神灵天理鉴察。

【二黄原板】听谯楼打罢了二更鼓下\protect\hyperlink{fn143}{\textsuperscript{143}},越思越想把事来做差。悔不该把家属一旦撇下,悔不该弃县令抛却了乌纱。我只说贼是个宽宏量大,汉室后来贼是惹祸的根芽。

明公。

睡着了。

【二黄原板】观此贼睡卧真潇洒,安眠好似井底之蛙。贼好比蛟龙未生鳞甲,贼好比猛虎未曾长牙。虎在笼中我不打,我岂肯放虎归山又把人抓。

【二黄散板】执宝剑将贼的头割下,

【二黄散板】险些儿把事又做差。

我若将他(一剑)杀死,旁人岂不道我与董卓同党?

(这这这\ldots{}\ldots{})

看桌案之上现有笔砚,我不免题诗一首,打动于他。

但不知以何为题?

就以四更为题。

(念)鼓打四更月正浓,心猿意马归旧踪。杀死吕家人数口,方知曹------

(啊,明公。

明公,睡着了。)

(念)方知曹操是奸雄!

陈宫题。

趁此天色朦胧,我不免寻找马匹逃走了罢。

唉!

【二黄散板】也是我陈宫做事差,不该随贼走天涯(或:悔不该随贼奔天涯)。落花有意随流水,流水无情恋落花。

我好悔也!

\newpage
\hypertarget{ux501fux8d75ux4e91-ux4e4b-ux5218ux5907}{%
\subsection{借赵云 之
刘备}\label{ux501fux8d75ux4e91-ux4e4b-ux5218ux5907}}

\textbf{{[}第一场{]}}

\textbf{(念)千军容易得,一将最难求。}

\textbf{俺刘备,只因曹操带领十万雄兵攻打徐州,要与他父曹嵩报仇。陶恭祖向吾弟兄求救。怎奈我兵微将寡,恐难取胜。为此去至北}邳\textbf{公孙瓒那里借兵解围。}

\textbf{军士们,人马缓缓而行!}

\textbf{{[}第二场{]}}

\textbf{【西皮原板】我心中恨曹操奸雄太甚,欺天子霸中原想谋朝廷。屡次里兴人马抢夺沛郡,因此上到北}邳\textbf{亲走一程。忙吩咐众将校前把路引,}

\textbf{【西皮摇板】但愿得见公孙借将回程。}

\textbf{公孙兄!哈哈哈\ldots{}\ldots{}(笑介)}

\textbf{备来得鲁莽,仁兄海涵。}

\textbf{今有曹操,因张闿杀死他父曹嵩,他就赖在陶恭祖的身上。为此带领十万雄兵攻打徐州,要与他父报仇。陶恭祖向弟借兵相助。怎奈备兵微将少,不是他人对手。是以到此与仁兄借兵解围。}

\textbf{望求赵子龙将军前往。}

\textbf{不妨,破曹之后,弟亲自送将回营。}

\textbf{岂肯失信?}

\textbf{如此当面谢过!}

\textbf{告辞。}

\textbf{来此就要讨扰。}

\textbf{请------}

\textbf{干。}

\textbf{徐州危在旦夕,备不敢久停。}

\textbf{告辞了!}

\textbf{【西皮摇板}\protect\hyperlink{fn144}{\textsuperscript{144}}\textbf{】陶恭祖望救兵营门立等}\protect\hyperlink{fn145}{\textsuperscript{145}}\textbf{,备哪有闲心肠来饮杯巡。施一礼辞仁兄足踏金镫,破曹后弟亲自送将回营。}

\textbf{{[}第三场{]}}

\textbf{【西皮摇板】在帐中辞别了公孙仁兄,一心要会一会大将子龙。}

\textbf{【西皮摇板】来至在校场地用目观定,我见了赵将军施礼打躬。}

\textbf{那厢敢是赵将军?}

\textbf{赵将军!呵呵哈哈哈\ldots{}\ldots{}(笑介)}

\textbf{请。}

\textbf{诶,这是则甚?}

\textbf{哎呀,不敢不敢!}

\textbf{与将军掸座。}

\textbf{有坐。}

\textbf{前者磐河之役,使备久已倾倒,今日得见,三生有幸。}

\textbf{岂敢。}

\textbf{只因曹兵甚强,陶恭祖邀我弟兄破曹,不能取胜,为此特来公孙兄帐下借兵解围。今有赵将军前去,大功必成}\protect\hyperlink{fn146}{\textsuperscript{146}}\textbf{也!}

\textbf{怎么,赵将军是顺情而去么?}

\textbf{这个\ldots{}\ldots{}失言了!}

\textbf{但不知将军几时起兵?}

\textbf{赵将军,你来看,天色尚早哇。你我吩咐众将缓缓而行,你我马上叙谈叙谈,以慰平生渴慕,不知赵将军意下如何?}

\textbf{将军请来传令!}

\textbf{你我一同传令:}

\textbf{众将官,一路之上不准扰害百姓,马踏青苗,违令者斩!}

\textbf{天色尚早,我与赵将军马上叙谈叙谈,尔等缓缓而行。}

\textbf{带马!}

\textbf{待我来与将军牵马坠镫。}

\textbf{呃,岂敢岂敢呐。}

\textbf{请------}

\textbf{这------}

\textbf{啊------呵呵哈哈哈\ldots{}\ldots{}(笑介)}

\textbf{赵将军,你看当今之世,天下大乱,汉室衰微。众诸侯各霸一方,争雄赌胜,看来日后成王立帝
,不知又是哪一家了。}

\textbf{哦,赵将军不知?}

\textbf{请------}

\textbf{赵将军,备倒想起一家来了哇:}

\textbf{想河北袁绍,四世三公,文有田丰、沮授、逢纪、郭图之谋,武有颜良、文丑万夫不当之勇,日后汉室基业,定属此人了。}

\textbf{比作何来?}

\textbf{哦,凤毛鸡胆。}

\textbf{呃,守户之犬。}

\textbf{怎见得?}

\textbf{哦,袁绍,他不能。}

\textbf{这该是哪一家呢?}

\textbf{赵将军,备又想起一家来了:}

\textbf{袁绍之弟淮南袁术,占据寿春,兵精粮足,又有纪灵、桥蕤,勇冠三军。况且又得了天子玉玺,不久必要称帝。日后只怕是此人了。}

\textbf{比作何来?}

\textbf{诶------怎么将当世英雄比作了冢中枯骨?}

\textbf{哦,呃,他不能。}

\textbf{哦,是了。我想他每每兴兵,霸占民女,掳抢民财,鞭挞士卒,焉能成王霸业呀。}

\textbf{呃,呃,呃\ldots{}\ldots{}备又想起一家来了:}

\textbf{想那荆襄王刘表,坐镇荆州,占有长江之险,统戍一十三郡,兵多将广,又有蔡瑁、张允善习水战,嗯,想天下大势,一定是我那宗兄刘表的了!}

\textbf{呃,正是刘景升。}

\textbf{不能?}

\textbf{哦,请------}

\textbf{唉!如今曹操在青州招贤纳士,将勇兵强,占得天时}\protect\hyperlink{fn147}{\textsuperscript{147}}\textbf{,挟天子以令诸侯。难道说,炎汉基业就归于曹操不成么?}

\textbf{哦,那曹操不足论哉。}

\textbf{呃,请------}

\textbf{啊,赵将军,备想起我那公孙兄,宽宏量大,汉室功臣,足智多谋,又有赵将军辅助。日后成王霸业,呃,一定是我那公孙兄了!}

\textbf{呃,他不能?}

\textbf{如此说来,天下竟无一人了!}

\textbf{备倒明白了:想赵将军天生威武,英雄盖世,智勇双全。日后天下定是赵将军的了!}

\textbf{原要直言!}

\textbf{哪个?}

\textbf{我刘备?!}

\textbf{唉呀呀,哈哈哈\ldots{}\ldots{}(笑介)}

\textbf{小小平原县令,兵不过三千,将不过关、张。}

\textbf{有道是:(念)天上无云难降雨,掌中无剑怎杀人?}

\textbf{【西皮摇板】刘备生来命运薄,小县令焉能掌山河。}

\textbf{啊,赵将军,你看光阴似箭。人生在世------}

\textbf{着啊!备乃困水蛟龙,陷阱猛虎。}

\textbf{古人云:有美玉于斯,韫匮而藏之,求善价而沽之。沽之哉,沽之哉,我待贾者也!}\protect\hyperlink{fn148}{\textsuperscript{148}}

\textbf{【西皮摇板】刘备身旁少英雄。}

\textbf{【西皮摇板】你好比皓月在当空。}

\textbf{赵将军,你问我的志啊------}

\textbf{将军!}

\textbf{【西皮摇板】有朝一日春雷动,得会风云上九重。}

\textbf{哎呀,失言呐失言!}

\textbf{分明``失言''怎说``实言''?}

\textbf{哦,赵将军在磐河牧马的时节,就有心扶助刘备么?}

\textbf{哎呀呀真乃桃园之幸也!}

\textbf{赵将军,你看天色不早,你我马上加鞭。}

\textbf{请!}

\textbf{【西皮摇板】明明知道故意问,侥幸打动子龙心。}

\textbf{我好恨!}

\textbf{【西皮摇板】恨只恨足下不生云,}

\textbf{呵呵哈哈哈\ldots{}\ldots{}(笑介)}

\textbf{【西皮摇板】}聪明不过赵将军。二人催马朝前进。

\textbf{{[}第四场{]}}

将军\textbf{请,将军请!}

\textbf{好将!}

\textbf{{[}第五场{]}}

\textbf{啊,赵将军来此已是徐州境界,备要先行一步。}

\textbf{请。}

\textbf{三弟,代我下马。}

\textbf{请。}

\textbf{{[}第六场{]}}

\textbf{赵将军请坐。}

\textbf{为国勤劳,何言辛苦。}

\textbf{此番借得兵马三千,大将一员。}

\textbf{就是赵将军。}

\textbf{这是我三弟。}

\textbf{三弟,见过赵将军。}

\textbf{赵将军请坐请坐,想我那三弟张翼德,乃是鲁莽之夫,言语傲慢。}

\textbf{备这厢赔罪了!}

\textbf{呃,备这厢有礼了!}

\textbf{{[}第七场{]}}

\textbf{唉!糟糕!}

\textbf{赵将军,我三弟万分粗鲁,言语冒犯,得罪将军,休得见怪。}

\textbf{我这厢下马------}

\textbf{哎呀,将军呐,典韦人马犹如潮水一般,我只得马上赔礼,马上赔礼!}

\textbf{{[}第八场{]}}

\textbf{且住!赵云到此,一战未交,为何将人马撤回北}邳\textbf{?}

\textbf{呃,莫非此人外实内虚,不免去至两军阵前,假意败在他的马前,看他救我不救。}

\textbf{正是呀:(念)大事安排定,打动子龙心。}

\textbf{{[}第九场{]}}

\textbf{啊三弟!}

\textbf{哼,我看那赵云,不如三弟------你呀!}

\textbf{{[}第十场{]}}

\textbf{呵呵哈哈哈\ldots{}\ldots{}(笑介)}

\textbf{后帐摆宴与赵将军贺功!}

\newpage
\hypertarget{ux6253ux9f13ux9a82ux66f9-ux4e4b-ux7962ux8861}{%
\subsection{打鼓骂曹 之
祢衡}\label{ux6253ux9f13ux9a82ux66f9-ux4e4b-ux7962ux8861}}

{[}第一场{]}

{[}引子{]}天宽地阔,运机谋,智广才多。

(念)口似悬河语似流,全凭舌尖运机谋。男儿若得擎天手,自然谈笑觅封侯。

卑某(或:卑末)姓祢名衡,字正平,乃平原孝义村人氏。幼读诗书,深通战策。(虽怀王佐之才,)少游北海,偶遇孔融。他将我荐与曹府门下作幕\protect\hyperlink{fn149}{\textsuperscript{149}}。我想那曹操,名为汉相,实为汉贼,焉能敬贤礼士?此番进得相府(或:去至相府),必须见机而行。

正是:(念)未遇圣命主,又愧栋梁才。(或:未逢圣明主,又愧栋梁才。)

【西皮快三眼】平生志气运未通,似蛟龙困在浅水中。有朝一日春雷动,得会风云上九重。

{[}第二场{]}

(来也。)

【西皮快板】相府门前杀气高,密密层层排枪刀。画阁雕梁双凤绕,亚赛天子九龙朝。

丞相在上,卑某(或:卑末)礼到。

卑某(或:卑末)姓祢名衡,字正平,乃平原孝义村人也。

呜哙呀!人言曹操轻贤慢士,今日一见果然名不虚传(或:果然话不虚传)。孔大夫,你把我错荐了。

【西皮快板】人言曹贼多奸巧,果然亚似秦赵高。欺君误国非正道,全凭势力压当朝。站在丹墀微微笑,哪怕虎穴与笼牢。

呵呵哈哈哈\ldots{}\ldots{}(冷笑介)

吾笑天地宽阔,并无一人。

你(或:丞相)道你帐下,文能安国武能定邦(或:文能安邦,武能定国)。请问丞相,帐下文有谁能,武有谁高?

卑某(或:卑末)愿闻一二。

呵,呵,呵呵呵呵呵\ldots{}\ldots{}(冷笑介)

你道你帐下,尽是英雄豪杰。依卑某(或:卑末)看来,尽是些无用之辈呀。

听道: 荀彧、荀攸,可使吊丧问疾;

郭嘉、程昱,可使看墓守坟;

乐进、李典,可使牧羊放马;

许褚、张辽,

哎,也只可使击鼓鸣金呐。

曹子孝,呼为要钱太守;

夏侯惇,人称完肤将军。

余下者,尽是些衣架、饭囊,酒桶、肉袋,碌碌之辈,何足道哉?

区区不才,幼读诗书,深通战策。天文地理之书,无所不读;三教九流之事,无所不晓。上,可以致君为尧、舜;下,可以配德与孔、颜。吾乃天下名士,岂与你这奸贼同党。孔大夫,你把我错荐了。

【西皮快板】平生志气与天高,不愿金钱结富豪。我本是堂堂青史表,岂与犬马共同槽。

(量你也不敢呐。)

这\ldots{}\ldots{}

愿为鼓吏。

呵,呵,呵呵呵\ldots{}\ldots{}(冷笑介)

【西皮二六】丞相委用恩非小,用为鼓吏怎敢辞劳。背转身来微微笑,孔融做事也不高。明知曹贼多奸巧,全凭势力【转西皮快板】压当朝。我越思越想心头恼,安排巧计骂奸曹。罢罢罢暂且忍下了,明天自有我的巧妙高。

{[}第三场{]}

【西皮导板】适才与贼一席话,

【西皮散板】气得我正平乱如麻。

(念)酒逢知己千杯少,语不投机半句多。

适才进得相府,与那贼深施一礼,他坐在上面,昂然不动,倒还罢了哇(或:还则罢了),反道我的礼貌不周。明日大宴群臣,将我用为鼓吏。分明是羞辱于我哇。我不免明日当着满朝文武,将贼(或:将他)辱骂一回。纵然将我斩首,也落得个青史名标!正是:

(念)明知山有虎,偏向虎山行。

【西皮快板】昔日里韩信受胯下,英雄落魄(或:落魄英雄)走天涯。到后来登台把帅挂,扶保汉室锦邦家。到明天进帐把贼骂,拚着一命染黄沙。纵然将我的头割下,落一个骂贼的名儿扬天涯。

{[}第四场{]}

来也!

(内)【西皮导板】谗臣当道谋汉朝,

【西皮原板】楚汉相争动枪刀。汉王爷咸阳登大宝,一统山河乐唐尧。到如今出了个奸曹操,上欺天子下压群僚。我有心替主爷把贼讨\protect\hyperlink{fn150}{\textsuperscript{150}},掌中缺少杀人的刀。陪席坐定【转西皮快板】奸曹操,左右文武众群僚。元旦节与贼个不祥兆,假装疯魔骂奸曹。我把蓝衫\protect\hyperlink{fn151}{\textsuperscript{151}}来脱掉,

【西皮原板】破衣褴衫摆摆摇。怒气不息登甬道,帐下的儿郎闹吵吵。

【西皮快板】你二人休得呵呵笑,有辈古人听根苗:昔日太公曾垂钓,张良进履在圯桥。为人受得苦中苦,脱去蓝衫换紫袍。

呸!

【西皮快板】你二人把话讲差了,休把虎子当狸猫。有朝一日时运到,拔剑要斩海底鳌。

【西皮快板】休道我白日梦颠倒,顷刻就要上青霄。我把破衣也脱掉,

【西皮快板】赤身露体逞英豪。耀武扬威往上跑,

【西皮快板】你丞相降罪有我承招。

【西皮快板】将身来在西廊道,\protect\hyperlink{fn152}{\textsuperscript{152}}

【西皮散板】看奸贼他把我怎开销。

曹操。

你叫得我祢衡,我就叫得你曹操!

我露父母之遗躰\protect\hyperlink{fn153}{\textsuperscript{153}},方显我是清洁的君子。

你就是混浊的小人!

听道:你不识贤愚,眼浊也;不纳忠言,耳浊也;不读诗书,口浊也;(或:不读诗书,口浊也;不纳忠言,耳浊也;)常怀篡逆,乃是心浊也!

我乃天下名士,将我用为鼓吏,犹如臧仓毁孟子,阳货轻仲尼。曹操啊,奸贼!(你)真乃匹夫之辈也!

【西皮快板】开言怒发三千丈,大骂曹操听比方:昔日文王访姜尚,亲临渭水请栋梁。臣坐君辇联辔往,为国求贤理所当。我本是堂堂奇男子,把我当作小儿郎。枉在朝中为首相,狗奸贼不识臭和香(或:不知臭和香)。

【西皮散板】曹操把话错来讲,无水怎把蛟龙藏。

【西皮散板】鼓打一通天地响,

【西皮散板】鼓打二通震朝纲。

【西皮散板】鼓打三通灭奸党,

【西皮散板】鼓打四通国安康。

【西皮散板】鼓伐一阵连声响,

【西皮散板】管教你狗奸贼死无下场。

列公啊。

【西皮二六】未曾开言我的心头恨,尊一声列公大人听详情:家住在平原孝义村,姓祢名衡字表正平。我胸中颇有安邦论,曾与孔融当过了幕宾。他将我荐与曹奸佞,贼有眼不识宝和珍。我宁做那忠良门下客,不愿做奸贼帐下的人。

【西皮快板】贼道我正平舌辩徒,舌辩之徒有张、苏。苏秦六国为相首,全凭舌辩压诸侯。有朝大展昆仑手\protect\hyperlink{fn154}{\textsuperscript{154}},要把奸贼一笔勾。

【西皮快板】贼那里道我井底蛙,井底下蛙也不差。有朝一日风云驾,要把奸贼一把拿(或:一把抓)。

【西皮散板】狗奸贼他那里故意问道,尊一声列公卿细听根苗:自幼儿举孝廉官职卑小,他本是夏侯子过继姓曹。到如今做高官忘了祖考\protect\hyperlink{fn155}{\textsuperscript{155}}(或:忘了宗祧\protect\hyperlink{fn156}{\textsuperscript{156}}),全不怕骂名儿万载笑嘲。

量尔也不敢呐。(或:哼,我量你也不敢呐。)

住了! (或:呀呸!)

【西皮散板】要往荆州不能够,岂与奸贼作马牛。

(哦。)

【西皮二六】列公大人齐来劝我,犹如方醒(或:犹如推醒)梦南柯。自古道未曾责人先要责己过,手摸胸膛自揣摩。罢罢罢暂息我的心头火,

【西皮快板】学一个陆贾与随何。丞相有事交与我,顺说刘表做定夺。

【西皮摇板】丞相宽心安闲坐,披星戴月奔江河(或:渡江河)。顺说事儿若不妥,

【西皮散板】愿死他乡做鬼魔。

\newpage
\hypertarget{ux6c49ux6d25ux53e3-ux4e4b-ux5173ux516c}{%
\subsection{汉津口 之
关公}\label{ux6c49ux6d25ux53e3-ux4e4b-ux5173ux516c}}

{[}第一场{]}

{[}引子{]}威震乾坤,扶汉室,一点丹忱。

(念)忠义一腔贯古今,补天扶日志平生。英雄几见称夫子,豪杰如斯乃圣人。\protect\hyperlink{fn157}{\textsuperscript{157}}

某,汉寿亭侯关------。可恨曹操,诓哄孺子,刘琮献了荆襄,反遭其害。刘皇叔弃了新野,欲取荆州。曹兵百万,追赶甚急。因此诸葛军师,命某前来江夏,向大公子刘琦,搬兵取救。怎奈他连日染病未痊,不能发兵。某今在此,心悬两地,好不焦虑人也!

【西皮原板】想国家气运衰令人悲悼,叹不尽创业难英雄功高。刘皇叔帝室后欲将国保,时不至空使人忧虑焦劳。

大公子。

有座。

公子贵恙既已痊愈,克即发兵,与某前去接应,恐刘皇叔悬念之至。

即刻点将发兵。

哦,军师为何来此?快快有请。

哦,军师到了。

得令!

【西皮二六】某正在心悬急军师驾到,好一似风云会波浪腾蛟。府堂上领雄兵谕令军校,

【西皮摇板】斩曹操准备某偃月钢刀。

{[}第二场{]}

(刘备
【西皮散板】败当阳过长桥夏口而奔,猛回头又只见襄阳古城。实可怜数万的无辜百姓,懦弱子失荆州苦及黎民。)

(刘备
【西皮原板】自桃园结义起同扶汉鼎,我三人投公孙屡建奇勋。在安喜鞭督邮弃了信印,仗大义救孔融陶谦让城。收吕布却反被吕布兼并,饮曹操青梅酒饱受【转西皮二六】虚惊。失徐州投河北袁绍不信,兄弟们遭失散相会古城。好容易得新野暂时安稳,)

(刘备 【西皮散板】到如今依然是颠沛飘零。)

(刘备
【西皮摇板】说什么年半百儿是根本,说什么汉宗室儿是皇孙。为蠢子叹糜氏自投枯井,为蠢子险伤我股肱之臣。今日里势已败要儿做甚?)

(刘备 【西皮散板】我岂肯学袁氏溺爱不明?)

{[}第三场{]}

【西皮导板】青龙偃月\protect\hyperlink{fn158}{\textsuperscript{158}}威风凛,

【西皮快板】赤兔胭脂起风云。桃园弟兄忠心耿,誓挽汉室天日倾\protect\hyperlink{fn159}{\textsuperscript{159}}。英雄此时当效命,

【西皮散板】除奸扶汉镇乾坤。

{[}第四场{]}

曹操休得逞强,关老爷来也!

\newpage
\hypertarget{ux4e34ux6c5fux4f1a-ux4e4b-ux5218ux5907}{%
\subsection{临江会 之
刘备}\label{ux4e34ux6c5fux4f1a-ux4e4b-ux5218ux5907}}

{[}第一场{]}

{[}引子{]}奸雄并立,起戈矛。怎能够,中原尽扫。

(念)临难仁心存百姓,登舟挥泪动三军。至今凭吊襄江口,父老犹然忆使君。

孤,刘备,自败当阳,兵屯夏口,只因孔明先生去往东吴,共议破曹之策。只是未见音信回来,教孤放心不下。

来,唤糜竺进帐。

罢了。

只因孔明先生去往东吴,一去渺无音信。意欲命你备下礼物,去往东吴。名为犒军,暗探先生。听我吩咐。

【西皮摇板】过江去探孔明虚实动静,必须要一同回免孤忧心。

【西皮摇板】叹只叹汉刘备未得天运,似困龙何日里平步登云。

【西皮摇板】恨曹瞒勒逼我困于江夏,每日里操兵将习练战法。

{[}第二场{]}

【西皮摇板】糜子仲往江东探其虚诈,待等那先生回细问根芽。

罢了,孔明先生何在?

哦,想是周瑜已定破曹之策,邀我面议。如此准备船只,即刻便行。

二弟少礼,请坐。

兄欲往江东赴会,二弟为何阻拦?

这\ldots{}\ldots{}

无有哇。

啊二弟,如今孙、刘结盟,共破曹操。周瑜相请,必有大事商议。今若不去,则两下猜疑,事不谐矣。

二弟同去,兄无忧矣。

就命三弟、四弟守寨,你我弟兄即刻过江。

{[}第三场{]}

看,江水波涛,水天一色。好一派江景也。

【西皮原板】汉阳江上把船开,波涛滚滚风云来。两旁排列旌旗摆,临江会上逞英才。

二弟。孔明自往江东,渺无音信。今周郎邀我临江赴会,难免有诈,你我弟兄须要留心一二。

{[}第四场{]}

都督,备过江来了。

不敢,都督请。

啊,呵呵哈\ldots{}\ldots{}(陪笑介)

(周瑜 \ldots{}\ldots{}参拜。)

不敢,都督名闻天下,备不才无学,怎当将军重礼。

来,就分宾主而坐。

请坐。

岂敢,都督身挂金印,备少来恭贺,望祈海涵。

今日孙刘结盟,共破曹贼,乃天下之幸也。

不敢,摆下就是。

都督这是何意?

不敢,请起。

都督请!

【西皮原板】多蒙美意礼相邀,临江会上似琼瑶。孙、刘两家结盟好,同心协力破奸曹。

【西皮原板】都督英名天下晓,

【西皮摇板】定有妙计展雄韬。

多谢都督。

干。

乃备二弟云长。

呃,乃他昔年之事,何足都督挂齿。

岂敢呐岂敢。

请便。

请问都督,帐下多少人马?何计破曹?

如此待等都督大功成就,备专当叩贺。

告别了。

【西皮摇板】临江会上备讨扰。

{[}第五场{]}

【西皮散板】此来不见诸葛亮,倒教刘备挂心肠。弟兄同回江夏往。

哎呀,先生呐,你想煞我也。

呃,不知呀。

哎呀,险呐。

先生,同回夏口去罢!

【西皮散板】诸葛亮果奇才世间少有,准备着迎他回整顿貔貅。

\newpage
\hypertarget{ux7fa4ux82f1ux4f1a-ux4e4b-ux9c81ux8083ux8bf8ux845bux4eae}{%
\subsection{\texorpdfstring{群英会\protect\hyperlink{fn160}{\textsuperscript{160}}
之
鲁肃、诸葛亮}{群英会160 之 鲁肃、诸葛亮}}\label{ux7fa4ux82f1ux4f1a-ux4e4b-ux9c81ux8083ux8bf8ux845bux4eae}}

{[}第一场{]}

鲁肃 来也!

鲁肃 (念)运筹除汉逆,参赞保东吴。

鲁肃 参见都督,孔明先生到。

鲁肃 现在帐外。

鲁肃 有请诸葛先生。

诸葛亮 嗯哼!

诸葛亮 (念)不惜一身探虎穴,智高哪怕入龙潭。

诸葛亮 (啊)都督。

诸葛亮
有座。(亮来得卤莽,都督海涵。\protect\hyperlink{fn161}{\textsuperscript{161}})

诸葛亮 岂敢。(都督)相召,有何见谕?

诸葛亮 人马未动,粮草先行。

诸葛亮 承都督委派,亮自当效劳。

诸葛亮 就请都督传令。

诸葛亮 得令。

诸葛亮 正是:(念)明知周郎借刀计,佯装假作不知情。

诸葛亮 哈哈哈\ldots{}\ldots{}(笑介)

鲁肃 啊,都督,命孔明劫粮却是何意?

鲁肃 哦,是是是。

(鲁肃出帐在大边台口略一沉思介下)

鲁肃
哈哈哈\ldots{}\ldots{}(笑介)\protect\hyperlink{fn162}{\textsuperscript{162}}

鲁肃 【西皮摇板】诸葛亮背地里将人嘲笑,他道那周都督用计不高。

鲁肃
那孔明他回到馆驿,哈哈大笑,道他水战、陆战,车战、步战,件件精通。\protect\hyperlink{fn163}{\textsuperscript{163}}非比都督只习水战一能耳!

鲁肃 然也。

鲁肃 是是是。

鲁肃 咳,原令追回。

(鲁肃出帐在大边台口轻摊手轻叹下)

{[}第二场{]}

(周瑜 但不知\ldots{}\ldots{})

鲁肃 乃是荆襄降将蔡瑁、张允。

鲁肃 啊,都督闻得蒋干过江,为何发笑?

鲁肃 哦哦哦\ldots{}\ldots{}

鲁肃 啊,都督,想那蒋干乃都督同窗契友,恐识笔迹,肃来代笔。

鲁肃 是是是。

(鲁肃出帐在大边台口左手轻捋胡子、轻转腰看右手信)

鲁肃 噗。(笑介)

(鲁肃用手中信一挡头,边转身边把信送入左袖口内下)

(鲁肃藏书,有细致做工\protect\hyperlink{fn164}{\textsuperscript{164}})

(鲁肃藏完书出帐先往上场门边走一小步,听见蒋干来了,立即撤左脚往下场门边轻退,边退边把灯从右手交左手,右手投袖遮灯,站稳偷眼一望、一点头,沉着转身轻轻稳步下)

鲁肃 啊,蒋先生,请了请了。

鲁肃 都督醒来,都督醒来。

(周瑜出帐子)

鲁肃 哈哈哈\ldots{}\ldots{}(笑介)

鲁肃 那蒋干果然他盗书逃走了。

鲁肃 都督请看。

鲁肃 呵,哈哈哈\ldots{}\ldots{}(笑介)

鲁肃 量他们不知。

鲁肃 那孔明么,哼,他也未必料到。

鲁肃 呵,哈哈哈\ldots{}\ldots{}(笑介)

鲁肃 【西皮摇板】周都督运机谋神鬼不觉。

(周瑜唱完第三句【西皮摇板】下,鲁肃跟过去到下场门边向里站住,一想、轻摇头,回身右手投袖、颠袖露手)

鲁肃 【西皮摇板】怕只怕瞒不过南阳诸葛。

(用右手指,抓袖转身,\textless{}\textbf{大锣抽头}\textgreater{}稳步甩下摆,微摇头下)

{[}第三场{]}

鲁肃 哈哈哈\ldots{}\ldots{}(内笑介出)

鲁肃 【西皮摇板】曹孟德果杀了蔡瑁、张允,周都督可算得第一能人。

鲁肃 呵哈哈哈\ldots{}\ldots{}(笑介)

(周瑜 \ldots{}\ldots{}为何发笑?)

鲁肃
那曹操果然中了都督借刀之计,杀了蔡瑁、张允。水军头目换了毛玠、于禁掌管了。

鲁肃 量他们不知。

鲁肃 量他也不晓。

鲁肃 遵命。有请诸葛先生。

诸葛亮 嗯哼。

诸葛亮
【西皮摇板】昨夜晚听消息早已料定(或:昨夜晚观天象早已算就;或:昨夜晚观天象早已料定),曹孟德中巧计误杀水军。

诸葛亮 (啊)都督。

诸葛亮 有座。恭喜都督,贺喜都督。

诸葛亮
那曹操中了都督借刀之计,杀了蔡瑁、张允,水军头目换了毛玠、于禁。此二人不习水战,岂非一喜?

鲁肃 怎么他\ldots{}\ldots{}(惊异介,背躬)(知道了。)

诸葛亮 你我不必言明,各写一字在手,看看心意如何。

诸葛亮 请------

诸葛亮 大夫请看。

鲁肃 嗳呀!他二人俱是``火''字。

(鲁肃退至小边)

鲁肃 你二人俱是``火''字。

诸葛亮 未必。

诸葛亮 啊,哈哈哈\ldots{}\ldots{}(笑介)

鲁肃 哦,哈哈哈\ldots{}\ldots{}(笑介)

诸葛亮 军国大事焉能泄漏。

诸葛亮 水面交锋弓箭当先。

诸葛亮 愿当此任。

诸葛亮 但不知宽限多少日期?

(周瑜 三月如何\ldots{}\ldots{})

诸葛亮 忒多了哇。

(周瑜 十日\ldots{}\ldots{})

诸葛亮 倘若曹操进军,岂不误了大事?还多啊哇。

(周瑜 七日如何\ldots{}\ldots{})

诸葛亮 军务紧急,还多哇。

(周瑜 \ldots{}\ldots{})

诸葛亮 容山人计算,三日交箭。

鲁肃 啊?三日焉能造得齐十万支雕翎(狼牙)箭呐?先生!

(周瑜 三日无箭?)

诸葛亮 三日无箭,甘当军令。

(周瑜 军无戏言。)

诸葛亮 愿立军状。

诸葛亮 亮乎?

鲁肃 使不得!

鲁肃 完了。

诸葛亮 大夫,这是山人的军令状,请大夫收藏。

诸葛亮 三日后江边搬箭。

鲁肃 搬你的尸灵吧。

诸葛亮 取笑了。

诸葛亮 告辞了。

诸葛亮 【西皮摇板】在帐中辞公瑾再别子敬,三日后到江边搬取雕翎。

鲁肃 啊,都督,那孔明莫非有逃走之意?

鲁肃 是。

鲁肃 啊,都督,此二人恐是诈降。

(周瑜叫鲁肃``出帐去罢'')

鲁肃 哦,是是是。

(鲁肃到大边台口)

鲁肃
(小\textless{}\textbf{叫头}\textgreater{})哎呀且住!分明是诈,怎说是实?哎呀,这这这\ldots{}\ldots{},呵,有了,我不免去至馆驿,问过孔明先生,呃,问过孔明先生。(\textbf{不念对儿}\protect\hyperlink{fn165}{\textsuperscript{165}})

(鲁肃转身扬右手、轻摇)

鲁肃 哈哈哈\ldots{}\ldots{}(笑介)

(鲁肃边笑边下)

{[}第四场{]}

诸葛亮
【西皮原板】周公瑾命鲁肃行监坐守,叫山人背地里暗笑不休。他那里要杀我怎能得够,一桩桩一件件记在心头。

鲁肃 【西皮原板】限三天造雕翎不多时候,

(鲁肃 嘿嘿。)

鲁肃 【西皮原板】为什么坐一旁不睬不愁。

(鲁肃 嗨!)

鲁肃 【西皮快板】昨日里在帐中夸下海口,这时候倒叫我替你担忧。

诸葛亮 我(又)没有什么大事,要大夫替我担得什么忧哇?

鲁肃
啊?!昨日你(或:你昨日)在帐中夸下海口,立了军(令)状:三日造齐十万支狼牙箭。你算算,今天第几天了?

诸葛亮 怎么还有此事么?

鲁肃 呵?

诸葛亮 不错不错,不是大夫提出,我倒忘怀了。

鲁肃 哎呀,哎呀,你怎么忘怀了。

诸葛亮 来来来,(我们)算算日期吧。昨日,

鲁肃 一天。

诸葛亮 今日,

鲁肃 两天,

诸葛亮 明日。

鲁肃 三天。拿来。

诸葛亮 什么?

鲁肃 拿箭来呀。

诸葛亮 哎呀,我是一支也无有哇!

鲁肃 哎呀,这这\ldots{}\ldots{}

诸葛亮 (哎呦)大夫,你要救我一救(,你要救我一救)哇。

鲁肃 咳,事到如今,倒不如你驾一小舟逃回江夏去罢!

诸葛亮
呵,我奉主之命前来同心破曹。如今寸功未立,回去怎样回覆吾主?我如何走得?(呃,)走不得呀!

鲁肃 (哎呀)走不得\ldots{}\ldots{}咳,我只有这一条主意了。

诸葛亮 大夫有何高见。

鲁肃 你呀,投江死了罢。

诸葛亮 呵。

鲁肃 还落一全尸呀。

诸葛亮
嗳,蝼蚁尚且贪生\protect\hyperlink{fn166}{\textsuperscript{166}},为人岂不惜命?你救不了我,还则罢了,你不该劝我一死。这叫作什么朋友哇。

鲁肃 唉,叫你走你说走不得,叫你死你又舍不得。真真的教我鲁肃为难呐!

诸葛亮 大夫哇!

鲁肃 大夫哇,不能下药了。

诸葛亮 【西皮摇板】鲁大夫平日里待人宽厚,

鲁肃 本来的不错哇。

诸葛亮 【西皮摇板】你保我过江来无祸无忧。

鲁肃 是我的保荐呐。

诸葛亮 【西皮摇板】周都督要杀我你不来搭救,

鲁肃 我是怎样救你呀?

诸葛亮 【西皮摇板】看起来算不得好朋友哇。

鲁肃 呵。

鲁肃 【西皮快板】这件事本是你自作自受,为什么反把我埋怨不休。

鲁肃 你怎么倒埋怨起我来了?

诸葛亮 大夫,(大夫)你是救不了我了。

鲁肃 咳,我怎样救你呀?

诸葛亮 嗯,你救不了我,我与你借上几样物件如何哇?

鲁肃 什么物件啊?呵呵,我早已预备下了。

诸葛亮 什么?

鲁肃 寿衣、寿帽,大大(的)一口棺木。

诸葛亮 要这些物件何用呐?

鲁肃 事后将你盛殓起来,送回江夏。你就不必挂念了。

诸葛亮 哎呀呀,不是这些宝贝呐。

鲁肃 什么宝贝?

诸葛亮 乃是军中需用之物。

鲁肃 什么军中需用之物?

诸葛亮
战船二十只,军士五百名,茅草千担,青布帐幔,金鼓全份,还要备酒一席。

鲁肃 啊,(这)备酒一席何用呐?

诸葛亮 少时我与大夫同往舟中饮酒取乐哇。

鲁肃 明日无箭,我看你是饮酒哇,还是取乐哇。

诸葛亮 大夫,千万莫对人言,你去办呐。

鲁肃 咳,办呐。

鲁肃
【西皮摇板】十万箭焉能够(或:焉能得)一夜造就,为朋友我只得顺水推舟(右手指)。

(鲁肃左手撩官衣左转身右手盖头,上场门反下(示背着周瑜办事之意))

诸葛亮
【西皮摇板】这件事料鲁肃难以猜透,哪知我袖儿中(或:他哪知我袖中)暗藏机谋。要借箭待等到四更时候,趁大雾到曹营去把箭收。

{[}第五场{]}

鲁肃 【西皮快板】一桩桩一件件俱已办就,请先生到江边即刻登舟。

诸葛亮 大夫,备齐了?

鲁肃 备齐了。

诸葛亮 请。

鲁肃 请到何处(去)哇?

诸葛亮 同往舟中饮酒取乐哇。

鲁肃
要去你去,我不去。(右手手心朝上往外一挥,托胡子扔出去、摇左手右转身要走同时递左手,孔明左手拉住鲁肃左手,右手从鲁肃左胳膊上方过去,用扇柄挑鲁肃胡子)

诸葛亮 走走走。(孔明拉鲁肃下)

诸葛亮 将船往江北而发。

鲁肃 呵,慢来慢来,曹营现在江北,那如何去得的?来来来,待我下船。

诸葛亮 (呃,)船已离岸,(你)下不去了。

鲁肃 便怎么样呀?

诸葛亮 你我一同吃酒(或:一同饮酒)哇。

鲁肃 诶,什么吃酒哇。

诸葛亮
【西皮原板】一霎时白茫茫漫江雾厚,顷刻间观不出在岸在舟。似这等巧机关世间少有,学轩辕造指南以制蚩尤。

(鲁肃 哎!)

鲁肃
【西皮原板】鲁子敬在舟中浑身战抖,把性命当儿戏全不担忧。这时候他还有心肠饮酒\protect\hyperlink{fn167}{\textsuperscript{167}},

(鲁肃 唉!)

鲁肃 【西皮原板】顷刻间到曹营难保人头。

诸葛亮 将船直往曹营进发。

鲁肃 呵,我要下船。

诸葛亮 船行半江,你(是)下不去了。

鲁肃 (呃,)便怎么样呀?

诸葛亮 请来吃酒哇。

鲁肃 呃,吃酒,吃酒,好,吃酒(、吃酒)哇。

诸葛亮 大夫哇,

诸葛亮 【西皮摇板】劝大夫放宽怀且自饮酒,些许事又何必这等担忧。

诸葛亮 擂鼓呐喊。

鲁肃 (哎呀,)不要擂鼓。

(曹操 \ldots{}\ldots{}吩咐放箭。)

(\textless{}\textbf{风入松}\textgreater{}头段)

(军士 \ldots{}\ldots{}经受不住。)

诸葛亮 拨转船头,军士大喊三声:南阳诸葛先生谢曹丞相赠箭。

诸葛亮 大夫,请来观看呐。

(鲁肃先是右手翻袖盖头,往台中间一望,再用左袖盖头、右手撩官衣、窝下)

{[}第六场{]}

(\textless{}\textbf{风入松}\textgreater{}二段,鲁肃、诸葛亮上)

鲁肃 先生,你怎样知道(或:知晓)今晚有此一场大雾哇?

诸葛亮 为谋士者,不知天文,不晓地理,乃庸才也。

鲁肃 先生真乃神人也。

诸葛亮 岂敢。大夫看看,这令可以交得的了么?

鲁肃 交令呐,有我。

诸葛亮 大夫请。

鲁肃 呃,先生请转。

诸葛亮 何事?

鲁肃 我实实服了你了。

诸葛亮 大夫服我何来(呢)?

鲁肃 我服你好阴阳,好八卦。好大的胆量呐。

诸葛亮 我也服了你了。

鲁肃 你服我何来呢?

诸葛亮 我服你在舟中这样呵------(抖介)

鲁肃 𠳶,𠳶,𠳶。(同时``欺''孔明、三指孔明,孔明小撤步三挡下,鲁肃追下)

{[}第七场{]}

鲁肃 参见都督。

(周瑜 \ldots{}\ldots{})

鲁肃 他造齐了。

(周瑜 \ldots{}\ldots{}怎样造\ldots{}\ldots{})

鲁肃
都督容禀:那孔明他回到馆驿,一天也不慌,两天也不忙。到了三日也不用我国工匠人等,只用战船二十只,军士五百名,茅草千担,青布帐幔,金鼓全份。四更时分,去至曹营,擂鼓呐喊。那时满江(的)大雾,(那)曹贼闻知,吩咐水陆两寨一齐放箭。顷刻之间,借来十万支雕翎。特来交令呐。

鲁肃 可算得是活神仙。

鲁肃 有请呵,呵,活神仙。

诸葛亮 (念)狼牙已造就,只在险中求。

诸葛亮 有座。

诸葛亮 都督关照。

诸葛亮 请。

(周瑜 \ldots{}\ldots{}一百脊杖。)

(鲁肃 哎呀!)

(``打盖''时黄盖是跪左腿面里躬\textbf{身受脊杖};诸葛亮正襟危坐,\textbf{不是在喝酒}\protect\hyperlink{fn168}{\textsuperscript{168}};鲁肃跪黄右侧,双袖覆黄背、两轰牢子手,\textbf{是阻周瑜},\textbf{欲向孔明作色},\textbf{不是一劲作揖})

(周瑜下)

鲁肃 这一下我可不服你了(或:我可又不服你了)。

诸葛亮 怎么(又)不服我了?

鲁肃
方才周都督怒责黄公覆,我等俱是他麾下之人,不好讲情呐。你是客位,坐在席前,一言不发,是何道理(或:是何理也)?呵,难倒(说)这酒就是这样(的)好吃的么?

诸葛亮 他二人一个愿打,一个愿挨,与我何干呐?

鲁肃 世间之上只有愿打,哪个愿挨?你愿挨我就来打。

诸葛亮 他二人又是一计呀。

鲁肃 呵,又是一计?呃,倒要领教。

诸葛亮 大夫哇,

诸葛亮 【西皮摇板】他二人定下了苦肉之计,

鲁肃 收蔡中与蔡和呢?

诸葛亮 【西皮摇板】收蔡中与蔡和暗通消息。

鲁肃 今日之事?

诸葛亮 【西皮摇板】黄公覆受苦刑俱是假意,

鲁肃 (哎呀,)我是哪里晓得呀!

诸葛亮 【西皮摇板】进帐去切莫说我诸葛先知呀。

(孔明溜下,胡琴\textless{}\textbf{行弦}\textgreater{}鲁肃转身朝外、左手托右肘、右手摸脖子(手不动颈动))

鲁肃 我是哪里晓得呀!

(鲁肃回身拱手)

鲁肃 呵,先生。(看孔明不见了)

鲁肃 先生,先生,先生!(同时招手边追孔明下)

\newpage
\hypertarget{ux534eux5bb9ux9053-ux4e4b-ux5173ux516c}{%
\subsection{华容道 之
关公}\label{ux534eux5bb9ux9053-ux4e4b-ux5173ux516c}}

(刘曾复 饰 关公、王家祺 饰 曹操;李斌植 司鼓、屠楚材 操琴)

关公 {[}引子{]}一片丹心,辅汉室,锦绣疆宏。

关公
(念)军师将令守华容,恼得某家怒气冲。怎肯因私废公义,擒曹方显肝胆忠。

关公 某,汉室关------某与军师赌头争印,擒拿曹贼。

关公 众军校,

(众 哦!)

关公 备马伺候。

(众 啊!)

关公 【西皮导板】背地里笑诸葛用兵不到,

【西皮原板】在大营他那里藐视英豪。自幼儿读春秋韬略颇晓,为不平斩雄虎\protect\hyperlink{fn169}{\textsuperscript{169}}怒诛土豪。蒙圣母赐清泉呐改换容貌,与大哥和三弟【转西皮快板】结下了生死的故交。初起手\protect\hyperlink{fn170}{\textsuperscript{170}}破黄巾立功不小,酒未寒斩华雄吓坏群僚。过五关斩六将力保皇嫂,古城边斩蔡阳匹马单刀。今奉命埋伏在华容小道,

关公 【西皮散板】今日里一心要活拿奸曹。

关公 小路埋伏,大路点起烟火。曹贼到此,速报我知。

(众 啊!)

曹操 【西皮导板】曹孟德在马上长吁短叹,

曹操 唉!

曹操
【西皮原板】手捶胸眼落泪恨怨苍天。在许昌领人马八十三万,实指望讨东吴夺取江南。庞士元他把那连环计献,诸葛亮借东风火烧战船。在赤壁烧得我头焦肉烂,只剩下十八骑好不惨然。曹孟德勒丝缰喜笑满面,

(曹将 【西皮散板】丞相发笑为哪般?)

曹操 将军!

曹操
【西皮快板】笑只笑周郎见识浅,孔明心中无计谈。此处若有人和马,大家的性命难保全。

曹操 啊!

曹操 【西皮散板】正然说话人呐喊,莫非此地有狼烟。

曹操 看看前面什么旗号哇?

(曹将 乃是``关''字旗号。)

曹操 怎么讲?

(曹将 ``关''字旗号。)

曹操 啊哈,啊哈,啊呵呵哈哈\ldots{}\ldots{}(笑介)

(曹将 丞相为何发笑?)

曹操 你们哪里知道,那关云长曾许下老夫我三不死,难道今日一次不饶?

曹操 如此说来,不用你们杀了。

(曹将 我们杀不得了。)

曹操 不用你们战了。

(曹将 我们也战不得了。)

曹操 席地而坐。老夫亲自向前------嗯哼------搭话。

(曹将 啊。)

曹操
【西皮快板】听说来了关美髯,不由得孟德喜心间。走上前来把礼见,许昌一别有数年。

(众 曹操到,曹操到啊!)

关公 【西皮导板】耳边厢又听得人喊马闹,

(众 曹操到啊!)

关公 【西皮原板】睁开了单凤眼仔细观瞧。

曹操 二君侯,别来无恙。

关公 【西皮原板】狭路上莫不是冤家来到,

曹操 诶,你我朋友相交,何出此言?

关公
【西皮原板】今日里用武时\protect\hyperlink{fn171}{\textsuperscript{171}}怎念故交。

曹操 呃,此言忒重了。

关公
【西皮原板】三国中论奸雄算你曹操,一派的假殷勤笑里藏刀。观天时巳已完午刻来到,拿住了奸曹贼岂肯相饶。

曹操 君侯!

曹操
【西皮原板】曹孟德近前来满面赔笑,尊一声二君侯细听根苗:误中了小周郎牢笼圈套,诸葛亮借东风把我的战船烧。只剩下十八骑残兵来到,望君侯你不信仔细观瞧。

关公 周仓,

周仓 在呃!

关公 查点曹操多少人马。

周仓 得令。

周仓 嘿!站起来嘿!

曹操 (呃,)站起来!站起来!站起来!

周仓
一五,一十,十五,一、二、三。啊哈,啊哈,哇呀呀呀\ldots{}\ldots{}呵呵哈哈\ldots{}\ldots{}(笑介)

周仓 启父王,只有一十八骑残兵败将呃。

关公 怎么讲?

周仓 一十八骑残兵败将。

关公 起过了。

周仓 呃!

关公 \textless{}\textbf{叫头}\textgreater{}军师!

关公 慢说是一十八骑残兵败将,就是一十八只猛虎,关某何惧?

关公 【西皮二六】你好比钓金鲤怎能游遨,大鹏鸟------

曹操 惊弓之鸟。

关公
【西皮二六】褪翎羽难腾青霄。似蛟龙脱离了蓬莱海岛,伤弓鸟纵有翅也难飞逃。

曹操 君侯!

曹操
【西皮快板】想当初我待你恩德不小,上马金、下马银美酒红袍。官封到寿亭侯可算不小,大英雄怎忘却当年故交。

关公
【西皮快板】你虽然待我的恩高义好,我亦曾答报过你的功劳。斩颜良、诛文丑立功报效,挂信印封黄金留柬辞曹。

曹操 【西皮快板】我也曾派张辽文凭送到,酴醿酒大红袍送至灞桥。

关公
【西皮快板】休提起送文凭令人可恼,诛孔、卞,刺孟、韩,王植被枭。过黄河斩秦琪文凭才到,似丞相假殷勤呃哪放心梢。

曹操 唉!

曹操 【西皮摇板】你曾经许下我云阳三报,难道说今日里一次不饶。

关公
【西皮快板】非是我忘却了云阳答报,皆因你这奸曹其罪难逃。在许田射鹿时把君欺藐,挟天子命诸侯势压群僚。逼死了董贵妃其罪非小,董承毙、马腾斩欲谋汉朝。恨不得把奸贼剥皮楦草,

关公 刀来!

关公 【西皮摇板】来来来,试一试偃月钢刀。

曹操 哎呀!

曹操 【西皮摇板】一见君侯变了脸,倒教孟德无话言。往日的恩情无半呃点,

曹操 \textless{}\textbf{哭头}\textgreater{}君侯啊!

曹操 【西皮摇板】杀了我曹孟德你、你、你\ldots{}\ldots{}算不得能员。

曹操 \textless{}\textbf{叫头}\textgreater{}二君侯,二将军!

曹操
想当年在我营中,三日一小宴,五日一大宴,上马黄金,下马白银。临别之时,曾许我三不死。呃,大丈夫需要言而有信呐。今日若放我等逃生,漫说是曹操,就是众将,也感君侯的大恩大------德啊\ldots{}\ldots{}(哭介)

关公
【西皮摇板】往日里杀人不眨眼,铁打心肠软似绵。关某岂是无义汉,宁斩我头挂高竿。叫人来------

(众 哦!)

关公 (接)【西皮摇板】一字长蛇忙开展(或:忙开道),

(众 啊!)

关公 【西皮摇板】认识此阵你快加鞭。

曹操 来。

(曹将 在。)

曹操 向前看看什么阵式呃。

(曹将 乃是一字长蛇大阵。)

曹操 怎么讲?

(曹将 一字长蛇大阵。)

曹操 哦,想是那关云长有放你我逃走之意。趁此机会,咱们溜了罢。

(曹将 逃了罢。(或:呃,逃走了罢。)

曹操 溜了吧。

曹操 【西皮快板】多谢云长开恩典,放俺曹操回中原。三军与爷朝前趱,

曹操 【西皮散板】扭转回头谢美髯。此番回到许昌转,

曹操 周郎啊!

曹操 【西皮散板】重整人马我二下江南。

曹操 走了呃。

(众 曹瞒勿走!)

关公 回营交令呐!

关公
【西皮快板】悔当初错许他云阳答报,今日里徇人情又犯律条。叫小校辕门来通报,就说你爷放奸曹。七星剑将头斫,一腔热血洒战袍。盖世英雄辜负了,

关公 【西皮散板】汗马功劳一旦抛。

\newpage
\hypertarget{ux6218ux957fux6c99}{%
\subsection{战长沙}\label{ux6218ux957fux6c99}}

(李舒遗作 录 刘曾复先生传本,按余叔岩、王凤卿演法)

{[}第一场{]}

\textbf{(大帐、堂桌、印盒、文房。\textless{}发点\textgreater{},四绿龙套执月华旗站门,关羽上。夫子盔、黑三、千金、绿靠褶绿蟒、红彩裤、黑厚底靴。以袖挡脸上,至台口中间)}

\textbf{关羽
{[}引子{]}正气冲霄汉(放袖\textless{}一锣\textgreater{}),秉忠心,保汉疆宏。}

\textbf{(\textless{}发点合头\textgreater{},归内座)}

\textbf{关羽
(念)蚕眉凤目丹赤心,青龙偃月建奇勋。苍天若助三分力,扭转汉室锦乾坤。}

\textbf{关羽
(白)某,汉室关------三弟、四弟取了武陵、桂阳。某奉军师将令夺取长沙。众军校,}

\textbf{众 有!}

\textbf{关羽 听某一令!}

\textbf{(关羽坐)}

\textbf{关羽 【西皮导板】某奉军师将令差,}

\textbf{关羽
【西皮原板】威风凛凛坐将台。旌旗不住空中摆,大小将官逞雄才(或:抖雄才)。正气冲开凌霄汉,文光射入斗、牛开。某家出世英名在,哪把长沙挂心怀。吩咐三军把马带,}

\textbf{(出位、上马,龙套插门下,关收腿)}

\textbf{关羽 (接唱)【西皮原板】一战成功列三台。}

\textbf{(打一下战马,下)}

{[}第二场{]}

\textbf{(黄忠、魏延双起霸}\protect\hyperlink{fn172}{\textsuperscript{172}}\textbf{。黄硬扎巾、白三、黄靠;魏硬扎巾、黑满、紫靠}\protect\hyperlink{fn173}{\textsuperscript{173}}\textbf{。黄大边、魏小边,念)}

\textbf{黄忠 (念)老将年高大,}

\textbf{魏延 (念)镇守在长沙(或:威镇在长沙)。}

\textbf{黄忠 (念)丹心贯日月,}

\textbf{魏延
(念)保主锦中华。}\protect\hyperlink{fn174}{\textsuperscript{174}}

\textbf{黄忠 某,姓黄名忠字汉升(或:某,黄忠)。}

\textbf{魏延 某,姓魏名延字文长(或:某,魏延)。}

\textbf{黄忠 将军请了,}

\textbf{魏延 请了。}

\textbf{黄忠 都督升帐,你我两厢伺候。}

\textbf{魏延 请。}

\textbf{(黄、魏分坐双门椅,黄大边,魏小边。四白龙套站门,堂桌、文房、大座,打上。韩玄纱帽、黪三、缃色蟒或白蟒)}

\textbf{韩玄 {[}引子{]}镇守长沙,秉忠心,扶保乾坤。}

\textbf{(归内大座)}

\textbf{韩玄 (念)执掌兵权印,决策扫烟尘。忠心扶社稷,赤胆保龙庭。}

\textbf{韩玄
本督,韩玄。奉我主之命镇守长沙。今有(或:闻)桃园弟兄前来夺取长沙,不免传(或:与)黄、魏二将进帐,议论迎敌之策。来,传黄、魏二将进帐。}

\textbf{众 都督有令,黄、魏二位将军进帐。}

\textbf{黄忠、魏延 来也!黄忠、魏延告进!(挖进去)参见都督。}

\textbf{韩玄 二位将军少礼,请坐。}

\textbf{黄忠、魏延
谢座。(双跨椅,黄大边,魏小边)传末将(等)进帐,有何军情议论?}

\textbf{韩玄
今有桃园弟兄前来夺取长沙。请二位将军议论迎敌之计(或:请二位将军进帐,商议退敌之策)。}

\textbf{黄忠
启禀都督(或:元帅}\protect\hyperlink{fn175}{\textsuperscript{175}}\textbf{),想长沙、桂阳、零陵、武陵四郡各有军兵把守,何惧桃园(弟兄)!}

\textbf{魏延 但不知何人领兵前来?}

\textbf{韩玄 且听探马一报(,自见分晓)。}

\textbf{(报子上)}

\textbf{报子 报,关羽讨战。}

\textbf{韩玄 再探!}

\textbf{探子 啊!}

\textbf{(探子下。\textless{}冲头\textgreater{}黄、魏站立,撤椅)}

\textbf{韩玄 二位将军,关羽讨战,哪位将军出马?}

\textbf{黄忠
\textless{}叫头\textgreater{}都督,既是关羽讨战,黄忠情愿带领一哨人马(或:俺请出马),生擒那关羽入帐。}

\textbf{魏延
\textless{}叫头\textgreater{}且慢呐!老将军,我想关羽出世以来过五关斩六将,何等威风,人人皆知。诚恐老将军前去,不是他人对手哇!}

\textbf{黄忠 魏将军此言差矣。}

\textbf{魏延 何差?}

\textbf{(黄忠、魏延站)}

\textbf{黄忠
俺老只老头上发,项下须(,胸中韬略却还不老)。有道是:(念)虎老雄心在,这年迈力刚强!}

\textbf{(起\textless{}夺头\textgreater{})}

\textbf{黄忠
【西皮二六】魏将军(或:魏文长)把话错来讲,长他人的威风灭自强。
周室(或:昔日)有个姜吕望,八十三岁遇文王。战国(或:秦国)的姬颜韬略广,他也曾赴会到过(了)湘江。黄忠今年六旬上,杀人妙计腹中(或:腹内)藏。}

\textbf{魏延 老将军!}

\textbf{魏延 【西皮摇板】老将军说话欠思量,}

\textbf{魏延
【西皮快板】某家言来听端详:关羽生来韬略广,千里迢迢保皇娘。过五关曾斩六员将,擂鼓三通斩蔡阳。你今出征阵头上,只恐难胜关云长。}

\textbf{(黄、魏比粗)}

\textbf{韩玄 呃------将军!}

\textbf{(黄、魏分开,面里,不必跪。四上手持枪两边暗上)}

\textbf{韩玄
【西皮摇板】魏将军说话欠思量(或:不必多言讲),黄老将军(或:黄忠近前)听端详:一支将令往下降,命你大战关云长。}

\textbf{黄忠 得令!}

\textbf{黄忠 【西皮摇板】黄忠接令出宝帐,}

\textbf{黄忠 马来!}

\textbf{(黄忠上马,上手插门下)}

\textbf{黄忠 (接唱)【西皮摇板】会一会蒲州关云长。}

\textbf{(黄忠下)}

\textbf{韩玄 将军!}

\textbf{(四下手两边上)}

\textbf{韩玄
【西皮摇板】黄忠接令(或:得令)出宝帐,开言叫声魏文长。四路催粮需谨慎,鞍前马后要提防!}

\textbf{魏延 得令!}

\textbf{魏延 【西皮摇板】元帅(或:都督)将令往下降,}

\textbf{(魏延上马,下手插门下)}

\textbf{魏延 (接唱)【西皮摇板】不分昼夜去催粮。}

\textbf{(魏延下,韩玄出位)}

\textbf{韩玄 (接唱)【西皮摇板】黄忠、魏延出宝帐,}

\textbf{(四白龙套分下)}

\textbf{韩玄 (接唱)【西皮摇板】且听探马报端详。}

\textbf{(韩玄下)}

{[}第三场{]}

\textbf{(\textless{}风入松\textgreater{}头段,四绿龙套、关羽脱蟒穿靠执刀上)}

\textbf{关羽 夺取长沙去者!}

\textbf{(\textless{}风入松\textgreater{}二段,龙套二龙出水会阵,黄忠上,关、黄架住)}

\textbf{黄忠 \textless{}叫头\textgreater{}呔!马前来的敢是关羽?}

\textbf{关羽 既知某姓,何不下马归顺?}

\textbf{黄忠
\textless{}叫头\textgreater{}住口(或:关羽)。你桃园弟兄有多大本领,竟敢前来夺取长沙!}

\textbf{关羽 你且听道哇:}

\textbf{关羽 【西皮导板】勒马停蹄站疆场,}

\textbf{关羽
(脸冲里,拄刀,边转身边用刀指黄忠,接唱)【西皮二六】黄忠老将听端详:某大哥堂堂帝王相,当今的皇叔天下扬。某三弟翼德英雄将,}\protect\hyperlink{fn176}{\textsuperscript{176}}

\textbf{关羽
【西皮快板】大吼一声桥断梁。温酒未寒某华雄斩(或:某破黄巾兵百万),颜良、文丑丧疆场。过五关曾斩六员将,擂鼓三通斩蔡阳。劝你早把长沙让,稍有迟延刀下亡。}

\textbf{黄忠 【西皮小导板】稳坐雕鞍用目望,}

\textbf{黄忠
(接唱)【西皮快板】关公打扮非寻常:丹凤眼、眉蚕样,五绺长髯洒胸膛。(头戴金盔明又亮,身穿铠甲闪秋霜。)胯下赤兔胭脂马,青龙偃月放毫光(或:闪秋霜)。勒住了丝缰把话讲,开言叫声关云长:你夺长沙休妄想,除非你死或我亡(或:你死并我亡)。}

\textbf{(剜萝卜,钻烟筒,一扯两扯,一合两合,一拉转身,关大边,黄小边,架住,往外一绕两绕三绕,往里一绕两绕三绕,关推黄刀,打腰封,\textless{}四击头\textgreater{}亮相。关先下,黄后下}\protect\hyperlink{fn177}{\textsuperscript{177}}\textbf{)}

\textbf{(关羽\textless{}水底鱼\textgreater{}上)}

\textbf{关羽
(\textless{}叫头\textgreater{})且住!黄忠甚是骁勇(或:刀法厉害;或:十分骁勇),(再若来时)拖刀计伤他。}

\textbf{黄忠 哪里走!}

\textbf{(黄上,关用拖刀计,黄抢背落马,跪大边,关小边举刀亮相}\protect\hyperlink{fn178}{\textsuperscript{178}}\textbf{)}

\textbf{黄忠 (呔,)关羽,俺今落马,为何不杀(或:你为何不斩)?}

\textbf{关羽
(黄忠,)关某(或:某家)出世以来,不斩落马之将,回营换马再战。}\protect\hyperlink{fn179}{\textsuperscript{179}}

\textbf{(黄上马,回身一望,下。关一望两望)}

\textbf{关羽 好将!}

\textbf{(关羽挥手,四绿龙套插门下。关打下)}

{[}第四场{]}

\textbf{(四白龙套站门,韩玄上)}

\textbf{韩玄
(念)眼观旌旗起}\protect\hyperlink{fn180}{\textsuperscript{180}}\textbf{,耳听好消息。}

\textbf{(\textless{}水底鱼\textgreater{},黄忠上,下马,挖进门,站大边)}

\textbf{黄忠 末将交令。}

\textbf{韩玄 一旁坐下。}

\textbf{黄忠 谢坐。}

\textbf{(黄忠坐大边,跨椅坐)}

\textbf{韩玄 老将军胜负如何?}

\textbf{黄忠 (末将出兵,)两军阵前不分胜负。明日定要生擒关羽入帐。}

\textbf{韩玄 好,听本督一令:}

\textbf{韩玄
【西皮散板】本帅帐中(或:本督帐前)传令号,黄忠老将听根苗:你若擒得关羽到,凌烟阁上美名标。}

\textbf{(四白龙套下,韩玄下)}

\textbf{黄忠 得令。}

\textbf{(黄忠接令,转身站台中间)}

\textbf{黄忠 【西皮散板】黄忠接令出宝帐,}

\textbf{(黄忠出门)}

\textbf{黄忠 【西皮散板】背转身来自参详:}

\textbf{黄忠
(\textless{}叫头\textgreater{})且住!想某(或:适才)在两军阵前被关羽打下(或:挑下)马来,不忍杀害(或:伤害)于我。想俺这百步穿杨百发百中,我若暗地陷害于他(或:我若暗箭伤害于他),岂不被天下英雄耻笑?也罢!明日两军(或:明日去至)阵前,(我不免)只射盔缨不射咽喉,以报阵前不杀之义(或:恩)也。}

\textbf{黄忠 【西皮散板】明日战场来会阵,}

\textbf{黄忠 【西皮散板】百步穿杨射盔缨。}

\textbf{(\textless{}抽头\textgreater{},黄忠下)}

\textbf{(\textless{}长锤\textgreater{}四绿龙套站门上,关羽上)}

\textbf{关羽 【西皮散板】黄忠老将(或:老儿)失了计,}

\textbf{关羽
【西皮快板】他与某家比高低。我把黄忠好一比,绵羊遇虎把头低。(或:中了某家拖刀计,不忍杀之放他回。)将身且坐虎榻椅,(或:某家不杀放他回,将身且坐宝帐里,)}

\textbf{(关羽外场椅小座)}

\textbf{关羽 【西皮散板】且听探马报端的。}

\textbf{探子 报!黄忠讨战。}

\textbf{关羽 再探。}

\textbf{(关羽站起)}

\textbf{关羽 【西皮散板】三军(或:人来)带过赤兔骑。}

\textbf{(\textless{}扫头\textgreater{},关拿刀上马,四绿龙套下,关站大边,黄上站小边)}

\textbf{关羽 黄忠,昨日饶尔不死,你又来则甚?}

\textbf{黄忠 今日一定要与你决一死战。}

\textbf{关羽 口出大言,终何用耳(或:中何用了)。放马过来。}

\textbf{(关黄开打,一合,关小边,黄大边,一拉转身,架住往里一绕两绕三绕,,往外一绕两绕三绕,黄压关刀,关撤刀,用刀鐏盖黄刀头,鼻子,削头。黄败下,关追下)}

\textbf{(\textless{}扭丝\textgreater{},四白龙套站门,韩玄骑马上)}

\textbf{韩玄
【西皮散板】催马加鞭到战场,观看两军比刚强。下得马来敌楼上,}

\textbf{(韩玄下马,四白龙套下,韩玄上城)}

\textbf{韩玄 (接唱)【西皮散板】旌旗招展尘飞扬(或:土飞扬)。}

\textbf{(黄忠执弓箭,\textless{}扭丝\textgreater{}上)}

\textbf{黄忠 【西皮散板】催马加鞭战场到,关公追赶不肯饶。
(或:关公刀法真奥妙,念他不斩恩义高。)箭射盔缨把恩报,}

\textbf{(关上,黄放箭,关接箭,黄下)}

\textbf{关羽
(\textless{}扭丝\textgreater{}接唱)【西皮散板】接过雕翎箭一条。}

\textbf{(关拿箭到台口)}

\textbf{关羽
(\textless{}凤点头\textgreater{}接唱)【西皮散板】明知深山藏虎豹,大胆单身去採樵。}

\textbf{(关用刀头拨箭到小边台口,回身看箭,下。黄忠上)}

\textbf{黄忠
(\textless{}扭丝\textgreater{})【西皮散板】某家箭法人知名,只为报答(或:答报)不斩恩。二次盔缨射得准,}

\textbf{(黄放箭,关上接箭,黄下,关把箭扔到大边台口)}

\textbf{关羽
(\textless{}扭丝\textgreater{}接唱)【西皮散板】接过雕翎箭二根。恼恨黄忠欺人甚,放箭哪有接箭能。(或:黄忠不伤某性命,箭射盔缨为何情?)}

\textbf{(关下。黄\textless{}水底鱼\textgreater{}上)}

\textbf{黄忠
(\textless{}叫头\textgreater{})且住!关羽(或:关公)不解其意,后面紧紧跟随(或:不知进退;或:紧紧追赶),如何是好?也罢,待某磕掉(或:去)箭头伤他一箭,惊吓于他。呔,看箭!}

\textbf{(黄下。关上接箭。四绿龙套两边上)}

\textbf{关羽
(\textless{}叫头\textgreater{})且住!黄忠百步穿杨百发百中,连射三箭,只射盔缨不射咽喉是何意也?嚯嚯是了(或:唔),想是黄忠有降顺桃园之意,军士们(或:众军校),将长沙团团围住者。}

\textbf{(\textless{}风入松\textgreater{}三段,堂鼓\textless{}急急风\textgreater{},四绿龙套抄过合,分下,关往里扔箭,关下)}

\textbf{韩玄 哎呀!}

\textbf{韩玄
(\textless{}扭丝\textgreater{})【西皮散板】敌楼之上看分明,黄忠老儿起反心。三军带路宝帐进,}

\textbf{(韩下城,拉幕换堂桌。四白龙套上挖门,韩上,一番两番,下马,进门入大座)}

\textbf{韩玄 (接唱)【西皮散板】快传老将黄汉升。}

\textbf{(黄忠上)}

\textbf{黄忠
(\textless{}扭丝\textgreater{})【西皮散板】辕门下马心不定,}

\textbf{(黄下马,挖进门站大边,韩怒视,拍桌子)}

\textbf{黄忠 (接唱)【西皮散板】都督发怒为何情?}

\textbf{黄忠 都督,为何发怒?}

\textbf{韩玄
我来问你(或:我且问你),你的箭法百步穿杨百发百中,今日连射三箭,为何只射盔缨,不射咽喉,是何意也?}

\textbf{(黄忠 这\ldots{}\ldots{})}

\textbf{(韩玄 讲!)}

\textbf{黄忠
(\textless{}叫头\textgreater{})都督!末将昨日在两军阵前,被关羽挑下马来,是他不忍伤害于我(或:末将昨日在两军阵前,马失前蹄,关羽不忍伤害于我),故而末将今日只射盔缨,不射咽喉,以报(昨日阵前)不杀之意也!}

\textbf{韩玄 你待怎讲(或:怎么讲)?}

\textbf{黄忠 以报(昨日阵前)不杀之意也。}

\textbf{韩玄 住口。}

\textbf{(黄忠面朝里跪中间)}

\textbf{韩玄
(\textless{}扭丝\textgreater{}接唱)【西皮散板】听一言来怒气生,老贼不该起反心。吩咐两旁刀斧手,绑出(或:推出)辕门问斩刑。}

\textbf{(黄忠转身向外屁股坐子,两刀斧手上,给黄上绑)}

\textbf{黄忠 【西皮散板】情愿一死仁义尽,岂肯做那无义人。}

\textbf{(\textless{}叭嗒仓\textgreater{}、\textless{}冲头\textgreater{},黄、刀斧手出门下)}

\textbf{韩玄 (接唱)【西皮散板】滚木擂石安排定,}

\textbf{(韩玄出位,四白龙套下)}

\textbf{韩玄 (接唱)【西皮散板】等候魏延定计行。}

\textbf{(韩玄\textless{}抽头\textgreater{}下)}

{[}第五场{]}

\textbf{(\textless{}长锤\textgreater{},四下手执车旗站门,魏延换箭衣、马褂上)}

\textbf{魏延 【西皮摇板】某家奉了都督命,解押(或:押运)粮草转回营。}

\textbf{魏延
某(或:俺),魏延。奉了都督{之命}(或:将令),押运粮草军前{需用}(或:听用)。粮草催齐回营交令。军士们,}

\textbf{众 有!}

\textbf{魏延 催军!}

\textbf{魏延 (接唱)【西皮摇板】三军押粮(或:与爷)往前进。}

\textbf{(下手插门下)}

\textbf{魏延 (接唱)【西皮摇板】见了都督说分明(或:问军情)。}

\textbf{(打下)}

{[}第六场{]}

\textbf{(黄忠换红箭衣,戴条子)}

\textbf{黄忠 (内)【西皮导板】将令(或:号令)一出绑帐口,}

\textbf{(二刀斧手上,黄忠\textless{}四击头\textgreater{}上至九龙口)}

\textbf{黄忠 【西皮原板】汗马功劳一笔勾。}

\textbf{(扯正,扯四门)}

\textbf{黄忠
(接唱)【西皮原板}\protect\hyperlink{fn181}{\textsuperscript{181}}\textbf{】桃园弟兄来争斗,一来一往统貔貅。误中了(或:某中了)关公拖刀计,蒙他不斩把我留(或:将我留;或:把情留)。都只为他人情谊(或:恩义)厚,百步穿杨把恩酬,都督一见冲牛、斗,绑出了辕门(或:绑出了营门)要斩头。移步儿来在营门首,}

\textbf{(黄忠坐大边门椅,二刀斧手站身后)}

\textbf{黄忠 (接唱)【西皮散板】到此时我只得气忍咽喉。}

\textbf{(四下手``一条鞭''上,魏延\textless{}长锤\textgreater{}上)}

\textbf{魏延 【西皮散板】来在营门下走兽,}

\textbf{(魏延下马,四下手下,魏延看)}

\textbf{魏延 啊?!}

\textbf{魏延 (接唱)【西皮散板】老将军醒来问根由。}

\textbf{魏延 老将军醒来!}

\textbf{黄忠 【西皮小导板】法场上绑得我如醉酒,}

\textbf{魏延 老将军!}

\textbf{黄忠
(\textless{}凤点头\textgreater{})【西皮散板】抬头只见魏参谋。}

\textbf{魏延
(接唱)【西皮快板】老将军身犯何罪由?快对某家(或:快与某家)说从头。}

\textbf{黄忠
(\textless{}三锣\textgreater{})【西皮快板】都只为龙争并虎斗,两军阵前运机谋。误中关公拖刀计,蒙他不杀将我留。都只为他人情谊(或:恩义)厚,箭射盔缨把恩酬。进帐去不容我开口,绑出(了)辕门要斩头。}

\textbf{魏延 (接唱)【西皮快板】老将军不必心担忧,末将进帐把情求。}

\textbf{黄忠
(接唱)【西皮快板】你与韩玄长沙守,不可(或:休要)为我结冤仇。}

\textbf{魏延
(接唱)【西皮快板】他若是(或:倘若是)人情来准下,万般事儿一旦丢。韩玄若是不罢手,定教老贼一命休。吩咐两旁(或:开言叫声)刀斧手,你把老将留一留。}

\textbf{(\textless{}叭嗒仓\textgreater{}\textless{}扭丝\textgreater{},魏延下,黄忠站)}

\textbf{黄忠 (接唱)【西皮散板】一见魏延(或:魏延讲情)进帐口,}

\textbf{(二刀斧手扯斜)}

\textbf{黄忠 (接唱)【西皮散板】法场之上我担忧(或:心忧)。}

\textbf{(二刀斧手押黄忠下)}

{[}第七场{]}

\textbf{(\textless{}扭丝\textgreater{}四白龙套站门,韩玄上)}

\textbf{韩玄
【西皮散板】闷坐帐中(或:斩了黄忠)心不定(或:神不定),眼跳心惊为何情?闷恹恹坐在大堂(或:坐在宝帐;或:且坐宝帐)等,}

\textbf{(韩玄归大座)}

\textbf{韩玄 (接唱)【西皮散板】魏延到来定计行。}

\textbf{(魏延上)}

\textbf{魏延 (接唱)【西皮散板】将身且把(或:怒气不息)宝帐进,}

\textbf{(魏延挖进,到大边)}

\textbf{魏延
(接唱)【西皮散板】见了都督来求情(或:讲人情)。(\textless{}住头\textgreater{})}

\textbf{魏延 末将交令。}

\textbf{韩玄 将军请坐。}

\textbf{魏延 谢座。}

\textbf{(\textless{}五击头\textgreater{},魏延坐大边跨椅)}

\textbf{韩玄 粮草可曾催齐?}

\textbf{魏延 俱已催齐,都督查点。}

\textbf{韩玄 将军之功也!}

\textbf{(魏延一望两望)}

\textbf{韩玄 将军你看些什么?}

\textbf{魏延 请问都督,黄老将军哪里去了?}

\textbf{韩玄  老贼起了降刘之心,绑赴法场问斩去了。}

\textbf{魏延
啊,都督斩了黄忠,不值紧要(或:要紧),桃园弟兄兴兵前来,如何是好(或:何人出马)?}

\textbf{韩玄 自然是将军出马呀。}

\textbf{魏延
(哼哼,)赦了黄忠,我便出马;不赦黄忠,哼,我就不管你的闲事了。}

\textbf{(韩玄 你待怎讲?)}

\textbf{(魏延 不管你的闲事了。)}

\textbf{韩玄 (我)定斩不赦。}

\textbf{魏延 你待怎讲?(或:怎么讲?)}

\textbf{韩玄 定斩不赦。}

\textbf{魏延 啊?!(或:呀呸!)}

\textbf{(\textless{}扭丝\textgreater{},魏延站起)}

\textbf{魏延
【西皮散板】放了(或:赦了)黄忠我便罢,不放(或:不赦)黄忠我的怒气发!}

\textbf{韩玄 唗!(或:大胆!;或:住口!)}

\textbf{韩玄
【西皮散板】骂声魏延真胆大,敢在帐前(或:帐中)乱军法。快将魏延来拿下,推出(或:绑至)辕门把他杀(或:将他杀)!}

\textbf{(\textless{}崩登仓\textgreater{},魏延站台口打``哇呀呀'')}

\textbf{魏延
(\textless{}扭丝\textgreater{}接唱)【西皮散板】听一言来怒气发,不由某家咬钢牙。两旁儿郎一起杀,}

\textbf{(四白龙套杀死下。韩玄抱印下,魏延追下,韩玄抱印上,一举两举,魏延一漫头,两漫头,回身刺韩玄倒,魏延拿印)}

\textbf{魏延
(\textless{}扭丝\textgreater{}接唱)【西皮散板】看你饶他(\textless{}叭嗒、仓、仓、仓\textgreater{},魏用剑打``彩头''三下)不饶他。}

\textbf{(\textless{}叭嗒仓\textgreater{}亮相,\textless{}一锤锣\textgreater{}下)}

{[}第八场{]}

\textbf{(二刀斧手、黄忠下场门上,\textless{}扭丝\textgreater{},黄坐大边台口)}

\textbf{黄忠
【西皮散板】魏延进帐时已久,为何一去不回头。眼观日落西山后(或:眼观红日疾行走),法场一刻似千秋。}

\textbf{(魏延上)}

\textbf{魏延
(\textless{}扭丝\textgreater{})【西皮散板】开刀先杀刽子手(或:先杀两旁刀斧手),}

\textbf{(魏延杀刀斧手)}

\textbf{魏延
(\textless{}凤点头\textgreater{}接唱)【西皮散板】再与老将说从头(或:老将军醒来听从头)。(\textless{}住头\textgreater{})}

\textbf{魏延 老将军醒来,老将军醒来。}

\textbf{黄忠 【西皮散板】霎时(或:一时)昏迷如梦走,}

\textbf{魏延 老将军醒来。}

\textbf{黄忠 【西皮散板】再与将军说从头。}

\textbf{(黄忠站大边,魏延小边)}

\textbf{黄忠
将军进帐讲情,都督可准(或:都督可曾应允)?(或:魏将军,你这是何意呀?)}

\textbf{魏延 老贼不准,是我将他杀死了!}

\textbf{黄忠 我却不信。}

\textbf{魏延 首级在此,拿去看来(或:人头在此。你且看来)。(黄接``彩头'')}

\textbf{黄忠 哎呀!(拿``彩头''在台口一对)}

\textbf{黄忠
(\textless{}扭丝\textgreater{})【西皮散板】一见人头珠泪滚,{怎不叫人痛伤情}}(\textbf{或:}点点珠泪痛伤情)\textbf{。(哭、哭一声韩太守,我叫、叫一声{忠良臣}(或:韩大人)。)可叹你为国家丧了(\textless{}哭头\textgreater{})命。}

\textbf{魏延 你拿过来罢!}

\textbf{(魏延抢彩头)}

\textbf{黄忠 (接唱)【西皮散板】回头再叫魏将军,你我同把后堂进!}

\textbf{(黄拉魏欲往里进)}

\textbf{魏延 哪里去?}

\textbf{黄忠 (接唱)【西皮散板】后堂去见韩夫人。}

\textbf{魏延 嘿嘿!都被我杀光了(或:也被我杀了)哇!}

\textbf{黄忠 哎呀!}

\textbf{黄忠 (接唱)【西皮散板】你我同把许昌进(或:许昌奔)。}

\textbf{(黄拉魏欲往外(台口)去)}

\textbf{魏延 哪里去?(或:做什么?)}

\textbf{(放手,回来。黄、魏八字站,黄大边,魏小边)}

\textbf{黄忠
(接唱)【西皮散板】魏王台前{领罪名}(或:请罪名)。}\protect\hyperlink{fn182}{\textsuperscript{182}}

\textbf{魏延 哎呀老将军,某家(或:我)把你好有一比。}

\textbf{黄忠 比作何来?}

\textbf{魏延 咸鱼放生。}

\textbf{黄忠 此话怎讲?}

\textbf{魏延 你连死活都不知道了。(你我不去逃生,反来送死不成?)}

\textbf{黄忠 依将军之见?}

\textbf{魏延 (以我之见,)你我{归顺}(或:投降)桃园弟兄岂不是好。}

\textbf{黄忠 怎么讲?}

\textbf{魏延 归顺桃园。}

\textbf{黄忠 嗯------(要去)你去,我不去。}

\textbf{魏延 你不去?(或:当真不去?)}

\textbf{黄忠 我不去。}

\textbf{魏延 你若不去,嘿嘿,我就是一刀杀了你呀!}

\textbf{(魏漫黄头,欺黄)}

\textbf{黄忠 哎!}

\textbf{黄忠 【西皮散板】长沙的儿郎散了队,}

\textbf{魏延 (接唱)【西皮散板】东逃西奔各自归。}

\textbf{黄忠 (接唱)【西皮散板】你做此事悔不悔?}

\textbf{魏延 (接唱)【西皮散板】事到临头(或:事到头来)埋怨谁。}

\textbf{黄忠 (接唱)【西皮散板】我哭,哭一声韩太守,}

\textbf{魏延 呔!我不准(或:我不教)你哭!}

\textbf{(黄忠 (接唱)【西皮散板】叫,叫一声韩元戎啊\ldots{}\ldots{})}

\textbf{魏延 我不许你嚎!}

\textbf{黄忠 (接唱)\textless{}哭头\textgreater{}啊\ldots{}\ldots{}}

\textbf{(魏延 我看你们哪一个敢嚎!)}

\textbf{黄忠 魏延!}

\textbf{黄忠 (\textless{}哆啰\textgreater{}接唱)【西皮散板】你这冒失鬼!}

\textbf{魏延 走,走。}

\textbf{(\textless{}冲头\textgreater{},二人往里转身,往外转身,魏轰黄幺二三,(换双楗子),黄托胡子,扔胡子,下)}

{[}第九场{]}

\textbf{(四绿龙套上,关羽上)}

\textbf{关羽 【西皮摇板】奉令夺取长沙地,}

\textbf{关羽 【西皮快板】黄忠箭法果然奇,一统汉室三分鼎,}

\textbf{(关羽归坐外场椅)}

\textbf{关羽 (接唱)【西皮快板】扶保兄王锦华夷。}

\textbf{(报子上)}

\textbf{报子 报!黄忠、魏延辕门投降!}

\textbf{关羽 知道了。升帐!}

\textbf{(关羽归坐内场椅)}

\textbf{报子 升帐。}

\textbf{(报子下)}

\textbf{关羽 架起刀门,传黄忠、魏延进帐!}

\textbf{众 黄忠、魏延进帐!}

\textbf{(\textless{}长锤\textgreater{}黄忠、魏延上,亮弦,\textless{}闪锤\textgreater{},小边台口)}

\textbf{黄忠 【西皮摇板】来在辕门(或:营门)用目觑,}

\textbf{魏延 【西皮摇板】刀枪剑戟摆得齐。}

\textbf{黄忠 【西皮摇板】我不归降转回去,}

\textbf{(魏延拦)}

\textbf{魏延 【西皮摇板】你不归降我不依。}

\textbf{黄忠 【西皮摇板】进得帐来屈膝跪,}

\textbf{(黄忠、魏延双挖门,跪,黄在大边,魏在小边)}

\textbf{黄忠、魏延 【西皮摇板】黄忠、魏延归降迟。}

\textbf{关羽 【西皮摇板】丹凤眼来观仔细,}

\textbf{关羽
【西皮快板】只见二将跪丹墀,你今归降因何意(或:尔等归降从何起;或:你今归降因何起)?一一从头说端的。}

\textbf{黄忠 【西皮摇板】只为韩玄不仁义,}

\textbf{魏延 【西皮摇板】要斩老将命归西。}

\textbf{黄忠 【西皮摇板】今日归降桃园地,}

\textbf{魏延 【西皮摇板】赤胆忠心永不移。}

\textbf{(魏延献``彩头'',关看,手一摆)}

\textbf{关羽
【西皮摇板】一见人头心惨凄,只因为国血染衣(或:命归西)。人头悬挂辕门地(或:辕门里),}

\textbf{(龙套回身}\protect\hyperlink{fn183}{\textsuperscript{183}}\textbf{)}

\textbf{关羽 【西皮摇板】二位可算将中奇。}

\textbf{(关羽出位)}

\textbf{关羽 (接唱)【西皮摇板】下得位来搀扶起,}

\textbf{(关羽搀二将,黄坐大边,魏坐小边,关坐中间)}

\textbf{关羽 (接唱)【西皮摇板】兄长到此把功提。}

\textbf{关羽 二位将军请坐。}

\textbf{黄忠、魏延 谢坐。}

\textbf{(黄忠大边,魏延小边,两边跨椅,关羽中间)}

\textbf{关羽 昨日阵前交战(或:交锋),老将军果然好刀法也。}

\textbf{黄忠
二君侯夸奖了。(或:二君侯刀法神妙。或:二君侯的刀法黄忠不及!)}

\textbf{关羽 岂敢,(或:休得过谦。啊,)昨日阵前为何不见魏将军?}

\textbf{魏延
末将奉命押解粮草,为此不在军前(或:军中),(现有)长沙印信(或:信印)呈上。}

\textbf{(魏交印,关拿印放袖内)}

\textbf{关羽 兄长到此必有封赠。}

\textbf{(内(白,搭架子) 主公(或:主上)到。)}

\textbf{关羽 二位将军暂退,}

\textbf{黄忠、魏延 是。}

\textbf{(黄忠、魏延下。黄先魏后)}

\textbf{关羽 有请!}

\textbf{(黄、魏应是下场门下。\textless{}吹打\textgreater{}四红龙套上``一条鞭'',诸葛亮、刘备上,挖进门。刘坐中,孔明大边,关羽小边,外场椅)}

\textbf{刘备 恭喜二弟(,贺喜二弟),一战成功,可喜可贺!}

\textbf{关羽 此乃兄长(或:兄王)洪福,先生妙算。小弟何功之有?}

\textbf{诸葛亮 二千岁虎威。(\textless{}撕边一锣\textgreater{})}

\textbf{关羽 小弟收得降将黄忠、魏延,现在帐外(候令)。}

\textbf{诸葛亮 二将进帐!}

\textbf{众 二将进帐!}

\textbf{(\textless{}冲头\textgreater{},黄忠、魏延上,挖门进)}

\textbf{黄忠、魏延 参见主公。}

\textbf{刘备 少礼,见过先生。}

\textbf{黄忠、魏延 参见先生。}

\textbf{诸葛亮 老将军(请起,)后帐歇息。}

\textbf{黄忠 谢先生。}

\textbf{(黄忠\textless{}小锣五锤\textgreater{}下)}

\textbf{诸葛亮 来,将魏延推出斩了(或:将魏延绑了)!}

\textbf{关羽 且慢!啊,先生,为何将魏延斩首(或:绑了)?}

\textbf{诸葛亮 魏延脑后有一反骨,故而将他斩首。}

\textbf{关羽
啊,先生,若是斩了魏延,只恐天下英雄道俺桃园(弟兄)就不义了!}

\textbf{刘备 着哇!}

\textbf{诸葛亮
日后魏延若有反意,休怪山人。来,将魏延解下桩来(或:将魏延松绑)。}

\textbf{关羽 魏延松绑。}

\textbf{魏延 多谢军师(或:先生)不斩之恩!}

\textbf{诸葛亮
非是山人不斩于你,此乃二君侯讲情,从今以后莫要离开山人左右,违令者斩。(站起)你要小心了!}

\textbf{魏延 是。}

\textbf{诸葛亮 你要打点了!}

\textbf{魏延 是。}

\textbf{诸葛亮 下去!}

\textbf{魏延 喳,喳,喳。(魏出门)嘿!}

\textbf{(\textless{}冲头\textgreater{}魏延下)}

\textbf{关羽 (现有)长沙印信(或:信印)献上。}

\textbf{(关羽左手拿印右手扶,站台口,刘、诸葛站,刘接印交孔明)}

\textbf{刘备 后帐摆宴,与贤弟贺功!}

\textbf{关羽 请驾。(或:请------)}

\textbf{(\textless{}尾声\textgreater{} 同下)}

\textbf{注:《战长沙》传统经典剧目,程长庚、余三胜、张二奎、汪桂芬、王鸿寿均擅演。}

《战长沙》对刀和接箭(王凤卿演法)

王凤卿演关公戏除用胭脂揉个淡红脸之外,其它与一般老生没有什么特殊分别,这与钱金福演周仓戏与一般武二花一样,不使``判儿''身段的情况相似。王的关公戏身段稳重,把子简捷。

他的《战长沙》对刀比较简单:

\textbf{第一场}开打是黄忠唱毕剜萝卜,架住,钻烟筒,一扯两扯,一合两扯(\textbf{不是大刀花过合}),回大边,刀头一拉转身架住,往外一二三绕,往里一二三绕,架住用刀头把黄刀推出去,面向里斜身回头望黄,扁刀先下。

\textbf{第二场}开打是关念``放马过来''一合过小边,刀头一拉转身架住,往里一二三绕,往外一二三绕,在外边架住,用鐏盖黄刀头,打鼻子,捋胡子不转身削黄头,亮,掠刀追下。

\textbf{三场}拖刀计,是原地漫头,转身从里边褪到小边抱刀亮(黄忠从外边翻)。

三次接箭使人有细致寓于平凡之中的感觉:

\textbf{第一箭}是黄放箭时,关上场门上,听见弓弦小锣声立即左手抄箭,右手用刀挡,先不向前大走,边走边在锣鼓中看箭,在九龙口站住,接唱``接过雕翎箭一条'',在锣鼓中走到台中间看一下箭站住,唱,唱完撤步到下场门一边再看箭,用刀头一绕箭把箭拨到上场门外边台上,再面向里胸前斜着横刀,回头望一下箭,扁刀下。

\textbf{第二箭}接箭法同上,接往后一看,在锣鼓中边走边把箭扔到下场门一边上,一直走到台中间,唱,唱得尺寸较快,唱毕刀画圈掠刀追下。

\textbf{第一次表示一惊后沉着追赶},\textbf{第二次是激怒拍马紧追}。

\textbf{第三箭}射中盔缨,又一惊,用手扶住箭走到台中间再拔下箭来看箭起叫头念,龙套两边上

念完龙套两边抄下,关下。

\newpage
\hypertarget{ux9ec4ux9e64ux697c-ux4e4b-ux5218ux5907}{%
\subsection{黄鹤楼 之
刘备}\label{ux9ec4ux9e64ux697c-ux4e4b-ux5218ux5907}}

{[}第一场{]}

{[}引子{]}义得人和,灭孙曹,孤心安乐。

(念)日月重明照英雄,全凭卧龙建奇功。虽得土地归王化,未能遂意高祖风。

孤,刘备,大树楼桑人氏。自与关、张结义桃园,三顾茅庐,请来卧龙先生,屡建奇功。客荆虽得安顿,只为孙、曹未得安定,叫孤常忧心也。正是:(念)苍天遂孤意,重整汉帝基。

罢了,进帐何事?

呈上来。东吴有书信到来,待孤拆开一观。

有请先生。

先生少礼,请坐。

东吴有书信到来,先生请来观看。

此番过江,那东吴是好意,还是歹意?

既然如此,孤就不去了。

先生计将安在?

四弟少礼,见过先生。

坐下,先生有差。

呃,慢来,慢来,前番去至东吴,就是我君臣二人,险些命丧周郎之手;此番又是我君臣二人。要去你去,孤是不去的了!

【西皮原板】先生把话错来讲,休提起当年赴会河梁。孙、刘仇结山海样,孤岂肯把性命送与周郎。

【西皮摇板】他二人把话一样讲,倒教孤王少主张。回头便对先生讲,孤王言来听端详。倘若孤王东吴丧,引孤的灵魂入庙廊。

去,孤便去。

还是多带人马才是啊。

四弟,打开一观。

哪里是不灵,分明是孤的引魂幡喏!

呃,迎孤的灵魂吧!

【西皮摇板】好个大胆诸葛亮,勒逼孤王过长江。虎穴龙潭孤去闯,

【西皮散板】你分明是送孤王去见阎王。

{[}第二场{]}

啊,都督,备过江来了。

都督请。

四弟子龙。

{[}第三场{]}

有坐。

远隔大江,少来问安,都督海涵。

为何不见吴侯?

告便。

进宫问安。

谨遵台命。

(周瑜 (念)相逢花中锦,)

(念)知己叙衷肠。

{[}第四场{]}

呃,大夫,备过江来了。

呵呵哈哈哈\ldots{}\ldots{}

有劳大夫。

大夫请便。

啊,啊,啊,都督请。

都督有何金言,当面请讲。

啊\ldots{}\ldots{}

呃,这\ldots{}\ldots{}

唉,都督哇,呃\ldots{}\ldots{}(哭介)

(周瑜 又来了。)

【西皮原板】周都督他那里提前情,倒教我汉刘备有话难云。借荆州取西川以为根本,望都督禀吴侯再等几春。

放肆。

下站。

【西皮摇板】四弟做事太莽撞,恶言恶语把人伤。周都督他倒有容人量,

都督,四弟莽撞,备这厢赔礼了。

备这厢赔礼了。

哎,都督哇!

【西皮摇板】还望都督好商量。

诸葛亮啊,害死孤王也。

【西皮摇板】勒逼孤王把宴饮,黄鹤楼上遇杀星。周郎苦苦要孤命,

四弟。

【西皮摇板】想一良谋好逃生。

四弟,有何妙计?

又是他那长坂坡!

四弟,在长坂坡前,你胯下有马,掌中有枪;今日在这黄鹤楼上,难道说你拳打------足踢------不成?

那是妖道的谣言呐。

``水军都督周''。

嗯哼,真乃是孤的好先生!

四弟,搀孤下楼。

告辞了。

\textbf{甘露寺 之 乔玄}

\textbf{{[}第一场{]}}

(刘备 看------江水波涛,水天一色,好一派江景也!)

(刘备
【西皮原板】看长江白茫茫银蛇滚滚,水与天共一色白浪纷纷。回头来再对四弟论,此一番到东吴见机而行。)

\ldots{}\ldots{}

(刘备 四弟,准备厚礼。明日你我君臣前去拜访。)

(刘备 正是:(念)来到东吴地,)

(赵云 (念)先去见乔玄。)

\textbf{{[}第二场{]}}

\textbf{呃------哼!(}内嗽介\textbf{)}

{[}引子{]}丹心镇国,辅君王,社稷安康。

(念)天子渊源重老臣,为子孝亲臣奉君。皇图永固民安乐,但愿东吴万万春。

老夫------乔玄,字嵩山,乃江东人氏。吴侯驾前为臣,官居首相,执掌江东十二内阁。夫人姜氏,膝下无儿,所生二女,长女大乔,许配孙策;次女小乔,配许周郎。适才朝罢归来,见街市之上,悬灯结彩;府下人等,一个个交头接耳,也不知他们说些甚么。

啊------家院。

老夫问你话呀。

老夫问你话呀。

适才老夫朝罢而归,见街市之上,悬灯结彩;府下人等,一个个交头接耳,不知他们说些么?

孙、刘两家结亲?呃,怎么老夫一些儿也不晓得呀!

哦,有这等事?老夫当朝首相,怎么一些儿也不知呢。

(思忖介)既是刘皇叔过江,也该前来拜拜老夫啊。

好好好,你且门上伺候!

(刘备 (念)身在东吴地,)

(赵云 (念)昼夜费心机。)

哦,果然来了。动乐有请。

啊------皇叔!

过江来了。

啊------呵呵呵哈哈哈\ldots{}\ldots{}(笑介)

请------

请坐。

皇叔驾到,蓬荜生辉。老朽有失远迎,望祈恕罪。

岂敢。

哦,罢了。

皇叔,此位是------

哦?!这就是在长坂坡前救幼主的子龙将军么?

真乃是虎将也。

哎呀呀,老朽怎敢受礼,万难从命。

呃,不、不,不敢收啊。

哎,老夫还未曾吩咐,你怎么就收下了?

好不中用!\\
呃呃呃,皇叔,如此我愧领了!

为何去心太急?

是啊,他们那里也该走走啊,只是老朽未得领教。

另日奉迎。

好,送客。

呃------嗯,皇叔到此,乃是贵客,我不肯收他的礼物,怎么你就大胆地收下了啊?

有道是:无功不受禄哇。

呃,我功在哪里?

呵呵呵,你这老狗才的话,倒也中听。

呵!

哎呀且住!刘备既已过江,孙、刘两家若能结亲,一同出兵,共敌曹操,与我东吴大大有利。我不免进宫,与太后贺喜。

来,吩咐外厢打道进宫!

\textbf{{[}第三场{]}}

\textbf{呃------哼!(}内嗽介\textbf{)}

(念)天上生瑞彩,人间配鸾凰。

来此已是,待我叩环!

乔玄求见。

领旨。

臣------乔玄见驾,国太千岁!

千千岁。

谢座。

呃,恭喜太后,贺喜太后!

太后将郡主招赘刘备,岂不是一喜?

这样的大事,太后不知,谁敢作主?

(思介)呃------莫非二千岁\ldots{}\ldots{}主意。

领旨!

太后有旨,二千岁进宫!

老臣参驾。

谢座。

太后醒来!

啊,千岁,若用此计,岂不被旁人耻笑么?

怎么?!又是周郎?

唉!他明明是害你呀。

呵呵\ldots{}\ldots{}我多口,多口哇\ldots{}\ldots{}

(孙权   【西皮原板】\ldots{}\ldots{}誓不休!)

千岁!

【西皮原板】劝千岁杀字休出口,细听老臣说根由:那刘备他本是------【转西皮二六】靖王后,【西皮快板】汉帝玄孙一脉流。他有个二弟关羽汉寿亭侯,青龙偃月神鬼皆愁。他斩颜良、诛文丑,古城又斩蔡阳的头。他三弟翼德性情有\protect\hyperlink{fn184}{\textsuperscript{184}},丈八蛇矛惯取咽喉。虎牢关前来争斗,枪挑金冠战温侯。当阳桥前一声吼,喝断桥梁水倒流。他四弟赵云常山将,盖世英名贯九州。长坂坡,救阿斗,杀得曹兵个个愁。这班武将哪国有?还有诸葛运计谋。杀了刘备不要紧,荆州岂肯来罢休?若是兴兵来争斗,曹操坐把渔利收。扭转回身启太后,老臣言来听从头:龙凤呃呈祥天造就,将计就计结鸾俦。

太后,孙、刘若能结亲,\protect\hyperlink{fn185}{\textsuperscript{185}}一同出兵,共灭曹操,与我东吴大大有利,不可失此机会也。

可以配得。

国太若相得上?

啊,太后,那刘备乃英雄之相,不相也罢。

领旨!

\textbf{{[}第四场{]}}

唉!明日太后在甘露寺面相刘备,我想刘备须发苍白,太后若相他不上,必被周郎所害,唉呀,这这这这\ldots{}\ldots{}

唉!他人闲事,不管也罢呀。

呃------都是你这个老狗才,我不肯收他礼物,你就大胆地收下了,如今岂不叫老夫作难么?

是啊,总要想个计策才是啊------

哦,有了!

乔福过来。这有乌须药一匣,命你送到馆驿,面交刘皇叔,教他连夜将须发染黑,明日在甘露寺中一相就相上了,快去快去!

啊\ldots{}\ldots{}转来!

对刘皇叔去说:明日席前,恐其有诈,命保驾将军内穿铠甲,外罩袍服,作一个``防而不备,备而不防''!

记下了。

快去快去。

这个老狗才。

唉\ldots{}\ldots{}从今以后,老夫再也不贪人家的小利了。

唉,这才是``不经一事,不长一智''哦!

\textbf{{[}第五场{]}}

刘皇叔到。

是。

啊------皇叔。

上面就是太后,见了就拜呀。

啊,太后,新姑老爷,总是要拜的。

皇叔,你要多拜几拜。

领旨。

太后有旨,二千岁上佛殿呐------

啊太后,可知皇叔的根基呀?

皇叔乃中山靖王之后,汉景帝陛下之玄孙,荆襄王刘表之堂弟,当今天子之皇叔。喏喏喏,太后请看------生得是龙眉凤目,两耳垂肩,双手过膝,真不愧是帝王的根本呐!

呃,帝王的根本!

说说也无妨啊!

哦,是,是,是。

啊太后,关美髯太后可晓得?

此人姓关名羽字云长,乃蒲州解良人也。弟兄桃园结义以来,在徐州失散,万般无奈,暂归曹营。那曹操待他十分恩厚,三日一小宴,五日一大宴,上马金、下马银,美女十名,俱不肯受哇。闻得皇叔有了下落,彼时挂印封金,在灞桥挑袍,过五关、斩六将,弟兄在古城相会。这位将军的义气------哼,不小哇!

好义气!

呃,虽不是我亲眼得见,谁人不知,呃呃,哪个不晓哇!

哦哦,好、好、好。

啊太后,张翼德太后可知?

此人姓张名飞字翼德,乃涿郡范阳人也。这位将军,在当阳桥前大吼一声,吓得曹操收去青龙伞,惊死夏侯杰。这位将军好威风,好煞气呀!

呃好威风,好煞气!

啊太后,赵子龙太后可晓得?

这位将军姓赵名云字子龙,乃真定常山人也。在长坂坡前与曹兵交战,杀入曹营,是七进七出!

呃不不不,七进七出!

七出七进,是七进七出啊!

呃本来是七进七出啊!

诸葛亮太后可知啊?这位先生,复姓诸葛名亮字孔明,道号卧龙,乃阳都人也。皇叔三顾茅庐,他是才得出山。这位先生在我东吴南屏山,高设一台,名曰七星祭风坛,借来三日三夜东风,烧退曹兵八十三万,好烧哇好烧!

领旨。

哪个的主意?

莫非二千岁?

太后有旨,二千岁上佛殿。

啊太后,我东吴有员大将,名叫贾化。

呃太后,新姑老爷讲情,总是要准的呀!

下去!

是。

太后,老臣眼力如何?

遵旨。

太后回宫。

带马------

\textbf{回荆州 之 鲁肃}

\textbf{(众将 得令!)}

\textbf{慢------慢,慢(,慢)\ldots{}\ldots{}慢着!}

\textbf{【西皮散板】美人计成画饼早已料就,到此时切莫要另结冤仇哇。鲁子敬怎能够旁观袖手啊,劝都督三思行再定良谋(或:劝都督再思行另定良谋)}\protect\hyperlink{fn186}{\textsuperscript{186}}\textbf{。}

\textbf{(哎呀)都督哇!(想)那郡主(随)同刘备回转荆州,乃是正理。你为何要将她赶回(或:你为何拦阻),是何意也?}

\textbf{哎呀,使不得,使不得呀。}

\textbf{那刘备乃是东吴的娇客呀。}

\textbf{(唉,难为你献那美人之计,诓哄刘备过江招亲,谁想以假成真。)}

\textbf{太后在甘露寺中面相刘备(或:太后做主),将郡主招赘刘备,(呃,那)岂不是东吴的娇客吗?}

\textbf{呃,不难,不难呐,}

\textbf{都督,再备美人,连那张飞也诓了前来!(或:只要都督,再备美人,漫说是那刘备,就是那张飞,嘿,他也是要来的呀。)}

\textbf{太后未必依你。}

\textbf{荆州兴兵?}

\textbf{哎呀,我怕呀------}

\textbf{(唉,)那诸葛亮的诡计,实在地厉害呀!(或:是厉害得很呐!)}

\textbf{(啊)都督,难道你就忘怀了?}

\textbf{(周瑜 忘怀了什么?)}

\textbf{(想)当年赤壁鏖兵,他在(那)南屏山上祭借东风,都督派了丁奉、徐盛,刺杀于他,尚且被他逃走(或:他尚且逃走)。}

\textbf{(都督又派他二人驾舟追赶,又被赵云箭射篷索而回。)}

\textbf{(哼,那时)几乎将你气死啊。}

\textbf{哼,难道这不是孔明的诡计么?(或:难道这不是孔明的诡计吗?)}

\textbf{啊都督,(你)不要生气呀。}

\textbf{这生气的日子还在后头呢。(或:那生气的日子还在后头呢。)}

\textbf{呃都督\ldots{}\ldots{}}

\textbf{(周瑜 不要管我的闲事。)}

\textbf{啊呀,这可不是闲事啊。}

\textbf{国家大事,唉,我不能不管呐。(或:军国大事,我是不能不问呐。)}

\textbf{呃呃呃,少弟,少弟。(或:呃呃,不敢不敢。)}

\textbf{不敢,不敢。(或:少弟。)}

\textbf{我本是个老实人,老实人才说这老实话呀!(或:唉,老实人才讲这老实话呀!)}

\textbf{哦哦,我,我,我吃醉了?(或:呃呃,呃,呃,呃呃,我,我\ldots{}\ldots{}我醉了。呃呃,我醉了?)}

\textbf{呃,我不曾吃酒,怎么我醉了?}

\textbf{诶------}

\textbf{(念)此时不听我言语,损兵折将后悔迟!}

\textbf{回荆州 之 刘备}\protect\hyperlink{fn187}{\textsuperscript{187}}

\textbf{{[}第一场{]}}

\textbf{【西皮原板】深宫无处不飞花,年老得配女娇娃。朝欢暮乐无牵挂,愿把东吴当故家。}

\textbf{【西皮散板】听说曹操发人马,攻破荆州把孤拿。四弟之言并非假,想一良谋\ldots{}\ldots{}}\protect\hyperlink{fn188}{\textsuperscript{188}}

\textbf{郡主,备要逃走了。}

\textbf{【西皮散板】本当在此多潇洒,失却荆州无有家。见郡主难说分别\textless{}哭头\textgreater{}话,}

\textbf{【西皮散板】花言巧语瞒哄她。}

\textbf{【西皮散板】根深哪怕狂风大,树正何惧日影斜。}

\textbf{{[}第二场{]}}

\textbf{(刘备穿箭衣上)}

\textbf{【西皮散板】郡主进宫辞太后,为何一去不回头。四弟且站宫门口,准备鳌鱼脱金钩。}

\textbf{认得也!}

\textbf{【西皮散板】多蒙太后恩德厚,此去只怕孙仲谋。}

\textbf{【西皮散板】四弟与孤带走兽,}

\textbf{(赵云接唱收腿,刘备下)}

\textbf{{[}第三场{]}}

\textbf{(刘备上)}

\textbf{【西皮导板】身躬步臃路途远,}

\textbf{【西皮散板】\ldots{}\ldots{}往前赶,怕的吴兵追赶还。}

\textbf{【西皮散板】你姑老爷要走呃你们谁敢拦。}

\textbf{(刘备下)}

\textbf{{[}第四场{]}}

\textbf{(刘备上)}

\textbf{【西皮散板】急急赶来如风涌,插翅难飞到九重(或:上九重)。}\protect\hyperlink{fn189}{\textsuperscript{189}}

\textbf{四弟,前有大江,后有追兵,如何是好?}

\textbf{(赵云 待我望来。)}

\textbf{\ldots{}\ldots{}}

\textbf{刘备 搭了扶手。}

\textbf{(诸葛亮 诸葛亮接驾。)}

\textbf{\ldots{}\ldots{}}

\newpage
\hypertarget{ux8ba9ux6210ux90fd-ux4e4b-ux5218ux748b}{%
\subsection{让成都 之
刘璋}\label{ux8ba9ux6210ux90fd-ux4e4b-ux5218ux748b}}

{[}第一场{]}

{[}引子{]}坐镇西川,恨张松,降顺桃园。

(念)君为民忧,又为国愁。忧国忧民,何日罢休?

孤,刘璋字季玉。祖镇西川。前者误听张松之言,致招刘备入川,指望同振汉室基业,不想他暗起图谋之意。是孤在张鲁王驾前,聘请一将,名唤马超,也曾命他在葭萌拒敌,今与桃园交战,未知胜负,且听探马一报。(或:孤,刘璋字季玉。祖镇西川。前者误听张松之言,致招刘备入川,指望同兴汉业,谁知他暗起图谋之心。是孤在张鲁王驾下,聘请一将,名曰马超,孤命他镇守葭萌关,今与桃园弟兄交战,不知胜负如何,且听探马一报。)

再探。

不,不,不好了!

【西皮散板】闻报不由心内惊(或:忽听探马报一声),不料(或:大胆)马超降他人。王到敌楼把贼问,

【西皮散板】皇儿上殿问分明。

皇儿平身,一旁坐下。(或:平身,赐座。)

(皇儿上殿,有何本奏?)

孤想刘备,兵多将广,意欲将成都让与他人就是。

【西皮原板】皇儿奏本欠思论,哪有能将敌雄兵。心中只把张松恨,竟将那地理图献与他人。老严颜巴州【转西皮二六】早降顺,张任不降命归阴。聘来的马超威风凛,反顺刘备取都城(或:反顺刘备降他人)。王有心开城把贼问,文武个个起异心。左思右想心不定,王倒做进退两难人。

【西皮快板】那刘备仁义从天命,诸葛先生赛苏秦。孤把好话对他论,难道不念同宗人。

平身。(赐座。)

卿家上殿,有何本奏?

我父子正为此事筹议,何言(或:何谓)坐视不理?

何人(或:何臣)保驾?

卿家保驾,孤无忧也。

听孤旨下。

【西皮摇板】皇儿敌楼把贼问,大事全仗王爱卿。四门人马安排定,莫教那马超贼杀进都城。

(窝下)

{[}第二场{]}

摆驾。

【西皮散板】适才王累进宫报,王儿(或:皇儿)敌楼赴阴曹。

【西皮散板】耳旁又听放火炮,马超贼放火把孤的民房烧。侍内臣摆驾上城道,

唉,皇儿\ldots{}\ldots{}(哭介)

【西皮散板】那旁来了贼马超。

(马超 蜀主请了!)

【西皮散板】见马超不由我(或:不由孤)心如刀绞,尊一声马孟起细听根苗:为王的待你是哪些儿不好,你那里降刘备所为哪条。

唉!

【西皮散板】王聘你原本为(或:王聘你为的是)西川有靠,看起来你是个无义儿曹。

【西皮散板】一言怒恼贼马超,放火把孤的民房烧。

【西皮散板】只烧得众黎民苦哀\textless{}\textbf{哭头}\textgreater{}告,

【西皮摇板】刘季玉失疆土就在今朝。

马将军,将人马暂退一箭之地,孤将成都让与刘备就是。

(马超 休得失信于我。)

唉!岂肯失信于你?

唉!

【西皮摇板】这也是成都地兵微将少,眼见得锦绣春付与水漂。

众将,

开城。

(王累 且慢!此城开不得!)

怎样开不得?

卿家,你来看!

为孤一人,岂肯连累百姓?

【西皮摇板】宁愿失却成都郡,岂肯连累好子民。

(王累 哎呀!\ldots{}\ldots{}我只为\ldots{}\ldots{}不如碰头死在都城。)

哎呀!

【西皮摇板】一见卿家丧了命,斩断擎天柱一根。

【西皮摇板】但愿你(或:但愿得)灵魂呐归仙\textless{}\textbf{哭头}\textgreater{}境,

【西皮摇板】凌烟阁上第一名(或:凌烟阁上标美名)。

众将,

开城。(哭介)

{[}第三场{]}

啊,宗兄!

宗兄请。

宗兄到此,乃是客位。

还是宗兄请

如此你我挽手而行。

呵,呵,呵,呃\ldots{}\ldots{}(哭介)

{[}第四场{]}

(宗兄,)此位是?

(刘备 这就是诸葛先生。)

哦,这就是卧龙先生。

请坐。

不知宗兄驾到,未曾远迎,当面恕罪。

啊,宗兄,前番敦请宗兄入川,共掌汉室之基业。(或:前番将成都让与宗兄执掌,)宗兄言道:不夺同宗之基业,致招天下人耻笑(或:惹天下人笑骂)。如今宗兄又兴此无义之师(或:今日兴此无义之师),失信于天下,是何意也?

(诸葛亮 这个\ldots{}\ldots{}我主乃是不得以而为之。)

哦,呵,呵,呵\ldots{}\ldots{}(冷笑介)

好一个不得已而为之!

(刘备 【西皮原板】\ldots{}\ldots{}进成都城。)

宗兄!(或:宗兄啊!)

【西皮原板】你我本是【转西皮二六】同宗姓,你今到来王心惊。有什么大事早议论,又何必带兵夺取我都城。先前让你掌蜀郡,一心要做仁义人。实指望两下【转西皮快板】结秦晋,又谁知反学吴越动刀兵。勒逼孤让成都郡,难道要我命不成。

【西皮快板】这几句言语实难听,俱是诸葛定计行。大胆难免把头刎,胆小也要见阎君。走向前来将他问(或:把话论),问他几语(或:问他几句)待怎生。

【西皮快板】此处好比鸿门宴,缺少樊哙保驾臣。孤若不念同宗姓,岂肯容你进都城。

哎呀!

【西皮摇板】两旁武将杀气生,

哎呀!

【西皮摇板】只见严颜老将军。孤命你(或:孤让你)镇守那巴州郡,为什么背孤王降顺他人。

呀呸!

【西皮摇板】孤道你(或:指望你)年迈苍苍忠心耿,却原来背主求荣狗肺心。

(众 让印!)

哎呀!

【西皮摇板】蝼蚁尚且贪性命,不让成都命难生。无奈何取出了先王印,

\textless{}\textbf{哭头}\textgreater{}先王啊,

【西皮摇板】从今后让你掌龙庭(或:掌乾坤)。

呵,还要拜过。

(过去\textbf{,}坐大边外场)

事到如今,但凭你君臣所为。

哦!

【西皮慢板】听说是一声要饯行,好一似狼牙箭攒心。舍不得成都花花美景,实难舍西川老少子民。含悲忍泪换衣巾(或:换衣衿\protect\hyperlink{fn190}{\textsuperscript{190}}),

【西皮原板】辞别了宗兄就要启行。但愿你把曹【转西皮二六】早扫定,但愿你在此享太平。但愿你各国把贡进,但愿你天降福禄亚似个尧君。西川的文武刀刀斩尽,尽都是那贪生怕死臣。王失却西川无怨恨,望宗兄开恩照看孤的这些好子民。

【西皮摇板】到此时他还是假殷勤,花言巧语宽王心。咽喉紧哽跨金蹬,

(刘备 唉,宗兄啊\ldots{}\ldots{})

【西皮摇板】刘备一旁假悲淋。(或:刘备送我假殷勤。)

【西皮摇板】我刘璋不把你别事愿,

【西皮摇板】但愿你后辈的儿孙也照孤样行。

\newpage
\hypertarget{ux767eux5bffux56fe}{%
\subsection{百寿图}\label{ux767eux5bffux56fe}}

\textbf{{[}第一场{]}}

\textbf{管辂 {[}引子{]}乾坤兴衰,日月相连,一卷生平。}

\textbf{管辂
(念)天上星辰日月,人间山水物华。争长论短空嗟呀,还是天伦为大。}

\textbf{管辂
贫道,姓管名辂字公明,乃平原人氏。自幼生就一双慧眼,能知过去、未来之事。在这十字街前,摆了一座卦棚,无非是指引世人。看今日天气晴和,我不免卦棚走走。}

\textbf{管辂
【二黄慢板】叹光阴似箭穿过目烟云,猛抬头见草木又已发青。观前面山岗上松竹茂盛,又看见山坡下花开缤纷。花开时比人生越开越盛,花败落}\protect\hyperlink{fn191}{\textsuperscript{191}}\textbf{比人老迟暮光阴。有等人贪酒色昏迷不醒,有等人为妻妾家业凋零。有等人为财产伤了性命,有等人为小事大祸临身。平安日必须要安守本分,切不可倚势力}\protect\hyperlink{fn192}{\textsuperscript{192}}\textbf{欺压旁人。轻移步来至在十字路径,等候了繁华世痴迷之人。}

\textbf{赵颜 【西皮摇板】在家中遵奉了双亲严命,手牵着青牛儿去把田耕。}

\textbf{赵颜 小生,赵颜。奉了双亲之命,下田耕种。就此走走。}

\textbf{赵颜 【西皮导板】世间人必须要耕种为本,}

\textbf{赵颜
【西皮原板】官出民民出土土内生金。来至在十字路用目观定,卦棚内坐定了算命的先生。放下犁拴上牛卦棚来进,看一看他手托哪部古文。}

\textbf{管辂
【西皮原板】稳坐在卦棚内心中烦闷,观前朝和后世累代帝君:前三皇后五帝尧王传舜,舜传禹、禹传商、商王为君。殷纣王坐江山天心不顺,宠爱妃妲己女残害忠臣。摘星楼摆筵宴比干丧命,黄飞虎反五关去投明君。且不论前朝事用目观定,}

\textbf{管辂 【西皮摇板】猛抬头见小哥令人吃惊。}

\textbf{管辂 可叹呐,可叹!}

\textbf{赵颜 啊,先生叹者何来?}

\textbf{管辂 请问小哥,家住哪里,姓氏名谁?身背犁杖,要往何方?}

\textbf{赵颜 小子赵颜,奉了双亲之命,下田耕种。}

\textbf{管辂
贫道管辂,字公明,能知未来。我劝你休要耕种呃;急速回家,好酒好饭,饱餐三天才是啊。}

\textbf{赵颜 先生你何出此言?}

\textbf{管辂 我看你气色不正,三日后你定夭寿而亡。}

\textbf{赵颜 诶,先生,你看我行路有影,痰嗽有声,怎见得我三日后必死呢?}

\textbf{管辂 唉,小哥啊!}

\textbf{管辂 【西皮摇板】休道你行有影痰嗽有声,岂不知天有那不测风云。}

\textbf{赵颜 【西皮摇板】管先生说此话我却不信,哪有个平白地死了好人。}

管辂 小哥!

\textbf{管辂
【西皮摇板】劝小哥听此话休得不信,待贫道下位去观看五行:耳属金金不能生水半寸,眉属木木生火枝叶凋零。口属水水已干犹如枯井,眼属火火无光是不能生金。鼻属土土入陷死气已真,三日后你必定命赴幽冥。}

\textbf{赵颜
【西皮摇板】听他言不由我心神不定,背转身我这里自己思忖。急忙忙向前去先生来问,三日后我不死有何为凭。}

\textbf{管辂
【西皮摇板】三日后你不死只管议论,如不然我和你去到公厅}\protect\hyperlink{fn193}{\textsuperscript{193}}\textbf{。再不然将我的卦棚拆损,任你羞任你辱任你施行。}

\textbf{赵颜
【西皮摇板】听他言吓得我心意不定,想必是三日后要见阎君。转过身牵犁牛急往家奔,见双亲把此话细说分明。}

\textbf{管辂 【西皮摇板】可叹他少年人大数已尽,这也是五阎君造定死生。}

\textbf{{[}第二场{]}}

\textbf{赵范 【西皮摇板】一家人全凭着耕种为本,小娇儿下田去未见回程。}

\textbf{赵颜 走啊!}

\textbf{赵颜 【西皮摇板】放下犁拴上牛家门来进,见双亲泪汪汪跪在埃尘。}

\textbf{赵颜 喂呀,爷娘啊\ldots{}\ldots{}(哭介)}

\textbf{赵范、赵母 儿啊,为何啼哭?}

\textbf{赵颜
爷娘有所不知,孩儿下田耕种,行至十字街前,有一算命先生,与儿看了一相。他道孩儿三日后夭寿,唉,而亡啊\ldots{}\ldots{}(哭介)}

\textbf{赵范、赵母 不好了!}

\textbf{赵范 【西皮摇板】听说是三日后我儿丧命,}

\textbf{赵母 【西皮摇板】只恐怕绝了我赵氏后根。}

\textbf{赵母 呃,那算命先生姓氏名谁,现在何处?}

\textbf{赵颜 此人姓管名辂,现在十字街前。}

\textbf{赵范、赵母 待我二老前去哀求先生,唉,倘有活命,亦未可知。}

\textbf{赵范、赵母 儿啊,带路!}

\textbf{赵颜 是。}

\textbf{赵范 【西皮摇板】叫妈妈你那里将门栓定,卦棚内去哀告算命先生。}

\textbf{{[}第三场{]}}

\textbf{管辂 【西皮摇板】将身儿来至在卦棚坐定,算人间吉凶事不差毫分。}

\textbf{赵范、赵母 (内)走啊!}

\textbf{赵范 【西皮摇板】教娇儿你与我把路来领,见先生泪汪汪跪在埃尘。}

\textbf{赵范、赵母 哎呀,先生呐!}

\textbf{管辂 【西皮摇板】耳边厢又听得悲声一阵,见二老泪汪汪跪在埃尘。}

\textbf{管辂 赵颜。}

\textbf{赵颜 有。}

\textbf{管辂 这二老是你何人?}

\textbf{赵颜 二老双亲。}

\textbf{管辂 哎呀,年迈之人,快快请起呀。}

\textbf{赵范、赵母 多谢先生!}

\textbf{管辂 你二老到此何事?}

\textbf{赵范、赵母 唉,先生呐!}

\textbf{赵范 【西皮摇板】小老儿名赵范六十三岁,}

\textbf{赵母 【西皮摇板】我二老年半百有此娇生。}

\textbf{赵范 【西皮摇板】先生道我娇儿三日丧命,}

\textbf{赵母 【西皮摇板】望先生发慈悲搭救娇生。}

\textbf{管辂
【西皮摇板】我不是五阎君秦广宫殿,我不是阴曹府掌簿判官。我不是观世音救苦救难,我不是西天佛法力无边。}

\textbf{赵范、赵母 唉呀,先生呐!}

\textbf{赵范 【西皮摇板】我哭、哭一声管先生,}

\textbf{赵母 【西皮摇板】叫、叫一声管辂仙。}

\textbf{赵范 【西皮摇板】为娇儿我二老朝山拜顶,}

\textbf{赵母 【西皮摇板】为娇儿我二老把香来焚。}

\textbf{赵范 \textless{}哭头\textgreater{}管先生,}

\textbf{赵母 \textless{}哭头\textgreater{}仙长爷,}

\textbf{赵范、赵母 \textless{}哭头\textgreater{}啊,先生呐。}

\textbf{管辂
【西皮摇板】见二老只哭得我心好惨,不由人一阵阵心内痛酸。这时候怎救得残生命转,}

\textbf{管辂 哦,有了!}

\textbf{管辂
【西皮摇板】又只见南北斗已奔}\protect\hyperlink{fn194}{\textsuperscript{194}}\textbf{高山。}

\textbf{管辂 二老请起。}

\textbf{赵范、赵母 多谢先生!}

\textbf{管辂 赵颜有了救了。}

\textbf{赵范、赵母 救在哪里?}

\textbf{管辂 回到家去,准备鹿脯美酒,去至终南山,有二\ldots{}\ldots{}}

\textbf{赵颜 先生,二什么?}

\textbf{管辂
有二位仙长在那里着棋,你将这鹿脯美酒,暗暗献上。他饮了你的酒,必要与你添寿。}

\textbf{赵范、赵母 请问先生,这二位仙长怎样打扮?}

\textbf{管辂 你们听了!}

\textbf{管辂
【西皮摇板】有一个穿白袍斯文体相,有一个穿红服气宇轩昂。你将这鹿脯酒暗暗献上,饮了酒必与你添寿绵长。}

\textbf{赵范、赵母 先生!}

\textbf{赵范 【西皮摇板】辞别了管先生忙往家奔,}

\textbf{赵母 【西皮摇板】准备下鹿脯酒送到山林。}

\textbf{赵颜 【西皮摇板】辞别了管先生忙回家门,}

\textbf{管辂 转来!}

\textbf{赵颜 【西皮摇板】问先生唤回我所为何情。}

\textbf{管辂 【西皮摇板】你把这鹿脯酒暗暗献上,必须要隐身形跌跪一旁。}

\textbf{赵颜
【西皮摇板】仙长爷你不必仔细叮咛,我赵颜纵一死不忘大恩。此一番到南山把酒来敬,见仙长求增寿小心殷勤。}

\textbf{管辂
【西皮摇板】这也是小赵颜不该命丧,他二老前世里积下善良。哀告那南北斗将寿添上,到后来子孙多瓜瓞绵长。}

\textbf{{[}第四场{]}}

\textbf{南斗、北斗 (内)请啊!}

\textbf{南斗 【西皮摇板】观天地和日月乾坤浩荡,}

\textbf{北斗 【西皮摇板】水连天天连水渺渺茫茫。}

\textbf{南斗、北斗 吾乃南、北斗星君是也。}

\textbf{南斗 星君请了。}

\textbf{北斗 请了。}

\textbf{南斗
你我奉了玉帝敕旨,巡查人间善恶。来此已是南瞻部洲,今日闲暇无事,将历代君王之事,细表一番。}

\textbf{南斗 请------}

\textbf{南斗 【西皮原板】自盘古分天地乾坤始创,}

\textbf{北斗 【西皮原板】先太极分两仪八卦阴阳。}

\textbf{南斗 【西皮原板】按金木水火土五行方向,}

\textbf{北斗 【西皮原板】先君臣后父子三纲五常。}

\textbf{南斗 【西皮原板】尧传舜舜传禹天下揖让,}

\textbf{北斗 【西皮原板】夏桀暴商纣淫自取灭亡。}

\textbf{南斗 【西皮原板】秦始皇归一统山河执掌,}

\textbf{北斗 【西皮原板】他不该焚诗书兴建阿房。}

\textbf{南斗 【西皮原板】楚霸王他倒有帝王之相,}

\textbf{北斗 【西皮原板】他不该杀义帝强霸为王。}

\textbf{南斗
【西皮原板】把前朝君王事【转西皮快板】暂且慢讲,有一辈忠良臣细说端详:淮阴侯小韩信功高智广,为什么未央宫一命身亡。}

\textbf{北斗
【西皮快板】休道那小韩信功高智广,他不该活埋母九里山旁。他不该问道路把樵哥斩丧,他不该逼高祖拜他为王。他不该逼霸王乌江命丧,因此上未央宫一命身亡。}

\textbf{南斗 【西皮快板】成萧何败萧何萧何该丧,为什么那老儿寿命延长。}

\textbf{北斗 【西皮快板】休道那汉萧何该当命丧,他本是忠良臣寿命延长。}

\textbf{南斗 【西皮快板】叹不尽前朝的忠臣良将,}

\textbf{北斗 【西皮摇板】松林内摆棋盘散闷一场。}

\textbf{赵颜 走啊!}

\textbf{赵颜
【西皮摇板】手捧着鹿脯酒终南山上,又只见二仙长分坐两旁。我这里将鹿脯暗暗献上,吞着气躲着身跌跪一旁。}

\textbf{南斗、北斗 请呐!}

\textbf{南斗 【西皮原板】在石台摆棋盘一帅一将,}

\textbf{北斗 【西皮原板】红棋先黑棋后各霸一方。}

\textbf{南斗 【西皮原板】走一步当头炮千军难挡,}

\textbf{北斗 【西皮原板】还一个连环马士相奔忙。}

\textbf{南斗 【西皮原板】又只见鹿脯酒从空而降,}

\textbf{北斗 【西皮原板】想必是天赐我美味清香。}

\textbf{南斗 【西皮原板】我和你棋不胜共饮佳酿,}

\textbf{北斗 【西皮摇板】飞来物饮几杯又有何妨。}

\textbf{南斗 你我再下一盘。}

\textbf{北斗 请啊!}

\textbf{南斗 【西皮摇板】战胜了好一似汉高皇上,}

\textbf{北斗 【西皮摇板】战败了好一似西楚霸王。}

\textbf{赵颜 求寿啊!}

\textbf{南斗 【西皮摇板】耳边厢又听得有人喧嚷,}

\textbf{北斗 【西皮摇板】猛抬头见小子跌跪道旁。}

\textbf{北斗 那一小子,家住哪里,姓氏名谁,到此何事?慢慢讲来。}

\textbf{赵颜 二位仙长容禀!}

\textbf{赵颜
【西皮摇板】家住在城厢外绿柳村上,我的名叫赵颜耕种田庄。都只为在卦棚先生看相,他道我三日后一命身亡。}

\textbf{南斗 哦!}

\textbf{南斗 【西皮摇板】见小子说此话倒也响亮,}

\textbf{北斗 【西皮摇板】仙家事是何人泄漏阴阳。}

\textbf{南斗 啊,星君,你我在此着棋,凡人怎能知晓?}

\textbf{北斗
星君有所不知,只因凡间有一个管辂,生就一双慧眼,能知人间过去未来之事,想是他指引前来,也未可知。}

\textbf{南斗 星君何不将他阳寿查上一查。}

\textbf{北斗 待我查来。}

\textbf{北斗
查得山西平原郡绿柳村赵范之子,名唤赵颜,前生作恶多端,今投赵门为子。注定大汉建安一十二年,寿活一十九岁,夭寿而亡。}

\textbf{北斗 赵颜,你今年多大了?}

\textbf{赵颜 一十九岁。}

\textbf{南斗 嘿嘿,完了!}

\textbf{北斗
【西皮摇板】叫小子抬头看生死簿上,这上面造定了字字行行。十九岁你就该把命夭丧,这时候并无有解救良方。}

\textbf{赵颜 不好了!}

\textbf{赵颜
【西皮摇板】听他言吓得我魂魄飘荡,不由我小赵颜无有主张。望仙长发慈悲将寿添\textless{}哭头\textgreater{}上,}

\textbf{赵颜 【西皮摇板】可怜我家还有二老爷娘。}

\textbf{南斗、北斗 哦!}

\textbf{南斗 【西皮摇板】小赵颜只哭得泪如雨降,}

\textbf{北斗 【西皮摇板】可怜他家还有二老爷娘。}

\textbf{南斗 啊,星君!看这赵颜哭得可怜,何不将他阳寿与他添上。}

\textbf{北斗
星君说哪里话来。你我奉了玉帝敕旨,巡查人间善恶。私添阳寿,玉帝闻知,吃罪不起。}

\textbf{南斗 上苍也有好生之德。何况你我\ldots{}\ldots{}}

\textbf{北斗 这是你吃了好酒!}

\textbf{南斗 你也好贪杯!}

\textbf{北斗 彼此?}

\textbf{南斗 一样!}

\textbf{南斗、北斗 啊,呵呵哈哈哈\ldots{}\ldots{}(笑介)}

\textbf{北斗星 如此说来,这阳寿添得的?}

\textbf{南斗星 添得的!}

\textbf{北斗星
赵颜,你命活一十九岁,将这``一''字改为``九''字。寿活九十九岁,也就够了。}

\textbf{赵颜 啊,仙长,将那一岁添上,岂不是百岁老人?}

\textbf{南斗、北斗 诶------贪心不足。听我等道来:}

\textbf{南斗 【西皮原板】我本是南斗星从空而降,}

\textbf{北斗 【西皮原板】我本是北斗星降下天堂。}

\textbf{南斗 【西皮原板】我掌生他掌死分毫不爽,}

\textbf{北斗 【西皮原板】查人间生和死善恶昭彰。}

\textbf{南斗 【西皮原板】我赐你子孙多富贵永享,}

\textbf{北斗 【西皮原板】我赐你财源盛金玉满堂。}

\textbf{南斗 【西皮原板】我赐你椿萱茂代代兴旺,}

\textbf{北斗 【西皮原板】我赐你一家人无有灾殃。}

\textbf{南斗 【西皮原板】我赐你百寿图悬挂堂上,}

\textbf{北斗 【西皮原板】我赐你九十九大寿延长。}

\textbf{南斗 【西皮原板】在人间休得要胡言乱讲,}

\textbf{北斗 【西皮摇板】泄露了仙家事五雷身亡。}

\textbf{南斗、北斗 去罢!}

\textbf{赵颜 【西皮摇板】谢罢了二仙长忙下山岗,}

\textbf{南斗、北斗 转来!}

\textbf{赵颜 【西皮原板】星君爷唤回我所为哪桩。}

\textbf{南斗
回去见了管辂,叫他从今以后,不要胡言乱语;再若胡言乱语,难免五雷殛顶。那旁有人来了。}

\textbf{赵颜 在哪里?}

\textbf{赵颜 哎!二位仙长不见,待我望空一拜。}

\textbf{赵颜 【西皮原板】手捧着百寿图忙下山林,回家去见爷娘细说分明。}

\newpage
\hypertarget{ux5b9aux519bux5c71-ux4e4b-ux9ec4ux5fe0}{%
\subsection{\texorpdfstring{定军山\protect\hyperlink{fn195}{\textsuperscript{195}}
之
黄忠}{定军山195 之 黄忠}}\label{ux5b9aux519bux5c71-ux4e4b-ux9ec4ux5fe0}}

{[}第一场{]}

慢着。

黄忠来也。

参见军师。

谢军师呃。

军师呃。攻取葭萌关,何劳三千岁。赐末将一哨人马,生擒那张郃入帐呃。

\textless{}\textbf{叫头}\textgreater{}军师呃。

(右手弹髯口,跨左腿,反云手,骗右腿,拱手从左向右转身(腰))

(念)末将年迈勇,血气贯长虹。杀人如削土,跨马走西东。两膀千斤力,能开铁胎弓。若论交锋事,还算老黄忠。

得令!

【西皮二六】师爷说话言太差,不由得黄忠怒气发。一十三岁习弓马,威名镇守在长沙。自从归顺皇叔爷的驾,匹马单刀取过了巫峡。抢关夺寨功劳大,师爷不信你在功劳簿上查一查。不是我黄忠夸大话,

弓来!

【西皮快板】铁胎宝弓手中拿。满满搭上朱红扣,帐下的儿郎把咱夸。

【西皮快板】二次运动这千斤的力,

【西皮散板】人有精神气又加。

【西皮散板】三次开弓秋月样,

【西皮散板】再与师爷把话答。

得令。

在。

得令。

【西皮摇板】黄忠接令把帐下,

(严颜 【西皮摇板】不由严颜笑哈哈。)

【西皮摇板】一不用战鼓这嗵嗵地打,

(严颜 【西皮摇板】二不用副将把队押。)

【西皮摇板】事不宜迟把马跨,

马来呃!

(严颜 【西皮摇板】\ldots{}\ldots{}把张郃一马踏。)

{[}第二场{]}

(罢了,)一旁坐下。

可曾与那贼会过阵来(或:见过阵来)。

(严颜 且慢,军家胜败,古之常理啊。)

(严颜 老将军开恩。)

还不谢过严老将军。

陈式听令啊,城头之上高扯红旗二面,上写黄忠、严颜。那贼闻名丧胆。

再探。

老将军,张郃小儿他来了啊!

(严颜 你我会他一会。)

(一派胡言。)

带马。

{[}第三场{]}

老夫黄忠

(严颜 严颜。)

尔为何发笑?

一派胡言,放马过来。

{[}第四场{]}

老将军追赶何人?

那贼去远了。

韩浩、夏侯尚,便宜了他们。

老将军抬头观看。

老将军,看前面已是天荡山,乃曹操屯粮之所。此山不破,你我大功难成。

老将军你有何妙计?

老将军妙计!(或:此计甚好,你我一同传令。)

众将官。

照计而行。

{[}第五场{]}

【西皮快板】背地里暗笑诸葛亮,他道老夫少刚强。虽然年迈精神爽,杀人犹如宰鸡羊。催马来在阵头上,

【西皮摇板】那旁来了送死郎。

来将通名。

(韩浩 韩浩。)

你来则甚?

(韩浩 替兄报仇。)

放马过来。

【西皮快板】走投鱼儿入罗网,败阵绵羊敢逞强。老夫倒有容人量,怎奈宝刀世无双。眼前若有诸葛亮,

【西皮摇板】管教他含羞带愧他的脸无光。

{[}第六场{]}

(念)大将军八面威风。

有请。

在。

得令。

后帐留宴。

带马。

老将军,你我一笑而别了哇。呵呵哈哈\ldots{}\ldots{}(笑介)

带马。

{[}第七场{]}

请。

一赖\protect\hyperlink{fn196}{\textsuperscript{196}}主上洪福,二赖先生妙算。末将(或:老臣)何功之有。

不敢呐不敢。

慢着。(或:且慢呐)

军师啊,攻取定军山,何劳二千岁远路而来。赐末将一哨人马,生擒那夏侯渊入帐呃。

军师呃,想那张郃乃中原有名上将,被末将杀得是望风而逃,何况那夏侯渊,乃一勇之夫(或:是一勇之夫)。

也罢。

俺若(或:倘某;我若)胜不过那夏侯渊,愿输项上的人头。

打赌啊,

得罪了!

在。

得令。

哦!

【西皮二六】在黄罗宝帐领将令,气坏了老将黄汉升。某昔年镇守长沙郡,偶遇云长(或:圣贤)二将军。某中了他人的拖刀计,我的百步穿杨射他的盔缨。弃暗投明【转西皮快板】来归顺,食王的爵禄当报王的恩。孝当竭力忠尽命,再与师爷把话论。一不用战鼓嗵嗵打,二不用副将随后跟。只要我黄忠一骑马,匹马单刀取定军。十日之内攻得胜\protect\hyperlink{fn197}{\textsuperscript{197}},军师的大印付与某的身。十日之内不得胜,愿将人头挂营门。来来来带过爷的马能行,

【西皮摇板】我要把定军山一扫平。

{[}第八场{]}

【西皮快板】吾主爷攻打葭萌关,将士纷纷取东川。可笑军师见识浅,道我难胜那夏侯渊。张郃被某杀破胆,卸甲丢盔奔荒山。坐至在雕鞍将令传,大小儿郎听爷言:埋鹿角、掘沟堑,金晶铠甲扣连环。上前个个把功建,退后的人头挂高竿。大吼一声催前站,

【西皮散板】十日之内取东川。

(唱``取东川'',右手马鞭,反云手,左手拉开平亮,正亮在``川''字上。再反云手,抬右腿,推左掌(向左斜场),抱鞭,跨右腿,转身,甩髯口,左脚找地方,左腿弓步,觑地,勒马,亮相,拿神。\textless{}\textbf{四击头}\textgreater{}接\textless{}\textbf{急急风}\textgreater{}。倒脚活法儿,起身,抬右腿,向右圆着走三步,反云手转身,跨左腿,打马出右腿(面外),向左甩髯口,左手勒马,抬头,\textless{}\textbf{八嗒仓}\textgreater{}亮相(面对下场门)。\textless{}\textbf{急急风}\textgreater{},下)

{[}第九场{]}

【西皮快板】(两军对垒动干戈,一来一往战几合。)夏侯渊打扮真不错,黑面长须似阎罗。劝你马前归顺我,宝刀下去尔的命难活。

{[}第十场{]}

【西皮快板】两家交锋来会过,一来一往动干戈。魏营打罢得胜的鼓,

【西皮摇板】我营缘何不鸣锣。

再探。

【西皮散板】听一言来心冒火,不由老夫咬牙车\protect\hyperlink{fn198}{\textsuperscript{198}}。人来与爷马带过。

(\textless{}\textbf{叫头}\textgreater{},右手扣腕横刀,左手弹髯口举手,看夏侯尚)

哈哈,

(顺手于左侧,双手握刀,刀头冲外,看刀)

哈哈,

(横刀,面向外亮)

啊哈哈哈\ldots{}\ldots{}(笑介)

(退步,大刀花)

{[}第十一场{]}

【西皮快板】夏侯渊武艺果然好,可算得中原将英豪。将身且坐(或:将身来在)宝帐到,

【西皮摇板】营外缘何闹吵吵。

传。

(罢了,)奉何人所差?

书信呈上,下去。(或:下面伺候。)

夏侯渊来的书信,待我拆开一观。

唤下书人。

回去言讲(或:回覆夏侯将军),就说老夫修书不及,照书行事。

且住。老夫正在营中无计可施,夏侯渊这封书信来得是将将凑巧啊,明日午时三刻与老夫(或:约定老夫明日午时三刻)走马换将。那时先教他放过我国先行陈式,然后再放他侄男夏侯尚。习就百步穿杨,将他侄男一箭射死。那夏侯渊必定带领人马与他侄男报仇。那时老夫杀一阵,败一阵,败至在荒郊。习学关公拖刀之计,将他斩于马下。

夏侯渊呐,我的儿啊。你若来时,定中老夫拖刀之计也。

【西皮快板】这一封书信来得巧,天助黄忠成功劳。站立在辕门传令号,大小儿郎听根苗:一通鼓战饭造,二通鼓紧战袍。三通鼓刀出鞘,四通鼓把兵交。向前个个俱有赏,退后项上吃一刀。就此与爷归营号,

【西皮散板】到明天午时成功劳。

{[}第十二场{]}

请了。

正为书信而来,但不知哪家先放?

呃,老夫到此,乃是客位,自然是你家先放。

老夫若有二意,日后死在药箭之\ldots{}\ldots{}

焉有不放之理。

来,将夏侯尚放了过去。

弓箭伺候。

夏侯尚,看箭。

{[}第十三场{]}

且住,夏侯渊来得厉害。再若来时,拖刀计伤他。

(向左转半圈,横握刀,\textless{}\textbf{冲头}\textgreater{},\textless{}\textbf{叫头}\textgreater{},向右转身上步,右手横握刀,左手举起,抬头)

哈哈,

(右手背刀,刀头在下,撤右步,左手托髯口,左弓步,冲左斜场,面对外)

哈哈,(此处左臂撑开,手对左腿马面,右臂伸直,手对右胯,两腕着力,身上不能使劲)

(收左步面右前,丁字步,横握刀,亮)

啊呵呵哈哈哈\ldots{}\ldots{}(笑介)

(退步,大刀花)

\newpage
\hypertarget{ux9633ux5e73ux5173-ux4e4b-ux9ec4ux5fe0}{%
\subsection{阳平关 之
黄忠}\label{ux9633ux5e73ux5173-ux4e4b-ux9ec4ux5fe0}}

{[}第一场{]}

哈哈,哈哈,啊呵呵哈哈哈\ldots{}\ldots{}(笑介)

老臣奉命,斩得夏侯渊首级,特来献上。

号令辕门。

臣,

谢主隆恩。

老臣怎敢?

折煞老臣了。

主公请。

黄忠愿往。

食王爵禄,当报王恩,何言``劳倦''二字。

俺今出马,立斩张郃头来,四将军以为如何?

四将军。

【西皮二六】讲什么军家无有常胜,仔细看一看我黄汉升。黄忠今年七十整,还要在阵前抖一抖老精神。一马直将曹营进,恰好似猛虎入羊群。眼前若有军师令,看看我老而------我是能不能。

【西皮摇板】长吾黄忠整几春。

【西皮摇板】躬身施礼请将令,

【西皮摇板】差池甘当军令行。

【西皮摇板】黄忠越老越好胜,

马来呃。

【西皮摇板】他那里越欺越侮我偏要行。

{[}第二场{]}

【西皮导板】亦非人前夸老硬,

【西皮快板】胸中韬略亘古今。我在宝帐领将令,赵云道我老无能。任他兵来如潮涌,任他人马似秋云。杀得他血流人头滚,杀得他尸横遍野马难行。洋洋得意朝前进,

啊?

【西皮摇板】张著赶来必有因。

将军赶来则甚?

哦,想是军师以我不能成功,要你赶我回去不成么?

啊,哈哈哈哈\ldots{}\ldots{}(笑介)

军师可谓知我者也。

将军既来相助,你我今晚三更饱餐,四更时分,去至北山脚下,烧贼的粮草。那张郃必然前来营救,就而擒之,岂不美哉?

你我一同传令。

众将官,北山去者。

{[}第三场{]}

【西皮导板】越杀越勇精神好,

【西皮快板】层层密密似涌潮。我若今日遭圈套,一世英名任笑嘲。抖擞精神往前蹈\protect\hyperlink{fn199}{\textsuperscript{199}},

(赵云 【西皮摇板】好似天神下九霄。)

哦,四将军你来了。

杀啊!

\newpage
\hypertarget{ux4f10ux4e1cux5434}{%
\subsection{伐东吴}\label{ux4f10ux4e1cux5434}}

{[}第一场{]}

黄忠
(念)英雄回首忆长沙,百战威名逞虎牙\protect\hyperlink{fn200}{\textsuperscript{200}}。

黄忠 圣上宣召。

黄忠 一同进帐。

黄忠 臣等见驾。

黄忠 愿吾皇万岁,(万岁,)万万岁。

黄忠 宣臣等进帐有何旨意?

黄忠 领旨。

{[}第二场{]}

刘备
【西皮原板】风吹旌旗山岳动,关兴、张苞出御营。未知此去可得胜。举首翘望心不宁。

黄忠
【西皮原板】忆昔当年长沙镇,算来不觉有数春(或:转眼不觉数十春)。荆襄、阆中遭不幸,一心要把东吴平(或:吾主爷要把东吴平)。黄汉升撩袍御营进,

刘备 【西皮原板】老将军免礼且平身。暂陪朕坐消愁闷,

黄忠 【西皮原板】行兵不必泪伤心(或:兴兵不必泪常涔)。

张苞 【西皮摇板】斩将擒贼破敌阵,

关兴 【西皮摇板】弟兄御前显奇能。

张苞
启禀皇伯,儿臣出阵,不料谭雄暗放雕翎,射死战马;幸得关兴赶到,不然性命难保。

关兴 儿臣见张苞兄长落马,赶到阵前,刀劈谢旌,活捉谭雄,特来交令。

刘备 快将谭雄绑了上来!

刘备 好吴狗!

刘备
【西皮散板】四百年来争汉鼎,东吴不君也不臣。鼠窃犬偷真堪恨,快斩逆贼立即行。

黄忠 号令辕门。

刘备 将这厮首级祭奠二千岁灵前;洒下热血,以祭死马。搭了下去。

众 啊------

刘备 朕今兵发东吴,与二位贤弟报仇,幸得二虎侄头阵取胜,惊破吴人之胆。

刘备 左右,看酒。与二位皇侄贺功。

内侍 是。

黄忠 嗯哼。(黄忠痰嗽介)

刘备 哦------老将军你也来呀。

黄忠 臣领旨。

刘备
【西皮原板】庆贺功劳把酒饮,想起了当年破黄巾。战贼挽手威风凛,虎牢关前显威名。

刘备
想当年与尔父等桃园结义之后,破黄巾、得徐州、收襄阳、入西川,皆尔父等之力也,今他们一旦去世啊,所有当年之将,尽是些老迈无用。

刘备 幸有二皇侄斩将破敌,如此英勇,何愁东吴不平。

刘备 看酒来,朕亲为二皇侄贺功。

刘备 【西皮摇板】幸喜皇侄多英俊,此酒酬劳庆功勋。

黄忠 老了哇,老了哇!

黄忠
【西皮散板】主公说话不思忖,他道老将便无能。(或:主公说话欠思忖,怎知老将便无能。)

黄忠
且住,关兴、张苞子侄之辈,阵前擒来谭雄,无非是些许功劳。主公帐中(或:主公隆宠,)夸了又夸,讲了又讲。反讲当年五虎上将尽是些老迈无用。这这这\ldots{}\ldots{}

黄忠
也罢!我不免(或:俺不免)去至两军阵前,斩那东吴八员上将,看看俺黄忠老是不老。

黄忠 【西皮快板】太公八十方交运,廉颇七旬挡秦军。黄忠年迈有本领,

黄忠 【西皮摇板】再学走马取定军。

报子 黄老将军私自出营,人向东而去。

刘备 快去打探。

探子 得令。

刘备
哎呀且住。黄汉升绝非叛逆之人,想是适才朕言老将无能,故而一怒出营,意在斩将显能尔。既然如此,诚恐有失。

刘备 关兴、张苞,

关兴、张苞 在

刘备 命你二人急速前去保护。倘若老将军得胜,劝他回营,不得有误。

关兴、张苞 得令。

刘备 将宴撤去。

刘备
【西皮摇板】得意忘形错是朕,激怒老将黄汉升。但愿他马到成功呃早得胜,平安无事转回程。

{[}第三场{]}

黄忠 【西皮导板】黄忠马上呵呵笑,

黄忠 哈哈,哈哈,啊呵呵哈哈\ldots{}\ldots{}(笑介)

黄忠
【西皮快板】主公帐中论英豪。溺爱不明夸年少,反道老将无略韬。只要杀人胆量好,哪怕胡须似银条。催马来在阳关道,

关兴、张苞 老将军慢走。

黄忠 【西皮摇板】二小将赶来为哪条。

关兴、张苞 老将军且慢。

黄忠 二位小将赶来则甚?

关兴、张苞
(我等奉了)皇伯之命,请老将军回营,(诚恐)年迈有失。\protect\hyperlink{fn201}{\textsuperscript{201}}

黄忠 呀呸!

黄忠
【西皮快板】二小将把话讲差了,讲什么阵前把命抛。我也不图凌烟标,恢复汉室锦皇朝。见了主公好言告,你就说年迈的黄忠要立功劳。

张苞 【西皮摇板】黄忠年迈性情傲,

关兴
【西皮摇板】(相随)保护莫辞劳。\protect\hyperlink{fn202}{\textsuperscript{202}}

{[}第四场{]}

吴班
【西皮摇板】大将出川把贼剿,挂印先行不辞劳。连营下寨恐非妙,见机而行稳重高。

报子 报!

报子 黄老将军到。

吴班 有请。

吴班 啊,老将军。

黄忠 哼!

吴班 黄老将军,怒气不息,为着何来?

黄忠 呃!

黄忠 \textless{}\textbf{叫头}\textgreater{}吴将军,

黄忠
想那关兴、张苞乃子侄之辈,阵前擒来谭雄,无非是些许功劳。主公帐中(或:主公隆宠,)夸了又夸,讲了又讲。反讲当年五虎上将尽是些老迈无用。你道恼是不恼?

吴班 哎,本来的老了哇。

黄忠 啊?(或:呀呸!)

黄忠 【西皮摇板】为什么人人道我老哇,

吴班 唉,本来是老了。

黄忠 呀呸!

黄忠
【西皮快板】不由怒气上眉梢。吾十岁(或:某十岁)弓马颇知晓,十三、十四使宝刀。交锋对垒有多少,数十年未离马鞍鞒。战长沙已然须发皓,取东川谁不道我是英豪。我也曾天荡、定军一齐扫,夏侯渊一命赴阴曹。到如今八十三岁何曾老,我是哪些儿老,

吴班 老将军本来是老了啊。

黄忠 呃!

黄忠 【西皮快板】年迈也要逞英豪。来来来与爷带马到,斩几个人头你瞧一瞧。

吴班 【西皮摇板】老将人老心不老,

吴班 带马。

吴班 【西皮摇板】暗地保护走一遭。

{[}第五场{]}

崔禹、史蹟 俺,东吴大将------

崔禹 崔禹。

史蹟 史蹟。

崔禹
我等奉了吴侯旨意,镇守猇亭。探子报道,黄忠前来讨战,你我二人前去会他一会。

史蹟 请。

黄忠 来将通名!

崔禹 哼,连你家老爷东吴大将崔禹全不认识?

黄忠 通上名来。

崔禹 某乃东吴大将崔禹。

史蹟 俺乃史蹟。哇呀呀呀\ldots{}\ldots{}你还不跑?

黄忠 诶呀!我道是东吴八员上将,原来是两个无名的狗头。

崔禹、史蹟 什么狗头,这是人头。

黄忠 饶尔等不死,去罢!

崔禹、史蹟 什么人头、狗头的,你这老头儿叫什么名字?

黄忠 老夫黄忠。

崔禹 咦咦咦,

史蹟 哇啊啊------

崔禹、史蹟 呵呵哈哈哈\ldots{}\ldots{}(笑介)

黄忠 尔为何发笑?

崔禹、史蹟 呵呵黄忠啊,我道是天神下界,原来是一个老倭瓜。

黄忠 休得胡言,快(快)教那潘璋出马,饶尔等不死。去罢!

崔禹 我这里待我耍个``提枪花'',摘就一个老倭瓜。

崔禹、史蹟 呵呵将军呐。

(崔禹
人道黄忠乃是好将,未战两个回合,他为何败下阵去。)\protect\hyperlink{fn203}{\textsuperscript{203}}

史蹟 想是不忍杀害于你。

崔禹 哼,休得胡言,你我赶上前去。定然死在他手。

{[}第六场{]}

黄忠 啊?!

吴班 老将军刀劈崔禹、史蹟,就是莫大之功,可以回营交令了。

黄忠
俺要去至吴营,斩那东吴八员上将,看看我黄忠老是不老(或:俺要去至吴营,斩那东吴八员上将,方显我黄忠不老)。

吴班 唉,老将军呐,

吴班
【西皮摇板】老将军威风谁不晓,破敌须防战马劳。\protect\hyperlink{fn204}{\textsuperscript{204}}

黄忠 吴将军,

黄忠
【西皮摇板】这几句话儿讲得好,黄忠的怒气一半消。回营报功呃休取笑,暂且饶他这一宵。

吴班
老将军不如请暂回师。\protect\hyperlink{fn205}{\textsuperscript{205}}

黄忠
(啊)吴将军,你我今日暂回大营(或:你我暂且回营),教那些吴狗们多活上一夜。

吴班 是啊,教他们多活上一夜。

黄忠 便宜了他们。

吴班 呃,便宜了他们。

黄忠 啊吴将军,你看我黄忠老是不老?

吴班 呃,将军么------嗯,不老。

黄忠 嗯,不老?

吴班 呃,不老。

黄忠、吴班 啊,呵呵哈哈哈\ldots{}\ldots{}(笑介)

{[}第七场{]}

潘璋
【西皮摇板】探马不住急来报,黄忠斩我两英豪。\protect\hyperlink{fn206}{\textsuperscript{206}}

潘璋
俺,潘璋。前者同吕蒙定计袭取荆州\protect\hyperlink{fn207}{\textsuperscript{207}},我主大喜,将关羽刀、马赐俺,赤兔马不食草料而死;青龙刀虽在我手,却未斩一将。适才探子报道,黄忠踏营,岂肯容他张狂,待俺擒他便了。

黄忠 来将通名。

潘璋 东吴大将潘------

黄忠 潘什么?

潘璋 潘璋。

黄忠 啊!

黄忠
【西皮快板】一见潘璋把牙咬,手持青龙偃月刀。怎不教人珠泪掉,斩尔的狗头马后捎。

{[}第八场{]}

马忠
(念)旌旗飞龙影,干戈耀日月。\protect\hyperlink{fn208}{\textsuperscript{208}}

马忠 俺,马忠。只因潘璋出营,大战黄忠,不知胜负如何,俺且出营一望。

马忠 将军胜负如何?

潘璋 黄忠十分骁勇,难以取胜。

马忠 将军且退后阵,待俺前去会他。

潘璋 多加小心。

{[}第九场{]}

潘璋 将军。

马忠 将军。

潘璋 你我被黄忠杀败,主公降罪如何是好?

马忠 黄忠虽然骁勇,潘将军你且与他交战,待俺暗放一箭。

潘璋 黄忠善射,百步穿杨,若是射他不中,只恐你``画虎不成反类其犬''。

马忠 岂不知``会家不防''?

潘璋 既然如此,待俺再会他一阵。

马忠 须要小心。

{[}第十场{]}

黄忠 【西皮导板】黄忠今日遭圈套,

黄忠 【西皮快板】中了奸贼计笼牢(或:谅我插翅也难逃)。

黄忠 【西皮快板】大将临危有神保,

关兴、张苞 【西皮快板】来了关兴和张苞。

(潘璋
黄忠带箭,被二小将救出重围,你我速速追赶。)\protect\hyperlink{fn209}{\textsuperscript{209}}

马忠 赶上前去。

{[}第十一场{]}

刘备
【西皮摇板】黄忠性傲见识浅,不该匹马去争先。张苞、关兴料难劝,但愿平安得胜还。

黄忠
(念)\textless{}\textbf{金钱花}\textgreater{}渭城朝雨清尘、清尘;轮台古月黄云、黄云。催花羯鼓去从军。枕头上,别情人;刀头上,做功臣。

刘备 【西皮散板】一见老将身带箭,霎时胆落百丈渊。早知出兵遭凶呃险,

刘备 \textless{}\textbf{哭头}\textgreater{}将军呐------

刘备 【西皮摇板】朕悔一时错出言。

黄忠 【西皮散板】精神恍惚四肢软,耳旁又听有人言。大骂潘璋休弄呃险,

刘备 老将军!

黄忠 【西皮散板】只见主公在眼前。急忙叩谢龙恩典,黄忠的性命难保全。

刘备
唉呀老将军呐,朕一言之错,使你怒出大营,如今带箭而归,教朕痛断肝肠了哇啊\ldots{}\ldots{}

黄忠 哎呀主公啊,老臣出马,刀劈史蹟、崔禹------

刘备 就该回营。

黄忠
因见吴狗潘璋手持二君侯青龙宝刀,老臣一见,肝胆俱裂。正要擒贼下马,不想贼营暗放冷箭,中臣肩窝。

刘备 啊------

刘备 啊,老将军乃是善射的能手,为何不防?

黄忠 哎呀陛下呀。

黄忠
【西皮散板】老臣智不如王翦,临阵怎敢不当先。况且仇人两相见,心急哪顾听弓弦(或:哪有闲心听弓弦)。

黄忠 此乃老臣自不小心。

刘备
【西皮散板】真是风云不测变,空将血泪洒胸前。回头便把小将怨:年轻无知小儿男。

关兴、张苞 儿臣等知罪。

黄忠 陛下,此乃臣自不小心,休要埋怨二位小将军。

刘备 既然如此,待朕与老将军起箭。

黄忠 哎呀万岁呀,这箭上有药,箭在------臣在,这箭去------臣亡。

刘备
老将军带箭不起,那敢是怕痛?\protect\hyperlink{fn210}{\textsuperscript{210}}

黄忠 老臣死且不惧,焉能畏痛?一言永别,伏乞圣听:

黄忠
【反西皮二六】平生今洒泪几点,回首功名八十年。主上待臣恩非浅,粉身碎骨理当然。幸得全尸已无怨(或:幸得全身已无怨),叩谢圣恩归九泉。万岁须当谋虑远\protect\hyperlink{fn211}{\textsuperscript{211}}(或:主上须当韬略远;主上须当\textless{}\textbf{哭头}\textgreater{}谋略远),

黄忠 【西皮散板】平吴不及定中原。

刘备
【西皮散板】老将军休得心惊战,起箭医疗早愈痊。康复之后功臣宴,愿你康宁寿百年。

黄忠
【西皮散板】见主公说话(或:见主公只哭得)泪满面,关兴、张苞哭两边。大丈夫一死终难免,强打精神假流连。

刘备 【西皮散板】事到临头难挽转,张苞、关兴听朕言:

刘备 关兴、张苞,搀扶老将军,待朕与老将军起箭。

黄忠 且慢呐,大将取箭,不用人搀,待老臣自取。

黄忠 闪开了!

黄忠 唉呀------

刘备
【西皮散板】一见老将归九天,冷水浇头落空潭。从今何处再相见,\protect\hyperlink{fn212}{\textsuperscript{212}}

刘备 \textless{}\textbf{哭头}\textgreater{}老将军呐------

刘备 【西皮散板】热泪行行洒征衫。

张苞 【西皮散板】大将尸全世少见,

关兴 【西皮散板】皇伯不必损龙颜。

张苞 【西皮散板】尸首后帐好收殓,

关兴 【西皮散板】准备灭吴报仇冤。

刘备 【西皮散板】五虎大将三不见呐,

(刘备 \textless{}\textbf{三叫头}\textgreater{}汉升!二弟,三弟呀!)

刘备 【西皮散板】休想古城再团圆。黄忠有灵当应显,踏平东吴在眼前。

刘备 【西皮散板】张苞、关兴传令箭,

刘备 拿潘璋------

刘备 【西皮散板】刀出鞘来弓上弦。

\textbf{*王荣山教《定军山》、《阳平关》和《伐东吴》大刀把子}

王荣山说《定军山》、《阳平关》和《伐东吴》三出黄忠戏戏情不同,大刀把子不同,很明显的是《阳平关》有大战,打挡棒攒,其它两出没有,实际上许多处都不一样。

《定军山》有个小``三股档'',用在黄忠见韩浩、夏侯尚那一场。这场是韩、夏侯正式奉令出马,黄也是正式迎战交锋。黄忠抖擞精神把韩和夏侯杀得大败而逃。这场开打不能大又不能小,太大显不出韩、夏侯弱,太小显不出黄忠勇。用这个``三股档''黄把韩和夏侯拨拉过来,拨拉过去,两合就连削带抓
他们打下去,紧接着黄来一个大刀花下场,表示奋勇追击。

《阳平关》黄忠见张郃、杜袭一场,也是三个人,但情节与《定军山》黄见韩、夏侯不同。张、杜是曹营名将,非韩、夏侯可及,黄忠夜半劫营,他们仓猝迎战,溃散之下,投奔曹兵主力开始大战,这场开打如果用《定军山》的三股档就不太合适,可以打``硬三枪''头子或其它套子。

《伐东吴》是黄忠一肚子气,拼了老命要斩东吴八员上将,潘璋虽勇但招架不住,因此潘、黄开打不要多,但黄要耍三个下场以示其奋勇冲杀深入重围。

《定军山》三股档(谭派打法)

刘砚芳介绍:

韩浩、夏侯尚在大边台口,韩里,夏侯外,黄上场门上,一指,向大边台口漫夏侯头,夏侯过小边外边,黄用刀鐏勾韩腰到大边外边(即一肘),再从下场门向小边外边漫夏侯头,夏侯又过大边,黄到小边,回身与韩穿肚左转身回来打夏侯鼻子,右转身削韩头,打夏侯靠旗,亮,接大刀花下场。

《阳平关》见张郃、杜袭硬三枪头子(王凤卿打法):

黄众人搭轿上到台口站齐,黄上时左手抱刀右手扶刀,到台口刀交右手平出刀,举左手数更,(白)``放起火来'',往里一砍,张著上手下,下手张、杜两边抄过合,黄从中间到小边,张郃留在大边,余者分下,一扯,两扯,剜萝卜黄到大边,拉转身,一枪、两枪、三枪,打张鼻子,转到里面打腰封,接背躬。向外漫张头,向里两穿,向外两盖,打张鼻子,用鐏勾杜袭上,从中间向外起大刀花在台中间被压住,搅起来,用鐏勾杜腰过小边(即一肘),黄归大边,在二人中间一合到小边,拉肚转身,鼻子,削头,抓靠旗,看左拳,看马后,左手指,耍下场。\textbf{看拳和马后是看擒着人没有},\textbf{指是指张}、\textbf{杜跑了},\textbf{耍下场是猛追}。

\textbf{王荣山说这一场也可以少打些},打法是:

剜萝卜黄归大边后,张刺黄一压,打张腰封,勾杜上,一压,搅起来,用鐏勾杜腰过小边(即一肘),黄归大边,中间过合,穿肚往里转,鼻子,削头,抓靠旗,亮,耍下场下。

《阳平关》挡棒攒头子(王凤卿打法):

第一场曹操上,唱,上桌唱毕,黄上场门上出刀被徐晃漫头过小边,一滑,打上下左右,鐏一盖,打后蓬头。拉转身,搕,回花转身,大刀花转身搕,下叉,从里面漫头过去到大边,打鼻子,勾王平上,下接打棒攒,亮下场门下。

第二场曹又唱毕,黄下场门上,出刀,被徐晃漫头,搕反抄勾王平上,下接反攒,亮上场门下。

第三场,曹唱毕,黄内导(/倒)板,边唱上场门上,一亮,一趱子,台口刀头朝后洒,拨拉众上,到小边,一、二大扯,推到中间架住,唱。

\textbf{王荣山说前边不打硬三枪},\textbf{则接攒第一场可用硬三枪},打法是:

黄上场门上,晃漫头上黄归小边,一兜往里转身,一枪、两枪、三枪,绕刀鐏一盖前蓬头,一盖后蓬头,拉肚转身,漫头过去到大边,晃刺,黄压打腰封,左转身勾王平上,下略。

二场是黄下场门上,晃漫头,一勾打晃腰封,反勾王平上,下略。

三场,唱上一亮,一趱子,三甩胡,出刀拨拉众上,领起来由大边到小边,一指,左转身向里一合,向外两合,向里架住,唱。

《伐东吴》三个大刀花下场(王荣山演此戏用):

\textbf{第一个}:即常用的大刀花下场,出刀,三个正花转身过大边,串腕回花转身,正花转身,大刀花,劈马,正花转身,面外,亮住,串腕转身,出刀,转身背刀,弓箭步,向里亮下。

\textbf{第二个}:开始同第一个,到亮住,串腕转身,出刀,用右手背垫刀反手接刀,脸前绕大圈,平着往左齐腰横砍,撤右脚,向外正面正花转向大边,亮住,串腕右转身,向外正面出刀交左手反蹦子转身弓箭步左手握刀杆中间向外亮,下。

\textbf{第三个}:开始同第一个,到劈马,再耍三个正花转身到小边,大刀花劈马,正花转身到大边,面向小边左右两涮刀,右转身右手持刀撕开,刀上膀子左转身,刀交左手串腕,在大边弓箭步左手握刀杆中间向外亮,大绕下场门下。

\textbf{陈超老师说明:}

刘曾复先生传授的贾洪林配演刘备的词很讲究,而且很重要,兹举两例:

刘备念``昔年随朕开基创业之将,死的死了,亡的亡了'',而黄忠会错意,认为``开基创业之将''就是``五虎上将''。刘备没有针对,因此不念``昔年五虎上将,死的死了,亡的亡了''也不至于失言至此。

刘备见谭雄不唱``孤与孙权冤仇深\ldots{}\ldots{}''而是``四百年来争汉鼎,东吴不君也不臣。''

一句道出了刘备伐吴的真正原因,以报仇为借口,灭吴统一。

潘璋是东吴八员上将,黄忠再勇也不至于一个``扫头''就落荒而逃。因此有些演法是黄忠、潘璋对刀。

谭鑫培、余叔岩认为不能对刀的原因是:孙权将青龙刀赏赐潘璋,而潘璋并不会使青龙偃月刀。特别值得一提的是,钱金福为潘璋设计的开打,总是用刀鐏杵,不用刀头砍。加之黄忠勇武,因此一个\textless{}\textbf{扫头}\textgreater{},潘璋落花流水。

\newpage
\hypertarget{ux8fdeux8425ux5be8-ux4e4b-ux5218ux5907}{%
\subsection{\texorpdfstring{连营寨\protect\hyperlink{fn213}{\textsuperscript{213}}
之
刘备}{连营寨213 之 刘备}}\label{ux8fdeux8425ux5be8-ux4e4b-ux5218ux5907}}

{[}第一场{]}

(诸葛瑾
【西皮摇板】奉王命行路程不耽时候,此一番到蜀营讲和罢休。但愿得此一去旗偃歌奏\protect\hyperlink{fn214}{\textsuperscript{214}},免生灵遭涂炭民死蜉蝣。)

{[}第二场{]}

【西皮摇板】孙仲谋与孤王结成仇寇,只杀得他兵和将尸堆山丘。望空中二贤弟神灵保佑,灭却了东吴贼方肯罢休。

有请。

平身。

请坐。

到此乃依\protect\hyperlink{fn215}{\textsuperscript{215}}(或:乃是)客位,有话叙谈,哪有(或:焉有)不坐之理?(请坐。)

子瑜远来,有何事故?

哼,汝东吴现在危急,故命汝以巧言来说和。

住口!

汝东吴不仁,杀弟之仇,不共戴天。欲朕罢兵,哼哼,(或:汝东吴诡谋,损孤二弟,此仇不共戴天,欲孤罢兵,)除死方休!

不看我家丞相之面,先斩汝首(或:定斩汝首)。今且放汝回去,说与孙权,洗颈就戮。(或:今且放你回去,说与孙权,教他洗颈待戮。)

去罢!

(诸葛瑾 【西皮摇板】适才间在蜀营申述利害,见主公定良谋好把兵排。)

{[}第三场{]}

关兴、张苞,传令吩咐:满营大小将官,俱穿孝服。将你父等灵牌请在灵堂,一概仇人绑好。(或:关兴、张苞,打扫灵堂,安放灵位。满营将官,俱穿孝服。将一干人犯绑至灵堂。)为伯亲自祭奠\ldots{}\ldots{}(哭介)

摆驾!

【西皮摇板】想当年结桃园同天发咒,愿同年同月日(或:同日月)同刻罢休。到如今一旦间死别分手,孤岂肯独一人乐享无忧。

【西皮导板】白盔白甲白旗号,

\textless{}\textbf{哭头}\textgreater{}二弟呀,三弟呀!啊\ldots{}\ldots{}

【回龙】孤的好兄弟!

【西皮原板】满营将官哭嚎啕。孤王兴兵把仇报,扫灭了东吴恨方消。请过了神牌怀中抱,

【反西皮二六】点点珠泪往下抛。当年桃园结义好哇,胜似一母共同胞。不幸徐州失散了,万般无奈暂且降曹。那曹操待你的情义好,上马金银也曾赠过你锦袍(或:赐过你锦袍)。美女十名你不要,挂印封金辞奸曹。匹马单刀保皇嫂,过五关斩六将擂鼓三通把蔡阳的首级枭,可算得盖世的英豪。华容道上放曹操,大仁大义志量高(或:亘古流表\protect\hyperlink{fn216}{\textsuperscript{216}})。单刀赴会天下晓,英雄美名亘古标(或:志量高)。可恨(那)孙权行计巧,害孤二弟归天曹。愚兄兴兵把仇报,扫平了东吴气才消。还望二弟神灵保,

【西皮散板】神灵呐保,

\textless{}\textbf{哭头}\textgreater{}孤的好兄弟呀,(或:二弟呀,)

【西皮摇板】不灭孙权不回朝(或:不还朝)。

【西皮摇板】非是为伯伤心泪掉,孤与你父(或:我与你父)生死交。哭罢了二弟把三弟叫,

\textless{}\textbf{哭头}\textgreater{}翼德(弟)呀,桓侯哇,啊,孤的好兄弟呀,

【反西皮二六】叫声三弟听根苗:大破黄巾天下晓,敌人见你望风逃。虎牢关曾把吕布的发冠挑,长坂坡前喝断当阳桥(或:喝断灞桥)。夜战马超胆气好,义释严颜颇有略韬。可恨那范疆、张达两个贼强盗,谋害英雄二贼脱逃。愚兄兴兵与你把仇报,只杀得孙权魄散魂消。情愿罢兵写降表,同心合意共灭奸曹。锦绣山河孤不要,一心与你把仇消(或:一心只想把仇消)。哭哑了咽喉把三弟叫,把三弟\textless{}\textbf{哭头}\textgreater{}叫,豹头环眼的三弟呀,

【西皮摇板】拿住孙权两开销。

看酒来,待孤亲自祭奠。

(两弟受孤一拜。)

儿要多拜几拜。(或:儿等多拜几拜。呃\ldots{}\ldots{}(哭介))

一概仇人(或:将一干仇人),拿去开刀。

啊?

此贼为何不斩?

哦------剑来!

【西皮散板】到此时(或:到如今)还讲什么郎舅之分,献荆州贼有何亲戚之情。三尺剑正国法又报仇恨,死眼前看贼子你埋怨何人。

关兴、张苞,吩咐文武官员(或:传孤旨意,满营将官),歇兵三日,兵发东吴。

正是:(念)满腔怒气冲昊天,誓把(或:要把)东吴踏平川。

{[}第四场{]}

(念)起居梦寐恨吴寇,不报冤仇誓不休。

孤自兴兵以来,(势如破竹,)吾国(或:我国)人马屡屡得胜,他邦兵将(或:他国兵将)节节败溃。那些吴狗们望风而逃,此乃诸将之功也。

(众 主公妙计,臣等何功之有?)

诸将俱各有功。

(马良 启奏主公,闻得孙权又拜陆逊为都督,兵扎猇亭。)

坐下。

(马良 谢座。)

陆逊何路人也?

(马良
乃九江太守陆骏\protect\hyperlink{fn217}{\textsuperscript{217}}之后。)

哼,懦弱书生,统领人马(或:担此重任),岂不贻笑大方?

(马良
那陆逊虽是一介书生年幼,前番吕蒙白衣渡江,暗取荆州,乃此人之计也。)

哦!竖子诡谋(或:孺子诡谋),损孤二弟,今当擒之!

关兴、张苞,传令进兵。

(马良 且慢!主公请息龙怒。)

为何拦阻?

(马良 陆逊智胜周郎,不可轻敌。)

唉!孤用兵老矣!岂反不如一黄口孺子么?

(马良 如今陆逊不战不退,莫非有何诡计?)

哼!黄口孺子,有多大能为?既敢当此重任,就该领兵前来,与孤对敌。战又不战,退又不退,其情可恼!

(马良 倘若陆逊以逸待劳,如之奈何?)

孤兵精粮足,与他对守何惧(或:对垒何惧)?

(马良
堪堪天气炎热,暑气难当,兵扎离火之中,汲水不便。又恐将士多生疾病。)

不妨,孤将营寨,移于茂林深处(或:孤将人马,移至茂林深处),待(等)过夏到秋,并力进兵,东吴自然休矣(或:吴国可图也)。

(马良 我若兵动,倘陆逊踏营,如何是好?)

孤命关兴、张苞各带人马(或:带领精兵),埋伏山谷之中。倘陆逊来击,引兵突出,孺子可擒也。(或:倘陆逊劫营,我军伏兵杀出,一鼓擒之。)

(众 主公妙计,臣等不及也。)

(马良 主公要移营寨,可画成地图,问过丞相?)

孤亦颇知兵法,此事何必又问(或:再问)丞相?

(马良 古语云:兼听则明,偏听则暗。望陛下详之。)

也罢。卿可自去各营,画成四至八道\protect\hyperlink{fn218}{\textsuperscript{218}}图本,亲到东川,去问丞相。若有不便,即来回奏。孤再作裁处。(或:如此就命你将山势、营盘画成图本,去至东川,送与丞相观看。倘有不到,急速回来,孤再作裁处。)

(马良 领旨。)

关兴、张苞,就此移营者(或:择吉移营者)。

正是:(念)龙麟启祚\protect\hyperlink{fn219}{\textsuperscript{219}}如反掌,干戈霸业定太平。

{[}第五场{]}

(念)月当空乌鸦嘶叫,帅字旗无风自摇(或:无风自飘)。

(报子 报!东吴人与马缓缓移动。)

再探。

(此乃疑兵。)

啊,沙摩柯听令。

(沙摩柯 在。)

带领蛮兵女将前去探看虚实,相机击之。

(沙摩柯 得令。)

此乃疑兵,何足道哉?

(报子 报!江北营中火起。)

再探。

关兴前去救火。(或:关兴营救。)

(关兴 得令。)

江北营中火起,(此乃)我军自不小心。

(报子 报!两岸火起。)

再探。

张苞去救。(或:张苞急救。)

两岸火起,我军大不利也。

(报子 报!满营火起。)

(再探。)

不、不、不\ldots{}\ldots{}不好了!

(念)\textless{}\textbf{蛮牌令}\textgreater{}看、看、看,看呐,风助火威狂,火乘猛风飏。满天飞烈焰,遍地闪金光(或:撒金光)。祸从天降,祸从天降呃。寻不出路当央\protect\hyperlink{fn220}{\textsuperscript{220}},寻不出路当央。快带丝缰,快带丝缰。

{[}第六场{]}

(赵云 赵云接驾。)

哎呀四弟呀!你看孤被他们烧得乌焦巴弓了。(或:四弟你来了,杀出重围。)

(赵云 主公保重。)

(四弟,)孤命休矣!

杀呀。

(赵云 杀呀。)

(赵云 赵云救驾来迟,死罪呀死罪呀!)

唉,孤虽得脱,诸将奈何!

(赵云 由臣断后。)

前面什么所在?

(赵云 乃是白帝城。)

兵撤白帝城。

带马。

\textless{}\textbf{三叫头}\textgreater{}二弟,三弟,唉!兄弟啊(或:贤弟呀)\ldots{}\ldots{}(哭介)

(罢!)

(赵云 马僮,抬枪带马。)

\textbf{(安居)平五路}\protect\hyperlink{fn221}{\textsuperscript{221}}

{[}第一场{]}

(打朝,末扮贾诩、外扮辛毗、副扮曹真、净扮司马懿,上)

贾诩 (念)自古良禽择木栖,

辛毗 (念)而今喜得拜丹墀。

曹真 (念)男儿须当封侯印,

司马懿 (念)正是英雄得志时。

贾诩 请了。

辛毗、曹真、司马懿 请了。

贾诩 今日万岁升殿,必有军情议论。

辛毗、曹真、司马懿 大家分班伺候。

贾诩、辛毗、曹真、司马懿 请。

(四太监、一大太监,\textless{}\textbf{小开门}\textgreater{},小生扮曹丕\protect\hyperlink{fn222}{\textsuperscript{222}}上)

曹丕 {[}引子{]}驾坐朝阁受三分,重整山河。

曹丕
(念)献帝无福民不安,人心归朕乐尧天。上苍若肯遂孤愿,扫平东吴灭西川。

曹丕
寡人曹丕,国号黄初在位。蒙众卿忠勇,扶孤禅位,更改国号。深感上天之福佑也。朕闻刘备兵伐东吴,中了陆逊火攻之计,败入白帝城,气忿身亡。朕闻此信心无忧矣。孤有心攻取西川,我想必获全胜。众贤卿。

贾诩、辛毗、曹真、司马懿 万岁。

曹丕
朕想刘备新亡,乘他国内无主,人心未定,攻取西川,蜀可得矣。卿等意下如何?

贾诩 臣贾诩奏闻陛下。

曹丕 当面奏来。

贾诩
臣想刘备虽亡,必托孤与诸葛亮,那孔明感刘备知遇之恩,必要倾心竭力扶持嗣主,陛下不可轻伐。

司马懿 臣司马懿有本启奏。

曹丕 卿有何良谋,奏与朕知。

司马懿 西蜀新败,休容他养成锐气,若不乘此发兵,待等何时?

曹丕 卿言正合孤意,当用何计?

司马懿
若用中原之兵,恐难取胜。须用五路大兵,四面攻打。那诸葛亮首尾不能相顾,西川之地,必然唾手可得。

曹丕 哪五路呢?

司马懿
可修国书遣使臣去到鲜卑国见那国王轲比能,贿以金帛,令他起羌兵十万\protect\hyperlink{fn223}{\textsuperscript{223}}攻打西平关,此一路也;

曹丕 二路呢?

司马懿
差人直入蛮洞买通蛮王孟获,领起蛮兵十万攻打益州、永昌四郡,此二路也;

曹丕 三路呢?

司马懿
再遣能言使臣入吴和好\protect\hyperlink{fn224}{\textsuperscript{224}},许以割地为约,领起吴兵十万入峡口取涪城,此三路也;

曹丕 四路呢?

司马懿 急调孟达起上庸兵十万攻打汉中,此四路也;

曹丕 那五路呢?

司马懿
就命大将军曹真起中原大兵十万攻打阳平关,此五路也。五路大兵共五十万,併力攻取西川,那孔明纵有吕望之才,难逃五路雄兵也。

曹丕
此本奏之有理,孤王依计而行。曹真进位\protect\hyperlink{fn225}{\textsuperscript{225}}。

曹真 万岁。

曹丕 卿领大兵十万攻取阳平关,得胜回朝,另加升赏。

曹真 领旨。

曹真 (念)金殿领君命,校场排雄兵。

(曹真下)

曹丕 退班\protect\hyperlink{fn226}{\textsuperscript{226}}。

(众分下)

{[}第二场{]}

(二小军抬杠箱上,丑扮差官跳上)

差官
吾乃北魏国王驾下差官是也,今奉我主之命押解礼物去到鲜卑国,聘请国王轲比能,今起羌兵十万攻打西平关。身奉君命,不敢怠慢。军士们。

二丑小卒 有话说罢。

差官 快趱行。

(差官倒退跳走\textbf{干\textless{}度柳翠\textgreater{}}\protect\hyperlink{fn227}{\textsuperscript{227}},\textless{}\textbf{挑子}\textgreater{}下)

{[}第三场{]}

(报子上)

报子 马来。

报子 (念)胆量天生就,应变广机谋。探访邻邦事,名称夜不收。

报子
吾乃西蜀远探是也,探得曹丕发兵五十万,五路进兵攻取西川。探得真实,不免连夜飞报丞相知道便了。

(报子下)

{[}第四场{]}

(生扮诸葛亮上)

诸葛亮
【西皮\textbf{原}板】天命归人心归天时地利,一朝君一朝臣争夺华夷。西川地到而今虽归我主,普天下皆王土汉室地基。

诸葛亮 (念)铺谋设计正朝纲,国事纷纷费心肠。

诸葛亮
山人诸葛亮,字孔明,道号卧龙。只因\protect\hyperlink{fn228}{\textsuperscript{228}}先皇伐吴失利败入白帝城,气忿成疾,晏了圣驾。蒙托孤之重,扶保幼主登了龙位,安稳民心。可恨曹丕篡了汉位,更改国号。本当发兵问罪,怎奈我兵新败,不敢轻举妄动,待等兵精粮足再去发兵问罪。正是:(念)只为托孤恩义重,披肝沥胆报国恩。

(报子上)

报子 (念)探听北魏军情事,报与西蜀丞相知。

报子 来此已是府门,里面哪位在?

(副扮听事官上)

听事官 (念)侯门深似海,不许外人来。

听事官 什么人?

探子\protect\hyperlink{fn229}{\textsuperscript{229}} 探子要见丞相。

听事官 候着。探子求见丞相。

诸葛亮 传。

听事官 探子,丞相传,小心了。

探子 啊。探子叩头。

诸葛亮 探子,你探听哪路军情,一一讲来。

探子
相爷容禀:\textless{}\textbf{五字赞}\textgreater{}探子禀军情,相爷在上听:曹丕人马踴,五路起雄兵:中原兵十万,曹真攻阳平;上庸发人马,孟达取汉中;孙权入峡口,大兵攻涪城;西平羌兵众,国王轲比能;南蛮名孟获,四郡恶交锋。五路貔貅猛,十万虎狼兵。声如地裂山摇动,要把西川一扫平。

诸葛亮
赏你银牌一面,休叫成都\protect\hyperlink{fn230}{\textsuperscript{230}}军民知觉,不可走漏我的消息。

探子 谢相爷。

(探子下)

诸葛亮
这厮好生\protect\hyperlink{fn231}{\textsuperscript{231}}可恶,,明知我国老王驾崩,新君年幼,乘我国丧,兵发五路来克我国,如此猖狂,我自有道理。听事官。

听事官 在。

诸葛亮 传四路旗牌进府听令。

听事官 是。相爷有令:传四路旗牌进府听令。

(外扮北路旗牌,副扮西路旗牌,末扮南路旗牌,净扮东路旗牌;四旗牌内应,上)

四旗牌 (念)丞相来呼唤,忙步到府堂。

四旗牌 四路旗牌参见丞相。

诸葛亮 你等免参,听我分派。

四旗牌 愿听丞相军令。

诸葛亮
今有曹丕兵发五路攻取西川,要你等飞递边关,不叫成都军民知晓,莫要泄漏我的机关。

四旗牌 谨遵丞相军令。

诸葛亮 北路旗牌听令。

北路旗牌 在。

诸葛亮
命你赶到阳平关通知赵云,叫他暗设人马,不可交锋,待贼粮尽自退,督兵追杀,不得违误。

北路旗牌 得令。

(北路旗牌下)

诸葛亮 西路旗牌听令。

西路旗牌 在。

诸葛亮
命你赶到西平关急报马超,叫他虚立自己旗号,羌兵必不敢战,容他自退,不得违令。

西路旗牌 得令。

(西路旗牌下)

诸葛亮 南路旗牌听令。

南路旗牌 在。

诸葛亮
我有令箭一支,内有柬贴封好,赶到南郡交付魏延,叫他依计而行,不可错误。

南路旗牌 得令。

(南路旗牌下)

诸葛亮 东路旗牌听令。

东路旗牌 在。

诸葛亮 命你前去急调关兴、张苞各带汉中人马三万,四面接应,不得违误。

西路旗牌 得令。

(西路旗牌下)

诸葛亮
那上庸兵乃孟达督帅,不用劳动军卒,管叫他不战自退。我料东吴孙权必要兵扎三江口,虚作人情,静观两家胜败,就中攻取,事虽如此。怎奈我先皇昭烈兵伐东吴,结下仇怨,并未和解。我若兴兵伐魏,吴必攻取西蜀;昼夜思想,不得其人入吴和好\protect\hyperlink{fn232}{\textsuperscript{232}}。若得吴、蜀和好,结为唇齿,然后兴兵伐魏,也免我忧虑东吴之患也。

诸葛亮
【西皮\textbf{原}板】平生恨篡国贼欺君万恶,心想要灭贼子枉自揣摩。我本该去问罪天不容我,一桩桩一件件国事阻隔。到如今曹丕贼心威赫赫,乘国丧兵五路侵佔我国。西蜀中现有我区区诸葛,岂肯容贼猖獗奏唱凯歌。参想想和东吴长久计策,缺少个能言士前去说说。叹先皇心愿事敕命于我,灭国贼尽人力天意如何。居相位守臣节日日思索,我怎能负先皇临危重托。

(诸葛亮下)

{[}第五场\protect\hyperlink{fn233}{\textsuperscript{233}}{]}

(四旗牌同上)

北路旗牌 列位请了。

三路旗牌 请了。

北路旗牌
今有曹丕兵发五路,攻取西川,你我奉了丞相军令,通知各路依令而行。军情紧急,分路投递。正是:(念)将军不下马,

三路旗牌 (念)各自奔前程。

{[}第六场{]}

(四文堂、一中军站门,孟达上)

孟达 {[}引子{]}只为一着错,满盘棋势空。

孟达
(念)昔侍蜀君今侍魏(或:昔仕蜀君今仕魏),俱是三呼称万岁。叹想原郡故乡土,谁到坟前化纸灰。

孟达
俺,孟达,昔在汉中称臣,为事不平弃蜀投魏,命俺镇守上庸等处。日前圣旨到来,命俺起兵十万攻取汉中。我想永安宫乃李严镇守,我若攻打,有碍生死之交;如不攻打,又恐魏王识破疑我,好不两难也。

(副扮下书人上)

下书人 (念)四季关银饷,一年走慌忙。

下书人 来此已是,营门有人么?

中军 什么人?

下书人 永安宫李严差人下书。

中军 候着。启禀帅爷:永安宫李严差人下书。

孟达 传他进帐。

中军 是。下书人里面传你,小心了。

下书人 是。下书人叩头。

孟达 你奉何人所差?

下书人 奉永安宫李老爷所差,有书呈上。

孟达 后营用饭。

下书人 领爷赏赐。

(下书人下)

孟达 待我看来。

(孟达看信,起\textless{}\textbf{牌子}\textgreater{})

孟达 来,传下书人。

中军 下书人。

(下书人上)

下书人 (念)后营用罢饭,帐下听回音。

下书人 谢爷的酒饭。

孟达 回覆你爷:我这里修书不及,照书行事。

下书人 是,小人记下了。

(下书人下)

孟达
我正忧疑之间,李严有书到来,我岂忘了生死之交。不免假装重病,中军传令:帅爷偶得重病,暂将人马撤回,再听调用。

中军 得令。下面听者:

(内应介)

中军 元帅偶得重病,暂将人马撤回,再听调用。

众 (内)传令。

孟达 (念)谁人不思故乡土,洛阳虽好不如家。

孟达 唉哟哟,好不痛死人也。

(众搀扶孟达领下)

{[}第七场{]}

(四文堂、四水军众站门上,小生扮陆逊\textless{}\textbf{牌子}\textgreater{}上)

陆逊
吾乃东吴水军都督陆逊,今有北魏曹丕五路攻川,许以割地为约,令起大兵十万出峡口,攻打涪城。吾想吴、魏两国皆非诸葛之敌手,万难取胜。是我奏明主公,用两全之计,虚作人情,兵扎三江口,坐观胜败,就中取事。众将官------

(众应介)

陆逊 兵发三江口去者。

(\textless{}\textbf{牌子}\textgreater{}众原场)

众 前面已到三江口。

陆逊 安营下寨。

(众应介,同下)

{[}第八场{]}

(四朝官上,末扮许靖,白髯;外扮董允,黪髯;副扮杜琼,黑三;生扮邓芝,黑三)

许靖 (念)金钟响罢禁门开,

董允 (念)雨露恩深拜龙台。

杜琼 (念)常思汉鼎三分在,

邓芝 (念)灭魏伐吴待时来。

许靖 下官,司徒许靖。

董允 下官,黄门侍郎董允。

杜琼 下官,谏议大夫杜琼。

邓芝 下官,户部尚书邓芝。

许靖 请了。

董允、杜琼、邓芝 请了。

许靖 今日早朝圣驾登殿,必有国政议论。

董允、杜琼、邓芝 金钟三响,想是圣驾临朝。

许靖、董允、杜琼、邓芝 请。

(许靖、董允、杜琼、邓芝分班站介,四太监、一大太监站门,小生扮刘禅上)

刘禅 {[}引子{]}诏书赐孤王,驾坐成都称帝邦。

刘禅
(念)父皇白帝驾殡天,众卿扶保坐江山。但得吴、魏干戈定,永守西蜀心也安。

刘禅
孤刘禅,国号建兴,只因父皇兵伐东吴失利,兵退白帝城;圣驾殡天,托孤与诸葛丞相。扶孤登基,内理国政,外治民情,皆赖丞相之奇才也。今日早朝,内侍,展放龙帘。

大太监 领旨。

(黄门官上)

黄门官 (念)忙将动地惊天事,奏与君王御驾知。

黄门官 臣黄门官见驾,吾皇万岁。

刘禅 卿有何本奏?

黄门官 今有北魏曹丕兵发五路啊!( \textless{}\textbf{牌子}\textgreater{})

刘禅 既有此事,就命卿相府召丞相入朝理事。

黄门官 领旨。

黄门官 (念)五路雄兵起,三国战不息。

(黄门官下)

刘禅 适才黄门奏道:曹丕五路进兵,攻取我国。众卿,

众 万岁。

刘禅 有何良谋可退贼兵?

众
万岁圣意宽怀,暂请放心。\protect\hyperlink{fn234}{\textsuperscript{234}}待丞相入朝,必有良谋妙策。

刘禅 孤亦想到如此。

黄门官 (内)走啊!

(黄门官上)

黄门官 启奏万岁:丞相有病在府,不容进见,特来交旨。

刘禅 卿家暂退。

黄门官 领旨。

(黄门官下)

董允、杜琼 臣董允、杜琼同到相府求计,看有何说。

刘禅 二卿愿去,速来回奏。

董允、杜琼 领旨。

(董允、杜琼下)

众
诸事(已)毕\protect\hyperlink{fn235}{\textsuperscript{235}},请驾回宫。

刘禅 退班。

(众分班下)

{[}第九场{]}\protect\hyperlink{fn236}{\textsuperscript{236}}

(丑扮门官上,\textless{}\textbf{普贤歌}\textgreater{}干牌子)

门官
(念)剑戟峥嵘将相门,谁敢杂踏与高声。常闻细柳营,天子按辔行,何况文官与武臣。

门官
咱家诸葛丞相府下门吏便是。可怪我家相爷向来真正霄晓勿遑\protect\hyperlink{fn237}{\textsuperscript{237}},日理万机。近日何故,终朝不出内阁,一切政务不理。慢说羽书雪片,多官请事。就是方才圣明来召,也是推病不起,你想这样身价,可是亘古罕有的。闲言少说,只恐又有人来请见,我且坐守等候。

(董允、杜琼、二青衣扮随侍上)

董允 (念)宰正百官才独称,

杜琼 (念)仪型四海圣王尊。

董允 (念)仪门台下下了马,

杜琼 (念)好向阁内问安宁。

董允、杜琼 回避。

(二随侍下)

董允 看仪门肃静,人寂无声,同到门上。

杜琼 请。

董允 (念)月照牙旂肃,

杜琼 (念)风吹画角寒。

董允、杜琼 门官。

门官 原来二位大人,请坐。

董允、杜琼 请。

董允 我来问你,丞相往常勤于政事,近日不见升堂,是何缘故?

门官
小官不知(何)故\protect\hyperlink{fn238}{\textsuperscript{238}}。(念)宅门高挂止步牌,一切杂物何禀来。

门官 二位大人不曾见么:(念)门外朝事与边事,谁敢进内去相催。

董允 这却(为)何也?\protect\hyperlink{fn239}{\textsuperscript{239}}

杜琼 丞相连日起居如何,饮食可曾加减?

门官 这小官也曾打听:(念)起居倒觉不甚衰,肥肉三餐不吃斋。

董允、杜琼 既然身健食壮,缘何不出堂理事?

门官
据小官想来,相爷多管害心病\protect\hyperlink{fn240}{\textsuperscript{240}}。

董允、杜琼 甚么心病?

门官
二位大人,从来出将入相之家,不言歌童舞女成群,即便那娇妻美妾也却无数。可怜我们相爷就是一位黄夫人,心性却有姜嫄之德,其颜却如嫫母之陋。\protect\hyperlink{fn241}{\textsuperscript{241}}恁教相爷耐得住呢。

董允、杜琼 胡说。

门官 世情如此,不是小官妄说。

董允、杜琼 你可进去禀知说董允、杜琼特来问候金安,还有大事面启。

门官
哎呀呀\ldots{}\ldots{}相爷数日传谕:所有一应大小官员,勿得擅入禀事。方才圣命来召,尚尔\protect\hyperlink{fn242}{\textsuperscript{242}}辞去,何况大人。

董允、杜琼 我等今奉圣命而来,定要请见。

门官 既然如此,小官传禀便了。

(门官下)

董允
(念)似此葫芦闷\protect\hyperlink{fn243}{\textsuperscript{243}}难审,

杜琼 (念)只须等待听好音。

(门官上)

门官
回禀二位大人,丞相说:知道了,请二位大人不必进见,有甚军国大事,等待病体稍可,改日自出都堂会议。请回罢。

(门官下)

董允 哎呀,军情至急,哪还等得改日会议。

杜琼 量来难以进见,且回奏圣上再处。

董允、杜琼 请。

董允 (念)召命尚安难通问,

董允 带马。

(二随侍左右上,应)

杜琼 (念)何况区区僚佐臣。

(董允、杜琼急下)

{[}第十场{]}

(二宫女、一大太监引正旦扮吴后上)

(吴后
【二黄慢板】老王爷祖居在大树楼桑,在桃园三结义万世名扬。伐东吴中奸计全军俱丧,梦魂里白帝城痛断肝肠。)

吴后 {[}引子{]}珠帘高卷似蓬莱,追思先帝心痛哀。

吴后 (念)
老王祖居在楼桑,桃园结义万古扬。兵伐东吴全军丧,梦魂白帝断肝肠。

吴后
哀家吴后,先皇昭烈帝与二君侯报仇心切,兵伐东吴,连营失利,败入白帝,恸想二弟,思念桃园,气忿成疾,晏了圣驾,托孤与诸葛丞相。扶保皇儿,登了龙位,内修国政,外治民情,依赖丞相之贤也。正是:(念)谋猷\protect\hyperlink{fn244}{\textsuperscript{244}}人钦敬,调和鼎鼐臣。

刘禅 (内)摆驾。

(四太监一字引刘禅上)

刘禅
【西皮摇板/散板\protect\hyperlink{fn245}{\textsuperscript{245}}】内臣宰无计策孤心烦躁,老丞相不出府所为哪条。

内监 万岁朝罢回宫,与国太请安。

吴后 请。

内监 请驾进宫。

刘禅 参见母后千岁。

吴后
哀家\textbf{平善如常}\protect\hyperlink{fn246}{\textsuperscript{246}},坐了讲话。

刘禅 谢母后。唉!

吴后 王驾叹息为了何事?

刘禅 启母后:大事不好了!

吴后 有何大事如此惊慌?

刘禅 北魏曹丕今发大兵五十万,攻打西川,怎不惊怕?

吴后 众文武岂无退兵之策?

刘禅
文武虽多,惶惶无策\protect\hyperlink{fn247}{\textsuperscript{247}}。

吴后 诸葛丞相必有退兵之策,召来一问。

刘禅
母后有所不知,儿也曾诏他上殿,怎奈\protect\hyperlink{fn248}{\textsuperscript{248}}推病在府,不容使臣入见;又命董允、杜琼同到相府问计,未见回奏如何。

(董允、杜琼上)

董允 【西皮摇板/散板】忙步踉跄走御道,

杜琼 【西皮摇板/散板】气喘嘘嘘滚油浇。

董允 【西皮摇板/散板】丞相忠心改变了,

杜琼 【西皮摇板/散板】他把托孤火化消。

董允 【西皮摇板/散板】你我何言回奏好,

杜琼 【西皮摇板/散板】须将实言奏当朝。

董允、杜琼 原来老承奉在此,烦劳转奏:董允、杜琼回奏交旨。

内监 二位老大人回来了。

董允、杜琼 回来了。

内监 万岁在延寿宫与国太等候回奏,待咱家与你二人请驾。

董允、杜琼 有劳老承奉。

内监 是咧,交给咱家。启奏万岁:董允、杜琼宫外候旨。

刘禅 儿启母后:董允、杜琼回奏,待儿问明,回奏母后。

吴后
董允、杜琼乃旧日老臣,国事紧急,暂止肃避之条,宣进延寿宫,哀家面前回奏。

刘禅 母后之言极是。内侍,宣董允、杜琼进宫,在国太驾前回奏。

内监 董允、杜琼进宫,在国太驾前回奏。

董允、杜琼 领旨。臣董允、杜琼愿国太千岁。

吴后 二卿平身。

董允、杜琼 千千岁。

吴后 二卿同到相府求计,丞相有何良策回奏?

董允、杜琼 臣启国太:丞相推病在府,不容臣等轻入,特来回奏。

刘禅
丞相推病为辞\protect\hyperlink{fn249}{\textsuperscript{249}},并不入朝理事,又无良谋回奏,不如小王趁早死了罢。

吴后
王驾休得如此,我想老王曾将大事托孤与丞相,皇儿拜他以为相父\protect\hyperlink{fn250}{\textsuperscript{250}}。人臣之中位至极矣。今曹丕明知老王殡天,皇儿年幼,乘我国新丧,人心未定,五路进兵夺取西川,社稷危急之际,假以推病为辞,一谋不设。哀家亲到相府求计,看有何说。

刘禅 二卿意下如何?

董允、杜琼
万岁,依臣等愚昧之见,国太不可轻往。料丞相不出府门,必有奇谋妙策。暂请主公御驾亲往求计\protect\hyperlink{fn251}{\textsuperscript{251}},如有怠慢,再请国太召丞相入太庙对老王御影问之可也。

吴后 二卿所奏有理,皇儿速往,哀家立听回奏。

吴后
【西皮摇板】西川地到如今我蜀帝基,恨曹贼兴人马五路告急。却为何老相爷坐视不理,这内中必有那妙算神机。

(吴后\textless{}\textbf{小锣打下}\textgreater{})\protect\hyperlink{fn252}{\textsuperscript{252}}

刘禅 摆驾。

(四大铠两边上,四太监引刘禅上辇\textless{}\textbf{牌子}\textgreater{},当场见门官)

众 圣驾到。

门官 小臣接驾。

(刘禅下辇介,\textless{}\textbf{牌子}\textgreater{}停)

门官 小臣叩见万岁。

刘禅 起来。

门官 万万岁。

刘禅 前去通禀。

门官 丞相令出森严,不准小官通禀。

刘禅 丞相今在何处?

门官 臣亦不知。只有丞相钧谕:挡住百官,勿得擅入。

刘禅 起来。

门官 谢万岁!

刘禅 相父既有此谕,众臣府外等候,毋得喧哗。

众 领旨。

(众分下)

刘禅 门官引起。

门官 领旨。

刘禅
【二黄摇板】龙离潭凤离巢论礼不雅\protect\hyperlink{fn253}{\textsuperscript{253}},为的是平五路君到臣家。

{[}第十一场{]}

(诸葛亮持竹杖上)

诸葛亮
【二黄三眼】报国家报不过黎元为大,扭人心扭不过事理无差。恨曹丕受禅台惨行强霸,叛逆贼终有日报应相加。我本当去问罪发动人马,怎奈我兵新败难以去杀。哭献帝恸先皇淋漓泪洒,好教人肝胆碎心乱如麻。

(诸葛亮大边外场观鱼,门官引刘禅上)

刘禅
【二黄原板】入相府(或:进相府)穿廊厦肃静幽雅,过几层曲湾处倒也可夸。进花园见相父闲坐潇洒,

(刘禅指门官退下,门官作揖退下)

刘禅 【二黄原板】孤这里走近前侧耳听他。

诸葛亮 【二黄原板】汉高皇创基业治平天下,至孝平方五载丧了邦家。

(诸葛亮站)

诸葛亮
【二黄原板】光武兴白水村【转二黄三眼】重整人马,访邓禹、收岑彭到处征伐。剐王莽、诛苏献神惊鬼怕,洛阳城修宫殿一统中华。四百载东西汉六元七甲,传至在献帝朝国乱如麻(或:群寇如麻)。十常侍乱宫闱董卓强霸,许田射猎曹孟德把主欺压。曹丕贼篡汉位万民叫骂(或:万民怒发),吾主爷恨贼子咬碎齿牙。白帝城受血诏遗言留下,承受那托孤重(或:托孤情)怎敢有差。哭一声先帝爷在九泉之下,保佑臣增寿算扶保汉家。

(诸葛亮看鱼指介)

诸葛亮
【二黄摇板】这鱼儿\protect\hyperlink{fn254}{\textsuperscript{254}}比陆逊行兵狡诈,有此计无此人怎能退他。猛回头站身边当今圣驾,

(诸葛亮跪介,刘禅搀介)

诸葛亮 【二黄摇板】老孤臣轻慢君罪当重加。

刘禅 相父。

刘禅
【二黄摇板】相父病孤王我放心不下,因此上孤亲自来看卿家。见相父观鱼跃闲情潇洒,这几天小王我心乱如麻(或:这几天相父病我心乱如麻)。\textless{}\textbf{行弦}\textgreater{}

诸葛亮 陛下何事忧心?

刘禅
【二黄摇板】曹丕无故兴人马,五路大兵来战杀(或:来征杀)。文武百官心惊怕,相父有病又在家。西蜀倾危在眼下,求取良谋\protect\hyperlink{fn255}{\textsuperscript{255}}去退他。

诸葛亮 哦!

诸葛亮 【西皮导板】尊我主请正坐容臣参驾,

(刘禅正坐,诸葛亮拜,刘禅扶)

刘禅 相父免礼,请坐。

诸葛亮 谢座。

诸葛亮
【西皮原板】容忍老臣奏根芽:曹丕国贼多奸诈,贿买羌、蛮辅助他。乱臣贼子人叫骂,谁肯真心死战杀。先皇在日(或:先皇在时)常怒发,本当问罪去征伐。乘人之丧毒手下,就是百万何惧他。臣非妄奏言虚假,望我主稳听捷报奏国家。\textless{}\textbf{小拉子}\textgreater{}

刘禅 听相父之言,曹兵五路,如此容易退却?

诸葛亮 陛下只请放心(或:陛下只管放心,呃),且免忧虑。

刘禅 (呃,)望相父明言与孤,所调都是哪路人马?

诸葛亮 陛下。

诸葛亮
【西皮原板】臣不奏为的是行兵密法,怕的是成都民惊走天涯。非是臣瞒陛下事有虚假,都只为安人心保国保家。马孟起守西平威名颇大,魏文长疑兵计俱按兵法。赵子龙阳平关督理人马,一封书差人去赚走孟达。退吴兵臣已把良谋想下(或:东吴兵臣已罢良谋想下),缺少个能言士每日详查。

刘禅 呀。

刘禅
【西皮原板】孤王亲入相府地,君臣二人论军机。欺君篡位贼曹丕,兵发五路\protect\hyperlink{fn256}{\textsuperscript{256}}取川西。派将三员贼退去,孤王心内自猜疑。彼众我寡非容易,片纸岂退(或:片纸怎退)上庸敌?丞相在府观鱼戏,东吴怎肯卷旌旗。越思越想心忧虑,(收腿)孤必得拔树搜根仔细提。

诸葛亮 (啊,)万岁思索何事?

刘禅
孤王所虑,彼众我寡,孤闻贼兵五十万,五路攻川,相父所派蜀将三员,能否退敌。孟达、孙权何人抵挡?

诸葛亮
陛下!老王将大事托与老臣,臣怎敢不竭力?报答老王知遇之恩。况成都臣宰,不知兵法之妙论。若用成都人马,人民震动\protect\hyperlink{fn257}{\textsuperscript{257}},不能安稳,机关泄漏,大事去矣!臣身居相府之中,心在边关之外\protect\hyperlink{fn258}{\textsuperscript{258}}(呀)。知己知彼,略韬\protect\hyperlink{fn259}{\textsuperscript{259}}因人而使,量才择用。那马超祖居西土,声名远振。羌人称他为``神威将军''。羌人见是马超,必自退去。此西路之兵不必忧矣。

诸葛亮
【西皮二六】那羌王柯比能兴兵犯境,那马超继祖先世居西平。他父祖自从来声名远振,那羌人称马超神威将军。臣命他用奇兵四下伏定,那羌王见而丧胆决不敢前来相争。

刘禅 哦(哦),这一路退得妙!那蛮王孟获,闻他骁勇无比,谁可敌他?

诸葛亮
那南蛮孟获兵犯益州四郡,臣使魏延用疑兵之计。孟获虽勇,多生疑心,必要自退,我兵随后追杀,必然大获全胜。此南路之兵------陛下------心勿忧矣!

诸葛亮
【西皮快板】南蛮贼他夙习智量浅近,哪知晓孙、吴法机谋实深。忽见我左右军无数隐隐,管教他神魂不定不敢交兵。

刘禅
【西皮摇板】我相父素昔来谋猷谨慎,知己知彼着着胜人。还有那第三路汉中要紧,第四路恐难退贼将曹真。

诸葛亮
汉中是李严把守,孟达与李严有生死之交。臣已暗作一书,如严亲笔,着人潜递\protect\hyperlink{fn260}{\textsuperscript{260}}孟达,孟达见之,必然推病不出。况孟达并非李严对手,臣回成都,留严镇守永安宫,正为此也。东路之兵,万岁,不足忧也。

刘禅 哦哦哦\ldots{}\ldots{}

诸葛亮
那曹真领中原大兵十万,攻取阳平。那阳平本非用武之地,山岭险峻\protect\hyperlink{fn261}{\textsuperscript{261}},道路崎岖。行运粮草不便,臣命赵云暗设人马,坚守勿战。待彼粮尽,一战成功。曹真必败于赵云之手,此北路人马不足忧矣。呃,臣尚恐不能全保,又秘调关兴、张苞呵------

诸葛亮
【西皮摇板】恐四路有哪处力不胜任,故命他分要口结寨安营。有不虞急提兵前往救应,\textbf{此密事故未便先使人闻}。

刘禅 如此说来,五路贼兵,(呃,)已退四路了。

诸葛亮 正是。

刘禅 孤恐孙权必怀伐吴之恨,借此而入,当之如何?

诸葛亮 (唉,)陛下呀!

诸葛亮
【西皮摇板】量孙权必观望兵未出境,止需用一能士说彼连横。臣观鱼非潇洒寻思线引,看机变好得个这舌辩的苏秦呐。

刘禅 如相父之言,那五路大兵不日全退了?

诸葛亮 然。

刘禅
呵哈哈\ldots{}\ldots{}哎呀呀相父,你真有移星换斗之智,神鬼不测之机。使寡人万虑皆释矣。

诸葛亮 陛下既释怀疑,可请驾快快回宫,奏知太后要紧呐。

刘禅 既有此万全之策(或:万全之计),何必这等着忙,定要赶奏太后。

诸葛亮
不是呵,陛下若不及早回奏\protect\hyperlink{fn262}{\textsuperscript{262}},诚恐太后已向太庙召臣,那时教老臣何以担当得起?

刘禅 (呃,)此话(\ldots{}\ldots{}呃,)相父何以知之?

诸葛亮 (呃,)不过是推情度理(而已)。

刘禅 (唉,)真正羞煞群僚也(或:真正愧煞群僚也)。

刘禅 【西皮摇板】你谋猷真使人钦敬,不愧调和鼎鼐臣。

诸葛亮
【西皮摇板】鞠躬尽瘁臣之分,敢忘先帝委托恩(或:怎敢忘却先帝恩)。请主回宫心安定,

(\textless{}\textbf{牌子}\textgreater{}刘禅、诸葛亮出门,众上,刘禅上辇介)

诸葛亮 老臣送驾。

刘禅 啊哈哈哈\ldots{}\ldots{}

(刘禅笑介,邓芝点头,诸葛亮望)

诸葛亮 邓大夫暂留一步。

(众、刘禅下,留邓芝,\textless{}\textbf{牌子}\textgreater{}停)

诸葛亮 【西皮摇板】请留大夫说分明。

诸葛亮 请。

邓芝 请。

诸葛亮 大夫请坐。

邓芝 谢座。

(诸葛亮右上,邓芝左偏,坐)

邓芝 不知丞相有何面谕?

诸葛亮 挽留大夫,非为别事,有一桩国事难心领教。

邓芝 何事难心?

诸葛亮 今有魏、蜀、吴,鼎分三国,欲讨二国,一统中兴,请问大夫,先讨哪国?

邓芝
(这\ldots{}\ldots{})若以愚意论之,魏虽汉贼,占据中原,其势甚大,急难摇动(呵),合当徐徐缓图;今当主上新登宝位,民心未安。当与东吴连合\protect\hyperlink{fn263}{\textsuperscript{263}},结为唇齿之邦,永结盟好,一洗先帝之怨,此乃长久之计。未审丞相钧意若何?

诸葛亮 呵哈哈哈哈\ldots{}\ldots{}(笑介)吾亦思之久矣,无奈不得其人。

邓芝 其人何用?

诸葛亮
吾欲使人往结东吴,大夫既明此意,必能不辱君命。这一大任,非大夫不可。

邓芝 邓芝才疏智浅,诚恐有负丞相所托。

诸葛亮 明日我便奏知天子,大夫休要谦让,有负吾意。

(诸葛亮
【西皮快板\protect\hyperlink{fn264}{\textsuperscript{264}}】休谦让莫推辞听我言讲,我和你作臣宰同侍君王。须念在先皇爷恩如海样,谈国政量人才非比寻常。同受过托孤重遗命曾降,也是你明此意劳苦应当。与东吴结唇齿好言讲上,灭汉贼(或:灭国贼)报国仇美名远扬。伯苗你若推辞【转西皮摇板】(你)不肯前往,笑西蜀无能将\textbf{志}成风霜。况先皇待臣宰手足一样,秉赤胆方显你干国忠良。)

(邓芝 丞相。)

邓芝 【西皮摇板】先皇爷托孤情恩深海样,去东吴见孙权自有主张。

邓芝 谨遵台命,邓芝告退。

诸葛亮 (且慢,)书房小酌,聊佐行色。

邓芝 多谢丞相。

诸葛亮 请。

诸葛亮 【西皮摇板】我与你到书房饮酒欢畅,到明天同入朝启奏君王。

诸葛亮 大夫请。

邓芝 丞相请。

诸葛亮 正是:(念)但愿仲谋纳君训,

邓芝 (念)得统中华贺升平。

诸葛亮 啊,

邓芝 啊,

诸葛亮、邓芝 哈哈哈哈\ldots{}\ldots{}(笑介)

诸葛亮 大夫请。

邓芝 丞相请。

诸葛亮、邓芝 呵呵哈哈哈\ldots{}\ldots{}(笑介)

(诸葛亮、邓芝下)

{[}第十二场{]}

(\textless{}\textbf{牌子}\textgreater{}旦扮祝融夫人\textless{}\textbf{点绛唇}\protect\hyperlink{fn265}{\textsuperscript{265}}\textgreater{}上)

祝融夫人
(念)自幼生长在南方,喜读战策演刀枪。上阵能斩千员将,谁人敢犯我边疆。

祝融夫人
咱家乃洞府都蛮王孟获之妻祝融夫人是也。只因中原皇帝曹丕兵督五路,攻取西蜀。遣臣前来聘请咱家大王起蛮兵十万攻打川南四郡,去之日久,不见回来。是咱家放心不下,为此催办粮草,置买水牛、菜蟒,咱家亲身押赴军营。嘟,众蛮兵------

众 啊。(众应)

祝融夫人 咱家命你们所办粮草等物可曾齐备?

众 齐备多时。

祝融夫人 随咱家解送军营。

众 啊。(众应)

祝融夫人 【西皮导板】汉室三分争江山,

祝融夫人
【西皮原板】北魏使臣把兵搬。大王率领兵十万,攻打四郡夺西川。咱家算来日期远,不见大王转回还。解押粮草日夜赶,到军营花沾雨露续团圆。

{[}第十三场{]}

(四文堂,净扮魏延上)

魏延 {[}引子{]}奉命守边关,敌将心胆寒。

魏延
(念)少年英勇走天涯,杀死韩玄献长沙。弃暗保定刘先主,占据西川定邦家。

魏延
某乃西蜀大将魏延,奉了军师将令,命俺挡住蛮王孟获,不要临阵交锋;又道贼心性多疑,必要自退。我不免照书行事。

(报子上)

探子 报,蛮兵退去。

魏延 再探。

报子 得令。

(报子下)

魏延 且住,果然不出军师妙算,趁此追杀前去。众将官,杀!

(魏延下)

{[}第十四场{]}

(\textless{}\textbf{牌子}\textgreater{}众、净扮孟获上)

孟获
孤,都蛮王孟获,今有北魏皇帝聘请孤家帮助,因此领了蛮兵十万攻打川南四郡。孤自安营以来,蜀将并不出马交锋,见他军马每日左出右入,右入左出,不知是何缘故。孤家素闻诸葛亮诡计多端,不要入他圈套,孤不免将人马撤回,暂归蛮洞,再作计较。

(报子上)

报子 报,蜀兵追杀前来。

孟获 再探。

报子 得令。

(报子下)

孟获 嘟,众蛮兵,迎上前去。

(魏延、众人会阵上)

孟获 蜀将通名。

魏延
听者,某乃西蜀大将魏延,尔知道某家厉害\protect\hyperlink{fn266}{\textsuperscript{266}},快些下马受死。

孟获 魏延,孤家开恩,饶尔不死,竟敢大胆追赶孤王,前来送死。

魏延
住了,蛮贼无故兴兵,助逆侵犯边界,占据疆土。要想逃走,留下尔的人头。

孟获 少要多言,看枪(或:看刀)。

(魏延、孟获二人开打,下)

{[}第十五场{]}

(四手下站门上,关兴、张苞上)

张苞 俺,张苞。

关兴 俺,关兴。

张苞 贤弟请了。

关兴 请了。

张苞
你我奉了军师将令,带领人马四路接应。适才探马报道:魏延追赶蛮王孟获,不知胜败如何。

关兴 你我前去接应。前去接应杀退那贼。

张苞 好,就此迎上前去。

张苞、关兴 众将官,杀上前去。

(众全下)

{[}连场{]}

(众围魏延,关兴、张苞上,挑开起打;众围孟获,孟获败下,关兴、张苞、魏延追下)

{[}第十六场{]}

(众、祝融夫人上,报子报上)

报子 报,大王遭了围困。

祝融夫人 再探。

报子 得令。

(报子下)

祝融夫人 蛮兵们。

众 有。

祝融夫人 杀上前去。

(众全下)

{[}第十七场{]}

(众围孟获、孟优,旦上救下;关兴、张苞追下;祝融夫人与魏延架住,磕开)

魏延 杀来杀去,杀出一个蛮婆来了。呔,那蛮婆少要送死,老爷开恩饶你去罢。

祝融夫人
住着。我乃都蛮王孟获之妻祝融夫人是也。你们知道咱家厉害,就在马前磕头饶你们不死。

魏延 休得胡言,放马过来。

(魏延、祝融夫人开打,关兴、张苞上战介;蛮众败下,蜀众追下;祝融夫人上)

祝融夫人 这厮们果然厉害,待咱家飞刀伤他便了。

(张苞等追上)

祝融夫人 看咱家飞刀取你。

(张苞坠马介,众救下;祝融夫人拉孟获,蛮众随下;魏延、众上)

魏延 张小将军怎么样了?

张苞 末将身无伤损,可惜战马被她杀死。

魏延、关兴 此乃万幸。谢天谢地。

张苞 快快换马。待俺追上蛮妇,好报杀马之仇。

魏延 将军不必如此,天色已晚,道路不明,趁此收兵。

关兴 老将军之言甚是。

张苞 便宜了蛮婆。

魏延、关兴、张苞 众将官,收兵。

(\textless{}\textbf{牌子}\textgreater{}众下)

{[}第十八场{]}

(祝融夫人搀孟获上)

孟获
孤自兴兵以来,从无如此大败,似这等狼狈不堪,有何颜面回见各家洞主。我不免碰死了罢。

(祝融夫人拉孟获)

祝融夫人
大王不要如此短见,自古道:军家胜败乃古之常理。依咱主意,暂将人马撤回蛮洞,养足锐气,平整人马,再来报仇不迟。

孟获 \textless{}\textbf{叫头}\textgreater{}诸葛亮,孔明!

孟获 孤家与你誓不两立也。

祝融夫人 走了的好。

孟获 悄悄地收兵。

(孟获、祝融夫人同下)

{[}第十九场{]}

(四太监、一大太监站门,净扮孙权上)

孙权 {[}引子{]}坐镇江东,三分鼎;半壁山河。

孙权
(念)陆逊年幼智超群,蜀兵百万尽皆焚。看来孤王有福分,刘备命丧白帝城。

孙权
孤,孙权,今有曹丕聘孤兵伐西蜀,孤王难作决策,也曾命人探听各路消息,未见回报。

虞翻 (内)走啊。

(虞翻上)

虞翻
【西皮摇板\protect\hyperlink{fn267}{\textsuperscript{267}}】奉使连朝暗徵听,不道诸葛果然能。忙上银安将情禀,见了主公说分明。

虞翻 虞翻参见主公。

孙权 命你徵听曹兵各路进取西蜀消息怎么样了?

虞翻
主公容启:(念)曹兵四路寇蜀,诸葛调军相迎:马超西平退羌兵,疑兵孟获远遁;孟达推病不出,子龙拒走曹真。眼见四路尽解纷,谁与西蜀争胜。

孙权 啊,听你所言,那西蜀四路的曹兵,竟多被孔明暗调兵马,全皆逐退了。

虞翻 便是主公,幸喜听了陆逊之言,未曾动兵,不然今日也要羞归江东矣。

孙权 果然。咳,孔明啊\ldots{}\ldots{}你真有神通也。

孙权 【西皮摇板】似此神通令人敬,堪笑曹丕枉用心。兴衰此际难拟定,

(薛综上)

薛综 【西皮摇板】狂儒胆敢来批鳞。

薛综 臣薛综启事:今有西蜀邓芝特来请见主公。

孙权 他来见孤何意?

薛综 此定是孔明遣他来作说客,退我第五路兵耳。

孙权 他来何以答之?

薛综
依臣鄙意,可于殿前\protect\hyperlink{fn268}{\textsuperscript{268}}设一大大油鼎,贮油数百斤,下用木炭烧得烈烈腾沸;再选身长面大\protect\hyperlink{fn269}{\textsuperscript{269}}武士千人,执利刃从宫门直排至殿角,后唤邓芝入见。他自胆裂魂飞,彼若开言责以郦食其说齐故事,可效那田广旧例而烹之。看他怕也不怕。

孙权 甚好。你就去传旨安排者。

虞翻 领旨。

(虞翻下)

孙权 你去候殿廷排列齐整,然后引那邓芝来见。

薛综 领旨。

(薛综应下)

孙权
【西皮原板】建邺王气承天运,黄武称号顺人心。虽云蜀、魏峙如鼎,谁似孤江东群俊英。武略初仗周公瑾,他壮猷深谋谁比伦。曹兵百万犯吾境,势如压卵好惊人。天意存吴东风趁,火炬烧他无几存、若不是关公释曹命,魏家何能鼎足分。近日里全仗小陆逊,年幼智广独超群。刘备妄想报仇恨,连络七百结下营。猇亭用计火攻盛,蜀兵百万尽皆焚。看来孤王有福分,刘备命丧白帝城。眼前刘禅遣使命,孤岂做齐王烹郦生。銮杖此际排齐整,

(吹打,四值殿、四小太监执銮杖左右上,往内抄;孙权上高台,众在高台后;四校尉抬油鼎上,放前中间,虞翻、薛综上)

虞翻、薛综 万岁。

孙权
【西皮快板】器杖刀剑亮如银。殿角之下设油鼎,汤沸火烈焰腾腾。教那蜀使来认一认,方知东吴不虚名。

孙权 来,

孙权 【西皮快板】众卿替孤传一令,速宣邓芝来觐至尊。

众 宣邓芝上殿。

邓芝 (内)来也。

(邓芝上)

邓芝 啊。

邓芝
【西皮摇板】我自谓钦承皇王命,到此连合去灭魏人。只见他列杖设油鼎,这其间教我解不明。

邓芝
且住,我今奉命而来,原想与他连合伐魏。看他不以礼待,反而列杖设鼎召我。呜哈哈哈\ldots{}\ldots{}(笑介)

邓芝 (念)孤身到虎穴,气节足凌云。从来试威武,岂能屈儒生。

邓芝
【西皮快板】孤身入了虎穴境,志谋气节贯凌云。撩袍端带金殿进,看那吴王怎样行。

邓芝 啊大王,邓芝奉揖了。

孙权
呃嗯------你既奉使来朝,缘何见孤长揖不拜\protect\hyperlink{fn270}{\textsuperscript{270}}?

邓芝 吾乃上国天使,来此小邦,汝不倒履相迎,便为不恭,何必拜汝。

孙权
哦,想尔\protect\hyperlink{fn271}{\textsuperscript{271}}欲掉三寸之舌,意效郦生说齐王么?但孤非田广者比,汝却作了食其之惨。速赴鼎镬,毋得饶舌。

邓芝 哈哈哈\ldots{}\ldots{}(笑介)

邓芝 人人皆言东吴是个名贤之邦,今日一见,竟惧怕一儒生,何其鄙哉。

孙权
寡人富有半壁山河,强有兵将数百余万,何惧一匹夫乎?量汝不过为诸葛亮来做说客,欲孤绝魏向蜀,可是么?

邓芝
嗯。吾实奉大汉丞相诸葛孔明钧旨,特为汝来陈说利害。请问近日还是向蜀乎,还是向魏乎?

孙权 而今曹丕据有中原,磐石之坚,尔那西蜀刘禅焉能与魏抗衡乎?

邓芝
唉,鄙哉斯言也。夫魏虽窃据中原\protect\hyperlink{fn272}{\textsuperscript{272}},实为篡贼苗裔;蜀虽暂处西川,实为大汉帝脉。自上古以来,几见篡贼之后而能长享之理。

邓芝
【西皮原板】何故出言直不慎,全无高下枉批评。试问那王莽移汉鼎,他能可\protect\hyperlink{fn273}{\textsuperscript{273}}遗享与子孙?一朝事败身家尽,惨祸波及家满门。曹丕贼目前似侥幸,我量他不久祸临身。炎汉昭穆承天运,不日旧业重复兴。凡有疑惑不自省,则恐怕前车覆而后车跟。

孙权 啊。

孙权
【西皮摇板】他言出金石令人信,不似张、苏说连横\protect\hyperlink{fn274}{\textsuperscript{274}}。

孙权 邓芝。

孙权 【西皮摇板】兴亡大势孤且不问,如今之势怎样行。

邓芝
【西皮快板】承王问,芝直禀,非敢虚谬论世情:大王诚算命世主,吾丞相诚算佐命臣。一边仗有山川险,一边仗有江河奫\protect\hyperlink{fn275}{\textsuperscript{275}}。若能连合为唇齿,兼併山河可二分。如或弃蜀归魏贼,我川兵不久顺流征。那时节蜀、魏同临境呃,凭便是铁桶的山河也保不成。

孙权 哦\ldots{}\ldots{}罢!

孙权 【西皮摇板】孤便绝魏从伊请,谁为介绍两通情。

邓芝 喏。

邓芝 【西皮摇板】王欲使臣臣从命,若还疑臣便烹臣。

邓芝 大王。

邓芝 【西皮摇板】如不信臣即赴油鼎,

邓芝 罢!

(邓芝扑鼎介)

孙权 快快拉住。

(众应)

孙权 【西皮摇板】先生自是有信人。

孙权 将油鼎抬下,请邓先生上殿相见。

(众应;吹打,孙权下位,吹打住;校尉抬鼎下)

邓芝 大王。

孙权 先生。

孙权 【西皮摇板】你浩然之气令人敬,莫哂小量待高人。

邓芝 哎呀,大王过谦了。

孙权 先生一番金石之言,使孤顿开茅塞,从此吴、蜀连合,协力伐魏。

邓芝 (念)承蒙金诺予,看太平有待。

孙权 哈哈哈\ldots{}\ldots{}(笑介)

孙权 吩咐摆酒,与先生接谈。

(众应)

邓芝 多谢大王。

孙权 先生:(念)从此蜀、吴连合定,孤与先生叙主宾。

(孙权拉邓芝下,众下)

\textbf{天水关 之
诸葛亮}\protect\hyperlink{fn276}{\textsuperscript{276}}

{[}第一场{]}

{[}引子{]}带砺山河\protect\hyperlink{fn277}{\textsuperscript{277}},掌丝纶,运筹帷幄。(或:威权秉正,掌丝纶,运筹帷幄。或:运筹帷幄,掌丝纶,威权秉正。)

(念)先王晏驾白帝城,托孤之事嘱孔明。身受皇恩无可报,忠心耿耿扶后君。(或:汉室纷纷齐弄谋,军中战马不停留。北魏东吴不扫定,黎民涂炭何日休。)

山人复姓诸葛名亮,字孔明,道号卧龙。蒙先帝厚恩,三顾茅庐聘请下山,恢复汉室。老王宴驾,辅保幼主继位,重整汉室基业。今以魏、蜀、吴鼎足三分,昨日修下出师表章。启奏后主,兵出祁山,征服中原。(或:老夫诸葛亮。蒙先帝三顾之恩,托孤之重。一要扫灭孙吴,二要恢复汉室。前者平定孙、曹五路之兵,刻今\protect\hyperlink{fn278}{\textsuperscript{278}}蜀中安稳,为此写就出师表章。呈请幼主,准予兵出祁山。)

童儿,捧了表本、朝服,带路朝房去者。(或:来,带好朝服,金殿见驾去者。)

【二黄摇板】蒙先主三顾请才下山林,白帝城托付事谨记在心。诸葛亮出祁山汉室整顿,怎能够辜负了托孤厚恩。(或:蒙圣恩三顾请才把山下,承受那托孤恩怎敢有差。此一番上金殿参王见驾,但愿得扫北魏重整汉家。)

{[}第二场{]}

嗯哼!

(念)手捧出师表,把本奏当朝。

臣,诸葛亮见驾(或:老臣见驾)。吾主万岁。

万万岁。

谢座。

为臣修下出师表章,启奏万岁,龙目御览。

(刘禅 \ldots{}\ldots{}寡人怎生舍得。)

陛下!(或:万岁呀!)

【二黄慢板】先帝爷白帝城龙归海境,传口诏命老臣常挂(或:时刻)在心。命老臣辅陛下汉室重整,命老臣把孙、曹定要扫平。(或:命老臣辅我主社稷重整,命老臣把孙、曹定要扫平。)出师表并无有别端议论(或:别桩议论),望陛下准臣本臣要发兵。

【二黄原板】食君禄当报效臣把忠尽,臣怎敢劳陛下长亭饯行。

{[}第三场{]}

【二黄原板】接过了吾主爷皇封御饮,便转身祭过了旗纛尊神。

【二黄原板】我朝中有二臣忠心耿耿,蒋公琰、费文伟两大贤臣。主临朝大小事与他们议论,宫中的事阃外事依理\protect\hyperlink{fn279}{\textsuperscript{279}}而行。

【二黄散板】请我主龙驾归臣要发兵(或:臣要启程)。

【二黄散板】传一令众将官齐跨金镫,

【二黄散板】文武官且免送响炮启程(或:响炮起营)。

众将官,

兵出祁山!

{[}第四场{]}

(念)安排诓军计,三军掌握中。(或:眼观旌旗起,耳听好消息。)

老将军回来了。

胜负如何?

可曾问过来将名姓(或:此人名姓)?

姜维\ldots{}\ldots{}

他乃大孝之人,待山人略施小计,收服于他。

老将军后帐歇息。

来,传旗牌。

命你去往冀州,迎接姜维老母。一路之上,小心侍奉,不可怠慢。记下了。

众将进帐。

站立两厢。

众位将军------

今日出马,非比寻常。

听山人令下:

【西皮摇板】一枝将令往下传,

【西皮快板】镇北将军名魏延。假扮姜维关前站(或:假扮姜维去骂关),口口声声出反言。\protect\hyperlink{fn280}{\textsuperscript{280}}

【西皮摇板】再把马岱(或:西凉马岱)一声唤,

【西皮快板】山人言来听根源:若遇姜维莫交战,引他兵至(或:引他兵败)凤凰山。

【西皮摇板】龙虎二将一声唤,

【西皮快板】自古英雄出少年。二马连环来交战,杀得他四路无门跌跪马前。

【西皮摇板】人来看过四轮辇,

【西皮摇板】一战成功定中原。

{[}第五场{]}

【西皮散板】四面设下(或:四下安排)天罗网,姜维小儿无躲藏。四轮车且把(或:下得车来)山岗上,

【西皮散板】准备倭弓(或:弓弩)射虎狼。

(姜维 【西皮散板】\ldots{}\ldots{},手执长枪往上闯。)

大胆!

【西皮散板】姜维小儿莫逞强。要保来保(或:要降来降)圣明主,为何扶保篡位王。

(众 姜维归降!)

【西皮导板】玄天大数如反掌,

(众 姜维归降!)

啊------哈哈哈\ldots{}\ldots{}(笑介)

【西皮原板】事不遂心意彷徨。我不爱将军(你)的韬略广,爱将军是一个贤孝儿郎(或:行孝儿郎)。

(姜维 【西皮原板】冀州还有老萱堂。)

【西皮原板】我早已安排下令堂母,

(姜维 谢丞相。)

【西皮原板】将军不必(或:休得)挂心旁。一出祁山收此将(或:收良将),心中得意喜气洋洋。怕只怕(或:怕的是)五丈原秋风降,【转西皮二六】军国大事(或:军中大事)付与他承当。

【西皮摇板】请将军同把车辇上。

(姜维 【西皮摇板】姜维跌跪跪道旁。)

【西皮散板】将酒宴摆至在中军帐,我与将军论军机叙叙衷肠。

\newpage
\hypertarget{ux7a7aux57ceux8ba1-ux4e4b-ux8bf8ux845bux4eae}{%
\subsection{空城计 之
诸葛亮}\label{ux7a7aux57ceux8ba1-ux4e4b-ux8bf8ux845bux4eae}}

{[}第一场{]}

{[}引子{]}羽扇纶巾,四轮车,快似风云。阴阳反掌定乾坤,保汉家两代贤臣。

列位(或:众位)将军少礼。

(众 啊!)

(念)忆昔当年居卧龙,万里乾坤掌握中。扫尽狼烟扶汉统,人曰男儿大英雄。

老夫,复姓诸葛名亮,字孔明,道号卧龙。先帝爷白帝城托孤遗言,扫荡中原,保留汉室。闻得司马懿兵至岐山,必然夺取街亭。必须派一能将,前去防守。啊,众位将军,

(众 啊!)

哪位将军带领人马,镇守街亭,甘当此任。

(马谡 \ldots{}\ldots{}镇守街亭。)

那司马(懿)虽则年迈,用兵如神,将军不可轻敌。

(马谡 \ldots{}\ldots{}攻无不取\ldots{}\ldots{}何况小小街亭。)

(呃嗯------)

(马谡躬揖)

街亭虽小,干系甚重啊。

军无戏言。

(马谡 愿立军状。)

好,当帐立来。

帐外候令。

(马谡下)

众位将军,

(众 丞相。)

哪位将军愿协同马谡,镇守街亭,当帐请令。

王将军素来谨慎;此番到了(或:去至)街亭,必须靠山近水,安营扎寨。扎寨已毕,画一四至八道地理图,速报我知。

(王平 得令。)

赵老将军听令:带领三千人马,镇守列柳城。

马岱听令:押解粮草,军中(或:军前)需用。

马谡进帐。

(马谡上)

一旁坐下。

(马谡 \ldots{}\ldots{}有何密令?)

今逢大敌,非比寻常。我有一言,将军听了:

【西皮原板】两国交锋龙虎斗,各为其主统貔貅。管带三军要宽厚,赏罚中公平莫要自由。此一番领兵去镇守,靠山近水把营守(或:把营收;把陉\protect\hyperlink{fn281}{\textsuperscript{281}}守)。

(马谡
【西皮摇板】\ldots{}\ldots{}辞别丞相出帐口,\ldots{}\ldots{}顺水去推舟。)

【西皮摇板】先帝爷白帝城叮咛就,汉诸葛扶幼主岂能无忧。但愿得此一去扫平贼寇,免得我亲自去把贼收。

{[}第二场{]}

(念)兵扎祁山地,要擒司马懿。

(旗牌上)

(旗牌 门上哪位在?)

传。

罢了。

奉何人所差?

(旗牌 王平王将军所差。)

手捧何物?

(旗牌 地理图。)

展开。

命你去到列柳城,速速将赵老将军调回营来!快去快去!

(旗牌下)

啊------?!好大胆的马谡哇。临行怎样吩咐(或:嘱咐)与你?靠山近水,安营扎寨。怎么,你偏偏要在山顶扎营?!哎呀,大略街亭难保哇。

(探子上)

(探子 报!)

(探子 \ldots{}\ldots{}失守街亭!)

再探!

(探子下)

如何,果然把街亭失守了。唉------呀!虽然马谡失守街亭,乃诸葛(亮)之罪也。

(探子上)

(探子 报!)

(探子 \ldots{}\ldots{}带兵夺取西城!)

再探!

(探子下)

呜哙呀!司马懿居然带兵夺取西城来了。唉------

(诸葛亮站起)

当初先帝爷白帝城托孤之时言过:马谡言过其实,不可大用。悔不听先帝遗言,今日错差马谡,失守街亭,悔之晚矣呀!

(探子上)

(探子 报!)

再,再探!

(探子下)

啊?!,司马懿的兵,他来得好快呀!嗯------人言司马,用兵如神,今日一见,令人可敬呐,令人可服!

哎呀且住!

(诸葛亮站起)

想这西城的将官,俱被老夫调遣在外,所剩下尽是些个老弱残兵。倘若司马兵到,难道说教我束手被擒,这束手------被擒\ldots{}\ldots{}哎呀!\textless{}\textbf{乱锤}\textgreater{}

老军们进见。

(老军甲 司马兵到,)

(老军乙 心惊肉跳。)

(老军甲 见了丞相,)

(老军乙 急忙跪倒。)

(二老军 有何吩咐?)

命尔等将四门大开,每门上二十名老军,洒扫街道。司马兵到,不可惊慌浮躁,违令者斩。

(老军甲 丞相吩咐我,)

(老军乙 准死不能活。)

天呐,天------

汉室兴败就在这空城一计也!

【西皮摇板】我用兵数十年从来谨慎,错用了小马谡无用之人。无奈何定空城计我的心神不定,望空中求先帝大显威灵。

{[}第三场{]}

【西皮摇板】恨马谡失街亭令人可恨,这时候倒教我难以调停。

呃------

【西皮摇板】老军们因何故纷纷议论,

【西皮摇板】国家事用不着尔等劳心。

【西皮摇板】这西城地原本是咽喉路径,

【西皮摇板】我城内早埋伏有十万神兵。

(老军甲 我再到里头瞧瞧去,)

(老军乙 你瞧见什么没有?)

(老军甲
什么我也没瞧见,瞧见李佩卿在那拉胡琴呢!\protect\hyperlink{fn282}{\textsuperscript{282}})

【西皮摇板】叫老军扫街道把宽心放稳(或:把宽心拿稳),

【西皮摇板】退司马保空城全仗此琴。

(司马懿上)

(司马懿 【西皮原板】大队人马往前进,\ldots{}\ldots{})

【西皮慢板】我本是卧龙岗散淡的人,评阴阳如反掌保定乾坤。先帝爷下南阳御驾三请,算就了汉家的业鼎足三分。官封到武乡侯执掌帅印,东西战南北剿博古通今。周文王访姜尚周室大振,汉诸葛怎比得前辈的先生。闲无事在敌楼我亮一亮琴音,

(诸葛亮抚琴介)

呵呵哈哈哈\ldots{}\ldots{}(笑介)

【西皮原板】我眼前缺少个知音的人。

【西皮二六】我正在城楼观山景,又听得城外乱纷纷。旌旗招展空翻影,原来是司马发来的兵。我也曾差人去打听,打听得司马领兵往西行。一来是马谡无谋少才能,二来是将帅不和(或:二将不和;两将不和)失街亭。连得三城多侥幸,贪而无厌又夺我西城。诸葛亮在敌楼把驾等,等候你到此谈、谈、谈谈心。西城的街道(或:城外的街道)打扫净,准备司马好屯兵。到此并无有别的敬,早备下羊羔美酒犒赏你的三军(临)。既到此就该把城进,为什么你犹豫不定、进退两难为的是何情。我只有琴童人两个,我是又无有埋伏又无有兵(或:我是又没有埋伏又没有兵)。你不要胡思乱想心不定,来来来,请上城来听我抚琴。

(探子 \ldots{}\ldots{}兵退四十里呐!)

(险呐!)

【西皮散板】人言司马善用兵,到此不敢进空城呐。诸葛从来永不弄险,险中又险显才能。

哎呀老将军呐!方才司马懿兵临城下,被我(或:被山人)用空城计将他哄走。必然复返,老将军速速抵挡一阵。

(赵云 得令!)

正是:(念)虎在深山人咸远,蛟龙得水又复还。

险呐!

{[}第四场{]}

【西皮摇板】算就汉家三分鼎,险些一旦化灰尘呐。

(探子 报!)

(探子 马谡、王平回营请罪!)

升帐。

有请。

带王平!

【西皮摇板】怒上心头难消恨,

(王平 丞相。)

【西皮快板】抬头只见小王平。临行再三嘱咐你,靠山近水扎大营。大胆不听我的令,失守街亭你的罪不轻。

(王平 丞相!)

(王平
【西皮快板】丞相不必怒气生,王平言来听分明:马谡不听丞相令,他在山顶扎大营。丞相若是不肯信,现有画图作证凭。)

【西皮快板】若不是画图来得紧,定与马谡同罪名。将王平责打【转西皮摇板】四十棍\protect\hyperlink{fn283}{\textsuperscript{283}},

【西皮摇板】快带马谡这无用的人呐。

(马谡 唉呀!)

【西皮快板】见马谡跪帐下,不由老夫怒气发。大胆不听我的话,失守街亭差不差。

【西皮散板】吩咐两旁刀斧手,快斩马谡正军法。

【西皮摇板】见马谡只哭得珠泪\textless{}\textbf{哭头}\textgreater{}洒,

【西皮摇板】我心中好似乱刀扎。

\textless{}\textbf{三叫头}\textgreater{}马谡!幼常!唉------参军呐!(哭介)

马谡,你临行之时,(当着满营的将官,)先立下军令状啊。如今若不将你正法,何以服众?

\textless{}\textbf{三叫头}\textgreater{}马谡!幼常!唉------参军呐!

(众 哦\ldots{}\ldots{})

来!斩!

招回来!

马谡哇,方才言道(或:方才言过):家有八旬老母,无人侍奉。你死之后,将你兵马钱粮,拨与你老母,以为养老之费。

\textless{}\textbf{三叫头}\textgreater{}马谡!幼常!唉------参军呐!(哭介)

来!斩,斩,斩,斩\ldots{}\ldots{}

【西皮散板】我哭、哭一声马参军,叫、叫、叫一声马幼常啊。未出兵先立下军令状,可叹你为国家刀下身亡。

\textless{}\textbf{哭头}\textgreater{}马谡哇!参谋啊!啊!马幼常啊!

(赵云 \ldots{}\ldots{}为何落泪?)

唉!老将军呐!(我哪里哭的是马谡啊!)当初先帝爷(白帝城)托孤之时言过:马谡言过其实,不可大用。悔不听先帝遗言,至有今日之过。我哪里哭的是马谡哇,乃深恨己之不明,追思先帝遗言呐,呃呃呃\ldots{}\ldots{}(哭介)

也罢,待山人拜本进京,奏明幼主,贬去武乡侯。整顿人马。再与司马决战。

后帐有宴,与老将军贺功。

\newpage
\hypertarget{ux6218ux5317ux539f-ux4e4b-ux8bf8ux845bux4eae}{%
\subsection{战北原 之
诸葛亮}\label{ux6218ux5317ux539f-ux4e4b-ux8bf8ux845bux4eae}}

{[}第一场{]}

{[}引子{]}掌握兵权,运奇谋,拨乱扶贤。六出祁山,取中原,扫狼烟,全图归汉。

列位将军少礼。

(念)汉室纷纷齐弄谋,军中战马不停留。司马纵有拿云手\protect\hyperlink{fn284}{\textsuperscript{284}},老夫先下钓鱼钩。

老夫,复姓诸葛名亮字孔明,道号卧龙,汉室为臣,官拜武乡侯之职。蒙先帝三顾之恩,托孤之重,一要扫灭孙、曹,二要恢复汉室。我想北原乃魏邦咽喉之要地,必需兴兵夺取。

啊,众位将军,人马可齐?

吩咐兵发北原去者。

{[}第二场{]}

【西皮慢板】想当年在隆中何等潇洒,闲无事听鸟音观看山花。先帝爷三顾请才把山下,曾受过托孤的恩怎敢有差。在蜀中奉王命统领人马,兵行在祁山地来把营扎。这几天我未曾去把仗打,司马懿他必然笑我怕他。选一个黄道日发动人马,

【西皮摇板】扫中原灭北魏重整汉家。

再探!

来,

唤众将进帐。

众位将军少礼。

适才探马报道,司马营中有一员大将,名唤郑文,前来投降。尔等可知此人?

原来如此。

来,

吩咐击鼓升帐。

传郑文进帐。

【西皮摇板】见一将跪帐下身躯高大,

【西皮摇板】两眼中含珠泪滚滚似麻。问将军因甚事反背司马,表你的名和姓哪里有家。

【西皮摇板】人道那司马懿识高才大,却为何今日里把事做差。

【西皮摇板】见过了蜀营将请坐叙话,待山人奏幼主定把功加。

再探!

郑将军,

司马营中又来一将,名唤秦朗,他的武艺如何?

哦,原来如此。待山人差一能将前去对敌。

郑将军为何阻令?

哦------好,这一头功,就让与郑将军。

郑文听令!

命你大战秦朗,不得有误。

哈哈哈哈\ldots{}\ldots{}(笑介)

【西皮摇板】看起来吾主爷洪福天大,收此将好一似锦上添花。

【西皮摇板】待山人去观阵众将退下,

【西皮摇板】看郑文与秦朗如何杀法。

【西皮摇板】他二人见了面刀枪并架,未战到三两合就把人杀。莫不是司马懿叫他来行诈,

不错,是的。

回营。

【西皮摇板】等郑文回营来仔细盘查。

{[}第三场{]}

郑将军头功,号令辕门。请坐。

郑将军,

司马营中,有几个秦朗?

哦?并无第二。呵呵呵呵\ldots{}\ldots{}(冷笑介)

郑将军,你这是何苦哇?

【西皮原板】在茅庐曾学过孙吴兵法,仗三韬与六略扶保汉家。适才间斩秦朗多多劳驾,我在那祁山上活活笑煞。

【西皮原板】你道那小秦朗武艺高大,未战到三两合就把他杀。分明是司马懿教你行诈,你可知诸葛亮料事如神半点不差。

郑文!

【西皮摇板】谁不知诸葛亮智谋广大,尔好比螳螂臂井底之蛙。区区的诈降计敢来戏耍,瞒得过诸葛亮?你瞒不了他。

大胆!

呵呵呵呵\ldots{}\ldots{}(冷笑介)

【西皮二六】在帐中说与你一派好话,谁教你自逞那满腹才华。诸葛亮兴人马谁不惧怕,我把那司马懿当作小娃。区区的诈降计敢来行诈,我心内似明镜鉴照无差。我劝你在此间讲了实话,待山人奏幼主定把功加。你若是满口中胡言乱答,顷刻间传将令尔的血染黄沙。谁不知诸葛亮能知兵法,我服尔好大胆,我服尔好大胆敢在那虎口扳牙。

大胆!

真真的大胆!

【西皮摇板】郑将军你既然讲了实话,从今后弃暗投明扶保汉家。

郑将军请坐。

郑将军你与司马懿定下何计?

何谓苦肉之计?

如此待山人与他个计上就计。

就烦郑将军修书一封,下到司马营中,教他三更时分,前来偷营劫寨。我这里埋伏停当,纵然擒不住司马懿,呃,也要杀他个片甲不归。

呃,料无推辞的了哇。

传旗牌。

郑将军有差。

转来。要说郑将军暗差。

记下了。

来,将郑文绑了。

郑将军暂受一时捆绑,待山人擒住司马懿,再来发放于你。

押了下去。

呵哈哈哈\ldots{}\ldots{}(笑介)

【西皮摇板】可笑司马少才学,做事全然不揣摩。诈降之计错又错,些小事怎瞒我南阳诸葛。

{[}第四场{]}

【西皮摇板】老夫背地笑司马,

【西皮快板】郑文诈降智谋差。打虎之计安排下,

【西皮摇板】撒出鹰鹞把兔拿。

回来了,司马懿见了书信,怎样回答?

记功一件。

传令下去,将郑文斩首。

命你将郑文首级用白绫裹好,放在食匣之内,司马兵败之后,送至他营,就说老夫念他连日用兵辛苦,送来些小小的薄礼,望祈笑纳。

来,传众将进见。

命你等整顿军马,埋伏祁山脚下,高挑红灯,上写``郑''字旗号。等候司马前来偷营劫寨,一拥杀出,司马兵败,不可追赶,老夫在祁山口等候。

司马懿呀司马懿!这是你自讨其祸,两军阵前,看你有何面目前来见我。

【西皮快板】自从卧龙把山下,排兵布阵从无差。四轮辇且往祁山脚下,

【西皮摇板】两军阵前取笑他。

{[}第五场{]}

【西皮摇板】四下安排天罗网,准备弩弓射虎狼。耳旁又听鸾铃响,

【西皮摇板】功劳簿上写几行。

【西皮摇板】旌旗招展龙蛇样,

司马!

呵呵哈哈哈\ldots{}\ldots{}(笑介)

【西皮摇板】原来司马到战场。自己用兵不思量,诈降之计太荒唐。今日损兵又折将,我看你何颜回转营房。

住口!

【西皮摇板】你人困马乏打什么仗,败阵之将敢逞强。

【西皮摇板】回头派出一员将,

马岱,杀!

且慢。

呵呵哈哈哈\ldots{}\ldots{}(笑介)

【西皮摇板】我这一只虎能挡你一群羊。

【西皮摇板】坐在车辇把令降,大小三军听端详:人马扎在祁山上,选一个黄道吉日动刀枪。

司马!

【西皮摇板】我少陪------少陪,我要收兵将,

收兵。

【西皮摇板】你父子回营去养养伤再摆战场。

请。

\textbf{七星灯}\protect\hyperlink{fn285}{\textsuperscript{285}}

\textbf{{[}第一场{]}}

\textbf{诸葛亮 (念)万事不由人作主,一心难与命争衡。}

\textbf{旗牌 参见丞相。}

\textbf{诸葛亮 你回来了?}

\textbf{旗牌 回来了。}

\textbf{诸葛亮 司马看了衣服、书信有何举动?}

\textbf{旗牌
司马受了巾帼女衣、看了书信,并不嗔怒,呃,反请小人饮宴,席中只问丞相寝食及事之烦简如何,饭间未提军旅之事。}

\textbf{诸葛亮 尔如何抗对?}

\textbf{旗牌
呃,小人言道:``丞相夙兴夜寐,罚二十以上皆亲览焉。所啖之食,不过数升。''}

\textbf{诸葛亮 司马何言?}

\textbf{旗牌 呃,司马并无言语。}

\textbf{诸葛亮 呃------下去。}

\textbf{诸葛亮 唉,司马深知吾也!}

\textbf{诸葛亮
【二黄原板】仰面朝天【转二黄慢板】自己嗟叹,司马懿可算得将中魁元。送脂粉和钗裙不恼不怨,反与那旗牌官酒食来餐。有刚有柔是好汉,我诸葛比司马难上加难。先帝爷下南阳君臣相见,感恩}深重要扭转汉室河山。博望坡烧夏侯(或:博望坡烧曹兵;博望坡烧贼兵)初次交战,借东风助周郎火烧战船。烧藤甲用火攻孟获丧胆(或:蛮夷丧胆),谁不知诸葛亮智能扭天。到如今与司马两下会战,葫芦峪设地雷安排机关。我料他父子们必遭此难,又谁知天不遂也是枉然。一时里心血涌浑身是汗\protect\hyperlink{fn286}{\textsuperscript{286}},

\textbf{诸葛亮 【二黄摇板}】传姜维和魏延速到帐前。

\textbf{旗牌 姜维、魏延进帐。}

\textbf{姜维、魏延 来也!}

\textbf{魏延 【二黄摇板】帐上一声唤,}

\textbf{姜维 【二黄摇板】上前问根源。}

\textbf{姜维、魏延 参见丞相,有何将令?}

\textbf{诸葛亮 魏延。}

\textbf{魏延 在!}

\textbf{诸葛亮 命你巡营瞭哨,司马叫阵,不可出兵,紧守大营,不得有误!}

\textbf{魏延 得令。}

\textbf{诸葛亮 姜维。}

\textbf{姜维 在。}

\textbf{诸葛亮 命你随我后帐安排七星祭坛,一干人等,勿得入内。}

\textbf{姜维 遵令}(或:\textbf{遵命)。}

\textbf{诸葛亮 唉!}

\textbf{诸葛亮
【二黄摇板】设坛拜星求北斗,但愿天意早回头。三寸气在千般用,一旦无常万事休。扫荡中原难回首,}

\textbf{诸葛亮 【二黄摇板】怕的是天意不遂不自由。}

\textbf{{[}第二场{]}}

\textbf{司马懿 【二黄导板】谯楼鼓打罢了初更时分,}

\textbf{司马懿 【回龙】静悄悄出魏营观看天星。}

\textbf{司马懿
【二黄原板】叫人来前引路高岗来进,司马懿观天象细算详情:观东方甲乙木木能生火,观南方丙丁火火能克金。正西方庚辛金金能生水,观北方壬癸水水遇土屯。佔中央戊己土仔细看定,}

\textbf{司马懿 【二黄摇板】五行生克观不清。}

\textbf{司马懿
【二黄摇板】北斗星垣来观看,主星暗淡光不明。看罢天象心拿稳,}

\textbf{司马懿 回营!}

\textbf{司马懿 【二黄摇板】安排巧计擒孔明。}

\textbf{{[}第三场}\protect\hyperlink{fn287}{\textsuperscript{287}}\textbf{{]}}

\textbf{姜维 有请丞相!}

\textbf{诸葛亮 (内)先王呀!}

\textbf{(}正场小座,``七星灯''不摆在正场桌,而是摆在下场门斜场,诸葛亮拄宝剑上\textbf{)}

\textbf{诸葛亮
【二黄慢板】为汉家把我的心血用尽,都只为先帝爷托孤之恩。执法剑进祭坛(或:执宝剑上坛台)实难扎挣,}

\textbf{诸葛亮 【二黄原板】险些儿把老夫跌倒埃尘。}

\textbf{诸葛亮
(念)亮,谨书尺素,上告穹苍:伏望天慈,俯垂鉴听:亮生于乱世,甘老林泉;承昭烈皇帝三顾之恩,托孤之重,誓讨国贼,永延汉祚}\protect\hyperlink{fn288}{\textsuperscript{288}}\textbf{。上求北斗,曲延臣算,非敢妄祈,实由------唉------情切。诶,呃\ldots{}\ldots{}(哭介)}

\textbf{(\textless{}小开门\textgreater{}烧符箓)}

\textbf{诸葛亮 上苍呐!}

\textbf{诸葛亮
【二黄原板】诸葛亮不敢扭天行,为的是我主锦乾坤。拜南斗和北斗}\protect\hyperlink{fn289}{\textsuperscript{289}}\textbf{赐我阳寿,掌簿官执笔吏留下人情。佔中央戊己土深深拜定,}

\textbf{(诸葛亮叩头,拜后下坛台,踱步至上场门,回身看星灯)}

\textbf{诸葛亮 【二黄摇板】见将星比往常显见光明。}

\textbf{诸葛亮 【二黄摇板】虽然是星明亮吉凶未定,}

\textbf{(诸葛亮归小座)}

\textbf{诸葛亮 【二黄散板】怕的是(或:怕只怕)天意难违大事难成。}

\textbf{魏延 【二黄摇板】司马来踏营,近前说分明。}

\textbf{诸葛亮
【二黄摇板】这是我大限有一定,魏延扑熄我的本命灯。将本命灯撇在尘埃地,}

\textbf{姜维 【二黄摇板】丞相发怒为何情。}

\textbf{诸葛亮
【二黄摇板】我拜斗今日六天整,堪堪}\protect\hyperlink{fn290}{\textsuperscript{290}}\textbf{七天大功成。恨魏延他把我本命灯扑熄,我性命就要哇一旦倾。}

\textbf{姜维 啊?!}

\textbf{姜维
【二黄摇板】听一言来怒气生,魏延贼子起反心。手执宝剑将尔斩,}

\textbf{魏延 你要斩哪个?}

\textbf{姜维 要杀你。}

\textbf{魏延 你杀不得。}

\textbf{诸葛亮 将军!}

\textbf{诸葛亮 【二黄摇板】将军息怒且消停。}

\textbf{诸葛亮 魏延。}

\textbf{魏延 在。}

\textbf{诸葛亮 莫非司马前来踏营?}

\textbf{魏延 正是。}

\textbf{诸葛亮 前去抵挡,出帐去罢!}

魏延 遵命!

\textbf{魏延 (念)堪堪孔明不长久,管教蜀营众将休!}

\textbf{魏延 哼!}

\textbf{诸葛亮 姜维搀我出坛!}

\textbf{诸葛亮 【二黄摇板】姜维后营一声请,快快请出李大人。}

\textbf{姜维 有请李大人!}

\textbf{李福 (内)来也!}

\textbf{李福 【二黄摇板】忽听帐上一声请,急忙进帐看分明。}

\textbf{李福 参见丞相!}

\textbf{诸葛亮 李大人。}

\textbf{李福 在。}

\textbf{诸葛亮 这有表章一轴,连夜送往成都,替吾转奏,请吾主龙目御览。}

\textbf{李福 遵命。}

\textbf{诸葛亮 搀扶!}

\textbf{诸葛亮
【二黄摇板】远望成都忙跪定,拜谢我主爵禄恩。羞愧难见刘先主,李大人速速转奏快快登程。}

\textbf{李福 【二黄摇板】辞别丞相忙登程,不分昼夜奔都城。}

\textbf{诸葛亮 姜维!}

\textbf{姜维 在!}

\textbf{诸葛亮 听我吩咐!}

\textbf{姜维 啊!}

\textbf{诸葛亮 【二黄碰板三眼】我和你虽为将帅倒有那师徒之义,}

\textbf{诸葛亮
【二黄原板】必须要秉忠心扶保华夷。一封锦囊交与你,内藏着妙算与神机。我死后三件大事托与你,一桩桩一件件莫要泄机:第一件我死后休得挂孝,第二件必须要缓缓移营。第三件我死后那魏延必反,}

\textbf{姜维 啊?!}

\textbf{诸葛亮
【二黄散板】我自有妙计除此人。我将这奇门遁甲传授你,阵阵不离此图形。这一弩能发十条箭,九伐中原你担承。将军与我传将令,快传那杨仪、马岱与王平。}

\textbf{姜维 杨仪、马岱、王平速速进帐!}

\textbf{杨仪、马岱、王平 【二黄摇板】丞相帐中传将令,一同上前看分明。}

\textbf{杨仪、马岱、王平 丞相醒来!}

\textbf{诸葛亮
【二黄摇板】指望}\protect\hyperlink{fn291}{\textsuperscript{291}}\textbf{霸业兴炎汉,谁知半途不周全。猛然睁开昏花眼,又只见众将官站立面前。}

\textbf{诸葛亮 杨仪!}

\textbf{杨仪 在。(\textless{}小拉子\textgreater{})}

\textbf{诸葛亮
【二黄摇板】我死后军师大印你掌管,事事谨慎要周全。我今与你这小柬,我死之后再来观。}

\textbf{杨仪 遵命。(\textless{}住头\textgreater{})}

\textbf{诸葛亮 子均!}

\textbf{王平 在。(\textless{}小拉子\textgreater{})}

\textbf{诸葛亮
【二黄摇板】王子均近前听召唤,一封小柬带身边。事到头来}\protect\hyperlink{fn292}{\textsuperscript{292}}\textbf{再观看,内有如此与这般。}

\textbf{王平 遵命。}

\textbf{诸葛亮 马岱!}

\textbf{马岱 在。(\textless{}小拉子\textgreater{})}

\textbf{诸葛亮
【二黄摇板】西凉马岱听我言,我有言来记心间。倘若是魏延来造反,这封小柬临阵观。}

\textbf{马岱 遵命。}

\textbf{诸葛亮
【二黄摇板】众将官搀扶我吾主叩见,诸葛亮在营中拜别龙颜。叩罢头抽身起心血上泛,}

\textbf{诸葛亮 呜\ldots{}\ldots{}(吐血介)}

\textbf{诸葛亮 【二黄摇板】我面前站定了庞统士元。}

\textbf{诸葛亮
【二黄摇板】在荆州对把八字算,我二人各有不周全。我算他落凤坡前身带箭,他算我难逃五丈原。霎时间胸内痛(或:霎时间心内痛)心血上泛,}

\textbf{诸葛亮 呜\ldots{}\ldots{}(吐血介)}

\textbf{诸葛亮 【二黄摇板】昏沉沉一旦间命归九泉。}

\textbf{众 丞相啊!}

\textbf{李福 丞相钧体如何?}

\textbf{众 已归仙境。}

\textbf{李福 哎呀,误了吾主大事了!}

\textbf{诸葛亮 嗯哼\ldots{}\ldots{}}

\textbf{姜维 哦,丞相醒转!大人有何圣谕,快快禀来!}

\textbf{李福 启禀丞相:万岁问道:丞相之后,何人接替。}

\textbf{诸葛亮 蒋公琰。}

\textbf{李福 公琰之后?}

\textbf{诸葛亮 费文伟。}

\textbf{李福 文伟之后?}

\textbf{诸葛亮 三国归于\ldots{}\ldots{}}

\textbf{众 丞相啊\ldots{}\ldots{}}

\textbf{姜维 列公且免悲泪,待我打开丞相钧谕观看。}

\textbf{姜维 原来如此。}

\textbf{姜维
(丞相命我等)用沉香木塑成钧体,安放四轮车上。倘若司马踏营,将车推至阵前,司马必然不战自退。}

\textbf{众 原来如此。}

\textbf{姜维 你我后营安排,准备一切便了!}

\textbf{众 请呐!}

\newpage
\hypertarget{ux9664ux4e09ux5bb3-ux4e4b-ux738bux6d5aux5468ux5904}{%
\subsection{除三害 之
王浚、周处}\label{ux9664ux4e09ux5bb3-ux4e4b-ux738bux6d5aux5468ux5904}}

\textbf{{[}第一场{]}}

\textbf{(王浚
【西皮快二六】趁青年你莫当朝嬉夕宴}\protect\hyperlink{fn293}{\textsuperscript{293}}\textbf{,董仲舒他三载未曾窥园。幼而学壮而行经纶开展,那时节报皇家荣耀门田。若得儿洗旧污重新向善,不唯对儿那高、曾、祖,我二老亦可对湛湛青天。)}\protect\hyperlink{fn294}{\textsuperscript{294}}

\textbf{{[}第二场{]}}

\textbf{(王浚 (内)走哇。)}

\textbf{王浚} 【二黄摇板】摘去乌纱换儒巾,谁人识我大元勋。

\textbf{王浚}
老夫(或:下官)王浚,散操回衙,黎民百姓状告恶霸周处。我想周处乃周舫之子,我若将他(当真)查办,教我怎能(或:教我怎样)对得过他那去世先人?为此乔装改扮,出衙私访于他。用言语打动,若(或:倘)能改邪归正亦未可知。只是教我哪里去寻,哪里去找?

\textbf{王浚}
看那旁来一红脸大汉,想是周处。我不免在此等候。等他到来,他有来言,我有去语。

\textbf{王浚} 【二黄摇板】浑玉不琢(或:璞玉不琢)多壑陵,当头棒喝返本真。

周处 【二黄摇板】终日饮酒消愁闷,半世悠悠困风云。

\textbf{王浚} 唉!

周处 【二黄摇板】老丈缘何冲天恨,

\textbf{王浚} 唉,不成世界了!

周处 啊?!

周处 【二黄摇板】叫人心中解不明。

周处 啊,老丈------请了。

\textbf{王浚} 哦,原来是一位壮士。(这厢有礼。)

\textbf{周处 啊,老丈,为何一人在此长叹?莫非有人欺压于你?}

\textbf{王浚}
想老汉(或:老朽)乃(是)唾面自干之人,纵有人欺压于我,亦何敢较量。

\textbf{周处 既然如此,为何在此长叹?}

\textbf{王浚 唉,可叹这宜兴的百姓好不苦也。}

\textbf{周处 却是为何?}

\textbf{王浚} 皆因此地出了三害。

\textbf{周处 哦,出了三害?但不知是哪三害?}

\textbf{王浚} 壮士愿听(或:壮士愿闻)?

\textbf{周处 愿}闻\textbf{。}

\textbf{王浚} 愿闻(或:愿听)?

\textbf{周处 你且讲来。}

\textbf{王浚} 听了------

\textbf{王浚} 【二黄三眼】若提起这三害令人可恨,

\textbf{周处 你慢慢讲来。}

\textbf{王浚} 【二黄三眼】讲出来连壮士(或:连壮士闻此言)也要心惊:

\textbf{周处 第一害------}

\textbf{王浚} 【二黄三眼】第一害那南山出了猛虎,

\textbf{周处 哦,出了猛虎便怎样?}

\textbf{王浚} 【二黄三眼】它遇着(或:倘遇着)行路人骨肉全吞。

\textbf{周处 嗯,这第二害------}

\textbf{王浚} 【二黄三眼】第二害它比那猛虎还狠,

\textbf{周处 哦,那又是什么妖魔鬼怪?}

\textbf{王浚 【}二黄三眼\textbf{】长桥下又出了恶魔蛟精。}

\textbf{周处 哦,出了蛟精,它是怎样的厉害?}

\textbf{王浚}
【二黄三眼】在水中兴波浪吞舟荡\textbf{艇(或:}荡\textbf{坉}\protect\hyperlink{fn295}{\textsuperscript{295}}\textbf{)},到旱道作毒雾苦害行人。

\textbf{王浚}
【二黄三眼】第三害讲出口令人可恨,他比那南山猛虎、长桥孽蛟还狠十分。

\textbf{周处 哦,它是什么妖魔鬼怪?,又是怎样的厉害?}

\textbf{王浚}
【二黄快三眼】若论他是英雄亦非是禽兽之类,他本是有须眉、有志气、雄赳赳、气昂昂是一个有志的能人。

\textbf{周处 哦,既然并非禽兽之类,为何被列为``三害''之内?}

\textbf{王浚}
【二黄快三眼】都只为他父丧早无人教训,因此上习下流做了歹人。

\textbf{周处 怎样为害?}

\textbf{王浚
【}二黄快三眼\textbf{】仗势力在宜兴习为光棍,欺贫贱、诈富贵苦害良民。}

\textbf{周处 哦,怎样地不法?}

\textbf{王浚
【}二黄快三眼\textbf{】有钱的还则可苦苦地曲奉,只可怜(或:实可怜)那无钱的人儿典了庄田、鬻了妻儿、也难少他的半分。}

\textbf{周处 哦,何不去县衙状告于他?}
\protect\hyperlink{fn296}{\textsuperscript{296}}

\textbf{王浚
【}二黄快三眼\textbf{】也有那被害的家与他来理论}\protect\hyperlink{fn297}{\textsuperscript{297}}\textbf{(或:议论),怎奈他膂力过人、力能扛鼎、有势有财,大小的衙门谁敢哼声。}

\textbf{周处 哇呀呀\ldots{}\ldots{}(周扔扇子)}

\textbf{王浚 【二黄摇板】都只为宜兴城出了恶棍,害得那众黎民难度光阴。}

\textbf{周处 老丈!}

\textbf{周处
【二黄摇板】听一言来怒气生,不由豪杰动无名。快快说出他的名和姓。}

\textbf{周处 我要剥了他的皮。【接二黄摇板】抽了他的筋。}

\textbf{王浚 【二黄摇板】我若是讲出了他人名姓,怕的是我老命要活不成。}

\textbf{周处 老丈!}

\textbf{周处 【二黄摇板】有俺在此何足论。}

\textbf{周处 任凭他铜金刚、铁罗汉,【接二黄摇板】难近某的身。}

\textbf{王浚} 壮士愿听?

\textbf{周处 愿听。}

\textbf{王浚} 愿闻?

\textbf{周处 愿闻。}

\textbf{王浚} 两厢看来。

\textbf{周处 讲来。}

\textbf{王浚 听了------}

\textbf{王浚} 【二黄摇板】他姓周名处

\textbf{周处 啊。}

\textbf{王浚} 【接二黄摇板】字子隐,

\textbf{王浚} 嘿嘿!

\textbf{王浚} 【二黄摇板】壮士闻言你惊不惊。

\textbf{周处 哎呀。}

\textbf{周处 【}二黄摇板\textbf{】好似霹雷当头震,周处做了不义人。}

\textbf{王浚} 【二黄摇板】问声壮士名和姓,

\textbf{周处 【}二黄摇板\textbf{】周处就是我的名。}

\textbf{王浚} 哎呀,饶命呐!

\textbf{周处
【}二黄摇板\textbf{】老丈在此等一等。\textless{}扫头\textgreater{}}

\textbf{(周处下)}

\textbf{王浚} 哈哈哈\ldots{}\ldots{}(笑介)

\textbf{王浚}
【二黄摇板】他好似酒醉方才醒,一言惊起懵懂人。但愿三害俱除尽(或:早除尽),

(王浚捡扇子)

\textbf{王浚} 【二黄摇板】黎民百姓享太平。

\textbf{审刺客 之 闵觉}\protect\hyperlink{fn298}{\textsuperscript{298}}

{[}第一场{]}

搀扶!

【西皮原板】自那日朝驾归精神不爽(或:身体不爽),因此上染重病倒卧在床。这几天我未曾(或:并未曾)朝见皇上,宫闱中出刺客搅乱朝纲。

老夫闵觉。晋王驾前为臣,官居刑部尚书之职。只因我主在粉宫楼前,拿住刺王杀驾之徒。万岁命六部审问。想老夫身染重病,不知哪部大臣代审。今日身体稍愈,不免去至朝房观看。

家院,

带定老爷朝服、官诰,朝房去者(或:带路朝房)。

【西皮原板】晋王爷坐山河人称有道,普天下众黎民快乐逍遥。到如今吾主爷国运衰了(或:看起来吾主爷国运不好),宫闱中出刺客搅乱当朝(或:搅乱九朝)。叫家院忙带路向前引道(或:叫家院你与爷向前引道;或:叫家院你与爷忙登御道),

【西皮摇板】看一看审问官是怎样开销。

{[}第二场{]}

(贺道庵\protect\hyperlink{fn299}{\textsuperscript{299}}、贾昱上\textless{}\textbf{小锣打上}\textgreater{},靠后站;四朝官搭轿上,靠台前站)

(朝官甲 (念)世事不由人机变,)

(朝官乙 (念)宰相专权贺道庵。)

(朝官丙 (念)列位不信抬头看,)

(朝官丁 (念)谁是忠来谁是奸。)

(四朝官 啊,大丞相。

(贺道庵 审得的?)

(贺道庵 问得的?)

(贺道庵 有僭了。)

(谢四\protect\hyperlink{fn300}{\textsuperscript{300}}上,闵觉暗上)

(皂隶 刺客当堂有刑。)

(贺道庵 松刑。)

(贺道庵 刺客。)

(谢四 有。)

(贺道庵 那日\ldots{}\ldots{}一一招来。)

(谢四
丞相容禀:小人名叫史龙。宫中史娘娘乃是小人的姑母。\ldots{}\ldots{}这就是我亲口所招。)

(贺道庵 那日\ldots{}\ldots{}一一招来。)

(贺道庵 来,叫他画押。)

(贺道庵 众位大人,请来画押。)

(四朝臣 闵大人未到,我等不敢画押。)

(贾昱 你们不肯画押,待咱家替你都画上罢!)

(且慢呐!)

怪道啊,怪道------

(贾昱 哎呦,我也先别画了。)

【西皮原板】站立在朝房下用目观看,

【西皮原板】看一看晋朝中文武两班。

【西皮原板】大丞相他那里颜色改变,审刺客这内中有他牵连。我岂肯把先人门墙辱玷,既到此我就该舍命向前。

罢!

【西皮摇板】怒冲冲将大丞相抓下公案,

(闵拉贺出桌,到台中间,贾站起)

【西皮摇板】我问你审刺客哪部官员。(或:审刺客何劳你宰相专权。)

你待怎讲?

哈哈。

呵呵!

(贺道庵 \ldots{}\ldots{}轻慢\ldots{}\ldots{}当朝宰相。)

啊------呵呵哈哈哈哈!(笑介)

(大丞相,)你道我轻慢你当朝首相(或:当朝宰相),想(或:有道是)这宰相之家,表率天下文武百官,为内外大小群臣之领袖,燮理阴阳,调和鼎鼐。兼放天下主考,开科取士,考取文章;翰林之家,选拔天下人材。晋朝之中,圣上设立六部乃是:吏、户、礼,兵、工、刑。这六部大堂,各有专司,各有专责。这吏部:掌管天下百官职爵,升迁调补,提选咨留,参革简放,考功稽勋,验封文选,征辟选举,论秀书升,才能参见当今万岁;户部:掌管内务府库金银,天下各省,丁漕赋税,各项钱粮;礼部:掌礼仪祭祀,祭坛祭庙,庵观寺院,陵寝庙宇。天地三界,十方万灵,春夏秋冬,祀祭典礼;兵部:掌管副参游都守,千把外委,五营四哨,兵丁将士,兵马钱粮;那工部:掌管三宫六院,五府六部,九卿科道,城池营垒,河工粮道,道路桥梁,天下大小工程。惟有我这(小小的)刑部:督理刑名,执掌生杀之大权,凡有人命攸关,私杀、擅杀、逼杀、格杀、谋杀、斗杀、故杀、误杀,各种案件,俱是我刑部所管。想这刺客,乃弑君之徒,朝廷要犯,理当我刑部所审,理当我刑部所问,何劳大丞相你来审问?

大丞相你要审问,却也不难,你我手挽手同上金殿。到晋王驾前,奏上一本:这晋朝之中,有几部大臣。圣上言道:六部。你就奏道:晋朝之中,不要六部,只要五部。圣上必然问将下来:哪部可裁,哪部可减?那时大丞相你须奏道:刑部可裁,刑部可减。圣上准了你的本章,裁了我这刑部,你方可审得,方能问得;圣上准不了你的本章,裁不了我这刑部,你便审不得,你也问不得。此乃刑部法堂之地,非是你大丞相议事之所,你要端端正正坐定了!(或:此乃王法所在,你与我站下些,你与我退后些,你与我坐下了!)

(闵推贺到边上,贺坐,贾上前到台口)

(贾昱 难道你还不念这同朝的情面吗?)

呀呸!

【西皮摇板】非是我不念在同朝情面(或:同朝脸面),审刺客何劳你宰相专权(或:理当我刑部掌权)。

【西皮摇板】对列公施一礼忙上公案(或:忙登公案),

(闵入大座。贾到闵身边)

(贾昱 你要还审不了?)

【西皮摇板】审不清问不明你启奏龙颜(或:审不公问不明面奏龙颜)。

(来,列位大人再次审问。)

(四朝官 恐刺客受刑不过,牵连不便。)

(请列位大人将台座升上一步。)

(四朝官 是。)

(四朝官站,再坐下)

带刺客!

(皂隶带谢四上)

(皂隶 当面上刑。)

(松刑。)

(贺道庵 史龙。)

(啊?大丞相,你怎么知道他叫史龙?)

(贺道庵 呃,这这这\ldots{}\ldots{}适才供状他叫史龙,故而叫他史龙。)

(哼,他乃弑君之徒,天子重犯,叫不得史龙。)

(贺道庵 要叫什么?)

(叫------要叫刺客。要叫刺客!)

(贾昱
咳,要叫刺客就叫刺客。刺客,忙将言语\ldots{}\ldots{}还要开脱于你。)

刺客,

(谢四 有。)

刺客!

刺客,你是何人将你带进宫去,藏在粉宫楼前,刺王杀驾,被御前侍卫拿住?你要从实招来,免受\textbf{五刑}之苦!

你待怎讲?

(谢四 \ldots{}\ldots{}亲口所招。(或:句句实言))

尔就该掌嘴!

【西皮原板】朝房中比不得荒野小县(或:晋朝中比不得旷野小县),本部堂岂容你信口胡言(或:信口诬陷)。

(刺客!)

(谢四 有。)

【西皮原板】你不过受他人些许情面,为什么把性命付与九泉。

教他招来。

唗!

【西皮散板】好一个小刺客真个大胆,四十板管叫你吐露实言。

有招无招?(或:来,问他有招无招。)

(皂隶 有招无招。)

(谢四 无有什么招的。)

(皂隶 无招。)

唗!。

(【西皮散板】骂一声小刺客真个大胆,不招供管教你鲜血不干(或:四十板管教你口吐真言)。)

(来,打!)

(皂隶把谢按倒在台中间)

(朝官甲 【西皮原板】好一个小刺客真个大胆,)

(皂隶 一十。)

(朝官乙 【西皮原板】责打他四十板鲜血不干。)

(皂隶 二十。)

(朝官丙 【西皮原板】贺道庵在一旁颜色改变,)

(皂隶 三十。)

(朝官丁 【西皮原板】倒教我审问官一体为难。)

(皂隶 四十打完。)

(谢四 哎呀。)

(谢四
【西皮散板】上堂来责打我四十大板,只打得小豪杰鲜血不干。自幼儿出娘胎未遭此难,你不敢把某家送上刀山。)

(住口!)

【西皮散板】任凭你就是那铜打铁炼,铜夹棒管教你尸不周全(或:尸骨不全)。(或:任尔是铜打铁炼,少时节(或:少时间)管教你尸骨不全。)

夹起来!(或:来,夹了起来)

(朝官甲 【西皮原板】铜夹板夹得他一声呐喊,)

(朝官乙 【西皮原板】吓得我战兢兢不敢胡言。)

(朝官丙 【西皮原板】他若是招实了你我不便,)

(朝官丁 【西皮原板】咱三人这性命全凭老天。)

有招无招? (或:来,问他有招无招。)

(皂隶 有招无招。)

(收。)

(皂隶 \ldots{}\ldots{}晕刑。)

(松刑。)

(谢滚堂,摸腿,爬起)

(谢四 【西皮导板】铜夹棒夹得我皮开肉烂,\ldots{}\ldots{})

(谢四 嘿!好一个刑部大人\ldots{}\ldots{}我愿招。)

(皂隶 他有招。)

教他画供。

(皂隶 招上来画供。)

(谢看贺,贺示意谢不招)

(谢四 哎呀,招不得呀招不得。闵觉,方才俺招的俱是实言,还教我招得什么?)

哎呀!

【西皮散板】小刺客不招承浑身是汗(或:小刺客不招供难以判断),

(闵觉吐血介)

【西皮散板】不由人一阵阵心血上泛(或:不由我一阵阵血往上泛)。

【西皮散板】到如今好教我难以审断,用尽了百般刑也是枉然。

【西皮散板】无奈何下位去将他来劝(或:我只得下位去将他来赚),

(闵觉出座,坐台口椅)

【西皮散板】慢慢地相劝他好吐实言(或:好言诓哄他好漏实言)。

刺客,

(谢四 有哇!诶呀\ldots{}\ldots{})

刺客!我看你乃是一条英雄好汉,并非真心刺王杀驾,不过是受了哪部大人之托,不想被御前侍卫拿住,圣上命六部审问。方才你胡说史娘娘是你的姑母,你是史娘娘的内侄。想那史娘娘,自从进宫以来,并无三亲六眷,哪有你这样的内侄?你若信口胡言,害得史娘娘身遭斩首,你岂不怕天下人咒骂于你?你若招出真情,请列位大人,作一本首,老夫作一本尾,保你在朝,做上大官,你是何等不喜,哪些不乐?你若执意不招,一刀将你斩首,想你乃天下英雄,岂能做这刀头之鬼?

(刺客!)

(谢四 有!)

有道是:(念)要学天下奇男子(或:为人学得梁鸿志),方显男儿大丈夫!(你要再思啊再想。哈哈,哈哈,啊,哈哈哈\ldots{}\ldots{}(笑介))

【西皮原板】你本是天下的英雄好汉,

(谢四 诶,本来是英雄好汉。)

【西皮原板】为什么当刺客下贱不堪。招实供史娘娘感你的恩德匪浅,晋王爷他必然封你在帘外为官(或:圣天子龙心喜必封你帘外为官)。

【西皮原板】如不然我和你把帖来换,

(谢四 我不敢呐。)

【西皮原板】我为兄你为弟同列朝班。

(谢四 越发地不敢。)

【西皮原板】你好比错行路大大地弯转,这件事何须我替你为难(或:替你忧烦)。

(来,劝他招了口供,尔等皆有赏。)

(闵觉归大座)

(皂隶 朋友,我家大人下得位来百般相劝与你,叫你招了实供,邀请列位大人作一
本头,我家大人作一本尾,保你在晋朝之中,做一员大官。是何等儿不喜,
哪些儿不乐?朋友,你想做官的好,是挨刀的好呢。)

(谢四 待我思忖思忖。)

(谢四 好一个闵大人,好一位闵青天!下得位来百般相劝与俺,教俺招了实供,
邀请列位大人作一本头,闵大人作一本尾,保俺在这晋朝之中做一员大官。
是何等儿不喜,是哪些儿不乐?我为报他人之恩,断送俺的性命。哎,看在做官的份上,我有招,我有招,招了上来。)

教他画供。

(谢四
诶呀!招不得呀招不得!俺若招了实情,俺的恩人岂不是一刀两断,哪里还有什么官做?咳,我想世上恩只将恩报,哪有恩将仇报之理!
呔,闵大人!你为史娘娘乃是一忠,俺为俺主乃是一义,你做你的忠臣,俺做俺的义士!你教我招的什么?!)

将他吊了起来。(或:与我吊了起来。)

(问他有招无招。)

(皂隶 有招无招。)

(谢四 \ldots{}\ldots{}无有招对。)

敲牙一颗。

(谢四 呜\ldots{}\ldots{})

(问他有招无招。)

(皂隶 有招无招。)

(谢四 \ldots{}\ldots{}无有招对。)

再敲!

(谢四 呜\ldots{}\ldots{})

(问他有招无招。)

(谢四 无有什么招的。)

(贾昱 割他的舌头!)

且慢呐!

哎呀!

【西皮散板】听说是割舌根心惊肉颤,吓得我魂灵儿飞上九天。下位来我这里用目观看,(或:听说是割舌尖神魂散乱,吓得我一阵阵胆战心寒。无奈何下位去刺客来看,)

【西皮散板】尊一声众大人细听我言(或:见他口内吐鲜血我心不安)。

列位大人,下官有意,将他带回衙去,审问明白,再奏龙颜。不知列位大人,意下如何?(或:列位大人,且慢启奏龙颜,待下官将刺客带回衙去,审明回奏。)

(四朝官 但凭大人。)

带回衙去。(或:搭了下去。)

(皂隶搀谢下)

告辞了。

【西皮摇板】辞别了众大人忙回衙转(或:辞列公施一礼抽身回转) ,

(闵觉出来,四朝官下,贾昱(或:贺道庵)过去)

(贾昱 拿来我看。(或:贺道庵 拿来我看。))

(呀呸!)

【西皮摇板】这件事你二人定有牵连 (早知道这内中有你牵连) 。

哈哈,哈哈,啊------

呜\ldots{}\ldots{}(门子搀闵下)

(贾昱 请呐。)

(贺道庵 请呐。)

\newpage
\hypertarget{ux6851ux56edux5bc4ux5b50-ux4e4b-ux9093ux4f2fux9053}{%
\subsection{桑园寄子 之
邓伯道}\label{ux6851ux56edux5bc4ux5b50-ux4e4b-ux9093ux4f2fux9053}}

{[}第一场{]}

{[}引子{]}家道兴隆,训子嗣,早成功名。

(念)人生在世几度秋,好似杨花水上浮。有朝一日狂风落,大限来时一笔勾。

老汉邓伯道。兄弟伯俭,不幸中年丧命,今当他周年之期,我不免带领两个孩儿,去至坟前一祭。

家院,

有请二位少爷。

罢了。

你二人坐下\protect\hyperlink{fn301}{\textsuperscript{301}}。

今当你叔父周年之期,为父备定祭礼,带领你们去往坟前一祭。

祭礼走上。

{[}第二场{]}

唉,难得见的兄弟呀!呃,呃,呃\ldots{}\ldots{}(哭介)

【二黄慢板】叹兄弟遭不幸一旦丧命,丢下了年幼儿好不伤情。眼望着孤坟台珠泪难忍,见坟台不见人刀割我心。

\textless{}\textbf{叫头}\textgreater{}伯俭!兄弟!

今当你周年之期,愚兄带领两个孩儿,前来祭奠于你。来来来,受愚兄一陌纸钱。也不枉你我手足一场\protect\hyperlink{fn302}{\textsuperscript{302}}。

\textless{}\textbf{三叫头}\textgreater{}兄弟!伯俭!唉,兄弟呀!

【二黄导板】见坟台不由人珠泪滚滚,

\textless{}\textbf{三叫头}\textgreater{}伯俭!兄弟! 呜哙呀难得见的兄弟呀!

【回龙】叫一声同胞弟细听兄云。

【二黄快三眼】曾记得弟在世何等的侥幸,兄与弟同商议家道隆兴。料不想身得病一旦丧命,兄弟丧------命,兄弟呀!此黄土埋却了无价宝珍。\protect\hyperlink{fn303}{\textsuperscript{303}}

哦,儿要儿的叔父么?

这里面就是儿的叔父哇,

你要叫啊。

哦,儿要儿的爹爹么?

这里面就是儿的爹爹,你去叫他起来,同我们回去呀。呃,呃\ldots{}\ldots{}(哭介)

\textless{}\textbf{叫头}\textgreater{}伯俭!

你可曾听见呐,这两个孩儿,一个问我要他的叔父,一个问我要他的爹爹,你可曾听见呐!你聋了?你哑了?你,你,你睡死了哇\ldots{}\ldots{}(哭介)

【二黄散板】这一个要叔父我的心酸难忍,\protect\hyperlink{fn304}{\textsuperscript{304}}

【二黄散板】那一个要天伦刀割我心。

【二黄散板】哭一声同胞弟慢慢相等。

何事惊慌?

唉------呀,这才是福无双至,祸不单行!

儿呀,犹恐你母亲在家悬望,我们快快地回去呀!

{[}第三场{]}

回来了。

弟妹,大事不好了!

今有黑水国石勒造反,逢州夺州,遇县抢县,堪堪杀到我庄来了!

带领两个孩儿后面收拾收拾。

家丁们,看衣改换!

家丁们请上,受我全家------唉,一拜!(哭介)

【二黄散板】一家人跪草堂珠泪滚滚,叫一声众家丁细听分明:但愿得贼兵退此地安靖\protect\hyperlink{fn305}{\textsuperscript{305}},但愿得贼兵退也好回程。

【二黄散板】诉不尽衷肠苦急忙投奔。

{[}第四场{]}

【二黄慢板】走青山望白云------

【二黄慢板】山又高水又深难以忍耐,

【二黄慢板】手攀藤带娇儿忙登山界,忙登山------\protect\hyperlink{fn306}{\textsuperscript{306}}

【二黄慢板】眼望着白茫茫但不知何方地界,

【二黄散板】眼观得旌旗飘把我吓坏,

【二黄散板】又只见众贼兵蜂拥而来。

【二黄散板】手挽手带娇儿忙下山界。

{[}第五场{]}

吓煞我也,吓煞我也!

嗯。

儿啊,你的婶母呢?

儿啊,你的母亲呢?

怎么讲?

哎呀!

【二黄散板】听说是贼兵抢三魂不在,

\textless{}\textbf{三叫头}\textgreater{}弟妹!金氏!唉,弟妹呀!

【二黄散板】眼见得一家人四散分开。

儿啊,量他们走得不远,我们速速地赶上呃。

儿啊,你这是怎么样了?

这个\ldots{}\ldots{}

那旁有一土台,儿且站了上去,待为父的背儿一程就是。

不妨,儿只管的上去。

诶,儿呀,待我背了你哥哥,然后再来背你呀。

哦,来了。

哎!

【二黄散板】前世里欠下了冤孽魔债,老的老、小的小,好不伤怀。

哎呀!

我儿醒来。

儿啊,不要啼哭,你也站了上去,待为伯的也背儿一程就是。

为伯么\ldots{}\ldots{}

呃,不老,不老!

你,你,你只管的上去呀,呃\ldots{}\ldots{}(哭介)

诶,你大,他小,你还是哥哥呢。

哦,来了!

【二黄散板】小娇儿年纪小聪明可爱,可怜他无父的儿珠泪满腮。

你怎么又不走了?!

哎呀!

且住,看这两个小冤家,挨挨蹭蹭,行走不动。倘若贼兵到此,将我儿杀死,唉,倒也罢了哇;若是将我侄儿这一刀杀死------教我怎样对得过我那亡故的兄弟。这,这,这\ldots{}\ldots{}(搓手介)

有了!

看那旁有一桑园,我不免将我儿绑在树上,与他留下血书一道,然后背着我侄儿逃走,纵死九泉,也能对得过我那亡故的兄弟。我就是这个主意呀,我、我就是这个主意呀,呃\ldots{}\ldots{}(哭介)

你们站了起来!

行走半日,可曾饥饿?

抬头观看!

那旁有一桑园,桑葚长得茂盛,你们哪个上去,摘将下来,也好充饥呀。

呃,慢来慢来。

你小哇,看摔下来呀。

着哇,原要你上去呀,呃呃\ldots{}\ldots{}(哭介)

儿啊,脸朝外站。

儿啊!非是为父的心狠呐。儿来看------只因你兄弟年小啊,行走不动。倘若贼兵到此,将我儿杀死,为父的倒也落得个干净;若是将你兄弟这一刀杀死,教为父的怎样对得过你那亡故的叔父病床之上托孤之情?万般无奈,将儿绑在树上。与儿留下血书一道,然后背着你兄弟逃走。但愿贼兵不打此经过,有那仁人君子将儿救下。儿啊!你就有了活命了。你我父子日后还有相逢之日;倘若贼兵到此将我儿这一刀------杀死,儿啊,这也是儿遭劫在数,大数难逃。你我父子今生今世再若相逢,只怕是万万不得能够哇,呃呃\ldots{}\ldots{}(哭介)

\textless{}\textbf{三叫头}\textgreater{}邓方!我儿!唉,儿啊!

【二黄散板】此时间顾不得父子恩爱,眼见得亲骨肉两下分开。

【二黄散板】急忙忙扯下了这衣襟一块呀,

【二黄散板】咬指尖腹内痛珠泪满腮。

【二黄散板】我家住在太原府文水县界,我的名叫邓伯道逃难此来。舍亲生救侄儿流传后代,也免得旁人骂我年老无才。

【二黄散板】将儿的年庚月血书上载,仁君子你、你、你\ldots{}\ldots{}救了去,我佛如来。

哎呀!

【二黄散板】血迹干书不尽恩深似海呀。

儿啊,走哇。

为何?

哎呀!

儿啊,你的母亲来了!

呃,在这里呃------

\textless{}\textbf{三叫头}\textgreater{}弟妹!金氏!儿啊\ldots{}\ldots{}(哭介)

罢!

\item
  \leavevmode\hypertarget{fn136}{}%
  ``奉行''是``遵照执行''之意;``头戴乌纱奉行先''陈宫身为县令,为一县表率;\protect\hyperlink{fnref136}{↩}
\item
  \leavevmode\hypertarget{fn137}{}%
  ``开可''是``许可''的意思;\protect\hyperlink{fnref137}{↩}
\item
  \leavevmode\hypertarget{fn138}{}%
  ``家邑''本意为``采地'',这里表示陈宫管辖之中牟县;

  ``循吏''见于《史记》的《循吏列传》,一般指实施、推行善政、口碑声好的州、县级地方官;\protect\hyperlink{fnref138}{↩}
\item
  \leavevmode\hypertarget{fn139}{}%
  ``水地天''乃``尧天、舜地、禹水''之意。

  吴小如先生学的定场诗作``\textbf{头戴乌纱奉孝先},\textbf{慈祥恺悌万民欢}。\textbf{嘉言犹如湖中地},\textbf{得配汪洋水底天}。''后两句源自《后汉书·黄宪传》:郭林宗评黄宪(叔度)``汪汪若千顷之陂,澄之不清,淆之不浊,不可量也。''\protect\hyperlink{fnref139}{↩}
\item
  \leavevmode\hypertarget{fn140}{}%
  段公平君建议作``王升'',此处从``中国京剧戏考''网站《戏考》第一册本。\protect\hyperlink{fnref140}{↩}
\item
  \leavevmode\hypertarget{fn141}{}%
  夏行涛君建议作``定妥''。\protect\hyperlink{fnref141}{↩}
\item
  \leavevmode\hypertarget{fn142}{}%
  陈超老师介绍:``宿店''一场,陈宫坐小边,【二黄慢板】也坐小边虎头椅。\protect\hyperlink{fnref142}{↩}
\item
  \leavevmode\hypertarget{fn143}{}%
  陈超老师介绍:起二更时陈宫有个身段:搭左腿,左手水袖搭在左膝上,右手扶座。\protect\hyperlink{fnref143}{↩}
\item
  \leavevmode\hypertarget{fn144}{}%
  ``\textbf{陶恭祖望救兵营门立等,备哪有闲心肠来饮杯巡。}''也可以唱\textbf{【西皮二六】,刘曾复先生说戏时亦作了示范。}\protect\hyperlink{fnref144}{↩}
\item
  \leavevmode\hypertarget{fn145}{}%
  夏行涛君建议作``迎门立等''。\protect\hyperlink{fnref145}{↩}
\item
  \leavevmode\hypertarget{fn146}{}%
  夏行涛君建议作``大功易成''。\protect\hyperlink{fnref146}{↩}
\item
  \leavevmode\hypertarget{fn147}{}%
  刘曾复先生录音中念的是``占得天子'',但按文意``天时''更通顺。\protect\hyperlink{fnref147}{↩}
\item
  \leavevmode\hypertarget{fn148}{}%
  《论语·子罕》:子贡日:``有美玉于斯,韫匮而藏诸?求善贾而沽诸?''子日:``沽之哉!沽之哉!我待贾者也。''\protect\hyperlink{fnref148}{↩}
\item
  \leavevmode\hypertarget{fn149}{}%
  段公平君指出,``作幕''疑作``作掾'',因形似``作录''以致讹作。掾,后为副官佐或官署属员的通称。《全唐文》杜牧有``作掾京兆'';《元好问集》有``先夫人每以作掾为讳''。\protect\hyperlink{fnref149}{↩}
\item
  \leavevmode\hypertarget{fn150}{}%
  ``中国京剧戏考''网站《戏考》第一册本作``把贼扫''。\protect\hyperlink{fnref150}{↩}
\item
  \leavevmode\hypertarget{fn151}{}%
  一般俗作``褴衫''。李楠君按:``蓝衫''是职位低下的官吏的职服,考诸剧情,祢衡先着蓝衫觐见曹操,继而换破衣褴衫,最后赤身裸体,当是。\protect\hyperlink{fnref151}{↩}
\item
  \leavevmode\hypertarget{fn152}{}%
  吴焕老师整理的剧本记作``将身来在东廊道'',并注:``刘老云,老本旧词此句唱`将身来在西廊道',下面所接锣鼓为\textless{}\textbf{快长锤}\textgreater{},而并非\textless{}\textbf{双楗子}\textgreater{}''。\protect\hyperlink{fnref152}{↩}
\item
  \leavevmode\hypertarget{fn153}{}%
  旧谓子女的身体为父母所生,因称子女的身体为父母的``遗躰''。《大戴礼记·曾子大孝》:``身者,亲之遗躰也。''一本作``遗体''。\protect\hyperlink{fnref153}{↩}
\item
  \leavevmode\hypertarget{fn154}{}%
  此句李楠君从刘曾复先生学作``有朝大展经纶手''。\protect\hyperlink{fnref154}{↩}
\item
  \leavevmode\hypertarget{fn155}{}%
  祖考,泛指父祖之辈。\protect\hyperlink{fnref155}{↩}
\item
  \leavevmode\hypertarget{fn156}{}%
  ``祧''为远祖之庙。宗祧即宗庙,引申为祖业。\protect\hyperlink{fnref156}{↩}
\item
  \leavevmode\hypertarget{fn157}{}%
  ``志平生''夏行涛君建议作``助平升'',此处从《京剧汇编》第八十五集
  马连良藏本;录音中刘曾复先生念``天子''应作``夫子'',据夏行涛君告,``英雄几见称夫子,豪杰如斯乃圣人''是清代中叶理学家夏力恕为湖北孝感关帝庙作的对联。\protect\hyperlink{fnref157}{↩}
\item
  \leavevmode\hypertarget{fn158}{}%
  据樊百乐君告,刘曾复先生曾言,前辈艺人沿袭``尊崇关帝''的旧俗,台上往往将``青龙刀''或``青龙偃月''的``龙''念``铜(tóng)''音,也可视为一种``避讳''。刘先生学戏时``青龙刀''念法即此路数。\protect\hyperlink{fnref158}{↩}
\item
  \leavevmode\hypertarget{fn159}{}%
  刘曾复先生录音中似作``天日行'',参考《京剧汇编》第八十五集
  马连良藏本,此处作``实望汉室天日倾'',故``誓挽汉室天日倾''文意更确。\protect\hyperlink{fnref159}{↩}
\item
  \leavevmode\hypertarget{fn160}{}%
  刘曾复先生钞本注:``汪桂芬、王凤卿、余叔岩、贾洪林
  派,与富(连成)社不同''。

  (戏中所有鲁肃下场为王凤卿授)

  剧本中有关人物、场次和调度由段公平君协助整理。\protect\hyperlink{fnref160}{↩}
\item
  \leavevmode\hypertarget{fn161}{}%
  刘曾复先生钞本注:``此句可不念''。\protect\hyperlink{fnref161}{↩}
\item
  \leavevmode\hypertarget{fn162}{}%
  陈超老师介绍:这是鲁肃的第二个上场。\protect\hyperlink{fnref162}{↩}
\item
  \leavevmode\hypertarget{fn163}{}%
  此处不念《三国演义》原文中``伏路把关饶子敬,临江水战有周郎。''两句。\protect\hyperlink{fnref163}{↩}
\item
  \leavevmode\hypertarget{fn164}{}%
  刘曾复先生在为樊百乐君说戏时详细介绍了藏书的做工和细节。\protect\hyperlink{fnref164}{↩}
\item
  \leavevmode\hypertarget{fn165}{}%
  这个对儿是``真假难凭信,好歹问知音。''\protect\hyperlink{fnref165}{↩}
\item
  \leavevmode\hypertarget{fn166}{}%
  刘曾复先生钞本此处记为``蝼蚁尚生'',系脱漏,据录音补正。\protect\hyperlink{fnref166}{↩}
\item
  \leavevmode\hypertarget{fn167}{}%
  陈超老师介绍:此处鲁肃不端酒杯,更没有把酒泼在自己脸上的表演。\protect\hyperlink{fnref167}{↩}
\item
  \leavevmode\hypertarget{fn168}{}%
  陈超老师介绍:贾洪林说谭鑫培不允许(诸葛亮饮酒)。\protect\hyperlink{fnref168}{↩}
\item
  \leavevmode\hypertarget{fn169}{}%
  段公平君注``雄虎''亦作``熊虎''。\protect\hyperlink{fnref169}{↩}
\item
  \leavevmode\hypertarget{fn170}{}%
  夏行涛君建议作``起首''。姜骏按:``起首''为开始、起先之意;``起手''有动手、下手之意,亦有开始之意。\protect\hyperlink{fnref170}{↩}
\item
  \leavevmode\hypertarget{fn171}{}%
  段公平君建议作``荣辱事''。\protect\hyperlink{fnref171}{↩}
\item
  \leavevmode\hypertarget{fn172}{}%
  \textbf{原来老路子是单起霸,与《群英会》黄盖、甘宁同。}\protect\hyperlink{fnref172}{↩}
\item
  \leavevmode\hypertarget{fn173}{}%
  \textbf{金少山戴大镫,后改倒缨盔。}\protect\hyperlink{fnref173}{↩}
\item
  \leavevmode\hypertarget{fn174}{}%
  双起霸黄忠、魏延分着念黄忠单起霸的词。如单起霸,魏延的念为``(念)威风凛凛杀气飘,万马军中逞英豪。丹心一片扶社稷,深谢皇恩保汉朝。''\protect\hyperlink{fnref174}{↩}
\item
  \leavevmode\hypertarget{fn175}{}%
  刘曾复先生多次强调,黄忠、魏延对韩玄的称呼,应该称呼``都督''或``太守'',不称呼``元帅''。\protect\hyperlink{fnref175}{↩}
\item
  \leavevmode\hypertarget{fn176}{}%
  陈超老师介绍:这段\textbf{【西皮二六】设计得}很别致。七个字一句\textbf{【二六】},按十个字一句\textbf{【二六】}唱。\protect\hyperlink{fnref176}{↩}
\item
  \leavevmode\hypertarget{fn177}{}%
  此为北京谭派、汪派打法。上海有``四门斗''、\textless{}\textbf{柳青娘\textgreater{}}唢呐牌子打法。\protect\hyperlink{fnref177}{↩}
\item
  \leavevmode\hypertarget{fn178}{}%
  \textbf{陈超老师介绍:黄忠落马,念时,关羽始终举刀纹丝不动。}\protect\hyperlink{fnref178}{↩}
\item
  \leavevmode\hypertarget{fn179}{}%
  \textbf{陈超老师介绍:}关羽念到``换马再战'',右手一拄刀,左手慢捋髯,很威严。\protect\hyperlink{fnref179}{↩}
\item
  \leavevmode\hypertarget{fn180}{}%
  夏行涛君注:``旌旗起''当作``旌节旗(或:旌捷旗)''------《金瓶梅》第十二回作``眼望旌节旗'';《琵琶记》作``眼望旌捷旗''。\protect\hyperlink{fnref180}{↩}
\item
  \leavevmode\hypertarget{fn181}{}%
  此处按老路子不转【西皮快板】。\protect\hyperlink{fnref181}{↩}
\item
  \leavevmode\hypertarget{fn182}{}%
  此处及\textbf{以上几句一般没有唱,谭派、余派有唱。}\protect\hyperlink{fnref182}{↩}
\item
  \leavevmode\hypertarget{fn183}{}%
  \textbf{表示悬挂韩玄人头示众。}\protect\hyperlink{fnref183}{↩}
\item
  \leavevmode\hypertarget{fn184}{}%
  ``性情有''也有唱``性情拗''的。\protect\hyperlink{fnref184}{↩}
\item
  \leavevmode\hypertarget{fn185}{}%
  此处刘曾复先生录音疑似有缺失,\textbf{根据《马连良演出剧本选集》添加。}\protect\hyperlink{fnref185}{↩}
\item
  \leavevmode\hypertarget{fn186}{}%
  段公平君注:这两句刘曾复先生另有作:\textbf{鲁子敬再不能旁观袖手,望都督三思行另定良谋。}\protect\hyperlink{fnref186}{↩}
\item
  \leavevmode\hypertarget{fn187}{}%
  段公平君据2005年5月29日刘曾复先生与段公平、樊百乐谈话录音整理(非正式说戏录音)\protect\hyperlink{fnref187}{↩}
\item
  \leavevmode\hypertarget{fn188}{}%
  \textbf{末句刘曾复先生未能忆起。}

  \textbf{陈超老师介绍:他随刘曾复先生学的词句是``曹贼兴动人和马,进犯荆州把孤拿。四弟之言非虚假,瞒哄郡主及早还家。''尤其最后一句是八个字,挺特别。}\protect\hyperlink{fnref188}{↩}
\item
  \leavevmode\hypertarget{fn189}{}%
  刘曾复先生说戏时说明:此处\textbf{后来多不唱,刘备直接\textless{}水底鱼\textgreater{}上。}\protect\hyperlink{fnref189}{↩}
\item
  \leavevmode\hypertarget{fn190}{}%
  ``衿''的本意为正装,同``襟'';也可指系衣裳的带子。\protect\hyperlink{fnref190}{↩}
\item
  \leavevmode\hypertarget{fn191}{}%
  《京剧汇编》第十三集
  马连良藏本作``花瓣落''。\protect\hyperlink{fnref191}{↩}
\item
  \leavevmode\hypertarget{fn192}{}%
  《京剧汇编》第十三集
  马连良藏本作``以势力''。\protect\hyperlink{fnref192}{↩}
\item
  \leavevmode\hypertarget{fn193}{}%
  ``公厅''是官衙的意思。\protect\hyperlink{fnref193}{↩}
\item
  \leavevmode\hypertarget{fn194}{}%
  《京剧汇编》第十三集
  马连良藏本作``移奔''。\protect\hyperlink{fnref194}{↩}
\item
  \leavevmode\hypertarget{fn195}{}%
  该戏的相关场次的身段表演参阅李舒先生遗作《涉艺所得》所录的《刘曾复谈话、书信摘录》部分。\protect\hyperlink{fnref195}{↩}
\item
  \leavevmode\hypertarget{fn196}{}%
  《京剧汇编》第一百零七集作``一来''、``二来''。\protect\hyperlink{fnref196}{↩}
\item
  \leavevmode\hypertarget{fn197}{}%
  段公平君建议作``功得胜''。\protect\hyperlink{fnref197}{↩}
\item
  \leavevmode\hypertarget{fn198}{}%
  据樊百乐君告知,刘曾复先生强调,``牙车''是``牙床''的意思。\protect\hyperlink{fnref198}{↩}
\item
  \leavevmode\hypertarget{fn199}{}%
  陈超老师介绍,他随刘曾复先生学的此句是``抖擞精神上山道''。\protect\hyperlink{fnref199}{↩}
\item
  \leavevmode\hypertarget{fn200}{}%
  据《三国志·蜀书》载:``黄忠、赵云强挚壮猛,并作爪牙,其灌、滕之徒欤?''

  \textbf{陈超老师按}:

  《伐东吴》如果带``小桃园'',则是黄忠与吴班同上念此对儿。

  \textbf{陈超老师介绍带``小桃园''演法如下}:

  刘备封黄忠``以为随军副帅'',封吴班``前站先行'',二人领旨下;

  刘备再传令``哪位将军愿领副先锋?''关兴、张苞争功,比武、折箭后,再接黄忠``忆昔当年''。

  一般演出都不带``小桃园``。\protect\hyperlink{fnref200}{↩}
\item
  \leavevmode\hypertarget{fn201}{}%
  此处据《京剧汇编》第一百零一集
  马连良藏本增补。\protect\hyperlink{fnref201}{↩}
\item
  \leavevmode\hypertarget{fn202}{}%
  此处据《京剧汇编》第一百零一集
  马连良藏本增补。\protect\hyperlink{fnref202}{↩}
\item
  \leavevmode\hypertarget{fn203}{}%
  此处据《京剧汇编》第一百零一集
  马连良藏本增补。\protect\hyperlink{fnref203}{↩}
\item
  \leavevmode\hypertarget{fn204}{}%
  此处据《京剧汇编》第一百零一集
  马连良藏本作``吴班有言来禀告,破敌须防战马劳。老将军威风谁不晓,何妨饶他这一遭。''\protect\hyperlink{fnref204}{↩}
\item
  \leavevmode\hypertarget{fn205}{}%
  此处刘曾复先生只念``暂回师'',据上下文增补。\protect\hyperlink{fnref205}{↩}
\item
  \leavevmode\hypertarget{fn206}{}%
  此处刘曾复先生唱的是``\textbf{\ldots{}\ldots{}}斩将论英豪'',似欠通,此处从《京剧汇编》第一百零一集
  马连良藏本。\protect\hyperlink{fnref206}{↩}
\item
  \leavevmode\hypertarget{fn207}{}%
  此处据《京剧汇编》第一百零一集
  马连良藏本增补。\protect\hyperlink{fnref207}{↩}
\item
  \leavevmode\hypertarget{fn208}{}%
  《京剧汇编》第一百零一集
  马连良藏本此处作``旌旗飞龙影,干戈耀日明''。\protect\hyperlink{fnref208}{↩}
\item
  \leavevmode\hypertarget{fn209}{}%
  此处据《京剧汇编》第一百零一集
  马连良藏本增补。\protect\hyperlink{fnref209}{↩}
\item
  \leavevmode\hypertarget{fn210}{}%
  此句刘曾复先生录音不清楚,据文意添加。存疑。\protect\hyperlink{fnref210}{↩}
\item
  \leavevmode\hypertarget{fn211}{}%
  陈超老师介绍,他跟刘曾复先生学的此句``谋虑远''唱,``平吴不及定中原''一句【散板】。\protect\hyperlink{fnref211}{↩}
\item
  \leavevmode\hypertarget{fn212}{}%
  此处至本剧结尾,刘曾复先生只是大致示范,能听清个别词句。因此剧本中词句据《京剧汇编》第一百零一集
  马连良藏本增补。\protect\hyperlink{fnref212}{↩}
\item
  \leavevmode\hypertarget{fn213}{}%
  陈超老师介绍:《连营寨》前半出西皮,后半出昆腔,是这出戏的特点,陆逊先唱\textless{}\textbf{粉蝶儿}\textgreater{}、\textless{}\textbf{醉太平}\textgreater{},然后再唱九支曲子。\textbf{陆逊唱北曲},\textbf{其他人唱南曲}。\protect\hyperlink{fnref213}{↩}
\item
  \leavevmode\hypertarget{fn214}{}%
  段公平君建议作``旗偃戈收''。\protect\hyperlink{fnref214}{↩}
\item
  \leavevmode\hypertarget{fn215}{}%
  夏行涛君建议作``乃一''\protect\hyperlink{fnref215}{↩}
\item
  \leavevmode\hypertarget{fn216}{}%
  夏行涛君建议作``亘古流标''。\protect\hyperlink{fnref216}{↩}
\item
  \leavevmode\hypertarget{fn217}{}%
  据《三国志·吴书》载,陆逊是九江都尉陆骏之子。\protect\hyperlink{fnref217}{↩}
\item
  \leavevmode\hypertarget{fn218}{}%
  ``四至八道''是旧时标志土地界域的用语。表示四面八方所到之处及通往的道路。\protect\hyperlink{fnref218}{↩}
\item
  \leavevmode\hypertarget{fn219}{}%
  据李元皓君告知,``启祚'',是发祥、开创帝业之意。``中国京剧戏考''网站《戏考》第六册本作``起坐'',似非。\protect\hyperlink{fnref219}{↩}
\item
  \leavevmode\hypertarget{fn220}{}%
  段公平君建议作``路当阳''。\protect\hyperlink{fnref220}{↩}
\item
  \leavevmode\hypertarget{fn221}{}%
  根据刘曾复先生钞本整理,刘曾复先生说戏``平五路''是本剧的``观鱼遣邓''部分。刘曾复先生的钞本与《清车王府藏曲本(全印本)》\textsuperscript{{[}14{]}}第二册所收录的``安五路(总讲)''基本一致,但``安五路(总讲)''没有最后的``邓芝扑油鼎''部分。\protect\hyperlink{fnref221}{↩}
\item
  \leavevmode\hypertarget{fn222}{}%
  刘曾复先生钞本注``小生戴髯'',即曹丕归小生行应工。\protect\hyperlink{fnref222}{↩}
\item
  \leavevmode\hypertarget{fn223}{}%
  刘曾复先生钞本作``令起起羌兵十万'',似欠通;此处从``安五路(总讲)''。\protect\hyperlink{fnref223}{↩}
\item
  \leavevmode\hypertarget{fn224}{}%
  刘曾复先生钞本作``合好''。\protect\hyperlink{fnref224}{↩}
\item
  \leavevmode\hypertarget{fn225}{}%
  进位是进升爵位,封号的意思。\protect\hyperlink{fnref225}{↩}
\item
  \leavevmode\hypertarget{fn226}{}%
  刘曾复先生钞本作``还班'',此处从``安五路(总讲)''原文。\protect\hyperlink{fnref226}{↩}
\item
  \leavevmode\hypertarget{fn227}{}%
  刘曾复先生钞本与``安五路(总讲)''均作``差官倒退跳赶\textless{}\textbf{度柳翠}\textgreater{}'',经何毅老师指教,\textless{}\textbf{度柳翠}\textgreater{}是牌子名,``干''表示是``干牌子''。\protect\hyperlink{fnref227}{↩}
\item
  \leavevmode\hypertarget{fn228}{}%
  刘曾复先生钞本作``自因''。\protect\hyperlink{fnref228}{↩}
\item
  \leavevmode\hypertarget{fn229}{}%
  刘曾复先生钞本以下``报子''均作``探子'',应可通。\protect\hyperlink{fnref229}{↩}
\item
  \leavevmode\hypertarget{fn230}{}%
  刘曾复先生钞本和``安五路(总讲)''所有``成都''均作``城都''。\protect\hyperlink{fnref230}{↩}
\item
  \leavevmode\hypertarget{fn231}{}%
  刘曾复先生钞本``好生''二字不确认,疑作``如此''或``着实'',此处从``安五路(总讲)''。\protect\hyperlink{fnref231}{↩}
\item
  \leavevmode\hypertarget{fn232}{}%
  刘曾复先生钞本作``合好''。\protect\hyperlink{fnref232}{↩}
\item
  \leavevmode\hypertarget{fn233}{}%
  刘曾复先生钞本注``带`扑油鼎'可考虑不要此场。''\protect\hyperlink{fnref233}{↩}
\item
  \leavevmode\hypertarget{fn234}{}%
  刘曾复先生钞本作``万岁圣宽怀,暂且放心。'',此处从``安五路(总讲)''。\protect\hyperlink{fnref234}{↩}
\item
  \leavevmode\hypertarget{fn235}{}%
  刘曾复先生钞本作``诸事毕''``安五路(总讲)''此处补``已'',此处从之。\protect\hyperlink{fnref235}{↩}
\item
  \leavevmode\hypertarget{fn236}{}%
  这一场是``安五路(总讲)''没有的。\protect\hyperlink{fnref236}{↩}
\item
  \leavevmode\hypertarget{fn237}{}%
  ``霄晓勿遑''即不分昼夜之意。\protect\hyperlink{fnref237}{↩}
\item
  \leavevmode\hypertarget{fn238}{}%
  刘曾复先生钞本作``小官不知故''。\protect\hyperlink{fnref238}{↩}
\item
  \leavevmode\hypertarget{fn239}{}%
  刘曾复先生钞本作``这却何地'',文意欠通。\protect\hyperlink{fnref239}{↩}
\item
  \leavevmode\hypertarget{fn240}{}%
  段公平君注:``多管'',即多半,大概之意。多见于元明小说、话本等。。\protect\hyperlink{fnref240}{↩}
\item
  \leavevmode\hypertarget{fn241}{}%
  姜嫄是传说中上古农神``后稷''之母,非常贤德,后世尊为``圣母'';嫫母是传说中的丑女,是黄帝的次妃。\protect\hyperlink{fnref241}{↩}
\item
  \leavevmode\hypertarget{fn242}{}%
  段公平君注:``尚尔'':即尚且之意。如纪昀《阅微草堂笔记·滦阳消夏录五》:``对神尚尔,对人可知''。\protect\hyperlink{fnref242}{↩}
\item
  \leavevmode\hypertarget{fn243}{}%
  刘曾复先生钞本作``胡卢闷''。``闷葫芦''比喻极难猜透或令人纳闷的事或话。\protect\hyperlink{fnref243}{↩}
\item
  \leavevmode\hypertarget{fn244}{}%
  谋猷为计谋,谋略之意。\protect\hyperlink{fnref244}{↩}
\item
  \leavevmode\hypertarget{fn245}{}%
  刘曾复先生钞本未注明板式,下同。\protect\hyperlink{fnref245}{↩}
\item
  \leavevmode\hypertarget{fn246}{}%
  刘曾复先生钞本作``加常'',系误;此处从``安五路(总讲)''。\protect\hyperlink{fnref246}{↩}
\item
  \leavevmode\hypertarget{fn247}{}%
  刘曾复先生钞本与``安五路(总讲)''均作``慌慌无策'',此处从《三国演义》原文。\protect\hyperlink{fnref247}{↩}
\item
  \leavevmode\hypertarget{fn248}{}%
  刘曾复先生钞本作``岂不等'',此处从``安五路(总讲)''。\protect\hyperlink{fnref248}{↩}
\item
  \leavevmode\hypertarget{fn249}{}%
  刘曾复先生钞本与``安五路(总讲)''均作``推病为词''。\protect\hyperlink{fnref249}{↩}
\item
  \leavevmode\hypertarget{fn250}{}%
  段公平君注:刘曾复先生钞本此句作``皇儿拜他以为相父称之'',文意欠通,疑是``皇儿拜他以为相父''和``皇儿以相父称之''两句错杂而成。考``安五路(总讲)''原亦作``皇儿拜他以为相父称之'',后删去``称之'',作``皇儿拜他以为相父'',此处从``安五路(总讲)''。\protect\hyperlink{fnref250}{↩}
\item
  \leavevmode\hypertarget{fn251}{}%
  ``安五路(总讲)''此处原作``求计'',改为``问计''。\protect\hyperlink{fnref251}{↩}
\item
  \leavevmode\hypertarget{fn252}{}%
  刘曾复先生钞本注``以下`观鱼遣邓'''。\protect\hyperlink{fnref252}{↩}
\item
  \leavevmode\hypertarget{fn253}{}%
  ``安五路(总讲)''本作``礼论不雅'',旁注``也是无法'';李元皓君注``不雅'',犹言``君不登臣门''之意。\protect\hyperlink{fnref253}{↩}
\item
  \leavevmode\hypertarget{fn254}{}%
  刘曾复先生说戏录音作``这鱼你'',此处从``安五路(总讲)''。\protect\hyperlink{fnref254}{↩}
\item
  \leavevmode\hypertarget{fn255}{}%
  刘曾复先生钞本与
  ``安五路(总讲)''均作``求条良谋''。\protect\hyperlink{fnref255}{↩}
\item
  \leavevmode\hypertarget{fn256}{}%
  刘曾复先生钞本作``兵伐五路''。\protect\hyperlink{fnref256}{↩}
\item
  \leavevmode\hypertarget{fn257}{}%
  刘曾复先生钞本作``人民振动''。\protect\hyperlink{fnref257}{↩}
\item
  \leavevmode\hypertarget{fn258}{}%
  刘曾复先生说戏录音中似作``神在边关之外'';段公平君认为说戏录音误作``身在边关之外''。\protect\hyperlink{fnref258}{↩}
\item
  \leavevmode\hypertarget{fn259}{}%
  刘曾复先生说戏录音作``韬略''。\protect\hyperlink{fnref259}{↩}
\item
  \leavevmode\hypertarget{fn260}{}%
  刘曾复先生说戏录音作``潜送''。\protect\hyperlink{fnref260}{↩}
\item
  \leavevmode\hypertarget{fn261}{}%
  刘曾复先生说戏录音作``山岭峻险''。\protect\hyperlink{fnref261}{↩}
\item
  \leavevmode\hypertarget{fn262}{}%
  刘曾复先生钞本作``急早回奏''。\protect\hyperlink{fnref262}{↩}
\item
  \leavevmode\hypertarget{fn263}{}%
  刘曾复先生钞本作``连和'',此处从《三国演义》原文。\protect\hyperlink{fnref263}{↩}
\item
  \leavevmode\hypertarget{fn264}{}%
  刘曾复先生钞本注,此段可不唱。\protect\hyperlink{fnref264}{↩}
\item
  \leavevmode\hypertarget{fn265}{}%
  刘曾复先生钞本此处径写``点将诗''。疑是\textless{}\textbf{点绛唇}\textgreater{}牌子,后接\textless{}\textbf{定场诗}\textgreater{}四句。\protect\hyperlink{fnref265}{↩}
\item
  \leavevmode\hypertarget{fn266}{}%
  刘曾复先生钞本中``厉害''均作``利害''。\protect\hyperlink{fnref266}{↩}
\item
  \leavevmode\hypertarget{fn267}{}%
  刘曾复先生钞本未注明板式。\protect\hyperlink{fnref267}{↩}
\item
  \leavevmode\hypertarget{fn268}{}%
  刘曾复先生钞本作``可于光殿前''(``光''字不确认,疑此字误衍或脱漏,如作``光明''),此处从《三国演义》原文。\protect\hyperlink{fnref268}{↩}
\item
  \leavevmode\hypertarget{fn269}{}%
  刘曾复先生钞本作``身长大面'',此处从《三国演义》原文作``身长面大''。\protect\hyperlink{fnref269}{↩}
\item
  \leavevmode\hypertarget{fn270}{}%
  刘曾复先生钞本作``常揖不拜'',此处从《三国演义》原文。\protect\hyperlink{fnref270}{↩}
\item
  \leavevmode\hypertarget{fn271}{}%
  刘曾复先生钞本作``尔想'',似欠通。\protect\hyperlink{fnref271}{↩}
\item
  \leavevmode\hypertarget{fn272}{}%
  刘曾复先生钞本作``窃𢫑中原'',``𢫑''同``據''。\protect\hyperlink{fnref272}{↩}
\item
  \leavevmode\hypertarget{fn273}{}%
  刘曾复先生钞本疑``为''或``不''字,段公平君注:``他能可'',系``他可能''颠倒。全句为反诘语气。\protect\hyperlink{fnref273}{↩}
\item
  \leavevmode\hypertarget{fn274}{}%
  ``连横''亦作``连衡''。\protect\hyperlink{fnref274}{↩}
\item
  \leavevmode\hypertarget{fn275}{}%
  奫,水深广的样子。刘曾复先生钞本注``奫(音
  氲)''。\protect\hyperlink{fnref275}{↩}
\item
  \leavevmode\hypertarget{fn276}{}%
  此戏的文字结合了樊百乐君提供的刘曾复先生说戏的实况录音整理的。刘先生为百乐君说此戏时,因为没有找到剧本,部分词句是临时回忆介绍的,因此有些小的地方与为戏曲学院说戏的词句略有出入。\protect\hyperlink{fnref276}{↩}
\item
  \leavevmode\hypertarget{fn277}{}%
  ``带砺山河''亦作``带厉山河'',这里借指诸葛亮的忠贞之心恒久不变。\protect\hyperlink{fnref277}{↩}
\item
  \leavevmode\hypertarget{fn278}{}%
  ``刻今''犹``刻下''之意,即现在,当下。\protect\hyperlink{fnref278}{↩}
\item
  \leavevmode\hypertarget{fn279}{}%
  段公平君建议作``一理'',并注:《出师表》有``宫中府中,俱为一体'',故选``一理''。夏行涛君建议``依例''或``依礼''也都合文意。\protect\hyperlink{fnref279}{↩}
\item
  \leavevmode\hypertarget{fn280}{}%
  这段【西皮快板】原词较长,兹录如下:

  ``镇北将军名魏延。自从长沙来降汉,跟随山人二十年。今日战比不得往日战,比不得当年大战在渭南。四更时分造战饭,要出兵来五更天。假扮姜维关前站,口口声声出反言。''\protect\hyperlink{fnref280}{↩}
\item
  \leavevmode\hypertarget{fn281}{}%
  ``陉''是山脉中断的地方,这样的地方往往是重要的关隘。这里特指街亭。\protect\hyperlink{fnref281}{↩}
\item
  \leavevmode\hypertarget{fn282}{}%
  刘曾复先生示范说戏时介绍,这是慈瑞泉临场抓的哏。\protect\hyperlink{fnref282}{↩}
\item
  \leavevmode\hypertarget{fn283}{}%
  刘曾复先生示范说戏时介绍,此句原作``叉出帐去免责问''。\protect\hyperlink{fnref283}{↩}
\item
  \leavevmode\hypertarget{fn284}{}%
  ``拿云手''比喻远大的志向。\protect\hyperlink{fnref284}{↩}
\item
  \leavevmode\hypertarget{fn285}{}%
  陈超老师介绍:刘曾复先生所传的是贾丽川家的路子。\protect\hyperlink{fnref285}{↩}
\item
  \leavevmode\hypertarget{fn286}{}%
  刘曾复先生钞本作``遍身是汗''。\protect\hyperlink{fnref286}{↩}
\item
  \leavevmode\hypertarget{fn287}{}%
  陈超老师介绍了这一场相关的舞台布局及调度。\protect\hyperlink{fnref287}{↩}
\item
  \leavevmode\hypertarget{fn288}{}%
  《三国演义》原文为``\textbf{永延汉}祀''。\protect\hyperlink{fnref288}{↩}
\item
  \leavevmode\hypertarget{fn289}{}%
  刘曾复先生钞本作``南斗合北斗''。\protect\hyperlink{fnref289}{↩}
\item
  \leavevmode\hypertarget{fn290}{}%
  刘曾复先生钞本作``看看''。\protect\hyperlink{fnref290}{↩}
\item
  \leavevmode\hypertarget{fn291}{}%
  刘曾复先生钞本作``只望''。\protect\hyperlink{fnref291}{↩}
\item
  \leavevmode\hypertarget{fn292}{}%
  刘曾复先生钞本作``事到临头''。\protect\hyperlink{fnref292}{↩}
\item
  \leavevmode\hypertarget{fn293}{}%
  李楠君认为作``朝喜夕厌''。\protect\hyperlink{fnref293}{↩}
\item
  \leavevmode\hypertarget{fn294}{}%
  这一场戏一般省去,是王浚教训自己的儿子所唱。\protect\hyperlink{fnref294}{↩}
\item
  \leavevmode\hypertarget{fn295}{}%
  ``\textbf{坉''的意思是}用草袋装土筑墙或堵水。\protect\hyperlink{fnref295}{↩}
\item
  \leavevmode\hypertarget{fn296}{}%
  夏行涛君建议此句作``何不写状告于他?''\protect\hyperlink{fnref296}{↩}
\item
  \leavevmode\hypertarget{fn297}{}%
  此处原来唱``议论'',刘曾复先生听从吴小如先生建议,改唱``理论'',唱词文意更通顺。\protect\hyperlink{fnref297}{↩}
\item
  \leavevmode\hypertarget{fn298}{}%
  此戏的文字也结合了刘曾复先生两次为樊百乐君说戏的实况录音整理完成的。\protect\hyperlink{fnref298}{↩}
\item
  \leavevmode\hypertarget{fn299}{}%
  贺道庵也有本记作``贺道安''或``何道安''的。此处从《戏考》。\protect\hyperlink{fnref299}{↩}
\item
  \leavevmode\hypertarget{fn300}{}%
  刘曾复先生记忆中刺客本名作``谢二'',但在为樊百乐君说戏时所本作``谢四''。

  据段公平君告知:《曲海总目摘要》载清传奇《九莲灯》,丞相名``霍道南'',刺客名``獬儿''。\protect\hyperlink{fnref300}{↩}
\item
  \leavevmode\hypertarget{fn301}{}%
  此句陈超老师作``两旁坐下''。据陈超老师介绍:``两边坐下''是坐垫子,由检场的扔。\protect\hyperlink{fnref301}{↩}
\item
  \leavevmode\hypertarget{fn302}{}%
  陈超老师介绍,此处先烧纸,再上香。\protect\hyperlink{fnref302}{↩}
\item
  \leavevmode\hypertarget{fn303}{}%
  陈超老师介绍,此段王荣山教王又宸坐着唱,王又宸后改为站着唱。夏行涛君建议``此黄土''作``似黄土''。\protect\hyperlink{fnref303}{↩}
\item
  \leavevmode\hypertarget{fn304}{}%
  陈超老师介绍,唱此句场面起\textless{}\textbf{快扭丝}\textgreater{},一定站着唱。\protect\hyperlink{fnref304}{↩}
\item
  \leavevmode\hypertarget{fn305}{}%
  李舒先生遗作《涉艺所得》录《刘曾复修润剧本四篇》中《\textless{}御碑亭\textgreater{}及其他》一文中此句作``但愿得贼兵退此地安静''。\protect\hyperlink{fnref305}{↩}
\item
  \leavevmode\hypertarget{fn306}{}%
  陈超老师介绍此处老生表演为:

  左手往外转一个水袖,一抓藤,背身右手往外转一个水袖,一抓藤,滑步,倒步,把旦角和俩小孩挤到台口。旦角坐地,邓方扶旦角,老生扶邓元,再上山石片。

  \begin{quote}
  陈超老师按:这是老谭的身段。
  \end{quote}

  \protect\hyperlink{fnref306}{↩}

%\addcontentsline{toc}{section}{\hfill[\hei 隋·唐]\hfill}
\newpage
\chead{隋·唐} % 页眉中间位置内容
\textbf{当锏卖马}\protect\hyperlink{fn307}{\textsuperscript{307}}

{[}第一场{]}

王老好 (内)啊哈!

(\textless{}\textbf{小锣打上}\textgreater{})

王老好
【数板】不赊不欠不算店,赊赊欠欠不见面。他在前街走,我在后街转。二人见了面,他说不方便。改日再见,改日再见。(白)开的是店,卖的是饭。一个人吃半斤,三人\protect\hyperlink{fn308}{\textsuperscript{308}}吃斤半。我王老好。怎么叫这个名儿呢,我在这天堂地面,开了一座小小的店房,有那南来北往的客人,有钱儿没钱儿的吃了就走,人们就给我送了个绰号,叫王老好,这且不言。前些日子,我这店里来了个山东好汉,名叫秦琼,住了这么多天啦,一个大钱儿也没给,人的饭,马的料,我能老垫着吗,今儿个把他请出来,商量商量,就是这个主意。

(站,向上场门)

王老好 二爷起来了没有,请出来凉快凉快罢。

秦琼 (内)嗯喷。

(小锣上,揉手揉眼,伸懒腰,到小边台口\textless{}\textbf{哭相思}尾\textgreater{})好汉英雄困天堂,不知何日回故乡。

(指,转身望王老好)

王老好 二爷。

(秦琼拱手,让)

秦琼 店主东,请到里面,请坐。

(进门,中间小坐,王老好大边旁坐)

秦琼
(啊,)店主东,将你二爷请了出来,是饮酒哇(或:还是吃酒哇),还是吃饭呐?

(左手做拿杯,双手做吃饭状)

王老好
出来就是吃喝,二爷酒也要喝,饭也要用,今儿小店家有两句话,不知当讲不当讲?

秦琼 店主东有话请讲当面。

(撩鸾带、左腿搭右腿上,鸾带搭在左腿上、左手下,右手放在左手上,一块儿放在膝上,望王老好)

王老好
没别的,二爷您在我这儿的店里,日子可也不少啦,人的饭食,马的草料,我有点儿垫不起啦。

(秦琼做睡状)

王老好 你瞧,着啦,二爷你醒醒,别睡。

秦琼 你讲你的。

(睡着说)

王老好 那么您呐。

秦琼 我哇,我睡我的呀。

(仍睡状)

王老好 那我说给谁听呐?

秦琼 你讲我听得见呐。

王老好 二爷您还是别睡啦。

秦琼 好,我不睡就是。

(醒听王老好讲,放下腿)

王老好
我说二爷,您在我这儿店里日子可也不少啦,人的饭食,马的草料,我可有点垫不起啦,您是有银子有钱,拿出来,我好垫补着花。

秦琼 听你之言是要钱呐?

王老好 不敢说要,跟您借俩钱使唤使唤。

秦琼 (店主东,)我进店的时节,(我)也曾对你讲过哇。

王老好 真格的,您说什么来着?

秦琼
只因我押解了一十八名江洋大盗,(天气炎热,)损伤一名。那蔡大老爷不与我批票回文,故而我在此等候(或:故而被困在此)。待等批票发下,那时节再算还你的店钱不迟,(或:待等那蔡大老爷发下路费银两,算还你的饭钱也还不迟,)哦,你何必这样着急呀。

(略转右,不听状)

王老好 哦,是我着急,比方这么说吧 ,这批票回文要是一个月发不下来?

秦琼 你就等上他一个月。

(看一下王老好,再转过来)

王老好 一个月三十天不多,好等,那么一年不下来呢?

秦琼 何妨等上他一年呐。

(身子放右转)

王老好 哦,要是一辈子发不下来呢?

秦琼 (哦,这一辈子么,)那就算你倒了运了。

(右转身,右腿架左腿上,右臂架椅背上,不理状)

王老好 嘿,您别着急,我再跟您说,人吃五谷杂粮,
可没有不生病的,瞧您这样儿,要是您死在我这个店里呢。

秦琼 怎么?

(站椅右侧)

秦琼
(我)若是死在你的店(中)么,哈哈哈\ldots{}\ldots{}那你就大大的发了财了。

(坐下)

王老好 哦哦哦,我这财是怎么个发法呢?

秦琼
等你二爷死后,你必须买上寿衣、寿帽,大大的一口棺木(或:棺材),将你二爷成殓起来,(比势)灵前立一牌位,上写山东好汉秦琼之位。(王应)(啊,)店主东那时节你可就不要这样打扮了。

王老好
对,我发啦财\protect\hyperlink{fn309}{\textsuperscript{309}},得捯饬捯饬。

秦琼
你必须头戴麻冠(王老好应),身穿重孝(王老好应),手拿哭丧棒,再与你二爷摔丧盆子,(拿棒状,摔盆状,伸右手,切手,点两下)然后你再大大的请上(他)一个份子,岂不就发了财了么?(双手比势)

王老好 啊哟,照你这么一说,我不就成了你的儿子了吗?

秦琼
哎呀呀,(我不敢呐,)我没有那样的造化呀。(立椅右侧,双手摇,托胡子向王老好扔,坐)

王老好
好哇,你吃饭不给钱,还转着弯儿骂人,你身上没穿树叶儿,我今儿要剥你。

秦琼 怎么,你要剥(上口)你二爷?(哼,量你也不敢呐。)

王老好
上口也得剥你,说剥就剥。(王上前剥状,秦琼绕王老好右手抓着一拧,往后拉住一按,王转身弯腰叫,秦推王手推开王,王起甩胳膊揉)

王老好 瞧这瘦样,怎么这么大劲,不信我打不过,我呀,喊叫臊你的皮。

秦琼 (怎么,你要喊叫?)任凭你去喊叫。

王老好
说叫就叫,街坊、邻舍,我这来了个山东好汉秦琼,白吃不给钱,还要打人。

(王老好出门,向上场门叫,回身,秦琼摊右手,出门,右手堵王嘴,放手)

王老好 你要堵死我哇?

秦琼 我有策划呀。

王老好 要钱不给,还要拆毁。

秦琼 就是商量商量呐,进来进来。

(秦琼进门归中间,王老好跟进归大边,立)

秦琼 槽头之上,二爷的黄骠马,牵到市上卖了银钱,(算)还你的店饭钱就是。

王老好 就是那匹马呀,瘦得成马灯啦,没人儿要!

秦琼 你是不懂呐,这货哇要与识家呀。(右手在上边一画,再指眼)

王老好 对,我外行。

秦琼 店主东。

王老好 怎么着?

秦琼 牵马呀\ldots{}\ldots{}

王老好 嘿,别哭哇。

(王老好过小边拉马回大边)

秦琼 【西皮慢板】(王老好唱中夹白)店主东带过了黄骠马(王老好
\textbf{给您带过来了}),不由得秦叔宝两泪如麻(王老好
\textbf{您怎么哭啦})。提起了此马来头大(王老好
\textbf{怎么个来头呢}),兵部堂皇甫爷相赠与咱(王老好
\textbf{不该卖呐})。遭不幸困至在天堂下,还你的店饭钱无奈何只得来卖它。摆一摆手儿你就牵去了吧,(秦唱中比势黄骠马,握右拳牵马状,``泪如麻''右手擦眼泪,``来头大''抬右大指,``与咱''轻拍右腹部,``卖它''左手捋胡子右手指,``去了吧''右手两画圈挥手叫王牵马走,下场门下,秦跟出门望下场门,回身进门,转身面外接唱末句)但不知此黄骠落于谁家(或:但不知此马落在谁家)。

(秦琼上步转身走下)

{[}第二场{]}

(\textless{}\textbf{抽头}\textgreater{}接\textless{}\textbf{快长锤}\textgreater{},四青袍引单雄信上,站九龙口唱)

单雄信 【西皮摇板】自幼闯荡江湖下,

(众归正)

单雄信 (接唱)【西皮摇板】人人道来是豪家。闲来无事大街耍,

(单雄信众圆场,王老好牵马从外边抄回去到大边台口牵马亮,单转到台小边台口,上步背扇一望,王牵马下,单归中间)

单雄信 (接唱)【西皮摇板】只见黄骠人爱煞。人来与爷忙赶下,

(众下,单雄信到下场门边回身)

单雄信
(接唱)【西皮摇板】\protect\hyperlink{fn310}{\textsuperscript{310}}不知此马是谁家?

(单雄信下)

{[}第三场{]}

(轻\textless{}\textbf{快长锤}\textgreater{}秦琼上,九龙口望,叹,到台口,\textless{}\textbf{闪锤}\textgreater{}唱)

秦琼
【西皮摇板】店主东卖黄骠不见回转,倒叫我(或:好教我)秦叔宝两眼望穿。

(坐中间小座,王急上大边栓马进门)

王老好 马给你牵回来啦,一根马毛也没短。买马的在后头,有话你们说罢。

(王老好急下,秦摊手、立,出店门立大边。单雄信众上站门,单立小边,一青袍解马牵马,单望念,\textless{}\textbf{小锣二三锣}\textgreater{})

单雄信
(念)此马是黄骠,四蹄似雪飘。浑身发金色,遍体无杂毛。胜似南山豹,亚塞浪里蛟。

\begin{quote}
好马呀,好马!
\end{quote}

(秦右手比画一挥,单雄信望秦)

秦琼 啊,连夸好马,敢是有爱马之意?

单雄信 好马人人皆爱,只是膘头忒瘦了。

秦琼 草料不佳之故。此处不是讲话之所,请到里面(或:店房一叙)。

单雄信 请呐。

(单雄信、秦琼挖门进店,众跟进站门,单大边秦小边八字。里边坐,单望秦)

单雄信 听兄台讲话不像此地人氏。

秦琼 本不是此地人氏。

单雄信 哪里人氏?

秦琼 山东历城县人氏。

单雄信 山东历城县,弟有一家好友,兄台可知。

秦琼 有名便知,无名不晓。

单雄信 提起此人大大有名。

秦琼 但不知是哪一家。

单雄信 此人姓秦名琼字叔宝。

秦琼 秦琼,此人落魄潦倒哇(或:唉,此人落魄得紧呐)。

单雄信 何出此言?

秦琼 这,愚下(或:在下)就是秦琼。

(单雄信、秦琼立,单托秦双手架住)

单雄信 你是秦二哥?

秦琼 不敢。(或:岂敢。)

单雄信 叔宝。

秦琼 正是。(或:不敢。)

单雄信 哈哈哈\ldots{}\ldots{}请来上坐。

(单雄信让秦琼,秦过大边,单过小边,同坐下)

(秦琼 哈哈哈\ldots{}\ldots{})

秦琼 听兄台讲话也不像此地人氏。

单雄信 本不是此地人氏。

秦琼 哪里人氏?

单雄信 河南二贤庄人氏。

秦琼 (河南)二贤庄弟有一(位)好友,兄台可知。

单雄信 有名便知,无名不晓。

秦琼 提起此人大大有名。

单雄信 但不知是哪一家。

秦琼 姓单名通字雄信。

单雄信 小弟就是单通。

(秦琼、单雄信立架住,秦望单)

秦琼 你(就)是单通?

单雄信 正是。

秦琼 单员外。

单雄信 不敢。

(秦琼让坐)

秦琼
请来上坐,哈哈哈\ldots{}\ldots{}(或:啊哈哈哈\ldots{}\ldots{}请来上坐。)

(秦琼单雄信换座,秦掸座、单拦,坐)

单雄信 二哥为何这等模样?

秦琼
只因愚兄押解一十八名江\ldots{}\ldots{}俱是我们绿林中的朋友哇,天气炎热,中途路上,损伤一名。那蔡大老爷不与我批票回文,故而被困在此。

单雄信 这有何难,待小弟拿我名帖讨来就是。

秦琼 有劳贤弟。

单雄信 前者伯母寿诞之期,小弟有一份薄礼可曾收到。

秦琼 但不知打从哪道而去。

单雄信 打从那黑\ldots{}\ldots{}

秦琼 收到了,当面谢过。(或:哦,当面谢过。)

(\textless{}\textbf{冲头}\textgreater{}家院上,进门站大边)

家院 启员外:大事不好了。

单雄信 何事惊慌?

家院 大员外在临潼山被李渊一箭射死。

单雄信 不好了!

(\textless{}\textbf{急三枪}\textgreater{}单雄信擦泪,秦琼摊手)

秦琼 天气炎热,就该搬灵的才是呀(或:就该搬尸的才是呀)。

单雄信 怎奈无有乘骑。

家院 这儿不有匹马吗?

单雄信 秦二爷的马焉能乘骑。

秦琼
呵贤弟,(有道是:)乘肥马,衣轻裘,与朋友共,敝之而无憾\protect\hyperlink{fn311}{\textsuperscript{311}}呐。哈哈哈\ldots{}\ldots{}

单雄信 怎么,骑得的。

秦琼 骑得的。

单雄信 带马。

(王老好暗上,\textless{}急三枪\textgreater{}单雄信秦琼出门,单上马,秦小边,单回身)

单雄信
\textless{}\textbf{叫头}\textgreater{}二哥!小弟此去多则半月,少则十天,将马匹送回,请。

(单雄信\textless{}\textbf{抽头}\textgreater{}下,秦琼过大边望)

秦琼
(啊,)贤弟慢走,(恕)愚兄不能远送了。(啊,哈哈哈\ldots{}\ldots{}(笑介))

(秦琼看王老好)

秦琼 这才是我的好朋友。(这才是我的好朋友。)

(秦琼拍右腹,伸右手大指)

王老好 这才是好桐油。

秦琼
哦,分明是好朋友,怎么说是好桐油哇。(或:呃,分明是好朋友,什么好桐油哇)

(挥右手,探手)

王老好
好朋友?我们这里,他是响马头儿。这儿三岁小孩子都认得他,他把你的马给拐走啦。

秦琼
(单通单二员外呀,)就是他。(秦琼右手弹胡子,左手捋胡子,抬左腿,左转身向下场门,右手指,弓箭步矮相)

王老好 不是他还是我?

(秦琼右手捋胡子,右转身回来,抬手切手)

秦琼
唉!【西皮摇板】骂一声秦琼瞎了眼呐,把响马当作好宾朋。我拉住了(或:我抓住了)店家撒一个赖呀。

(背供指,抓王老好领)

秦琼
好店家,你勾结响马把我的马骗了去了,快快还我的马便罢,不然呐,我就要你的老命呐。

王老好 你先等等,咱们讲个理儿。

秦琼 讲。

王老好 我问问你,你的马在槽头上拴着,是前门撬了锁啦?

秦琼 不曾。

王老好 后门挖了窟窿啦?

秦琼 也不曾。

王老好
这不结了吗。你把你的马送给你的好朋友啦,跑我这儿撒赖,别不害臊啦。(扔秦琼胳膊,秦臂画一圈垂右侧,望王老好)

秦琼
唉!(接唱)【西皮摇板】如此说我和你呀(就)两丢开。(右拳击左掌,双手分开摊,手把王老好碰倒,秦琼进门正小坐,王起来,王进门站大边)

王老好 摔着啦,两丢开两丢开,还得拿钱来。

秦琼 (唉,)还是没有钱呐。

王老好 你没钱,我照方抓药,我还给你喊叫去。

秦琼 任凭你去喊叫。

(王老好出门)

王老好 街坊、邻舍:我这儿\ldots{}\ldots{}

(王老好先向下场门,回身,秦琼出门,秦出门挡王,王往后一退)

王老好 我早预备啦。

秦琼 (呃,)我还有策划呀,进来进来。

王老好 那咱们就进去,你说罢。

(二人进门,秦琼中、王老好大边站,小边架锏)

秦琼 兵刃架上劈抡双锏,拿到市上卖了银钱,(算)还你的饭钱。

王老好 就那俩家伙儿,当通条嫌短,当火筷子太长,没人儿要。

秦琼 呵,(有道是)货卖与识家呀。

王老好 还卖呐,马都让人给识了去啦。

秦琼 惭愧。

王老好 蝉蜕?卖给药铺啦。

秦琼 你且取来。

王老好 好。(王老好过小边,拿锏拿不动)

王老好 回二爷的话,它拿我不动。

秦琼 敢是你拿它不动罢。(或:你敢是拿它不动罢。)

王老好 有那么点儿。

秦琼 闪开了。

(左右两枕,紧大带,过小边右手拿锏,交左手抱,右手山膀亮住)

王老好 我短这两手。

秦琼 【西皮摇板】(王夹白)家住山东历城县(王老好
好地方),秦琼的名儿天下传。我本是顶天立地男儿汉(王老好
拿钱来),好汉无钱到处难(或:处处难)(王老好
甭瞎充)。无奈何出店门我就卖\ldots{}\ldots{}

(秦琼右手撩带迈右脚出门,左望,退脚右手挡脸,左转身向小边)

王老好 你卖什么,我的爹。

秦琼 嗳!(接唱)【西皮摇板】卖锏呐,

(\textless{}\textbf{快长锤}\textgreater{}秦琼出门,王老好跟着,秦弹胡子,山膀,往左走圆场,王伯当\protect\hyperlink{fn312}{\textsuperscript{312}}、谢映登二人上,从外边抄过去,到大边,秦捋胡子站住,与王、谢对望亮住,王、谢下,秦弹胡子\textless{}\textbf{紧锤}\textgreater{}向下场门过去望,捋胡子转身向外)

(王伯当、谢映登同上,过场,同下。)

秦琼
(接唱)【西皮快板】两匹马跑得似雪花。分明知道(或:明明知道)是响马,无有批票不好拿。叫声店家(或:叫住店家)快来吧,还你的饭钱就是他。(秦琼一指,到台口,王老好到小边,秦往外一指,往里甩胡子,左手抱锏,由下往右往上往左外甩胡子,右手往外指台下,左腿抬站住,王跟着望,秦弹胡子转身向下场门山膀下)

王老好 您哪一位给呀?

(王老好回头一看)

王老好 跑啦,追。

(王老好下)

{[}第四场{]}

(\textless{}\textbf{水底鱼}\textgreater{}王伯当、谢映登上)

王伯当 王伯当。

谢映登 谢映登。

王伯当 贤弟,前面一座酒肆,你我歇息歇息。

谢映登 好,店家哪里?

(\textless{}\textbf{小锣}\textgreater{},店家上)

店家 (念)杏花村店酒,开门十里香。

\begin{quote}
二位是喝酒的吗?
\end{quote}

王伯当、谢映登 正是。将马带过。

(店接马拴大边,王、谢进店骑马八字坐)

店家 二位用些什么?

王伯当 好酒取来。

店家 好酒一壶!酒到。(店家拿酒摆桌上)

谢映登 唤你再来。

(店家下)

王伯当 请。

(王、谢饮酒,秦琼抱锏、王老好先后上,秦到台口)

秦琼 卖锏!

王老好 卖脸。

(秦琼看王老好)

秦琼 卖锏呐!

王老好 卖脸呐!

秦琼 (呃,)分明是卖锏,怎说是卖脸呐(或:卖的什么脸呐)?

王老好
前街走到后巷,后巷又到大街,连个搭理的也没有,这不是卖脸还卖什么呀?

秦琼 (卖锏,)还是卖锏的受听呐。

王老好 卖脸,卖脸定啦。

(秦琼转小圆场到台口)

秦琼
君子不得第(或:君子不得志),反被这小人欺\textless{}\textbf{哭相思}\textgreater{}。

王老好
店家倒了运,遇见了白吃的\textless{}\textbf{哭相思}\textgreater{}。

秦琼 哪个白吃?

王老好 就是你,你拿钱来。

(秦琼在台口望大边叫王老好)

秦琼 (好好好,)店主东你来看,那旁有两骑高头大马。

王老好 怎么,你要偷人家。

秦琼
啊,骑马之人(或:乘马之人)不是好人,定是响马,你去问问他们可要锏呐。

(秦琼右手指大边再指锏,王、秦换位)

王老好 啊,我去问问,你可别走。

(秦琼 只管地前去。)

王老好 嘿,是我亲家这儿,亲家哪儿?

店家 那是谁这么嚷呐,哟,亲家,你好哇,干什么来啦?

王老好 你们要锏不要呐?

店家 我不洗衣裳,要碱干什么?

王老好 不是,是兵器。有俩骑马的罢,你去问问他们要不要。

店家 好,我问问去,做成了二八扣。

王老好 你去吧。

店家 (店家进门)二位客官要不要锏呐,是兵器。

王伯当 叫那卖锏之人进来,当面言价。

店家 是,亲家,人家要,可是得当面讲价儿。

王老好 是,二八扣你也吹啦,我说二爷,着啦,二爷,

(秦琼打盹,听见叫,把锏推在地上)

秦琼 哎呦砸了我的脚了。

王老好 你得了罢,还上口呐,锏躺这边,会砸了你的脚,没砸着我就不错。

(随便捡起锏要递给秦琼)

秦琼 放下,你不是拿它不动吗?

(王老好赶快放下锏)

王老好 我忘啦。

(秦琼拾锏,秦、王换位)

秦琼 他们讲些什么?

王老好 要倒是要,叫你进去当面讲价。

秦琼 好,走走走。(或:呃,走走走。我们一同前去。)

(秦琼拉王老好)

王老好 我不进去,你拉着干嘛?

秦琼 我怕你跑了哇。

王老好 你该我的,我不怕你跑,你干嘛怕我跑。

秦琼 你跑了我吃哪一个哇?

王老好 去你的吧,官人响马我别掺混啦。

(秦琼 势利的小人。)

(王老好下,秦琼、店家先后挖进,秦大边,店小边,秦望二人)

秦琼 请了。

王伯当 此锏可是要卖?

秦琼 正是要卖。

王伯当 借来一观,放在桌上。

秦琼 有些沉重。

(锏放桌上)

店家 别砸坏桌子。

王伯当 有我等包赔。那一汉子你可会使?

秦琼 略知一二。

王伯当 耍来我等观看。

秦琼 这。(摸肚子)

王伯当 店家,带他前去用饭。

店家 走哇。

秦琼 做什么?

店家 吃饭去。

秦琼 吃饭呐,走哇。(一挥手,跟店家下)

王伯当 我看此人,莫非秦琼。

谢映登 少时问过。

店家 (店家上小边)二位,这位好大饭量,五十包子,外带十碗粥。

王伯当 我等开销。

秦琼 (秦琼上,拱手) 多谢二位酒饭。

王伯当 可曾用好?

秦琼 (秦琼拍腹)(无非是)充饥而已。

王伯当 耍来我等观看。

秦琼 (秦琼看地方,指)此地狭小。(或:此地窄小)

王伯当 哪里宽阔?

店家 后面宽阔。

王伯当 带路。

(秦琼抱锏中间,王伯当、谢映登、店家两边一翻两翻,秦望,王、谢归坐,店下,秦归大边站)

秦琼 献丑了。

(躬揖,回身中间站)

秦琼
【西皮摇板】站在店中用目𥋌\protect\hyperlink{fn313}{\textsuperscript{313}},

(右手弹胡子,小边外边山膀,望王伯当、谢映登,回来正面捋胡子)

秦琼
(接唱)【西皮快板】不由得叔宝怒气发。明明认得他是响马,江湖路上也曾会过他。骂一声贼子真胆大,杀人放火海走天涯。今日里相逢在潞州天堂下,无有批票不好拿(或:不敢拿)。眼前若在历城县,定要将他锁拿到公衙。板子打夹棍夹,看他犯法不犯法。减头去尾耍一耍,(倒叫二位耻笑咱。在舞台演出时此句不唱)

(耍\textbf{锏架子}\protect\hyperlink{fn314}{\textsuperscript{314}},\textless{}\textbf{扫头}\textgreater{}归中间坐,王伯当、谢映登左右两边坐,秦放锏椅侧)

王伯当 听兄台讲话不像此地人氏。

秦琼 本不是此地人氏。

王伯当 哪里人氏?

秦琼 山东历城县人氏。

王伯当 山东历城县弟有一家好友,兄台可知?

秦琼 (有名便知,无名不晓。)但不知哪一家?

王伯当 姓秦名琼字叔宝。

秦琼 愚下(或:在下)就是秦琼。

王伯当、谢映登 原来秦二哥!失敬了。

秦琼 岂敢。请问二位上姓?

王伯当 小弟王伯当。

谢映登 小弟谢映登。

秦琼 原来是二位贤弟,失敬了。

(店家暗上)

王伯当 二哥为何这等模样?

秦琼
只因愚兄解押一十八名绿林(中的)朋友,天气炎热,中途路上,损伤一名。那蔡大老爷不与我批票回文,故而被困在此。

王伯当 这有何难,店家拿我二人名帖,去到蔡大老爷那里,请他发下回文。

(秦琼 有劳了。)

(店家接帖下)

王伯当 前者伯母寿诞之期,弟等有份薄礼可曾收到?

秦琼 但不知打从哪道而去?

王伯当 打从那黑\ldots{}\ldots{}

秦琼 收到了(或:这\ldots{}\ldots{}收下了),当面谢过。

(店家上)

店家 批票领到。

(店家交王伯当,王交秦琼,店下)

王伯当 回文在此 。

(秦琼接文,谢映登取银交秦)

王伯当 弟等散碎银两,二哥收下。

秦琼 二位贤弟银两愚兄怎能用得。

王伯当 二哥不必过谦。

秦琼 (如此)愧领了。

(三人站,秦琼抱锏,三人躬揖)

秦琼
【西皮摇板】心中恼恨单雄信,不该骗我马能行。有朝犯在秦琼手,我打一锏来我要问一声。

(两手分锏举锏亮,合锏右手平划,右转整身面外,右食指向上指,手放下,王伯当、谢映登揖,秦还揖)

王伯当 看在我二人份上,

秦琼 (接唱)【西皮摇板】二贤弟只管把响马来放,

(撩带出门,弹胡子,山膀转身里走向下场门站住,王伯当、谢映登跟出站小边)

王伯当 闯出祸来?

(秦捋胡子回身)

秦琼 (接唱)【西皮摇板】闯出祸来由秦琼担承。

(秦琼拍肚子,向外甩胡子,双手揖,左手抱锏向右捋胡子,右手山膀下。王、谢请,二人挖门进去八字立)

王伯当 店家,酒饭钱在此,我等去也。

(王、谢二人出门拉马,上马\textless{}\textbf{扫头}\textgreater{}下)

\newpage
\hypertarget{ux5357ux9633ux5173}{%
\subsection{南阳关}\label{ux5357ux9633ux5173}}

{[}第一场{]}

(四红龙套,尚师徒、麻叔谋,韩擒虎大锣打上,小座)

韩擒虎
\textless{}\textbf{点绛唇}\textgreater{}奉王钦命,统领雄兵,军威盛,将勇兵精,干戈定太平。

韩擒虎
(念)堂堂男儿立帝基,巍巍武将挂铁衣。咚咚战鼓惊天地,杀气腾腾鸟难飞。

韩擒虎
(白)本帅,韩擒虎。奉了新主钦命,统领人马,捉拿伍云召进京问罪,二位将军,人马可齐?

尚师徒、麻叔谋 俱已齐备。

韩擒虎 兵发南阳。

尚师徒、麻叔谋 兵发南阳去者。

(\textless{}小\textbf{朱奴儿}\textgreater{}众领下)

{[}第二场{]}

伍保 (内白)马来。

(\textless{}\textbf{水底鱼}\textgreater{}前半,上)

伍保
俺,伍保。太老爷不知身犯何罪,敲牙割舌而亡。不免回转南阳关,报与老爷知道,就此马上加鞭。

(\textless{}\textbf{水底鱼}\textgreater{}后半,下)

{[}第三场{]}

(\textless{}发点\textgreater{}四文堂站门,伍云召上)

伍云召 {[}引子{]}威风浩荡,统雄师,镇守南阳。

(\textless{}\textbf{发点}合头\textgreater{},大座)

伍云召
(念)统雄师东征西荡(或:东杀西挡),每年间杀砍战场。食君禄哪得安享,与祖先廊庙争光。

伍云召
(白)本帅伍云召。隋帝驾前为臣,吾父官居太宰,(或:本帅武云召,吾父伍建章,文帝驾前为臣,官居当朝太宰。)本帅镇守南阳,只因夫人新生(或:只因夫人生下)一子,也曾命家将(或:也曾命家人)伍保进京,一来与圣上问安,二来与爹娘报喜,一去数日未见回报。今当操演之期,站堂军,教场去者。(或:这几日本帅心惊肉跳,不知为了何事。今当操演之期,左右,教场去者。或:这几日本帅心惊肉跳,不知为了何事。站堂军,伺候了!或:这几日本帅心惊肉跳,不知为了何事。这且不言,今当三六九日操演之期,站堂军,教场去者。)

伍保 (内白)走哇!

(上,小边下马,进门站小边)

伍保 参见老爷。

伍云召 伍保回来了?

伍保 回来了,大事不好了!

伍云召 (伍云召惊介)什么大事?(或:何事惊慌?)

伍保 太老爷、太夫人不知身犯何罪,敲牙割舌而亡。

(伍云召出位,台口拉伍保)

伍云召 你待怎讲?(或:怎么讲?!)

伍保 敲牙割舌而亡。

(伍云召拉住伍保带到大边去)

伍云召
\textless{}\textbf{叫头}\textgreater{}爹爹,母亲,哎呀!(昏坐下,伍保挡)

伍保 老爷醒来。

伍云召
【西皮导板】闻惊耗不由人魂魄掉,\textless{}\textbf{叫头}\textgreater{}爹爹(或:父亲),母亲,爹娘啊,啊!

伍云召 (接唱)【西皮散板】珠泪点点往下抛。忍泪含悲(或:本帅开言)叫伍保,

伍云召 伍保,

伍云召 (接唱)【西皮散板】被害的(或:犯罪的)情由(细)说根苗。

伍保 老爷。

伍保
(接唱)【西皮散板】杨素化及行奸巧,太宰割舌把牙敲。擒虎领兵人马到,捉拿老爷转回朝(或:转还朝)。

伍云召 好贼!(推胡子,指)

伍云召
【西皮散板】听一言来心头恼,二目圆睁似火烧。站在大堂传令号,大小三军(或:大小儿郎)听根苗:本帅有心把仇报,尔等可敢反皇朝?(翻袖指)

众 我等情愿。

(众跪,伍云召翻双袖扶众,众起立,右回身拿令旗亮住,面向大边伍保)

伍云召 (接唱)【西皮散板】伍保与爷改旗号,

(绕旗扔递给伍保,众、伍保举旗与众归小边,伍云召撩袍面向下场门、扔袍回身立大边面向台口)

伍云召 (接唱)【西皮散板】南阳关杀一个浪里蛟。

(\textless{}\textbf{四击头}\textgreater{}亮相,撩袍走到大边,撩袍抬左腿面向里亮相,走下,众跟下)

{[}第四场{]}

韩擒虎 (内)【西皮导板】南阳关前放号炮,

(韩众上站斜门,韩擒虎上)

韩擒虎 【西皮原板】对对旌旗空中飘。

(众归正场)

韩擒虎 (接唱)
【西皮原板】左先行骑的是呼雷豹,右先行稳坐在马鞍桥。大小三军齐开道,韩擒虎马上叹英豪。伍建章他本三朝元老,为国忠良无下梢。

尚师徒 (接唱) 【西皮原板】新主降下旨一道,元帅何必挂心劳?

麻叔谋 (接唱) 【西皮原板】在王驾前说王好,食王爵禄当效劳。

韩擒虎
【西皮快板】二先行\protect\hyperlink{fn315}{\textsuperscript{315}}说话志量高,不由老夫喜心稍。就将南阳齐围绕,

(韩众领下,\textless{}\textbf{纽丝}\textgreater{})

韩擒虎 (接唱)【西皮散板】捉拿云召转还朝。

(\textless{}\textbf{大锣打下}\textgreater{},起鼓)

{[}第五场(连场){]}

伍云召 (内)【西皮导板】恨杨广斩忠良谗臣当道,

(\textless{}\textbf{急急风}\textgreater{}韩擒虎众上站门,打上、韩上,站台中间,伍云召上城,\textless{}\textbf{四击头}\textgreater{}起唱,云弹胡子亮、哭``爹娘啊'')

伍云召
【西皮原板】叹双亲不由人珠泪双抛。手扶着垛口往下瞧,韩擒虎虽年迈杀气高。尚师徒胯下呼雷豹,麻叔谋使钢鞭稳坐在马鞍桥(或:麻叔谋使长枪鞭插在马鞍桥;或:麻叔谋打将鞭稳跨在马鞍桥)。左右先锋把帅保,耀武扬威逞英豪。搌干了(或:擦干了)泪痕伯父(哇)叫,【西皮二六】侄男有话禀年高:自古(道)臣尽忠来子当尽孝,方不愧人间走一遭。我的父忠心把国保,敲牙割舌为的是哪条?连四员虎将俱都斩了,我那年迈的娘也受那一刀。【西皮快板】到此时就该把气消了,兵困南阳为哪条?世代的忠良难话表,叫儿泪抛不泪抛。

韩擒虎
【西皮快板】贤侄休得珠泪掉,为伯言来听根苗:毁谤新主罪非小,随同为伯转回朝。

伍云召
【西皮快板】老伯父把话讲差了,侄儿言来听根苗:宇文化及行奸巧,杨广无道霸当朝。纵然将侄儿拿去了,绝了伍家后代根苗。既与我父同朝好,就该宽放路一条。伍家有朝把仇报,早烧香,晚唪经(或:晚点灯),供奉年高,饶是不饶?

韩擒虎
【西皮摇板】贤侄休把事看小,非是为伯不肯饶。新主降下旨一道,左右先行(或:二位先行)杀气高。

尚师徒 【西皮摇板】快将南阳城开了。

麻叔谋 【西皮摇板】枪对枪来刀对刀。

(伍云召 呀呸!)

伍云召
【西皮散板】伯父与我(或:我与伯父)好言告,匹夫竟敢逞英豪。叫伍保与爷城开了。

(扫一句,``叫伍保''时,伍保应)

(\textless{}\textbf{扫头}\textgreater{}下城\textless{}\textbf{急急风}\textgreater{}开打,韩败下,伍云召龙套追过场,云耍下场亮大边,下场门下\protect\hyperlink{fn316}{\textsuperscript{316}})

{[}第六场{]}

(伍保上,尚师徒、麻叔谋上,开打尚师徒、麻叔谋败下,韩擒虎上,伍保败下,伍云召上,开打韩败下,云耍下场追下\protect\hyperlink{fn317}{\textsuperscript{317}})

{[}第七场{]}

(韩擒虎众\textless{}\textbf{乱锤}\textgreater{}败上)

韩擒虎 伍云召杀法厉害,安营扎寨。

(韩擒虎众上)

{[}第八场{]}

(\textless{}\textbf{风入松}\textgreater{}四上手推粮车,宇文成都上)

宇文成都
某,宇文成都。奉了新主旨意,押运粮草,南阳关前听用,军士们,催军。

(\textless{}\textbf{风入松}合头\textgreater{},宇文成都,众下)

{[}第九场{]}

(四龙套站门,引韩擒虎上,小坐)

韩擒虎 眼观旌旗起,耳听好消息。

(内白:``无敌将军到。'')

韩擒虎 有请。

(\textless{}\textbf{吹打}\textgreater{}韩擒虎出门迎,宇文成都众上,挖门进,宇文、韩大小边八字坐)

韩擒虎 不知贤侄驾到未曾远迎,当面恕罪。

宇文成都 岂敢。小侄来的鲁莽,伯父海涵。

韩擒虎 岂敢。

宇文成都 可曾与那贼会过阵来?

韩擒虎 会过一阵,大败而回。

宇文成都 待侄会他。

韩擒虎 须要小心。

(韩擒虎下)

宇文成都 众将官,杀。

(宇文成都脱蟒,拿镋,众引下)

{[}第十场{]}

(伍云召、伍夫人打上,夫人抱喜神)

伍云召 (念)父母冤仇恨,

伍夫人 (念)时刻挂在心(或:常挂一片心)。

(八字坐,伍保报上站小边)

伍保 报,宇文成都讨战。

伍云召 (再探!)不,不\ldots{}\ldots{}好了!

伍云召
【西皮散板】宇文成都领兵到,娃娃的武艺比我高,眼见得冤仇不能报,爹娘呀!

(伍保拿枪)

伍保 请老爷上马。

(伍云召脱开氅,提枪上马,伍保下,伍云召回身收腿)

伍云召 (接唱)【西皮散板】老天爷助我成功劳。

(伍云召下)

伍夫人 【西皮散板】一见老爷跨金镫,倒教奴家挂在心。

(夫人下)

{[}第十一场{]}

(二龙出水,伍云召众大边、宇文成都众小边,伍、宇文一、二过合,搕开)

宇文成都 【西皮散板】一见云召心好恼,

(宇文成都打伍云召蓬头,伍保挑出架开)

宇文成都
(接唱)【西皮散板】骂声无知小儿曹。既知某家领兵到,就该一同转还回朝。

伍云召 (接唱)【西皮散板】你父不该行奸巧。

宇文成都 (接唱)【西皮散板】你父不该骂当朝。

伍云召 (接唱)【西皮散板】任凭尔是天神到,

(扫一句。剜萝卜,钻烟筒,开打伍云召败下,宇文成都追下\protect\hyperlink{fn318}{\textsuperscript{318}})

{[}第十二场{]}

(伍夫人上)

伍夫人 【西皮散板】老爷出兵去会阵,不知胜负与输赢。

(伍保上)

伍保 老爷回府。

(伍云召上亮住下马,与夫人推磨进门坐中间,昏)

伍夫人 老爷醒来。

伍云召 【西皮导板】这一阵杀得我昏迷了,(白)看枪!(立)

伍夫人 喂呀!

(伍云召 唉呀!)

伍云召
【西皮散板】不由本帅心内焦。回头忙把夫人叫(或:便把夫人叫),放我父子把命逃。

伍夫人 (接唱)【 【西皮散板】娇儿交与伍保抱,后花园中赴阴曹。

(伍保接喜神,伍夫人碰死下)

伍保 夫人自尽。

伍云召
【西皮散板】一见夫人命丧了(或:自尽了),不由本帅(或:怎不教人)泪双抛。叫伍保(将)尸首掩埋了。\textless{}\textbf{扫头}\textgreater{}

(伍保交喜神给伍云召,伍云召后场面里做束子介,伍保埋尸,带马,伍云召上马,向上场门,伍保下,宇文成都上,漫云头,伍云召过小边,勾伍云召到大边,里盖,打云蓬头,保挑开,伍云召下,宇文、保开打,打死保,宇文成都追下)

{[}第十三场{]}

(朱灿\textless{}\textbf{五击头}\textgreater{}上,站中间)

朱灿
一载干戈动,十载不太平。某朱灿。只因隋帝无道,隐居家园,今日闲暇无事不免闲游一回。(鼓架子)呀,那旁人马呐喊,待我登高一望。

(朱灿上桌望,伍云召、宇文成都龙套上,伍云召双望门,先上后下,平端枪,龙套追下)

朱灿
前面走的伍云召,后面追赶宇文成都,我想一个人怕一个人也就是了,为何苦苦追赶?待我上前打一个抱不平,又恐不是他人对手。(钟声)昔日王员外修造关帝庙宇,我不免去至庙中,借那周仓老爷大刀盔铠,前去搭救,吓死这些亡八东西。(下)

{[}第十四、五场(连场){]}

(龙套、伍云召上望门,先下再上,打马捋枪下,众再追下。朱灿下场着盔铠拿刀上椅站大边台口,伍云召\textless{}\textbf{水底鱼}\textgreater{}上,台口回头望上场门起\textless{}\textbf{乱锤}\textgreater{},到上场门再到小边台口见朱灿\protect\hyperlink{fn319}{\textsuperscript{319}},朱灿招手,伍云召下马拉马蹉步过去,站朱灿身后下马,宇文成都众上跟朱灿架住)

宇文成都 何人挡住某家去路?

朱灿 吾神周仓。

宇文成都 哎呀!

(宇文成都众下。朱灿下椅,伍云召抱喜神,朱灿、伍大边推磨,一二枕,伍大边、朱灿小边立)

(伍云召 多谢贤弟搭救。)

朱灿 仁兄为何这等模样?

伍云召
只因(或:可恨)杨广无道,将我父敲牙割舌而亡,愚兄逃出(或:杀出)南阳,不是贤弟搭救,险遭不测。

朱灿 仁兄意欲何往?

伍云召 愚兄意欲往雄阔海那里搬兵报仇。只是娇儿无人抚养。

朱灿 待小弟抱回家中抚养,日后父子自有相逢之日。

伍云召
如此贤弟请上,受我父子一拜。(或:如此有劳贤弟,请上受我父子一拜。)

(二人拜,喜神交朱灿,站)

伍云召
【西皮快板】幸喜贤弟遇得巧,救我父子命二条。娇儿付与贤弟抱,昼夜之间(多)受辛劳(或:要辛劳)。辞别贤弟跨虎豹,

(伍云召上马,回身)

伍云召 (接唱)【西皮散板】学一个伍子胥往吴国逃。

(转身\textless{}\textbf{叫头}\textgreater{})

伍云召 登科,我儿,(唉,)儿吓\ldots{}\ldots{}罢!

(伍云召下)

朱灿
仁兄已去,我不免送还周爷老爷大刀盔铠。(小孩哭)儿吓,不要啼哭,我给你买糕干去。

(朱灿下)

{[}第十六场(连场){]}

(\textless{}\textbf{乱锤}\textgreater{}宇文成都众上)

宇文成都
且住!正要擒那伍云召下马,周仓老爷显圣,不是俺马走如飞,险遭不测,不免回朝启奏。众将官,

(众应)

宇文成都 收兵回朝。

(\textless{}\textbf{尾声}\textgreater{}众下)

*王荣山关于此戏的把子研究:

《南阳关》是隋炀帝命韩擒虎挂帅到南阳去捉拿伍云召进京问罪。隋炀帝知道伍本领大,所以不会派无能之辈挂帅,韩擒虎不是没本事的人。但是戏中表出韩擒虎同情伍家冤枉,不愿意拿伍云召,与伍交战比画几下,希望伍逃走,他就算追赶不上收兵交差。可是伍不理解韩的心情,一心要报父仇,不顾一切奋勇追击,直到宇文成都到来才被迫逃亡。照这样的戏情来安排,伍与韩的开打就不能多,与宇文更不能多打,因为宇文非常厉害。与韩开打是二人都使枪,头场开城会阵\ldots{}\ldots{},二场打是追韩上\ldots{}\ldots{}这些都是表示韩擒虎节节退让,伍云召则不明此理,奋勇追杀,头场打还来个龙套追过场作衬托。后半出韩擒虎不露面,由宇文成都追伍云召,让朱灿装神把宇文蒙回来传令收兵,伍云召逃走,追赶不上的责任落在宇文身上,既合乎戏情,又满足观众愿望。

\textbf{陈超老师介绍:}

《南阳关》把子中甩发的使用是特色:

开打中有三下甩发。

见宇文成都也有三下甩发。

\textbf{陈超老师按:}

王凤卿有三出戏绝不穿马褂,《南阳关》、《战樊城》、《穆天王》,这三出现在都穿马褂了。

《南阳关》、《战樊城》余叔岩、王荣山、王又宸都不穿马褂,杨宝森学余也不穿。

\newpage
\hypertarget{ux6253ux767bux5dde-ux4e4b-ux79e6ux743c}{%
\subsection{打登州 之
秦琼}\label{ux6253ux767bux5dde-ux4e4b-ux79e6ux743c}}

{[}第一场{]}

【西皮导板】无情铁索困蛟龙,

【西皮原板】一腔怒气贯长虹。俺平生交友义气重,侠肠义胆论英雄。靠山王令出山岳动,历城县内捉拿秦琼。舍不得老娘【转西皮快板】无人侍奉,舍不得妻和子泪洒前胸。舍不得亲眷们同衙伙众,实难舍邻居们仁义宾朋。

【西皮摇板】前思后想心酸痛,可叹我闯荡江湖十数春有始无终。

【西皮散板】耳边厢又听得悲声大放,

呀!

【西皮散板】抬头只见儿的娘。不想灾祸从天降,此去恐难转还乡。

【西皮散板】老娘亲休把儿盼望,全当是未生儿一场。

哎呀,母亲呐!靠山王捉拿孩儿,定是为了大反山东之故。儿今此去,吉凶不保。望母亲静养身体,休要挂念你这苦命的------唉,孩儿啊\ldots{}\ldots{}(哭介)

哎呀,妻呀!事已至此,我纵有千言万语,一时焉能说得尽?我今此去,只恐有死无生,望你在母亲面前多多孝敬,管教我儿,长大成人也好接续香烟。倘得生还,一家还有相逢之日,母亲请上,孩儿就此叩别了!

【西皮散板】含悲忍泪拜慈亲,

(秦母 【西皮散板】\ldots{}\ldots{}顷刻两离分。)

(贾氏 【西皮散板】\ldots{}\ldots{}泪淋淋,)

【西皮散板】你休得要恨天怨地泪淋淋。

\textless{}\textbf{哭头}\textgreater{}老娘亲呐,

(秦母 我儿。)

\textless{}\textbf{哭头}\textgreater{}受苦的妻啊,啊,

\textless{}\textbf{哭头}\textgreater{}儿的娘哪!

{[}第二场{]}

【西皮摇板】历城县内上了杻,儿行千里母担忧。眼观日落西山后,望求差爷把店投。

{[}第三场{]}

这\ldots{}\ldots{}

病倒有,难道你会医治?

但不知何药为引?

记下了。

想俺秦琼,英名盖世,不想夜宿三家店中,受此苦刑,好不伤感人也\ldots{}\ldots{}(哭介)

【二黄三眼】在店中吊上杆威风难展,龙困在无水沙滩难把身翻。良马渴思饮长江水,人到了难中仗金兰。魏大哥、徐三弟难得见面,咬金、俊达,金甲、童环。鲁明星、鲁明月,伯当、国远,燕山罗成相见难。眼前若有那罗成士信,

【二黄散板】一定要搭救我免受熬煎。

【二黄散板】差官不必将我问,我与罗成姑表亲。

【二黄散板】问声差官是何人。

【二黄散板】不该不该大不该,不该将兄吊起来。

【二黄散板】你不知来我不怪,弟兄对坐叙开怀。

我若逃走,岂不连累于你?

哦,里八个,外八个,开门看看是哪个?

正是愚兄。

走不得。

现有公差在此。

杀不得。

他不是外人呐,乃是一门内亲。

这倒使得。

罗贤弟见过史贤弟。

这做什么?

呵,胆量是好的。

是好的。

我若逃走,一来连累罗表弟;二来家中还有老母、妻室,如何走得?

我两膀疼痛,难以提笔。

何人下书?

好,有劳二位贤弟!

【西皮导板】三家店内把计定,

这做什么?

胆量是好的。

【西皮散板】好似龙虎会风云。上写秦琼把首顿,

这做什么?

呃,你自己揉上一揉也就好了。

【西皮散板】拜上同盟结义人。八月十五登州进,

待我看来。

外面无人,乃是灯光照得你我弟兄的人影。

【西皮散板】搭救愚兄上山林。一封书信修齐整,

贤弟呀!

【西皮散板】有劳贤弟走一程。

杀不得。

方才言过,他乃是一门内亲。

且慢,惊动店家,只恐走不成了。

愚兄全知。

啊贤弟,史贤弟在瓦岗寨上不算第一也算第二。

真乃英雄也。

【西皮散板】待等到中秋节登州放火,

(罗周 【西皮散板】那时节做一个里应外合。)

{[}第四场{]}

参见千岁。

正是。

千岁将小人拿来,敢莫是为了长叶岭之事(或:原来为的是大反山东之事)?

(杨林 着,着,着!)

【西皮散板】长叶岭谁是谁不是?

【西皮散板】千层浪里翻身滚,百尺高竿又复生。

哎呀贤弟呀!适才教那老贼一棒将兄打死,也免得贤弟你挂心呐。

噤声!

贤弟呀!

【西皮散板】蛟龙正在沙滩困,切盼春雷响一声。待等八月中秋到,

(罗周 【西皮散板】再与老贼定输赢。)

{[}第五场{]}

【西皮散板】长叶岭前一着错\protect\hyperlink{fn320}{\textsuperscript{320}},事到头来无奈何。

正合我意。

就烦贤弟先行,打探瓦岗兄弟的消息便了。

【二黄导板】登州城闷坏了秦叔宝,

【回龙】走过来、行过去大街之上腕带杻锁倒教我好不心焦。

【二黄原板】十数载马上威风浩,英雄四路\protect\hyperlink{fn321}{\textsuperscript{321}}美名标。到如今屋漏偏遭连阴雨,船到江心失了篙。

那旁来了敢是茂\ldots{}\ldots{}

我是你二哥秦琼。

贤弟,莫非救我来了?

【二黄原板】徐茂公阴阳算得好,何不救我出笼牢。是是是来明白了,内中定有巧计高。

来的敢是尤\ldots{}\ldots{}

我是你二哥秦琼。

贤弟,敢是救我来了?

【二黄原板】尤俊达反山东劫过宝,官兵拿获难脱逃。若不亏我秦叔宝,何人放他往外逃。

那旁来的敢是雄信?

我是你二哥秦琼。

莫非念在结义之情前来救我来了?

【二黄原板】单雄信他本是江洋盗,枣阳山前逞英豪。被我一锏来打倒,我也曾饶过他性命一条。

来的敢是表弟?

我是你二哥秦琼。

想是贤弟念在姑表之亲救我来了?

啊?!

【二黄散板】罗成是我亲姑表,不认我秦琼为哪条?越思越想心头恼。

(罗周 【二黄散板】二哥为何心内焦?)

哎呀贤弟呀,适才众家兄弟俱不相认愚兄,如何是好?

有劳了!

【二黄散板】罗周待我情义好,他与秦琼似同胞。

【二黄散板】站在阳关心焦躁,

啊?!

【二黄散板】来了咬金故旧交。

来的敢是程\ldots{}\ldots{}

咬金!

我是二哥秦琼。

你我幼年相交,想是救我来了。

啊?!

【二黄散板】好个聪明的程咬金,他把救字暗藏身。急难之人心不稳。

打探瓦岗弟兄之事如何?

好哇------

【二黄散板】待等明日中秋到。

(罗周 【二黄散板】再与老贼动枪刀。)

{[}第六场{]}

参见千岁。

(杨林 为何这等白胖?)

(罗周 这个\ldots{}\ldots{})

(秦琼 海风吹得浮肿。)\protect\hyperlink{fn322}{\textsuperscript{322}}

不足千岁一观。

遵令。

【西皮二六】老杨林校场令传下,不由得秦琼怒气发。既然他疑心我通响马,到此为何不把我来杀。今日里校场试锏法,又恐其中事有差。我今既在矮檐下,

【西皮快板】生死二字何惧他。约定了瓦岗弟兄把山下,搭救秦琼把老贼拿。

【西皮快板】减头去尾耍一耍。

(秦琼耍锏\protect\hyperlink{fn323}{\textsuperscript{323}})

(杨林 不足为奇。)

小人马上武艺还好。

小人不敢!

【西皮摇板】鹞子翻身上黄骠,

【西皮快板】来了瓦岗众英豪。斜跨雕鞍高声叫,

杨林!老贼!

【西皮摇板】敢与老爷动枪刀。

多谢众位贤弟搭救。

(请呐!)

\newpage
\hypertarget{ux65adux5bc6ux6da7-ux4e4b-ux738bux4f2fux5f53}{%
\subsection{断密涧 之
王伯当}\label{ux65adux5bc6ux6da7-ux4e4b-ux738bux4f2fux5f53}}

\textbf{{[}第一场{]}}

\textbf{(念)习就百步箭穿杨,一片丹心扶瓦岗。}

\textbf{俺王勇}\protect\hyperlink{fn324}{\textsuperscript{324}}\textbf{,奉了大王之命,追赶徐勣、魏徵,赶至三岔路口,念在贾家楼结义之情,将他二人释放。不免回山懵懂启奏。}

\textbf{众好汉,回山!}

\textbf{{[}第二场{]}}

\textbf{(念)留得五湖明月在,何愁无处下金钩。}

\textbf{报------王勇告呃进!}

\textbf{参见大王,伯当交令。}

\textbf{谢座。}

\textbf{他二人去远,追赶不上,特地回山启奏。}

\textbf{大王------}

\textbf{【西皮原板】大王说话太痴迷,细听伯当把话提:三十六人曾结义,生死相交永不离。二劫皇杠把祸起,大反山东【转西皮二六】赶惹是非。王伯当呃单人【转西皮快板】赶唐璧,秦叔宝匹马取}金堤\protect\hyperlink{fn325}{\textsuperscript{325}}\textbf{。夜夺瓦岗非容易,三月三日拜帅旗。那程咬金有福登龙位,兵多将广人马齐。自从大王到此地,锦绣江山化灰泥。恨飞鼠盗去仓粮米,满山喽罗似雀飞。众家弟兄散了队,一个东来一个西。只剩下王勇来保你,反说为臣把君欺。倘若哪国刀兵起,祸到临头悔不及。}

\textbf{(李密 【西皮快板】\ldots{}\ldots{},莫做三心二意呃的。)}

\textbf{【西皮快板】说什么三心共二意,为臣怎敢把君欺。王勇保主有假意,气化清风肉化泥。}

\textbf{(李密 好哇!)}

\textbf{(李密
【西皮快板】好一个忠良王贤弟,亚赛过当年介子推。孤王若把良心昧,乱箭攒身不回归。)}

\textbf{(李密 \ldots{}\ldots{},必须重整瓦岗才是。)}

\textbf{瓦岗已散,难以重整。}

\textbf{(李密 这便如何是好?)}

\textbf{臣保定大王前去降唐。}

\textbf{(李密 \ldots{}\ldots{},只是那南牢之事?)}

\textbf{敢保大王无事。}

\textbf{且慢,这样不能前往。}

\textbf{必须换了亵衣小帽,方可前往。}

\textbf{事到如今,舍不得,也要舍!}

\textbf{改换。}

\textbf{(李密 \ldots{}\ldots{},传令。)}

\textbf{领旨。}

\textbf{下面听者:大王前去降唐,愿随者随,不愿随者,各自散去。}

\textbf{(李密 贤弟,与孤------带呃马!)}

\textbf{臣------领呐------旨。}

\textbf{(李密 【西皮快板】\ldots{}\ldots{}被犬欺。)}

\textbf{【西皮快板】大王不必长叹息,伯当言来听端的:阿房宫,今何在,铜雀楼台化灰泥。江山自古有兴废,男儿能伸又能屈。改邪归正投唐去,青史名标万古题。}

\textbf{(李密 【西皮快板】怕只怕唐童把仇记,笼中之鸟也难飞。)}

\textbf{【西皮摇板】杀身大祸臣愿替,愿保大王挂紫衣。}

\textbf{{[}第三场{]}}

\textbf{(李密 【西皮散板】昔日螳螂去捕蝉,)}

\textbf{【西皮散板】偶遇黄雀把路拦。}

\textbf{(李密 【西皮散板】黄雀又被金弹打,)}

\textbf{【西皮散板】打弹之人被虎餐。}

\textbf{(李密 【西皮散板】猛虎落在陷阱内,)}

\textbf{【西皮散板】仇报仇来冤报冤。}

\textbf{(李密 【西皮散板】勒住丝缰用目看,)}

\textbf{【西皮散板】只见箭雁落马前。}

\textbf{``西府秦王,百发百中。''}

\textbf{乃是箭雁。}

\textbf{(李密 待孤看来。)}

\textbf{大王请看。}

\textbf{且慢,有了箭雁,就有了进身}\protect\hyperlink{fn326}{\textsuperscript{326}}\textbf{之策。}

\textbf{待等唐童到此,将箭雁献上,岂不是进身之策?}

\textbf{啊------敢保大王无事。}

\textbf{松林等候。}

\textbf{请呐------}

\textbf{(李密 【西皮摇板】\ldots{}\ldots{}雕翎箭,)}

\textbf{【西皮摇板】狭路相逢天凑缘。}

\textbf{(众 \ldots{}\ldots{}挡道!)}

\textbf{王勇!}

\textbf{参见千岁!}

\textbf{(李世民 来此则甚?)}

\textbf{为臣拾得箭雁,特来献上。}

\textbf{(李世民 到此何事?)}

\textbf{臣保定一家,前来降唐。}

\textbf{(李世民 但不知哪一家?)}

\textbf{就是那西魏王李密。}

\textbf{(李世民 哦呵,请来相见。)}

\textbf{犹恐千岁记起南牢仇恨。}

\textbf{多谢千岁。}

\textbf{松林等候。}

\textbf{啊千岁,须要言而有信。}

\textbf{谢千岁。}

\textbf{【西皮快板】好一个仁义二主君,他比尧、舜强十分。松林忙把大王请,}

\textbf{(李密 【}西皮摇板\textbf{】\ldots{}\ldots{}不安宁。)}

\textbf{也曾会过。}

\textbf{请大王前去相见。}

\textbf{一概不究。}

\textbf{吾王且慢,见了千岁怎生行礼?}

\textbf{他乃一君,你乃一臣,必须下一全礼。}

\textbf{你前来做甚?}

\textbf{却又来!}

\textbf{【}西皮快板\textbf{】说什么瓦岗你为君,说什么低头不拜人。上前去施一个君臣礼,我保你头戴乌纱入朝门。}

\textbf{(李密 【}西皮摇板\textbf{】人言唐童似尧、舜,)}

\textbf{【}西皮摇板\textbf{】话不虚传果是真。}

\textbf{(李密 【}西皮摇板\textbf{】降唐事儿呃心拿稳呐,)}

\textbf{【}西皮摇板\textbf{】似狂风吹散了满天云呐。}

\textbf{{[}第四场{]}}

\textbf{(李密 (念)低头入朝门,)}

\textbf{(念)叩见圣明君。}

\textbf{(李密 臣李密,)}

\textbf{王勇,}

\textbf{(李密、王勇 见驾,万岁。)}

\textbf{臣有本启奏。}

\textbf{谢万岁!}

\textbf{{[}第五场{]}}

\textbf{正是:(念)王勇生来秉性刚,一臣不保二君王。}

\textbf{【西皮摇板】蟒袍玉带我不爱,一片丹心揣在怀。}

\textbf{{[}第六场{]}}

\textbf{【西皮摇板】将身来在宫墙外,大王慌张为何来。}

\textbf{大王慌慌张张,为了何事?}

\textbf{我却不信。}

\textbf{哎呀!}

\textbf{【西皮}摇板\textbf{】一见公主倒尘埃,怎不教人痛伤怀。}

\textbf{【西皮}摇板\textbf{】河阳公主今何在,}

\textbf{(李密 呵哈哈哈\ldots{}\ldots{}(笑介))}

\textbf{呀呸!}

\textbf{【西皮}摇板\textbf{】忘恩负义怎安排。}

\textbf{似你这样忘恩负义,谁来救你?}

\textbf{大王请起。}

\textbf{征剿山东。}

\textbf{吏部旨意可在身旁?}

\textbf{你我趁此黑夜,诈出皇城,再作道理。}

\textbf{走哇!}

\textbf{{[}第七场{]}}

\textbf{【西皮摇板】摇头摆尾再不来。}

\textbf{(李密 【西皮摇板】只身逃出天罗网,)}

\textbf{【西皮摇板】翻身跳出是非墙。}

\textbf{你我去往山后刘武周那里安身便了。}

\textbf{唐童追兵甚急,倘若赶上,那还了得。}

\textbf{又有你!}

\textbf{俺王勇的性命,断送------你手!}

\textbf{(李密 呵呵哈哈哈\ldots{}\ldots{}(笑介))}

\textbf{(李密 【西皮原板】\ldots{}\ldots{}}带惆怅?\textbf{)}

\textbf{【西皮原板】你杀那河阳公主因何故,忘恩负义所为哪桩。}

\textbf{【西皮快板】闻言怒发三千丈,太阳头上冒火光。可叹三十六员将,东逃西奔各一方。单单剩下王伯当,大胆保你来降唐。唐王待你恩德广,河阳公主招东床。谋朝篡位心妄想,顺者昌来逆者亡。}

\textbf{(李密 【西皮快板】昔日韩信谋家邦,)}

\textbf{【西皮快板】未央宫中一命亡。}

\textbf{(李密 【西皮快板】毒死平帝是王莽,)}

\textbf{【西皮快板】千刀万剐无下场。}

\textbf{(李密 【西皮快板】曹丕也曾把中原掌,)}

\textbf{【西皮快板】留得骂名天下扬。}

\textbf{(李密 【西皮快板】李渊也是臣谋主,)}

\textbf{【西皮快板】他本是真龙下天堂。}

\textbf{(李密 【西皮快板】说什么\ldots{}\ldots{},封你一字并肩王。)}

\textbf{【西皮快板】说什么一字并肩王,羞得王勇脸无光。你好比人心不足蛇吞象,你好比困龙思想上天堂。手摸胸膛想一想,你是人面兽心肠。}

\textbf{(李密
【西皮快板】\ldots{}\ldots{}君臣一路好商量,李密打马朝前闯呃,)}

\textbf{【西皮散板】伯当错保无义王。}

\textbf{{[}第八场{]}}

\textbf{【西皮摇板】大王休得心慌忙,自有王勇做主张。}

\textbf{千岁!}

\textbf{【西皮快板】稳坐雕鞍把话讲,尊声千岁听端详:王勇生来性情刚,一臣不保二君王。}

\textbf{【西皮摇板】千岁不把臣来放,情愿战死在沙场。}

\textbf{【西皮摇板】王勇前来救大王。}

\textbf{{[}第九场{]}}

\textbf{下马观看。}

\textbf{``断密涧''。}

\textbf{``断密涧''。}

\textbf{不好了!}

\newpage
\hypertarget{ux5343ux79cbux5cad}{%
\subsection{千秋岭}\label{ux5343ux79cbux5cad}}

\textbf{{[}第一场{]}}

\textbf{(罗成上,起霸)}

\textbf{罗成
(念)束发金盔显少年,气吐虹霓贯九天。银枪摆动龙戏水,战马------}

\textbf{(四下手上)}

\textbf{罗成 (念)驰驱似火焰。}

\textbf{罗成
俺、罗成。今在洛阳王世充帐下为将。闻听唐营兵马到来,岂肯容他张狂}\protect\hyperlink{fn327}{\textsuperscript{327}}\textbf{。}

\textbf{罗成 众将官!}

\textbf{众 有!}

\textbf{罗成 起兵前往!}

\textbf{众 啊!}

\textbf{(四下手带马同下)}

\textbf{{[}第二场{]}}

\textbf{(秦琼、尉迟恭上,【打上】)}

\textbf{秦琼 (念)劈抡锏盖世无双,}

\textbf{尉迟恭 (念)水磨鞭保定唐王。}

\textbf{(秦琼大边、尉迟恭小边)}

\textbf{秦琼
某、护国公}\protect\hyperlink{fn328}{\textsuperscript{328}}\textbf{秦琼。}

\textbf{尉迟恭 鄂国公敬德。}

\textbf{秦琼 请了!}

\textbf{尉迟恭 请了!}

\textbf{秦琼 二主升帐,两厢伺候!}

\textbf{尉迟恭 请!}

\textbf{(四文堂引徐勣、李世民上)}

\textbf{李世民
\textless{}点绛唇\textgreater{}\ldots{}\ldots{}儿郎虎豹,军威浩,地动山摇,要把狼烟扫。}

\textbf{(李世民归大座)}

\textbf{众人 参见主公。}

\textbf{李世民 众卿少礼。}

\textbf{李世民 (念)奉了父王令,征战洛阳城。虽然我为主,全仗众公卿。}

\textbf{李世民
小王李世民。奉了父王之命,征战洛阳王世充,可恨他战又不战,降又不降。也曾命探马前去打探,未见回报。}

\textbf{(报子上)}

\textbf{报子 罗成讨战。}

\textbf{李世民 再探!}

\textbf{报子 啊。}

\textbf{(报子下)}

\textbf{李世民 先生!}

\textbf{徐勣 主公!}

\textbf{李世民 罗成讨战,命何将出马?}

\textbf{徐勣 命马三保带领本部人马,攻打头阵。}

\textbf{李世民 先生传令。}

\textbf{徐勣 得令。}

\textbf{徐勣 下面听者!二主有令:命马三保带领本部人马攻打头阵!}

\textbf{马三保 啊!}

\textbf{徐勣 主公请至后帐,且听好音便了。}

\textbf{(众同下)}

\textbf{{[}第三场{]}}

\textbf{(马三保带四龙套上,\textless{}四边静\textgreater{}头段)}

\textbf{马三保 某、马三保。奉命攻打头阵。}

\textbf{马三保 众将官,杀!}

\textbf{(\textless{}急急风\textgreater{},罗成、四下手同上,会阵)}

\textbf{罗成 马三保,你乃久败之将,擅敢起兵到此!}

\textbf{马三保 一派胡言。放马过来!}

\textbf{(马三保败下,罗成追下)}

\textbf{{[}第四场{]}}

\textbf{(\textless{}四边静合头\textgreater{}四文堂、尉迟恭、秦琼、徐勣、李世民上,李世民归大座)}

\textbf{(报子上)}

\textbf{报子 马三保败阵!}

\textbf{秦琼 再探!}

\textbf{报子 啊!}

\textbf{(报子下)}

\textbf{李世民 先生,马三保败阵,当命何将出马?}

\textbf{徐勣 待臣思忖回话。}

\textbf{尉迟恭 嗯------}

\textbf{徐勣
呜哙呀!看此黑贼耀武扬威,我不免在二主面前搬动是非,教他出马打一败仗,也好灭灭他的火性。}

\textbf{徐勣
臣启主公:罗成骁勇,我营将士,无人对敌,将营盘暂退四十里,打本进京,奏与老王,遣来能将大战罗成!}

\textbf{李世民 先生传令!}

\textbf{徐勣 得令。}

\textbf{徐勣 令出\ldots{}\ldots{}}

\textbf{尉迟恭 且慢呐!}

\textbf{徐勣 尉迟恭为何阻令?}

\textbf{尉迟恭
先生!你道那罗成天上少有,地下难寻,眼前若有二主将令,某要生擒那罗成进帐呃。}

\textbf{徐勣 你若擒得住罗成,山人愿将军师大印,付你执掌。你呢?}

\textbf{尉迟恭 我若擒不住罗成,愿输项上的人头。}

\textbf{徐勣 空口无凭,敢与山人击掌?}

\textbf{尉迟恭 击掌?请啊!}

\textbf{徐勣 尉迟恭听令!}

\textbf{尉迟恭 在。}

\textbf{徐勣 命你带领本部人马,大战罗成!}

\textbf{尉迟恭 得令!}

\textbf{李世民 皇兄须要小心。}

\textbf{尉迟恭 二主!}

\textbf{尉迟恭
【西皮摇板】二主不必太小量,强中自有强中强。辞别二主出宝帐,}

\textbf{(尉迟恭过大边,秦琼到小边)}

\textbf{秦琼 哪里去?}

\textbf{尉迟恭 大战罗成!}

\textbf{秦琼 你不能得胜!}

\textbf{尉迟恭 你怎样知晓?}

\textbf{秦琼 他与我乃是姑表之亲。}

\textbf{尉迟恭 你们俱是一党!}

\textbf{秦琼 哼!}

\textbf{尉迟恭 【西皮摇板】战鼓嗵嗵下校场。}

\textbf{(尉迟恭下)}

\textbf{秦琼
【西皮摇板】一见尉迟出唐营,回头埋怨徐先生。明知罗成多骁勇,不该命他去出征。先生传令某出马,}

\textbf{徐勣 你也不能得胜!}

\textbf{秦琼 【西皮摇板】不得胜落一个两太平。}

\textbf{徐勣
【西皮摇板】二哥有所不知情,小弟言来听分明。自从黑贼进唐营,屡屡欺压弟兄们。教他出兵打败阵,看他逞能不逞能。}

\textbf{秦琼
【西皮摇板】好个仁义徐先生,把我弟兄看得清。辞别二主出唐营,}

\textbf{秦琼 【西皮摇板】且听探马报分明。}

\textbf{(秦琼下)}

\textbf{李世民 【西皮摇板】带过御马跨金镫,}

\textbf{(李世民上马)}

\textbf{李世民 【西皮摇板】去到阵前看分明。}

\textbf{(众人同下)}

\textbf{{[}第五场{]}}

\textbf{(四文堂、尉迟恭上)}

\textbf{尉迟恭
【西皮摇板】胯下}\protect\hyperlink{fn329}{\textsuperscript{329}}\textbf{一骑乌骓马,打将钢鞭手中拿。三军与爷催战马,大小将官听根芽。别的将官休出马,单要罗成小娃娃。}

\textbf{(尉迟恭站大边,里边)}

\textbf{四下手 (内)唐将骂阵!}

\textbf{罗成 (内)【西皮摇板】正在后帐习兵法,}

\textbf{(四下手引罗成上,站小边)}

\textbf{罗成
【西皮摇板】听得唐将叫骂咱。三军带过爷的马,管教敌将染黄沙。 }

\textbf{(尉迟恭、罗成驾住)}

\textbf{尉迟恭 来将通名!}

\textbf{罗成 少爷罗成!}

\textbf{尉迟恭 哈哈!哈哈!啊,呵哈哈\ldots{}\ldots{}(笑介)}

\textbf{罗成 黑贼为何发笑?}

\textbf{尉迟恭
罗成呐,孺子!我道你,天上少有,地下难寻;原来是个黄毛的孺子,怎当某家一战?}

\textbf{尉迟恭 三军的,报与二主,上某家大大的头功。}

\textbf{罗成 黑贼到此,一战未交,为何上尔大大头功?}

\textbf{尉迟恭 娃娃,慢说交战,提起你老爷昔年的威风,吓破尔的苦胆!}

\textbf{罗成 你要讲啊!}

\textbf{尉迟恭 你要听呐!}

\textbf{尉迟恭
【西皮快板】勒住马头慢交战,细听老爷表家园:家住山西麻邑县,聚贤村内有家园。日抢三关夺八寨,杀得唐将心胆寒。慢说与爷来交战,提起了威风吓尔还。}

\textbf{罗成
【西皮摇板】忽听黑贼表家园,吓得少爷心胆寒。不战黑贼走了罢,}

\textbf{尉迟恭 敢是怯战?}

\textbf{罗成 呸!}

\textbf{罗成
【西皮快板】细听少爷表家园:我七岁气吹檐前瓦,八九学艺在燕山。十岁姑表枪换锏,十一岁结拜在济南。十二山东放响马,十三岁威名天下传。十四夜打登州府,十五扬州夺状元。十六岁景阳曾打虎,十七染病在山前。少爷今年十八岁,}

\textbf{尉迟恭 十八岁你该死了!}

\textbf{罗成 【西皮摇板】杀得唐将丧黄泉。}

\textbf{(尉迟恭、罗成俩人一兜,合扇下)}

\textbf{{[}第六场{]}}

\textbf{(四龙套引徐勣、李世民同上)}

\textbf{李世民 【西皮摇板】君臣打马出唐营,}

\textbf{徐勣 【西皮摇板】观看两家动刀兵。}

\textbf{李世民 【西皮摇板】下得马来高坡近,}

\textbf{(李世民、徐勣下马)}

\textbf{徐勣}\protect\hyperlink{fn330}{\textsuperscript{330}}
\textbf{【西皮摇板】看是谁胜哪家赢。}

\textbf{(罗成先上,尉迟恭上。罗成把尉迟恭勾过去,把尉迟恭打败,打下)}

\textbf{李世民 【西皮摇板】尉迟好比南山豹,}

\textbf{徐勣 【西皮摇板】罗成好比浪里蛟。}

\textbf{尉迟恭 (内)【西皮导板】越杀越勇罗士信,}

\textbf{(尉迟恭上)}

\textbf{尉迟恭 哎呀!}

\textbf{尉迟恭
【西皮摇板】杀得某家少精神。只杀得襆头戴不稳,乌骓马倒退不前行。背地只把二主怨,回头埋怨徐先生。明知罗成多骁勇,不该命某来出兵。}

\textbf{尉迟恭 【西皮摇板】慈悲大士救八难,缘何不救某难中人。}

\textbf{尉迟恭 罢!}

\textbf{尉迟恭 【西皮摇板】不顾生死将他战,}

\textbf{(罗成上,压住)}

\textbf{罗成 哼,哼,哼\ldots{}\ldots{}(冷笑介)}

\textbf{尉迟恭 【西皮摇板】倒被娃娃笑一声。}

\textbf{尉迟恭 娃娃!}

\textbf{尉迟恭 【西皮摇板】你若马前来归顺,老爷收儿做螟蛉。}

\textbf{(剜萝卜,完了一躲,尉迟恭败下)}

\textbf{李世民 【西皮摇板】越杀越勇罗士信,}

\textbf{徐勣 【西皮摇板】战败我朝鄂国公。}

\textbf{李世民 罗成好将啊,好将!}

\textbf{徐勣 主公连夸好将,莫非有爱将之意?}

\textbf{李世民 小王爱他,他不归顺,也是枉然!}

\textbf{徐勣 待臣顺说他来降,不知二主待将如何?}

\textbf{李世民 皇兄啊!}

\textbf{李世民 【西皮摇板】只要他真心来归顺,我与他皇兄御弟称。}

\textbf{徐勣
【西皮摇板】好个仁义二主君,把我弟兄看得清。辞别主公跨金镫,顺说罗成降唐营。}

\textbf{(徐勣下桌子,下)}

\textbf{李世民 【西皮摇板】人来带过御马乘,}

\textbf{(李世民下桌子)}

\textbf{李世民 【西皮摇板】且候罗成降唐营。}

\textbf{(众人同下)}

\textbf{{[}第七场{]}}

\textbf{(尉迟恭、罗成上)}

\textbf{罗成 黑贼,敢是怯战?}

\textbf{尉迟恭 住了!非是你老爷怯战,我家元帅鸣金收兵,明日再战!}

\textbf{罗成 你少爷不收兵,一定要战!}

\textbf{尉迟恭 你不收兵?}

\textbf{罗成 不收兵!}

\textbf{尉迟恭 你不收兵?}

\textbf{罗成 不收兵!}

\textbf{尉迟恭 嘿!我收兵呃。}

\textbf{(尉迟恭下)}

\textbf{罗成 哼哼哼\ldots{}\ldots{}(冷笑介)}

\textbf{罗成
【西皮摇板】催命鼓来救命锣,阵前战败黑阎罗。豪杰打马过山坡,}

\textbf{徐勣 贤弟慢走!}

\textbf{罗成 【西皮摇板】那边来了徐三哥。}

\textbf{(徐勣上)}

\textbf{徐勣 【西皮摇板】扬鞭打马出唐营,见了贤弟说分明。}

\textbf{罗成 三哥!小弟有礼。}

\textbf{徐勣 贤弟!众家弟兄俱已降唐,惟有贤弟不降,是何道理?}

\textbf{罗成 小弟早有此心,只是无有引荐之人。}

\textbf{徐勣 愚兄愿作引荐之人。}

\textbf{罗成 不知二主,待将如何?}

\textbf{徐勣 贤弟呀!}

\textbf{徐勣
【西皮摇板】只要你真心来归顺,他与你皇兄御弟称。辞别贤弟把马乘,莫作三心二意人。}

\textbf{(徐勣上场门下)}

\textbf{罗成 【西皮摇板】好个仁义徐先生,顺说罗成降唐营。}

\textbf{罗成 众将官!}

\textbf{众 有}

\textbf{罗成 随爷降唐去者!}

\textbf{众 (内)啊!}

\textbf{(罗成上场门下)}

\textbf{{[}第八场{]}}

\textbf{(\textless{}扫头\textgreater{},四红文堂引徐勣、程咬金、秦琼、李世民上,尉迟恭在\textless{}扫头\textgreater{}中上)}

\textbf{尉迟恭
二主,某正要擒那罗成下马,被这牛鼻子老道,鸣金收兵。你要罪他,你要与我怪他!}

\textbf{程咬金 嘿,我说老黑呃!耀武扬威的,敢是得了胜了?}

\textbf{尉迟恭 这个------嘿!}

\textbf{程咬金
嘿,败了!哼,卖不了的秫秸,你那边戳戳}\protect\hyperlink{fn331}{\textsuperscript{331}}\textbf{吧。}

\textbf{李世民
【西皮摇板】一见尉迟败了阵,回头埋怨徐先生:明知罗成多骁勇,不该命他去出兵!}

\textbf{徐勣
【西皮摇板】二主休要罪微臣,把话说与尉迟听:你出兵好似一只虎,}

\textbf{尉迟恭 某家好比一只猛虎!}

\textbf{程咬金 你呀,嘿,墙上爬着的蝎拉虎子!}

\textbf{尉迟恭 猛虎!}

\textbf{程咬金 你呀,蝎拉虎子。}

\textbf{尉迟恭 猛虎!}

\textbf{程咬金 得得得,就算是猛虎。}

\textbf{徐勣 【西皮摇板】恨不得把罗成一口吞。}

\textbf{尉迟恭 着哇!某恨不得把那娃娃吞吃在腹内!}

\textbf{程咬金 你呀,诶,嘴大嗓子眼小,你咽不下去。}

\textbf{尉迟恭 吞吃在腹内!}

\textbf{程咬金 你咽不下去!}

\textbf{尉迟恭 吞吃腹内!}

\textbf{徐勣 【西皮摇板】只杀得襆头戴不稳,}

\textbf{尉迟恭 先生,那是风吹歪了的。}

\textbf{程咬金
哼,你呀,哼,那是我把弟爷给你挑歪了}\protect\hyperlink{fn332}{\textsuperscript{332}}\textbf{。}

\textbf{尉迟恭 风吹歪了的!}

\textbf{程咬金 枪挑歪了的。}

\textbf{徐勣 【西皮摇板】乌骓马倒退不前行。}

\textbf{尉迟恭 先生,那是罗成的马!}

\textbf{程咬金
哼,我那把弟呀,骑的是白龙马,你呀,骑的是那个------蛤蟆。}

\textbf{尉迟恭 罗成的马。}

\textbf{徐勣 【西皮摇板】慈悲大士救八难,缘何不救某难中人!}

\textbf{尉迟恭 先生,那是罗成讲的呀。}

\textbf{程咬金 嘿,你说的,我听见了!}

\textbf{尉迟恭 罗成讲的!}

\textbf{程咬金 你说的!}

\textbf{徐勣 【西皮摇板】不用杀来不用战,点点手儿唤罗成。}

\textbf{尉迟恭 某家不信。}

\textbf{徐勣 【西皮摇板】不信与我来击掌!}

\textbf{尉迟恭 击掌?请!}

\textbf{程咬金
我说老黑呀!你先前就打过赌,输了个脑袋了;你还打赌啊,别不害臊啦!}

\textbf{尉迟恭 嘿!}

\textbf{(尉迟恭坐大边)}

\textbf{徐勣
【西皮摇板】谅你不敢赌输赢。回头启奏二主君:有罪罗成降唐营。}

\textbf{徐勣 罗成降唐。}

\textbf{李世民 宣他进帐!}

\textbf{徐勣 二主有令:罗成进帐!}

\textbf{(罗成上)}

\textbf{罗成 【西皮摇板】催动坐骑到唐营,}

\textbf{秦琼 贤弟!}

\textbf{罗成
【西皮摇板】唐营中来会众宾朋。别的宾朋我不问,二哥是我姑表亲。}

\textbf{(秦琼出门)}

\textbf{罗成 【西皮摇板】二哥报门小弟进,}

\textbf{(秦琼报门)}

\textbf{秦琼 报,罗成告进!}

\textbf{(罗成挖进去,站小边)}

\textbf{罗成 【西皮摇板】有罪罗成降唐营。}

\textbf{李世民 【西皮摇板】只要你真心来归顺,封你为越国公永在朝门。}

\textbf{罗成 【西皮摇板】叩罢头来谢罢恩,}

\textbf{(罗成坐小边)}

\textbf{罗成 【西皮摇板】那旁坐的对头人。}

\textbf{尉迟恭 啊!}

\textbf{尉迟恭
【西皮摇板】一见罗成讨了封,不由某家怒气升。唐营中哪有你的座,}

\textbf{尉迟恭 站过去!}

\textbf{(\textless{}快长锤\textgreater{}里,尉迟恭拉罗成,换坐位,罗成归大边,尉迟恭归小边)}

\textbf{尉迟恭 【西皮摇板】这座让与俺鄂国公,哼!}

\textbf{程咬金 【西皮快板】程咬金怒不息,骂声敬德你把人欺!}

\textbf{尉迟恭 你才欺人!}

\textbf{程咬金 【西皮快板】曾记得美良川鞭对斧,你鞭鞭打在我胸窝里。}

\textbf{尉迟恭 某家我服了你了。}

\textbf{程咬金 你服我什么?}

\textbf{尉迟恭 我服你呀,好捱挨打哦。}

\textbf{程咬金 我也服了你!}

\textbf{尉迟恭 服我何来?}

\textbf{程咬金 服你好狠心哦!}

\textbf{尉迟恭 只是便宜了你!}

\textbf{程咬金 便宜了你!}

\textbf{程咬金 【西皮快板】不看罗成你看在我。}

\textbf{尉迟恭 唐营之中好大的一个你呃!}

\textbf{程咬金 哎,好大的一个你呃!}

\textbf{程咬金 【西皮快板】岂不知我们是把兄弟。}

\textbf{尉迟恭 哎,你们是猪兄狗弟!}

\textbf{程咬金 哎,龙兄虎弟。}

\textbf{尉迟恭 猪兄狗弟!}

\textbf{程咬金 我说二哥啊,把弟啊,咱们大伙打他。}

\textbf{程咬金 【西皮摇板】唐营中哪有你的座?}

\textbf{程咬金 你站站呐!你站站呐!}

\textbf{(程咬金揪尉迟恭,尉迟恭不起来)}

\textbf{程咬金
【西皮快板】你不起来我不依。扭回头搬是非,叫声二哥听端的:他不欺罗成欺的是你,你不骂他我不依。}

\textbf{秦琼
【西皮摇板】听一言来心好恼,大骂黑贼听根苗:曾记得美良川鞭对锏,三鞭两锏斗英豪。你三鞭打不动秦叔宝,俺两锏打得你望影而逃。唐营中哪有你的座?}

\textbf{秦琼 站过去!}

\textbf{(\textless{}快长锤\textgreater{}里,秦琼拉尉迟恭,尉迟恭与罗成换坐位,尉迟恭归大边,罗成归小边)}

\textbf{秦琼 【西皮摇板】这座让与我表弟英豪。}

\textbf{秦琼 贤弟,进得唐营,为何这等懦弱?}

\textbf{罗成 小弟初进唐营,不敢造次。}

\textbf{秦琼 待愚兄教导与你。}

\textbf{程咬金 是啊,二哥你教给他!}

\textbf{秦琼
俺罗成一不降唐,二不归顺;多蒙仁义徐先生,顺说俺来降,二主见爱,赐俺一个座位,又被廊下的匹夫占去。要杀抬枪,要打何惧!}

\textbf{罗成
唐营众将听者:俺罗成一不降唐,二不归顺;多蒙仁义徐先生,顺说俺来降。二主见爱,赐俺一个座位,又被廊下匹夫占去。要杀------抬枪,要打------何惧!}

\textbf{尉迟恭 呔!罗成你要打哪个?}

\textbf{罗成 我要打你。着打。}

\textbf{(罗成打尉迟恭)}

\textbf{尉迟恭 哎呀!}

\textbf{尉迟恭 【西皮摇板】唐营中俱是他猪兄狗弟,}

\textbf{程咬金 我们都是龙兄虎弟。二哥,把弟!咱们打他!打他!}

\textbf{尉迟恭 哎!}

\textbf{尉迟恭 【西皮摇板】惟有俺尉迟恭是个外来的。}

\textbf{程咬金 你呀,咳,你好比茄子地里长蒺藜,嘿!坏种独苗大紫包。}

\textbf{尉迟恭 哼!}

\textbf{尉迟恭 【西皮摇板】怒气不息打进去!}

\textbf{程咬金 干什么啊?}

\textbf{尉迟恭 要打你。}

\textbf{程咬金 讲打?一个打一个,哼,不算好朋友,二哥,把弟,打他!打他!}

\textbf{尉迟恭
【西皮摇板】险些逼起一窝蜂。豪杰低头进大营,把话说与二主听。}

\textbf{尉迟恭 二主!当初御果园有难,何臣保驾?}

\textbf{李世民 小王失记。}

\textbf{尉迟恭 为臣全记。}

\textbf{李世民 奏来!}

\textbf{尉迟恭 容奏呃!}

\textbf{尉迟恭
【西皮摇板】御果园中曾有难,口口声声叫尉迟皇兄。有了罗成忘了臣,有了新臣忘旧臣。怒气不息出大营,把话说与众三军。}

\textbf{尉迟恭 三军的!降唐人马,随爷反出大营!}

\textbf{(尉迟恭又坐大边,向下场门叫三军)}

\textbf{李世民 呀!}

\textbf{李世民
【西皮摇板】一见尉迟反唐营,倒教小王吃一惊。走向前来忙跪定,}

\textbf{(李世民跪在尉迟恭座位前,众齐跪)}

\textbf{李世民 【西皮摇板】俱是皇兄御弟称。}

\textbf{(众搀扶李世民起,众起)}

\textbf{众人 臣等和睦!}

\textbf{李世民 后帐摆宴,与众卿贺功!}

\textbf{众人 谢主公!}

\textbf{(\textless{}尾声\textgreater{}。众人同下)}

\textbf{(罗成、尉迟恭坐左右``虎头椅'')}

\newpage
\hypertarget{ux5babux95e8ux5e26ux5341ux9053ux672c-ux4e4b-ux674eux6e0aux891aux9042ux826f}{%
\subsection{宫门带·十道本 之
李渊、褚遂良}\label{ux5babux95e8ux5e26ux5341ux9053ux672c-ux4e4b-ux674eux6e0aux891aux9042ux826f}}

\textbf{{[}第一场{]}}

\textbf{李渊
【二黄慢板】都只为御梓童命归仙境,因此上为王的染病在身。内侍臣搀扶王龙床安定,还需要却烦虑静养精神。}

\textbf{(李世民 【二黄摇板】内侍摆驾进龙廷,父王台前问安宁。)}

\textbf{(李世民 儿臣见驾,父王万岁!)}

\textbf{李渊 皇儿平身。}

\textbf{(李世民 万万岁!)}

\textbf{李渊 赐座。}

\textbf{(李世民 谢座。)}

\textbf{李渊 皇儿进宫为了何事?}

\textbf{(李世民 儿臣在太医院,取得太平汤药,进宫与父王熬煎。)}

\textbf{李渊 我儿真乃孝道(或:孝心)。}

\textbf{(李世民 内侍,金炉伺候!)}

\textbf{(李世民
【二黄原板】父王将息龙床养,儿臣进宫煎药汤。屈膝跪在尘埃地,拜天拜地拜三光。但愿父王身无恙,焚香顶礼谢上苍。)}

\textbf{李渊
【二黄原板】儿孝心感动天和地,药下咽喉病离身。空养建成、元吉子,并不进宫问安宁。日后为父(或:为父日后)归仙境,儿就是东宫的守阙人。谯楼鼓打三更时分,}

\textbf{李渊 【二黄摇板】皇儿回避(或:暂且)出宫廷。}

\textbf{(李世民 【二黄摇板】辞别父王出宫门,)}

\textbf{(李世民 【二黄摇板】为何还有作乐声。)}

\textbf{(李世民 \ldots{}\ldots{}明白便了!)}

\textbf{(李世民 【二黄摇板】听谯楼鼓打三更尽,看是何人作乐声。)}

\textbf{{[}第二场{]}}

\textbf{李渊 【西皮摇板】宫中服药精神爽,悼念御妻神暗伤。}

\textbf{(张妃、刘妃 万岁呐!)}

\textbf{李渊 梓童为何这等模样?}

\textbf{(张妃、刘妃 今有二主秦王,二更二点进宫调戏我二人。万岁做主!)}

\textbf{李渊 世民素行仁孝,为王(或:孤王)不信。}

\textbf{(张妃、刘妃 \ldots{}\ldots{}玉带为证。)}

\textbf{李渊 呈上来。}

\textbf{(张妃、刘妃 万岁请看。)}

\textbf{李渊 哎呀!}

\textbf{李渊
【西皮摇板】一见玉带怒气生,胆大奴才乱宫廷。你二人暂且回宫禁(或:后宫进),}

\textbf{(张妃、刘妃 【西皮摇板】\ldots{}\ldots{}世民丧残生。)}

\textbf{李渊 【西皮摇板】内侍摆驾金殿进,快宣皇儿李世民。}

\textbf{(李世民 【西皮摇板】忽听父王宣世民,急忙上殿问分明。)}

\textbf{(李世民 儿臣见驾,父王万岁!)}

\textbf{李渊 儿是世民?}

\textbf{(李世民 是世民。)}

\textbf{李渊 好奴才!}

\textbf{李渊
【西皮摇板】把儿当作擎天柱,奴才竟是忤逆人。吩咐两旁武士手,推出午门问斩刑。}

\textbf{(李世民
【西皮摇板】一言未发来问斩,教我有话不敢言。因何将儿推出斩,说明儿死也心甘。)}

\textbf{李渊 【西皮摇板】奴才不必将父问,现有玉带作证凭。}

\textbf{(李世民 【西皮散板】却原来为的是联珠带,)}

\textbf{(李世民 父王,父王,父王啊,呃\ldots{}\ldots{}(哭介))}

\textbf{(李世民
【西皮散板】吓得三魂少二魂。本当说出二兄长,又恐伤了手足情。望父王饶了儿的\textless{}哭头\textgreater{}命,父王啊,还望看在父子情。)}

\textbf{李渊
【西皮散板】手摸胸膛想一想,此事可行不可行。吩咐殿前(或:吩咐两旁)武士手,推出午门问典刑。}

\textbf{(李世民 【西皮摇板】含悲忍泪下龙廷,看是何人把本升。)}

\textbf{(长孙无忌 刀下留人!)}

\textbf{(武士 啊!)}

\textbf{(长孙无忌 【西皮摇板】迈步撩袍上龙廷,品级台前臣见君。)}

\textbf{(长孙无忌 臣长孙无忌见驾,吾皇万岁!)}

\textbf{李渊 上殿有何本奏?}

\textbf{(长孙无忌 二主秦王身犯何罪,\ldots{}\ldots{}午门问斩?)}

\textbf{李渊 蠢子不正,扰乱宫廷,故而问斩。}

\textbf{(长孙无忌 想秦王有十大汗马功劳,只可一赦,不可一斩。)}

\textbf{李渊 孤王龙心已定,定斩不赦。}

\textbf{(长孙无忌 万岁呀!)}

\textbf{(长孙无忌
【西皮摇板】当年驾坐太原省,隋炀帝无道灭人伦。二主大战王世充,才保我主坐龙廷。)}

\textbf{李渊 呃!(或:唗!)}

\textbf{李渊
【西皮摇板】无忌奏本太欺情,敢在金殿藐寡人。吩咐殿前(或:两旁)武士手,他与奴才同罪名。}

\textbf{李渊 绑了下去!}

\textbf{(长孙无忌 【西皮摇板】\ldots{}\ldots{},看是何人保我生。)}

\textbf{{[}第三场{]}}

\textbf{(徐勣
【西皮摇板】一见秦王上了刑,不由徐勣心内惊。\ldots{}\ldots{}忙往龙殿奔,)}

\textbf{(秦琼、程咬金 先生慢行。)}

\textbf{(秦琼、程咬金 【西皮摇板】见了先生礼相迎。)}

\textbf{(徐勣 二公慌慌张张为了何事?)}

\textbf{(秦琼、程咬金
二主秦王不知身犯何罪,推出午门斩首。我等上殿保本。)}

\textbf{(徐勣 此本你我保不下来。有人来了,你我暂退朝房便了。)}

\textbf{(徐勣 【西皮摇板】三人一同朝房进,)}

\textbf{褚遂良 (内)先生慢走!}

\textbf{(徐勣 【西皮摇板】那旁来了褚先生。)}

\textbf{(褚遂良 反了哇,反了呃!)}

\textbf{褚遂良
【西皮散板】听说要斩二主君呐,斩断了(或:斩坏了)擎天柱一根。万岁不听(或:万岁不准)忠良本,长孙无忌问斩刑。这都是二奸妃用计狠,谁知我主假作真。}

\textbf{褚遂良 哎呀!}

\textbf{褚遂良
这,这\ldots{}\ldots{},罢!(或:这,这,这\ldots{}\ldots{}哎呀!罢!)}

\textbf{褚遂良
【西皮散板】歪戴乌纱斜插带,假装疯魔去见君。大摇大摆金殿进,}

\textbf{褚遂良 【西皮散板】与他个君不君来臣不臣。}

\textbf{褚遂良 臣,褚遂良见驾,吾主万岁,万万岁!呃呃呃,请了!}

\textbf{李渊
呃嗯------胆大褚遂良,上得殿来衣冠不整,莫非你疯了?(或:呃嗯------卿家莫非你疯了?)}

\textbf{褚遂良
呃,臣倒不曾疯啊,只恐(或:只怕)万岁你昏了。二主秦王身犯何罪,推出午门斩首?}

\textbf{李渊 奴才扰乱宫廷,因此斩首!}

\textbf{褚遂良
想二主秦王,东挡西杀,南征北剿,有十大汗马功劳。将他斩首,君心何忍,这臣心何安呐?!}

\textbf{褚遂良
【西皮快板】想当年驾坐太原郡,三搜晋阳才为君。二主大战王世充,瓦岗寨收下众英雄。美良川,收敬德,千秋岭下收罗成。大唐收了罗世信,才保我主坐龙廷。挣来的江山多安稳,为何要斩创业人。}

\textbf{李渊 呃嗯------胆大褚遂良,上殿言君之过,绑了!}

\textbf{褚遂良 万岁!臣有十道条陈,容臣奏完,(诶,)再斩不迟。}

\textbf{李渊 呈上龙案,寡人御览。}

\textbf{褚遂良 臣修本不及,乃是口奏。}

\textbf{李渊 奏来。}

\textbf{褚遂良
容奏:臣这第一道条陈奏的是夏禹王坐了一十七代,四百五十八载。后出一君,名曰桀王,宠爱一妃,名唤妹喜。那桀王听信妹喜之言,以酒为池,以肉为林,忠臣良将,俱已遭害呀。}

\textbf{褚遂良
【西皮快板】自古道有道反无道,汤王定计安黎民。南巢岭桀王丧了命,只落得江山一旦倾。}

\textbf{李渊 大胆褚遂良,毁谤孤王。武士手,绑了!}

\textbf{褚遂良
啊,万岁!臣只奏过一道,还有九道未奏啊,容臣奏完,诶,再斩不迟。}

\textbf{李渊 奏来!}

\textbf{褚遂良
容奏:臣这第二道条陈奏的是成汤王得了桀王天下,传至三十一代。后出一君,名曰纣王,宠爱一妃,名叫妲己。他驾前有两个谗臣,一名费仲,一名尤浑。那纣王听信妲己之言,盖一楼名曰``摘星(楼)''。造下炮烙之刑,糟害百姓(或:残虐百姓)。比干丞相剖心而亡,贾氏夫人坠楼而死,姜后娘娘挖目剁手(或:剜目剁手),东宫太子一旦逐出,黄家父子反出五关。到后来姜尚兴兵伐纣,可叹那纣王啊,只落得火焚摘星楼台而亡。万岁,你看他也是宠爱奸妃的无道昏君喏。}

\textbf{李渊 呃------}

\textbf{李渊
【西皮摇板】褚遂良奏本孤心恨,把孤比作无道君。寡人至德平天下,学尧舜不差半毫分。}

\textbf{李渊 再将三道条陈奏来!}

\textbf{褚遂良
容奏:臣这道条陈奏的是周朝。那周文王得了纣王天下,后出一君,名曰幽王,宠爱一妃,名曰褒姒,生得(是)面貌如花。怎奈进宫以来,
永无笑容。那幽王无计可施,他驾前有一谗臣,名叫尹球。是他奏道:万岁要娘娘发笑不难。在骊山设宴,火焚烟墩。那幽王听信尹球所奏,就在骊山设宴,火焚烟墩。
各路诸侯见烟墩火起,想必国家有难,一个个顶盔贯甲,兵临城下。观见他君妃在楼台饮酒取乐哇,一个个乘兴而来,(是)败兴而返呐;那褒姒一见是呵呵地大笑。后来犬戎作乱,那幽王又将烟墩点起。各路诸侯言道:想必(又是)他君妃(又)在那里饮酒取乐啊,你我各保汛地}\protect\hyperlink{fn333}{\textsuperscript{333}}\textbf{要紧。(一个个是按兵不动。啊)万岁,你看那幽王为褒姒一笑不值紧要啊,失落周室家邦,他还死在了乱军之中。}

\textbf{褚遂良
【西皮快板】幽王无道掌乾坤,骊山设宴焚烟墩。各路诸侯无救应,江山一旦化灰尘。}

\textbf{李渊 幽王无道,戏耍诸侯,提他则甚?再将四道条陈奏来。}

\textbf{褚遂良
容奏:臣这道条陈奏的是东周列国,周惠王驾前有一家诸侯,名曰晋献公。他(或:那晋献公)宠爱一妃名唤骊姬。前妃所生二子,长子申生,次子重耳。那骊姬在献公面前搬动是非,要害(那)申生太子一死,那献公是执意地不听呐。骊姬一计不成,又生二计。用蜂蜜擦头,到御花园观花(或:采花),命申生太子保驾采花。蜜蜂围绕头上,申生太子不解其意,在后面用扇搧开。那献公在楼台之上观见,言道:这奴才果有戏母(或:残母}\protect\hyperlink{fn334}{\textsuperscript{334}}\textbf{)之心。吩咐殿前武士,将(或:把)申生太子推出午门问斩。来在午门,众大臣拦路言道:
千岁(你)有满腹含冤,为何不奏知你父王?申生太子言道:我若奏知我父王,我父王大怒,必将骊姬斩首。斩了骊姬不关紧要哇,有日我父王思想骊姬成病,岂不是小王之罪?小王只可一死,不做那不忠不孝之人。万岁,为臣看来,二主秦王与前朝申生太子一般无二。}

\textbf{(太监 着啊!)}

\textbf{(李渊 呃嗯------)}

\textbf{李渊
【西皮摇板】晋献公本是无道君,听信谗言斩亲生(或:听信谗言斩申生)。世民本是不肖子,淫乱宫闱问斩刑。}

\textbf{李渊 再将五道条陈奏来!}

\textbf{褚遂良
呃呃,臣这道条陈奏的是楚平王在临潼斗宝,多亏伍子胥力举千斤鼎,压定各国为下邦。到后来秦、楚结亲,(那)楚平王闻得(或:楚平王闻听)无祥女生得是天姿国色,有意纳妾,怎奈儿媳不好启齿啊。他驾前有一谗臣,名叫费无极,奉旨(前)往秦国迎亲。行至在钟离山前,用金顶轿改换银顶轿,无祥女改换马昭仪。好个伍子胥,保定皇家四口反出昭关,去往吴国借兵。可叹那平王(啊,他)死后,只落得鞭尸三百有余。}

\textbf{褚遂良
【西皮快板】楚平王本是无道君,父纳子妻乱人伦。子胥后来【转西皮摇板】发人马,鞭尸三百留骂名。}

\textbf{李渊 那父纳子妻,乃(是)酒色昏王,提他则甚?再将六道条陈奏来。}

\textbf{褚遂良
呃,呃,臣这六道条陈奏的是姑苏吴王宠爱一妃,名曰(西施或:名唤西施)。他驾前有一谗臣,名唤伯嚭。那吴王听信西施、伯嚭之言,起造一台,名曰姑苏台。挑选天下出色的女子,去往楼台饮酒取乐。到后来勾践兴兵前来,只杀得那吴王有家难奔,有国难投哇。}

\textbf{李渊 呃------嗯。}

\textbf{李渊
【西皮摇板】姑苏吴王无道君,听信谗言选红裙。越王勾践发人马,吴国从此不太平。}

\textbf{李渊 再将七道条陈奏来。}

\textbf{褚遂良
容奏:臣这道条陈奏的是齐宣王在桑园射猎,收来一妃,名唤无盐,手持春秋大棒(或:春秋大棍)压定各国。只因宠爱一妃,名曰夏迎春。那无盐娘娘身怀六甲,那夏迎春讨下收生代劳的旨意,用金丝狸猫剥去皮尾。启奏大王言道:那无盐娘娘产生妖魔鬼怪。齐宣王大怒,将无盐娘娘推出斩首,多亏满朝文武保奏,打入冷宫。后来吴起伐齐,只落得跪门求救哇。}

\textbf{褚遂良
【西皮摇板】齐宣王本是无道君,宠爱奸妃夏迎春。后来吴起发人马,只落得跪门去求兵。}

\textbf{李渊 齐宣王宠妃害贤,怎比孤王?将八道条陈奏来!}

\textbf{褚遂良
容奏:臣这道条陈奏的是齐湣王宠爱一妃,名曰邹赛花。他驾前有一宦官名叫伊立。那湣王听信邹妃之言,要害东宫太子一死。后来乐毅兴兵,前来追赶湣王。赶得他有家难奔,有国难投。只落得日晒湣王,路剐邹妃(或:路卧}\protect\hyperlink{fn335}{\textsuperscript{335}}\textbf{邹妃)。万岁!这就是前朝宠妃灭子的报应喏!}

\textbf{李渊 呃------}

\textbf{李渊
【西皮摇板】宠妃灭子害忠臣,他将湣王比寡人。待等奏完十道本,定与奴才同罪名。}

\textbf{李渊 再将九道条陈奏来。}

\textbf{褚遂良
容奏:臣这道条陈奏的是前朝杨广欺娘奸妹,败坏人伦,后来亡国丧身。}

\textbf{褚遂良
【西皮摇板】杨广本是无道君,欺娘奸妹乱宫廷。五花棒奸王丧了命,才保我主坐龙廷。}

\textbf{李渊 将十道条陈奏完,孤王定要将你碎尸万段。}

\textbf{褚遂良 这个\ldots{}\ldots{}}

\textbf{(太监
我说褚先生,十道条陈奏了九道,只管奏来,自有咱家帮助于你呃。)}

\textbf{褚遂良 万岁,臣这十道条陈奏的是前朝君王与本朝皇帝一般无二。}

\textbf{李渊 唗!}

\textbf{李渊
【西皮摇板】褚遂良奏本孤心恨,道道条陈刺寡人。吩咐殿前武士手,推出午门问斩刑。}

\textbf{褚遂良 冤枉------}

\textbf{武士 褚遂良冤枉。}

\textbf{李渊 召回来!}

\textbf{武士 啊!}

\textbf{褚遂良 谢万岁不斩之恩。}

\textbf{李渊 非是寡人不斩于你,为何口喊冤枉?}

\textbf{褚遂良 臣有一事不明,要在万岁驾前领教!}

\textbf{李渊 何事不明?}

\textbf{褚遂良 二主秦王什么时候进宫?}

\textbf{李渊 一更一点。}

\textbf{褚遂良 什么时候煎汤熬药?}

\textbf{李渊 二更二点。}

\textbf{褚遂良 什么时候出宫?}

\textbf{李渊 三更三点。}

\textbf{褚遂良
二位娘娘(或:皇娘)奏道,抓袍夺带是什么时间(或:什么时候)?}

\textbf{李渊 这个\ldots{}\ldots{}}

\textbf{(太监 二更二点。)}

\textbf{褚遂良
着哇,二主秦王,一更一点进宫,二更二点煎汤熬药,三更三点才得出宫。二位娘娘(或:二位皇娘)奏的是二更二点,这不是(或:岂不是)冤枉吗?}

\textbf{李渊 现有玉带,拿去看来。}

\textbf{褚遂良 待臣看来。}

\textbf{褚遂良
万岁,我想这抓袍夺带,必定是你这一拉,我这一扯。这玉带之上并无一点伤损,岂不是大大的冤枉么?}

\textbf{李渊 寡人看来(或:待孤看来)。}

\textbf{褚遂良 万岁请看。}

\textbf{李渊 嘿嘿!}

\textbf{褚遂良 嘿嘿!}

\textbf{李渊 【西皮摇板】寡人如醉方才醒,险些错斩李世民。}

\textbf{李渊
【西皮摇板】孤王急忙下龙廷,手提羊毫写分明:一赦皇儿李世民,二赦长孙无忌卿。忙将赦旨交与你,快到法场走一程。}

\textbf{褚遂良
【西皮散板】手捧赦旨下龙廷,笑坏了两班文武臣。文班中笑坏了徐勣先生,武班中笑坏了叔宝、咬金二位(或:众位)将军。都道我褚遂良不怕死。}

\textbf{褚遂良 哈哈,哈哈,啊呵呵哈哈哈\ldots{}\ldots{}(笑介)}

\textbf{{[}第四场{]}}

\textbf{(李世民 【西皮摇板】父王传旨斩世民,)}

\textbf{(长孙无忌 【西皮摇板】听信谗言斩忠臣。)}

\textbf{(李世民 【西皮摇板】忍泪含悲法场进,)}

\textbf{(长孙无忌 【西皮摇板】两眼睁睁等时辰。)}

\textbf{褚遂良 赦旨下。}

\textbf{(武士 赦旨下。)}

\textbf{(李世民 接旨。)}

\textbf{褚遂良 圣旨下。跪!)}

\textbf{(李世民、长孙无忌 万岁!)}

\textbf{褚遂良
听宣读。诏曰:只因孤王误听谗言,错斩皇儿李世民与国舅长孙无忌。多亏褚遂良保奏,将他二人赦回金殿加封。旨意读罢(或:旨意读奏),望诏谢恩。}

\textbf{(李世民、长孙无忌 万万岁!)}

\textbf{褚遂良 请过圣命。}

\textbf{(李世民 有劳先生保奏。)}

\textbf{褚遂良 保本来迟,千岁恕罪。}

\textbf{(李世民 岂敢。)}

\textbf{褚遂良 一同上殿交旨。}

\textbf{(李世民 请!)}

\textbf{{[}第五场{]}}

\textbf{李渊 (念)可恨奸妃做事错,平白无故起风波。}

\textbf{褚遂良 (念)忙将赦旨事,启奏万岁知。}

\textbf{褚遂良 启万岁:二主千岁、长孙无忌宣到。}

\textbf{李渊 宣他二人冠带上殿!}

\textbf{褚遂良 二人冠带上殿!}

\textbf{(李世民 (念)法场得活命,)}

\textbf{(长孙无忌 (念)死而又复生。)}

\textbf{(李世民 儿臣李世民\ldots{}\ldots{})}

\textbf{(长孙无忌 臣长孙无忌,)}

\textbf{(李世民、长孙无忌 谢万岁不斩之恩。)}

\textbf{李渊 皇儿、国舅平身。赐座。}

\textbf{(李世民 谢座。)}

\textbf{李渊 长孙无忌为皇儿误受一绑,加升三级,免朝一月。下殿!}

\textbf{(长孙无忌 谢万岁!)}

\textbf{李渊 褚遂良上殿听封。}

\textbf{褚遂良 臣有本启奏。}

\textbf{李渊 奏来!}

\textbf{褚遂良 臣不愿加官封赠。}

\textbf{李渊 愿者何来?}

\textbf{褚遂良 请我主差哪部大臣,将宫中查明。}

\textbf{李渊 赐座。}

\textbf{褚遂良 谢座。}

\textbf{李渊 皇儿,你二姨母怎样害你(或:怎生害你),一一奏来!}

\textbf{(李世民 父王啊!)}

\textbf{(李世民
【二黄原板】未开言不由人珠泪滚滚,尊父王听儿臣细说分明:二皇兄与姨母行事不正,儿戏君妃乱胡行。儿本当进宫细查问,又恐失了手足情。因此上将玉带宫门挂定,这就是一桩桩一件件父王详情。)}

\textbf{李渊
【二黄原板】劝皇儿休得要【转二黄快三眼】珠泪滚滚,为父的心中明如灯:将二妃贬至在冷宫禁,她自羞自惭自丧生。为江山儿何曾略得安静,为江山东挡西除、南征北剿未享安宁。今日里儿活命实称万幸,改日里过府去酬谢先生。武德君迈虎步忙下九重,}

\textbf{(褚遂良跪)}

\textbf{李渊
【二黄快三眼】用手儿挽定了褚先生(或:搀扶起褚先生)。满朝中文武臣袖手不问,怎当得先生你赤胆忠心。为皇儿把卿家的心血用尽,为皇儿哪顾得费尽辛勤。为皇儿在朝房一番议论(或:一番争论),为皇儿可算得擎天柱一根。(为皇儿把卿家的心血用尽,为皇儿哪顾得费尽辛勤。)为皇儿假装作疯魔急病,为皇儿衣冠不整来见当今。为皇儿把君臣大礼全然不论,为皇儿哪顾得舍死忘生。为皇儿连奏过十道表本,为皇儿把夏桀与商纣、前朝后代历代的昏王一代一代比与孤听。加封你吏部大堂带管那都察院,太子少保伴君正卿(或:太子少保外加正卿;或:太子少保陪伴寡人)。再赐你尚方剑如山压定,【垛板】压定了九卿四相、满朝文武、大小的官员哪一个不遵,先斩后奏启奏寡人,你是捍国}\protect\hyperlink{fn336}{\textsuperscript{336}}\textbf{的良臣。}

\textbf{褚遂良
【二黄原板】非是臣我不愿(或:我不爱)加官封赠,为的是我主锦乾坤。从今后主休听宫闱谗本(或:主休听宫中谗本),普天下众黎民乐享太平,都道你是(海不扬波是一个)有道明君。}

\textbf{李渊
【二黄原板】好一个孝道李世民,赤胆忠心褚先生。孤的皇儿残生性命亏你救应,命皇儿与先生结为师生。侍内臣把酒宴宫中摆定,孤与那皇儿、先生来压惊。左手带定世民子,右手带定褚先生。孤的皇儿李世民,孤的爱卿褚先生,你本是皇儿的恩人、孤的爱卿。劝皇儿休流泪、免悲声,放大胆一步一步随定寡人。(或:孤的皇儿李世民,孤的爱卿褚先生,你二人一步一步随定寡人。)}

\textbf{取帅印}\protect\hyperlink{fn337}{\textsuperscript{337}}

{[}第一场{]}

(吹\textless{}\textbf{点绛唇})\textgreater{}

徐勣 (念)朝臣待漏月坠西,

尉迟恭 (念)文臣武将整朝衣。

程咬金 (念)金钟玉磬连声响,

徐勣、尉迟恭、程咬金 (同念)三跪九叩拜丹墀。

徐勣 山人徐勣。

尉迟恭 鄂国公敬德。

程咬金 鲁国公咬金。

徐勣 列公请了!

尉迟恭、程咬金 请了!

徐勣
今有张士贵,在绛州龙门,招军已满,有本回朝。少刻万岁登殿,一同启奏。

徐勣、尉迟恭、程咬金 看,香烟缭绕,圣驾临朝,分班伺候。请!

李世民 {[}引子{]}海宴河淸,喜的是,四海升平。

徐勣、尉迟恭、程咬金 臣等见驾,吾皇万岁!

李世民 平身。

徐勣、尉迟恭、程咬金 万万岁!

李世民 赐座。

徐勣、尉迟恭、程咬金 谢座!

李世民
(念)父王宴驾命归天,孤王接位掌江山。征扫北国回朝转,可恨辽东起狼烟。

李世民
孤,李世民。国号贞观在位。父王宴驾,众卿保孤登基。可恨辽东盖苏文,打来连环战表,教寡人御驾亲征。是寡人命张士贵,在绛州龙门,招军集将,王君可监造战船。二人出京,数月有余,并无本章回朝,教孤日夜忧虑也。

徐勣 臣启万岁:今有张士贵有本还朝,请我主龙目御览!

李世民 呈上来!

李世民
【西皮原板】日出扶桑万道霞,群臣歌颂帝王家。张环有本奏陛下,请主龙目细详察。\protect\hyperlink{fn338}{\textsuperscript{338}}奉王旨意招人马,英雄投效到王家。并无有仁贵投帐下,

李世民 【西皮摇板】再与先生把话答。

李世民
先生,张士贵绛州龙门招军,为查访应梦贤臣。这本章上面,并无``仁贵''二字,张环莫非有欺君之意?

徐勣 张环焉敢欺君。万岁此番征东,若无贤臣保驾,臣之罪也。

李世民 秦恩公染病在床,先生保奏何人?

徐勣 臣保尉迟恭挂帅,征伐辽东,一战成功。

程咬金 万岁,休听军师之言,尉迟恭挂不得帅印。

尉迟恭 程将军,军师保我挂帅,你为何拦阻?想我开国元勋不挂,谁人能挂?

程咬金 得了罢,动不动就是开国元勋,难道我老程就不是开国元勋吗?
\protect\hyperlink{fn339}{\textsuperscript{339}}

尉迟恭 你不能。

程咬金 哼,我不能,那你也不能啊。

李世民
且慢!二卿不必争论,帅印现在秦府,就命程皇兄前去取印回来,再作定夺。

程咬金 臣领旨。

李世民 转来!

程咬金 臣在。

李世民 他乃有病之人,必须见机而行。听孤旨下!

李世民
【西皮摇板】恩公投唐功劳大,东挡西除定邦家。虽然卧病在床榻,雄心依然保唐家。卿家要说温柔话,随机应变把印拿。

程咬金 领旨!

程咬金
【西皮摇板】万岁叮咛一席话,为臣一一转秦家。辞王别驾把殿下,背转身来自咂牙。

程咬金
哎呀且住。我想这颗帅印,乃是秦二哥执掌多年;我若取回,岂不白白地送与黑贼之手?呃呃呃,我有了!我不免午门闲游一番,急回谎奏,就说二哥染病在床,昏迷不醒,他不肯交印。也免得那个黑贼痴心妄想也!

程咬金 【西皮摇板】急回谎奏君王驾,痴心妄想不归他。

黄门官 【西皮摇板】春城无处不飞花,随王并无半日暇。

黄门官 臣黄门官见驾,吾皇万岁!

李世民 平身。

黄门官 万万岁!

李世民 上殿有何本奏?

黄门官 今有王君可,有本回朝。我主龙目御览!

李世民 呈上来,待孤观看!

李世民
【西皮摇板】奉王旨意到海下,王君可修本奏皇家。请主旨意发人马,扫平辽东定中华。看罢本章记心下,黄门官近前听根芽。吩咐众将免见驾,三日之后候旨发。

黄门官 领旨!

黄门官 【西皮摇板】辞王别驾把殿下,晓谕文武百官家。

程咬金 【西皮摇板】胸藏妙计说假话,急忙上殿我骗皇家。

程咬金 臣交旨。

李世民 赐坐。

程咬金 谢座。

李世民 秦恩公病体如何?

程咬金 照常一样,呕吐不止。

李世民 唉,恩公啊\ldots{}\ldots{}(哭介)

李世民
【西皮摇板】恩公病势不见佳,不由孤王泪如麻。东挡西除功劳大,病体缠身难挣扎。

程咬金 他乃久病之人,万岁何必忧虑。

李世民 他乃有功之臣,倘有不测,孤心不忍。

程咬金 万岁真乃有道明君。

李世民 取印之事如何?

程咬金
万岁休要提起取印之事。为臣走进了病房之间,言道:二哥,你病体如何。他说:照常一样,呕吐不止。是臣言道,万岁因你染病在床,龙心悬念,命我前来探望于你。他说真是有道明君。他又问起为臣,呃,征东一事如何。臣言:万岁今日设立早朝,张士贵有本回朝,招军已满。万岁因你染病在床,龙心未定。军师力保尉迟恭挂帅。他听说尉迟恭挂帅,哼,是一派的好埋怨\protect\hyperlink{fn340}{\textsuperscript{340}}啊。

李世民 埋怨何来?

程咬金
呃,呃,他言道:想我秦琼,自投唐以来,攻无不胜,战无不取,才挣下,呃,这颗帅印。如今染病在床,倘有不测,呃,还有我儿怀玉呀。再一说,还有咬金兄弟,也是文武双全,可以挂得帅印。想那尉迟恭,与我秦琼,并无半点瓜葛之情。况且他目不识丁,何能决胜千里之外?呵呵,他一派的好埋怨呐!

徐勣 哼,你一派说谎。

程咬金 嘿,你又不曾听见,你怎么知道是谎言呢?

程咬金 臣见他那般光景呵!

程咬金
【西皮摇板】气力不佳难讲话,病势未减只又加。为臣一见心害怕,万岁龙心细详察。

李世民
【西皮摇板】孤王闻言头低下,心中辗转泪如麻。辽东若不去征伐,定说孤王惧怕他。

李世民 先生,秦恩公昏迷不醍,不肯交印,如何是好?

徐勣 万岁明日过府,一来探病,二取帅印。

程咬金
呵,万岁,休听军师之言,取印乃是一桩小事,不论差哪部大人前去也就是了,何劳御驾亲往!

尉迟恭 程将军,君入臣门,蓬荜生辉,你为何拦阻?

徐勣 你呀,真是多口!

程咬金 哎呀!你们俩呀,嘿嘿,打成了合同了。

程咬金 臣启万岁:呃,臣好有一比。

李世民 比作何来?

程咬金 掌上的乌鸦,呵,我开不得口啊!

李世民 开口便怎样?

程咬金
开口便是祸。方才说了一句话,一个,道臣拦阻,一个,道臣多口。明日过府,呃,必须有几句言语要讲;不讲,呃,反倒得罪他们,我还是不去的为妙啊。

李世民
呃,你与秦恩公昔年结为好友,只管大胆,保孤前去。有什么祸事,寡人与你担待。

程咬金 我说三哥哟,这可是万岁教我去的。呃,我这可是奉了旨的了!

徐勣 真乃一张油口。

程咬金 哼,我又油口了。

李世民 听孤旨下!

李世民
【西皮摇板】昔日恩公走天涯,锏打杨广救全家。明日文武齐保驾,孤王亲自去看他。

程咬金
【西皮摇板】黑贼金殿夸大话,军师一旁暗保他。就是帅印归他挂,也教他口念活菩萨呀。

尉迟恭
【西皮摇板】军师金殿抬爱咱,咬金一旁把话答。若是帅印归我挂,寻一良谋摆布他。

徐勣
【西皮摇板】适才咬金一席话,蒙哄万岁弄巧牙。辞王别驾把殿下,明日保主到秦家。

李世民
【西皮摇板】龙楼凤阁紫雾霞,金殿祥光绕瑞华。内侍与孤摆銮驾,探望功臣到秦家。

{[}第二场{]}

(程咬金 (内)掌灯。)

(程家院引程咬金\textless{}\textbf{小锣打上}\textgreater{})

程咬金 【西皮摇板】无事关心心不乱,有事关心心不安。

程咬金
老夫程咬金。适才在金殿用花言巧语,蒙哄万岁。可恨那个牛鼻子老道,奏了一本,请万岁明日过府,一来探病,二来取印。哎呀,我想这件事,二哥可是一概不知呀。倘若万岁明日过府,问起情由,二哥一概不知,我岂不是有蒙君之罪么。我不免连夜过府,与二哥送上一信。倘若明日万岁问起情由,也免得二哥临时,诶,失于机变。

程咬金 家院。

程家院 有。

程咬金 掌灯秦府。

程咬金
【西皮摇板】谋事在人成事天,心中恼恨黑炭丸。一心要放暗中箭,摆布黑贼有何难。

{[}第三场{]}

(秦家院、秦怀玉搀秦琼上)

秦琼
【西皮摇板】投唐保国扶江山,东挡西除马上眠。自从扫北回朝转,疾病缠身整一年。眼观红日西山晚,

(秦琼入大座)

秦琼 【西皮摇板】心中焦躁不耐烦。

(秦琼睡介,程咬金上)

程咬金
【西皮摇板】万岁金殿把旨传,晓谕文武众两班。明日过府把病探,怕的泄漏这机关。

程家院 来在(或:来到)秦府。

程咬金 前去通禀,鲁国公求见。

程家院 门上哪位在?

秦家院 做什么的?(或:什么人?)

程家院 鲁国公求见。

秦家院 候着。

秦家院 启少爷:鲁国公求见。

秦怀玉 启爹爹:程叔父到。

秦琼 怀玉迎接!

秦怀玉 遵命。

秦怀玉 【西皮摇板】怀玉出了府门前,见了叔父礼当先。

秦怀玉 迎接叔父。

程咬金 哎,罢了,罢了。怀玉,你父病体如何?

秦怀玉 照常一样,呕吐不止。

程咬金 你父今在何处?

秦怀玉 现在病房。

程咬金 带路病房啊!

程咬金 【西皮摇板】来在(或:来到)病房用目看,

程咬金 唉(或:哎呀)!

程咬金 【西皮摇板】看见二哥病容颜。重病缠身容颜变,精神恍惚非当年。

程咬金 二哥醒来。

秦琼 【西皮摇板】适才朦胧将合眼,耳旁又听有人言。猛然睁开昏花眼,

程咬金 二哥。

秦琼 【西皮摇板】只见贤弟在眼前(或:只见贤弟在面前)。

秦琼 贤弟来了,请坐。

程咬金 嘿,谢坐哟。请问二哥,(你)病体如何(呀)?

秦琼 照常一样,呕吐不止。

程咬金 你乃久病之人,何须忧虑。

秦琼 啊,贤弟连夜过府(或:夤夜到此)何事?

程咬金
哎呀二哥呀,小弟有一件为难之事(或:要紧之事)。二哥,你要依我才是啊。

秦琼 哎,贤弟,你我自结金兰,患难相顾。今日有何为难之事,愚兄依你就是。

程咬金
我说是啊,二哥啊,你有所不知啊。张士贵绛州龙门招军已满,有本回奏(或:有本回朝)。万岁见你染病在床,无人挂帅,是龙心未定啊。可恨那个牛鼻子老道,诶,他保尉迟恭挂帅呀。

秦琼 哦,那尉迟恭么\ldots{}\ldots{}

程咬金 嘿,是哦。

秦琼 哼,一介村夫,况且目不识丁,焉能(够)掌得帅印。

程咬金
是啊。小弟就是因为此事,与黑贼跟------诶,军师争论了几句,圣上命我前来取印。我想这颗帅印,二哥你执掌了多年,我若取回(或:我要取回),岂不白白地送给黑贼之手么。那时小弟,在午门闲游了一番,急回谎奏(或:急忙谎奏)。呃,实指望说几句言语,蒙哄万岁;诶嘿嘿,谁知那个牛鼻子老道,又启奏了一本:明日过府,一来探病,二来取印。我想这些个事,二哥啊,你可是一概不知啊,我特来通报与你。二哥啊,要是圣驾过府,问起取印之事,二哥啊,你就说小弟我来过了,周全小弟无罪。想你我弟兄,自投唐以来嚯!

程咬金
【西皮摇板】算来倒有数十年,并未分首各一天。生死相交共患难,这件事儿要周全。

秦琼 贤弟。

秦琼
【西皮摇板】可笑(或:堪笑)万岁见识浅,听信军师入蜚言(或:宠信军师入蜚言)。

秦琼 贤弟,只管放心,些许小事,由我担承。

程咬金 是,谢二哥。

秦琼 怀玉,取印过来!

秦怀玉 遵命。帅印在此!

秦琼 放在床前。

秦琼 夜已经深了,贤弟回府去罢。

程咬金
二哥,明日那黑贼保驾前来,你必须用言语摆布于他。小弟的言语,你可要牢牢地紧记。我告辞啦!

程咬金 【西皮摇板】辞别二哥回身转,

秦琼 怀玉代送。

程咬金 【西皮摇板】猛然想起巧机关。

秦怀玉 送叔父!

程咬金 哎,我说怀玉,你可知你父的病体,是因何而得呢?

秦怀玉 唉,前者在金殿赌力而得。

程咬金
嘿,这可就不对啦。为叔的我要不说,你哪儿知道哇。昔年大战美良川,三鞭换两锏。你父在马鞍鞒上,气堵胸膛,故而成病,至今才发。我要是不说呀,你这一辈子也不明白(或:也不知道)。

秦怀玉 依叔父之见?

程咬金
依我之见呐,明日圣驾过府探病,那黑贼一定是保驾前来。你父用言语摆布于他,他必然叫骂你父啊,你听见之后啊,就只管地打他。

秦怀玉 哎呀,他乃是开国元勋,侄儿怎敢呐。

程咬金
他是开国元勋,那你父就不是开国元勋吗?为叔我------诶,那也不是开国元勋吗?哼。

秦怀玉 侄儿打他不过呀。

程咬金
诶,小小的年纪,就说这种软弱的话。你打不过,不碍事啊,为叔父的,诶,我帮着你。

秦怀玉 哦,侄儿遵命。

程咬金 你要记下了!

程咬金
【西皮摇板】昔年大战美良城,三鞭两锏赌输赢。明日只管将他打,打出祸来有我老程(或:打出祸来我担承)。

秦怀玉
【西皮摇板】适才叔父对我言,为的当年旧仇冤。为子当把父仇报,暗藏心机不漏言。

{[}第四场{]}

(吹\textless{}\textbf{牌子}\textgreater{},大铠等引李世民、徐勣、尉迟恭、程咬金上)

秦怀玉 怀玉接驾!

李世民 你父病体如何?

秦怀玉 照常一样,呕吐不止。

李世民 前去通禀,孤王前来探病。

秦怀玉 领旨。

秦怀玉 (念)君入臣门第,蓬荜又生辉。

(秦怀玉下)

李世民 程皇兄,吩附銮驾,府外伺候!

程咬金 銮驾府外伺候哇。

(大铠等众下)

李世民 尉迟皇兄!

尉迟恭 万岁。

李世民 少时秦恩公问起征东之事,孤必命卿挂帅。他乃有病之人,你必须忍耐。

尉迟恭 领旨。

秦怀玉 启万岁:臣父叫之不应,请驾回宫。

李世民 孤是为你父病而来,再去通禀,孤在前厅等候。

秦怀玉 领旨。

(秦怀玉下)

李世民 唉!皇兄啊!

李世民 【西皮摇板】孤摆銮驾到府门,

(李世民站起)

李世民
【西皮摇板】亭台以外柳青青。山水古画多齐整,厅前瑞草送芳馨。君臣且在前厅等,

(李世民坐下)

李世民 【西皮摇板】等候怀玉报信音。

秦怀玉 【西皮摇板】君入臣门多侥幸,龙行一步百草生。

秦怀玉 臣启万岁:臣父昏迷不醒,叫之不应,请驾回宫。

李世民 平身。

李世民 先生,秦恩公昏迷不醒,如何是好?

徐勣 万岁请至(或:请到)病房,将他唤醒。

李世民 怀玉,

秦怀玉 在。

李世民 带路病房!

秦怀玉 领旨。

李世民 【西皮摇板】恩公昏迷不得醒,去至病房看真情。

(\textless{}\textbf{大锣打下}\textgreater{})

{[}第五场{]}

(秦家院搀秦琼上)

秦琼 【西皮摇板】昨日咬金来送信,果然今日驾临门。假装昏迷睡不醒,

(秦琼入大座,睡介)

秦琼 【西皮摇板】且看万岁怎样行(或:圣上怎样行)。

(秦怀玉引李世民、徐勣、尉迟恭、程咬金上)

李世民 【西皮摇板】孤王亲自来探病,功劳打动帝王心。君臣且把病房进,

(众挖门进)

李世民 【西皮摇板】只见恩公睡沉沉。

秦琼
【西皮散板】适才朦胧将睡定(或:适才朦胧荏苒\protect\hyperlink{fn341}{\textsuperscript{341}}动),耳旁又听有人声。猛然睁开昏花眼,抬头只见圣明君。

秦琼 怀玉,圣驾到此,为何不来通报?

秦怀玉 孩儿呼唤爹爹不醒,未敢惊动。

秦琼 哼------只恐儿难免欺君之罪。

李世民 啊皇兄,此乃寡人自进,与小爱卿无干。

秦琼 若非圣上开恩,定要加罪。还不谢过万岁!

秦怀玉 谢万岁!

李世民 平身。

秦琼 万岁驾到臣门,奈臣有病,不能接驾。万岁恕罪!

李世民 皇兄,你乃有病之人,谁来怪你。

秦琼 谢万岁!

李世民 孤今前来问病,但不知你病体如何?

秦琼 照常一样,呕吐不止。

李世民 你乃久病之人,且免忧虑。

秦琼
微臣久病,日加沉重,今见万岁一面,再不能朝见的了,呃\ldots{}\ldots{}(哭介)

尉迟恭 元帅,某这几日有朝事在身,少来问候。今日保驾前来,问候元帅金安。

秦琼 有劳台驾。

程咬金
二哥,您听见了没有?他说这几天有朝事在身,少来问候。今日保驾前来,捎带着给你问个好儿。这个人情,诶,你可得领他的。

徐勣 你又来多口!

程咬金 嘿,我又------哼,多口了。

秦琼 有劳众位皇兄前来问病,请坐。

徐勣、尉迟恭、程咬金 有座。

秦琼 万岁征东一事如何?

李世民
孤王昨日设立早朝,张士贵有本回朝,招军已满;王君可海下战船造齐,二人俱有本章还朝。孤见恩公染病在床,无人挂帅,孤心未定。

秦琼 万岁征东事大,奈臣,唉,这病\ldots{}\ldots{}(哭介)

秦琼
【西皮原板】疾病缠身整一春,吉凶二字解不明。残生难逃幽冥境,再不能替主扫烟尘。

李世民
【西皮原板】孤王闻言心酸痛(或:泪双淋),好似狼牙箭攒心。恩公有病难挂印,有何人替孤领雄兵。

秦琼
【西皮原板】臣子怀玉年纪轻,文韬武略智超群。胸中颇有安邦论,可命他替主(或:挂帅)统雄兵。

李世民
【西皮原板】怀玉虽然有本领,年幼怎能压(或:年幼怎能服)老臣。况且皇兄身有病,还要膝下奉晨昏。

秦琼
【西皮原板】甘罗十二为宰臣,无志空活百岁龄。怀玉年幼难挂帅,有何人替主掌权衡。

李世民
【西皮原板】昨日金殿【转西皮快板】同议论,公保尉迟老将军。因此为王来取印,即日兴兵往东征。

秦琼 啊?!

秦琼
【西皮快板】听说尉迟挂帅印,急得我心头似火焚。尉迟恭有勇无学问,焉能挂帅统雄兵。非是为臣抗君命,军师分明你错用了人呐。

徐勣 【西皮摇板】休道万岁错用人,

徐勣
【西皮快板】病缠有力不从心(或:病缠身力不从心)。将军染病一年整,无人挂印掌权衡。尉迟恭暂挂元帅印,剿灭辽东盖苏文(或:扫平辽东盖苏文)。且等将军病安稳,再往辽东扫烟尘。我主龙心似尧、舜,岂负你开国老元勋。

尉迟恭 元帅。

尉迟恭 【西皮摇板】元帅息怒且消停,

尉迟恭
【西皮快板】末将言来听分明:你双锏打下唐社稷,(某)单鞭挣下锦乾坤。末将虽然无学问,可以挂帅领雄兵。征伐辽东干戈定,元帅大印付将军(或:元帅大印还将军)。非是某有意来夺印,元帅还要三思行。

程咬金 老黑。

程咬金
【西皮快板】黑贼休要逞舌能,我等俱是有功臣。投唐国公三十六,咬金也能我统雄兵。

尉迟恭 你不能。

程咬金 我不能啊,你也不能。

尉迟恭 你不能。

程咬金 诶,你也不能。

李世民 且慢。

李世民
【西皮快板】二位皇兄免争论,俱是开国老元勋。尉迟皇兄能挂印,程皇兄也能领雄兵。太平原是将军定,原是将军定太平。

李世民
【西皮快板】回头来把话论,尊一声恩公听分明:四海的狼烟俱扫尽,不伏辽东盖苏文。恩公不肯让帅印,征东的事儿(或:征东大事)孤去不成。

秦琼 【西皮导板】狼烟一起主亲征呐,

秦琼 【西皮摇板】怎敢违抗圣明君。

秦琼
\textless{}(\textbf{三})\textbf{叫头}\textgreater{}万岁!我主!(唉,万岁啊!)

李世明
\textless{}(\textbf{三})\textbf{叫头}\textgreater{}恩公!皇兄!(唉,恩公啊!)

秦琼
万岁此番征东,三年不知,五载不晓。主在边庭,臣在京内(或:阃内)。臣不能见君,君不能见臣。臣子怀玉,尚未授职,又无婚配;臣病入膏肓,不久于人世。(或:臣病入膏肓,不久于人世;臣子怀玉,尚未授职,又未婚配。)臣纵死九泉,唉!也是不能瞑目的了哇,呃\ldots{}\ldots{}(哭介)

李世民 哦!(或:呀!)

李世民
【西皮二六】孤王闻言心酸痛,句句言辞记龙心(或:记在心)。倘若是皇兄遭不幸,细听孤王加荣封:追封王位归正品,儿孙代代(或:子子孙孙)入朝门。孤有公主银屏女,赐与怀玉配为婚。众家国公为媒证,即日里婚配驾登程,老皇兄请放宽心。

秦琼
【西皮摇板】叔宝闻言心安稳,纵死九泉也甘心呐。在枕边谢恩把首顿,转面再谢众公卿。

秦琼 怀玉。

秦琼
【西皮二六】叫怀玉近前来听父命,万岁爷的金言要谨记心。倘若是为父遭不幸,追封王位葬至在山林。我的儿子袭父职品,银屏公主配儿为婚。列位国公为媒证,即日里婚配驾启程。这就是(或:这都是)圣上面应允,儿要三跪九叩谢王(的)恩。

秦怀玉 【西皮摇板】怀玉床前遵父命,

秦怀玉
【西皮快板】含悲忍泪谢皇恩。恩赐子婿父极品,食王爵禄当报恩。叩罢头来抽身起,

秦怀玉
【西皮快板】转面再谢众公卿。小侄在朝受皇恩,还要叔父好看承。倘有一点不到处,休怪怀玉乱胡行。

程咬金 儿啊!

程咬金
【西皮快板】我儿只管任意行,凡事有我程咬金。你父身归蓬莱境,我的儿,你是这当朝的驸马,你还怕何人呐。

秦琼 呃------

秦琼
【西皮快板】咬金不必逞舌能,不使良言\protect\hyperlink{fn342}{\textsuperscript{342}}训子孙。怀玉(或:我儿)在朝受皇恩,大事全仗徐先生。怀玉年轻须教训,念在当年结义的情。怀玉请过元帅印,

秦琼
【西皮快板】再听为父细叮咛:后堂酒宴安排整(或:安排定),先敬君来后敬臣。

秦怀玉 【西皮摇板】怀玉床前遵父命,准备酒宴好看承。

秦琼
【西皮快板】见床前(或:见床头)摆列元帅印,见物情伤好惨心(或:惨凄)。咽喉紧哽话难尽,

秦琼 【西皮摇板】再叫尉迟猛将军。

秦琼 尉迟将军!

尉迟恭 元帅。

秦琼 万岁此番征东,可是挂你为帅?

尉迟恭 正是末将为帅。

秦琼 既然挂你为帅,你可晓得为将帅之道?

尉迟恭 身为武将,焉有不知为帅之道。

秦琼 好,今当万岁金面,你且讲来!

尉迟恭
元帅容禀(或:元帅听了):为帅之道,必须盔缨灿烂,铠甲鲜明;刀枪锋利,金鼓齐鸣(或:锣鼓齐鸣);安营巩固(或:安营坚固),谨守大营;擂鼓而起(或:擂鼓而进),鸣金收兵。战马须要强壮,上阵观看动静。众将不能取胜,某就单鞭匹马------我就杀------杀入万马营中。三合九战,方可收兵。这才是为帅的之道。

秦琼 (真乃是)一派的胡言!

程咬金 嘿,简直是放屁(或:如同放屁)呀!

秦琼 跪近床前,待本帅教导于你!

尉迟恭 元帅有何金言,当面请教,何必要跪!

秦琼 一定要跪!

尉迟恭 谁来跪你!

程咬金 要我挂帅啊,嘿,跪一辈子我也跪(或:跪一辈子我也干呐)!

李世民 啊尉迟皇兄,看孤份上,你就跪他一膝(或:你且屈膝)。

尉迟恭 嗯,臣(或:某家)这里跪下了。

程咬金 嘿,我说老黑诶,你要跪,就两条腿都跪下罢,这条腿干什么呐!

(程咬金踹介,尉迟恭跪)

秦 琼
听道:为将帅者,必须饱读经纶,深通战策;运筹帷幄之中,决胜千里之外呀。行兵不可马踏青苗;将士不可扰害百姓。逢高山莫先登,
遇空城莫乱入。高防围困,低防水淹;森林防埋伏,芦苇防火攻。身为元帅,令不乱传。有功嘉赏,有过责罚。有道是:(念)朝中天子三诏宣,阃外将军令出山岳动,这言发鬼神惊\protect\hyperlink{fn343}{\textsuperscript{343}}。渴饮刀头血,困卧马上眠。受得苦中苦,方为人上人。这都是为将帅之道(或:此乃是为将帅之道),须要牢牢谨记!

(程咬金 一点儿没病。)

尉迟恭 末将全记。

秦琼 帅印在此。

尉迟恭 拿来。(或:鲁莽了。)

秦琼
唗!想这帅印,乃圣上钦赐与我,理当交还万岁(或:交还圣上)。你是甚等样人(或:尔是甚等样人),竟敢前来夺取帅印,真正是无羞无耻,匹夫之辈(或:真正是匹夫之辈,毫不知羞耻)也!

秦琼
【西皮快板】不记得当年美良城,三鞭两锏赌输赢。不看万岁、先生面,定要叉出帅府门。

尉迟恭 【西皮快板】可恨秦琼太欺心,藐视尉迟有功臣。怒气不息到前厅,

程咬金
【西皮摇板】程咬金与你个腚后跟\protect\hyperlink{fn344}{\textsuperscript{344}}。

秦怀玉 【西皮摇板】准备酒宴多齐整,特请万岁饮杯巡。

秦琼
【西皮摇板】请主后堂把宴饮,军师伴驾饮杯巡。(或:后堂酒宴安排整,请先生陪驾饮杯巡。)

(秦琼作揖,佯睡介)

李世民 【西皮摇板】辞别恩公把宴饮,

(李世民、徐勣出来,李世民指印,徐勣抱印,李世民下)

徐勣 【西皮摇板】后堂奉陪天子君。

(徐勣下,秦琼醒,出座)

秦琼
【西皮摇板】好个有道圣明君,赛似(或:亚似)尧、舜掌龙庭。可叹(或:嗟叹)秦琼身染病,不能保主\protect\hyperlink{fn345}{\textsuperscript{345}}扫烟尘。

(秦琼下)

{[}第六场{]}

程咬金 【西皮摇板】黑贼前厅怒不息,咬金正好搬是非。迈步且进二堂里,

秦怀玉 【西皮摇板】见了叔父问端的。

程咬金 怀玉你慌慌张张,是为了何事?

秦怀玉 请尉迟赴席。

程咬金 嘿嘿,是呀?他骂你的父可骂出理来了!

秦怀玉 他骂我父何来?

程咬金
嘿,他骂你父啊,病不死的老牛精。他说一颗帅印,让与不让,但凭于你,为何当着众家的国公,羞侮于我呀。从前你血气方刚,可以耀武扬威;如今你咽喉啊,就只有一口虚气喽,还待这样的性傲啊。嘿嘿,教你死在阴山背后,永不翻身。这场骂哟!

秦怀玉 叔父之言,侄儿不信。

程咬金 哎呀,为叔父这大的年纪,还跟你撒谎啊。

秦怀玉 依叔父之见?

程咬金 嘿,还是昨晚那话呀,忘了啊?打他呀!

秦怀玉 哎呀,他乃是开国元勋,侄儿不敢打呀。

程咬金
哼,他是开国元勋呐,嘿,如今晚儿,你可就是当朝驸马了。你只管地打他,打完了,嗯,还得教他给你赔个礼儿。

秦怀玉 为何与侄儿赔礼呀?

程咬金
嘿,你听我告诉你,他必然在前厅,叫骂你父。你悄悄地走在他的背后,给他来个饿虎扑食啊,劈拳就打。那时为叔父的去至后堂,把万岁给请来。你听我再咳嗽这么一声,诶,赶紧翻身在地,百般地你就喊叫。

秦怀玉 喊叫什么?

程咬金
你就说,诶,儿臣好好请尉迟恭赴宴,谁知他以大压小,将儿臣暴打了一顿。若非父王到此啊,孩儿性命可有亏。那时候为叔就一句话,他就得给你赔礼。

秦怀玉 哪一句话?

程咬金 我说老黑诶,你敢打当朝驸马,犹如欺君之罪。你说他给你赔礼不赔礼啊?

秦怀玉 自然要赔礼。

程咬金 嘿,你要记下了!

秦怀玉 遵命!

秦怀玉 【西皮摇板】叔父之言牢谨记,顷刻之间打尉迟。

程咬金 【西皮摇板】娃娃中了我的计,管教老黑儿他暗吃亏。

{[}第七场{]}

尉迟恭 走哇!

尉迟恭 哇啊啊\ldots{}\ldots{}

尉迟恭
【西皮摇板】恼恨秦琼太无礼,一阵火起\protect\hyperlink{fn346}{\textsuperscript{346}}往上提。将身且坐前厅椅,气坏当朝老尉迟。

尉迟恭 \textless{}\textbf{叫头}\textgreater{}秦琼啊!匹夫!

尉迟恭
一颗帅印让与不让,但凭于你。从先血气方刚,可以耀武扬威;如今咽喉只有一口虚气,还是这样地性傲。我把你这病不死的老牛精!

秦怀玉 着打!

(尉迟恭、秦怀玉扭打介)

程咬金 嘿嘿,老黑呀,你当着万岁,你还打他呐!

程咬金
起来,起来,起来,起来\ldots{}\ldots{}嘿,有事别哭,别哭!有什么话,你只管地讲啊。

秦怀玉 \textless{}\textbf{叫头}\textgreater{}父王!

秦怀玉
儿臣好意请尉迟恭赴席,谁想他以大压小,将儿臣暴打一顿。若非父王到此,唉,儿的性命有亏啊\ldots{}\ldots{}(哭介)

尉迟恭 哎呀万岁呀,他打了老臣了!

程咬金 怎么着?他打了你了,诶,那你怎么在上头,他在底下呢?

尉迟恭 哎呀!这个\ldots{}\ldots{}喳,喳,喳,喳\ldots{}\ldots{}

程咬金 哎呀,我说老黑诶,你敢打当朝驸马呀,这就是欺君之罪呀!

李世民 着哇!

程咬金 你摸摸,还有脑袋吗?

李世民 唗!

李世民 【西皮摇板】孤王闻言怒冲起,

李世民
【西皮快板】开言大骂黑面皮。你本是堂堂国公体,全然不知高和低。怀玉本是东床婿,打他犹如把孤欺。罚俸三载赎你罪,快与驸马把罪赔。

尉迟恭 \textless{}\textbf{叫头}\textgreater{}万岁。

尉迟恭
【西皮快板】万岁容臣本奏启,细听为臣辩是非:为臣坐在前厅椅,他背后将椅往下推。就势将臣推在地,脊背打得响如雷。他见万岁来到此,翻身在地假悲啼。花言巧语奏万岁,反说为臣把他欺。臣本是堂堂的国公体,岂与他无知少年把罪替。

程咬金 我说怀玉呀,你可别哭,有什么话,你倒是说呀。

秦怀玉 \textless{}\textbf{叫头}\textgreater{}父王。

秦怀玉
【西皮快板】父王在上容奏启,细听儿臣辩是非:我父功劳谁能比,盖世忠良属第一。临潼山,把功立,双锏保驾把名提\protect\hyperlink{fn347}{\textsuperscript{347}}。如今病在牙床里,来在帅府夺帅旗。兵权大印让与你,反来叫骂无礼仪。怀玉虽然小年纪,出山的猛虎抖毛衣。万岁招我东床婿,当今驸马谁不知。以大压小把我欺,你不赔礼我不依。

尉迟恭
【西皮快板】娃娃说话太无礼,花言巧语你骂谁。你父与我同一辈,论什么高来论什么低。他双锏打来唐社稷,某单鞭挣下锦华夷。你父临潼把功立,御果园单鞭救驾回。你父功劳谁能比,某的功劳也不亏。花言巧语就是你,要想我赔礼日出西。

程咬金 你呀,赔礼罢!

尉迟恭 我不能!

程咬金 嗨嗨\ldots{}\ldots{}打红眼了,我可不惹你!

徐勣 尉迟呀!

徐勣
【西皮快板】尉迟恭暂忍心头气,我有一言听端的:怀玉虽然得罪你,大能容小休要提。为人休得心生气,又无烦来又无非。你执意不肯去赔礼,有道是君王有命怎能违。

尉迟恭 先生。

尉迟恭
【西皮快板】先生说话大有理,背转身来自猜疑。马行夹道难回避,船到江心补漏迟。罢罢罢,暂忍心头气,保主征东挂帅旗。走向前来我忙赔礼,

程咬金
嘿,怀玉你看呐,你看瞧尉迟恭这么大的岁数,跪在你的面前,你呀,饶了他罢!

秦怀玉 呵,我饶恕于你!

程咬金 嘿,老黑诶!你瞧瞧咯,这人情可是我讲的。

尉迟恭 怎么,你讲的?诶,我这儿谢谢你喽!

尉迟恭
【西皮快板】驸马爷宽宏休要提。万岁驾前告过罪\protect\hyperlink{fn348}{\textsuperscript{348}},臣不该把他少年欺。

李世民
【西皮摇板】一见尉迟来赔礼,满天浮云一扫归。怀玉近前听旨意,安排花烛结光辉。孤王回到昭阳去,急送\protect\hyperlink{fn349}{\textsuperscript{349}}公主出宫闱。

(李世民、徐勣、尉迟恭、程咬金分下)

秦怀玉 怀玉送驾!

{[}第八场{]}

李世民 尉迟皇兄,挂你为帅,当殿谢过。

尉迟恭 领旨。

李世民 程皇兄,你为三十六路都先锋,带领八十三万人马,去至海口扎营!

尉迟恭、程咬金 谢万岁!

\newpage
\hypertarget{ux6c7eux6cb3ux6e7e-ux4e4b-ux859bux4ec1ux8d35}{%
\subsection{汾河湾 之
薛仁贵}\label{ux6c7eux6cb3ux6e7e-ux4e4b-ux859bux4ec1ux8d35}}

{[}第一场{]}

(四将起霸,发点,四文堂站门,薛仁贵上)

\textless{}\textbf{点绛唇}\textgreater{}跨海征东,英名远震(或:威名远震)。军威盛,扫荡烟尘,保主锦绣春。

(薛仁贵入大座)

(念)忆昔跨海去征东,拔山举鼎显异能。可恨张环行毒计,埋没英雄汗马勋。

本爵薛仁贵,(乃)山西绛州龙门县人氏。只因保定唐王跨海征东,立下十大汗马功劳,唐王见喜,封俺为平辽王之位(或:封我为平辽王之职)。只因我离家日久,不知妻室怎生度日,为此辞王别驾,回家探望。来此离家不远,我不免改换行装,以免惊动乡邻。

中军(或:左右),看衣更换。

(\textless{}合龙\textgreater{},吹\textless{}\textbf{牌子}\textgreater{},薛仁贵换衣,分开)

中军听令。

(中军 在。)

传令下去,(吩咐)大小三军就在此地靠山近水安营扎寨,不可踏践青苗,扰害百姓。(或:不可骚扰百姓,马踏青苗,违令者斩。)

(中军 \ldots{}\ldots{}传令已毕。)

(众下)

(与爷)带马------

(薛仁贵扯四门中唱)

【西皮原板】忆昔当年去投军,张士贵是我的对头人。打虎遇着程千岁,他带我仁贵见了当今。卖弓计打破了摩天岭,三枝神箭辽东平。\protect\hyperlink{fn350}{\textsuperscript{350}}前三日修下了辞王本,

【西皮摇板】回家探望妻迎春。

(\textless{}\textbf{撤锣}\textgreater{}薛仁贵下;王禅老祖上)

(王禅老祖 (念)\ldots{}\ldots{})

(王禅老祖唤虎童,下;盖苏文魂上,过场)

(盖苏文 【西皮摇板】\ldots{}\ldots{}驾起阴风朝前走,要报当年一箭仇。)

(盖苏文魂下,接柳迎春上)

{[}第二场{]}

马来!

【西皮摇板】催马来在汾河湾,见一顽童打弹丸。弹打,弹打南来宾鸿雁,

枪挑呃,

【西皮摇板】枪挑鱼儿水浪翻。翻身下了马雕鞍,再与顽童把话言。

那一顽童在此作何玩耍?

一弹上去,能打几雁落地?

为军的不信呐。

好,你且打来。

呜哙呀,小小年纪有此本领,我若将他带回朝去,(将来)定是大大膀臂。

我自有道理。

啊,顽童。

弹打双雁落地,不足为奇。为军的(不才,呃,)我也会打雁呐。

(薛丁山 一弹上去,能打几雁落地?)

一弹上去,我能打三雁落地。(或:呃呃,我能打三雁落地。)

(薛丁山 我却不信。)

(哦,)打来你看呐。

(薛丁山 你且打来。)

(呃------呃,)借弓弹一用。

且住!南山之上下来猛虎,有伤顽童之意。身旁带有袖箭,不免伤它一箭。

呔!顽童闪开,猛虎来了!看箭!\protect\hyperlink{fn351}{\textsuperscript{351}}

唉呀!

【西皮摇板】打虎误伤顽童命,是非之地莫久停呐。仁贵拉马朝前奔(或:朝前进)。

{[}第三场{]}

马来!

【西皮快板】适才离了(或:适才路过)汾河境,一马儿来在柳家村。勒住丝缰来观定,

【西皮快板】见一个妇人在窑门(或:见一位妇人站窑门\protect\hyperlink{fn352}{\textsuperscript{352}})。布裙荆钗容貌整,看她好像柳迎春。翻身下马来询问(或:下了马能行),

(薛仁贵下马介)

【西皮快板】躬身施礼把话云。

大嫂请了。

(柳迎春 还礼。军爷敢是失迷路途。)

正是失迷路途,请问大嫂,此处可是柳家村么?

(柳迎春
\ldots{}\ldots{}这面也是柳家村,这面也是柳家村。\ldots{}\ldots{}问的是哪个?)

此地有一柳氏迎春,大嫂可晓得?

(柳迎春 \ldots{}\ldots{}问她做甚?)

大嫂有所不知,我与他丈夫同营吃粮。与她(或:托我)带来万金家书,故而动问。

(我那薛大哥言道:书信要面交本人。)

(柳迎春 不见本人呢?)

(原书带回。)

请便。

(柳迎春 啊军爷,与你打个哑谜你可晓得?)

这哑迷么?略知一二。

(柳迎春 这远------)

远在天边,不能相见。

(柳迎春 这近------)

哦!莫非你就是薛大嫂么?

哎呀呀,问来问去,问到本人的头上来了。

来来来,重见一礼呀。

(柳迎春 见过礼了。)

礼多人不怪呀。

大嫂请稍待。

哎呀且住,想我仁贵离家一十八载,不知她光景如何?

嗯,我自有道理。

啊大嫂,我实对你说了吧(或:我实对你讲了吧):我那薛大哥,在营中已欠我二十两银子(或:在营中借了我二十两银子),将大嫂你就卖与我了。

呃,呃,呃,我有婚书为证呐。

呃,慢来慢来,我看大嫂变脸变色,婚书诓到手中,三把五把扯碎,为军的岂不落一个人财两空啊?

(柳迎春 依你只见?)

呃,你我去至前村,大户人家,请上三老四少,同拆同观。

当真。

哪个骗你呀?

(柳迎春 【西皮导板】狠心的强盗啊\ldots{}\ldots{})

呵呵,她倒骂起来了哇!

哦,在哪里?

(柳迎春关门)

哎,妻呀!

【西皮摇板】叫声贤妻快开门,我是你丈夫薛仁贵转回程。

妻呀!

【西皮导板】家住绛州县龙门,

【西皮原板】薛仁贵好命苦无亲无邻呐。幼年间父早亡母又丧命,丢下了仁贵无处身存。常言道姻缘一线引,柳家庄上招了亲。你的父嫌贫(他的)心太狠,将你我二人赶出了门庭\protect\hyperlink{fn353}{\textsuperscript{353}}。夫妻们双双【转西皮快板】无投奔,破瓦寒窑\protect\hyperlink{fn354}{\textsuperscript{354}}暂安身。每日里窑中苦难尽,无奈何立志去投军。结交了兄弟们周青等,跨海征东把贼平。幸喜得狼烟俱扫尽,保定圣驾转回京。前三日修下了辞王的本,特地回来探望柳迎春。我的妻若还不肯信,来来来,算一算,连来带去十八春。

(柳迎春 \ldots{}\ldots{}薛郎,你好啊?)

我好,你可好啊?

你也不像从前了。这就是:(念)少年子弟江湖老。

(柳迎春 (念)红粉佳人白了头。)

彼此?

一样。

啊,啊,哈哈哈\ldots{}\ldots{}(笑介)

(柳迎春 你临行之前,你还讲过什么言语你可记得?)

我讲过什么(言语),我倒是记不起来了哇。

(柳迎春 想必是做官回来了。(或:你不做官是不回来的,必定是做了官了。))

(呃,)再(也)不要提起做官呐,早去三天也好,晚去三天也好哇。

(柳迎春 \ldots{}\ldots{}刚刚凑巧。)

呃,凑巧倒还凑巧哇,

只是做了一名马头军\protect\hyperlink{fn355}{\textsuperscript{355}}。

(柳迎春 哦,马头军?)

(或:正是。)

(柳迎春 但不知你有多大的品级?)

呵,大得很呐!若论这品级台位么,呃,少不得,(少不得)也有它个七、八、十来品呐。

(柳迎春 呃,做官有七、八、十来品,但不知掌管什么?)

妻呀,为丈夫在家的时节,我管些什么?

(柳迎春 与人家看马。)

我如今呐,还是与人家牵马(或:看马)------

(柳迎春 \ldots{}\ldots{}看马。)

和从前是一样啊。

呃,有心胸,

(柳迎春 \ldots{}\ldots{}你有志气。)

有志气。

我这个志气还小吗?

(柳迎春 喂呀\ldots{}\ldots{}(哭介))

呃,我不回来,你是盼我回来。我好容易回来了,你又是这样鼻子、脸子的。

好好好,我在家中住上三五天,呃,我还是出外啊。

呃,葬埋在龙头山。

何谓马头山?

呃,还是龙头山的受听呐。

龙头山,龙头山。(或:龙头山,龙头\ldots{}\ldots{})

(柳迎春 \ldots{}\ldots{}马头山。)

啊,妻呀,我那岳父岳母百年之后,葬埋在何处啊?

呵,你看你看,到了他们家就成了凤凰山了。

依我看来,呵,不叫作凤凰山呐。

要叫作穷苦山。

你想啊,我在家的时节,你就是这样的受苦;我(如今)出外一十八载,如今回来,你还是这样的受苦(或:你还是住在这个破窑)。你爹娘生下你这受苦的女儿,呃,岂不是叫作穷苦山么?

呃,这也是你家的坟地里的风水呀。

呃,呵呵呵,穷苦山呐。

呵,穷苦山,穷\ldots{}\ldots{}

呃,这我倒不晓得呀。

哦,你是为我哇。

唉,我在外面一十八载,(省吃俭用,)受尽风霜之苦------

哦,我为的是哪个啊?

我啊,我也是为的是你呀。

我不为你,还为这座破窑不成么?

呃------我乃是受尽风霜之人,你不要呕我哇。

你不要呕我哇。

呵。

噗。

薛礼呀薛礼,你真真地岂有此理(呀)!

你今日回得家来,乃是一桩喜事,你偏偏要(来)呕她。

哎呀你看你看,把她气得这个样儿。

不妨不妨,待我取出一件东西,教她来看看,她就不生气了。

啊妻呀,为丈夫与你带回来好东西来了。

哎呀,(你也)特以地挖苦了,不是这些(或:这般)物件呐。

你拿去看来。

你看仔细。(或:仔细看来。)

你呀,拿过来吧。

这是我保定唐王跨海征东,立下十大汗马功劳,唐王见喜,封我为平辽王之位(或:之职)。这就是平辽王的虎头金印呐!

砷黄铜\protect\hyperlink{fn356}{\textsuperscript{356}}?!像这样的砷黄铜你见过几块呀。

呵呵呵,你(呀,)不开眼呐。

砷黄铜不要看了。

还要看看?

但要小心了。(或:你要看仔细。)

哎呀!

你还是拿过来吧。

你要把(或:要将)我这平辽王吞吃在腹内呀。

(柳迎春 \ldots{}\ldots{}饿怕了。)

惭愧!

啊妻呀,为丈夫一路行来,有些口渴,有什么香茶取来一用。

用些什么? (好,与我取来。)

好好好,快些取来。

【西皮摇板】在长安何曾吃白水,此水难饮泼埃尘。

不用啊。

(妻呀,为丈夫)腹中有些饥饿,有什么好酒好饭,取来一用呃。

用些什么?

何谓鱼羹?

好好好,与我取来。

【西皮摇板】用手接过鲜鱼羹,

呃,

【西皮摇板】这样腥气实难闻。

不用了啊。

鞍马劳顿,身体困倦呐。

哦,怎么还有后窑?

好,快快打扫。(或:与我打扫)

我如今回来了。

啊?

【西皮摇板】听她言来自思忖,莫非相交有情人。(或:察言观色详其情,教人心中解不明。)出得窑去观动静,

【西皮摇板】窑外并无一个人。

【西皮摇板】将马拴在柳荫下,

【西皮摇板】鞍辔放在了地埃尘。

【西皮摇板】站在窑中来观定,

【西皮摇板】这只男鞋必有因。

且住!怪道她变脸变色,原来她有了外遇了!

呀呀呸!(柳氏啊柳氏,)你在你丈夫跟前露出马脚来了。

贱人,你与我走了出来呀!

(呀呸!)你自己做的事还要问我?

你呀,你就是与我------

死呃。(或:呃!)

要赃。

要双。

呵呵,怕无有你的赃证?!这不是你的赃?这不是你的证?

你就是与我\ldots{}\ldots{}

唉!

(柳迎春 \ldots{}\ldots{}可问的这穿鞋的人\ldots{}\ldots{})

呃,我不问这穿鞋的人儿,还问我这穿靴子的人么?

(柳迎春 \ldots{}\ldots{}比你强得多啊!)

(是啊,)自然比我强啊!

我如今有了这个讨厌的东西了。

是啊,你若是靠着我,饿啊,(或:你要靠着我,这一十八载,饿啊,)也把你饿干了哇。

什么新鲜的事情?(或:哦,什么新鲜之事。)

哎呀呀,你真是无羞无耻呀!

你不死待我来碰。

(你啊,你就是与我------)

(唉!)

嗯,有的。

也是有的。

一十七岁的孩儿------不大,不小,是正穿呐。

哎呀,她倒端起来了。

哎呀呀,你拿过来吧。

妇道人家,拿刀、动杖,呃,成什么样儿啊?

薛礼呀薛礼,难为你还是个平辽王啊,做事就是这样粗鲁。

哎呀呀,这窑前窑后,也无人前来解劝呐\ldots{}\ldots{}这这这\ldots{}\ldots{}

这这这\ldots{}\ldots{}这这这\ldots{}\ldots{}这这这\ldots{}\ldots{}

唉,上前赔个笑脸可也就拉倒了。

妻呀!为丈夫的不是,喏喏喏,我这厢(与你)赔礼了。

妻呀!为丈夫的不是,我这厢又赔礼了。

唉,妻呀!为丈夫这厢(或:这里)跪下了哇。(或:俱是为丈夫的不是,我这厢跪下了。)

哎呀你这是怎么样了?(或:呃,你这做什么?)

哎呀,耍出汗来了。

(啊)妻呀,将你我的儿子唤将出来,教他看看(或:教他看一看)我这不成器的老子。

哦,他往哪里去了?

(惊介)

(啊)妻呀!我来问你,这窑前窑后,(可)还有别人家的孩儿会打雁?

贤妻你这里来啊!

你我的儿子出窑的时节,这头戴?

身穿?

左手?

右手?

唉呀!

【西皮导板】听一言来吓掉魂,

\textless{}\textbf{三叫头}\textgreater{}丁山!我儿!唉!儿啊\ldots{}\ldots{}(哭介)

(柳迎春 \ldots{}\ldots{}儿的老子。)

唉!

【西皮散板】凉水浇头怀抱冰。适才路过汾河境,见一个顽童打弹能。弹打南来宾鸿雁,枪挑鱼儿水浪分。

他不会来了\protect\hyperlink{fn357}{\textsuperscript{357}}!

【西皮摇板】本当与她实言论,又恐吓坏这受苦的人呐。

唉!

【西皮摇板】事到临头难瞒隐,咬定牙关说真情。

\textless{}\textbf{叫头}\textgreater{}唉呀妻呀!

适才为丈夫打从汾河湾前经过,观见你我的儿子在那里射雁。南山之上,下来猛虎,身旁带有袖箭,实望将虎打走,不想这一箭呐------将你我的儿子就射死了!

射死了!

唉呀,又是一条人命呐!

醒来!

汾河湾前。

随我来呀。(或:一同寻找。)

丁山\ldots{}\ldots{}

我儿\ldots{}\ldots{}

\newpage
\hypertarget{ux6c99ux6865ux996fux522b}{%
\subsection{沙桥饯别}\label{ux6c99ux6865ux996fux522b}}

{[}第一场{]}

(唐玄奘上)

玄奘 {[}引子{]}一年气象一年新,抛却红尘念佛经。

玄奘 (念)正在佛前打坐,回头观见五岳。一班俱是神像,为何欺善怕恶。

玄奘
贫僧玄奘。只因唐王天子为游地府许下大愿,要往西天拜佛,取经回朝,设立坛台,超度众魂。是我情愿替主一往。今乃黄道吉日,不免上朝,请主发下通关文凭,即日启程便了。

玄奘
【二黄慢板】有玄奘离娘怀身遭大难,蒙吾师搭救我来到金山。取法名唤玄奘苦读经卷,每日里在殿前把佛来参。因唐王游地府许下大愿,为的是斩神龙起下祸端。传旨意将众僧道法考选,我情愿替君王取经回还。

(玄奘下)

{[}第二场{]}

(徐勣、殷开山、程咬金、尉迟恭同上)

徐勣 (念)日出山高一片红,

殷开山 (念)唐王江山掌握中。

程咬金 (念)长安多少花似锦,

尉迟恭 (念)堪叹不觉白头翁。

徐勣、殷开山、程咬金、尉迟恭 老夫,

徐勣 徐茂公。

殷开山 殷开山。

程咬金 程咬金。

尉迟恭 尉迟敬德。

徐勣 列公请了。

殷开山、程咬金、尉迟恭 请了。

徐勣
只因吾主曾命金山法师,去往西天拜佛取经,今日上殿见驾领凭,同在朝房伺候。请。

(玄奘上)

玄奘 (念)离了金山寺,上殿见圣君。

玄奘 众位国公在上,贫僧稽首。

徐勣、殷开山、程咬金、尉迟恭 有礼相还。

殷开山 儿啊\ldots{}\ldots{}

徐勣、程咬金、尉迟恭 请问国公,他是何人?

殷开山 唉!乃是老朽外孙。

徐勣、程咬金、尉迟恭
原来如此。法师请在殿角伺候,圣驾临朝,我等启奏。请。

(玄奘下)

徐勣、殷开山、程咬金、尉迟恭 金钟三响,圣驾临朝,分班伺候。

(四小太监、二大太监引李世民上)

李世民 {[}引子{]}先王晏驾,龙归藏,孤掌朝堂。

徐勣、殷开山、程咬金、尉迟恭 臣等见驾,愿吾皇万岁。

李世民 平身。

徐勣、殷开山、程咬金、尉迟恭 万万岁。

李世民
(念)忆昔当年战洛阳,收得瓦岗众豪强。可叹恩公秦琼丧,寡人日夜不安康。

李世民
寡人大唐天子,贞观在位。因游地府,曾许大愿。考得僧人玄奘,道法甚高,愿替寡人西天拜佛取经。今乃黄道吉日,命他前往。徐皇兄。

徐勣 臣。

李世民 玄奘可曾宣到?

徐勣 今在殿角候旨。

李世民 宣他上殿。

徐勣 领旨。万岁有旨,玄奘上殿。

玄奘  (内)领旨。

(玄奘上)

玄奘 (念)金殿传旨宣,别驾往西天。

玄奘 玄奘见驾,愿------吾皇万岁。

李世民
法师\protect\hyperlink{fn358}{\textsuperscript{358}}替朕西天取经,封卿御弟三藏,如朕亲临。

玄奘 愿吾皇万岁。

李世民 平身。

玄奘 万万岁。

李世民 赐座。

玄奘 谢座。

(玄奘坐大边跨椅)

玄奘 启奏万岁:今乃黄道吉日,请驾发下通关文凭,即日启程。

李世民 内侍,文房四宝伺候。

李世民
【二黄慢板】王因为游地府许愿斋醮,超度那泾河龙重回天曹(或:轮回阴曹)。孤将这众高僧传旨选考(或:众僧人传旨选考),唯有那金山的(或:金山寺)玄奘法高。他情愿往西天见佛拜祷,他情愿取真经替朕代劳。孤想你往西行无穷(或:无数)路道,今日去何日归才得还朝?

玄奘
【二黄三眼】请吾主修文凭休迟即早(或:请我主写牒文休迟即早),仗吾皇洪福大何惧山遥。只要人秉诚心见佛拜祷,吾主爷何需要替僧心焦。

李世民
【二黄原板\protect\hyperlink{fn359}{\textsuperscript{359}}】提龙笔王亲书大唐国号,命御弟唐三藏奉旨出朝。各国的众王子【转二黄三眼】休挡禁道,到西天取经回替朕代劳。赐御弟锦袈裟霞光万道,孤赐你(或:赐御弟)紫金钵、禅杖一条。孤赐你(或:赐御弟)装经箱、毗卢僧帽,孤赐你四徒儿鞍前马后、涉水登山好把经挑(或:赐御弟四小童好把经挑)。内侍臣与孤王将宝抬到(或:替孤王将宝抬到),金銮殿王与你改换佛袍(或:改换法袍)。

(\textless{}合龙\textgreater{}玄奘改装,下)

李世民 【二黄摇板】王传旨即便把众卿宣召,随同孤送御弟饯行沙桥。

(李世民下)

{[}第三场{]}

(大太监上)

太监 (念)朝朝随驾走,时时伴龙行。除了当今主,咱家第一人。

太监
咱家,大唐天子驾前,掌朝内监是也。奉了万岁爷的旨意,在沙桥备酒,与三藏法师饯行。酒宴备齐,等候圣驾与众家国公前来。正是:

太监 (念)吾主许下诚心愿,就有高僧往西天。

(徐勣、殷开山、程咬金、尉迟恭上)

李世民 (内)【西皮导板】出午门到沙桥王下车辇,

(李世民搀玄奘同上)

太监 奴俾接驾。

李世民
【西皮原板】叫一声贤御弟细听王言:孤想你数万里路途崎险,孤愁你何日里得到西天。但愿你此一去早把佛见,但愿你路途上免带愁颜。但愿你见佛祖取经回转,百里外排銮驾接到殿前。

玄奘
【西皮慢板】万岁爷休得要将臣怜念,容为臣一一地细奏根源(或:细表根源):僧的父蒙恩赐七品正县,上任去遇刘贼劫了官船。将僧父用绳捆丢在水面(或:抛在水面),那贼子霸官亲就印为官。贤德母怀小僧十月孕满,想自尽又恐怕绝了后传。那一天生下僧时乖运\emph{蹇},刘洪贼他一见怒气冲天。霎时间(或:顷刻间)要将臣一刀两断,贤德母跪尘埃才得保全。用匣装写血书【转西皮二六】抛在水面(或:丢在水面),取名字江流儿性命由天。金山寺老禅师道法非浅,算定臣不该死(或:算就臣不该死)救至在山前。取法名唤玄奘苦把经念,看破了红尘路世事不贪:【转西皮快板】一不贪富与贵做官为宦,二不贪妻共子游玩清闲。三不贪吃珍馐五荤三厌,四不贪走花街观看红颜。五不贪住龙楼凤阁温暖,六不贪五花马銮驾旌幡。七不贪用奴仆随身使唤,八不贪出门庭拥后呼前。九不贪红颜女把酒来献,十不贪穿龙袍受王官衔。但愿得见佛祖取经回转,保唐室国泰民安万万年。

李世民
【西皮二六】内侍臣看过了皇封御宴,孤爱你道德好十事不贪。孤愿你此一去无灾无难,孤愿你足生云(或:孤愿你足登云)早到佛前。孤赐你饯行酒金杯玉盏,太平去吉日归\protect\hyperlink{fn360}{\textsuperscript{360}}孤谢上天。

玄奘
【西皮快板】谢吾皇饯行酒金杯玉盏,怎敢当吾的主龙恩海宽。转身来对苍天把酒来奠,祝告了天和地日月星官。

徐勣
【西皮摇板】领王命在沙桥把行来饯,尊一声大法师细听吾言:受老朽这一礼非为别干,替吾主取经回大大相烦。

玄奘
【西皮快板】小贫僧有何能怎敢领饯,公本是国王师八卦先天。三贤府盖过了臣救君难,保圣驾坐长安万万余年。

尉迟恭
【西皮摇板】老尉迟敬酒宴遮住了英雄脸,手捧着紫金杯躬身向前。但愿得此一去取经回转,某在那百里外接进朝班。

玄奘
【西皮快板】老国公家住在麻邑贵县,抢三关、夺八寨好不威严。征辽东挂帅印威风八面,访白袍保唐室万古名传。

程咬金
【西皮快板】程咬金平日里讲理不惯,尊法师休怪我粗鲁之言。烦公公你与我将酒斟满,

程咬金 【西皮摇板】见佛祖取真经早早回还。

玄奘
【西皮快板】程千岁可算得忠心赤胆,在瓦岗聚英雄人闻胆寒。弃暗地投明主官高爵显,但愿你寿延年快乐清闲。

殷开山 酒来。

殷开山
【西皮散板】论国法本应当国师称唤,论家法你本是老夫孙男。在长亭替你母把行来饯,我的孙何日里才得回还。

玄奘
【西皮散板】见外公不由我心中凄惨,烦外公拜儿母不孝之男。就说儿奉王旨不敢迟慢,多拜上贤德母少来问安。

玄奘 【西皮散板】在沙桥抬头看红日西转,请万岁、众国公驾回朝班。

李世民 【西皮散板】内侍臣带龙驹孤把缰挽,叫御弟跨金镫早奔阳关。

玄奘 哎呀!

玄奘 【西皮散板】君带马与臣骑世间稀罕,阻王驾休咒臣忙把王拦。

玄奘 将马带过!

玄奘 【西皮散板】辞王驾别国公忙把路趱,到西天拜佛祖取经回还。

(玄奘下,众太监、内侍、徐勣、殷开山、程咬金、尉迟恭引李世民同下)

(\textless{}\textbf{尾声}\textgreater{})

\newpage
\hypertarget{ux4e7eux5764ux5e26-ux4e4b-ux674eux4e16ux6c11}{%
\subsection{乾坤带 之
李世民}\label{ux4e7eux5764ux5e26-ux4e4b-ux674eux4e16ux6c11}}

(内)摆驾!

【二黄慢板】想当年老王爷带兵出征,下江南十余载才得回程。得了胜回朝来交旨复命,麒麟阁摆筵宴犒赏功臣。小杨广在席前言语不正,紫金杯打奸王惹下祸根。因此上修下了辞王表本,连夜里带家属转回故林。行至在临潼山被贼围困,多亏了秦恩公搭救满门。隋炀帝坐山河天心不顺,下扬州观琼花涂炭黎民。天降下五花棒奸王丧命,众公卿保父皇驾坐乾坤。遭不幸老王爷龙归海境,众老臣一个个辅孤王驾坐九重。恨只恨摩里沙兴兵犯境,命驸马秦怀玉前去剿平。但愿得此一去旗开得胜,但愿得此一去马到功成。内侍臣摆御驾九龙口进,又听得殿角下大放悲声。

(\textbf{或}:【二黄慢板】想当年老王爷带兵出征,下江南十余载得胜回程,得胜归回朝来交旨复命,麒麟阁摆筵宴犒赏功臣。小杨广在席前言语不正,紫金杯打奸王惹下祸根。因此上修下了辞王表本,连夜里带家属转回故林。行至在临潼山前被贼围困,多亏了秦恩公搭救满门。隋炀帝坐江山天心不顺,下扬州观琼花涂炭黎民。天降下五花棒奸王丧命,众公卿保父皇驾坐龙庭。遭不幸老王爷龙归海境,窦太后望儿楼凤驭上宾。众老臣一个个忠心耿耿,一个个辅孤王驾坐金龙。恨只恨摩里沙打来奏本,他要夺孤王的锦绣乾坤。为王的在金殿传下旨意,命驸马秦怀玉去把贼平。但愿得此一去旗开得胜,但愿得此一去马到功成。侍内臣摆御驾九龙口进,又听得后宫院大放悲声。)

梓童为何这等模样?

呜哙呀,有这等事?

梓童平身。

赐座。

内侍(或:来),宣银屏公主带子上殿。

(公主 万岁!)

皇儿,你可知罪?

这才是皇儿的道理。

平身。

殿前武士,将秦英绑上殿来。

唗!胆大秦英,前番将程雄打死,孤不降罪于你,也就是了。怎么,今日又将詹老太师打死(金水桥前),二罪归一。

殿前武士,将秦英推出午门斩首(或:斩了)。

(长孙皇后 吾皇万岁。)

御妻平身。

赐座。

(梓童上殿有和本奏?)

(长孙皇后 \ldots{}\ldots{}是哪位大臣?)

小将秦英。

(长孙皇后 \ldots{}\ldots{}所犯何罪?)

前番将程雄打死,不降罪于他。今日又将詹老太师打死金水桥前,故而推出斩首。

(这个\ldots{}\ldots{})

寡人龙心已定了,御妻不必多奏。

你又来多事了。

【西皮慢板】劝御妻休得要把本奏上,孤怎比开河运无道隋炀。孤岂肯听信那谗言毁谤,孤岂肯斩忠良绝了那秦门后香。慢说是打死了詹老丞相,就是那庶民人也要抵偿。

是啊,你母女在金殿奏得本,爱梓童,哎,连一句话都讲不得吗?

梓童你有本?当殿奏来,寡人与你作主。

梓童平身。

梓童赐座。

哼,这还了得!

呃,梓童奏来。

【西皮导板】这桩事教孤王难以发放,

(长孙皇后 儿啊\ldots{}\ldots{}(哭介))

【西皮原板】娘哭儿、女哭父好不惨伤。孤传旨斩了那秦英小将,

唉!

【西皮原板】孤皇儿在一旁两泪汪洋。孤传旨赦了那秦英小将,

\textless{}\textbf{哭头}\textgreater{}教孤好为难呐,(老皇妻呐,)

【西皮原板】爱梓童殿角下哭断肝肠。唐贞观在龙书案前思后想,

【西皮原板】爱梓童近前来【转西皮二六】细听端详:你的父并不曾欺君罔上,可怜他金水桥一命身亡。孤劝你把此事休挂心上,哪有个人死后又能还阳。孤传旨挑选那能工巧匠,孤传旨修一座忠义祠堂。孤传旨赐你父金井玉葬,孤传旨文武臣送至在山岗,王去拈香,孤的爱梓童,你那里且免愁肠。

(詹妃 【西皮摇板】\ldots{}\ldots{}母女二人。)

呃------

【西皮摇板】唐贞观亦非是懦弱(的)皇上,为的是安黎民整顿朝纲。哪一个大胆人敢来违抗?

【西皮二六】叫皇儿近前来父女商量。小秦英打死了皇亲国丈,论国法就应该叫他抵偿。念秦门昔年间东杀西挡,念秦门只有这一脉后香。金銮殿父赐儿玉液琼浆,殿角下去哀求詹妃娘娘。

【西皮摇板】好一个爱梓童宽宏大量,不由孤心内喜(或:不由孤龙心喜)暗称贤良。

【西皮摇板】为王的在金殿把旨来降,午门外快赦回秦门儿郎。

非是寡人不斩于你,詹娘娘讲情,将你饶恕。一旁谢过詹娘娘(或:上前谢过詹娘娘)。

(秦英 谢过姨姥哦!)

御妻、梓童、皇儿回避。

内侍(或:内臣),宣徐勣上殿。

平身。

赐座。

卿家上殿,有何本奏?

呈上来。

待孤(或:寡人)看来。

哎呀!原来驸马被困,卿家计将安在(或:有何良策)?

小将秦英,打死皇亲国戚(或:皇亲国丈)。

已然赦却(或:赦回)。

依卿所奏。

内侍(或:内臣),宣秦英上殿。

秦英,今有你父被困摩里沙。命你带领人马前去征剿,得胜还朝(或:得胜回朝),将功折罪。外赐乾坤宝带,以振军威。

(秦英 谢万岁!)

见过儿徐祖父。

退班。

\textbf{芦花河 之
薛丁山}\protect\hyperlink{fn361}{\textsuperscript{361}}

(内)马来!

(薛丁山上)

【西皮二六】奉主旨意往西征,数年铠甲未离身。先父当年挂帅印,在白虎关前命归天庭。多亏了智勇樊夫人,她也能提调众三军。来至在辕门下金镫,

(薛丁山下马)

啊?!

【西皮快板】辕门外绑定薛应龙。我儿犯了何条令,缘何捆绑问典刑(或:问斩刑)?

(醒来。)

哦!

【西皮摇板】我道是犯了那皇王的军令,却原来为的是这临阵招亲------

儿啊!

【西皮快板】我的儿只管心放稳,为父进帐讲人情。进帐只用三两语,管教你母饶儿身。本帅撩袍宝帐进------

哦!

【西皮快板】王法条条不徇情。我若讲情她不允(或:我若讲情她不准),把娇儿反送在这枉死城。

【西皮快板】进帐去先行周公礼,必然念在夫妻的情。

【西皮摇板】秦、窦二将往上禀(或:秦、窦二将一声禀),你就说二路的元帅转回大营。

(薛丁山下)

(秦汉、窦一虎 二路元帅到!)

(樊梨花 【西皮导板】\ldots{}\ldots{})

(薛丁山上,进帐,樊梨花出帐,二人撞肩膀)

(樊梨花 王爷!)

夫人!

(樊梨花 王爷请!)

夫人请!

夫人请坐!

(二人换位,薛丁山大边、樊梨花小边,坐)

(樊梨花 【西皮原板】迎接元帅进大营,\ldots{}\ldots{}打听得哪路发来兵?)

【西皮原板】一来是夫人威名盛,各国闻名不敢动兵。

(樊梨花 【西皮原板】\ldots{}\ldots{}闲事情。)

呀!(或:哦!)

【西皮快板】樊夫人她倒有(或:樊夫人她倒能)隔山照镜,就知本帅讲人情。未曾开言把罪请呐。

(樊梨花 【西皮原板】问王爷施礼为何情?)

【西皮摇板】应龙儿犯了(或:身犯)何条令,缘何捆绑问典刑(或:缘何捆绑在辕门)?

(樊梨花 王爷问的是他?)

正是!

(樊梨花 王爷呀!)

(樊梨花 【西皮二六】\ldots{}\ldots{}问斩刑?)

(哦。)

【西皮摇板】我道是犯了那皇王(的)军令,却原来为的是这临阵招亲。提起来招亲的事,话也难尽,难道说贤夫人你心不明。想当年大战在那寒------

(樊梨花 噤声。)

掩门。

【西皮二六】寒江岭,寒江关前动刀兵。我与夫人来会阵,夫人与我来提亲。

(薛丁山拉樊梨花,樊梨花羞介,二人换位)

【接西皮二六】本帅再三不应允,夫人又把巧计生。使下了(或:设下了)移山倒海阵,

(二人换位)

【接西皮二六】将本帅吊在那(或:吊至在)半空存。那时我唤天,天不应;我待入地,地又无门。万般无奈才应允,夫妻双双进唐营。若论这临阵招亲,是你我先来做定,常言道前人开路,这后人行呐。

(樊梨花 【西皮快板】王爷说话\ldots{}\ldots{},军无私来就法无情。)

【西皮摇板】应龙犯罪理当斩,

(樊梨花 谢王爷!)

且慢!

【西皮摇板】还要看他的年纪轻。

(樊梨花 【西皮快板】 \ldots{}\ldots{}不是娘生?)

【西皮快板】本帅与你讲人情,哪个和你比古人。大夫人生下麒麟子,二夫人也有后代根。唯独夫人无有后,收下应龙作螟蛉。到如今夫人有了梦熊信\protect\hyperlink{fn362}{\textsuperscript{362}},便把应龙当外人。倘若是娇儿有伤损,旁人道你两样心。

【西皮快板】你若是赦了应龙子,唐王降罪我担承。

(【西皮快板】你今赦了应龙子,满营将官我也担承。)

【西皮快板】不能不能万不能呐。

(樊梨花 【西皮摇板】 你把你\ldots{}\ldots{}看大了。)

【西皮摇板】威宁侯啊,也不放在本帅心。

哎呀!

【西皮快板】一见宝剑挂营门,吓得三魂少二魂。眼望娇儿无救\textless{}\textbf{哭头}\textgreater{}应,我的儿啊,

【西皮摇板】父子们做鬼一路行。

\textless{}\textbf{哭头}\textgreater{}薛应龙,我的儿啊(或:小娇儿啊),啊\ldots{}\ldots{}

夫人,

\textless{}\textbf{哭头}\textgreater{}我的儿啊!

夫人你看,众将皆服了(或:满营将官皆服了)。

(樊梨花 【西皮摇板】\ldots{}\ldots{}平身。)

解下桩来。

(樊梨花欲踹薛应龙介)

夫人方才赦过了。

出帐去罢。\protect\hyperlink{fn363}{\textsuperscript{363}}

(探子 讨战。)

再探!

夫人,贼人摆下阵势,你我夫妻敌楼一观------(或:军士们,带马城头去者。)

(樊梨花 带马。)

【西皮散板】适才探马报一声,芦花河贼子发来兵。

【西皮散板】下得马来敌楼进,观看贼阵是何名(或:观看贼子发来兵)。

夫人,贼人摆的是何阵势?

(樊梨花 此乃是金------)

噤声!

【西皮散板】叫声夫人莫高声,

(薛丁山、樊梨花下城)

【西皮散板】休要惊动那贼兵(或:休要惊动这贼兵)。

【西皮散板】下得马来宝帐进(或:下得马来大营进),

【西皮散板】再与夫人把话云(或:夫妻对坐论军情)。

夫人方才讲的金什么? (或:适才摆的什么阵势?)

(樊梨花 乃是金光大阵。)

可有破法?

如此(待)本帅二次回转仙山,哀求师父,求来法宝,再破此阵。

即刻启程,我有一言,夫人听了:

(薛丁山拉樊梨花到台口)

【西皮快板】手挽手,站营门,尊声梨花樊夫人。芦花河摆下金光阵,莫教应龙去出征。倘若娇儿有伤损,那时失了夫妻情。辞别夫人上马行,

【西皮摇板】我嘱咐你言语呀,你(要)谨记在心呐。

\newpage
\hypertarget{ux54edux5c38}{%
\subsection{哭尸}\label{ux54edux5c38}}

樊梨花 (念)眼跳心惊,未知吉凶。

秦汉、窦一虎 元帅,大事不好了。

樊梨花 什么大事?

窦一虎 小本官芦花河阵中身亡。

樊梨花 不好了!

樊梨花 尸首可曾带回?

秦汉、窦一虎 带回来了。

樊梨花 搭了上来!

樊梨花
【西皮快板】一见娇儿丧了命,不由为娘痛在心。千言万语说不\textless{}\textbf{哭头}\textgreater{}尽,喂呀我的儿啊,

樊梨花 【西皮快板】哪个教你去出兵?

薛丁山 (内)马来!

薛丁山 【西皮摇板】深山奉了师尊命,回到大营说分明。

樊梨花 \textless{}\textbf{叫头}\textgreater{}元帅!

樊梨花 大事不好了!

薛丁山 何事惊慌?

樊梨花 应龙丧命。

薛丁山 在哪里?

樊梨花 在这里。

薛丁山 哎呀!

薛丁山 【西皮摇板】一见娇儿丧了命,

樊梨花 【西皮摇板】怎不教我痛在心。

薛丁山 【西皮摇板】你在那玉泉山何等安静呐,

樊梨花 【西皮摇板】哪个教你去出兵?

樊梨花 【西皮摇板】千言呐------

薛丁山 【西皮摇板】万语啊------

薛丁山、樊梨花 【西皮摇板】说不尽\ldots{}\ldots{}

薛丁山
【西皮快板】把话说与夫人听:芦花河摆下金光阵。休教应龙去出兵,今日我儿丧了命,看你心疼不心疼?

樊梨花 \textless{}\textbf{哭头}\textgreater{}薛应龙,

薛丁山 \textless{}\textbf{哭头}\textgreater{}小娇儿啊\ldots{}\ldots{}

樊梨花
\textless{}\textbf{哭头}\textgreater{}喂呀,我的儿啊\ldots{}\ldots{}

樊梨花
【西皮快板】元帅息怒容我禀,为妻言来你试听\protect\hyperlink{fn364}{\textsuperscript{364}}:我在营中传将令,不知他私自去出兵。如今娇儿丧了命,难道我不痛在心?

薛丁山 \textless{}\textbf{哭头}\textgreater{}薛应龙,

樊梨花 \textless{}\textbf{哭头}\textgreater{}小娇儿啊,

薛丁山 \textless{}\textbf{哭头}\textgreater{}啊, 我的儿啊!

薛丁山
【西皮快板】大夫人生下麒麟子,二夫人也有后代根。唯独夫人无有后,你把应龙当亲生。今日夫人心不稳,有意绝我后代根。

樊梨花 \textless{}\textbf{哭头}\textgreater{}薛应龙,

薛丁山 \textless{}\textbf{哭头}\textgreater{}小娇儿啊,啊------

樊梨花 \textless{}\textbf{哭头}\textgreater{}我的儿啊!

樊梨花
【西皮快板】元帅有所不知情,为妻言来听分明:先禁自己后禁人,怕的三军心不平。

薛丁山 \textless{}\textbf{哭头}\textgreater{}薛应龙,

樊梨花 \textless{}\textbf{哭头}\textgreater{}小娇儿啊,

薛丁山 \textless{}\textbf{哭头}\textgreater{}啊------我的儿啊!

薛丁山 呀呸!

薛丁山
【西皮快板】自古道青竹蛇儿口,自古道黄蜂尾上针。自古道万般皆由命,自古道最毒妇人心。

樊梨花 \textless{}\textbf{哭头}\textgreater{}薛应龙,

薛丁山 \textless{}\textbf{哭头}\textgreater{}小娇儿啊,啊------

樊梨花 \textless{}\textbf{哭头}\textgreater{}喂呀, 我的儿啊!

樊梨花
【西皮快板】元帅道我两样心,对着苍天把誓盟:梨花待子有假意,死在千军万马营。

薛丁山 言重了!

薛丁山
【西皮摇板】听一言来才知情,本帅错怪樊夫人。叫人来抬尸首后营进呐,父子们相逢万不能。

樊梨花 \textless{}\textbf{叫头}\textgreater{}薛应龙,

薛丁山 \textless{}\textbf{叫头}\textgreater{}小娇儿!

薛丁山、樊梨花 唉!儿啊\ldots{}\ldots{}(哭介)

\textbf{法场换子}\protect\hyperlink{fn365}{\textsuperscript{365}}

{[}第一场{]}

夫人 (念)夫受皇家爵,妻沾雨露恩。

徐策 (内)开道!

(\textless{}\textbf{小锣六幺令}前段\textgreater{},徐策下轿,\textless{}\textbf{小锣原场}\textgreater{},进门)\protect\hyperlink{fn366}{\textsuperscript{366}}

徐策 唉!

夫人 相爷今日下得朝来,为何这等长叹?

徐策
夫人有所不知,(或:夫人呐------听道:)今日早朝,可恨张泰奸贼将薛猛夫
妻调进京来,要害他二人一死,(唉,)倒也罢了哇。最可叹未满三月小薛蛟,
也要受皇家一刀之苦。怎不令人长叹呐!

夫人 就该寻一计策,搭救忠良才是。

徐策 (下官正为此事回来,与夫人商议。)计策倒有哇,只是要应在夫人的身上。

夫人 难道说教妾身替他不成?

徐策
不是哟。我看金斗孩儿,面带七煞\protect\hyperlink{fn367}{\textsuperscript{367}},终难抚养。意欲带到法场,将薛蛟调换
下来,以接薛门宗嗣。不知夫人你的意下如何?

夫人
相爷说哪里话来,想你我夫妻,年将半百,只有此子,若是替人------万万不能。
\protect\hyperlink{fn368}{\textsuperscript{368}}

徐策 唉!夫人呐,呃,唉\ldots{}\ldots{}(哭介)

夫人 万万不得能够!

徐策 唉!夫人呐,呃\ldots{}\ldots{}(哭介)

徐策
【二黄快三眼】恨薛刚小奴才不如禽兽,吃醉了酒全不顾满面惭羞。闯下了滔天祸一人逃走,连累他二爹娘不能到头。把一个两辽王午门斩首,樊夫人拔宝剑自刎人头。眼见得忠良臣乏嗣无后,可怜他斩草除根、寸草不留、天地含忧,怎教我看水流舟,夫人呐!

夫人
【二黄原板】老相爷说此话情理不周,听妾身把此事再说从头:张泰贼与薛家结成仇扣,满朝中文武臣不敢出头。怕的是画虎不成反类狗,那时节船到江心倒做了逆水行舟。

徐策
【二黄散板】贤夫人舍不得娇儿金斗,眼见得小薛蛟一命罢休。为忠良我只得屈膝叩首哇,

夫人 【二黄散板】老相爷跪埃尘情理不周。

夫人 相爷不必如此,妾身应允就是。

徐策 多谢夫人。

徐策 家院(过来)。

家院 有。

徐策 将你家少公爷放在食盒之内,抬到法场。再拿我名帖,去见张泰,就说老夫
要亲自祭奠。

(家院 是。)

(家院招丫鬟抱小孩(喜神)同下)

徐策 附耳上来(,记下了)。

家院 遵命。

夫人 啊相爷,妾身也要跟随前去。

徐策 法场之上,耳目甚众,去之无益。

夫人 妾身要去。

徐策 夫人要去?到了法场,看下官眼色行事。

夫人 遵命。

徐策 (如此)夫人请。

夫人 相爷请。

徐策 正是:(念)可叹薛家世代贤,

夫人 (念)忠良无故把刀餐(或:忠良无辜被刀餐)。

徐策 (念)苍天有灵睁开眼,

夫人 (念)仇报仇来冤报冤。

徐策 着哇!好一个``仇报仇来冤报冤''。

徐策 夫人,

夫人 相爷。

徐策 随我来。

{[}第二场{]}

(\textless{}\textbf{大锣六幺令}后段\textgreater{},张泰上)

张泰
(念)树大遮天盖地,根深哪怕风狂。(任他皇亲国戚,一本斩草除根。)\protect\hyperlink{fn369}{\textsuperscript{369}}

张泰 老夫张泰,奉圣命监斩薛猛夫妻。刀斧手,将薛猛夫妻押了上来。

马氏
哎吓老爷呀,你我夫妻一死,不值要紧,可叹三月孩儿,也要受皇家一刀之苦
哇\ldots{}\ldots{}(哭介)

马氏
【二黄散板】叫你反来你不反,叫你行来你不行(或:叫你行来逃生你不行)。
你我一死不要紧,可怜那娇儿也受酷刑。\protect\hyperlink{fn370}{\textsuperscript{370}}

薛猛 夫人呐!

薛猛
【二黄散板】薛家世代忠良后,\ldots{}\ldots{}怎做那叛逆臣。回头再把张泰论:苦害薛
家为何情?恨不得一足将尔踏,阴曹地府勾尔魂。\protect\hyperlink{fn371}{\textsuperscript{371}}

张泰
校尉等,将他夫妻绑上法标。有人讨祭,报我知道。\protect\hyperlink{fn372}{\textsuperscript{372}}

家院
(念)奉了相爷命,法场走一程。\protect\hyperlink{fn373}{\textsuperscript{373}}

家院 法场之上哪位听事。\protect\hyperlink{fn374}{\textsuperscript{374}}

校尉 做什么的?

家院 徐老相爷有名帖奉上,前来法场祭奠。

校尉 候着。

校尉 启相爷,徐相爷有帖拜上。

张泰 呈上来。

张泰 呜哙呀,这老儿又来多事。

(校尉 他的夫人也来了。)\protect\hyperlink{fn375}{\textsuperscript{375}}

张泰
(哦,夫人爷来了。)\protect\hyperlink{fn376}{\textsuperscript{376}}命他一祭(或:容他一祭),时辰一到,速报我知。

校尉 容你们一祭。

家院 祭礼走上。

(丫鬟随家院、四青袍上,中间两个青袍抬盒,家院站台中间,四青袍脸朝里,开盒,丫鬟取出小孩与马氏怀中小孩交换。家院令众人下,青袍领下,丫鬟随青袍后,右手抱小孩、用左袖盖小孩随下,家院留场上)

家院 有请相爷夫人。

徐策 (内)夫人,随我来!

(家院下)

徐策 【二黄散板】夫妻双双到法场呃,

夫人 【二黄散板】不见忠良在哪厢。

徐策 \textless{}\textbf{叫头}\textgreater{}薛猛!

夫人 \textless{}\textbf{叫头}\textgreater{}马氏。

徐策 唉!儿啊\ldots{}\ldots{}(哭介)

徐策 【二黄散板】他夫妻好比一张弓,

夫人 【二黄散板】万马营中抖威风。

徐策
【二黄散板】未把箭放弦又\textless{}\textbf{哭头}\textgreater{}断,我的儿啊,

夫人 【二黄散板】一到法场一场空。

徐策 夫人,天色不早,先回府去吧。

(夫人过大边,徐策过小边,夫人在大边面向小边)

夫人 待我辞别辞别。

夫人 \textless{}\textbf{叫头}\textgreater{}薛猛!

夫人
\textless{}\textbf{叫头}\textgreater{}马氏------我那金\ldots{}\ldots{}

(徐策面向大边,阻拦介)

徐策 噤声!

夫人 今生今世难得见的\ldots{}\ldots{}亲儿啊\ldots{}\ldots{}(哭介)

(夫人下场门下)

徐策
正是:(念)法场之上冷嗖嗖,绳拿索绑不自由。盖世忠良遭毒手,(或:法鼓嗵
嗵打,西山月影斜。黄泉无客店,)

徐策 \textless{}\textbf{叫头}\textgreater{}薛猛!

徐策 \textless{}\textbf{叫头}\textgreater{}马氏!

徐策
(念)花开花落(或:花开花谢)籽未丢哇。(或:今晚宿谁家。)啊,呃\ldots{}\ldots{}(哭介)

徐策
【反二黄慢板】见夫人哭出了席棚以外,可怜她抛撇下十月怀胎。催命鼓响嗵嗵魂飞天界,勾命锣仓啷响魄散泉台。这壁厢绑的是薛猛元帅,那壁厢绑的是马氏裙钗。马夫人使双刀名扬四海,女将中可算得出色英才。你夫妻原本是镇守边塞,为什么一心心闯进京来。儿好无才,我的儿啊!

徐策
【反二黄三眼】千不该万不该是儿不该,大不该命薛刚私出府来。那奴才寿堂上把寿来拜,二爹娘一见娇儿,溺爱不明,把酒戒来开。三杯酒下咽喉劣性还在,酒壮胆、胆包天闯下祸来。驸马爷张登荣被他踢呃坏,太子爷紫金冠也打落尘埃。保驾的官、文武臣一齐打坏,最不该持香炉去打张泰。张泰贼奏一本将你来害,将儿的一家人捆绑御街。你夫妻尽了忠留名后代(或:你夫妻尽了忠留名四海;你夫妻双双死命里所在),【垛板】最可叹,断送了未满三月小婴孩,捆绑到御街,刀下赴泉台,儿好无才!冤哉冤哉,令人悲哀,好不伤怀,我的儿啊!

徐策
【反二黄原板】我也曾送儿的信,儿怎生不解,书信中藏密语儿解之不开。我教儿领人马反出了边塞,儿为何一心心闯进网来(或:闯入网来)。老徐策见此情无计可奈,舍亲生将薛蛟调换下来。待老夫替你家抚养几载,将养\protect\hyperlink{fn377}{\textsuperscript{377}}起忠良后祭扫泉台。可怜我年半百绝了后代,绝了后代,

徐策 【反二黄散板】恨不得将张泰斧斫刀开。

徐策
【反二黄散板】这一旁搀扶起薛猛元帅,马夫人我不便搀你、你\ldots{}\ldots{}你自呃己起来。到九泉见先人呐把我话带,你把我舍子的情细说开怀。

徐策
【反二黄散板】悲切切哭出了法场以\textless{}\textbf{哭头}\textgreater{}外啊,

徐策 【反二黄散板】等候了大炮响啊,收儿的尸骸。

徐策 \textless{}\textbf{叫头}\textgreater{}薛猛!

徐策 \textless{}\textbf{叫头}\textgreater{}马氏!

徐策 我那金\ldots{}\ldots{}(惊介)

徐策 今生今世难得见的\ldots{}\ldots{}唉,亲儿啊\ldots{}\ldots{}(哭介)

徐策 罢!

(徐策一跺脚,下)

张泰 校尉等,时辰可到?\protect\hyperlink{fn378}{\textsuperscript{378}}

校尉 时辰已到呃。

张泰 拿去开刀!

薛猛、马氏 好贼------

校尉 斩首已毕,现有一婴孩。

张泰 呈上来。

张泰
呜哙呀,这一婴孩,生得是眉清目秀,不免带回府去,收为义子。唉------呃,``斩草不除根,萌芽依旧生;斩草除了根,萌芽永不生''。

张泰 校尉等,将这婴孩,腰铡三截。

校尉 斩首已毕。

张泰 打道,上殿交旨。

\textbf{按}:此戏中徐策法场哭祭的大段``反二黄''唱腔是余叔岩根据李吉甫的《法场换子》的本子(用余自己的《焚绵山》的本子换取)设计的唱腔。刘曾复先生为樊百乐君另外还示范了传统的谭派《法场换子》的``反二黄''唱法(可参考程君谋、蒋锡康的《法场换子》唱片录音),兹照录如下:

【反二黄慢板】见夫人哭出了席棚以外,可怜她年半百十月怀胎。催命鼓响嗵嗵魂飞天界,救生锣仓啷响魂又转来。

【反二黄中三眼】站席棚先埋怨薛猛元帅,大不该命薛刚私出府来。进什么京来把什么寿拜,二爹娘爱子心又把宴排。三杯酒下咽喉劣性还在,酒壮胆胆包天闯下祸来。御花园众神像打成土块,太子爷紫金冠也打落尘埃。探花郎张登荣也被打坏,最不该上金殿去打张泰。张泰贼奏一本将你来害,将你来害,我的儿啊!

【反二黄原板】因此上薛门中降下祸灾。这一边哭坏了薛猛元帅,转面来再埋怨马氏裙钗。在阳河你就该反出了边塞,为什么将娇儿带进京来。你夫妻双双死情理所在,【垛板】最可叹,小薛蛟,未满三月也被刀开,我的儿啊!

【反二黄原板】只为你薛门中绝了后代,舍金斗将薛蛟调换下来。待老夫替你家抚养几载,将养起忠良后祭扫泉台。可怜我年半百绝了后代,绝了后代,

(薛猛、马氏跪,徐策不看二人)

【反二黄散板】恨不得把张泰斧斫刀开。

哎呀!

【反二黄散板】这一旁搀扶起薛猛元帅,马夫人我不便搀你、你\ldots{}\ldots{}你自己起来。

【反二黄散板】悲切切哭出了法场以\textless{}\textbf{哭头}\textgreater{}外啊,

【反二黄散板】等候了大炮响收儿的尸骸。

\newpage
\hypertarget{ux53ccux72eeux56fe-ux4e4b-ux5f90ux7b56}{%
\subsection{\texorpdfstring{双狮图\protect\hyperlink{fn379}{\textsuperscript{379}}
之
徐策}{双狮图379 之 徐策}}\label{ux53ccux72eeux56fe-ux4e4b-ux5f90ux7b56}}

{[}第一场{]}

(内)开道!

【二黄原板】朝罢圣天子转回府门,见狮子并一处所为何情?

家院,今日(府门)何人值日?

唤书僮。

罢了。

今日可是你的值日?

我来问你:府门外玉石狮子缘何并在一处?

哦,是你并在一处的么?

好。当着老夫的面前,还要与我分开(或:再与我分开)。

哦?狮子会讲话?(它)讲些什么?

来,掌嘴!

哦?是你家少公爷并在一处的么?

(好,)快快唤他前来。

不像话。

(我儿)罢了,一旁坐下。

儿啊,为父今早下朝回来,观见府门外玉石狮子,缘何并在一处?

呃,呃,你这做什么?

就该打手。

儿啊(你)慢慢讲来。

呃,你这又做什么?

就该掌嘴。

(还不)下去。

儿啊,你只管地讲来。

怎么?是我儿并在一处的么?

为父的不信呐。

哦,你还能分开?

(好,你)看仔细。

啊,呵呵哈哈哈\ldots{}\ldots{}(笑介)

【二黄散板】他父是英雄儿好汉,强将手下无弱兵。张泰贼这就是尔对头到,薜家出了报仇人。

儿啊,自从你(或:自从儿)长大成人,还未曾祭过祖先(或:拜过祖先)。今日随为父祖先堂上一祭。

家院,祖先堂打扫。附耳上来。

儿啊,随为父的来呀!

呵呵哈哈哈哈\ldots{}\ldots{}(笑介)

{[}第二场{]}

(\textless{}\textbf{柳摇金}\textgreater{}上)

儿啊,随为父的来呀!

儿(啊,)要多拜几拜。

一旁坐下。

我儿有所不知。此乃我朝一家忠良,被奸臣陷害。全家问斩,后辈无人。为父的与他家世代交好,故而将他的真容悬挂在祖先堂上一祭。

这头一排么,此人姓薛名礼,字仁贵,乃山西绛州龙门县的人氏啊。此人英雄盖世,武艺超群。保定唐王,跨海征东。立下十大汗马功劳,唐王见喜,封为平辽王之位。

平辽王之位。

第二排,此乃仁贵之子,名唤丁山。那旁樊梨花樊氏夫人。夫妻二人,征西有功,唐王见喜,(到后来)封为两辽王之职(或:两辽王之位)。

正是。

此乃丁山长子,名唤薛猛,那旁双刀马氏夫人。夫妻二人镇守阳河,被奸臣陷害,调进京来,双双而死。呃\ldots{}\ldots{}(哭介)

那黑汉在哪里?那黑汉在\ldots{}\ldots{}

呀呀呸!好你大胆黑汉,闯下塌天之祸(或:满门被害,皆因你一人所起),还敢在此发笑,还不与我走、走、走\ldots{}\ldots{}

真真气、气\ldots{}\ldots{}气煞------我也\ldots{}\ldots{}(哭介)

非是为父的动怒啊,满门被害,皆因他一人所起,怎不令人发怒啊?(或:我儿有所不知,此乃丁山三子,满门被害,皆因他一人所起,故而为父的动怒啊!)

那小孩子在哪里?

(那)小孩子\ldots{}\ldots{}

\textless{}\textbf{三叫头}\textgreater{}金斗!儿啊!唉!儿啊\ldots{}\ldots{}(哭介)

呵呵哈哈哈\ldots{}\ldots{}(笑介)

非是为父悲中带喜,你看那小孩子虽然是腰铡三截\protect\hyperlink{fn380}{\textsuperscript{380}},他还不曾死啊!

何谓呆话?

我儿哪里知道,只因我朝有一家忠良,与他薛门世代交好,不忍他断绝香烟。将自己亲生的儿子,在法场之上调换下来,故而他还不曾死啊。

还在。

他今年多大年纪了?

儿啊,你站了起来。

(一旁)坐下。

与我儿般长般大。

论他的本领么? (或:问他的本领么?)

他,他,他\ldots{}\ldots{}他能力举千斤。

儿为何发笑?

唉,只因他单丝不线,孤树不林。故而也就耽误下了。

怎么(或:哦)?(我)儿要替他代报冤仇么?

呀呀呸!儿自己有血海冤仇,尚未报得,还要替人家报的什么冤仇?

儿啊!

(儿)说什么前堂有父,后堂有母\ldots{}\ldots{}

慢说没有冤仇,纵有冤仇,儿是即刻就报。可惜我不是儿的亲\ldots{}\ldots{}

儿啊,为父的今早起来,吃了几杯早酒,说话么有些个颠三倒四。儿不必细问,快快(快)攻书去罢。

【二黄导板】未开言不由人珠泪滚滚,

\textless{}\textbf{三叫头}\textgreater{}徐忠(或:薛蛟)!我儿!唉,儿啊\ldots{}\ldots{}(哭介)

【回龙】待为父细说那以往原因。我的儿啊!

【二黄原板】头一排(或:第一排),儿曾祖哇薛仁贵,跨海征东立下功勋。

【二黄原板】第二排,儿祖父丁山元帅,那一旁樊梨花樊氏夫人。

【二黄原板】双尸无头是儿的亲生父母,亲生父\textless{}\textbf{哭头}\textgreater{}母,我的儿啊!

【二黄原板】他夫妻双双问典刑。

【二黄原板】那黑汉是儿的三叔父,

【二黄垛板】都只为,进都城、逛花灯、吃醉酒、打伤人,连累了一家满门,绑赴法场,俱丧残生,是一个起祸根。

【二黄原板】腰铡三截是我的亲生子,掉换呐\textless{}\textbf{哭头}\textgreater{}你,喂呀我的儿啊!

【二黄散板】张泰贼是儿的对头人。

我儿哪里去?

我儿一人焉能报得?

无妨,儿有一三叔父,现在韩山招兵聚将。待为父修书一封,儿去往那里搬兵报仇就是。

(念)我儿换衣巾。

家院,溶墨。

【二黄碰板】说明了十数载冤仇恨,血海的冤仇要报清。老徐策在祖先堂上修书信,打发娇儿早早登程。

【二黄原板】未曾提笔泪难忍,骂一声小薛刚不肖的畜生。当初进京把寿进,吃醉酒、打伤人连累满门。老夫见情心不忍,法场之上舍亲生。到如今此子长成(或:到如今薛蛟长成)有本领,他两膀之上力千斤。见书即刻发人马,老夫内应共灭仇人。一封书信忙写定,

【二黄摇板】我儿此去要小心(或:打发娇儿早动身)。

(\textless{}\textbf{哭头}\textgreater{}啊,我的儿啊!)

转来。

我儿此番搬兵,不定是三年五载,才得回来。儿来看------

为父的年迈呀。倘若我二老下世,儿必须买上几陌纸钱,去至坟前烧化。也不枉为父的抚养儿一十几载,养育之恩呐!呃\ldots{}\ldots{}(哭介)

为何?

呃,话虽如此,为父的虽则年迈,身体倒还康健。儿只管地前去。

你当真不去(或:儿当真不去)?

果然不去?

不去为父就要打!

哦?打死儿也是不去的么?

呀呀呸!徐策呀徐策,你好没来由!倘若留得自己亲生儿子在世,焉能如此这般倔强(或:这等倔强)。唉!喂呀,我那亲生的儿啊\ldots{}\ldots{}(哭介)

好!上马去罢!

\textless{}\textbf{三叫头}\textgreater{}薛蛟!我儿!唉,儿啊\ldots{}\ldots{}(哭介)

\textless{}\textbf{叫头}\textgreater{}薛蛟!我儿!

\textless{}\textbf{哭头}\textgreater{}啊,我的儿啊!

【二黄散板】见娇儿上了马能行,好似开弓箭一根呐。悲悲切切后堂进(或:二堂进),见了夫人说分明。

唉!儿啊\ldots{}\ldots{}(哭介)

\textbf{打金枝 之 唐王}\protect\hyperlink{fn381}{\textsuperscript{381}}

{[}第一场{]}

(郭暧
【西皮摇板】吾主爷有道君长安驾坐,全凭着驾下臣保定山河。安禄山反河东文武胆破,我父子扫狼烟才定干戈。蒙圣恩将金枝招赘于我,父王位、子东床\protect\hyperlink{fn382}{\textsuperscript{382}}扶保朝阁。今日里八旬寿群臣齐贺,奉王命回府去呀敬致三多\protect\hyperlink{fn383}{\textsuperscript{383}}。)

{[}第二场{]}

(郭暧
【西皮摇板】唐君瑞\protect\hyperlink{fn384}{\textsuperscript{384}}失却了周公之礼,有天地有父母才有夫妻。似这等不贤妇要她何益,倒不如在府中独宿孤栖。)

{[}第三场{]}

摆驾!

【西皮慢板】金乌东升玉兔坠,景阳钟三下响王出宫闱\protect\hyperlink{fn385}{\textsuperscript{385}}。唐室连年遭颠沛,国乱只为杨贵妃。安禄山在河东【转西皮二六】曾起反意,兵破潼关夺社稷。陈元礼兵变在马嵬驿,可怜那贵妃丧沟渠。先皇驾幸西蜀地,多亏皇兄郭子仪。血战三载狼烟息,擒住了贼子剑下劈。到如今乐享这太平世,黄河清、北海晏有凤来仪。内侍臣摆御驾九龙【回龙】里,

【西皮摇板】君王有道福寿齐。

【西皮快板】一见皇儿泪悲啼,打碎珠冠扯破衣。你与驸马因何起\protect\hyperlink{fn386}{\textsuperscript{386}},一一从头奏孤知。

皇儿平身。

赐座。

慢慢奏来!

【西皮摇板】御妻休得本奏启,

【西皮摇板】皇儿且莫泪悲啼(或:皇儿也莫泪悲啼)。

【西皮摇板】你母女暂且回宫去。

【西皮摇板】内侍与孤传旨意,快宣皇兄郭子仪。

【西皮二六】九龙口内红光起,来了皇兄郭子仪。昨日里皇兄悬弧喜\protect\hyperlink{fn387}{\textsuperscript{387}},王未曾去拜寿也曾赐过你珍奇。王坐江山全亏你,从今后赐你剑、履上丹墀。内侍臣与孤搀扶起,

【西皮摇板】君臣对坐把话提。

【西皮摇板】殿角绑的何臣子,一一从头说孤知(或:一一从头奏孤知)。

(郭子仪 【西皮摇板】请王传旨将他斩。)

【西皮散板】老皇兄做事太心急。况且驸马轻年纪\protect\hyperlink{fn388}{\textsuperscript{388}},公主又是少年妻。自古道清官难断家务事,他夫妻吵闹常有之。孤皇传旨不降罪,快与驸马去换朝衣。

皇兄平身!

赐座。

皇兄,昨晚宫中,驸马缘何与公主争论?

内侍,宣驸马冠带上殿。

平身。

驸马,昨晚为何与公主争论?

皇兄,驸马所言甚是。

从今以后,红灯撤去,只行夫妻常礼。

呃,往下奏来。

啊,皇兄,听驸马所奏,孤是明白了。

昨日皇兄八旬双寿,众家哥弟,一个个成双结对,拜寿堂前。公主不在,驸马一人拜寿,自觉孝道有亏,是与不是?

唉!皇兄啊,自古道:不痴不聋,难做公翁。从今以后,他夫妻之事,你不必劳心呐。

听孤旨下!

【西皮二六】驸马奏本孤的龙心爽,颇知三纲并五常。但愿皇兄多欢畅(或:臣心若得君欢畅),福寿齐眉永安康。老皇兄暂且回府往,王与驸马有商量(或:共商量)。

【西皮散板】驸马近前听旨降:忠臣孝子永留芳。孤王赐你尚方剑,命公主赔罪到汾阳。

【西皮散板】内侍臣摆驾后宫进,见了御妻说分明。

\newpage
\hypertarget{ux73e0ux5e18ux5be8-ux4e4b-ux674eux514bux7528}{%
\subsection{珠帘寨 之
李克用}\label{ux73e0ux5e18ux5be8-ux4e4b-ux674eux514bux7528}}

{[}第一场{]}

\textless{}\textbf{点绛唇}\textgreater{}荆棘铜驼\protect\hyperlink{fn389}{\textsuperscript{389}},唐室残破。离朝阁,自立山河,沙陀全归我。

(念)太白斗酒诗百篇,长安市上酒家眠。摔死国舅段文楚,唐王一怒贬北番。

孤,李克用呃,祖居沙陀,先父朱(姓,讳)国昌\protect\hyperlink{fn390}{\textsuperscript{390}},归顺唐室。讨贼有功,因赐国姓(或:只因屡建奇功,唐王见喜,恩赐国姓)。唐王见孤左眼小比龙,右眼大比虎,生就龙虎之姿,认孤为螟蛉义子殿下(或:认孤为义儿殿下),赐名``鸦儿''\protect\hyperlink{fn391}{\textsuperscript{391}}啊。

只因那年,孤王得胜还朝,唐王见喜,在五凤楼前恩赐御宴,文武百官庆贺千秋,(或:只因那年,孤王得胜还朝,唐王在五凤楼前恩赐御宴,以贺千秋,)内有国舅段文楚,笑孤坐席不正,礼貌不周。怒恼孤家,隔席抓过,(我)就摔呃------摔在丹墀,那贼就口吐鲜血而亡了。唐王大怒,将孤推出午门斩首,多亏恩官程敬思连保数本,唐王赦了死罪,将孤削职,贬回故土。(或:唐王死罪已免,活罪难饶,将孤谪贬沙陀为民。)

来到沙陀(或:是孤来在沙陀),众家王子,顶盔贯甲,拦住孤的马头,俱要与孤(王)比试。那时孤哪有什么闲情逸致与他们玩耍,是孤稳坐雕鞍,心生一计,将孤的九九八十一斤定唐宝刀------哗喇喇------耍上数路,众家王子一个个拜服马前,尊孤为首。一路之上,收了(或:收下)二位皇娘,一十一家太保。来到沙陀(或:来在沙陀),风调雨顺,国泰民安。朝朝饮宴,夜夜笙歌。好不洒乐人也!正是:(念)红尘一点不到处,

太保,回来了?

打来(或:打了)多少飞禽走兽?

(哦,)什么新闻?

嚯------胆大黄巢,欺我唐室无人。

太保(听令,)传令(下去):二位皇娘挂帅,众家太保以为前站先行,带领(或:发动)沙陀国四十五万番汉兵将,前去兴唐灭巢!

慢,慢\ldots{}\ldots{}慢着!唐王无道,将孤谪贬(或:当年唐王将孤谪贬),哪有人马与他解围。

太保,原令追回!

太保,你是怎样知晓?

哦,程恩官来了。

他乃孤王(或:孤家)活命恩人,(必须迎接于他。)太保------

吩咐摆队相迎。

{[}第二场{]}

久违了。(或:啊,恩官。)

你也皓然\protect\hyperlink{fn392}{\textsuperscript{392}}了哇。

一样。

啊------呵呵呵哈哈哈\ldots{}\ldots{}(笑介)

恩官到此,乃是客位。

恩官请。

(如此)你我挽手而行。

{[}第三场{]}

且慢,你乃孤活命恩人,受孤一拜。

太保,见过儿的程叔父(或:拜见程叔父)。

唐王驾安?

满朝文武可好?

有劳他们。

请坐。

不知恩官驾到,未曾远迎,当面恕罪。

岂敢。

看酒来,待孤把盏。

太保代敬。

恩官请。

干!

正是:(念)忆昔五凤楼,相隔有数秋(或:相隔数十秋)。

好哇------好一个``叙叙旧根由''。

【西皮导板】太保传令把队收,

干!

【西皮原板】孤与贤弟叙一叙旧根由。

【西皮原板】忆昔当年五凤楼,文武百官庆贺千秋。内有个文楚段国舅,他笑孤王坐席不正、礼貌不周。怒恼了孤王气冲牛斗,隔席抓过摔死龙楼。摔死了国舅段文楚,唐主爷一怒要斩头。自从那年离朝后,今日里相逢在北州。

(程敬思\protect\hyperlink{fn393}{\textsuperscript{393}}
【西皮原板】自从千岁离朝后,满朝中文武泪双流。为千岁懒把朝房走,为千岁懒观五凤楼。山遥路远少来问候,望千岁恕学生礼貌不周。)

【西皮导板】太保推杯换大斗,

【西皮快板】李克用跪席前脸带惭羞。当初不该打死国舅,怒恼了唐王要斩人头。如不是恩官把本奏,孤王焉有活命留。天高地厚恩少有,这一斗水酒你要饮下喉。

(程敬思
【西皮快板】用手儿接过梨花盏,学生大胆把话言:甲子年,开科选,山东来了一生员。家住曹州并曹县,姓黄名巢字巨天\protect\hyperlink{fn394}{\textsuperscript{394}}。三篇文章作得好,试官点他为状元。跨马三日游宫苑,宫娥、彩嫔笑连天。唐王嫌他容貌丑,斩了试官革状元。斩了试官不要紧,革了状元起祸端。祥梅寺\protect\hyperlink{fn395}{\textsuperscript{395}},造了反,将我主驾逼在西祁\protect\hyperlink{fn396}{\textsuperscript{396}}美良川。学生到此无别干,一来搬兵二问安。)

【西皮快板】听说黄巢造了反,不由得孤王笑连天\protect\hyperlink{fn397}{\textsuperscript{397}}。贤弟饮宴且饮宴,提起了唐王孤不耐烦。

(程敬思
【西皮快板】我这里提起唐天子,这老儿一旁不耐烦。是是是,明白了,老儿是个爱宝男。叫人来将宝搭上殿,特请千岁把宝观。)

【西皮快板】一见珠宝帐前摆,不由得孤王笑颜开。上有蟒袍和玉带,凤冠头上插金钗。明明知道佯不解,假意儿上前问开怀。你做清官数十载,此宝打从何处来。

(程敬思
【西皮快板】此宝出在山海外,三年五载进宝来。唐王爱将恩似海,特命学生进宝来。)

【西皮快板】贤弟进宝因何故,

(程敬思 【西皮快板】特请千岁把兵排。)

【西皮快板】年纪迈,血气衰,难作国家的栋梁才。

(程敬思 【西皮快板】千岁爷虎老雄心在,黄巢闻名他不敢来。)

【西皮快板】贤弟休得把孤抬,有一辈古人说上来:昔日有个姜吕望,稳坐钓鱼台他不下来。

(程敬思
【西皮快板】钓鱼台,不下来,他保周朝八百载。千岁不发人和马,黄巢笑你老无才。)

【西皮快板】笑只笑唐天子,他笑孤王为何来。中军帐,挂了帅,众家太保两边排。一马儿踏入唐室界,万里的乾坤扭转来。

(程敬思 【西皮快板】说此话就该发人马,)

【西皮摇板】唐王晏驾你再来。

(程敬思 【西皮摇板】问千岁此宝爱不爱?)

【西皮摇板】孤念你千里迢迢路远来,却之不恭呃,受之有愧,来来来,一体全收哇往后抬。

(程敬思
【西皮快板】这老儿做事不公平,收了宝物不发兵。用手取出唐王旨,我奉圣旨来调兵。)

(程敬思 圣旨下。)

呃!

【西皮快板】程敬思做事太无情,不该圣旨欺寡人。用手拿过(或:接过)皇王旨(或:唐王旨),回手压下帝王(的)文。哪一个再提发兵事,定斩沙陀不徇情。

(程敬思
【西皮快板】一见千岁变了脸,回头埋怨李嗣源。我在松林寻短见,不该救我活命还。)

【西皮快板】奴才做事真胆大,胡言乱语少家法(或:把话答)。(或:我与恩官来讲话,大胆奴才把话答。)吩咐两旁武士手(或:刀斧手),推出帐去(或:推出午门)把头杀。

(程敬思 【西皮摇板】千岁要斩把学生斩,快快赦回太保还。)

【西皮快板】我与恩公(或:恩官)来讲话,奴才一旁(或:奴才竟敢)把话答。恩公若回长安转,耻笑孤王无家法(或:少家法)。

(程敬思 【西皮摇板】有家法来无家法,看学生薄面绕过他。)

【西皮摇板】贤弟(或:恩官)不必礼恭敬,帐外赦回太保身(或:午门赦回小畜生)。

【西皮摇板】一足将儿踏帐下(或:恨不得一足将儿踏),

【西皮摇板】程恩官讲情儿要谢过他。

【西皮导板】昔日有个三大贤,

【西皮原板】刘、关、张结义在桃园。弟兄们徐州曾失散,古城相逢又团圆。关二爷马上呼三弟,张翼德在城楼怒发冲冠。你既然降了奸曹操,看来是无义反桃园(或:负义反桃园)。耳边厢又听【转西皮快板】人呐喊,老蔡阳的人马来到了古城边。城楼上助你三通鼓,日月旌旗壮壮威严。哗喇喇打罢了头通鼓,关二爷提刀跨雕鞍。哗喇喇喇打罢了二通鼓,人有精神马又欢。哗喇喇打罢了三通鼓,蔡阳的人头落在马前。一来是老儿该丧命,二来弟兄得团圆。贤弟休回长安转,就在这沙陀过几年,落得个清闲。

{[}第四场{]}

(程敬思
【西皮快板】过了一天又一天,心中好似滚油煎。眼望长安难回转,不知唐王驾可安。)

【西皮摇板】贤弟不必(或:休得)想唐朝,长安哪有(或:焉有)此地高。沙陀国有你的乌纱帽,沙陀国有你紫罗袍。

【西皮导板】贤弟随孤哇来观瞧,

【西皮快板】队队旌旗空中飘。(在后营有的是粮和草。众家太保杀气高:)大太保亚赛温侯貌,二太保上阵似白袍。三太保上山擒虎豹,四太保下海斩龙蛟。五太保惯使开山斧,六太保手持丈八矛。七太保金枪(或:银枪)真奥妙,八太保手持青龙偃月刀。九太保双锏(或:金锏)耍得好,亚赛秦叔宝,十太保钢鞭逞英豪(或:鞭插马鞍桥)。还有个十一小太保,他的武艺好,双手能打火龙镖。哪怕黄巢兵来到,孤与他枪对枪来刀对刀。

(程敬思 【西皮摇板】众家太保武艺好,你不发兵我心焦。)

【西皮摇板】贤弟休得心内焦(或:心烦恼),当饮酒时且逍遥。来来来,吃几杯解烦恼,

(程敬思 【西皮摇板】程敬思一旁闷无聊。)

慌什么?

对她二人(或:对她们)言讲,现有长安贵客在此,少时(或:少刻)退帐再来传见。

又慌什么?

方才言过,退帐(再来)传见,为什么又来啰唣。

呃,忒以的啰嗦了。

哎呀,得罪了(,得罪了)。

罢了,你(们)二人进帐何事?

唐王无道(或:当初唐王将孤谪贬),哪有人马与他解围。

那是送与孤家的,提它则甚? (或:呃,呃------俱都入了库了。)

(这凤冠霞帔么,)呃------(也)一齐入了库了。

孤心已定,休得多言(或:不必多奏)。

嗯------孤就是不发兵。(或:孤心已定,就是不发兵呐。)

(这做什么?)

啊?慢说是三声,(就是)三十声、三百声,又有何妨(或:又待何妨)啊?

呃,(呃,呃,我)一个不发兵。

嗯,(呃,呃,这)两个不发兵。

呃,你不必前来插足啊。(或:我们之间的事体,与台驾无关呐。)

呃,(诶------)我就是不发兵。

(诶呀!)

哦,倘若来迟呢?

(啊)太保,我们商量商量。

呃,我们商量商量。

唉!

【西皮摇板】大太保本是惹祸精呐,到后宫搬来了两个夜叉妇人呐。顺水推舟我把人情送,我为你点动了(或:我为你发动了)番汉兵。

(程敬思
【西皮摇板】千岁休得人情送,学生心内明如灯,皇娘人马来点动,程敬思不领你这空头情。)

【西皮摇板】贤弟休得笑盈盈,休笑愚兄(或:孤王)我怕,(我)怕,(我\ldots{}\ldots{})怕妇人呐。沙陀国内访一访你再问一问,怕老婆的人儿我是第一名。

{[}第五场{]}

(念)白发白须似银条,胸中韬略智谋高。也是黄巢气数到呃,试试------孤的定唐刀。

唉!只因黄巢造反,勒逼唐王驾幸西祁美良。程恩官解押珠宝来到沙陀,借兵解围(或:搬兵解围)。本当不发人马(或:兵马),可笑我那两个无知的妇人,一个要发兵,一个要挂帅。发兵也罢,挂帅也罢,(或:是一个要什么挂帅,一个要什么发兵;唉,挂帅也罢,发兵也罢,)也不知怎么(或:怎样)糊里糊涂地,把一个(或:这个)前站先行,弄到孤家的头上来了。

本当不遵,怎奈她们的家法,实在地厉害呀(或:怎奈是她的家法十分地厉害)。为此紧急料理宫廷善后之事,全身披挂,校场听点。(或:为此急忙料理完毕宫廷善后之事,急忙辕门听点。)

来(来来),带马带马。

呃呃呃,你二人为何争论起来。

哦,你不是看守宫殿的老军么?

你(前)来则甚呐?

诶------两军阵前,刀枪无眼呐,倘有伤损(或:倘有差错),那还了得?(你呀,)还是养养你这老命吧!

哦,看你不出,倒有一片爱国的精神。(或:听你之言,倒有一片爱国的精神呐。)

嗯,就命你与孤带马。(或:好好好,我就教你带马。)

{[}第六场{]}

【西皮快板】又听辕门(或:耳听辕门;又听营门)放号炮,众家太保杀气高(或:众家儿郎逞英豪;或:儿郎个个杀气高)。来在辕门下鞍鞒,

呵嘿!

【西皮摇板】误卯牌悬挂(或:误卯牌高挂)要糟糕。

呃,来了!

我早就来了,怎么(说)误了呢?

传旨进去,就说孤王驾到,教她们快快迎接。(或:传话进去,就说孤王到了,教她们下位迎接于我。)

呃,我们是夫妇顺呐。

怎么讲?(或:啊?!)

呀呸!不来(下位)迎接,还则罢了,反教孤王(或:反教我)报门而进。

哼哼,呵呵,(这人马是孤家的,)我不干了,另请高明罢!

反了哇,反了哇!

【西皮摇板】大太保传令理不通,不由孤王怒气生。(或:太保传令山摇震\protect\hyperlink{fn398}{\textsuperscript{398}},不由孤王胆战惊。)

【西皮摇板】本当进帐(或:本当与她)来争论,怎奈是她的家法比国法还要狠十分。孤若不遵她的令,到晚来不教孤进孤的(或:她的)卧室门呐。

【西皮摇板】东宫不留把西宫进,

【西皮摇板】西宫也是照样行------关门熄了灯。

【西皮摇板】闹得孤(或:恼得孤)黑夜里无投奔,银安殿上把闷气生。

【西皮摇板】孤王一生好把酒来饮(或:也是孤王好心性),好酒贪杯惯坏了她们呐。

【西皮摇板】沙陀国内访一访,你再问一问,家家有本难念的经,个个观世音。叫老军与孤王【回龙】你就报门进,

【西皮摇板】上面坐定两个夜叉精。她二人狼狈为奸端了一个稳,她那里不语(或:不言)我也不作声。

(小嗓)不错,是我啊!(或:是啊,来啦!)

我这个先行,不是花钱买来的,也不是运动来的。乃是你们(或:你二人)亲自委派的。

孤王料理宫廷善后之事,一步来迟,何必(这样的)大惊小怪呀。

哎呀,糟了\ldots{}\ldots{}

(呃呃呃,还好还好啊。)

(哎呀呀,)这还了得?!哼!

这才是(或:这就是)父子亲呐。

哦,不用我了?这就好了。(或:怎么,你不用我了?)

劳您驾------

在。

得令啊!

古来无有的事,如今都有了!古来无有的事,如今都有了!

呃,(古来就有,)你且讲来。

看你不出,你还知道这么多的历史故事啊(或:倒晓得这么些个历史的知识啊)。

你讲得不错。

呃,孤(王)就是,呃,有这样(或:这么)一点点的短处。

呃,这就好了,这就好了。

呃,你不曾听见吗?

皇娘传令,赐孤(或:赐我)五千名虎卫军,压住后队。

嗯,这倒是一个美差。

哦,难道孤听错了?(或:怎么,难道孤听错了?)

呃------你这个人,怎么这样势利眼呐?

那些太保,你们看来一个个如狼似虎,他们只能在围场之上,行围射猎;在沙场之上,交锋对垒,还要看孤家。(或:呃,你不要看那些太保们,一个个如狼似虎,他们只能拿强捕盗,行围射猎;战场交锋,还要看孤家的。)

那个自然,带马!

{[}第七场{]}

【西皮快板】将令一出山摇动(或:山摇震),儿郎个个胆战惊。来在营门下金镫(或:来在辕门下能行;或:催马来在辕门近),

【西皮摇板】这样的紧急为何情呐。

哦,恩官。(或:哦哦哦,请坐请坐。)

哦,恩官。(呃呃呃,哎呀,得罪了得罪了。)

呃,(那)我的座位呢?

(哦,谢坐谢坐。)

呃,家无常礼呀。

调我前来则甚呐?(或:调孤前来有何军情议论?)

哦,谈谈心?

呃,(那)我们就谈谈心呐。

想是那周德威呀?

无名小辈,草莽贼寇,(或:草莽贼寇,无名小辈,)何足道哉?

会他一会么?(又有何妨啊?)

(哦,)今天不耐烦。

不伺候。

呃,什么叫作好处?你说将出来,我听上一听。(或:哦,我倒不晓得有什么好处,呃,有什么好处呢?)

(哦哦哦\ldots{}\ldots{})

噫------(哈哈哈\ldots{}\ldots{}(笑介))你不是骗了我一次了!我再也不上当了。(或:我再也不信了!你骗了我不是一次了!)

你说将出来,教大家听上一听(或:看上一看),看看使得使不得。

打仗也吃酒,不打仗也吃酒。这酒么------不稀罕(或:呃,吃酒不稀罕)。

呃,我们是家务事啊,不劳台驾呀。(或:诶,这是我们家务事,你不要前来插足哇。)

(你说)哪个老了?

(哦,你说孤王老了?)

(孤王)我老只老头上发,项下须,胸中韬略却还不老!

有道是:(念)虎老雄心在,这------年迈呀------力刚强。

你(呀,)拿过来吧!

【西皮二六】老只老孤的须发老,胸中的韬略比人高。非是孤王不服老,上阵全凭马和刀。草莽的贼寇何足道,教他来试一试孤的九九八十一斤定唐刀。

【西皮快板】你把酒宴安排好,得胜回来贺贺功劳。叫老军与爷带马到,

【西皮散板】会一会山寇小儿曹。

{[}第八场{]}

(堂鼓轻击,龙套在九龙口左右站斜旗门,\textless{}\textbf{撕边}\textgreater{}李克用、周德威左右旗门冲出,双出门,李从台中间左转身回到台中间漫周头,打周鼻子(周与李相反走,\textless{}\textbf{撕边}\textgreater{}李、周分别左右转身回到旗门,同时抱刀亮、龙套领起来分站左右,李、周出来架住)

(念)呔!马前来的敢是周德威?

周德威,看你相貌堂堂,为何失身落草?
(或:我看你相貌堂堂,文韬武略,不该落草为寇。)依孤相劝,归顺孤家,封你以为一家太保,你且三思。

呜哙呀,他还惦记孤(家)的珠宝哩!

唉,全都烧光了。

哼,你若胜得过孤(王的)这定唐宝刀,(孤)将珠宝与你留下。

你若不胜?

丈夫一言------

(你我各传一令!)

众将官,压住阵脚。

【西皮导板】叫三军与爷战鼓伐,

(堂鼓轻击,钻烟筒,\textless{}\textbf{冲头}\textgreater{}一合两合,搕开,\textless{}\textbf{紧锤}\textgreater{})

【西皮快板】马前闪出年少娃。量儿本领有多大,敢与老夫动杀法伐。

(一合两合,李接上下左右,刀头拉转身,李被漫头过去到小边,向里回头打周后蓬头,拉肚转身,李被勾走马腰封到大边,绞起来李压周刀,往里一盖两盖,往外一盖两盖,起大刀花蹦子转身剁周头,亮住(李平端刀指周)。\textless{}\textbf{香柳娘}\textgreater{}亮收对面互看,夸奖,双出门分别左右转身对面拉开,搕,里面拉开(\textbf{不要正冠}),搕,李回花转身到大边里面抱刀单腿立亮相(周小边外面矮相),李向外边大刀花转身到大边外面斜横刀弓箭步矮相(周小边里面亮相),李向上场门出刀转身砍过去到中间里边,面外抱刀亮相(周里面矮相),拉转身到小边,对面拉开,搕,拉开,搕,回花转身到小边外面斜横刀矮相,向里边大刀花转身到小边里面,倒手外边面里斜横刀矮相,拉,回到大边(\textless{}\textbf{牌子}\textgreater{}停),搭,拉到大边里角,一合到小边外角,从小边外角经小边内角向大边外角退,倒提柳到大边外角,横着一个大刀花过合,两个大刀花过合,又到大边,打周上下左右,勾周走马腰封到大边,李归小边,原地被勾刀转身向里接鼻子,在原地勾周刀左转身向外打周鼻子,刀鐏盖周刀再打一个鼻子,转身切刀亮住,接耍下场,普通大刀下场下,但在劈马、正花转身面外亮住、串腕转身之后,再来一个正花转身面外串腕,弓箭步拿刀杆中间向外亮,缓刀掠刀追下)
\protect\hyperlink{fn399}{\textsuperscript{399}}

【西皮散板】接过雕翎箭一条。

【西皮散板】这样的射法不算好,放箭哪有接箭高。

【西皮散板】接过雕翎箭二根。

【西皮散板】这样的射法不算准,孔夫子门前你卖的什么文。

【西皮散板】勒马停蹄战场等,停箭不射为何情?

周德威,战又不战,降又不降,又在那里弄的什么诡计?

你的箭法呀,哼,孤(王)方才领教过了。

(要)怎样比试?

但不知哪家先射?

孤若先射,就无有你的份了(或:孤王先射,就无有尔的份了!让你先射)!

站定了!

【西皮散板】量尔不是汉李广,养由基再世又何妨。

【西皮散板】满满搭上朱红扣,

(诶,)你把我闹糊涂了!

【西皮散板】观不见金钱在何方。

【西皮散板】低下头来暗思想,

啊?!

【西皮散板】忽然一计上胸膛。

(周德威 李克用,\ldots{}\ldots{}停箭不射?)

非是孤王停箭不射,想这金钱(或:想那金钱)乃是一面死物,纵然射中,也不足为奇。

抬头观看------

空中飞的何物?

好哇------孤今(或:孤王一箭上去)要射它(一)个双雕落地。

丈夫一言------

(太保,)站定了。

【西皮散板】克用暗地告上苍,祝告天地日月光、过往的神灵听端详:我若有福收此将,箭射双雕落平阳。

(哪里走。)

【西皮散板】德威可算英雄将(或:忠良将),封你太保在朝堂(或:在朝廊)。(或:你若真心把孤保,封你太保辅大唐。)

(罢了,)见过众家哥弟。

不必查点。

转至御营(或:同至御营),见过你二位皇娘。

据\textbf{陈超老师}介绍,《珠帘寨》有``\textbf{烧宫}''一场,刘曾复先生有特别传授,兹照录如下:

程敬思 (唱)李克用依记前仇恨,且喜皇娘发救兵。

程敬思
是我解宝到此搬兵,不想李克用记恨前仇不肯出兵,多蒙二位皇娘发动倾国人马,李克用以为先锋,兵马已在四十里外扎营。是我悄悄折回,将李克用的宫殿焚毁,教他此去难回也。

程敬思 (唱)非是我程敬思忒心狠,为保我主锦乾坤。

李克用 (内)【西皮导板】王宫火光冲天境,

(李克用、老军上)

老军 老千岁,您可小心这点儿。

李克用 (唱)急忙转回看分明,顾不得即出征军情紧(圆场、下马)

老军 好大火啊(李克用扑火)。都烧成这样了,您就别忙活啦!

李克用 嘿嘿,

李克用 (唱)宫殿全已被火焚。

老军 回来您再盖新的吧。

李克用 唉!(唱)孤多年以来积攒的奇珍异宝------

老军 我的被窝褥子------

李克用 (唱)顷刻之间就化为飞灰,叫孤怎不心疼啊。

老军
老千岁您不错啦!我连被窝褥子都没啦!(\textless{}\textbf{三鼓}\textgreater{})

老军 完了!聚将鼓响了,咱们再不走可来不及啦!

李克用 (唱)不住战鼓来催命,

李克用 马来

李克用 (唱)此一去定回归重建宫廷。

\newpage
\hypertarget{ux592aux5e73ux6865}{%
\subsection{太平桥}\label{ux592aux5e73ux6865}}

{[}第一场{]}

(四绿龙套大锣打上,站门,\textless{}\textbf{四击头}\textgreater{}朱温上)

朱温
\textless{}\textbf{点绛唇}\textgreater{}光照旌旗,青虚钓实,龙唾涕,惊走鳌鱼,要掌锦华夷。(小座正座)

朱温
(诗)忆昔当年雅观楼\protect\hyperlink{fn400}{\textsuperscript{400}},赌头夺带面惭羞。孤今驻军汴梁地,安排巧计报前仇。

孤,梁王朱温,今有李克用兵扎驼龙岗,也曾命王刚打探虚实,未见回报。

王刚 (内)走哇!(王刚上)

王刚 打听晋王事,报与驸马知。参见驸马。

朱温 罢了。

王刚 谢驸马。(站小边)

朱温 命你打探晋王虚实,有何消息?

王刚
启禀驸马,晋王兵扎驼龙岗,命十三太保李存孝巡查黄河,营中空虚。晋王每日饮酒,醉而复醒,醒而复罪,不理军情,特来禀报。

朱温
起过了。且住。老儿一到训地不理军情,存孝不在营中,我不免趁此机会,请他汴梁阅兵,会上杀他,以报前仇。左右溶墨伺候。(朱温进大座)

朱温
【西皮导板】上写朱温多拜上,【导板】拜上千岁李晋王。皇姑宫院身不爽,思念天子泪成行。千岁驾临驼龙岗,未获拜趋在道旁。汴梁设下阅兵会,酒席宴前饮琼浆。写罢书信唤王刚,驼龙下书走一场。

(王刚接书下,朱温出位站台中间)

\begin{quote}
【摇板】汴梁设下天罗网,管教亚儿丧无常。(朱温众下)
\end{quote}

{[}第二场{]}

(四龙套站门,周德威、史敬思、李克用上,李中,周大边,史小边站台口)

李克用 旌旗遮日月。

周德威、史敬思 (同念)龙虎保乾坤(或:龙虎扶乾坤)。

(李克用中间小座)

周德威、史敬思 (同念)参见父王。

李克用 二皇儿免礼,赐座。

周德威、史敬思 (同念) (谢父王。)儿臣谢座。

周德威 将军,请坐。

史敬思 先生,请坐。

(周大边,史小边八字座)

李克用 二位皇儿。

周德威、史敬思 (同念)父王。(将儿臣唤出,有何军情议论?)

李克用 我父子兵扎驼龙岗,个月有余,缘何不见朱温前来问安?

周德威、史敬思 (同念)他乃叛逆之臣(或:他乃叛逆之人),提他做甚!

李克用 来。

龙套 有。

李克用 伺候了。

(王刚上,台口小边)

王刚 领了一件事,千金不敢移。来此已是,营门哪位听事?

龙套 做什么?

王刚 烦劳通禀千岁,就说朱驸马差来下书人求见。

龙套 营外稍站,待我禀报。启禀千岁,朱驸马差来下书人求见。

李克用 教他自进。

龙套 千岁命你自进,要小心了。

王刚 有劳了。(王刚进帐,跪)

王刚 参见千岁。

李克用 罢了,立起讲话。

王刚 谢千岁。(王站大边)

李克用 你奉何人所差?

王刚 今奉朱驸马所差,这有书信呈上。

(王呈信、李接)

李克用 命你回禀驸马,孤照书行事。

王刚 谢千岁。

(王刚下)

李克用 二位皇儿。

周德威、史敬思 (同念)父王。

李克用 朱温有书信到来,哪位皇儿观看。

周德威、史敬思 (同念)父王御览。

李克用 待孤拆书,父子同观。(\textless{}三枪\textgreater{}李克用看信)

李克用 呜哙呀,原来朱温设下阅兵大会,请孤汴梁赴会,哪位皇儿保驾?

周德威 十三太保不在营中,无人保驾。

史敬思 先生,俺史敬思不才,愿(带)领四十名长枪手,保定父王汴梁赴会。

周德威 保得去?

史敬思 保得去。

周德威 保得归?

史敬思 保得归。

周德威 未必!

史敬思 为何?

周德威
自盘古以来,只有湘江、临潼大会,无有什么阅兵大会,我想那朱温多奸多诈,父王还是不去为妙。

李克用
皇儿。【西皮原板】皇儿说话差又差,为父言来听根芽。朱温皇宫招驸马,金枝玉叶招赘他。转面我把德威唤,【摇板】你在那八卦之中仔细查。

周德威 儿遵命。(周德威立,大边台口)

周德威
【西皮摇板】父王命我查八卦,【快板】背转身来仔细查。一请前朝文王卦,二请周公与桃花。三才四象安天下,五行六爻定邦家。七星袖内查八卦,啊,【摇板】白虎当头有凶煞。(白)将军,(接唱)我劝将军休保驾,此去难免动杀伐。

史敬思
先生。【二六】先生说话理太差(或:言太差),长他人的威风灭却咱。战场交锋(或:交锋对垒;两军阵前;战场之上)如戏耍,我把那朱温当作小娃。上阵全凭胯下马,虎头金枪掌中拿。倘若是席前有奸诈,学一个单刀赴会名扬天涯(或:学一个单刀赴会万古夸)。

周德威
【摇板】将军不听我的话,再把言语叮咛他。(白)将军,(接唱)你此去逢桥休下马,

史敬思 (为何?)又是为何?

周德威 (接唱)``太平''二字谨防它。

史敬思
【摇板】先生休说懦弱话,非是末将把口夸。(或:先生说话理太差,末将言来听根芽:)大丈夫生至在三光下,生死二字何惧他。

李克用 周德威听令。

周德威 在。

李克用 为父赐你大令一支,命你镇守驼龙岗。

周德威 得令。

李克用 看衣改换。

(牌子合龙,李、史换衣介,四上手上)

李克用
【西皮摇板】人来带过白龙马,(上手带马,李克用、史敬思上马,上手、史下,李收腿)

李克用 (接唱)汴梁阅兵免征杀。(周德威送,李克用下。周归中间)

周德威 【西皮摇板】敬思不听我的话,(龙套斜撤)

周德威 (接唱)太平桥前有凶煞。

(周德威、龙套下)

{[}第三场{]}

(四绿龙套站门引朱温上)

朱温
【西皮摇板】汴梁设下阅兵会,要把克用性命追。将身且坐宝帐内,(小座正座)等候王刚下书回。

(王刚上,进门大边站)

王刚 晋王驾到。

朱温 传卞意随进见。

王刚 卞意随进见。

卞意随 (内)来也。(卞意随上)

卞意随 驸马传唤,急到帐前。

(卞意随进门参见、小边站)

卞意随 参见驸马,有何将令?

朱温 命你埋伏太平桥下,刺杀李晋王不得有误。

卞意随 得令。(卞意随下)

朱温 吩咐众将,摆队相迎。

王刚 摆队相迎。

(起牌子,龙套摆队反下,王刚、朱温反下)

{[}第四场{]}

(牌子中史敬思上,马上起霸,勒马站中场,四上手引李克用上过场,史敬思小趟马亮相下)

{[}第五场{]}

(牌子中绿龙套反上斜胡同,朱温反上望上场门场,四上手上斜胡同,李克用上,李下马)

李克用 驸马。

朱温 千岁,来在长亭,千岁请往前行。

李克用 驸马请往前行。

朱温 这就不敢,千岁前行。

李克用 你我挽手而行。(同笑介)

(李克用下,四上手随下,史敬思上扎过去,朱温众归小边,史敬思亮相下,朱温背供指,做杀介,领众下)

{[}第六场{]}

(牌子中四龙套、四上手上挖门,各走各边,李克用、朱温上,进门李坐大边,朱坐小边,牌子停)

朱温 千岁驾到,本宫未曾远迎,千岁恕罪。

李克用 岂敢,某来得鲁莽,驸马海涵。

朱温 岂敢。

李克用 告便。

(李克用、朱温立)

朱温 千岁意欲何往?

李克用 探望皇姑疾病。

朱温 皇姑闻得千岁驾到,疾病已然痊愈。

李克用 此乃驸马洪福。

朱温 全仗千岁虎威。

李克用 啊?

朱温 啊,(同笑介)

朱温 千岁请坐。

李克用 请坐。(同坐)

李克用
请问驸马,自盘古以来,只有湘江、临潼大会,未闻有何阅兵大会,驸马指教。

朱温 阅兵会上不过是水酒薄肴,与千岁同饮。

李克用 如此说来,到此就要叨扰。

朱温 千岁后宫请。

李克用 请。

(起牌子,李克用、朱温,龙套、上手两边分下)

{[}第七场{]}

(牌子中史敬思上,王刚反上,史、王回身望门,二人见面,史用手漫王头过去,回身右脚蹬王弓箭步左腿上,史拔剑三笑,剑不出鞘,漫王头,退望王,双撩下甲,转身下,王望,比势史身高,杀,怕介下)

{[}第八场{]}

(牌子中李克用、朱温、史敬思、王刚上,李进门,史挡朱随李进,朱、王进,李、朱分坐八字桌大座,牌子停)

朱温 开宴。

(王刚归中间站)

王刚 上宴。

(史敬思轰开,拔剑两望,挑桌袱搜桌下,站台中间亮相)

朱温 千岁请。

李克用 驸马请。

(史敬思、王刚退桌外侧,牌子,朱温、李克用饮酒,王撞钟介)

史敬思
啊,\textless{}\textbf{撞金钟}\textgreater{}【西皮摇板】忽听(或:又听)金钟一声响,

(绿龙套上,站小边一字,史出门双望,回中间)

史敬思 (接唱)刀枪剑戟列两旁。转面我对(或:上前忙对)父王讲,

(史敬思拉李克用出位站中间,史站旁边,朱温出位小边)

史敬思 (接唱)儿臣言来听端详。今日饮酒休放量,酒席宴前要提防。

李克用
皇儿呀。(接唱)皇儿不必心慌忙,为父言来听端详。汴梁纵有千员将,我儿保驾料无妨。

朱温
千岁,(接唱)千岁说话有志量,本宫言来听端详。千岁好比刘先主,太保亚赛关二王。

李克用 (接唱)驸马说话孤心爽,不由克用喜洋洋。人来将酒满斟上,

(李克用进位饮酒,史敬思暗拉,李饮,史急)

李克用 (【转散板】)多吃几杯又何妨。吃酒要学刘伶样,(念)酒来,

(史敬思再暗拉,李克用饮,史急,李饮)

李克用 (接唱)太白斗酒诗成行。

(李克用醉,王刚又撞钟介,史敬思惊介)

史敬思
哎呀,【散板】又听金钟二次响,此地一定有埋藏(或:倒教豪杰着了忙)。二次再对父王讲,(念)父王,

(拉李克用出位,朱温随出位)

史敬思 (接唱)请出皇姑问端详(或:做主张)。

李克用 (接唱)转面我对驸马讲,请出皇姑问安康。

(史敬思 有请皇姑。)

朱温 有请皇姑。

王刚 有请皇姑。

(公主上)

公主 【西皮摇板】耳听前帐声喧嚷,见了驸马问端详。(进门,站小边里边)

李克用 儿呀,前来见过皇姑。

史敬思 (参见)皇姑。

公主 罢了,你父子到此做甚?

李克用 赴阅兵大会。

公主 哪里是阅兵大会,你父子快快回去。

(朱温打公主嘴巴,公主下,王刚抓李克用,史敬思拔剑杀王,回身抓朱温带,大推磨,李逃下,史放朱带,史亮相下)

朱温 带路进宫。(朱温众下)

{[}第九场{]}

(公主上)

公主
【西皮散板】心中只把驸马恨,要害兄王为何情?(念)且住,驸马屡次要害兄王,是我今日走漏消息,驸马回宫,岂肯与我甘休,也罢,我不免拜谢父王母后养育之恩,自尽了罢,(接唱)走近前来忙跪定,拜谢父母养育恩,手执钢刀来自尽,罢!(刎下)

(朱温众上,其中一龙套带马鞭,挖门进宫介)

朱温 啊,【西皮散板】一见贱人丧了命,怎不叫人咬牙根,手执宝剑来砍定。

(砍三剑介)

朱温 马来。

(朱温上马,龙套领下)

{[}第十场{]}

(\textless{}\textbf{水底鱼}\textgreater{}史敬思,李克用上,史卸靠,剑插大带中,李褶马褂)

李克用 儿呀,那贼四门紧闭,如何是好?

史敬思
父王,休得(或:不必)惊慌,西南角下(或:西北角下)有一水门,你我父子托闸而走(或:托闸而逃;托闸出城)。

(史敬思、李克用走圆场,见闸,二人下马,史躬身,马交李带,史中场,李小边,史拔剑砍左闸门大横栓环,砍三下,有尘土,小边里边挡脸,同样砍右边栓环,砍中间锁,剑插带中,卸大横栓,放大边,推门,拉开门,开左扇门,开右扇门,假岔,见闸,靠闸,拔剑豁闸底土左右左三下,插剑,与李耳语,一枕,两枕,紧带,岔,托闸,李拉二马钻闸出城,二人上马下)

{[}第十一场{]}

(龙套引朱温上,报,朱众追下,朱用双锏)

{[}第十二场{]}

(卞意随上)

卞意随
(念)领了驸马令,埋伏太平桥。我卞意随,领了驸马将令,在太平桥下行刺,就此埋伏者。(过下场门桥,桥后藏身)

史敬思 (内)马来。(上,勒马站)

史敬思
【西皮散板】人困马乏难交战,不知父王落哪边(或:不见父王在哪边;或:不见父王落哪边)。(念)且住,我与朱温勇战一日一夜,也不知父王逃往何方去了?(望桥)来此已是太平桥,太平桥,呜哙呀,是我保驾临行之时,先生对我言讲,逢桥休下马,太平要提防。待俺加鞭催马过桥。(或:来此已是太平桥,太平桥。且住,临行之时,先生也曾言过(或:先生也曾嘱咐),教我见桥休下马,俺不免打马过桥。)

(打马上桥,马见水中人影,倒退不行,史敬思惊介)

史敬思
且住,看此桥,桥身高大,龟背鱼脊,若不下马怎能(或:焉能)得过。哎呀,俺史敬思一生一世,就是不信那些阴阳八卦,有道是圣天子百灵相助,大将军八面威风,俺今日偏偏要下得马来,牵马过桥。(或:哎呀,俺史敬思一生一世,就是不信那些鬼阴阳八卦,俺今日偏偏要下马过桥。或:哎呀,俺史敬思一生一世,就是不信那些阴阳八卦,俺今日偏偏要下马过桥。)

(下马,拉马上桥,马退,再拉上桥,卞意随从桥后刺史敬思腹中,史左手抓枪,史由桌上退到椅上,卞上桌,史扔马鞭,拔剑砍卞,一二三漫头,向后扔剑,摘盔扔上场门边,由检场接,捋甩发,双手握枪头,在椅上下腰,松枪,蹬椅边,摔硬僵尸头朝小边台口)

(卞意随三笑,由桥后下场门下,李克用上,下马,扶史起坐地,李站大边)

李克用 皇儿醒来。

史敬思
【西皮导板】耳旁又听(或:耳旁听得)父王到,(念)(唉,)父王呀!(史跪)

史敬思
(接唱)【散板】抬头只见(或:开言禀告)老年高。臣子年幼妻年少,一家老小无下梢。

李克用 (接唱)皇儿但把心放了,一家大小永在朝。

史敬思 (接唱)多谢父王加封号。

(鼓架子,史敬思立)

李克用
(接唱)朱温人马似涌潮,(念)哎呀儿呀,朱温人马犹如潮水一般,如何是好?

史敬思
父王休得惊慌,将战袍割下半幅,与儿包裹伤痍,儿与那贼决一死战。(或:父王,休得惊慌,将战袍割下半幅与儿包裹伤痕,与那贼决一死\ldots{}\ldots{};或:父王不必惊慌,将战裙割下半幅,与儿包裹伤痕,与那贼决一死\ldots{}\ldots{})

(李克用扶史敬思裹伤,史里边面内,李过小边,朱温众追过场)

李克用
【西皮散板】眼前若有李存孝,哪怕朱温计千条。(或:\ldots{}\ldots{}闹吵吵。)

史敬思
(接唱)有劳父王带马到,哎呀,伤痍(或:伤痕)疼痛似火烧,咬定牙关跨虎豹(或:含悲忍泪战场到;或:含悲忍泪跨虎豹;或:咬定牙关战场到)。

(李克用带马,史敬思上马,李抄下,卞上)

卞意随 哪里走?

(卞意随漫头过大边,史敬思小边,架住)

(史敬思 来将通名。)

(卞意随 卞意随。)

史敬思 太平桥行刺可是尔?

卞意随 正是你老爷。

(史敬思 放马过来。)

(史敬思夺卞意随枪,杀卞倒地)

史敬思
【西皮散板】只说儿的武艺好(或:适才道尔武艺好),老爷看来不为高(或:依我看来也不高)。这也是儿眼前报(或:这也是尔现成报)。

(史敬思拿枪刺卞意随腹,剑柄打枪鐏三下,枪劐卞腹,扔枪下,朱温上,史杀,朱败下,史追下)

{[}第十三场{]}

(朱温众上)

朱温 弓箭伺候。

史敬思 (内念)哪里走?

(史敬思 罢!)\\
(史敬思上,中箭下,朱温众追下)

{[}第十四场{]}

(李克用上,史敬思上,下马,李下马扶史上桌,拔箭,史下桌,向大边外指,做朱温兵来状,李望,史拔李剑自刎下,李哭介,拾剑上马,朱温上,李败下,朱追下)

{[}第十五场{]}

(四红龙套飞虎旗引李存孝牌子上)

李存孝
俺,十三太保李存孝。奉了父王将令,巡查黄河一带等处,巡查完毕,回营交令。(鼓架子)耳旁听得人马呐喊,军士们,登高一望。

(上中间桌子,朱温追李克用过场,存孝下桌)

李存孝 且住,原来朱温追赶我父王,此时不救等待何时?军士们,迎上前去。

(存孝站大边台口椅上,存众大边一字。李克用上,存孝招手,李下。朱温追上,存众小边一字,朱见存孝,存孝用挝三漫朱头,朱领众从外边抄过去下,与此同时,存孝下椅,存众从里往外小边走,存孝最后押队,绕回来,从下场门追下)

{[}第十六场{]}

(朱温众上,下场门边拉城)

朱温 扯起吊桥。

(朱温众进城下,关城门,李存孝众同上,城里摇旗,免战锣)

李存孝 收兵。

(存孝众上场门下,存孝大边台口亮相下。)

(收城。\textless{}\textbf{尾声}\textgreater{})

\newpage
\hypertarget{ux4e09ux51fbux638c-ux4e4b-ux738bux5141}{%
\subsection{三击掌 之
王允}\label{ux4e09ux51fbux638c-ux4e4b-ux738bux5141}}

\textbf{{[}引子{]}一枝花抛出墙外,为三女,常挂心怀。}

\textbf{(念)食禄君恩数十秋,辅保吾主坐龙楼。皇恩浩荡须答报,赤胆忠心直到头(或:赤胆忠心不到头)。}

\textbf{老夫王允。唐帝驾前为臣,官居当朝首相。夫人陈氏,膝下无儿,所生三女:长女金钏,许配苏龙为妻;次女银钏,许配魏虎为室;惟有三女宝钏,生性高傲,是她为母之病,在后花园中许下心愿,拈香三载。后宫娘娘闻知见喜,恩赐我儿五色绒线,织成彩球一朵。也曾择于二月二日,
在十字街头,(高搭彩楼,)抛球招赘。实望}\protect\hyperlink{fn401}{\textsuperscript{401}}\textbf{打中哪家王孙公子,不想打中了乞丐花郎薛平贵。想我儿乃千金之体,焉能与那花郎匹配?}

\textbf{老夫今日早朝,(在金殿之上,)观见新科状元姓蔡名端,人才出众,我意欲将花郎亲事打退,另将我儿改配}\protect\hyperlink{fn402}{\textsuperscript{402}}\textbf{那新科状元。(也不知她的心意如何,我)不免将她唤出堂前。与她商议。}

\textbf{家院,后堂传话:三姑娘出堂。}

\textbf{我儿罢了!(或:我儿免礼。)}

\textbf{一旁坐下。}

\textbf{恭喜我儿,贺喜我儿。}

\textbf{我儿在十字街头,(高搭彩楼,)抛球招赘,岂非一喜?}

\textbf{但不知那一彩球打中了哪家王孙公子?}

\textbf{唉,哪里是什么王孙公子,就是那乞丐花郎薛平贵。}

\textbf{儿呀,不必如此。}

\textbf{(想我儿千金之体,焉能与那花郎匹配?)为父今日早朝,观见新科状元姓蔡名端,人才出众,为父意欲将花郎亲事打退,另将我儿改配那新科状元。}

\textbf{不知我儿意中如何?(或:也不知儿的心意如何?)}

\textbf{罢了,恕你无罪(或:一旁坐下)。}

\textbf{坐下。}

\textbf{抱什么?}

\textbf{难道说这一彩球就定了儿的终身不成么?}

\textbf{啊,为父的与儿(或:与你)讲话,难道儿与为父的致气}\protect\hyperlink{fn403}{\textsuperscript{403}}\textbf{不成(或:你敢是与为父生气不成)?}

\textbf{既然不与为父致气,就该打退花郎亲事才是。}

\textbf{儿才怎讲(或:儿待怎讲)?}

\textbf{儿就该------}

\textbf{掌嘴!}

\textbf{【西皮原板】小奴才说此话全然不想}\protect\hyperlink{fn404}{\textsuperscript{404}}\textbf{,不由得年迈人怒满胸膛。你大姐配苏龙户部执掌啊;你二姐配魏虎兵部侍郎。唯独你小冤家娇生惯养,千金体配花郎}\protect\hyperlink{fn405}{\textsuperscript{405}}\textbf{脸面无光。}

\textbf{【西皮原板】薛平贵生来命运低,每日里在长街叫化行乞。衣衫不周实实的褴褛,他好比失林鸟无枝可栖。}

\textbf{【西皮原板】我的儿既知【转西皮快板】古礼义,可知晓韩信、张良魏苏秦。}

\textbf{【西皮快板】登台拜帅是韩信,未央宫斩的什么人。}

\textbf{【西皮快板】董永卖身葬父母,仙姬女下凡配何人呐。}

\textbf{【西皮快板】奴才说话理不顺,叫骂为父你为何情。}

\textbf{【西皮快板】要退要退偏要退。}

\textbf{【西皮摇板】我儿不遵为父命,两件宝衣脱下身(或:脱离身)。}

\textbf{乃是圣上所赐。}

\textbf{以表君臣之义(或:君臣情谊)。}

\textbf{哎呀儿啊,只要我儿将花郎亲事打退(或:只要儿打退这门亲事),慢说是两件宝衣,就是这府内的金银,都任儿取用(或:也任儿取用)啊!}

\textbf{哪里去?}

前堂无有儿父,后堂焉有儿母。

\textbf{家院、丫鬟,哪个(或:有人)去至后堂,打折(尔等的)两腿!}

\textbf{为父的怎样(的)把心死了呢?}

\textbf{哎呀儿啊,方才为父的言过:只要我儿打退花郎的亲事(或:哎呀儿啊,只要儿将这门亲事打退,这)府内的金银是任儿取用}\protect\hyperlink{fn406}{\textsuperscript{406}}\textbf{啊!}

\textbf{裁女不裁父!}

\textbf{裁为父的何来?}

\textbf{为父嫌贫爱富(我)为的是哪(一)个啊?}

\textbf{就为的是你这个小冤家啊!(或:我为的就是你这个小冤家呐!)}

\textbf{【西皮快板】膝下无儿怨我的命,养不得老来,儿送不得终。}

\textbf{【西皮快板】若是为父身染病,自有煎汤下药人。}

\textbf{【西皮快板】倘若是为父遭不幸,自有披麻戴孝人。}

\textbf{【西皮快板】父死不见王宝钏,后来若是来相见------}

\textbf{为父的不信!}

\textbf{嚯------}

\textbf{【西皮摇板】活活地气坏了哇我年迈的人呐。}

\textbf{唉!}

\textbf{【西皮摇板】无奈何(或:莫奈何)与我儿三击掌。}

\textbf{罢!}

\textbf{儿啊,呃\ldots{}\ldots{}(哭介)}

\textbf{【西皮摇板】一见宝钏出府门,怎不教人珠泪淋(或:两泪淋)。}

\textbf{【西皮摇板】含悲忍泪后堂进,见了夫人说分明。(或:悲悲切切后堂进,见了夫人定计行。)}

\textbf{儿啊,呃\ldots{}\ldots{}(哭介)}

\newpage
\hypertarget{ux8d76ux4e09ux5173-ux4e4b-ux859bux5e73ux8d35}{%
\subsection{赶三关 之
薛平贵}\label{ux8d76ux4e09ux5173-ux4e4b-ux859bux5e73ux8d35}}

\textbf{{[}第一场{]}}

\textbf{{[}引子{]}驾坐西凉,蒙公主,辅保孤王。}

\textbf{(念)离长安一十八载,思宝钏常挂心怀。恨魏虎将孤谋害,这冤仇何日解开!}

\textbf{孤,薛平贵,大唐人氏。只因当年征战西凉,可恨魏虎将我谋害,用酒灌醉,绑在红鬃烈马之上,赶至两军阵前,被公主擒获,多蒙老王不斩,反将公主匹配。不幸老王晏驾,文武百官辅孤登基。孤继位以来,风调雨顺,国泰民安,算来一十八载。今当设立早朝。}

\textbf{内侍,闪放龙门。}

\textbf{有这等事?}

\textbf{待孤观看。}

\textbf{宾鸿大雁,口吐人言,不祥之兆。}

\textbf{内侍,弓弹伺候。}

\textbf{【西皮散板】自从盘古立地天,宾鸿哪有吐人言。内侍看过弓和弹,对准宾鸿撒了弦。}

\textbf{呈上来。}

\textbf{哎呀!}

\textbf{【西皮导板}】见血书不由人泪流满面,

\textless{}\textbf{三叫头}\textgreater{}宝钏!贤妻!唉,妻呀\ldots{}\ldots{}(哭介)

\textbf{【西皮慢板】点点珠泪洒落胸前。常随官与孤王(或:替孤王)把朝散,撩龙袍端玉带孤离银安。展开了血书从头看(或:展开了罗衫仔细看),字字行行看周全:上写着拜上啊多拜上,拜上了平贵无义(儿)男。自从分别【转西皮二六】汾河岸(或:西河岸),光阴不觉十八年。夫在西凉常征战,妻在寒窑伴月眠。早来三日【转西皮快板】还相见,迟来三日难团圆。看罢血书望长安(或:肝肠断),}

\textless{}\textbf{哭头}\textgreater{}\textbf{王三姐呀,妻宝钏,啊,受苦妻呀。(或:宝钏我的妻啊!)}

\textbf{【西皮快板】想起了魏虎怒冲冠。有朝一日长安转,仇报仇来冤报冤。低下头来心暗算,}

\textbf{【西皮快板】忽然一计上心间。二次撩袍上银安,代战公主把驾参。}

\textbf{平身。}

\textbf{赐座。}

\textbf{公主连日操演人马,甚是辛苦,备得酒宴,与公主同饮。}

\textbf{内侍,看酒。}

\textbf{公主请!}

\textbf{【西皮原板】夫妻们对坐饮琼浆,她哪知血书袖中藏(或:有衷肠)。本当实言对她讲,还须要谨开口慢作商量。}

\textbf{【西皮原板】西凉本是王执掌,怕只怕南朝动刀枪(或:怕南朝兴兵动刀枪)。虽然公主你的(或:公主你虽然)韬略广,你一人怎敌百}万儿郎。

\textbf{【西皮摇板】公主进酒王心爽,多吃几杯又何妨。}

\textbf{公主操演辛苦,孤王要敬酒三杯。}

\textbf{从先能饮多少?}

\textbf{如今呢?}

呵呵哈哈哈\ldots{}\ldots{}(笑介)

还是一样啊!

\textbf{来,看大杯伺候。}

\textbf{公主请!}

\textbf{干!}

\textbf{请------}

\textbf{干!}

\textbf{请------}

\textbf{你也醉了!}

\textbf{【西皮快板】公主醉倒银安殿,中了平贵巧机关。内侍与孤把衣换,}

\textbf{【西皮摇板】番邦令箭带身边。内侍带马休迟慢,}

\textbf{【西皮摇板】难舍公主十分贤。桌案现有笔和砚,}

\textbf{【西皮快板】手提羊毫写周全:你若念在夫妻义,带领人马到关前;你若不念夫妻义,西凉女王坐江山(或:西凉国改作女儿川)。书信放在龙书案,}

\textbf{带马!}

\textbf{【西皮摇板】公主醒来对她言,说王去阅边。}

\textbf{{[}第二场{]}}

\textbf{开关!}

\textbf{奉了公主将令,出关另有公干。}

\textbf{令箭在此。}

\textbf{{[}第三场{]}}

\textbf{开关!}

\textbf{且住,来此三关地带。乃是他国地界,看城上好像莫老将军,待我冒叫一声。}

\textbf{莫老将军请了!}

\textbf{先行平贵在此。}

\textbf{何出此言?}

\textbf{此乃仇人咒骂于我。}

\textbf{红鬃烈马为证呐。}

\textbf{老将军,后面追兵甚急,快快开城再来叙话。}

\textbf{有劳了。}

\textbf{来了!}

\textbf{老将军有何话讲?}

\textbf{有劳了!}

\textbf{【西皮快板】莫老将军对我言,公主领兵(或:带兵)到关前。}

\textbf{【西皮摇板】放心不下敌楼看,}

\textbf{【西皮快板】旌旗遮住半壁天。马达、江海一声唤,快请公主到关前,【转西皮摇板】王有话言:}

\textbf{【西皮快板】那一日驾坐银安殿,宾鸿大雁口吐人言。手持金弓银弹打,打下了半幅血罗衫。展开罗衫仔细看(或:从头看),才知长安(或:才知道寒窑受苦的)王宝钏。非是孤(或:我)私自离宫院,为的长安王宝钏。}

\textbf{是孤的前妻呀!}

\textbf{【西皮摇板】本当与你说真言,公主不放也枉然。}

\textbf{【西皮摇板】一见公主变了脸,不由平贵(或:不由孤王)心胆寒。眼望长安难回}\textless{}\textbf{哭头}\textgreater{}\textbf{转,(宝钏)我的妻呀!}

\textbf{【西皮摇板】夫妻们见面难上难。}

\textbf{【西皮摇板】多蒙公主开恩典,放我平贵转回还。马达、江海一声唤,(孤王言来听根源:人马休回西凉转,)就在此地(或:关前)扎营盘。}

\textbf{【西皮摇板】交还鸽儿金鈚箭,到长安会一会妻宝钏(或:王氏宝钏)。}

\textless{}\textbf{三叫头}\textgreater{}\textbf{公主!我妻!唉,妻啊(或:公主啊)\ldots{}\ldots{}(哭介)}

\textbf{罢!}

\newpage
\hypertarget{ux6b66ux5bb6ux5761-ux4e4b-ux859bux5e73ux8d35}{%
\subsection{武家坡 之
薛平贵}\label{ux6b66ux5bb6ux5761-ux4e4b-ux859bux5e73ux8d35}}

\textbf{{[}第一场{]}}

\textbf{(内)【西皮导板】一马离了西凉界}\protect\hyperlink{fn407}{\textsuperscript{407}}\textbf{,}

\textbf{【西皮原板】不由人一阵阵泪洒胸怀。青的山绿是水花花世界,薛平贵好一似孤雁归来。那王允在朝中官居太宰,他把我贫穷人哪放在心怀。恨魏虎是内亲将我谋害,苦害我薛平贵所为何来。柳林下拴战马武家坡外,}

\textbf{【西皮摇板】见了那众大嫂细问开怀。}

\textbf{列位大嫂请了。}

\textbf{并非失迷路途,我乃找名问姓的。}

\textbf{(提起此人,大大有名,)王丞相之女,薛平贵之妻,王氏宝钏。}

\textbf{为何(或:怎么)不凑巧?}

\textbf{如今呢?}

\textbf{烦劳大嫂转达一声,就说他丈夫(与她)带来万金家书,教她前来接取。}

\textbf{有劳了。}

\textbf{【西皮原板】这大嫂去送信【转西皮快板}】太也迟慢,武家坡站得我两腿酸。下得坡来用目看,见一位大嫂把菜剜。前影好似王三姐,后影儿又像妻宝钏。本当上前把妻唤,错认了民妻礼不端。

\textbf{大嫂请了。(或:大嫂请来见礼。)}

\textbf{并非失迷路途,我乃找名问姓的。}

\textbf{提起此人,大大有名,就是那王丞相之女,薛平贵之妻,王氏宝钏。}

\textbf{非亲。}

\textbf{非故。}

\textbf{(大嫂有所不知,)我与她丈夫同营吃粮,与她带来万金家书,故而动问。}

\textbf{(我那薛大哥言道,书信么,要面交本人。)}

\textbf{(原书带回。)}

\textbf{请便。}

\textbf{这哑迷么,略知一二。}

\textbf{远在天边,不能相见。}

\textbf{哦,莫非你就是薛大嫂么?}

\textbf{哎呀呀,问来问去,问到本人的头上来了。}

\textbf{来来来,重见一礼呀。}

\textbf{礼多人不怪呀。}

\textbf{大嫂请稍待。}

\textbf{哎呀且住,想我平贵离家一十八载,不知她光景(到底)如何?}

\textbf{嗯,嗯,嗯\ldots{}\ldots{}我自有道理!}

\textbf{【西皮快板】洞宾曾把牡丹戏,庄子也曾戏过妻。秋胡戏过了罗敷女,薛平贵调戏自己妻。弓韔袋}\protect\hyperlink{fn408}{\textsuperscript{408}}\textbf{中摸一把,}

\textbf{哎呀!}

\textbf{【西皮快板】我把大嫂的书信失。}

\textbf{失落了。}

\textbf{弓韔袋中。}

\textbf{正是紧要的所在啊。}

\textbf{呃,呃,呃\ldots{}\ldots{}想是我前村抽弓打雁------}

\textbf{打雁充饥呀。}

\textbf{诶------一封书信,能值几何,你怎么开口伤人(或:出口伤人)呐?}

\textbf{哎呀呀,到底是丞相之女,出口便是文呐(或:出口成文)。}

\textbf{啊大嫂,你不要着急呀,这书信上的言语,呃,我还记得几句。}

\textbf{明白何来?}

\textbf{诶,(不是哟,)私看人家的书信是有罪名的呀。}

\textbf{呃,我那薛大哥修书的时节,我在一旁打点行李,我偷看了几句,倒是有的。}

\textbf{我若有心呐,还不失落你的书信呢。}

\textbf{呵呵哈哈哈\ldots{}\ldots{}(笑介)}

\textbf{【西皮导板】八月十五月光明呐,}

\textbf{军营中苦得很呐,哪有许多灯火。}\protect\hyperlink{fn409}{\textsuperscript{409}}

\textbf{【西皮原板】薛大哥在月下修书文呐。}

\textbf{(王宝钏 【西皮原板】我问他好来,)}

\textbf{【接西皮原板】他倒好,}

\textbf{(王宝钏 【西皮原板】再问他安宁,)}

\textbf{【接西皮原板】倒也安宁。}

\textbf{(王宝钏 【西皮原板】三餐茶饭,)}

\textbf{【接西皮原板】小军造,}

\textbf{(王宝钏 【西皮原板】衣服破了)}

\textbf{【接西皮原板】自己补缝。}

\textbf{【西皮原板】薛大哥这几年运不通,在西凉军营中受了酷刑}\protect\hyperlink{fn410}{\textsuperscript{410}}\textbf{。}

\textbf{呃,(不错)正是挨了打呀。}

\textbf{一捆四十。}

\textbf{大嫂不要啼哭,这苦哇------}

\textbf{还在后头呢。}

\textbf{【西皮原板}\protect\hyperlink{fn411}{\textsuperscript{411}}\textbf{】在营中失落了一骑马,}

\textbf{自然是官马呀。}

\textbf{哼,哪怕他不赔。}

\textbf{自然有哇------}

\textbf{【西皮原板】为赔马借了我十两纹银。}

\textbf{一份。}

\textbf{也是一份。}

\textbf{大嫂你有所不知呀,我那薛大哥啊,原先么,本是个好人呐。}

\textbf{后来他学坏了。交了些无业的游民,吃喝嫖赌(或:浪荡逍遥),呃,无所不为,把一份钱粮俱都花费。不怕大嫂你笑话,为军的我乃是个贫寒出身呐,从来不晓得什么叫作花钱(呐),积攒下几两银子,都借与他赔马了。}

\textbf{怎么不对呢?}

\textbf{哦,我那薛大哥也是个贫寒出身?}

\textbf{哎呀呀,薛大哥呀薛大哥,我今日才晓得你也是个贫寒出身呐。}

\textbf{呵呵哈哈哈\ldots{}\ldots{}(笑介)}

\textbf{【西皮原板】本利算来二十两,并不曾还我半毫分。}

\textbf{无有也是枉然。}

\textbf{(岂不伤了朋友的和气?)}

\textbf{防身宝剑,你问它则甚?}

\textbf{诶,清平世界(或:青天白日),朗朗乾坤,杀人(岂不)是要偿命的呀。}

\textbf{唉,有道是:善财难舍呀。}

\textbf{【西皮原板】那一日过营去将账讨,他言说长安城有一个王氏宝钏。}

\textbf{不该。}

\textbf{不欠。}

\textbf{大嫂,(我来问你,)有道是:这父债------}

\textbf{夫债呢?}

\textbf{妻,妻\ldots{}\ldots{}妻怎么样?}

\textbf{呵呵,你倒推得个干净呐。}

\textbf{(呃,)有道是:这汗呐,要出在病人的身上哦。}

\textbf{【西皮原板】他无钱便把妻来卖,将大嫂卖与了当军的人呐。}

\textbf{喏喏喏,就是在下。(或:不才,在下。)}

\textbf{呃呃呃,呃,我有婚书为证呐!}

呃呃,你慢来慢来,我看大嫂变脸变色,婚书诓至手中,三把五把扯碎,为军的岂不落一个人财两空么?

呃,你我去至前村,大户人家,请上三老四少,同拆同观。

当真。

哪个骗你呀?

\textbf{呵呵,她倒骂起来了。}

\textbf{(王宝钏 【西皮二六】\ldots{}\ldots{}主婚的人呐。)}

\textbf{【西皮快板】苏龙魏虎为媒证,那王丞相是我的主婚人呐。}

\textbf{【西皮快板】他三人与我有仇恨,咬定牙关就不认承。}

\textbf{【西皮快板】西凉川四十单八站,为军的要人不要钱。}

\textbf{【西皮快板】大嫂休得巧言辩,为军哪怕到官前?衙里衙外我打点,管保大嫂断与咱。}

\textbf{【西皮快板】好一个贞节王宝钏,百般调戏也枉然。(自古道青酒红人面,动人心,财帛金银钱。)(在)腰中取出银一锭,将银放在地平川。这锭银(子)三两三,送与大嫂作养奁(或:做妆奁)。买绫罗、做衣衫,打首饰、置簪环,做一个少年的夫妻就过几年呐。}

\textbf{【西皮快板】是烈女就该在闺房,缘何来在大路旁。为军的起下不良意,}

\textbf{【西皮摇板】来来来一马双跨到西凉呃。}

\textbf{走走走,上马呀。}

\textbf{在哪里?}

\textbf{诶呀!}

\textbf{呵呵哈哈哈\ldots{}\ldots{}(笑介)}

\textbf{【西皮摇板】一见宝钏回窑转,果然为我受熬煎(或:一十八载受熬煎)。不上马来步下赶,回到窑中两团圆。}

\textbf{{[}第二场{]}}

\textbf{【西皮摇板】后面跟随平贵男。}

\textbf{【西皮摇板】将为丈夫关至在(这)窑外边。}

\textbf{妻呀!}

\textbf{【西皮导板】想起当年泪不干呐,}

\textbf{【西皮原板】夫妻们在寒窑受尽了熬煎。自从我降了红鬃马,唐主爷驾前去讨官。官封我后军都督府哇,你的父上殿把本参。自从盘古【转西皮快板】立地天,哪有岳父把婿参。西凉国,造了反,薛平贵倒做了先行官。两军阵前遇代战,将我擒过了马雕鞍。多蒙老王不肯斩,反把公主配良缘。西凉的老王把驾晏,(众)文武保我坐银安。那一日驾坐银安殿,宾鸿大雁口吐人言。手执金弓银弹打,打下了半幅血罗衫。展开罗衫从头看,才知道寒窑受苦的王宝钏。不分昼夜往前趱,为的是回家夫妻团圆。三姐不信从头算,连来带呃去十八年。}

\textbf{【西皮摇板】水流千遭归大海,原物交还本人观。}

\textbf{【西皮摇板】少年子弟江湖老,红粉佳人两鬓斑。三姐不信菱花看,容颜不似当年在彩楼前。}

\textbf{水盆里面。}

\textbf{话已说明,开门相见才是。(或:话已说明,快快开门,夫妻相见。)}

\textbf{哦,退一步。}

\textbf{哦,又退了一步。(或:哦,再退后一步。)}

\textbf{唉呀妻呀,后面无有路了。}

\textbf{唉!}

\textbf{【西皮摇板】三姐不必寻短见,为丈夫跪在地平川。}

\textbf{(王宝钏 【西皮摇板】\ldots{}\ldots{}什么官。)}

\textbf{进得窑来,不问饥寒,开口便是官。难道说还吃官、穿官不成么?}

\textbf{(呃,)我临行的时节(或:临行之时),也曾(与你)留下安家度用啊。}

\textbf{十担干柴,八斗老米。}

\textbf{就该去借。}

\textbf{相府去借呀。}

\textbf{哦,你不曾去过相府(或:你不曾进过相府)?}

\textbf{呵(或:好),有志气。告便。}

\textbf{去至相府,与你那爹爹算这一十八载的老米账啊。}

\textbf{哦,他病了?得何病症呐(或:他得的什么病症呐)?}

\textbf{呵呵(或:哦),他见不得我?}

\textbf{难道说,我还见不得他?}

\textbf{有朝一日,孤王(或:我)得了唐室天下,他与我牵马坠镫,(呃,)我还嫌他老呢。}

\textbf{不曾睡着。}

\textbf{(句句实言。)}

\textbf{有道是:龙行有宝。}

\textbf{无宝呢?}

\textbf{三姐观宝。}

\textbf{【西皮快板】在头上整整沿毡帽,避尘珠金光照满窑。用手取出番王宝,三姐拿去仔细呀瞧。}

\textbf{下跪何人?}

\textbf{跪在孤王(或:我)的面前则甚呐?}

\textbf{(王宝钏 讨封。)}

\textbf{方才你在武家坡前骂得我好苦。(呃,)我是不能封的了啊。}

\textbf{若是知道,必然是不骂的了啊。(或:哦,倘若知道是我呢?)}

\textbf{呃,越发的不封了。}

\textbf{(当真不封。)}

\textbf{(果然不封。)}

\textbf{呃,慢来慢来,焉有不封之理。}

\textbf{三姐听封------}

\textbf{【西皮快板】非是孤不把你来封,有一个缘故在其中(或:在内中)。西凉国有个代------}

\textbf{【西皮快板】西凉国有一个(或:西凉川有个)代战女,她保孤王立大功。}

\textbf{(王宝钏 【西皮快板】\ldots{}\ldots{},她为正来我为偏。)}

\textbf{【西皮快板】讲什么正来论什么偏,你我结发在她前(或:在她先)。有朝一日登宝殿(或:登龙殿),封你昭阳掌正权。}

\textbf{【西皮摇板】平贵离家十八年。}

\textbf{(王宝钏 【西皮摇板】\ldots{}\ldots{}王宝钏。)}

\textbf{【西皮摇板】今日夫妻重相见。}

\textbf{(王宝钏 【西皮摇板】\ldots{}\ldots{}在梦间。)}

三姐,你看红日当头,不是做梦啊。

(不是做梦。)

三姐。

来了。

呵呵哈哈哈\ldots{}\ldots{}(笑介)

\newpage
\hypertarget{ux5927ux767bux6bbf-ux4e4b-ux859bux5e73ux8d35}{%
\subsection{大登殿 之
薛平贵}\label{ux5927ux767bux6bbf-ux4e4b-ux859bux5e73ux8d35}}

\textbf{(内)【西皮导板】长安城内把兵点,}

\textbf{【西皮原板】孤王得报旧仇冤。马达、江海把旨传,晓谕孤王驾坐长安。龙行虎步上金殿,薛平贵也有今一天}\protect\hyperlink{fn412}{\textsuperscript{412}}\textbf{。}

\textbf{【西皮原板】马达、江海一声唤,朝房中文武臣来把驾参。}

\textbf{【西皮摇板】你二人忠心把孤保,文武当朝一品官。}

\textbf{【西皮摇板】人来王允押上殿,}

\textbf{【西皮摇板】孤王言来听根源:先前设计将孤害,事到头来后悔难。人来推出午门斩,斩他的首级挂高竿。}

\textbf{【西皮摇板】他与魏虎将孤害,今日斩他报仇冤。}

\textbf{定斩不赦!}

\textbf{且慢!}

\textbf{【西皮摇板】梓童不必寻短见,午门外快赦王允还。}

\textbf{【西皮摇板】殿角赐你金交椅,事平之后再封官。}

\textbf{【西皮摇板】快将魏虎押上殿,}

\textbf{【西皮摇板】一见贼子怒冲冠。}

\textbf{【西皮摇板】马达、江海推出斩,}

\textbf{但凭于你。}

\textbf{【西皮摇板】杀魏虎方称孤心愿,}

\textbf{【西皮摇板】代战公主把驾参。}

\textbf{【西皮摇板】孤王金殿用目看,二梓童打扮似天仙。宝钏封在昭阳院,代战公主掌兵权。赐你二人龙凤剑,三人同掌锦长安。}

\textbf{【西皮摇板】宝钏相府忙回转,快请岳母上金銮。}

\textbf{【西皮摇板】孤王金殿出赦诏,晓谕天下众群僚:一赦钱粮并钱钞,二赦囚犯出监牢。}

\textbf{【西皮导板】二梓童搀岳母待王拜见。}

\textbf{【西皮二六】尊一声老岳母细听儿言:不幸我的亲娘亡故早,你比我亲娘甚是贤。薛平贵本是花郎汉,到如今驾坐在长安。宝钏封在昭阳院,代战公主掌兵权。将岳母奉至在养老院,你在那寿星宫内乐安然。彩女、宫娥常陪伴,一日三次王去问安。请请请,老岳母请至养老院。}

\textbf{【西皮摇板】快宣王允上金殿,}

\textbf{【西皮摇板】孤封你当朝太师在朝班,有职无权。}

\textbf{【西皮摇板】朝事已毕把班散,养老宫去问岳母安。}


\item
  \leavevmode\hypertarget{fn307}{}%
  陈超老师说明:谭鑫培特地给秦琼设计了很多江湖气的身段。\protect\hyperlink{fnref307}{↩}
\item
  \leavevmode\hypertarget{fn308}{}%
  段公平君建议此处从俗作``仨人''。\protect\hyperlink{fnref308}{↩}
\item
  \leavevmode\hypertarget{fn309}{}%
  一般作``我发财啦''或``我发了财'',此处从《京剧新序》。\protect\hyperlink{fnref309}{↩}
\item
  \leavevmode\hypertarget{fn310}{}%
  《京剧新序》此处原未注板式,据文意补。\protect\hyperlink{fnref310}{↩}
\item
  \leavevmode\hypertarget{fn311}{}%
  《论语·公冶长》:``子路曰:`愿车马衣轻裘,与朋友共,敝之而无憾。'''又《雍也》:``子曰:`赤(公西华)之适齐也,乘肥马,衣轻裘。''\protect\hyperlink{fnref311}{↩}
\item
  \leavevmode\hypertarget{fn312}{}%
  《京剧新序》中作``王伯党''。据史料载,王伯当是隋末济阳县(今河南兰考)人,起义领袖,瓦岗军名将。\protect\hyperlink{fnref312}{↩}
\item
  \leavevmode\hypertarget{fn313}{}%
  𥋌念``撒(sā)''音,是冀鲁方言看的意思。《京剧新序》原作``用目洒'',《京剧新序(修订版)》改作``用目𥋌'',此处从后者。\protect\hyperlink{fnref313}{↩}
\item
  \leavevmode\hypertarget{fn314}{}%
  王荣山此戏的耍锏有研究,现介绍\textbf{锏架子}如下:

  (台中间)\textbf{唱完}``\ldots{}\ldots{}耍一耍''\textbf{之后},转身面向里躬揖,左手抱锏,右手伸指切掌停在胸前,向里一步两步三步,右脚贴在左脚(/腿)后边,左腿单腿右转身面向大边台口外角(亮住),退三步到上场门里边,右手拉开山膀亮住。云手跨右腿踢左腿分锏,双手举锏亮。缓锏,正花转身、再正花转身到大边台口矗双锏,虚右脚亮。云手从台口过到小边矗双锏(台口左锏在右肘(/臂)后),虚左脚亮。缓锏,正花转身大走到下场门边,面小边台口举双锏亮。正花转身,横走正花、回花转身,回花搓步三倒手横推锏,在小边面向外,右上左下横锏,虚左脚亮。缓锏,正花转身(下场门边举双锏,站),横走正花、回花转身,回花搓步三倒手横(单腿亮)推锏,在大边面向里横锏,虚左脚亮。缓锏,正花、回花转身,横走到小边,回花双锏点地,双锏向后反画圈举锏,右脚贴左脚(/腿)后边单腿立亮。正花、转身缓锏,正花、回花转身,横走到大边,回花双锏点地,双锏向后反画圈举锏,右脚贴左脚(/腿)后边单腿立亮。正花撤步,回花转身,打右靴底、小蹦子打左靴底,回花在脸前起反云手转身到上场门里边拉开,在脸前起正云手转身向下场门外角。边走边向下方三扎。面斜向下场门外角,起三个大刀花,双锏点地,起三个揉(/回)花。正花蹦子转身、右锏打地。在台口中间反云手转身,踢右腿正面分锏亮住。

  \textbf{《当锏卖马》的锏架子是秦琼威吓王伯当和谢映登的,要大耍劈抡双锏,王荣山《卖马》锏架子受到内外行赞许,特别是收时在台口侧身回花,像车轮绕身一样,非常好看。}

  \textbf{陈超老师按:}刘曾复先生记录这套的锏套子为王荣山所传,是谭鑫培的演法。\protect\hyperlink{fnref314}{↩}
\item
  \leavevmode\hypertarget{fn315}{}%
  《京剧新序》误作``先生''。\protect\hyperlink{fnref315}{↩}
\item
  \leavevmode\hypertarget{fn316}{}%
  \textbf{与韩擒虎开打是二人都使枪}。

  \begin{quote}
  头场开城会阵,伍云召在城内上马介冲出城,在大边跟龙套走、到大边里边见面,一扯两扯,一合两合,幺二三,兜,勾韩走马腰封到大边,往里一盖,上下、打韩下。龙套追过场。伍耍大下场下。
  \end{quote}

  头场伍云召打韩擒虎下,缓枪上步在台中间向外站、右手拿枪斜垂,左手掏过去挥手,叫龙套追过场。捋胡子左转身面向上场门,跨左步,用枪扫右腿趋步,转身到上场门向外举枪亮,走,小绕到小边台口,两手斜托枪,走,到下场门方向,枪扎出去,伸左手、右手收回来斜着平举枪跺泥一亮,手、枪不动回走,绕到台中间向外涮枪转向里矗枪站住,耍三个背花,回身向外耍三个迎面花,面略斜向左,用枪打左脚,回身略斜向右,枪经左手背绕过左手反手接枪,由下往右往上往左绕一圈,与右手一块儿拿枪打右脚,枪从上往左小蹦子打左脚,跨右腿,右手向上场门出枪,向大边台口踢左腿,上步右手枪在脸前从左下向上往右再往左画大圈,左转身中枪交左手、在大边台口斜身面向外,左手拿枪,右手跟着一块儿在脸前从上往右再往左画大圈,弓箭步,左手平出枪,右手胸前按胡子亮住,大绕从下场门下。\protect\hyperlink{fnref316}{↩}
\item
  \leavevmode\hypertarget{fn317}{}%
  二场伍追韩上,原地漫头,用枪头别,一拉右转身一过合两过合,回身往里一裹,上下、打韩下。伍耍小下场(收兵)下。

  \begin{quote}
  二场打韩下,伍面向外提枪花转身,面向外三个提枪花,出枪,左手向左上方伸出平托枪,右手往右掠枪杆往下绕过枪鐏反手扶鐏,跨左腿,踢右腿,两手不离枪,向右往里翻身,面向下场门,左手在前扶住枪杆,右手顺着枪鐏转过来、握住枪下端在右侧腰间,一绕枪头,平端枪、弓箭步亮住下。
  \end{quote}

  \protect\hyperlink{fnref317}{↩}
\item
  \leavevmode\hypertarget{fn318}{}%
  唱完\textless{}\textbf{扫头}\textgreater{}双剜萝卜,架住,钻烟筒,一扯两扯,一合两合,接上下左右,往里一盖两盖,接蓬头,绕枪杆枪头斜向下、弓箭步败下。\protect\hyperlink{fnref318}{↩}
\item
  \leavevmode\hypertarget{fn319}{}%
  陈超老师介绍:伍云召见朱灿,\textless{}\textbf{水底鱼}\textgreater{}身段,富连成不下马,余叔岩、王凤卿下马,蹉步。谭富英不下马。\protect\hyperlink{fnref319}{↩}
\item
  \leavevmode\hypertarget{fn320}{}%
  《京剧汇编》第九集
  李万春藏本作``一朝错''。此处从李元皓君建议。\protect\hyperlink{fnref320}{↩}
\item
  \leavevmode\hypertarget{fn321}{}%
  《京剧汇编》第九集
  李万春藏本作``英雄四海''。\protect\hyperlink{fnref321}{↩}
\item
  \leavevmode\hypertarget{fn322}{}%
  秦琼见杨林后括号内的对白是陈超老师跟刘曾复先生学的。\protect\hyperlink{fnref322}{↩}
\item
  \leavevmode\hypertarget{fn323}{}%
  《打登州》是秦琼为骗杨林给他马骑好逃走,所以《打登州》只耍一点步锏,让杨林看不上,罗周才好替秦琼说马上武艺好,杨林才给秦琼马骑。该戏与《当锏卖马》戏情不同,锏的耍法各异。

  陈超老师介绍的秦琼耍步锏记录如下:

  一请,至小边里角拉开,分锏,缓锏,举锏一亮;一扫腿(外),俩扫腿(里);一扎锏、俩扎锏、三扎锏至大边台口,缓锏,横场大刀花;云手至小边台口,立锏虚步亮相。再到大边、台中如法炮制两遍。\protect\hyperlink{fnref323}{↩}
\item
  \leavevmode\hypertarget{fn324}{}%
  在传统话本小说中王伯当名勇,字伯当,以字行。\protect\hyperlink{fnref324}{↩}
\item
  \leavevmode\hypertarget{fn325}{}%
  段公平君注:金堤关,堤字正音为``敌(dí)'',艺人讹念为``提(tí)''。隋代关名,在今河南荥阳市广武镇霸王城村北黄河道中。以关置于汉代``金堤''而得名。\protect\hyperlink{fnref325}{↩}
\item
  \leavevmode\hypertarget{fn326}{}%
  进身,指被录用或提升。\protect\hyperlink{fnref326}{↩}
\item
  \leavevmode\hypertarget{fn327}{}%
  《京剧汇编》第九集
  苏连汉藏本作``猖狂''。\protect\hyperlink{fnref327}{↩}
\item
  \leavevmode\hypertarget{fn328}{}%
  段公平君注:\textbf{史载,秦琼死后追改封``胡国公'',疑后讹为``护国公''。}\protect\hyperlink{fnref328}{↩}
\item
  \leavevmode\hypertarget{fn329}{}%
  《京剧汇编》第九集
  苏连汉藏本作``跨下''。\protect\hyperlink{fnref329}{↩}
\item
  \leavevmode\hypertarget{fn330}{}%
  段公平君建议此处李世民唱,此处从《京剧汇编》第九集
  苏连汉藏本。\protect\hyperlink{fnref330}{↩}
\item
  \leavevmode\hypertarget{fn331}{}%
  《京剧汇编》第九集
  苏连汉藏本作``矗矗''。\protect\hyperlink{fnref331}{↩}
\item
  \leavevmode\hypertarget{fn332}{}%
  《京剧汇编》第九集
  苏连汉藏本作``我把弟一枪给你挑歪了''。\protect\hyperlink{fnref332}{↩}
\item
  \leavevmode\hypertarget{fn333}{}%
  \textbf{汛地是明、清时代称军队驻防地段。``汛''通``讯'',``讯地''即为军事烽火之地,以传消息,地界不大,故而汛地为基本驻防之地。}\protect\hyperlink{fnref333}{↩}
\item
  \leavevmode\hypertarget{fn334}{}%
  此处``残母''可能是``戏母''之误。\protect\hyperlink{fnref334}{↩}
\item
  \leavevmode\hypertarget{fn335}{}%
  段公平君指出,``路卧''可能是``路剐''之误。\protect\hyperlink{fnref335}{↩}
\item
  \leavevmode\hypertarget{fn336}{}%
  ``干国良臣''是旧时戏曲中常见词汇,``干国''是治理国家之意。干的本意是盾牌,引申为捍卫、保卫之意。``干国良臣''即``保国良臣''、``治国良臣''。此处``捍国''与``干国''同意。\protect\hyperlink{fnref336}{↩}
\item
  \leavevmode\hypertarget{fn337}{}%
  本剧本中标注人物台上位置主要由段公平君标注。刘曾复先生在为樊百乐君说戏时曾说明:《取帅印》是二十四本《龙门阵》的头一本。《龙门阵》是大老板程长庚排的。《白良关》也是《龙门阵》里的,但《凤凰山》、《独木关》、《汾河湾》等都不是《龙门阵》里的。\protect\hyperlink{fnref337}{↩}
\item
  \leavevmode\hypertarget{fn338}{}%
  夏行涛君建议此四句应为一人一句接唱,此处从《京剧汇编》第九集
  赵荣鹏藏本。\protect\hyperlink{fnref338}{↩}
\item
  \leavevmode\hypertarget{fn339}{}%
  刘曾复先生特别说明:程咬金对唐王念韵白,对其他臣宰念京白。\protect\hyperlink{fnref339}{↩}
\item
  \leavevmode\hypertarget{fn340}{}%
  ``埋怨'',《京剧汇编》第九集
  赵荣鹏藏本皆作``瞒怨''。\protect\hyperlink{fnref340}{↩}
\item
  \leavevmode\hypertarget{fn341}{}%
  ``荏苒''是逡巡、一刹那的意思。\protect\hyperlink{fnref341}{↩}
\item
  \leavevmode\hypertarget{fn342}{}%
  段公平君建议作``不识良言'';《京剧汇编》第九集
  赵荣鹏藏本作``不良之言教子孙''。\protect\hyperlink{fnref342}{↩}
\item
  \leavevmode\hypertarget{fn343}{}%
  《京剧汇编》第九集
  赵荣鹏藏本作``令出山摇动,严法鬼神惊''。\protect\hyperlink{fnref343}{↩}
\item
  \leavevmode\hypertarget{fn344}{}%
  《京剧汇编》第九集
  赵荣鹏藏本作``定后跟''。\protect\hyperlink{fnref344}{↩}
\item
  \leavevmode\hypertarget{fn345}{}%
  《京剧汇编》第九集 赵荣鹏藏本作``替主''\protect\hyperlink{fnref345}{↩}
\item
  \leavevmode\hypertarget{fn346}{}%
  段公平君建议作``一阵火气''。\protect\hyperlink{fnref346}{↩}
\item
  \leavevmode\hypertarget{fn347}{}%
  录音中``把名提''疑作``压名敌'',此处从《京剧汇编》第九集
  赵荣鹏藏本。\protect\hyperlink{fnref347}{↩}
\item
  \leavevmode\hypertarget{fn348}{}%
  段公平君建议作``告个罪''。\protect\hyperlink{fnref348}{↩}
\item
  \leavevmode\hypertarget{fn349}{}%
  夏行涛君建议作``即送'',此处从《京剧汇编》第九集
  赵荣鹏藏本。\protect\hyperlink{fnref349}{↩}
\item
  \leavevmode\hypertarget{fn350}{}%
  刘曾复先生介绍说,《马上缘》一剧中开始薛仁贵唱与此段基本相同,只是后两句为``我的儿出兵无音信,且听探马报信音。''\protect\hyperlink{fnref350}{↩}
\item
  \leavevmode\hypertarget{fn351}{}%
  陈超老师注:打虎的身段,老谭有两种演法。\protect\hyperlink{fnref351}{↩}
\item
  \leavevmode\hypertarget{fn352}{}%
  此处``站窑门''原来是``坐窑门''。\protect\hyperlink{fnref352}{↩}
\item
  \leavevmode\hypertarget{fn353}{}%
  吴小如先生认为``门庭''应该是``门桯'',``桯''是门槛的意思。\protect\hyperlink{fnref353}{↩}
\item
  \leavevmode\hypertarget{fn354}{}%
  据《秋声集》\textsuperscript{{[}15{]}.}载,程砚秋曾指出,陕西方言称有的窑洞为``坡洼寒窑'',即黄土坡下挖的窑洞,``坡洼''与``破瓦''谐音,艺人以讹传讹,久之就念成了``破瓦寒窑''。此处从俗。\protect\hyperlink{fnref354}{↩}
\item
  \leavevmode\hypertarget{fn355}{}%
  陈超老师注:老谭晚年《汾河湾》炉火纯青,舞台状态极其放松。王凤卿叙述过很多老谭晚年演此戏的即兴身段,如``马头军''身段很特别,左手捋髯,右手勾起食指指出去,等旦角重复完``马头军'',晃晃勾起的手指,犹如马头吃草,配合微点几下头,再念``马头军''。舞台效果极佳。\protect\hyperlink{fnref355}{↩}
\item
  \leavevmode\hypertarget{fn356}{}%
  ``砷黄铜''和``砷白铜''又分别被称为``药金''和``药银'',是古代方士为实现以铜``炼制''金、银时,用砷化物(包括雄黄、雌黄、砒霜等)``点化''铜而生成的产物(砷铜合金)。砷黄铜和砷白铜的区别主要是砷含量不同,铜中含砷小于~10\%,呈金黄色;当砷含量超过~10\%,则呈银白色。砷元素易挥发,所谓``真金不怕火炼'',就是因为高温下砷黄铜中的砷会遇热流失,恢复铜的本质。\protect\hyperlink{fnref356}{↩}
\item
  \leavevmode\hypertarget{fn357}{}%
  段公平君建议作``他不回来了''。\protect\hyperlink{fnref357}{↩}
\item
  \leavevmode\hypertarget{fn358}{}%
  此处刘曾复先生念作``法弟''。\protect\hyperlink{fnref358}{↩}
\item
  \leavevmode\hypertarget{fn359}{}%
  ``提龙笔''一段原来全部唱【二黄原板】,唱法刘曾复先生同样做了示范,词句如下:

  ``提龙笔王亲书大唐国号,命御弟唐三藏奉旨出朝。各国的众王子休挡禁道,到西天取经回替朕代劳。赐御弟锦袈裟霞光万道,孤赐你紫金钵、禅杖一条。孤赐你装经箱、毗卢僧帽,孤赐你四徒儿来把经挑。侍内臣与孤王将宝抬到,金銮殿王与你改换佛袍。''\protect\hyperlink{fnref359}{↩}
\item
  \leavevmode\hypertarget{fn360}{}%
  夏行涛君建议作``即日归''。\protect\hyperlink{fnref360}{↩}
\item
  \leavevmode\hypertarget{fn361}{}%
  本剧本中有关人物台上地方由段公平君协助整理。\protect\hyperlink{fnref361}{↩}
\item
  \leavevmode\hypertarget{fn362}{}%
  古人以梦中见熊罴为生男的征兆。后以``梦熊''作生男的颂语。语本《诗·小雅·斯干》:``吉梦维何?维熊维罴。''又:``大人占之,维熊维罴,男子之祥。''
  郑玄笺注:``熊罴在山,阳之祥也,故为生男。''\protect\hyperlink{fnref362}{↩}
\item
  \leavevmode\hypertarget{fn363}{}%
  段公平君注:若不带``\textbf{观阵}''一场,此处可按如下处理:

  \begin{quote}
  (樊梨花 仙师有何吩咐?)

  薛丁山
  仙师赐你丹药二料,一要戴之胸膛,二要用清水服下。\ldots{}\ldots{},料无妨碍。

  薛丁山 夫人请至后帐。

  (薛丁山、樊梨花下)
  \end{quote}

  \protect\hyperlink{fnref363}{↩}
\item
  \leavevmode\hypertarget{fn364}{}%
  段公平君建议此处作``你是听'',即``你在此听之意''。\protect\hyperlink{fnref364}{↩}
\item
  \leavevmode\hypertarget{fn365}{}%
  刘曾复先生在说该戏总讲时,除徐策外的人都说得比较简略,整理时个别地方参考了程君谋、蒋锡康录音唱片进行了增补;刘曾复先生在《京剧书文指伪录》\textsuperscript{{[}16{]}.}一文中介绍,徐策和夫人穿蓝帔,徐策戴员外巾。该戏有关场次调度也参照此文。\protect\hyperlink{fnref365}{↩}
\item
  \leavevmode\hypertarget{fn366}{}%
  刘曾复先生在为樊百乐君说戏时说明,\textless{}\textbf{六幺令}\textgreater{}行路时用;\textless{}\textbf{大锣六幺令}后段\textgreater{}留着张泰上场时用。\protect\hyperlink{fnref366}{↩}
\item
  \leavevmode\hypertarget{fn367}{}%
  七煞是紫微斗数中十四颗主星之一,七煞主``肃杀''。据说``面带七煞''的人往往寿命不长。\protect\hyperlink{fnref367}{↩}
\item
  \leavevmode\hypertarget{fn368}{}%
  此句程君谋、蒋锡康录音作``欸------相爷说哪里话来,想你我二老,年过半百,只生金斗孩儿,要他替旁人去死,万万不能!''\protect\hyperlink{fnref368}{↩}
\item
  \leavevmode\hypertarget{fn369}{}%
  据程君谋、蒋锡康唱片录音增补。\protect\hyperlink{fnref369}{↩}
\item
  \leavevmode\hypertarget{fn370}{}%
  据程君谋、蒋锡康唱片录音增补。\protect\hyperlink{fnref370}{↩}
\item
  \leavevmode\hypertarget{fn371}{}%
  此处程君谋、蒋锡康唱片录音作:

  薛猛 【二黄散板】夫人不必泪双淋,忠良哪怕丧残生。回头便对奸贼论:

  薛猛 贼!

  薛猛 【二黄散板】阴曹地府勾尔的魂。\protect\hyperlink{fnref371}{↩}
\item
  \leavevmode\hypertarget{fn372}{}%
  据程君谋、蒋锡康唱片录音增补。\protect\hyperlink{fnref372}{↩}
\item
  \leavevmode\hypertarget{fn373}{}%
  据程君谋、蒋锡康唱片录音增补。\protect\hyperlink{fnref373}{↩}
\item
  \leavevmode\hypertarget{fn374}{}%
  据程君谋、蒋锡康唱片录音增补。\protect\hyperlink{fnref374}{↩}
\item
  \leavevmode\hypertarget{fn375}{}%
  据程君谋、蒋锡康唱片录音增加。\protect\hyperlink{fnref375}{↩}
\item
  \leavevmode\hypertarget{fn376}{}%
  据程君谋、蒋锡康唱片录音增加。\protect\hyperlink{fnref376}{↩}
\item
  \leavevmode\hypertarget{fn377}{}%
  ``将养''即``抚养''之意。\protect\hyperlink{fnref377}{↩}
\item
  \leavevmode\hypertarget{fn378}{}%
  以下至结尾全部据程君谋、蒋锡康录音增补。\protect\hyperlink{fnref378}{↩}
\item
  \leavevmode\hypertarget{fn379}{}%
  刘曾复先生在樊百乐君说戏时说明,该戏徐策穿白开氅,戴相巾。\protect\hyperlink{fnref379}{↩}
\item
  \leavevmode\hypertarget{fn380}{}%
  段公平君建议作``腰铡三截''均作``腰铡三节''。\protect\hyperlink{fnref380}{↩}
\item
  \leavevmode\hypertarget{fn381}{}%
  根据剧中人物推测,此剧中的唐王应为唐代宗李豫。\protect\hyperlink{fnref381}{↩}
\item
  \leavevmode\hypertarget{fn382}{}%
  刘曾复先生说戏录音中作``父公侯、子王位'',此处从《京剧汇编》第三十二集
  邢威明藏本。\protect\hyperlink{fnref382}{↩}
\item
  \leavevmode\hypertarget{fn383}{}%
  《京剧汇编》第三十二集
  邢威明藏本作``共贺三多''。\protect\hyperlink{fnref383}{↩}
\item
  \leavevmode\hypertarget{fn384}{}%
  有些地方戏作``唐君蕊'',此处从《京剧汇编》第三十二集
  邢威明藏本。\protect\hyperlink{fnref384}{↩}
\item
  \leavevmode\hypertarget{fn385}{}%
  陈超老师介绍,唐王唱完此句后整冠捋髯。\protect\hyperlink{fnref385}{↩}
\item
  \leavevmode\hypertarget{fn386}{}%
  夏行涛君建议作``因何气''。\protect\hyperlink{fnref386}{↩}
\item
  \leavevmode\hypertarget{fn387}{}%
  古代习俗,生了男孩子,就在门的左首悬挂一张弓。因此用``悬弧之喜''指男性的生日。\protect\hyperlink{fnref387}{↩}
\item
  \leavevmode\hypertarget{fn388}{}%
  ``轻年纪'',即年纪尚轻之意。\protect\hyperlink{fnref388}{↩}
\item
  \leavevmode\hypertarget{fn389}{}%
  ``铜驼''即铜铸的骆驼,古代置于宫门外。借指京城、宫廷。同时也比喻朝代兴亡。\protect\hyperlink{fnref389}{↩}
\item
  \leavevmode\hypertarget{fn390}{}%
  据史料载,李克用之父李国昌,本名朱邪赤心,唐末沙陀部落首领,唐懿宗时因镇压庞勋起义之功,被赐名``李国昌''。``朱邪''姓亦作``朱耶'',艺人不识,误作朱姓;``国''字系入声字,此处保留湖北方言念法。\protect\hyperlink{fnref390}{↩}
\item
  \leavevmode\hypertarget{fn391}{}%
  据史料载,李克用别名``李鸦儿'',一目失明,其主力部队因穿黑衣服而以``鸦儿军''闻名。\protect\hyperlink{fnref391}{↩}
\item
  \leavevmode\hypertarget{fn392}{}%
  夏行涛君建议作``皓髯''。\protect\hyperlink{fnref392}{↩}
\item
  \leavevmode\hypertarget{fn393}{}%
  据《新五代史·唐本纪第四》载``黄巢已陷京师,中和元年,代北起军使陈景思发沙陀先所降者,与吐浑、安庆等万人赴京师,行至绛州,沙陀军乱,大掠而还。景思念沙陀非克用不可将,乃以诏书召克用于鞑靼,承制以为代州刺史、雁门以北行营节度使。率蕃汉万人出石岭关\ldots{}\ldots{}二年十一月,景思、克用复以步骑万七千赴京师。''戏中程敬思的原型即为陈景思。\protect\hyperlink{fnref393}{↩}
\item
  \leavevmode\hypertarget{fn394}{}%
  段公平君注:黄巢的字,于史无载。《残唐五代史演义》作``巨天'',此处从之。

  吴小如先生早年曾撰文\textsuperscript{{[}17{]}.}指出,此两句原作``家住曹州定陶县,姓黄名巢字霸天''。``并曹县''是``定陶县''的讹传;旧时艺人文化程度低,将``霸天''记作``垻天''字,以致讹成``具天''。

  \begin{quote}
  姜骏按:据《新唐书》载,黄巢是``山东曹州冤句人'',据史料推测,冤句在曹县、定陶一带(具体方位有争议)。因此``曹州并曹县''中``并''理解为衬字亦可。
  \end{quote}

  \protect\hyperlink{fnref394}{↩}
\item
  \leavevmode\hypertarget{fn395}{}%
  《残唐五代史演义》作``藏梅寺''。\protect\hyperlink{fnref395}{↩}
\item
  \leavevmode\hypertarget{fn396}{}%
  《京剧汇编》第四十六集
  郝寿臣藏本作``西岐'',此处从《残唐五代史演义》,下同。\protect\hyperlink{fnref396}{↩}
\item
  \leavevmode\hypertarget{fn397}{}%
  ``笑连天''亦可作``笑颜添''。\protect\hyperlink{fnref397}{↩}
\item
  \leavevmode\hypertarget{fn398}{}%
  此处``山摇震''或``山摇动''从俗,方与``胆战惊''对,下同。\protect\hyperlink{fnref398}{↩}
\item
  \leavevmode\hypertarget{fn399}{}%
  这是谭鑫培、钱金福在《珠帘寨》李克用与周德威对刀的``刀架子'',余叔岩演《珠帘寨》也打这一套刀架子:\textbf{这套刀架子},\textbf{特别是头子},\textbf{非常有内容},\textbf{一上来周德威显得很冲},\textbf{使人有李克用要招架不住之感},\textbf{但是几下之后李就轻巧地控制了周}。\textbf{这套把子层次分明},\textbf{很有戏}。\protect\hyperlink{fnref399}{↩}
\item
  \leavevmode\hypertarget{fn400}{}%
  《残唐五代史演义》作``鸦馆楼'',此处从《京剧新序》。\protect\hyperlink{fnref400}{↩}
\item
  \leavevmode\hypertarget{fn401}{}%
  刘曾复先生提供的钞本亦作``指望''。\protect\hyperlink{fnref401}{↩}
\item
  \leavevmode\hypertarget{fn402}{}%
  刘曾复先生提供的钞本作``改匹''。\protect\hyperlink{fnref402}{↩}
\item
  \leavevmode\hypertarget{fn403}{}%
  ``致气''一般作``置气'',下同。\protect\hyperlink{fnref403}{↩}
\item
  \leavevmode\hypertarget{fn404}{}%
  刘曾复先生提供的钞本作``全然不讲''。\protect\hyperlink{fnref404}{↩}
\item
  \leavevmode\hypertarget{fn405}{}%
  刘曾复先生提供的钞本作``千金体匹花郎''。\protect\hyperlink{fnref405}{↩}
\item
  \leavevmode\hypertarget{fn406}{}%
  刘曾复先生提供的钞本作``任儿取去''。\protect\hyperlink{fnref406}{↩}
\item
  \leavevmode\hypertarget{fn407}{}%
  吴焕老师整理的剧本(经刘曾复先生审订)注``汪派此处唱`一马离了三关界(或:三关境)'''。\protect\hyperlink{fnref407}{↩}
\item
  \leavevmode\hypertarget{fn408}{}%
  通常作``弓靫袋''。段公平君注:\textbf{韔(音chàng):弓袋,如《秦风·小戎》``虎韔镂膺'',``交韔二弓''。亦谓将弓放入弓袋,如《小雅·采绿》``之子于狩,言韔其弓''。}\protect\hyperlink{fnref408}{↩}
\item
  \leavevmode\hypertarget{fn409}{}%
  吴焕老师整理的剧本注:``谭派没有`皓月当空'的词句。''\protect\hyperlink{fnref409}{↩}
\item
  \leavevmode\hypertarget{fn410}{}%
  吴焕老师整理的剧本记作``受了苦刑''。\protect\hyperlink{fnref410}{↩}
\item
  \leavevmode\hypertarget{fn411}{}%
  吴焕老师整理的剧本注:``此句汪派唱【西皮快板】''。\protect\hyperlink{fnref411}{↩}
\item
  \leavevmode\hypertarget{fn412}{}%
  夏行涛君建议作``今日天''。\protect\hyperlink{fnref412}{↩}

%\include{Data/chap-05}
%\include{Data/chap-06}
%\include{Data/chap-07}
%\addcontentsline{toc}{section}{\hfill[\hei 唱腔·唱段]\hfill}
\chead{唱腔·唱段} % 页眉中间位置内容

\newpage
\phantomsection %实现目录的正确跳转
\section*{{\hei\large 硃痕记~{\small 之}~朱春登}$^{\ast}$}%*
\addcontentsline{toc}{section}{\hei 硃痕记~{\small 之}~朱春登}

\hangafter=1                   %2. 设置从第1⾏之后开始悬挂缩进  %
\setlength{\parindent}{0pt}{
{\centerline{\textrm{{[}\hei 第一场{]}}}}
\vspace{5pt}

\setlength{\hangindent}{56pt}{【{\akai 二黄散板}】听说是老娘黄泉命染,好一似刀割肉箭把心穿。问婶娘她婆媳何处埋掩。叫李仁备祭品\footnote{吴焕老师整理的剧本作``备祭礼''。}%\protect\hyperlink{fn631}{\textsuperscript{631}}
坟前祭奠,我只得身穿孝头戴麻冠。
}

\vspace{3pt}{\centerline{\textrm{{[}{\hei 第二场}{]}}}}\vspace{5pt}

【{\akai 二黄导板}】见坟台不由人泪流满面,

【{\akai 回龙}】尊一声去世娘细听儿言:~

\setlength{\hangindent}{66pt}{   %3. 设置悬挂缩进量                %
【{\akai 反二黄慢板}】都只为西域国黄龙造反,是孩儿替叔父去到军前。抖威风杀贼寇全凭神箭,灭黄龙平西域得胜回还。王封儿平西侯官高爵显,奉圣命回家来祭奠祖先。实指望母子们欢聚团圆,料不想儿的娘命染黄泉。哭老娘把儿的肝肠痛断,肝肠痛断,儿的娘啊,
}

\setlength{\hangindent}{66pt}{   %3. 设置悬挂缩进量                %
【{\akai 反二黄原板}】食什么爵禄做的是什么官。哭罢了老娘亲把妻房呼唤,叫一声贤德妻你在哪边。我和你夫妻情恩爱不浅,撇下我独一人凄凉孤单。哭一声贤德妻难得相见,难得相见,
}

【{\akai 反二黄散板}】要相逢除非是梦里团圆。

\setlength{\hangindent}{56pt}{【{\akai 西皮三眼}】听我妻赵锦棠言讲一遍,好一似刀割肉箭把心攒。婶娘道她婆媳黄泉命染,为什么她还在阳世之间。莫不是她死得苦冤魂不散,莫不是魍魉鬼来把我缠。我这里出席棚用目观看,观只见那红日未落西山。猛想起赵锦棠左手心硃砂红点,是不是真和假向前去细问根源。
}

\textless{}\!{\bfseries\akai 哭头}\!\textgreater{}啊,我的妻呀!

【{\akai 西皮散板}】问贤妻老娘亲何方避难。

【{\akai 西皮散板}】有劳你前引路把母来见,

【{\akai 西皮散板}】儿就是朱春登做官回还。}

\newpage
\phantomsection %实现目录的正确跳转
\section*{\hei\large 蟠桃会~\protect\footnote{此戏别名《海屋添筹》或《八仙庆寿》,是一出武旦戏。}~%\protect\hyperlink{fn632}{\textsuperscript{632}}
{\small 之}~吕洞宾$^{\ast}$}
\addcontentsline{toc}{section}{\hei 蟠桃会~{\small 之}~吕洞宾}

\hangafter=1                   %2. 设置从第1⾏之后开始悬挂缩进  %
\setlength{\parindent}{0pt}{
{\centerline{{{[}\hei 第一场{]}}}}
\vspace{5pt}

\setlength{\hangindent}{56pt}{【{\akai 西皮原板}】忆昔当年赴科场,科场中提笔做文章。文章幸喜龙颜赏,赏赐我进士伴君王。陪王伴君心不想,一心只想上天堂。天堂就在瑶池上\footnote{此句吴小如先生从刘曾复先生学的是``天堂远在瑶池上''。}%\protect\hyperlink{fn633}{\textsuperscript{633}}
,瑶池以上福寿绵长。}

\vspace{3pt}{\centerline{{{[}{\hei 第二场}{]}}}}\vspace{5pt}

{【{\akai 西皮散板}】离了洞府到仙界,见了众仙说开怀。}

{【{\akai 西皮散板}】瑶池以上寿筵席开。}

\setlength{\hangindent}{56pt}{{【{\akai 西皮原板}】今日里饮酒多爽快,好似仙子({\akai 或}:~好一似黄粱)赴瑶台。这仙女({\akai 或}:~这仙子)生得呀多娇态,眉清目秀送情来。趁此佳兴({\akai 或}:~趁此酒兴)破了戒,}
}

{【{\akai 西皮原板}】众仙道我理不该。将身且坐({\akai 或}:~将身来在)瑶池外,昏昏沉沉睡石台。}

\vspace{3pt}{\centerline{{{[}{\hei 第三场}{]}}}}\vspace{5pt}

{【{\akai 西皮导板}】沉醉东风}\footnote{樊剑{\scriptsize 君}此处建议作``沉醉洞府''。}%\protect\hyperlink{fn634}{\textsuperscript{634}}
{月儿高,}

\setlength{\hangindent}{56pt}{{【{\akai 西皮原板}】忆昔当初饮酕醄。两足徘徊任颠倒,湘子、仙姑发笑嘲。你道我当真吃醉了,任意随心乐逍遥。游戏三昧多奥妙,}
}

\setlength{\hangindent}{56pt}{【{\akai 西皮快板}】坎离\footnote{八卦中
``坎''(\includegraphics[height=10pt,width=10pt, viewport=120 59 225 140,clip]{Eight_Gua.jpeg})为水,
``离''(\includegraphics[height=10pt,width=10pt, viewport=5 59 100 140,clip]{Eight_Gua.jpeg})为火。此处寓为``水火既济''之意。}%\protect\hyperlink{fn635}{\textsuperscript{635}}
二字本相调。不觉来到东海道,海水接天浪滔滔。}

{【{\akai 西皮散板}】宝剑扔在东海道,你看我醉仙家的道法高不高?}

{【{\akai 西皮散板}】柳仙带路东海道,万丈波涛走一遭。}

{【{\akai 西皮散板}】湘子说话不中听,丢了宝贝你问旁人。}

{【{\akai 西皮散板}】洞宾主意拿得稳,从今后不管闲事情。}

(李铁拐\hspace{20pt}【{\akai 西皮散板}】$\cdots${}$\cdots${}{入东海道,)}

{【{\akai 西皮散板}】从今后戒酒最为高。}
}

\newpage
\phantomsection %实现目录的正确跳转
\section*{\hei\large 霸王别姬·山头~{\small 之}~ 韩信$^{\ast}$}
\addcontentsline{toc}{section}{\hei 霸王别姬~{\small 之}~韩信}

\hangafter=1                   %2. 设置从第1⾏之后开始悬挂缩进  %
\setlength{\parindent}{0pt}{

\setlength{\hangindent}{52pt}{   %3. 设置悬挂缩进量                %
\textrm{【{\akai 西皮散板}】李左车引霸王入了阵道,众诸侯齐奋勇争立功劳。直杀得血成河尸如山倒,灭西楚擒霸王就在今朝。}
}

\setlength{\hangindent}{52pt}{   %3. 设置悬挂缩进量                %
\textrm{【{\akai 西皮散板}】传一令犹如那泰山压倒,兵将涌如那海水临潮。楚项羽犹如那无翼之鸟,失彭城犹如那猛虎离巢。}
}

\setlength{\hangindent}{52pt}{   %3. 设置悬挂缩进量                %
\textrm{【{\akai 西皮散板}】直杀得楚项羽人喊马叫,直杀得子弟兵四路奔逃。直杀得天昏暗日无光耀,直杀得夜更深月挂松梢。}
}

\textrm{(项羽\hspace{30pt}【{\akai 西皮散板}】越杀越勇心焦躁,)}

\textrm{【{\akai 西皮散板}】三军带马回营道,请出张良作计较。}
}

\newpage
\subsubsection{\hei\large 逍遥津~{\small 之}~汉献帝$^{\ast}$}
\addcontentsline{toc}{subsection}{\hei 逍遥津~{\small 之}~汉献帝}

\hangafter=1                   %2. 设置从第1⾏之后开始悬挂缩进  %
\setlength{\parindent}{0pt}{

【{\akai 二黄导板}】苦汉帝在后宫伤心难忍,

【{\akai 回龙}】父子们悲切切好不伤情,贤御妻呀。

\setlength{\hangindent}{65pt}{   %3. 设置悬挂缩进量                %
【{\akai 二黄原板}】叹伏后此时间必定丧命,我君妃生离散惨不忍闻。二皇儿年幼小孩童天性,哭啼啼与孤王要他的娘亲。想奸贼不由孤咬牙愤恨,上欺寡人下压群臣。欺寡人贼带剑上殿孤见他不敢责问,欺寡人贼独霸朝纲、目无君王、自专自尊。欺寡人孤只得百般谨慎,欺寡人孤只得时刻留心。欺寡人贼奏本是非曲直孤不敢争论,欺寡人孤有命贼大胆妄为抗旨不遵。欺寡人贼一意孤行孤不敢过问,欺寡人孤怒不敢言、忍耐在心。欺寡人孤见他气色不正吓得孤乱了方寸,欺寡人孤见他带怒发威吓得孤胆战心惊。欺寡人蹂躏百般、万分难忍,欺寡人贼败坏朝纲、逆了五伦。欺寡人好一似【{\footnotesize 转}{\akai 二黄慢板}】奴仆受训,欺寡人好一似虐待家人。欺寡人好一似无辜良民被贼围困,欺寡人好一似冤屈罪犯无处冤申。欺寡人好一似蛇毒蝎狠,欺寡人好一似虎咽狼吞。欺寡人好一似前世冤孽今生报应,欺寡人好一似狭路相逢对头仇人。欺寡人好一似阎君索命,欺寡人好一似饿鬼孤魂。欺寡人好一似败阵残兵无投奔,反被贼困垓心难逃遁难存身,坐以待毙谁来救应,
}

【{\akai 二黄散板}】又听得一片喧哗声震乾坤。}

\vspace{15pt}
{\hei 附注}:~

\setlength{\parindent}{0pt}{
《顺天时报》曾刊《逍遥津任辰辙之唱词》一文,所载词句与刘曾复先生所传词句非常相近,照录供参考(张斯琦{\scriptsize 君}提供)
}

\vspace{10pt}
{\centerline{\textcolor{blue}{\hei《逍遥津》``任辰''辙之唱词}}}
\vspace{10pt}

\setlength{\parindent}{22pt}{     %
	{\hwfs 旧本《逍遥津》,``欺寡人''一段,俱用``由求''辙,戏中汉献帝唱``欺寡人好一比鹰抓兔胁''句,过于俚俗,殊伤大雅。刘鸿升未故时,将``由求''改``任辰'',虽亦不免俗,但较旧本,似觉雅驯。李桂芬唱《逍遥津》,亦用斯词,爰将改词录于左端:~}}

\vspace{15pt}
\setlength{\parindent}{0pt}{
	{\hei 汉献帝在后宫伤心难忍,可叹我父子们悲切切冷清清、求生不得求死不能、好不惨情。叹伏后到此时难保活命,我君妃生离散惨不忍闻。二皇儿年幼小孩童之性,哭啼啼与孤王要他的娘亲。想奸贼不由孤嚼牙愤恨,上欺天子下压群臣。欺寡人(贼)带剑上殿孤见他不敢责问,欺寡人(贼)独霸朝纲、目无君、自耑自尊。欺寡人孤只得百般谨慎,欺寡人孤只得时刻留神。欺寡人(贼)奏本是非曲直孤不敢\textcolor{red}{$\square~\square$},欺寡人孤有命贼大胆妄为抗旨不遵。欺寡人贼自由行孤不敢过问,欺寡人孤怒不敢言、忍耐在心。欺寡人孤见他气色不正嚇得孤乱了方寸,欺寡人孤见他怒发威嚇得吊胆提心。欺寡人蹂躏百般、惨忍万分,欺寡人贼败坏纲常、逆了五伦。欺寡人好一似主仆受训,欺寡人好一似虐待家人。欺寡人好一似无辜良民被贼围困,欺寡人好一似冤屈罪犯\textcolor{red}{$\square$}而受刑。欺寡人好一似蛇毒蝎狠,欺寡人好一似虎狼把孤吞。欺寡人好一似前世冤孽今生报应,欺寡人好一似夹路相逢对头仇人。欺寡人好一似阎王索命,欺寡人好一似饿鬼勾魂。欺寡人好一似败阵惨兵无投奔,反被贼困垓心、难逃命、难生存、认贼斩、恁贼擒,孤做一待毙谁来救应,又听得宫门外喧哗之声。
}}


%\addcontentsline{toc}{section}{\hfill[\hei 附录]\hfill}
\newpage
\chead{附~录} % 页眉中间位置内容
\textbf{柴桑口}\protect\hyperlink{fn636}{\textsuperscript{636}}

{[}第一场{]}

(四文堂,刘备、诸葛亮上)

刘备 {[}引子{]}祸福惟天造,岂在人谋计巧。

诸葛亮
{[}引子{]}秦、崔玉童\protect\hyperlink{fn637}{\textsuperscript{637}}到,从此知音稀少。

诸葛亮 参见主公。

刘备 先生少礼,请坐。

诸葛亮 谢座。

刘备
(念)周战场中祸难平,岂知转福结□□。\protect\hyperlink{fn638}{\textsuperscript{638}}

诸葛亮 (念)自古辛勤有天下,不在人谋定相星。

刘备 孤,刘备。

诸葛亮 山人诸葛亮。

刘备 先生。

诸葛亮 主公。

刘备
日前周郎设下``假途灭虢''之计,被先生奇谋,只气得他喷血坠马\protect\hyperlink{fn639}{\textsuperscript{639}},此时未闻周郎吉凶如何。

诸葛亮 亮夜观天象,见将星坠落。我料周郎刻下必死无疑也。

刘备 此话难料。

旗牌 (内)走哇。

(旗牌上)

旗牌
启上主公,昨日命小人持书递进吴营,周瑜拆开一观,忽然呕吐气绝身亡。今将灵柩移至柴桑去了。

诸葛亮 如何。

刘备 知道了。

诸葛亮 退下。

(旗牌下)

刘备 果然不出先生所料,周郎已死,还当如何?

诸葛亮
我料代周郎之权必是鲁肃。亮观天象,见将星聚在东吴,我当以过江吊孝为由,好觅贤士辅佐\protect\hyperlink{fn640}{\textsuperscript{640}}主公。

刘备
且慢,东吴将士恨先生如入骨髓,此番前去,必遭其害。且莫做那披麻救火,自惹其祸。

诸葛亮
周瑜在日,亮犹不惧,今瑜(已)死\protect\hyperlink{fn641}{\textsuperscript{641}},何足惧哉?

刘备 去得的?

诸葛亮 去得的。

刘备 无妨事?

诸葛亮 无妨事。

刘备
孤放心不下,可命三弟带领铁骑一万,战船千只,跟随先生前往。孤也好放心。

诸葛亮
既蒙主公垂念,就命四将军子龙随我前往,可命三将军在江南岸上等候。山人吊祭已毕,过江之时,用羽扇一招,前来接应便了。

刘备 先生呐。

刘备
【西皮散板\protect\hyperlink{fn642}{\textsuperscript{642}}】非是孤王人安顿,东吴尽是豺狼虎群。此番前去稍有伤损,岂不教孤两离分\protect\hyperlink{fn643}{\textsuperscript{643}}。

诸葛亮
【西皮散板\protect\hyperlink{fn644}{\textsuperscript{644}}】时不到兴与衰天心造定,当治乱自有那一辈时人。【转西皮快板】天生来周公瑾吴邦英俊,偏又有诸葛亮汉室称臣。三江口协力时同把曹併,彼爱我我爱彼各无异心。他起意三次里害我性命,心未遂气得他丧了残生。在帐中施一礼主心安稳,

刘备 先生保重了。

诸葛亮 【西皮散板】但看我一叶飘如风送云。

刘备
【西皮散板】他那里坦然不虑心安稳,孤心中不定胆战心惊。但愿得此一去吉星照定,且待他无凶险也好放心。

(同下)

{[}第二场{]}

(四白文堂、二旗牌、鲁肃上)

鲁肃
【西皮摇板\protect\hyperlink{fn645}{\textsuperscript{645}}】伤我那擎天柱一旦早丧,眼见得我东吴谁是栋梁。蒙主恩虽与我兵权执掌,愧匪才怎做得治国安邦。

鲁肃
下官鲁肃,不料孔明用计,竟将公瑾气死。蒙公瑾生前已奏表章,举我统领兵权。唉,我自愧匪才,焉能掌此大权,无奈主公再三\protect\hyperlink{fn646}{\textsuperscript{646}},难以推却,只得勉力而为。今将公瑾灵柩移至柴桑,等候他子周循\protect\hyperlink{fn647}{\textsuperscript{647}}到来,成服\protect\hyperlink{fn648}{\textsuperscript{648}}丧葬也。

四将 (内)走哇。

(四将上)

四将 参见都督。

鲁肃 列公少礼。

(鲁肃看介)

鲁肃 列公怒气不息,进帐何事?

四将
启禀都督,今有孔明带了祭礼前来祭奠先帅,故而进帐请教都督:还是将孔明杀了后祭,还是祭了后杀呢?

鲁肃 哦,那孔明竟敢前来吊祭先帅么?

四将 正是。

鲁肃 他带了多少人马?

四将 只有一叶扁舟,并无人马。

鲁肃 只有一叶扁舟,并无人马?

四将 正是。

鲁肃
(冷笑介)哼,呵呵\ldots{}\ldots{}这厮又来作怪。我东吴将士恨不得食尔之肉,他偏偏驾一叶扁舟而来。孔明啊孔明,你今前来岂不是羊入虎口。不知列公心意如何?

四将
吾家先帅被他气死,恨不得手刃此贼\protect\hyperlink{fn649}{\textsuperscript{649}}。他今自寻前来,都督传令可将他剖腹挖心\protect\hyperlink{fn650}{\textsuperscript{650}},活祭先帅,以报此仇。

鲁肃
不可。他今前来,定有一番道理。若凶惧而杀之,天下人道我东吴无容人之量。且待他祭奠之后,先用言语责他,然后治死也还不迟。不知列公意下如何?

四将 只是教那贼多活一时。

赵云 (内)孔明先生到。

鲁肃 列公不可造次,当遵吾令。

四将 遵命。

鲁肃 有请。

四将 有请。

(赵云、童儿、诸葛亮上)

鲁肃 先生。

诸葛亮 都督。

鲁肃 呀,先生一别有年,使人梦想。

诸葛亮
{久未晤面}(或:久违教益),如有所亡\protect\hyperlink{fn651}{\textsuperscript{651}}。

鲁肃
荷蒙\protect\hyperlink{fn652}{\textsuperscript{652}}足下远来吊祭,足见多情。

诸葛亮 故交之谊,聊此一行,以表寸意。

鲁肃 多谢了。

诸葛亮 先灵供在何处?

鲁肃 现在后帐。待下官引路。

诸葛亮 有劳了。

(同下)

{[}第三场{]}

(又上)

诸葛亮
\textless{}\textbf{三叫头}\textgreater{}公瑾,先生,唉,都督哇,呃\ldots{}\ldots{}(哭介)

诸葛亮
【二黄导板\protect\hyperlink{fn653}{\textsuperscript{653}}】{见陵寝}(或:见灵寝)不由人泪如雨降,想俊容不由人痛断肝肠。

诸葛亮
\textless{}\textbf{三叫头}\textgreater{}公瑾,先生,唉,都督哇,呃\ldots{}\ldots{}(哭介)

诸葛亮 【二黄散板】可惜你钟山秀{春年正旺}(或:春华正旺),

诸葛亮
【反二黄原板\protect\hyperlink{fn654}{\textsuperscript{654}}】可惜你美英才一旦夭亡。可惜你空碌碌一生容让,可惜你兢业业半世奔忙。实指望併曹瞒你我安享,\textless{}\textbf{哭头}\textgreater{}都督哇\ldots{}\ldots{}

诸葛亮
【反二黄原板】又谁知黄粱梦\protect\hyperlink{fn655}{\textsuperscript{655}}昙花一场。

旗牌 进位,上香,鞠躬。跪,叩首,二叩首,三叩首。兴,鞠躬。

赵云 (念)
呜呼公瑾,不幸夭亡!寿短固天,人岂不伤!我心实痛,酬酒一觞;君其有灵,想我衷肠。

诸葛亮
公瑾,想你英雄盖世,一代风流。贯精忠于日月,秉赤胆与东吴。不幸一旦身故,未遂你胸中之志,好不遗恨人也。

诸葛亮
【反二黄慢板\protect\hyperlink{fn656}{\textsuperscript{656}}】你是个霸业的忠贞良将,你是个振东吴豪杰贤良。谁似你天生来高智雅量,谁似你文武略器宇轩昂。谁似你青年人兵权执掌,谁似你定霸业扶弱抑强。奸曹贼统雄兵如风似浪\protect\hyperlink{fn657}{\textsuperscript{657}},只吓得江南士束手要降。若非你怀大志陈兵相抗\protect\hyperlink{fn658}{\textsuperscript{658}},运机谋烧得他抛甲弃枪。到今日稍得遂太平景象\protect\hyperlink{fn659}{\textsuperscript{659}},转瞬间\protect\hyperlink{fn660}{\textsuperscript{660}}天不佑大厦断梁。抛得我故人儿将谁依傍,\textless{}\textbf{哭头}\textgreater{}公瑾呐\ldots{}\ldots{}

诸葛亮
【反二黄原板】闪得我前后事与谁商量。今日里原比作那伯仲情况,我与你又好比鸡黍范、张\protect\hyperlink{fn661}{\textsuperscript{661}}。{望阴灵鉴吾这虔诚祭享,虔诚祭享,泪纷纷捧玉樽享受烝尝。}(或:望阴灵鉴之我祭奠,知我祭奠,诸葛亮泪纷纷痛断肝肠。\protect\hyperlink{fn662}{\textsuperscript{662}})

旗牌 进位,上香,退,鞠躬。

赵云
(念)呜呼公瑾!生死永别!朴守其真,冥冥灭灭,君如有灵,乞见我心:从此天下,更无知音!呜呼痛哉,伏惟上享。

诸葛亮 公瑾呐\ldots{}\ldots{}

诸葛亮
【反二黄散板】你非是妒贤辈胸怀愚量,都只是各为主不得不防。到今日奸曹在你命身丧,\textless{}\textbf{哭头}\textgreater{}都督哇\ldots{}\ldots{}

诸葛亮
【反二黄散板】闷得我诸葛亮心意彷徨。思至此哭得我{咽喉气颡}(或:咽喉难让),

众
【反二黄散板】只哭得满营中泪洒千行。\protect\hyperlink{fn663}{\textsuperscript{663}}

鲁肃 先生且免悲伤,还当同心破曹要紧。

众 是呀,同心破曹,全仗先生。

诸葛亮
亮纵有千言万语,一时难以尽诉。列公以奸曹为念,亮当佩服,与公同心破曹,就此告辞了。

鲁肃 且慢,备得水酒,聊表地主之情。

诸葛亮 本当领受,怎奈有公务在身,告辞了。

鲁肃 有慢了。

诸葛亮 \textless{}\textbf{叫头}\textgreater{}公瑾!

诸葛亮 你若有灵,须见我心呐!

诸葛亮 【反二黄散板】心问口、口问心牢骚千状,有万篇写不尽我心哀伤。

(诸葛亮出门)

众 送先生。

诸葛亮 【反二黄散板】送千里终须别何须谦让,

鲁肃 恕不远送了。

(赵云下)

诸葛亮
【反二黄散板】试看我一帆风雨洒康庄\protect\hyperlink{fn664}{\textsuperscript{664}}。

(诸葛亮下)

鲁肃 【反二黄散板】他那里诚恳恳哀泣模样,为朋友可算得古道热肠。

鲁肃 列公,人言先帅与孔明不睦,今日一见真乃伤情。

众 看将起来,诸葛先生乃是大大的好人。

二旗牌 (内)走哇。

(二旗牌上)

二旗牌 公子到。

鲁肃 有请。

众 有请。

(周循上)

周循 伯父。

鲁肃 公子,不知公子驾到,未曾远迎,面前恕罪。

周循 岂敢。小侄来的鲁莽,伯父、众位伯父恕罪。

众 岂敢。

周循 我父灵堂今在何处?

鲁肃 现在后帐,随我来。

周循 有劳伯父。

(同一翻两翻,周循看介、哭介)

周循 唉,爹爹呀\ldots{}\ldots{}(哭介)

周循 【二黄散板】在灵位不由我死去又醒,

周循
\textless{}\textbf{三叫头}\textgreater{}爹爹,我父,唉,爹爹呀\ldots{}\ldots{}(哭介)

周循
【二黄散板】想今生要见面万万不能。为国家受尽了千般苦衷,谁信那青史上万载标名。

鲁肃 啊公子,先帅已死,不能复生,请自保重要紧。

众 是啊,请自保重要紧。

周循 诚领众位伯父之教,小侄敢不从命。啊伯父,小侄有一言不知可听否?

鲁肃 公子有话请讲。

周循
先父执掌兵权有年,不料被孔明三气而死,我与他不共戴天之仇,望伯父助小侄一膀之力,杀往荆(州\protect\hyperlink{fn665}{\textsuperscript{665}}),生擒孔明,与我父报仇雪恨。伯父料无推辞的了。

鲁肃 啊公子,那孔明乃是个好人。

周循 啊?怎见得他是好人?

鲁肃 他闻先帅已死,过得江来哭了又祭,祭了又哭,岂不是个好人?

周循 啊,那孔明几时来的?

鲁肃 方才在此。

周循 如今何在?

鲁肃 回往荆州去了。

周循 啊伯父,你要与我统领人马追杀孔明,如若不然,我就碰死在灵前。

众 公子不必如此,大家追杀孔明便了。

周循 好,快快追赶。

鲁肃 去不得。

(同下)

{[}第四场{]}

(赵云、童儿、诸葛亮上)

诸葛亮
【西皮散板】非是我笑他们无有志量,怎知我袖儿内暗有行藏。{遇不着智谋人心中惆怅}(或:将身儿来至在江边岸上),

(庞统上)

庞统 呔,你往哪里走。

庞统 【西皮散板】你纵有托天手(或:托天胆)难逃罗网。

庞统
孔明呐孔明,你用计将周郎三气而死,又假意过江吊祭,分明笑我东吴无有能人。来来来,我与你较量。

诸葛亮 原来是凤雏先生。先生平生大才,今日出此不经之言,故意吓我。

诸葛亮、庞统 哈哈,哈哈,啊哈哈哈\ldots{}\ldots{}(诸葛亮、庞统对笑介)

诸葛亮
啊先生,我此来实为吾兄,我料仲谋必不能重用足下,玄德公宽仁厚德,(不负\protect\hyperlink{fn666}{\textsuperscript{666}})公生平所学,我有草书一封,趁便即往\protect\hyperlink{fn667}{\textsuperscript{667}}荆州共扶汉室,名垂千古,岂不美哉?

庞统
承蒙美意,自得遵教\protect\hyperlink{fn668}{\textsuperscript{668}}。

(起\textless{}\textbf{鼓架子}\textgreater{},诸葛亮、庞统两望)

庞统 啊先生,看那旁人声呐喊,必是周循追赶前来。

诸葛亮 公且自回避,亮要登舟去了。

庞统 后会有期。

诸葛亮 请呐。

诸葛亮 【西皮散板】此时节说不尽话言惆怅,

庞统 【西皮散板】暂分别改日里再会荆襄。

(庞统下)

诸葛亮 哈哈哈\ldots{}\ldots{}(笑介)

诸葛亮
【西皮散板】想人生荣与枯得失难量,际风云{显奇谋}(或:显奇能)姓字名扬。望一派{白茫茫}(或:白亮亮){翻江波浪}(或:滔天波浪),

张飞 (下场门内)嘚,开船。

(四黑龙套、二船夫、张飞上)

张飞 【西皮散板】张翼德接先生来到长江。

张飞 先生,搭了扶手。

(诸葛亮、赵云、童儿上船介;四白文堂、四将、周循上)

周循 呔,船头之上,可是诸葛亮?

诸葛亮 然也,来者何人?

周循 俺乃公瑾之子周循是也。

诸葛亮 哦,原来是公子到了,敢莫是与父谢孝的么?

周循 正是。

诸葛亮 为何持戈相向,是何理也?

周循 请先生上岸,周循有话言讲。

诸葛亮
哼呵呵呵\ldots{}\ldots{}(冷笑介)我若上岸,只恐你那小性命必随儿父去也。

周循 呔,孔明你若不上岸,休怪周循无礼了。

张飞
呔,我把你这不孝的乳臭小儿,汝父既死,儿不居守灵帐,执持器械,何以成孝?你这不忠不孝、不仁不义之人,要儿何用!先生闪开,待咱老张将他射死也。

诸葛亮 不可,饶他这条小命去罢。

张飞 也罢,念尔有重孝在身,暂且饶儿不死。嘚,开船!

(张飞三笑,诸葛亮众下)

周循 苍天呐苍天,

周循
【西皮摇板\protect\hyperlink{fn669}{\textsuperscript{669}}】满江洒下青丝网,怎奈鱼儿又脱缰。

周循 罢!

(周循跳水介,四将拦介)

四将 公子不必如此,驾船追杀孔明便了。

周循 好。驾船追者!

(同下)

注:钞本中诸葛亮祭奠周瑜的祭文,个别词句与《三国演义》原文音同字异,今将《三国演义》中诸葛亮祭文附后:

\textbf{呜呼公瑾,不幸夭亡!修短故天,人岂不伤?}

\textbf{我心实痛,酹酒一觞;君其有灵,享我烝尝!}

\textbf{吊君幼学,以交伯符;仗义疏财,让舍以民。}

\textbf{吊君弱冠,万里鹏抟;定建霸业,割据江南。}

\textbf{吊君壮力,远镇巴丘;景升怀虑,讨逆无忧。}

\textbf{吊君丰度,佳配小乔;汉臣之婿,不愧当朝,}

\textbf{吊君气概,谏阻纳质;始不垂翅,终能奋翼。}

\textbf{吊君鄱阳,蒋干来说;挥洒自如,雅量高志。}

\textbf{吊君弘才,文武筹略;火攻破敌,挽强为弱。}

\textbf{想君当年,雄姿英发;哭君早逝,俯地流血。忠义之心,英灵之气;命终三纪,名垂百世,哀君情切,愁肠千结;惟我肝胆,悲无断绝。}

\textbf{昊天昏暗,三军怆然;主为哀泣;友为泪涟。亮也不才,丐计求谋;助吴拒曹,辅汉安刘;}

\textbf{掎角之援,首尾相俦,若存若亡,何虑何忧?}

\textbf{呜呼公瑾!生死永别!朴守其贞,冥冥灭灭,魂如有灵,以鉴我心:从此天下,更无知音!}

\textbf{呜呼痛哉!伏惟尚飨。}

\newpage
\subsubsection{铁笼山·迷当发点~\protect\footnote{根据刘曾复先生钞录本整理。钞本作``铁龙山'',此处从《三国演义》原文。}}%\protect\hyperlink{fn670}{\textsuperscript{670}}
\addcontentsline{toc}{subsection}{\hei 铁笼山·迷当发点}

\hangafter=1                   %2. 设置从第1⾏之后开始悬挂缩进  %
\setlength{\parindent}{0pt}{
%{\centerline{\textrm{{[}\hei 第一场{]}}}}

	(\textless{}\!{\bfseries\akai 水龙吟}\!\textgreater{}{\hwfs 四}蛮兵、{\hwfs 二丑}校尉{\hwfs 站门};~\textless{}\!{\bfseries\akai 四击头}\!\textgreater{}迷当{\hwfs 上})

迷当\hspace{30pt} \textless{}\!{\bfseries\akai 点绛唇}\!\textgreater{}西羌英豪,儿郎虎豹,统雄骁,族裔三苗,灭魏蜀汉保。

(\textless{}\!{\bfseries\akai 水龙吟}{\akai 合头}\!\textgreater{}迷当{\hwfs上高台},{\hwfs 坐})

迷当\hspace{30pt} ({\akai 念})三国纷纷起战争,孔明火烧藤甲兵。七擒孟获孤得见,西羌领兵到如今。

\setlength{\hangindent}{60pt}{   %3. 设置悬挂缩进量                %
迷当\hspace{30pt} 孤,西羌国王迷当,长子迷强、次子迷能俱丧陈泰之手,为此每日操练人马,以防不测。看今日天气晴和,不免去往草上坡行围射猎,孩子们、马夫们走上。
}

(内\hspace{35pt} 马夫们走上。)

({\hwfs 四}马夫{\hwfs 上})

\setlength{\hangindent}{60pt}{   %3. 设置悬挂缩进量                %
四马夫\hspace{20pt} ({\akai 念})\textless{}\!{\bfseries\akai 马夫赞}\!\!\textgreater{}生来本是西凉娃,穿山越岭骑劣马。冲锋陷阵咱不怕,途程当玩耍,途程当玩耍。({\hwfs 边走边念})
}

四马夫\hspace{20pt} 参见大王。

迷当\hspace{30pt} 传蛮女们进见。

(马夫{\hwfs 下},马夫甲、乙、丙、丁{\hwfs 先后各拉}蛮女甲、乙、丙、丁{\hwfs 先后上},{\hwfs 分段唱}\textless{}\!{\bfseries\akai 粉蝶儿}\!\textgreater{})

蛮女甲\hspace{20pt} \textless{}\!{\bfseries\akai 粉蝶儿}\!\textgreater{}异国异苗,

蛮女乙\hspace{20pt} \textless{}\!{\bfseries\akai 粉蝶儿}\!\textgreater{}小蛮婆,异国异苗;

蛮女丙\hspace{20pt} \textless{}\!{\bfseries\akai 粉蝶儿}\!\textgreater{}天生就玉容花貌,

蛮女甲\hspace{20pt} \textless{}\!{\bfseries\akai 粉蝶儿}\!\textgreater{}镇日里舞剑操刀,背弯弓、发硬驽、穿杨技巧。

四蛮女\hspace{20pt} ({\akai 合})\textless{}\!{\bfseries\akai 粉蝶儿}\!\textgreater{}兴来时马上嬉游,弹一曲昭君宫调。

四蛮女\hspace{20pt} 参见大王。

迷当\hspace{30pt} 罢了。

(马夫、蛮女{\hwfs 分站两边})

迷当\hspace{30pt} 孩子们!

(众{\hwfs 应})

迷当\hspace{30pt} 草上坡去者!

(众{\hwfs 应},众{\hwfs 唱分段}\textless{}\!{\akai 北}{\bfseries\akai 泣颜回}\!\textgreater{})

众\hspace{40pt} \textless{}\!{\akai 北}{\bfseries\akai 泣颜回}\!\textgreater{}驱队出西郊,

(众{\hwfs 合龙},迷{\hwfs 下高台},{\hwfs 上马})

众\hspace{40pt} \textless{}\!{\akai 北}{\bfseries\akai 泣颜回}\!\textgreater{}逐骅骝人拥哎咆哮,

(众{\hwfs 转场},迷{\hwfs 上大边高台})

众\hspace{40pt} \textless{}\!{\akai 北}{\bfseries\akai 泣颜回}\!\textgreater{}貔貅簇拥,人如虎生翼英豪。

(众{\hwfs 转场},迷{\hwfs 下高台},{\hwfs 上小边高台})

众\hspace{40pt} \textless{}\!{\akai 北}{\bfseries\akai 泣颜回}\!\textgreater{}旗旛耀日,韵悠悠,画角连珠炮朴咚咚。

(迷{\hwfs 下高台})

众\hspace{40pt} \textless{}\!{\akai 北}{\bfseries\akai 泣颜回}\!\textgreater{}紧擂鼍鼓,布围场满塞弓刀,布围场满塞弓刀。

({\hwfs 归正场})
}

\newpage
\subsubsection{\textrm{美良川 {\small 之} 秦琼}~\protect\footnote{根据刘曾复先生钞录的秦琼``单词本''整理。\\
	\vspace{7pt}
\hangafter=1                   %2. 设置从第1⾏之后开始悬挂缩进  %
\setlength{\parindent}{0pt}{
	此戏花脸唱一支《八声甘州》,据《梅兰芳回忆录:舞台生活四十年》\upcite{Mei-Remember}%\textsuperscript{{[}28{]}.}
记载,词句为:~\\
	\vspace{5pt}
{\hei ``扬威奋勇,看愁云惨惨,杀气腾腾。鞭鞘指处,鬼神尽觉惊恐。三关怒冲千里振,八寨雄兵已成空。旌旗摇,剑戟丛,将军八面展威凤。人似虎,马如龙,伫看一战使成功!''}\\
	\vspace{7pt}
《铁笼山》一剧中姜维唱的《八声甘州》即出自此戏。}
}}%\protect\hyperlink{fn671}{\textsuperscript{671}}
\addcontentsline{toc}{subsection}{\hei 美良川~\small{之}~秦琼}

\hangafter=1                   %2. 设置从第1⾏之后开始悬挂缩进  %
\setlength{\parindent}{0pt}{
{\centerline{\textrm{{[}\hei 第一场{]}}}}
\vspace{5pt}

({\hwfs 上})

({\akai 念})头戴金盔凤翅飘,身穿铠甲络丝绦。劈抡双锏无人抵,保定我主锦龙朝。

俺,姓秦名琼字叔宝,唐室驾前为臣。奉主之命跟随二主千岁大战刘武周,可恨那贼战又不战,降又不降。今日闲暇无事,不免到二主营中问安。

吓!进得营来为何这样静悄悄的,待我两厢问来。三军们,主公可在营中?

哪里去了?

何人保驾?

不好了!

且住!二主夜探白璧关\footnote{刘曾复先生钞本作``北璧关'',此处从《说唐全传》;历史上李世民破刘武周麾下尉迟敬德于美良川,是其平定北方割据势力刘武周、宋金刚的关键战役(柏壁之战)的一部分,《旧唐书》、《新唐书》和《资治通鉴》均有记载。}%\protect\hyperlink{fn672}{\textsuperscript{672}}
,咬金保驾岂是那黑贼对手?众将官,迎上前去。({\hwfs 下})

\vspace{3pt}{\centerline{\textrm{{[}{\hei 第二场}{]}}}}\vspace{5pt}

({\hwfs 上})

呔!尔有何本领,擅敢追杀我主?

若问你老爷的,尔且听道。

呔!尔敢是怯战?你我两厢问来。

三军的,哪里宽阔?

打道美良川。({\hwfs 下})

\vspace{3pt}{\centerline{\textrm{{[}{\hei 第三场}{]}}}}\vspace{5pt}

({\hwfs 上})

来到美良川,你我怎样比试?

下得马来,何以为赌?何为打鞭换锏?

如此说来老爷先打。

老爷先打。

好,你我两厢问来。

三军的,何处地界?

黑贼,乃是我唐室地界,还是老爷先打。

哼!老爷打了就无有尔的份了。让尔先打。

这作什么?

你老爷站得稳,尔只管的打。

吐什么?

你老爷焉有吐红之理?尔只管的打。

又吐什么?

你老爷方才言过焉有吐红之理,尔只管的打。

你敢有逃走之意?若要你老爷不打,除非在老爷胯下趱将过去,俺便饶尔不死。

当真要打?

果然要打?

要打?

起鼓招打。

呔!尔为何闪你老爷这头一锏?

九十斤一根,

慢说是两锏,就是这一锏也要结果尔的性命。

当真要打?

果然要打?

要打?

起鼓招打\footnote{刘曾复先生钞本作``起鼓照打'',此处从段公平{\scriptsize 君}建议,为上下文统一改。}
%\protect\hyperlink{fn673}{\textsuperscript{673}}
。

与你老爷吐,吐红。

起鼓招打。

带马。({\hwfs 追下})

\vspace{3pt}{\centerline{\textrm{{[}{\hei 第四场}{]}}}}\vspace{5pt}

({\hwfs 上})

前道为何不行?

人马列开。

【{\akai 西皮摇板\footnote{刘曾复先生钞本未注明板式。}%\protect\hyperlink{fn674}{\textsuperscript{674}
}】秦琼生来不可当\footnote{``不可当''犹言``不得了''之意。}
%\protect\hyperlink{fn675}{\textsuperscript{675}}
,美良川前摆战场。三鞭打不动秦叔宝,两锏打得他吐红光。

败兵不可追赶,人马回营。({\hwfs 下})


\newpage
\phantomsection %实现目录的正确跳转
%\hypertarget{ux53d6ux91d1ux9675-ux4e4b-ux66f9ux826fux81e3}{%
\section*{取金陵~{\small 之}~曹良臣}%\label{ux53d6ux91d1ux9675-ux4e4b-ux66f9ux826fux81e3}}
\addcontentsline{toc}{section}{\hei 取金陵~{\small 之}~曹良臣}

\hangafter=1                   %2. 设置从第1⾏之后开始悬挂缩进  %
\setlength{\parindent}{0pt}{
{\centerline{\textrm{{[}\hei 第一场{]}}}}
\vspace{5pt}

({\akai 念})威震金陵谁敢犯,一片丹心保皇朝。

本帅,曹良臣。

今有红巾贼寇,兴兵犯境。自古道:兵行千里,不战自倦。今晚末将带兵前去劫营。都督大兵随后接应,大功必成也。

得令!

【{\akai 西皮摇板}\footnote{刘曾复先生钞本未注明板式,下同。}%\protect\hyperlink{fn676}{\textsuperscript{676}}
】都督传令如雷吼,扫灭红巾统貔貅。三军带马出帐口,不灭红巾誓不休。

\vspace{3pt}{\centerline{\textrm{{[}{\hei 第二场}{]}}}}\vspace{5pt}

【{\akai 西皮摇板}】旌旗招展绕星斗,金枪一举鬼神愁。三军催马朝前走,抬头只见一营头。

踹营!

哎呀!

【{\akai 西皮摇板}】只望劫营能成就,谁知贼有巧机谋。三军随爷绕营走,

\vspace{3pt}{\centerline{\textrm{{[}{\hei 第三场}{]}}}}\vspace{5pt}

【{\akai 西皮摇板}】适才闪出红巾寇,不由怒火起心头,三军催马踹营走,

杀!

哎呀!

【{\akai 西皮摇板}】只望今晚擒贼首,又恐中贼奸机谋。三军随爷夺路走,

\vspace{3pt}{\centerline{\textrm{{[}{\hei 第四场}{]}}}}\vspace{5pt}

【{\akai 西皮摇板}】四面俱是红巾寇,口口叫我把降投。东杀、西挡无路走------

然!

答话者何人?

唔哙呀!闻得徐达用兵如神,果然话不虚传。此乃天教俺归降也!

【{\akai 西皮摇板}】人言徐达韬略有,提兵调将似武侯。甩镫离鞍卸甲胄,含羞带愧把他投。

归降来迟,死罪呀死罪。

赤福寿人马,元帅要提防一二。\footnote{刘曾复先生钞本注明``以上《取金陵》曹良臣''。刘曾复先生钞本未注明场次,有关场次安排据《京剧汇编》第十九集~阎庆林~藏本,该藏本系阎岚秋(九阵风)生前演出本。}%\protect\hyperlink{fn677}{\textsuperscript{677}}
}

\item
  \leavevmode\hypertarget{fn636}{}%
  根据刘曾复先生钞录本整理。刘曾复先生钞本注``马少山本,民廿九 1940
  王存''。段公平君按:据何毅老师介绍,刘曾复先生本有意为此剧本润色文辞,后因言派《卧龙吊孝》已流行,``我就别招这个讨厌了'',竟未完成。因此抄本中讹误脱漏较多,且多有刘曾复先生修改和原文并存痕迹。这次整理,在文辞通顺的基础上拟对这些痕迹适当保留。刘曾老所改录在正文,抄本原文录在脚注中。抄本整理过程中,幸得``小豆子''老师惠赐《柴桑口》余胜荪藏本扫描件,和此抄本同质性很高,故多用为参考。\protect\hyperlink{fnref636}{↩}
\item
  \leavevmode\hypertarget{fn637}{}%
  秦即地狱秦广王,专司人间夭寿;崔即判官崔钰,掌生死簿。刘曾复先生钞本中,``童''字不确认,疑``章''字;玉童,即仙童。\protect\hyperlink{fnref637}{↩}
\item
  \leavevmode\hypertarget{fn638}{}%
  刘曾复先生钞本``周''字不确认,亦欠通;``结''字不确认,后二字缺。段公平君按:据余胜荪藏本,此四句诗为``干戈有时化玉帛,蜀吴修好结姻亲。炎汉正统有天相,不须人谋定隆兴。''和此本中的诗应该颇相关。因此推断第二句所缺两字也是``姻亲''之类。第一句``难''字不能确认,或疑为``虽'',结合文意,作``难''字似乎较合理。\protect\hyperlink{fnref638}{↩}
\item
  \leavevmode\hypertarget{fn639}{}%
  段公平君按:刘曾复先生钞本作``赞马'',似欠通,今据文意及《京剧汇编》第九十三集
  余胜荪藏本改。\protect\hyperlink{fnref639}{↩}
\item
  \leavevmode\hypertarget{fn640}{}%
  刘曾复先生钞本作``扶佐''。\protect\hyperlink{fnref640}{↩}
\item
  \leavevmode\hypertarget{fn641}{}%
  刘曾复先生钞本作``今瑜死``,此处从《三国演义》原文。\protect\hyperlink{fnref641}{↩}
\item
  \leavevmode\hypertarget{fn642}{}%
  刘曾复先生钞本未注明西皮或二黄板式,今据后文及《京剧汇编》第九十三集
  余胜荪藏本推测。\protect\hyperlink{fnref642}{↩}
\item
  \leavevmode\hypertarget{fn643}{}%
  刘曾复先生钞本``教''、``两''二字不确认,据文意推断。\protect\hyperlink{fnref643}{↩}
\item
  \leavevmode\hypertarget{fn644}{}%
  刘曾复先生钞本注``散(板)转快(板)或二六''。\protect\hyperlink{fnref644}{↩}
\item
  \leavevmode\hypertarget{fn645}{}%
  刘曾复先生钞本未注明板式,今据《京剧汇编》第九十三集
  余胜荪藏本添。\protect\hyperlink{fnref645}{↩}
\item
  \leavevmode\hypertarget{fn646}{}%
  刘曾复先生钞本原文如此,文意欠通。\protect\hyperlink{fnref646}{↩}
\item
  \leavevmode\hypertarget{fn647}{}%
  刘曾复先生钞本作``周巡'',此处从《三国演义》原文,作``周循''。\protect\hyperlink{fnref647}{↩}
\item
  \leavevmode\hypertarget{fn648}{}%
  旧时死者入殓后,其亲属穿着符合各自身分的丧服,称为``成服''。\protect\hyperlink{fnref648}{↩}
\item
  \leavevmode\hypertarget{fn649}{}%
  刘曾复先生钞本作``千刃此贼'',此处据《京剧汇编》第九十三集
  余胜荪藏本改。\protect\hyperlink{fnref649}{↩}
\item
  \leavevmode\hypertarget{fn650}{}%
  刘曾复先生钞本作``破腹挖心''。\protect\hyperlink{fnref650}{↩}
\item
  \leavevmode\hypertarget{fn651}{}%
  刘曾复先生钞本原录``如有所望'',后改``如有所忘'',段公平君注:``如有所忘''文意欠通。应为``如有所亡'',即``如有所失''意。\protect\hyperlink{fnref651}{↩}
\item
  \leavevmode\hypertarget{fn652}{}%
  荷蒙,犹``承蒙''之意。\protect\hyperlink{fnref652}{↩}
\item
  \leavevmode\hypertarget{fn653}{}%
  刘曾复先生钞本注``导(板)或散(板)''\protect\hyperlink{fnref653}{↩}
\item
  \leavevmode\hypertarget{fn654}{}%
  刘曾复先生钞本注``以下散(板)或反二黄原板''\protect\hyperlink{fnref654}{↩}
\item
  \leavevmode\hypertarget{fn655}{}%
  刘曾复先生钞本作``大数到'',此处据《京剧汇编》第九十三集
  余胜荪藏本改。\protect\hyperlink{fnref655}{↩}
\item
  \leavevmode\hypertarget{fn656}{}%
  刘曾复先生钞本仅注反二黄,今据《京剧汇编》第九十三集
  余胜荪藏本添。\protect\hyperlink{fnref656}{↩}
\item
  \leavevmode\hypertarget{fn657}{}%
  刘曾复先生钞本原录``暗兵机如风波浪'',先生改为``统雄兵如风似浪''。\protect\hyperlink{fnref657}{↩}
\item
  \leavevmode\hypertarget{fn658}{}%
  刘曾复先生钞本原录``沉兵相挡'',先生改为``陈兵相抗''。\protect\hyperlink{fnref658}{↩}
\item
  \leavevmode\hypertarget{fn659}{}%
  刘曾复先生钞本原录``烧得谁太平青浪'',先生改为``稍得遂太平景象''。\protect\hyperlink{fnref659}{↩}
\item
  \leavevmode\hypertarget{fn660}{}%
  刘曾复先生钞本原录``话未了'',先生改为``转瞬间''。\protect\hyperlink{fnref660}{↩}
\item
  \leavevmode\hypertarget{fn661}{}%
  ``范张鸡黍''指范式、张劭一起喝酒食鸡。比喻朋友之间情义与深情。刘曾复先生钞本记作``稷黍范张''。\protect\hyperlink{fnref661}{↩}
\item
  \leavevmode\hypertarget{fn662}{}%
  刘曾复先生钞本原有此两句,后改用前句。\protect\hyperlink{fnref662}{↩}
\item
  \leavevmode\hypertarget{fn663}{}%
  刘曾复先生钞本注``全体同唱,鲁(肃)末句''。\protect\hyperlink{fnref663}{↩}
\item
  \leavevmode\hypertarget{fn664}{}%
  刘曾复先生钞本原录``你看我一阵风如在康庄'',先生改。\protect\hyperlink{fnref664}{↩}
\item
  \leavevmode\hypertarget{fn665}{}%
  此处据《京剧汇编》第九十三集
  余胜荪藏本添加。\protect\hyperlink{fnref665}{↩}
\item
  \leavevmode\hypertarget{fn666}{}%
  据``中国京剧戏考''网站载《戏考》本添加。\protect\hyperlink{fnref666}{↩}
\item
  \leavevmode\hypertarget{fn667}{}%
  刘曾复先生钞本原录``趁乘谢之时,去往'',欠通,后改为``趁便即往''。\protect\hyperlink{fnref667}{↩}
\item
  \leavevmode\hypertarget{fn668}{}%
  刘曾复先生钞本作``尊教''。\protect\hyperlink{fnref668}{↩}
\item
  \leavevmode\hypertarget{fn669}{}%
  刘曾复先生钞本未注明板式,今据《京剧汇编》第九十三集
  余胜荪藏本添。\protect\hyperlink{fnref669}{↩}
\item
  \leavevmode\hypertarget{fn670}{}%
  根据刘曾复先生钞录本整理。钞本作``铁龙山'',此处从《三国演义》原文。\protect\hyperlink{fnref670}{↩}
\item
  \leavevmode\hypertarget{fn671}{}%
  根据刘曾复先生钞录的秦琼``单词本''整理。

  此戏花脸唱一支《八声甘州》,据《梅兰芳回忆录:舞台生活四十年》\textsuperscript{{[}28{]}.}记载,词句为:

  ``扬威奋勇,看愁云惨惨,杀气腾腾。鞭鞘指处,鬼神尽觉惊恐。三关怒冲千里振,八寨雄兵已成空。旌旗摇,剑戟丛,将军八面展威凤。人似虎,马如龙,伫看一战使成功!''

  《铁笼山》一剧中姜维唱的《八声甘州》即出自此戏。\protect\hyperlink{fnref671}{↩}
\item
  \leavevmode\hypertarget{fn672}{}%
  刘曾复先生钞本作``北璧关'',此处从《说唐全传》;历史上李世民破刘武周麾下尉迟敬德于美良川,是其平定北方割据势力刘武周、宋金刚的关键战役(柏壁之战)的一部分,《旧唐书》、《新唐书》和《资治通鉴》均有记载。\protect\hyperlink{fnref672}{↩}
\item
  \leavevmode\hypertarget{fn673}{}%
  刘曾复先生钞本作``起鼓照打'',此处从段公平君建议,为上下文统一改。\protect\hyperlink{fnref673}{↩}
\item
  \leavevmode\hypertarget{fn674}{}%
  刘曾复先生钞本未注明板式。\protect\hyperlink{fnref674}{↩}
\item
  \leavevmode\hypertarget{fn675}{}%
  ``不可当''犹言``不得了''之意。\protect\hyperlink{fnref675}{↩}
\item
  \leavevmode\hypertarget{fn676}{}%
  刘曾复先生钞本未注明板式,下同。\protect\hyperlink{fnref676}{↩}
\item
  \leavevmode\hypertarget{fn677}{}%
  刘曾复先生钞本注明``以上《取金陵》曹良臣''。刘曾复先生钞本未注明场次,有关场次安排据《京剧汇编》第十九集
  阎庆林藏本,该藏本系阎岚秋(九阵风)生前演出本。\protect\hyperlink{fnref677}{↩}


%%%%%%%%%%  FOR TEST %%%%%%%%%%%%
\newpage
%\hypertarget{ux9a6cux978dux5c71}{%
%\subsection{马鞍山}\label{ux9a6cux978dux5c71}}
\subsubsection{{\hei\large 马鞍山}}%\protect\hyperlink{fn8}{\textsuperscript{8}}
\addcontentsline{toc}{subsection}{\hei 马鞍山}

%\newcommand{\upcite}[1]{\hspace{0ex}\textsuperscript{\cite{#1}}} %
{\centerline{(李舒~遗作~~根据刘曾复先生手书原稿抄录)}}

\hangafter=1                   %2. 设置从第1⾏之后开始悬挂缩进  %
\setlength{\parindent}{0pt}{
{\centerline{{\bfseries\akai {[}\hei 第一场{]}}}}
\vspace{5pt}

(童儿、俞伯牙\textless{}\!{\bfseries\akai 小锣打上}\!\textgreater{})

\spacept{俞伯牙}{20pt} {[}{\akai 引子}{]}为访贤友,涉水登舟。

\spacept{俞伯牙}{20pt} {[}{\akai 诗}{]}青溪流过碧山头,空水澄鲜一色秋。隔断红尘三十里,白云红叶两悠悠。\protect\hyperlink{fn31}{\textsuperscript{31}}

\spacept{俞伯牙}{20pt} 下官姓俞名瑞字伯牙,(乃)鲁国人氏,晋国为官。只因去岁往各国催贡,船行马鞍山前,偶遇钟子期,我二人共谈琴律,情意相投,结为金兰之好,临行(之时,)赠他黄金二笏,约定今岁中秋还在马鞍山前相会。来此不见贤弟到来,昨晚琴音缭乱\protect\hyperlink{fn32}{\textsuperscript{32}},不知是何缘故,我不免去往集贤村寻访于他便了。

\spacept{俞伯牙}{20pt} 童儿。

(童儿 有。)

\spacept{俞伯牙}{20pt} 看衣改换。

(\textless{}\!{\bfseries\akai 小开门}\!\textgreater{},下,再上)

\spacept{俞伯牙}{20pt} 带了瑶琴,(随我往)集贤村去者!

(一翻两翻,半个\textless{}\!{\bfseries\akai 扯四门}\!\textgreater{},小童一直站大边)

\spacept{俞伯牙}{20pt} 【二黄三眼】我二人在山前金兰结好,今此来他不到所为哪条(或:今此来不见他所为哪条)。换冠裳我亲自义友寻找,此一去集贤村访见故交。

(\textless{}\!{\bfseries\akai 小锣打下}\!\textgreater{})

\vspace{3pt}{\centerline{\textrm{{[}{\hei 第二场}{]}}}}\vspace{5pt}

钟元普\protect\hyperlink{fn33}{\textsuperscript{33}} (内白)走哇。

(提篮子,\textless{}\!{\bfseries\akai 小锣抽头}\!\textgreater{}上)

\spacept{钟元普}{20pt} 唉!

\spacept{钟元普}{20pt} 【二黄摇板】屋漏偏遭连阴雨,破船又遇当头风。

\spacept{钟元普}{20pt} 老汉钟元普。吾儿(或:亡儿)名唤子期。只因去岁,中秋在马鞍山前砍柴,偶遇一位晋国大夫俞伯牙大人,他二人共谈琴律,情意相投,结为金兰之好。临别(或:临行)赠我儿黄金二笏,约定今岁中秋还在马鞍山前相会。谁想我儿回得家来,白日砍柴,夜晚攻书,朝暮积劳,染成疾病。他就此一命身亡了\ldots{}\ldots{}(钟元普哭介)

\spacept{钟元普}{20pt} 咳,今当吾儿百日之期(或:今当亡儿百日之期),为此备了几陌纸钱,去往坟前烧化。天呐,天,({\akai 念})家有万贯终何用,老来无子一场空。

\spacept{钟元普}{20pt} 【二黄原板】老眼昏花路难行,又闻得(或:又听得)松林内百鸟喧声。乌鸦倒有反哺意,羊羔也有跪乳情。似乌云遮住了天边月,似狂风吹散了满天云。这才是黄梅已老青梅落,白发人反送了黑发儿的身。我的儿呀!

(\textless{}\!{\bfseries\akai 小锣抽头}\!\textgreater{}下)

(俞伯牙接\textless{}\!{\bfseries\akai 小锣抽头}\!\textgreater{}上)

\spacept{俞伯牙}{20pt} 【二黄摇板】昨夜晚抚瑶琴暗藏悲调,看起来这内中事有蹊跷。移步儿来至在双阳岔道,(\textless{}\!{\bfseries\akai 小锣抽头}\!\textgreater{}圆场)寻不着集贤村路走哪条?

\spacept{俞伯牙}{20pt} 哎呀且住,来此已是双阳岔道,但不知这集贤村往哪条道路而走。

\spacept{钟元普}{20pt} (内嗽)嗯哼。

\spacept{俞伯牙}{20pt} 看那旁来一老丈。等他到来问明再走。

(钟元普上)

\spacept{钟元普}{20pt} 【二黄摇板】曲弯弯行过了溪边小道,哪有个父与子把纸来烧(或:把纸化烧)。(过大边)

\spacept{俞伯牙}{20pt} 老丈请转。

\spacept{钟元普}{20pt} 呃,原来是一位先生,这位先生可是失迷路途?

\spacept{俞伯牙}{20pt} 正是。

\spacept{钟元普}{20pt} 但不知问的是何所在?

\spacept{俞伯牙}{20pt} 我问的是集贤村。

\spacept{钟元普}{20pt} 先生,你来看:这东去十里也是集贤村,西去十里也是集贤村。但不知是哪个集贤村呢?

\spacept{俞伯牙}{20pt} 这\ldots{}\ldots{}哎呀,贤弟呀,现有两个集贤村(或:既有两个集贤村),为何不对愚兄说明,如今叫我作难了。

\spacept{钟元普}{20pt} 啊先生,敢是指路不明?

\spacept{俞伯牙}{20pt} 呃呃,久住三五载,

\spacept{钟元普}{20pt} 无处不亲连。

\spacept{俞伯牙}{20pt} 正是。

俞伯牙、\spacept{钟元普}{20pt} 啊哈哈哈哈。

\spacept{钟元普}{20pt} 但不知问的是哪一家?

\spacept{俞伯牙}{20pt} 我问的是钟子期。

\spacept{钟元普}{20pt} 哦,钟子期。

\spacept{俞伯牙}{20pt} 正是。

\spacept{钟元普}{20pt} 咳,儿呀。

\spacept{钟元普}{20pt} 【二黄摇板】相逢未说几句话,不由老汉泪如麻。(哭介)

\spacept{钟元普}{20pt} 先生你来迟了。

\spacept{俞伯牙}{20pt} (老丈)何言来迟?

\spacept{钟元普}{20pt} 老汉钟元普。吾儿子期(或:亡儿子期),只因去岁中秋与俞大人结拜(或:与俞大人结为金兰之好),分别之后回到家中,他白日砍柴,夜晚攻书,积劳成疾(或:积劳成病),百日前(他)一命身亡了\ldots{}\ldots{}

\spacept{俞伯牙}{20pt} 你待怎讲?

\spacept{钟元普}{20pt} 一命身亡了。

\spacept{俞伯牙}{20pt} 哎呀!(\textless{}\!{\bfseries\akai 崩登仓冲头}\!\textgreater{})

(俞伯牙昏介)

\spacept{钟元普}{20pt} 这是何人?

童儿 这就是俞大人。

\spacept{钟元普}{20pt} 哦哦(或:哎呀),大人醒来。

\spacept{俞伯牙}{20pt} 【二黄导板】听说是钟贤弟一命丧了,

\spacept{俞伯牙}{20pt} \textless{}\!{\bfseries\akai 三叫头}\!\textgreater{}贤弟! 子期!哎贤弟呀。

\spacept{俞伯牙}{20pt} 【二黄散板】此一番好一似马行断桥。他的父是尊长急忙拜倒,

\spacept{钟元普}{20pt} 【二黄散板】请大人莫折煞年迈山樵。

(\spacept{俞伯牙}{20pt} 老伯。)

\spacept{俞伯牙}{20pt} 【二黄散板】我就是俞伯牙伯父知晓,贤弟死留何言细说根苗。

\spacept{钟元普}{20pt} 【二黄散板】我的儿临危时也曾言道:葬埋在马鞍山候驾来瞧。

\spacept{俞伯牙}{20pt} 【二黄散板】烦伯父你与我坟台引道,

(\textless{}\!{\bfseries\akai 扭丝}\!\textgreater{},钟元普、俞伯牙同走圆场)

\spacept{钟元普}{20pt} 【二黄散板】这就是新坟土尚挂纸标。

(\spacept{俞伯牙}{20pt} 哎呀!)

\spacept{俞伯牙}{20pt} 【二黄散板】见坟台不由我双膝跪倒,呼不应、唤不醒生死故交。

\spacept{俞伯牙}{20pt} 贤弟呀\ldots{}\ldots{}(哭介)

\spacept{俞伯牙}{20pt} (啊,)老伯,那旁有一石台,老伯稍坐一时,待侄儿一祭。

(俞伯牙哭介)

\spacept{钟元普}{20pt} 有劳大人。

\spacept{俞伯牙}{20pt} 童儿。

(童儿 有。)

\spacept{俞伯牙}{20pt} 将我瑶琴摆在坟前。

(此处上渔、樵)

\spacept{俞伯牙}{20pt} 唉!({\akai 念})此来空枉费,人琴付东流\protect\hyperlink{fn34}{\textsuperscript{34}}。灵魂渺茫去呀,可叹一土丘。

\spacept{俞伯牙}{20pt} \textless{}\!{\bfseries\akai 帽子头}\!\textgreater{}【二黄慢板】想去岁中秋节论琴交好,今日里见坟台不见故交。来时喜去时悲愁云渺渺,又只见秋风起黄叶飘飘。为贤弟我不爱黄金荣耀,为贤弟我不爱玉带紫袍。为贤弟二双亲少行孝道,为贤弟辞王驾亲走这遭。为贤弟终日里梦魂颠倒,为贤弟千里迢迢,涉水登山,枉费徒劳。实指望与贤弟同饮香醪,实指望与贤弟共论琴操。实指望与贤弟朝夕欢笑,实指望与贤弟春游芳草,夏赏荷香,秋饮菊酒,冬藏梅阁,散淡逍遥。在坟台抚瑶琴以为祭吊,

(俞伯牙抚琴介)

(\spacept{俞伯牙}{20pt} 唉!)

\spacept{俞伯牙}{20pt} 【二黄散板】子期死少知音琴对谁调。我这里将瑶琴摔碎不要,

(俞伯牙摔琴介\textless{}\!{\bfseries\akai 乱锤}\!\textgreater{})

\spacept{钟元普}{20pt} 【二黄散板】问大人摔瑶琴所为哪条?

\spacept{俞伯牙}{20pt} 老伯,

\spacept{俞伯牙}{20pt} ({\akai 念})摔碎瑶琴凤尾寒,子期不在向谁弹?春风满面皆朋友,要会知音难上难。(俞伯牙哭介)

\spacept{俞伯牙}{20pt} 【二黄散板】问伯父贤弟死家有何靠,

\spacept{钟元普}{20pt} 【二黄散板】隐居在集贤村倒也逍遥(或:倒还逍遥)。

\spacept{俞伯牙}{20pt} 【二黄散板】这黄金与伯父甘旨\protect\hyperlink{fn35}{\textsuperscript{35}}养老,且待我迎接你替他代劳。

\spacept{俞伯牙}{20pt} 老伯,侄儿去后伯父不要思他。

\spacept{钟元普}{20pt} 我不思他。

\spacept{俞伯牙}{20pt} 不要想他。

\spacept{钟元普}{20pt} 我也不想他。(或:呃,我不想他。)

\spacept{俞伯牙}{20pt} 子期是我。

\spacept{钟元普}{20pt} (呃,)不敢。

\spacept{俞伯牙}{20pt} 我是子期。

\spacept{钟元普}{20pt} 实实不敢(或:唉,越发地不敢呐)。

\spacept{俞伯牙}{20pt} 小侄告辞了。

\spacept{俞伯牙}{20pt} 【二黄散板】辞伯父别坟墓扬长就道,

\spacept{钟元普}{20pt} 【二黄散板】虽异姓似手足犹如同胞。

\spacept{俞伯牙}{20pt} 【二黄散板】伯牙在\ldots{}\ldots{}

\spacept{钟元普}{20pt} 【二黄散板】子期死(啊)\ldots{}\ldots{}

\spacept{俞伯牙}{20pt} (接唱)【{\akai 二黄散板】知音缺少,摔瑶琴谢知音不负故交。

\spacept{俞伯牙}{20pt} \textless{}\!{\bfseries\akai 三叫头}\!\textgreater{}老伯,子期,唉贤弟呀。

(\spacept{钟元普}{20pt} \textless{}\!{\bfseries\akai 三叫头}\!\textgreater{}大人,我儿,唉儿呀。)

\spacept{俞伯牙}{20pt} 罢!

(俞伯牙下,小童同下)

(\spacept{钟元普}{20pt} 唉!)

\spacept{钟元普}{20pt} 【二黄散板】似这等金兰友如同管鲍,转身来见坟台不见儿曹。猛抬头见红日西山落了,回家去与老妻细说根苗。

\spacept{钟元普}{20pt} 儿呀!

(小锣打下\textless{}\!{\bfseries\akai 尾声}\!\textgreater{})
}

{\bfseries\akai 本戏人物扮相}:

\spacept{俞伯牙}{20pt} 纱帽,蓝帔,黑三,高方巾,宝蓝褶子,绦子。

\spacept{钟元普}{20pt} 白氈帽,白满,白老斗衣,腰包,鞋。

童儿 抓髻,白花褶子,鞋。

{\bfseries\akai 道具}:

小篮一只,内装纸钱。

石台,即倒椅两把。

{\bfseries\akai 附}:

渔、樵词 (念法很多,此为其中的一种)

两个小花脸,一老一少扮相如渔、樵。

(俞伯牙念``将摇琴摆在坟前\ldots{}\ldots{}'',渔、樵内``啊哈''\textless{}\!{\bfseries\akai 小锣五击}\!\textgreater{}上)

渔 ({\akai 念})渔翁夜傍西岩宿,

樵 ({\akai 念})更殚余力付樵苏。

渔 伙计,你说什么呐?

樵 我这念诗呐。

渔 你这长相还会念诗。

樵 就算我不会,那你干什么呐?

渔 我这可是念诗呐。

樵 许你念就不许我念。

渔 咱俩别吵,我是道听途说。

樵 我也是胡说八道。

渔 哎,你看这些人在这干什么呐?

樵 我看是上坟的。

渔 走累了,咱俩一边一个靠着树坐会儿。

樵 坐着坐着。

(俞伯牙弹完琴\ldots{}\ldots{})

渔 伙计,你听他们干什么呐?

樵 八成是弹棉花的。\\
渔 没事憩会儿好不好,跑这坟圈子里弹棉花干什么。

樵 吃饱了在这凉快凉快。

渔 别挨骂了。正是:({\akai 念})兰浦秋来烟雨深,

樵 ({\akai 念})几多情思在琴心。

渔 又拽上了。 别听弹棉花的了。

樵 回家睡大觉去喽,哈\ldots{}\ldots{}

(渔、樵同下)

{\bfseries\akai 注:}

\begin{enumerate}
\def\labelenumi{\arabic{enumi}.}
\item
  上渔、樵,俞、钟、童三人面向里。上渔、樵无非是不懂琴音,俞伯牙无知音,实际无此必要,故后来删掉。
\item
  《马鞍山》是乔玉林传下来的路子,传统、规矩,是学徒的唱法,中华戏曲专科学校的唱法略同与此,时慧宝的唱法与此有出入。此戏在过去是前三出的大路戏。
\end{enumerate}

         %
%%%%%%%%%%%%%%%%%%%%%%%%%%%%%%%%%
%-------------------The Figure Of The Paper------------------
%\begin{figure}[h!]
%\centering
%\includegraphics[height=3.35in,width=2.85in,viewport=0 0 400 475,clip]{PbTe_Band_SO.eps}
%\hspace{0.5in}
%\includegraphics[height=3.35in,width=2.85in,viewport=0 0 400 475,clip]{EuTe_Band_SO.eps}
%\caption{\small Band Structure of PbTe (a) and EuTe (b).}%(与文献\cite{EPJB33-47_2003}图1对比)
%\label{Pb:EuTe-Band_struct}
%\end{figure}

%-------------------The Equation Of The Paper-----------------
%\begin{equation}
%\varepsilon_1(\omega)=1+\frac2{\pi}\mathscr P\int_0^{+\infty}\frac{\omega'\varepsilon_2(\omega')}{\omega'^2-\omega^2}d\omega'
%\label{eq:magno-1}
%\end{equation}

%\begin{equation} 
%\begin{split}
%\varepsilon_2(\omega)&=\frac{e^2}{2\pi m^2\omega^2}\sum_{c,v}\int_{BZ}d{\vec k}\left|\vec e\cdot\vec M_{cv}(\vec k)\right|^2\delta [E_{cv}(\vec k)-\hbar\omega] \\
% &= \frac{e^2}{2\pi m^2\omega^2}\sum_{c,v}\int_{E_{cv}(\vec k=\hbar\omega)}\left|\vec e\cdot\vec M_{cv}(\vec k)\right|^2\dfrac{dS}{\nabla_{\vec k}E_{cv}(\vec k)}
% \end{split}
%\label{eq:magno-2}
%\end{equation}

%-------------------The Table Of The Paper----------------------
%\begin{table}[!h]
%\tabcolsep 0pt \vspace*{-12pt}
%%\caption{The representative $\vec k$ points contributing to $\sigma_2^{xy}$ of interband transition in EuTe around 2.5 eV.}
%\label{Table-EuTe_Sigma}
%\begin{minipage}{\textwidth}
%%\begin{center}
%\centering
%\def\temptablewidth{0.84\textwidth}
%\rule{\temptablewidth}{1pt}
%\begin{tabular*} {\temptablewidth}{|@{\extracolsep{\fill}}c|@{\extracolsep{\fill}}c|@{\extracolsep{\fill}}l|}

%-------------------------------------------------------------------------------------------------------------------------
%&Peak (eV)  & {$\vec k$}-point            &Band{$_v$} to Band{$_c$}  &Transition Orbital
%Components\footnote{波函数主要成分后的括号中,$5s$、$5p$和$5p$、$4f$、$5d$分别指碲和铕的原子轨道。} &Gap (eV)   \\ \hline
%-------------------------------------------------------------------------------------------------------------------------
%&2.35       &(0,0,0)         &33$\rightarrow$34    &$4f$(31.58)$5p$(38.69)$\rightarrow$$5p$      &2.142   \\% \cline{3-7}
%&       &(0,0,0)         &33$\rightarrow$34    &$4f$(31.58)$5p$(38.69)$\rightarrow$$5p$      &2.142   \\% \cline{3-7}
%-------------------------------------------------------------------------------------------------------------------------
%\end{tabular*}
%\rule{\temptablewidth}{1pt}
%\end{minipage}{\textwidth}
%\end{table}

%-------------------The Long Table Of The Paper--------------------
%\begin{small}
%%\begin{minipage}{\textwidth}
%%\begin{longtable}[l]{|c|c|cc|c|c|} %[c]指定长表格对齐方式
%\begin{longtable}[c]{|c|c|p{1.9cm}p{4.6cm}|c|c|}
%\caption{Assignment for the peaks of EuB$_6$}
%\label{tab:EuB6-1}\\ %\\长表格的caption中换行不可少
%\hline
%%
%--------------------------------------------------------------------------------------------------------------------------------
%\multicolumn{2}{|c|}{\bfseries$\sigma_1(\omega)$谱峰}&\multicolumn{4}{c|}{\bfseries部分重要能带间电子跃迁\footnotemark}\\ \hline
%\endfirsthead
%--------------------------------------------------------------------------------------------------------------------------------
%%
%\multicolumn{6}{r}{\it 续表}\\
%\hline
%--------------------------------------------------------------------------------------------------------------------------------
%标记 &峰位(eV) &\multicolumn{2}{c|}{有关电子跃迁} &gap(eV)  &\multicolumn{1}{c|}{经验指认} \\ \hline
%\endhead
%--------------------------------------------------------------------------------------------------------------------------------
%%
%\multicolumn{6}{r}{\it 续下页}\\
%\endfoot
%\hline
%--------------------------------------------------------------------------------------------------------------------------------
%%
%%\hlinewd{0.5$p$t}
%\endlastfoot
%--------------------------------------------------------------------------------------------------------------------------------
%%
%% Stuff from here to \endlastfoot goes at bottom of last page.
%%
%--------------------------------------------------------------------------------------------------------------------------------
%标记 &峰位(eV)\footnotetext{见正文说明。} &\multicolumn{2}{c|}{有关电子跃迁\footnotemark} &gap(eV) &\multicolumn{1}{c|}{经验指认\upcite{PRB46-12196_1992}}\\ \hline
%--------------------------------------------------------------------------------------------------------------------------------
%
%     &0.07 &\multicolumn{2}{c|}{电子群体激发$\uparrow$} &--- &电子群\\ \cline{2-5}
%\raisebox{2.3ex}[0pt]{$\omega_f$} &0.1 &\multicolumn{2}{c|}{电子群体激发$\downarrow$} &--- &体激发\\ \hline
%--------------------------------------------------------------------------------------------------------------------------------
%
%     &1.50 &\raisebox{-2ex}[0pt][0pt]{20-22(0,1,4)} &2$p$(10.4)4$f$(74.9)$\rightarrow$ &\raisebox{-2ex}[0pt][0pt]{1.47} &\\%\cline{3-5}
%     &1.50$^\ast$ & &2$p$(17.5)5$d_{\mathrm E}$(14.0)$\uparrow$ & &4$f$$\rightarrow$5$d_{\mathrm E}$\\ \cline{3-5}
%     \raisebox{2.3ex}[0pt][0pt]{$a$} &(1.0$^\dagger$) &\raisebox{-2ex}[0pt][0pt]{20-22(1,2,6)} &\raisebox{-2ex}[0pt][0pt]{4$f$(89.9)$\rightarrow$2$p$(18.7)5$d_{\mathrm E}$(13.9)$\uparrow$}\footnotetext{波函数主要成分后的括号中,2$s$、2$p$和5$p$、4$f$、5$d$、6$s$分别指硼和铕的原子轨道;5$d_{\mathrm E}$、5$d_{\mathrm T}$分别指铕的(5$d_{z^2}$,5$d_{x^2-y^2}$和5$d_{xy}$,5$d_{xz}$,5$d_{yz}$)轨道,5$d_{\mathrm{ET}}$(或5$d_{\mathrm{TE}}$)则指5个5$d$轨道成分都有,成分大的用脚标的第一个字母标示;2$ps$(或2$sp$)表示同时含有硼2$s$、2$p$轨道成分,成分大的用第一个字母标示。$\uparrow$和$\downarrow$分别标示$\alpha$和$\beta$自旋电子跃迁。} &\raisebox{-2ex}[0pt][0pt]{1.56} &激子跃迁。 \\%\cline{3-5}
%     &(1.3$^\dagger$) & & & &\\ \hline
%--------------------------------------------------------------------------------------------------------------------------------

%     & &\raisebox{-2ex}[0pt][0pt]{19-22(0,0,1)} &2$p$(37.6)5$d_{\mathrm T}$(4.5)4$f$(6.7)$\rightarrow$ & & \\\nopagebreak %\cline{3-5}
%     & & &2$p$(24.2)5$d_{\mathrm E}$(10.8)4$f$(5.1)$\uparrow$ &\raisebox{2ex}[0pt][0pt]{2.78} &a、b、c峰可能 \\ \cline{3-5}
%     & &\raisebox{-2ex}[0pt][0pt]{20-29(0,1,1)} &2$p$(35.7)5$d_{\mathrm T}$(4.8)4$f$(10.0)$\rightarrow$ & &包含有复杂的\\ \nopagebreak%\cline{3-5}
%     &2.90 & &2$p$(23.2)5$d_{\mathrm E}$(13.2)4$f$(3.8)$\uparrow$ &\raisebox{2ex}[0pt][0pt]{2.92} &强激子峰。$^{\ast\ast}$\\ \cline{3-5}
%$b$  &2.90$^\ast$ &\raisebox{-2ex}[0pt][0pt]{19-22(0,1,1)} &2$p$(33.9)4$f$(15.5)$\rightarrow$ & &B2$s$-2$p$的价带 \\ \nopagebreak%\cline{3-5}
%     &3.0 & &2$p$(23.2)5$d_{\mathrm E}$(13.2)4$f$(4.8)$\uparrow$ &\raisebox{2ex}[0pt][0pt]{2.94} &顶$\rightarrow$B2$s$-2$p$导\\ \cline{3-5}
%     & &12-15(0,1,2) &2$p$(39.3)$\rightarrow$2$p$(25.2)5$d_{\mathrm E}$(8.6)$\downarrow$ &2.83 &带底跃迁。\\ \cline{3-5}
%     & &14-15(1,1,1) &2$p$(42.5)$\rightarrow$2$p$(29.1)5$d_{\mathrm E}$(7.0)$\downarrow$ &2.96 & \\\cline{3-5}
%     & &13-15(0,1,1) &2$p$(40.4)$\rightarrow$2$p$(28.9)5$d_{\mathrm E}$(6.6)$\downarrow$ &2.98 & \\ \hline
%--------------------------------------------------------------------------------------------------------------------------------
%%\hline
%%\hlinewd{0.5$p$t}
%\end{longtable}
%%\end{minipage}{\textwidth}
%%\setlength{\unitlength}{1cm}
%%\begin{picture}(0.5,2.0)
%%  \put(-0.02,1.93){$^{1)}$}
%%  \put(-0.02,1.43){$^{2)}$}
%%\put(0.25,1.0){\parbox[h]{14.2cm}{\small{\\}}
%%\put(-0.25,2.3){\line(1,0){15}}
%%\end{picture}
%\end{small}

%------------------------------------直-接-插-入-文-件--------------------------------------------------------------------------------------
%\textcolor{red}{\textbf{直接插入文件}}:\verbatiminput{/home/jun_jiang/Documents/Latex_art_beamer/Daily_WORKS/Report-2020_model.tex} %为保险:~选用文件名绝对路径
%\textcolor{red}{\textbf{备忘录}}:\verbatiminput{/home/jun_jiang/Documents/备忘录.txt}
%---------------------------------------------------------------------------------------------------------------------------------------------%

%--------------------------------------------------------------------------The Biblography of The Paper-----------------------------------------------------------------%
%\newpage																				%

%--------------------------------------------------------------------------The Biblography of The Paper-----------------------------------------------------------------%
\newpage																				%
%-----------------------------------------------------------------------------------------------------------------------------------------------------------------------%
%\begin{thebibliography}{99}																		%
%%\bibitem{PRL58-65_1987}H.Feil, C. Haas, {\it Phys. Rev. Lett.} {\bf 58}, 65 (1987).											%
%	\bibitem{kp-method} \textrm{Zhenxi Pan, Yong Pan, Jun Jiang$^{\ast}$, Liutao Zhao}, \textrm{High-Throughput Electronic Band Structure Calculations for Hexaborides}, \textit{Intelligent Computing}, \textbf{Springer}, \textbf{P.386-395}, (2019).%
%	\bibitem{PAW-dataset} \textrm{姜骏},\textrm{PAW原子数据集的构造与检验}, \textit{中国化学会第十二届全国量子化学会议论文摘要集},\textbf{太原},(2014).
%\end{thebibliography}																			%
%-----------------------------------------------------------------------------------------------------------------------------------------------------------------------%
%\phantomsection\addcontentsline{toc}{section}{Bibliography} %直接调用\addcontentsline命令可能导致超链指向不准确,一般需要在之前调用一次\phantomsection命令加以修正%
\phantomsection\addcontentsline{toc}{section}{\CJKfamily{hei} 主要参考资料} %直接调用\addcontentsline命令可能导致超链指向不准确,一般需要在之前调用一次\phantomsection命令加以修正%
\thispagestyle{plain}
%\bibliography{../ref/Myref_from_2013}   %
\bibliography{/home/jun-jiang/Documents/Peking_Opera/Peking_Opera}   %
%\bibliography{/home/jun_jiang/Documents/Reports_Book/thuthesis-master/Peking_Opera/Peking_Opera}   %
%\bibliography{/home/jun-jiang/Documents/ref/Myref} %% 接近ieeert样式
\bibliographystyle{/home/jun-jiang/Documents/ref/mybib} %% 接近ieeert样式
%\bibliographystyle{/home/jun_jiang/Documents/Latex_art_beamer/ref/mybib} %% 接近ieeert样式
%\bibliographystyle{../ref/mybib} %% 接近ieeert样式
%%%%%%%%%%%%%%%%%%%%%%%%%%%%      \bibliographystyle         %%%%%%%%%%%%%%%%%%%%%%%%%%%%%%%%%%
%%%%%%      LaTeX 参考文献标准选项及其样式共有以下8种:                                %%%%%%%%
% plain,按字母的顺序排列,比较次序为作者、年度和标题.
% unsrt,样式同plain,只是按照引用的先后排序.
% alpha,用作者名首字母+年份后两位作标号,以字母顺序排序.
% abbrv,类似plain,将月份全拼改为缩写,更显紧凑.
% ieeetr,国际电气电子工程师协会期刊样式.
% acm,美国计算机学会期刊样式.
% siam,美国工业和应用数学学会期刊样式.
% apalike,美国心理学学会期刊样式.
%%%%%%%%%%%%%%%%%%%%%%%%%%%%%%%%%%%%%%%%%%%%%%%%%%%%%%%%%%%%%%%%%%%%%%%%%%%%%%%%%%%%%%%%%%%%%%%
%  \nocite{*}																				%
%-----------------------------------------------------------------------------------------------------------------------------------------------------------------------%


%-----------------------------------------------------------------------------------------------------------------------------------------------------------------------%
%-----------------------------------------------------------------------------------------------------------------------------------------------------------------------%

%-------------------------------------------------------------------------Thanks------------------------------------------------------------------------------------------------
%\newpage %%
%\newpage %%
%\thispagestyle{fancy}   % 首页插入页眉页脚 
%\section{致谢}

%\include{Data/chap-acknowledge}

%致谢内容
%-----------------------------------------------------------------------------------------------------------------------------------------------------------------------%

\clearpage     %\end{CJK} 前加上\clearpage是CJK的要求
%\end{CJK*}
\end{document}
