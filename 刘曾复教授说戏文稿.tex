\documentclass[10pt, oneside, a4paper]{article}      % Specifies the document class
%\documentclass[10pt, twoside, a4paper]{article}      % Specifies the document class

%%%%%%%%%%%%%%%%% CJK 中文版面控制  %%%%%%%%%%%%%%%%%%%%%%%%%%%%%%
%\usepackage{CJK} % CTEX-CJK 中文支持                            %
\usepackage{xeCJK} % seperate the english and chinese		 %
\usepackage{CJKutf8} % Texlive 中文支持                         %
\usepackage{CJKnumb} %中文序号                                   %
\usepackage{indentfirst} % 中文段落首行缩进                      %
%\setlength\parindent{22pt}       % 段落起始缩进量               %
\renewcommand{\baselinestretch}{1.2} % 中文行间距调整            %
\setlength{\textwidth}{16cm}                                     %
\setlength{\textheight}{24cm}                                    %
\setlength{\topmargin}{-1cm}                                     %
\setlength{\oddsidemargin}{0.1cm}                                %
\setlength{\evensidemargin}{\oddsidemargin}                      %
\usepackage{fancyhdr}           %使用页眉-页脚                   %
%%%%%%%%%%%%%%%%%%%%%%%%%%%%%%%%%%%%%%%%%%%%%%%%%%%%%%%%%%%%%%%%%%

\usepackage{authblk}					 %作者地址和E-mail
\usepackage{amsmath,amsthm,amsfonts,amssymb,bm}          %数学公式
\usepackage{mathrsfs}                                    %英文花体
\usepackage{tikz}					 %绘制平面图形
%\usepackage[dvipdfmx]{movie15_dvipdfmx} %插入视频
\usepackage{xcolor}                                        %使用默认允许使用颜色
%\usepackage{hyperref} 
\usepackage{graphicx}
\usepackage{subfigure}           %图片跨页
\usepackage{animate}		 %插入动画
\usepackage{caption}
\captionsetup{font=footnotesize}

%\usepackage[version=3]{mhchem}		%化学公式
\usepackage{chemformula}
\usepackage{chemfig}		%化学公式

\usepackage{fontspec} % use to set font
\setCJKmainfont{SimSun}
\XeTeXlinebreaklocale "zh"  % Auto linebreak for chinese
\XeTeXlinebreakskip = 0pt plus 1pt % Auto linebreak for chinese

\usepackage{longtable}                                   %使用长表格
\usepackage{multirow}
\usepackage{makecell}		%允许单元格内换行

\usepackage{arydshln}
\newcommand{\adots}{\mathinner{\mkern2mu%
\raisebox{0.1em}{.}\mkern2mu\raisebox{0.4em}{.}%
\mkern2mu\raisebox{0.7em}{.}\mkern1mu}}
%%%%%%%%%%%%%%%%%%%%%%%%%  参考文献引用 %%%%%%%%%%%%%%%%%%%%%%%%%%%
%%尽量使用 BibTeX(含有超链接,数据库的条目URL即可)                %
%%%%%%%%%%%%%%%%%%%%%%%%%%%%%%%%%%%%%%%%%%%%%%%%%%%%%%%%%%%%%%%%%%%

\usepackage[numbers,sort&compress]{natbib} %紧密排列             %
\usepackage[sectionbib]{chapterbib}        %每章节单独参考文献   %
%\usepackage{footbib}			   %脚注列出参考文献    %
\usepackage{hypernat}                                                                         %
%\usepackage[dvipdfm,bookmarksopen=true,pdfstartview=FitH,CJKbookmarks]{hyperref}              %
\usepackage[bookmarksopen=true,pdfstartview=FitH,CJKbookmarks]{hyperref}              %
\hypersetup{bookmarksnumbered,colorlinks,linkcolor=green,citecolor=blue,urlcolor=red}         %
%参考文献含有超链接引用时需要下列宏包,注意与natbib有冲突        %
%\usepackage[dvipdfm]{hyperref}                                  %
%\usepackage{hypernat}                                           %
\newcommand{\upcite}[1]{\hspace{0ex}\textsuperscript{\cite{#1}}} %
%%%%%%%%%%%%%%%%%%%%%%%%%%%%%%%%%%%%%%%%%%%%%%%%%%%%%%%%%%%%%%%%%%%%%%%%%%%%%%%%%%%%%%%%%%%%%%%
%\AtBeginDvi{\special{pdf:tounicode GBK-EUC-UCS2}} %CTEX用dvipdfmx的话,用该命令可以解决      %
%						   %pdf书签的中文乱码问题		      %
%%%%%%%%%%%%%%%%%%%%%%%%%%%%%%%%%%%%%%%%%%%%%%%%%%%%%%%%%%%%%%%%%%%%%%%%%%%%%%%%%%%%%%%%%%%%%%%

%---------------------------------xeCJK下设置中文字体-----------------------------------------%  
\setCJKfamilyfont{song}{SimSun}                             %宋体 song  
\newcommand{\song}{\CJKfamily{song}}                        % 宋体   (Windows自带simsun.ttf)  
\setCJKfamilyfont{xs}{NSimSun}                              %新宋体 xs  
\newcommand{\xs}{\CJKfamily{xs}}  
\setCJKfamilyfont{fs}{FangSong_GB2312}                      %仿宋2312 fs  
\newcommand{\fs}{\CJKfamily{fs}}                            %仿宋体 (Windows自带simfs.ttf)  
\setCJKfamilyfont{kai}{KaiTi_GB2312}                        %楷体2312  kai  
\newcommand{\kai}{\CJKfamily{kai}}                            
\setCJKfamilyfont{yh}{Microsoft YaHei}                    %微软雅黑 yh  
\newcommand{\yh}{\CJKfamily{yh}}  
\setCJKfamilyfont{hei}{SimHei}                                    %黑体  hei  
\newcommand{\hei}{\CJKfamily{hei}}                          % 黑体   (Windows自带simhei.ttf)  
\setCJKfamilyfont{msunicode}{Arial Unicode MS}            %Arial Unicode MS: msunicode  
\newcommand{\msunicode}{\CJKfamily{msunicode}}  
\setCJKfamilyfont{li}{LiSu}                                            %隶书  li  
\newcommand{\li}{\CJKfamily{li}}  
\setCJKfamilyfont{yy}{YouYuan}                             %幼圆  yy  
\newcommand{\yy}{\CJKfamily{yy}}  
\setCJKfamilyfont{xm}{MingLiU}                                        %细明体  xm  
\newcommand{\xm}{\CJKfamily{xm}}  
\setCJKfamilyfont{xxm}{PMingLiU}                             %新细明体  xxm  
\newcommand{\xxm}{\CJKfamily{xxm}}  
\setCJKfamilyfont{hwsong}{STSong}                            %华文宋体  hwsong  
\newcommand{\hwsong}{\CJKfamily{hwsong}}  
\setCJKfamilyfont{hwzs}{STZhongsong}                        %华文中宋  hwzs  
\newcommand{\hwzs}{\CJKfamily{hwzs}}  
\setCJKfamilyfont{hwfs}{STFangsong}                            %华文仿宋  hwfs  
\newcommand{\hwfs}{\CJKfamily{hwfs}}  
\setCJKfamilyfont{hwxh}{STXihei}                                %华文细黑  hwxh  
\newcommand{\hwxh}{\CJKfamily{hwxh}}  
\setCJKfamilyfont{hwl}{STLiti}                                        %华文隶书  hwl  
\newcommand{\hwl}{\CJKfamily{hwl}}  
\setCJKfamilyfont{hwxw}{STXinwei}                                %华文新魏  hwxw  
\newcommand{\hwxw}{\CJKfamily{hwxw}}  
\setCJKfamilyfont{hwk}{STKaiti}                                    %华文楷体  hwk  
\newcommand{\hwk}{\CJKfamily{hwk}}  
\setCJKfamilyfont{hwxk}{STXingkai}                            %华文行楷  hwxk  
\newcommand{\hwxk}{\CJKfamily{hwxk}}  
\setCJKfamilyfont{hwcy}{STCaiyun}                                 %华文彩云 hwcy  
\newcommand{\hwcy}{\CJKfamily{hwcy}}  
\setCJKfamilyfont{hwhp}{STHupo}                                 %华文琥珀   hwhp  
\newcommand{\hwhp}{\CJKfamily{hwhp}}  
\setCJKfamilyfont{fzsong}{Simsun (Founder Extended)}     %方正宋体超大字符集   fzsong  
\newcommand{\fzsong}{\CJKfamily{fzsong}}  
\setCJKfamilyfont{fzyao}{FZYaoTi}                                    %方正姚体  fzy  
\newcommand{\fzyao}{\CJKfamily{fzyao}}  
\setCJKfamilyfont{fzshu}{FZShuTi}                                    %方正舒体 fzshu  
\newcommand{\fzshu}{\CJKfamily{fzshu}}  
\setCJKfamilyfont{asong}{Adobe Song Std}                        %Adobe 宋体  asong  
\newcommand{\asong}{\CJKfamily{asong}}  
\setCJKfamilyfont{ahei}{Adobe Heiti Std}                            %Adobe 黑体  ahei  
\newcommand{\ahei}{\CJKfamily{ahei}}  
\setCJKfamilyfont{akai}{Adobe Kaiti Std}                            %Adobe 楷体  akai  
\newcommand{\akai}{\CJKfamily{akai}}  
%------------------------------设置字体大小------------------------%  
\newcommand{\chuhao}{\fontsize{42pt}{\baselineskip}\selectfont}     %初号  
\newcommand{\xiaochuhao}{\fontsize{36pt}{\baselineskip}\selectfont} %小初号  
\newcommand{\yihao}{\fontsize{28pt}{\baselineskip}\selectfont}      %一号  
\newcommand{\erhao}{\fontsize{21pt}{\baselineskip}\selectfont}      %二号  
\newcommand{\xiaoerhao}{\fontsize{18pt}{\baselineskip}\selectfont}  %小二号  
\newcommand{\sanhao}{\fontsize{15.75pt}{\baselineskip}\selectfont}  %三号  
\newcommand{\sihao}{\fontsize{14pt}{\baselineskip}\selectfont}%     四号  
\newcommand{\xiaosihao}{\fontsize{12pt}{\baselineskip}\selectfont}  %小四号  
\newcommand{\wuhao}{\fontsize{10.5pt}{\baselineskip}\selectfont}    %五号  
\newcommand{\xiaowuhao}{\fontsize{9pt}{\baselineskip}\selectfont}   %小五号  
\newcommand{\liuhao}{\fontsize{7.875pt}{\baselineskip}\selectfont}  %六号  
\newcommand{\qihao}{\fontsize{5.25pt}{\baselineskip}\selectfont}    %七号  

%%%%%%%%%%%%%%%%%%%%%  % 插图使用位置  %%%%%%%%%%%%%%%%%%%%%%%%%%%
\graphicspath{{Presentation_Beamer/}}                            %
%%%%%%%%%%%%%%%%%%%%%%%%%%%%%%%%%%%%%%%%%%%%%%%%%%%%%%%%%%%%%%%%%%

\usepackage{verbatim}			%Verbatim 宏包重新实现了 Verbatim 环境,并且提供一个命令可以导入一个 ASCII 文件到文档中
%\verbatiminput{filename}

%在beamer里面使用verbatim环境,可以通过在frame的参数里面添加 containsverbatim / fragile来解决,不过 containsverbatim 会导致pause失效
%\begin{frame}[containsverbatim] %也可以用 \begin{frame}[fragile]
%	\begin{verbatim}
%	\usepackage{xcolor}
%	TEST
%	\end{verbatim}
%\end{frame}

%%%%%%%%%%%%%%%%%%%  说戏剧本集特殊设置   %%%%%%%%%%%%%%%%%%%%%%
\setcounter{secnumdepth}{0} % 此处设置
\usepackage[center]{titlesec}
%%%%%%%%%%%%%%%%%%%%%%%%%%%%%%%%%%%%%%%%%%%%%%%%%%%%%%%%%%%%%%%%

%%%%%%%%%%%%%%%%%%%%%%%%%%%%% 用 authblk 包 支持作者和E-mail %%%%%%%%%%%%%%%%%%%%%%%%%%%%%%%%%
%\title{More than one Author with different Affiliations}				     %
\title{\hei\huge{刘曾复教授说戏剧本集文稿}}
%\author[a]{Author A}									     %
\author[]{}   %
%\author[a]{Author B}									     %
%\author[a]{Author C \thanks{Corresponding author: email@mail.com}}			     %
%\author[a]{Author/通讯作者 C \thanks{Corresponding author: cores-email@mail.com}}     %
%\author[b]{Author D}									     %
%\author[b]{Author/作者 D}									     %
%\author[b]{Author E}									     %
%\affil[a]{Department of Computer Science, \LaTeX\ University}				     %
%\affil[a]{作者单位-1 \authorcr 地址}    %\authorcr表示换行
%\affil[b]{Department of Mechanical Engineering, \LaTeX\ University}			     %
%\affil[b]{作者单位-2}			     %
											     %
%%% 使用 \thanks 定义通讯作者								     %
%%\affil命令后的{}中的内容,如果觉得需要换行的话,换行命令是\authorcr(不是\\)。
%%Email中可以吧相同邮箱的人@前面的内容写在一个{}里,用逗号隔开。注意{和}前面要加\。例如:
%%\affil[*]{单位1, \authorcr Email: \{zuozhe1, zuozhe2\}@yahoo.com, zuozhe3@sina.com}
											     %
\renewcommand*{\Authfont}{\small\rm} % 修改作者的字体与大小				     %
\renewcommand*{\Affilfont}{\small\it} % 修改机构名称的字体与大小			     %
\renewcommand\Authands{ and } % 去掉 and 前的逗号					     %
\renewcommand\Authands{ , } % 将 and 换成逗号					     %
\date{} % 去掉日期									     %
%\date{2020-12-30}									     %
%%%%%%%%%%%%%%%%%%%%%%%%%%%%%%%%%%%%%%%%%%%%%%%%%%%%%%%%%%%%%%%%%%%%%%%%%%%%%%%%%%%%%%%%%%%%%%
\begin{document}
%%%%%%%%%%%%%%%%%%%%%  % 页眉-页脚设计  %%%%%%%%%%%%%%%%%%%%%%%%%%%
%%\renewcommand{\headrulewidth}{3pt} %页眉(单)线宽(默认黑色),设为0可以去页眉线
%\makeatletter % 双线页眉
%\def\headrule{\color{blue}{\if@fancyplain\let\headrulewidth\plainheadrulewidth\fi%
%\hrule\@height 0.5pt \@width\headwidth\vskip1pt %上面线为0.5pt粗
%\hrule\@height 3.0pt\@width\headwidth  %下面3pt粗
%\vskip-2\headrulewidth\vskip-1pt}      %两条线的距离1pt
%  \vspace{6mm}}     %双线与下面正文之间的垂直间距
%\makeatother

%%\renewcommand{\footrulewidth}{3pt} %页脚线宽(默认黑色),设为0可以去页脚线
%\makeatletter % 双线页眉
%\def\footrule{{\color{blue}{\if@fancyplain\let\footrulewidth\plainfootrulewidth\fi%
%\hrule\@height 3.0pt \@width\headwidth}}
%  \vspace{2mm}}
%\makeatother

%\pagestyle{fancy}    %与文献引用超链接style有冲突
%\lhead{\bfseries Result} %页眉左边位置内容,并加粗 
%\chead{} % 页眉中间位置内容
%\rhead{\includegraphics[scale=0.20]{Figures/BCC_logo-1.png}}%在此处插入logo.pdf图片 图片靠右
%\lfoot{}  %页脚
%\rule{\temptablewidth}{1pt}
%\cfoot{}
%\rfoot{}
%\fancyfoot[C]{} %去掉页码
%%%%%%%%%%%%%%%%%  % pagestyleR常用格式  %%%%%%%%%%%%%%%%%%%%%%%%%
%% empty 无页眉页脚
%% plain 无页眉,页脚为居中页码
%% headings 页眉为章节标题,无页脚
%% myheadings 页眉内容可自定义,无页脚
%%%%%%%%%%%%%%%%%%%%%%%%%%%%%%%%%%%%%%%%%%%%%%%%%%%%%%%%%%%%%%%%%%

%\begin{CJK}{UTF8}{gbsn} %针对文字编码为unix %CJK自带的utf-8简体字体有gbsn(宋体)和gkai(楷体)
%\begin{CJK}{GBK}{hei}	%针对文字编码为doc
%\begin{CJK}{GBK}{hei}	 %针对文字编码为doc
%\CJKindent     %在CJK环境中,中文段落起始缩进2个中文字符
%\indent
%
\renewcommand{\abstractname}{\small{\CJKfamily{hei} 摘\quad 要}} %\CJKfamily{hei} 设置中文字体,字号用\big \small来设
\renewcommand{\refname}{\centering\CJKfamily{hei} 主~要~参~考~资~料}
%\renewcommand{\figurename}{\CJKfamily{hei} 图.}
\renewcommand{\figurename}{{\bf Fig}.}
%\renewcommand{\tablename}{\CJKfamily{hei} 表.}
\renewcommand{\tablename}{{\bf Tab}.}
%\renewcommand{\thesubfigure}{\roman{subfigure}}  \makeatletter %子图标记罗马字母
%\renewcommand{\thesubfigure}{\tiny(\alph{subfigure})}  \makeatletter %子图标记英文字母
%\renewcommand{\thesubfigure}{}  \makeatletter %子图无标记

%将图表的Caption写成 图(表) Num. 格式
\makeatletter
\long\def\@makecaption#1#2{%
  \vskip\abovecaptionskip
  \sbox\@tempboxa{#1. #2}%
  \ifdim \wd\@tempboxa >\hsize
    #1. #2\par
  \else
    \global \@minipagefalse
    \hb@xt@\hsize{\hfil\box\@tempboxa\hfil}%
  \fi
  \vskip\belowcaptionskip}
\makeatother

\newcommand{\keywords}[1]{{\hspace{0pt}\small{\CJKfamily{hei} 关键词:}{\hspace{2ex}{#1}}\bigskip}}

%%%%%%%%%%%%%%%%%%中文字体设置%%%%%%%%%%%%%%%%%%%%%%%%%%%
%默认字体 defalut fonts \TeX 是一种排版工具 \\		%
%{\bfseries 粗体 bold \TeX 是一种排版工具} \\		%
%{\CJKfamily{song}宋体 songti \TeX 是一种排版工具} \\	%
%{\CJKfamily{hei} 黑体 heiti \TeX 是一种排版工具} \\	%
%{\CJKfamily{kai} 楷书 kaishu \TeX 是一种排版工具} \\	%
%{\CJKfamily{fs} 仿宋 fangsong \TeX 是一种排版工具} \\	%
%%%%%%%%%%%%%%%%%%%%%%%%%%%%%%%%%%%%%%%%%%%%%%%%%%%%%%%%%

%\addcontentsline{toc}{section}{Bibliography}

%-------------------------------The Title of The Paper-----------------------------------------%
%\title{标题}
%----------------------------------------------------------------------------------------------%

%----------------------The Authors and the address of The Paper--------------------------------%
%\author{
%作者:
%\small
%Author1, Author2, Author3\footnote{Communication author's E-mail} \\    %Authors' Names	       %
%\small
%(The Address,City Post code)						%Address	       %
%}
%\affil[$\dagger$]{清华大学~材料加工研究所~A213}
%\affil{清华大学~材料加工研究所~A213}
%\date{}					%if necessary					       %
%----------------------------------------------------------------------------------------------%
%%%%%%%%%%%%%%%%%%%%%%%%%%%%%%%%%%%%%%%%%%%%%%%%%%%%%%%%%%%%%%%%%%%%%%%%%%%%%%%%%%%%%%%%%%%%%%%%%%%%%%%%%%%%%%%%%%%%%
\maketitle
%\thispagestyle{fancy}   % 首页插入页眉页脚 

%-------------------------------------------------------------------------------The Abstract and the keywords of The Paper----------------------------------------------------------------------------%
%\begin{abstract}
%The content of the abstract
%\end{abstract}

%\keywords{Keyword1; Keyword2; Keyword3}

\newpage
\setcounter{page}{0}
\pagenumbering{roman}
\hypertarget{ux8bf4-ux660e}{
\addcontentsline{toc}{subsection}{\hei 说~明}
	\subsubsection{\hei \large 说\hspace{35pt}明}\label{ux8bf4-ux660e}}
\pagestyle{fancy}    %与文献引用超链接style有冲突
\chead{说~明} % 页眉中间位置内容

	此为个人整理的刘曾复教授说戏录音的文本稿,\textbf{主要根据刘曾复先生为中国戏曲学院提供的百余出说戏录音为底本,并结合刘曾老在其他场合的说戏录音}\upcite{Liu-Shuoxi-Record}%\textsuperscript{{[}1{]}}
\textbf{整理完成的}。其中《太平桥》、《盗宗卷》、《梅龙镇》、《辕门斩子》、《摘缨会》、《上天台》、《一捧雪》、《卖马》、《南阳关》的``总讲本''主要依据《京剧新序》\upcite{Liu_Xinxu-I,Liu_Xinxu-II}%\textsuperscript{{[}2{]}.}
中收录的刘曾复先生整理的剧本并结合说戏录音整理完成;《马鞍山》、《战长沙》的``总讲本''则参考了李舒先生遗作《涉艺所得》\upcite{Li-SheyiSuode}%\textsuperscript{{[}3{]}.}
收录的刘曾复先生手书稿和传本并结合说戏录音整理完成的。\textbf{有关剧目中的把子,主要摘录自}《京剧新序》和《京剧老生把子见闻录》\upcite{XQYS1-32_1983}%\textsuperscript{{[}4{]}.}
一文记录的开打和舞台调度。

除了上述《太平桥》等十一出剧目,其余剧目的场次安排主要参考了《京剧汇编~(1-109集)》\upcite{Jingju-Huibian-1}%\textsuperscript{{[}5{]}.}
、《传统剧目汇编》\upcite{Jingju-Huibian-2}%\textsuperscript{{[}6{]}.}
、《京剧丛刊~(1-50集)》\upcite{Jingju-Congkan}%\textsuperscript{{[}7{]}.}
和``中国京剧戏考''网站\upcite{PekingOpera-Scripts}%\textsuperscript{{[}8{]}.}
上的相应的剧目的安排,个别剧目的词句也参考了,``中国京剧老唱片''网站\upcite{PekingOpera-OldRecords}%\textsuperscript{{[}9{]}.}
上载的老唱片戏词。

剧目按照剧中人物年代排列,部分剧目的年代排序参考了《京剧大戏考》\upcite{Chai-DaXikao}%\textsuperscript{{[}10{]}.}
和《京剧知识词典(增订版)》\upcite{PekingOpera-Dictionary}%\textsuperscript{{[}11{]}.}
中的剧目顺序。

\vskip 5pt
基于全面、客观、忠实的记录原则,整理剧目文字的标记说明如下:
\begin{enumerate}
\def\labelenumi{\arabic{enumi}.}
\item
	{\CJKfamily{hei}因为本人学识浅陋、加之录音带存年较久,因此文字中有不少存疑处。凡是存疑处,尽量用\textcolor{red}{红色字体}标出,}表明此处可能文辞欠通顺,或只是根据字音听写臆测的词句;
\item
	{\CJKfamily{hei}刘曾复先生腹笥渊博,在不同的场合说戏时,即使是同一出戏,个别词句也略有出入,文本中尽量作了标注:~}

\begin{enumerate}
\def\labelenumi{\arabic{enumi}.}
\item
  每个剧目中凡有出入的唱、念词句标注为:

\begin{quote}
	\underline{\textrm{XX}词1}~({\akai 或}:~\textrm{XX}词2;~\textrm{XX}词3;$\cdots${}$\cdots${})
\end{quote}
\begin{quote}
	\underline{\textrm{XX}句1}~({\akai 或}:~\textrm{XX}句2~{\akai 或}:~\textrm{XX}句3;$\cdots${}$\cdots${})
\end{quote}

\def\labelenumi{\arabic{enumi}.}
\setcounter{enumi}{1}
\item
  每个剧目中可不念或某些衬字的唱、念标注为:

\begin{quote}
	(\textrm{XX}词句)
\end{quote}
\end{enumerate}

\def\labelenumi{\arabic{enumi}.}
\setcounter{enumi}{2}
\item
  \textbf{除``总讲本''外,``单词本''中,与表演配合的其他人物唱、念(盖口)}标记为:
\begin{quote}
	(人物\hspace{30pt} 唱、念词句\textrm{XXX}。)
\end{quote}
\item
  \textbf{在本人的知识范围内,对一些生僻的典故、词汇作了简要的注解。}
\item
  \textbf{刘曾复先生对唱、念中的虚词非常重视,文本中的虚词标注有限,建议以先生的录音为准。}
\item
  \textbf{由于文字记录的功能有限,此书辑录的主要是说戏的文字内容,关于舞台表演过程中的唱、念关键都没有标注。}
\end{enumerate}


%-------------------------------------------------------------------------------The Content of The Paper----------------------------------------------------------------------------------------------%
\tableofcontents %% 制作目录(目录是根据标题自动生成的)
%-----------------------------------------------------------------------------------------------------------------------------------------------------------------------------------------------------%

\newpage	        % 每个新的/newpage 即可有新的\thispagestyle 引领      %
\thispagestyle{fancy}   % 插入页眉页脚                                        %
%----------------------------------------------------------------------------------------The Body Of The Paper----------------------------------------------------------------------------------------%
%Introduction

%\section{Introduction}
%导言
%\section{正文章节}
%参考文献的引用方式1\upcite{QCQC_2014}
%-------------------The Figure Of The Paper------------------
%\begin{figure}[h!]
%\centering
%\includegraphics[height=3.35in,width=2.85in,viewport=0 0 400 475,clip]{PbTe_Band_SO.eps}
%\hspace{0.5in}
%\includegraphics[height=3.35in,width=2.85in,viewport=0 0 400 475,clip]{EuTe_Band_SO.eps}
%\caption{\small Band Structure of PbTe (a) and EuTe (b).}%(与文献\cite{EPJB33-47_2003}图1对比)
%\label{Pb:EuTe-Band_struct}
%\end{figure}

%-------------------The Equation Of The Paper-----------------
%\begin{equation}
%\varepsilon_1(\omega)=1+\frac2{\pi}\mathscr P\int_0^{+\infty}\frac{\omega'\varepsilon_2(\omega')}{\omega'^2-\omega^2}d\omega'
%\label{eq:magno-1}
%\end{equation}

%\begin{equation} 
%\begin{split}
%\varepsilon_2(\omega)&=\frac{e^2}{2\pi m^2\omega^2}\sum_{c,v}\int_{BZ}d{\vec k}\left|\vec e\cdot\vec M_{cv}(\vec k)\right|^2\delta [E_{cv}(\vec k)-\hbar\omega] \\
% &= \frac{e^2}{2\pi m^2\omega^2}\sum_{c,v}\int_{E_{cv}(\vec k=\hbar\omega)}\left|\vec e\cdot\vec M_{cv}(\vec k)\right|^2\dfrac{dS}{\nabla_{\vec k}E_{cv}(\vec k)}
% \end{split}
%\label{eq:magno-2}
%\end{equation}

%-------------------The Table Of The Paper----------------------
%\begin{table}[!h]
%\tabcolsep 0pt \vspace*{-12pt}
%%\caption{The representative $\vec k$ points contributing to $\sigma_2^{xy}$ of interband transition in EuTe around 2.5 eV.}
%\label{Table-EuTe_Sigma}
%\begin{minipage}{\textwidth}
%%\begin{center}
%\centering
%\def\temptablewidth{0.84\textwidth}
%\rule{\temptablewidth}{1pt}
%\begin{tabular*} {\temptablewidth}{|@{\extracolsep{\fill}}c|@{\extracolsep{\fill}}c|@{\extracolsep{\fill}}l|}

%-------------------------------------------------------------------------------------------------------------------------
%&Peak (eV)  & {$\vec k$}-point            &Band{$_v$} to Band{$_c$}  &Transition Orbital
%Components\footnote{波函数主要成分后的括号中,$5s$、$5p$和$5p$、$4f$、$5d$分别指碲和铕的原子轨道。} &Gap (eV)   \\ \hline
%-------------------------------------------------------------------------------------------------------------------------
%&2.35       &(0,0,0)         &33$\rightarrow$34    &$4f$(31.58)$5p$(38.69)$\rightarrow$$5p$      &2.142   \\% \cline{3-7}
%&       &(0,0,0)         &33$\rightarrow$34    &$4f$(31.58)$5p$(38.69)$\rightarrow$$5p$      &2.142   \\% \cline{3-7}
%-------------------------------------------------------------------------------------------------------------------------
%\end{tabular*}
%\rule{\temptablewidth}{1pt}
%\end{minipage}{\textwidth}
%\end{table}

%-------------------The Long Table Of The Paper--------------------
%\begin{small}
%%\begin{minipage}{\textwidth}
%%\begin{longtable}[l]{|c|c|cc|c|c|} %[c]指定长表格对齐方式
%\begin{longtable}[c]{|c|c|p{1.9cm}p{4.6cm}|c|c|}
%\caption{Assignment for the peaks of EuB$_6$}
%\label{tab:EuB6-1}\\ %\\长表格的caption中换行不可少
%\hline
%%
%--------------------------------------------------------------------------------------------------------------------------------
%\multicolumn{2}{|c|}{\bfseries$\sigma_1(\omega)$谱峰}&\multicolumn{4}{c|}{\bfseries部分重要能带间电子跃迁\footnotemark}\\ \hline
%\endfirsthead
%--------------------------------------------------------------------------------------------------------------------------------
%%
%\multicolumn{6}{r}{\it 续表}\\
%\hline
%--------------------------------------------------------------------------------------------------------------------------------
%标记 &峰位(eV) &\multicolumn{2}{c|}{有关电子跃迁} &gap(eV)  &\multicolumn{1}{c|}{经验指认} \\ \hline
%\endhead
%--------------------------------------------------------------------------------------------------------------------------------
%%
%\multicolumn{6}{r}{\it 续下页}\\
%\endfoot
%\hline
%--------------------------------------------------------------------------------------------------------------------------------
%%
%%\hlinewd{0.5$p$t}
%\endlastfoot
%--------------------------------------------------------------------------------------------------------------------------------
%%
%% Stuff from here to \endlastfoot goes at bottom of last page.
%%
%--------------------------------------------------------------------------------------------------------------------------------
%标记 &峰位(eV)\footnotetext{见正文说明。} &\multicolumn{2}{c|}{有关电子跃迁\footnotemark} &gap(eV) &\multicolumn{1}{c|}{经验指认\upcite{PRB46-12196_1992}}\\ \hline
%--------------------------------------------------------------------------------------------------------------------------------
%
%     &0.07 &\multicolumn{2}{c|}{电子群体激发$\uparrow$} &--- &电子群\\ \cline{2-5}
%\raisebox{2.3ex}[0pt]{$\omega_f$} &0.1 &\multicolumn{2}{c|}{电子群体激发$\downarrow$} &--- &体激发\\ \hline
%--------------------------------------------------------------------------------------------------------------------------------
%
%     &1.50 &\raisebox{-2ex}[0pt][0pt]{20-22(0,1,4)} &2$p$(10.4)4$f$(74.9)$\rightarrow$ &\raisebox{-2ex}[0pt][0pt]{1.47} &\\%\cline{3-5}
%     &1.50$^\ast$ & &2$p$(17.5)5$d_{\mathrm E}$(14.0)$\uparrow$ & &4$f$$\rightarrow$5$d_{\mathrm E}$\\ \cline{3-5}
%     \raisebox{2.3ex}[0pt][0pt]{$a$} &(1.0$^\dagger$) &\raisebox{-2ex}[0pt][0pt]{20-22(1,2,6)} &\raisebox{-2ex}[0pt][0pt]{4$f$(89.9)$\rightarrow$2$p$(18.7)5$d_{\mathrm E}$(13.9)$\uparrow$}\footnotetext{波函数主要成分后的括号中,2$s$、2$p$和5$p$、4$f$、5$d$、6$s$分别指硼和铕的原子轨道;5$d_{\mathrm E}$、5$d_{\mathrm T}$分别指铕的(5$d_{z^2}$,5$d_{x^2-y^2}$和5$d_{xy}$,5$d_{xz}$,5$d_{yz}$)轨道,5$d_{\mathrm{ET}}$(或5$d_{\mathrm{TE}}$)则指5个5$d$轨道成分都有,成分大的用脚标的第一个字母标示;2$ps$(或2$sp$)表示同时含有硼2$s$、2$p$轨道成分,成分大的用第一个字母标示。$\uparrow$和$\downarrow$分别标示$\alpha$和$\beta$自旋电子跃迁。} &\raisebox{-2ex}[0pt][0pt]{1.56} &激子跃迁。 \\%\cline{3-5}
%     &(1.3$^\dagger$) & & & &\\ \hline
%--------------------------------------------------------------------------------------------------------------------------------

%     & &\raisebox{-2ex}[0pt][0pt]{19-22(0,0,1)} &2$p$(37.6)5$d_{\mathrm T}$(4.5)4$f$(6.7)$\rightarrow$ & & \\\nopagebreak %\cline{3-5}
%     & & &2$p$(24.2)5$d_{\mathrm E}$(10.8)4$f$(5.1)$\uparrow$ &\raisebox{2ex}[0pt][0pt]{2.78} &a、b、c峰可能 \\ \cline{3-5}
%     & &\raisebox{-2ex}[0pt][0pt]{20-29(0,1,1)} &2$p$(35.7)5$d_{\mathrm T}$(4.8)4$f$(10.0)$\rightarrow$ & &包含有复杂的\\ \nopagebreak%\cline{3-5}
%     &2.90 & &2$p$(23.2)5$d_{\mathrm E}$(13.2)4$f$(3.8)$\uparrow$ &\raisebox{2ex}[0pt][0pt]{2.92} &强激子峰。$^{\ast\ast}$\\ \cline{3-5}
%$b$  &2.90$^\ast$ &\raisebox{-2ex}[0pt][0pt]{19-22(0,1,1)} &2$p$(33.9)4$f$(15.5)$\rightarrow$ & &B2$s$-2$p$的价带 \\ \nopagebreak%\cline{3-5}
%     &3.0 & &2$p$(23.2)5$d_{\mathrm E}$(13.2)4$f$(4.8)$\uparrow$ &\raisebox{2ex}[0pt][0pt]{2.94} &顶$\rightarrow$B2$s$-2$p$导\\ \cline{3-5}
%     & &12-15(0,1,2) &2$p$(39.3)$\rightarrow$2$p$(25.2)5$d_{\mathrm E}$(8.6)$\downarrow$ &2.83 &带底跃迁。\\ \cline{3-5}
%     & &14-15(1,1,1) &2$p$(42.5)$\rightarrow$2$p$(29.1)5$d_{\mathrm E}$(7.0)$\downarrow$ &2.96 & \\\cline{3-5}
%     & &13-15(0,1,1) &2$p$(40.4)$\rightarrow$2$p$(28.9)5$d_{\mathrm E}$(6.6)$\downarrow$ &2.98 & \\ \hline
%--------------------------------------------------------------------------------------------------------------------------------
%%\hline
%%\hlinewd{0.5$p$t}
%\end{longtable}
%%\end{minipage}{\textwidth}
%%\setlength{\unitlength}{1cm}
%%\begin{picture}(0.5,2.0)
%%  \put(-0.02,1.93){$^{1)}$}
%%  \put(-0.02,1.43){$^{2)}$}
%%\put(0.25,1.0){\parbox[h]{14.2cm}{\small{\\}}
%%\put(-0.25,2.3){\line(1,0){15}}
%%\end{picture}
%\end{small}

%------------------------------------直-接-插-入-文-件--------------------------------------------------------------------------------------
%\textcolor{red}{\textbf{直接插入文件}}:\verbatiminput{/home/jun_jiang/Documents/Latex_art_beamer/Daily_WORKS/Report-2020_model.tex} %为保险:~选用文件名绝对路径
%\textcolor{red}{\textbf{备忘录}}:\verbatiminput{/home/jun_jiang/Documents/备忘录.txt}
%---------------------------------------------------------------------------------------------------------------------------------------------%

%--------------------------------------------------------------------------The Biblography of The Paper-----------------------------------------------------------------%
%\newpage																				%
%-----------------------------------------------------------------------------------------------------------------------------------------------------------------------%
%\begin{thebibliography}{99}																		%
%%\bibitem{PRL58-65_1987}H.Feil, C. Haas, {\it Phys. Rev. Lett.} {\bf 58}, 65 (1987).											%
%	\bibitem{kp-method} \textrm{Zhenxi Pan, Yong Pan, Jun Jiang$^{\ast}$, Liutao Zhao}, \textrm{High-Throughput Electronic Band Structure Calculations for Hexaborides}, \textit{Intelligent Computing}, \textbf{Springer}, \textbf{P.386-395}, (2019).%
%	\bibitem{PAW-dataset} \textrm{姜骏},\textrm{PAW原子数据集的构造与检验}, \textit{中国化学会第十二届全国量子化学会议论文摘要集},\textbf{太原},(2014).
%\end{thebibliography}																			%
%-----------------------------------------------------------------------------------------------------------------------------------------------------------------------%
\phantomsection\addcontentsline{toc}{section}{Bibliography} %直接调用\addcontentsline命令可能导致超链指向不准确,一般需要在之前调用一次\phantomsection命令加以修正%
%\bibliography{../ref/Myref_from_2013}   %
\bibliography{/home/jun-jiang/Documents/Peking_Opera/Peking_Opera}   %
%\bibliography{/home/jun-jiang/Documents/ref/Myref} %% 接近ieeert样式
\bibliographystyle{/home/jun-jiang/Documents/ref/mybib} %% 接近ieeert样式
%\bibliographystyle{../ref/mybib} %% 接近ieeert样式
%%%%%%%%%%%%%%%%%%%%%%%%%%%%      \bibliographystyle         %%%%%%%%%%%%%%%%%%%%%%%%%%%%%%%%%%
%%%%%%      LaTeX 参考文献标准选项及其样式共有以下8种:                                %%%%%%%%
% plain,按字母的顺序排列,比较次序为作者、年度和标题.
% unsrt,样式同plain,只是按照引用的先后排序.
% alpha,用作者名首字母+年份后两位作标号,以字母顺序排序.
% abbrv,类似plain,将月份全拼改为缩写,更显紧凑.
% ieeetr,国际电气电子工程师协会期刊样式.
% acm,美国计算机学会期刊样式.
% siam,美国工业和应用数学学会期刊样式.
% apalike,美国心理学学会期刊样式.
%%%%%%%%%%%%%%%%%%%%%%%%%%%%%%%%%%%%%%%%%%%%%%%%%%%%%%%%%%%%%%%%%%%%%%%%%%%%%%%%%%%%%%%%%%%%%%%
%  \nocite{*}																				%
%-----------------------------------------------------------------------------------------------------------------------------------------------------------------------%

%-------------------------------------------------------------------------Thanks------------------------------------------------------------------------------------------------
%\newpage %%
%\newpage %%
%\thispagestyle{fancy}   % 首页插入页眉页脚 
%\section{致谢}
\hypertarget{ux540e-ux8bb0-ux4e0e-ux81f4-ux8c22}{%
	\subsection{\hei{后记~与~致谢}}\label{ux540e-ux8bb0-ux4e0e-ux81f4-ux8c22}}

刘曾复教授(1914-2012)是我国前辈生理学家,毕生致力于生理学的教学与科研工作,怹对京剧也有深入、系统的研究。刘曾老学戏,早年师从王荣山先生,后又从王凤卿、刘砚芳、贯大元、钱宝森、侯喜瑞等诸家问艺。刘老不仅对京剧老生艺术有深厚的造诣,对京剧把子、脸谱亦无一不精,这在业余京剧爱好者中是极罕见的。

2004年夏,在樊百乐兄的引荐下,我有幸拜识刘曾复先生,此后的八年里,无论在学术研究还是戏曲欣赏方面,都曾得到刘老的热心提点,自感受益匪浅。现在由百乐兄整理的刘曾复先生的说戏剧本集即将面世,我由衷地感到高兴!因我也曾参与过其中一部分工作,华东师范大学钟锦副教授嘱我写一篇后记,我想借此机会,谈一下这本书的由来。

2009年9月,受吴小如先生(1922-2014)委托,我将吴先生珍藏的刘曾复先生为中国戏曲学院录制的百余出说戏磁带翻录整理成数字格式的音频文件,供吴先生脑梗后养病期间消遣。籍此机缘,我又将能找到的刘老在不同场合说戏录音汇集,编撰成``刘曾复教授说戏录音系列''光盘。记得当年11月初,我冒雪把制作的光盘送到刘曾老府上时,老先生非常高兴,跟我说:``没想到我还能见到这套录音。''

2011年6月间,我有幸拜读到钟锦先生根据这套录音整理的刘老说戏的唱词文稿(不含念白),钟老师告诉我,因当初的录音设备简陋,加之磁带放置的年代较久,有些录音听起来不很清晰,文稿中的一些词句记录不够准确。出于忠实保存前辈艺术的考虑,也为弥补我``点金成铁''的拙劣翻录技术,我萌生了将先生说戏录音整理成文的想法,可巧此间百乐兄受刘曾老之托,已着手启动这个``工程'',我因此参与了一部分工作。从2012年3月到2013年1月,经过我们的努力,终于完成了初稿。

在文稿整理过程中,我们注意到,刘曾老在不同的场合的说戏,即使是同一出戏,个别词句也略有出入。为完整纪录先生的说戏内容,对这类剧目,我们首先选定一个底本\footnote{《太平桥》、《盗宗卷》、《梅龙镇》、《辕门斩子》、《摘缨会》、《上天台》、《一捧雪》、《卖马》、《南阳关》的底本是《京剧新序》中收录的刘曾复先生亲自审订的``总讲本'';《马鞍山》、《战长沙》的底本主要参考李舒先生遗作《涉艺所得》收录的刘曾复先生手书稿和传本;《群英会》、\textbf{《柴桑口》、《平五路》、《七星灯》、《铁笼山·迷当发兵》、《美良川》、《三击掌》、《龙虎斗》、《审头刺汤》的底本则是自刘曾复先生保存的钞本复印件;除此之外的剧目,主要}以刘曾复先生为中国戏曲学院提供的说戏录音为底本。%\protect\hyperlink{fnref678}{↩}
}%\protect\hyperlink{fn678}{\textsuperscript{678}}
,在此基础上,整合刘老不同场合的说戏录音,完善成最后的剧本。凡唱、念词句与底本有出入的地方,尽可能作了标注。我们认为,这些不同处理的唱、念,也是刘老活用``三级韵''法则的示范,值得保留;希望文中的标注不至于影响读者阅读的顺畅。此外,在我们的知识范围内,对一些生僻的典故、词汇作了简要注解。而一些传统戏曲习惯用词,如``辅保''与``扶保''、``做甚''与``则甚''、``叫人''与``教人''等,整理时均未作统一。

书中有关剧目中的把子,主要摘录自《京剧新序》和《京剧老生把子见闻录》一文记录的开打和舞台调度。\textbf{除了《太平桥》等廿出剧目,其余剧目的场次安排则主要参考了《京剧汇编~(1-109集)》、《传统剧目汇编》、《京剧丛刊~(1-50集)》和``中国京剧戏考''网站上的相应的剧目的安排,个别剧目的词句也参考了``中国京剧老唱片''网站上载的老唱片戏词。剧目顺序基本按照剧中人物活动年代排列,个别剧目的年代排序参考了《京剧大戏考》}和《京剧知识词典(增订版)》\textbf{中的顺序。}

特别需要说明的是,各剧中唱腔、板式的标注,主要沿用钟锦老师提供的唱词文稿的纪录,此外也部分参考了相关剧本资料。另一方面,由于京剧发展史上``散板''和``摇板''一度互易,刘曾复先生的各类戏本中不少地方仍保留了这种痕迹,对此我们遵从原作,不作专门的统一。如果读者想要准确掌握相关剧目的唱腔、板式,建议以刘老的说戏录音为准。

本书初稿完成之际,承吴小如先生审阅了全部文稿,怹提出了非常详尽、细致的修改指导意见;刘曾老手录钞本中,《群英会》、《美良川》的复印件是刘老生前提供的,\textbf{《柴桑口》、《平五路》、《七星灯》、《铁笼山·迷当发兵》、《三击掌》、《龙虎斗》、《审头刺汤》}复印件则是刘老仙逝后由怹的女儿\textbf{祖敬阿姨在整理刘老遗物时提供的;北京的}段公平博士、\textbf{台湾中央大学}李元皓教授、上海的夏行涛先生分别为我们仔细校对了全部文稿,并修正了文稿的大量错误;\textbf{追随刘先生多年的}陈超老师、吴焕老师、娄悦老师、刘新阳老师、\textbf{何毅老师、美国芝加哥大学徐芃博士、北京市顺义区医院的樊剑医师、北京城市学院李楠博士以及北方工业大学钱盛博士、加拿大的网友``小豆子''老师、《健康时报》上海采访部的尹薇女士、北京大学的郝以鑫同学为我们提供和考订了很多准确、可靠的戏词;美国纽约梨园社青年团的马玢先生在此书出版后陆续为我们指正了诸多板式标注的舛误。诸位师长、同好的细致、无私的付出和帮助,为本书增色无数。}

\textbf{书稿完成过程中,我个人还曾得到台湾新竹交通大学邵锦昌教授、台湾大学王安祈教授以及首都医科大学李效义教授、上海广播电视台《绝版赏析》节目组柴俊为先生、复旦大学姜鹏博士和好友肖阳、刘鹏等的鼓励,也感谢我曾经的同事董栋博士、朱元慧博士。诸位师友的关心和肯定,给了我很大的动力。此外特别需要感谢爱妻高飞女士,她的宽容、理解和支持,为我能在业余时间参与有关工作创造了条件。}

\textbf{整理、出版此书的过程中,刘曾复先生和吴小如先生先后辞世,使我们更深刻地体会到腹笥渊博的前辈对于京剧的传承的重要性。保存、整理、抢救他们的艺术财富已是时不我待了。对于京剧我们都是外行,能做的也只是一点资料保存工作。本书所整理辑录的,只是我们所找到刘曾复先生的说戏资料,远非刘老所掌握剧目的全部。一些先生们生前提到的传统戏,像《清河桥》、《太行山》、《磐河战》、《葭萌关》、《凤鸣关》、《汜水关(锤换带)》等,因为没有找到相关的唱腔、唱词记录资料,只能付诸阙如;有些戏(如《柴桑口》、《平五路》、《美良川》、《取金陵》等)虽然刘老留下了钞本,但因为没有找到先生的说戏录音或说戏录音不完整,不少存疑处也无从请教。书稿虽已经同好多方指谬,但}书中存在的错误(特别是唱腔、板式的标注方面)肯定还有不少,诚恳期待京剧内行、对京剧研究有素的专家、学者和爱好京剧的读者不吝批评赐正(具体可发送电子邮件到
\href{mailto:czjiangjun@yeah.net}{{czjiangjun@yeah.net}})。

\textbf{樊百乐兄随侍刘曾老多年,尽得先生的真传。为本书的整理、出版,百乐兄耗费了大量的心血和精力,厥功至伟;本书能顺利出版,也得益于钟锦老师的默默付出。没有他们的努力,本书是不可能问世的。今年恰逢先生百周年诞辰,此书的出版想来也是对先生在天之灵很好的告慰。}

\textbf{时近甲午冬至,谨以本文表达对刘曾复先生和吴小如先生的深切缅怀!}

\begin{flushright}
\textbf{后学~ 姜骏~ 谨记~~~}

\textbf{2014-12-18~ 定稿~}

\textbf{2019-05-28~ 增补~}
\end{flushright}

%致谢内容
%-----------------------------------------------------------------------------------------------------------------------------------------------------------------------%

\clearpage     %\end{CJK} 前加上\clearpage是CJK的要求
%\end{CJK*}
\end{document}
