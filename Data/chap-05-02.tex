\newpage\hspace{30pt}~

{%

\subsubsection{\large\hei {战太平~{\small 之}~

花云}}

{\vspace{3pt}{\centerline{{[}{\hei 第一场}{]}}}\vspace{5pt}}

{({\akai 内})回府哇!}\hspace{20pt}~

{可恼哇!}\hspace{30pt}~

{夫人有所不知,今有反贼}陈友谅兴兵犯界,攻取太平城,千岁命我全身披挂,守住札子口,只是采石矶头缺一能将,故而烦恼。

有劳夫人。正是:~({\akai 念})青龙背上屯军马,

\setlength{\hangindent}{56pt}{【{\akai 二黄导板}】头戴着紫金盔齐眉盖鬓}\footnote{``齐眉盖鬓''俗作``齐眉盖顶'',据钱盛君告,后者是地方戏和鼓书词中常见``水词儿''。}{呐,}

\setlength{\hangindent}{56pt}{【{\akai 二黄散板}】为大将临阵时啊哪顾得残生呐。撩铠甲且把二堂进,}

\setlength{\hangindent}{56pt}{【{\akai 二黄散板}】有劳夫人点雄兵。}

\setlength{\hangindent}{56pt}{【{\akai 二黄散板}】接过夫人得胜饮呐,背转身来谢神灵。辞别夫人足踏镫,但愿此去扫荡烟尘({\akai 或}:~奏凯回程)。}

{\vspace{3pt}{\centerline{{[}{\hei 第二场}{]}}}\vspace{5pt}}

\setlength{\hangindent}{56pt}{【{\akai 二黄摇板}】一见贼子怒气发,不由老爷咬银牙。吾主洪福齐天大,把尔当作井底蛙。}

({花云与陈英(友)杰头场开打``大扫琉璃灯''}:~钻完烟筒花云大边,一扯两扯,半拉合到小边,回身盖下手的一刺,打下手蓬头,枪缓下来刺耳被勾走马腰封到小边,勾下手走马腰封,大转,下手到大边,花云在小边,里、外各一盖两盖,挑起来打下手腰封,往里漫下手头、下手低头左转身面朝里、背朝外,花云扎下手脖(即贴下手靠旗斗),下手小蹦子转身面朝外,右手推出花云枪,亮住。)\footnote{谭鑫培、李寿山设计。《战太平》花云与陈英杰(一般``英''字写成``友'',《英烈传》中是``英''字)的情况与《南阳关》伍云召、韩擒虎不同。头场开打是陈想攻下太平,花是想奋力杀退陈,等候救兵。这场开打也不能太多,太多显松,不能显示花云一鼓作气把陈击退,但又不能太少,少则不足以显示花云之勇。谭鑫培这一场采用过``大扫琉璃灯'',比``灯笼泡''火炽,但又干净利落,颇为得体。陈英杰由李寿山配演。花云得胜后是想回城防卫,所以不用龙套追过场,要一个小下场下。另一场是花云急于保护朱文逊杀出重围,与陈英杰无心恋战,因此不能多打,但打陈下后,紧接耍一个大下场,表示奋力突围。}

({花云与陈英(友)杰头场另一种开打``一百零八枪头子''}({没有幺二三的快枪}):~钻完烟筒,上手大边,一扯两扯,半拉合到小边,回身盖下手的一刺,打下手蓬头,缓下来到右边一刺接腰封,绕道左边盖下手一刺打腰封,再右边一刺接腰封,再左边一盖打腰封,缓下来,下手刺耳,勾下手走马腰封,上手到大边,下手到小边,往外漫下手头,一、二、三刺下手喉,下手跟着三刨,往里漫下手头,二人别,撤下来一盖下手的一刺、打腰封,一刺下手接腰封,再一盖下手一刺、打腰封,这三个一刺腰封是边盖边打、边刺边接、边盖边打、边打边走的,上手从大边到小边,下手到了大边,绕下来,别,下手刺上手马腿,上手挑起来往里刺肚,左转身刺下手马腿,接下手刺肚,原地一盖、两盖、扎脖,下手是递把转身,再一刺转身面向里,小蹦子转身面外,推上手扎脖的枪,二人亮住(所谓扎脖实际是枪贴靠旗斗)。)\footnote{余叔岩、钱金福设计。}

(\hspace{40pt}~

(花打陈下,面向外提枪花转身,面向外)三个提枪花,出枪,左手往左上方伸出平托枪(杆),右手往右掠枪杆往下绕过枪鐏反手扶鐏,跨左腿,踢右腿,两手不离枪,向右往里转身(/翻身),面向下场门,左手在前扶住枪杆,右手顺着鐏转过来、握住枪下端在右侧腰间,一绕枪头,平端枪,弓箭步(亮住),拧腰,亮住下。)\footnote{这是贯大元介绍的谭鑫培《战太平》头场花云开打后的下场。括号中隶体字标注的是《老生把子》一文与《京剧新序》中不同处。}

{\vspace{3pt}{\centerline{{[}{\hei 第三场}{]}}}\vspace{5pt}}

{花云!}\hspace{40pt}~

{来也!}\hspace{40pt}~

({花云保护朱文逊杀出重围小开打后枪下场:~}花打陈下,面向里矗枪,三个背花,转身面向外三个迎面花,扫左腿,扫右腿,在左边向后打右脚(右脚在左腿后反抬),枪顺过来单腿立背枪,向左上方斜着出左手,缓过枪来在脸前画三个大圈,同时左手也掏着画三圈,到第三圈左手向左斜上方伸出去,右手枪打左手,上膀子,跨左腿,右腿上步,右转身略偏小边面向外,在转身当中右手撤到胸前,左手绕枪鐏转过来面向外,右手平着出枪,左手绕到胸前按胡子弓箭步亮住,下场门下。)\footnote{杨小楼晚年耍完三个迎面花后减去扫腿等,只打一下,直接就在脸前画三圈。许多戏他都用这个下场,例如他演《下河东》时第二个下场就用它。}

{\vspace{3pt}{\centerline{{[}{\hei 第四场}{]}}}\vspace{5pt}}

{参见千岁!}\hspace{30pt}~

{千岁休得惊慌,随定为臣,杀出重围,去至金陵,搬兵取救({\akai 或}:~搬兵求救)。}

{唉呀,千岁呀,事到如今,你还顾得什么家眷呐!}

{诶------他为君的有家眷,我为臣的就无有家眷么?!}

{诶------我也要保护家眷去了!}

{\vspace{3pt}{\centerline{{[}{\hei 第五场}{]}}}\vspace{5pt}}

\setlength{\hangindent}{56pt}{【{\akai 西皮导板}】号炮一响啊惊天地呀,}

\setlength{\hangindent}{56pt}{【{\akai 西皮散板}】就是雀鸟也难飞。教花安与父带坐骑,舍不得妻儿两分离。}

{(【{\akai 西皮散板}】用手抱定娇儿体,我的儿啊,父子难免各东西。)}

\setlength{\hangindent}{56pt}{【{\akai 西皮散板}】夫人请上受一呃\textless{}\!{\bfseries\akai 哭头}\!\textgreater{}礼,夫人呐!}

\setlength{\hangindent}{56pt}{【{\akai 西皮散板}】下官言来听端的:~孙氏、娇儿托付你,这是我花家一脉系。}

\setlength{\hangindent}{56pt}{【{\akai 西皮散板}】狠心我把妻儿弃呀,落一个青史名标万古题。}

{\vspace{3pt}{\centerline{{[}{\hei 第六场}{]}}}\vspace{5pt}}

{千岁,你保得好家眷呐!}\hspace{10pt}~

{随定为臣({\akai 或}:~随臣马后),杀出重围!}

{\vspace{3pt}{\centerline{{[}{\hei 第七场}{]}}}\vspace{5pt}}

\setlength{\hangindent}{56pt}{【{\akai 西皮导板}】叹英雄失智呃入罗网,}

\setlength{\hangindent}{56pt}{【{\akai 西皮原板}】大将呃难免阵头亡。我主爷洪福齐呃天降,刘伯温八卦也平常呃。早知道({\akai 或}:~既知晓)采石矶被贼抢呃,早就该差能将前来提防。将身儿来在大街呀上,}\footnote{刘曾复先生为林瑞平、杨甲戌二位先生说戏录音接近刘先生为吴小如先生示范的王凤卿唱法:~ ``【{\akai 西皮原板}】{早就该差能将}【{\footnotesize 转}{\akai 西皮二六}】{前来提防}。【{\akai 西皮摇板}】将身儿来在大街呀上,''。}

\setlength{\hangindent}{56pt}{【{\akai 西皮摇板}】那旁来了疯婆娘。}

\setlength{\hangindent}{56pt}{【{\akai 西皮散板}】这一足哇踏在你地埃尘呐({\akai 或}:~踏你在地埃尘呐)。你是谁家疯魔女啊,}

\setlength{\hangindent}{56pt}{【{\akai 西皮快板}】怀中抱定小娇生。明明认得孙侍女}\footnote{刘曾复先生专门介绍,在《英烈传》小说中是``孙侍女'',后来戏台上变成了``二夫人''。}{,假装疯魔见夫君({\akai 或}:~见主人。)。你若念在夫妻义,去至金陵搬救兵。你若不念夫妻义,千万莫丢小娇生。({\akai 或}:~你若念在主仆义,去至金陵搬救兵。拜求圣上发人马,点动我国龙虎军。)使个眼色快逃走。}

\setlength{\hangindent}{56pt}{【{\akai 西皮散板}】大街上去了孙侍女,我的妻呀!夫妻们相逢万不能。({\akai 或}:~大街上去了孙侍女,父子们见面万不能。)}\footnote{陈超老师注:~这一场,括号内的词句也是他跟刘曾复先生学的原词。是三庆班的名演员冯瑞祥改的程长庚唱词,冯瑞祥与孙春恒是三庆班的``一文一武''。谭鑫培《战太平》的唱念和身段都学冯瑞祥。}

{\vspace{3pt}{\centerline{{[}{\hei 第八场}{]}}}\vspace{5pt}}

\setlength{\hangindent}{56pt}{【{\akai 西皮摇板}】虎落平川怎脱逃。}

{啊?!}\hspace{40pt}~

{自然是怒骂}\footnote{段公平君建议``怒骂''均作``辱骂''。}{的是啊。}

{诶,怒骂的是。}\hspace{20pt}~

{千岁!}\hspace{40pt}~

\setlength{\hangindent}{56pt}{【{\akai 西皮快板}】千岁爷休说懦弱话,非是为臣把君压。进帐去把贼骂,拚着一命染黄沙。纵然将你我头割下,落一个骂贼的名儿扬天涯。}

{怒骂的是!}\hspace{30pt}~

{诶------懦弱无刚啊!}

\setlength{\hangindent}{56pt}{【{\akai 西皮摇板}】站的是啊你老爷将花云呐。}

{唉呀千岁呀$\cdots{}\cdots{}$}

\setlength{\hangindent}{56pt}{【{\akai 西皮散板}】哗喇喇大炮一声响,血淋淋的人头滚一旁。先前怎样对你讲,一心降顺北汉王。那贼焉有容人量,顷刻之间一命亡啊。}

\setlength{\hangindent}{56pt}{【{\akai 西皮散板}】我把人头打进帐}\footnote{夏行涛君建议作``搭进帐''。}{呃,}

\setlength{\hangindent}{56pt}{【{\akai 西皮快板}】开言大骂北汉王:~既是兴兵来打仗,一来一往动刀枪。暗施诡计非为上}\footnote{据李楠君告知,余派原词作``暗地设计把太平抢'',他从刘曾复先生学的是``暗地设计非为上''。}{,你是人面兽心肠。}

{\textless{}{\!\bfseries\akai 叫头}\!\textgreater{}陈友谅!}

{呀------}\hspace{20pt}~

\setlength{\hangindent}{56pt}{【{\akai 西皮快板}】陈友谅他把好言讲,背转身来自思量:~我若是降了贼友谅,落得骂名天下扬。我若是不降贼友谅,顷刻之间一命亡。罢罢罢,屈膝呀我跪宝帐,}

{呃!}\hspace{40pt}~

\setlength{\hangindent}{56pt}{【{\akai 西皮摇板}】你老爷愿死我不愿降。}

{\vspace{3pt}{\centerline{{[}{\hei 第九场}{]}}}\vspace{5pt}}

\setlength{\hangindent}{56pt}{【{\akai 西皮导板}】盖世英雄遭毒手哇,}

\setlength{\hangindent}{56pt}{【{\akai 西皮快板}】好一似鳌鱼吞了钩。采石矶头入虎口,汗马功劳一笔勾。将身来在法标口,}

\setlength{\hangindent}{56pt}{【{\akai 西皮摇板}】为国忠良的下场头。}

\setlength{\hangindent}{56pt}{【{\akai 西皮快板}】大吼一声冲牛斗,大骂奸贼听从头:~要我归降不能够,岂与奸贼做马牛。}

\setlength{\hangindent}{56pt}{【{\akai 西皮快板}】怒气填胸({\akai 或}:~开言怒发)三千丈,太阳头上冒火光。要我归降休妄想,}

{贼呀,贼!}\hspace{30pt}~

\setlength{\hangindent}{56pt}{【{\akai 西皮快板}】除非红日出西方。}

{\vspace{3pt}{\centerline{{[}{\hei 第十场}{]}}}\vspace{5pt}}

{杀败了哇,杀败了!}

{不想------误入罗网,身带箭伤,我命休矣!}

{\textless{}{\!\bfseries\akai 叫头}\!\textgreater{}圣上啊,吾主!}

{臣力已竭,不能保主江山社稷了。}

{也罢!}\hspace{40pt}~

{我不免拜谢我主爵禄之恩,自刎疆场!免受贼辱!}

{罢!}\hspace{40pt}~

{*注:~王凤卿的《战太平》一剧唱法与常见的谭派唱法异趣,刘曾老曾为吴小如先生留下示范录音,有关唱词兹录如下:~}

\setlength{\hangindent}{56pt}{【{\akai 二黄导板}】受君禄当报效臣把忠尽,}

\setlength{\hangindent}{56pt}{【{\akai 二黄散板}】扫烟尘开疆土呃定主乾坤呐。古至今多少人不惜性命,君依礼臣尽忠哪顾宗亲。}

\setlength{\hangindent}{56pt}{【{\akai 二黄散板}】接过了得胜酒心中思忖,转身来祭虚空过往神灵。但愿得扫狼烟苍生之幸,方显得男儿汉盖国俊英。}

\setlength{\hangindent}{56pt}{【{\akai 二黄散板}】两军对垒在沙场,气得老爷怒满膛。无知匹夫敢犯上,狐群狗党一扫光。}

\setlength{\hangindent}{56pt}{【{\akai 西皮导板}】千岁说话真非理呀,}

\setlength{\hangindent}{56pt}{【{\akai 西皮散板}】把我当作小儿欺。}

\setlength{\hangindent}{56pt}{【{\akai 西皮散板}】花安与父带坐骑,二位夫人泪悲啼。用手抱定娇儿\textless{}\!{\bfseries\akai 哭头}\!\textgreater{}体,我的儿啊!}

\setlength{\hangindent}{56pt}{【{\akai 西皮散板}】父子难免各东西。}

\setlength{\hangindent}{56pt}{【{\akai 西皮散板}】夫人请上受一\textless{}\!{\bfseries\akai 哭头}\!\textgreater{}礼,夫人呐!}

\setlength{\hangindent}{56pt}{【{\akai 西皮散板}】这是花门一脉系。孙氏年轻全仗你,大小事体相扶持。狠心我把妻儿弃呀,}

\setlength{\hangindent}{56pt}{【{\akai 西皮散板}】落一个青史名标万古题。}

\setlength{\hangindent}{56pt}{【{\akai 西皮导板}】叹英雄失智落陷阱,}

\setlength{\hangindent}{56pt}{【{\akai 西皮原板}】一腔怒气贯长虹}\footnote{陈超老师注:~此句原是``止不住珠泪洒衣襟。''刘曾复先生认为``这句词太乏,我有时候就用《打登州》代替,马马虎虎''。王凤卿的传承的这套《战太平》唱词是程长庚的词句。给吴先生的录音中孙侍女身份已改成二夫人,这属于王凤卿的``过渡时期''的词句,王到后来索性就唱谭派词了。}{。刘伯温枉自挂帅印,用兵不到欠思忖呐。既知晓采石矶有伤损,为何不发【{\footnotesize 转}{\akai 西皮二六}】救兵临。}

\setlength{\hangindent}{56pt}{【{\akai 西皮摇板}】迈虎步来在大街境,}

\setlength{\hangindent}{56pt}{【{\akai 西皮摇板}】那旁来了疯魔人。}

\setlength{\hangindent}{56pt}{【{\akai 西皮散板}】这一足踏你在街心。你是谁家疯魔女呀,}

{(孙氏\hspace{40pt}~

哈哈,哈哈,啊$\cdots{}\cdots{}$({\hwfs 笑}{\hwfs 介}))}

\setlength{\hangindent}{56pt}{【{\akai 西皮快板}】怀中抱定小娇生。我若上前将妻认,泄露机关命难生。你若真心救夫命,去至金陵搬救兵。使个眼色快逃奔呐。}

{(孙氏\hspace{40pt}~

哈哈,哈哈,啊$\cdots{}\cdots{}$({\hwfs 笑}{\hwfs 介}))}

\setlength{\hangindent}{56pt}{【{\akai 西皮散板}】大街上去了孙侍\textless{}\!{\bfseries\akai 哭头}\!\textgreater{}女婢,我的妻呀!}

\setlength{\hangindent}{56pt}{【{\akai 西皮散板}】夫妻们重逢待来生。}
