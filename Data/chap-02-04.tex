\newpage
\subsubsection{\large\hei {汉津口~{\small 之}~关公}}
\addcontentsline{toc}{subsection}{\hei 汉津口~\small{之}~关公}

\hangafter=1                   %2. 设置从第1⾏之后开始悬挂缩进  %
\setlength{\parindent}{0pt}{

{\centerline{{[}{\hei 第一场}{]}}}\vspace{5pt}

{[}{\akai 引子}{]}威震乾坤,扶汉室,一点丹忱。

({\akai 念})忠义一腔贯古今,补天扶日志平生。英雄几见称夫子,豪杰如斯乃圣人。\footnote{``志平生''夏行涛{\scriptsize 君}建议作``助平升'',此处从《京剧汇编》第八十五集~马连良~藏本;~录音中刘曾复先生念``天子''应作``夫子'',据夏行涛{\scriptsize 君}告,``英雄几见称夫子,豪杰如斯乃圣人''是清代中叶理学家夏力恕为湖北孝感关帝庙作的对联。}

某,汉寿亭侯关------。可恨曹操,诓哄孺子,刘琮献了荆襄,反遭其害。刘皇叔弃了新野,欲取荆州。曹兵百万,追赶甚急。因此诸葛军师,命某前来江夏,向大公子刘琦,搬兵取救。怎奈他连日染病未痊,不能发兵。某今在此,心悬两地,好不焦虑人也!

\setlength{\hangindent}{56pt}{【{\akai 西皮原板}】想国家气运衰令人悲悼,叹不尽创业难英雄功高。刘皇叔帝室后欲将国保,时不至空使人忧虑焦劳。 }

大公子。

有座。

公子贵恙既已痊愈,克即发兵,与某前去接应,恐刘皇叔悬念之至。

即刻点将发兵。

哦,军师为何来此?快快有请。

哦,军师到了。

得令!

\setlength{\hangindent}{56pt}{【{\akai 西皮二六}】某正在心悬急军师驾到,好一似风云会波浪腾蛟。府堂上领雄兵谕令军校, }

\setlength{\hangindent}{56pt}{【{\akai 西皮摇板}】斩曹操准备某偃月钢刀。 }

\vspace{3pt}{\centerline{{[}{\hei 第二场}{]}}}\vspace{5pt}

\setlength{\hangindent}{56pt}{(刘备\hspace{30pt}【{\akai 西皮散板}】败当阳过长桥夏口而奔,猛回头又只见襄阳古城。实可怜数万的无辜百姓,懦弱子失荆州苦及黎民。) }

\setlength{\hangindent}{56pt}{(刘备\hspace{30pt}【{\akai 西皮原板}】自桃园结义起同扶汉鼎,我三人投公孙屡建奇勋。在安喜鞭督邮弃了信印,仗大义救孔融陶谦让城。收吕布却反被吕布兼并,饮曹操青梅酒饱受【{\footnotesize 转}{\akai 西皮二六}】虚惊。失徐州投河北袁绍不信,兄弟们遭失散相会古城。好容易得新野暂时安稳,)

\setlength{\hangindent}{56pt}{(刘备\hspace{30pt}【{\akai 西皮散板}】到如今依然是颠沛飘零。) }

\setlength{\hangindent}{56pt}{(刘备\hspace{30pt}【{\akai 西皮摇板}】说什么年半百儿是根本,说什么汉宗室儿是皇孙。为蠢子叹糜氏自投枯井,为蠢子险伤我股肱之臣。今日里势已败要儿做甚?) }

\setlength{\hangindent}{56pt}{(刘备\hspace{30pt}【{\akai 西皮散板}】我岂肯学袁氏溺爱不明?) }

\vspace{3pt}{\centerline{{[}{\hei 第三场}{]}}}\vspace{5pt}

\setlength{\hangindent}{56pt}{【{\akai 西皮导板}】青龙偃月\footnote{据樊百乐{\scriptsize 君}告,刘曾复先生曾言,前辈艺人沿袭``尊崇关帝''的旧俗,台上往往将``青龙刀''或``青龙偃月''的``龙''念``铜(tóng)''音,也可视为一种``避讳''。刘先生学戏时``青龙刀''念法即此路数。}威风凛, }

\setlength{\hangindent}{56pt}{【{\akai 西皮快板}】赤兔胭脂起风云。桃园弟兄忠心耿,誓挽汉室天日倾\footnote{刘曾复先生录音中似作``天日行'',参考《京剧汇编》第八十五集~马连良~藏本,此处作``实望汉室天日倾'',故``誓挽汉室天日倾''文意更确。}。英雄此时当效命, }

\setlength{\hangindent}{56pt}{【{\akai 西皮散板}】除奸扶汉镇乾坤。}

\vspace{3pt}{\centerline{{[}{\hei 第四场}{]}}}\vspace{5pt}

曹操休得逞强,关老爷来也!

}
