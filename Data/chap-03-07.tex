\newpage
\phantomsection %实现目录的正确跳转
\section*{\large\hei {沙桥饯别}}
\addcontentsline{toc}{section}{\hei 沙桥饯别}

\hangafter=1                   %2. 设置从第1⾏之后开始悬挂缩进  %}}}
\setlength{\parindent}{0pt}{

\vspace{3pt}{\centerline{{[}{\hei 第一场}{]}}}\vspace{5pt}

\setlength{\hangindent}{52pt}{({\hwfs 唐}玄奘{\hwfs 上})\hspace{10pt}}

\setlength{\hangindent}{52pt}{玄奘\hspace{30pt}{[}{\akai 引子}{]}一年气象一年新,抛却红尘念佛经。}

\setlength{\hangindent}{52pt}{玄奘\hspace{30pt}({\akai 念})正在佛前打坐,回头观见五岳。一班俱是神像,为何欺善怕恶。}

\setlength{\hangindent}{52pt}{玄奘\hspace{30pt}贫僧玄奘。只因唐王天子为游地府许下大愿,要往西天拜佛,取经回朝,设立坛台,超度众魂。是我情愿替主一往。今乃黄道吉日,不免上朝,请主发下通关文凭,即日启程便了。}

\setlength{\hangindent}{52pt}{玄奘\hspace{30pt}【{\akai 二黄慢板}】有玄奘离娘怀身遭大难,蒙吾师搭救我来到金山。取法名唤玄奘苦读经卷,每日里在殿前把佛来参。因唐王游地府许下大愿,为的是斩神龙起下祸端。传旨意将众僧道法考选,我情愿替君王取经回还。}

\setlength{\hangindent}{52pt}{(玄奘{\hwfs 下})}

\vspace{3pt}{\centerline{{[}{\hei 第二场}{]}}}\vspace{5pt}

\setlength{\hangindent}{52pt}{(徐勣、殷开山、程咬金、尉迟恭{\hwfs 同上})}

\setlength{\hangindent}{52pt}{徐勣\hspace{30pt}({\akai 念})日出山高一片红,}

\setlength{\hangindent}{52pt}{殷开山\hspace{20pt}({\akai 念})唐王江山掌握中。}

\setlength{\hangindent}{52pt}{程咬金\hspace{20pt}({\akai 念})长安多少花似锦,}

\setlength{\hangindent}{52pt}{尉迟恭\hspace{20pt}({\akai 念})堪叹不觉白头翁。}

%徐勣\\殷开山\\程咬金\raisebox{5pt}{\hspace{20pt}老夫,}\\尉迟恭
\raisebox{0pt}[30pt][26pt]{\raisebox{20pt}{徐勣}\raisebox{7pt}{\hspace{-21pt}殷开山}\raisebox{-7pt}{\hspace{-31pt}{程咬金}}\raisebox{-20pt}{\hspace{-31pt}{尉迟恭}}\raisebox{0pt}{\hspace{20pt}老夫------}}

\setlength{\hangindent}{52pt}{徐勣\hspace{30pt}徐茂公。}

\setlength{\hangindent}{52pt}{殷开山\hspace{20pt}殷开山。}

\setlength{\hangindent}{52pt}{程咬金\hspace{20pt}程咬金。}

\setlength{\hangindent}{52pt}{尉迟恭\hspace{20pt}尉迟敬德。}

\setlength{\hangindent}{52pt}{徐勣\hspace{30pt}列公请了。}

%殷开山\\程咬金\hspace{20pt}请了。\\尉迟恭
\raisebox{0pt}[24pt][16pt]{\raisebox{12pt}{殷开山}\raisebox{0pt}{\hspace{-31pt}{程咬金}}\raisebox{-12pt}{\hspace{-31pt}{尉迟恭}}\raisebox{0pt}{\hspace{20pt}请了。}}

\setlength{\hangindent}{52pt}{徐勣\hspace{30pt}只因吾主曾命金山法师,去往西天拜佛取经,今日上殿见驾领凭,同在朝房伺候。请。}

\setlength{\hangindent}{52pt}{(玄奘{\hwfs 上})}

\setlength{\hangindent}{52pt}{玄奘\hspace{30pt}({\akai 念})离了金山寺,上殿见圣君。}

\setlength{\hangindent}{52pt}{玄奘\hspace{30pt}众位国公在上,贫僧稽首。}

%徐勣\\殷开山\\程咬金\raisebox{5pt}{\hspace{20pt}有礼相还。}\\尉迟恭
\raisebox{0pt}[30pt][26pt]{\raisebox{20pt}{徐勣}\raisebox{7pt}{\hspace{-21pt}殷开山}\raisebox{-7pt}{\hspace{-31pt}{程咬金}}\raisebox{-20pt}{\hspace{-31pt}{尉迟恭}}\raisebox{0pt}{\hspace{20pt}有礼相还。}}

\setlength{\hangindent}{52pt}{殷开山\hspace{20pt}儿啊$\cdots{}\cdots{}$}

%徐勣\\程咬金\hspace{20pt}请问国公,他是何人?\\尉迟恭
\raisebox{0pt}[24pt][16pt]{\raisebox{12pt}{徐勣}\raisebox{0pt}{\hspace{-21pt}{程咬金}}\raisebox{-12pt}{\hspace{-31pt}{尉迟恭}}\raisebox{0pt}{\hspace{20pt}请问国公,他是何人?}}

\setlength{\hangindent}{52pt}{殷开山\hspace{20pt}唉!乃是老朽外孙。}

%徐勣\\程咬金\hspace{20pt}原来如此。法师请在殿角伺候,圣驾临朝,我等启奏。请。\\尉迟恭
\raisebox{0pt}[24pt][16pt]{\raisebox{12pt}{徐勣}\raisebox{0pt}{\hspace{-21pt}{程咬金}}\raisebox{-12pt}{\hspace{-31pt}{尉迟恭}}\raisebox{0pt}{\hspace{20pt}原来如此。法师请在殿角伺候,圣驾临朝,我等启奏。请。}}

\setlength{\hangindent}{52pt}{(玄奘{\hwfs 下})}

%徐勣\\殷开山\\程咬金\raisebox{5pt}{\hspace{20pt}金钟三响,圣驾临朝,分班伺候。}\\尉迟恭
\raisebox{0pt}[30pt][26pt]{\raisebox{20pt}{徐勣}\raisebox{7pt}{\hspace{-21pt}殷开山}\raisebox{-7pt}{\hspace{-31pt}{程咬金}}\raisebox{-20pt}{\hspace{-31pt}{尉迟恭}}\raisebox{0pt}{\hspace{20pt}金钟三响,圣驾临朝,分班伺候。}}

\setlength{\hangindent}{52pt}{({\hwfs 四}小太监、{\hwfs 二}大太监{\hwfs 引}李世民{\hwfs 上})}

\setlength{\hangindent}{52pt}{李世民\hspace{20pt}{[}{\akai 引子}{]}先王晏驾,龙归藏,孤掌朝堂。}

%徐勣\\殷开山\\程咬金\raisebox{5pt}{\hspace{20pt}臣等见驾,愿吾皇万岁。}\\尉迟恭
\raisebox{0pt}[30pt][26pt]{\raisebox{20pt}{徐勣}\raisebox{7pt}{\hspace{-21pt}殷开山}\raisebox{-7pt}{\hspace{-31pt}{程咬金}}\raisebox{-20pt}{\hspace{-31pt}{尉迟恭}}\raisebox{0pt}{\hspace{20pt}臣等见驾,愿吾皇万岁。}}

\setlength{\hangindent}{52pt}{李世民\hspace{20pt}平身。}

%徐勣\\殷开山\\程咬金\raisebox{5pt}{\hspace{20pt}万万岁。}\\尉迟恭
\raisebox{0pt}[30pt][26pt]{\raisebox{20pt}{徐勣}\raisebox{7pt}{\hspace{-21pt}殷开山}\raisebox{-7pt}{\hspace{-31pt}{程咬金}}\raisebox{-20pt}{\hspace{-31pt}{尉迟恭}}\raisebox{0pt}{\hspace{20pt}万万岁。}}

\setlength{\hangindent}{52pt}{李世民\hspace{20pt}({\akai 念})忆昔当年战洛阳,收得瓦岗众豪强。可叹恩公秦琼丧,寡人日夜不安康。}

\setlength{\hangindent}{52pt}{李世民\hspace{20pt}寡人大唐天子,贞观在位。因游地府,曾许大愿。考得僧人玄奘,道法甚高,愿替寡人西天拜佛取经。今乃黄道吉日,命他前往。徐皇兄。}

\setlength{\hangindent}{52pt}{徐勣\hspace{30pt}臣。}

\setlength{\hangindent}{52pt}{李世民\hspace{20pt}玄奘可曾宣到?}

\setlength{\hangindent}{52pt}{徐勣\hspace{30pt}今在殿角候旨。}

\setlength{\hangindent}{52pt}{李世民\hspace{20pt}宣他上殿。}

\setlength{\hangindent}{52pt}{徐勣\hspace{30pt}领旨。万岁有旨,玄奘上殿。}

\setlength{\hangindent}{52pt}{玄奘\hspace{30pt}({\akai 内})领旨。}

\setlength{\hangindent}{52pt}{(玄奘{\hwfs 上})}

\setlength{\hangindent}{52pt}{玄奘\hspace{30pt}({\akai 念})金殿传旨宣,别驾往西天。}

\setlength{\hangindent}{52pt}{玄奘\hspace{30pt}玄奘见驾,愿------吾皇万岁。}

\setlength{\hangindent}{52pt}{李世民\hspace{20pt}法师\footnote{此处刘曾复先生念作``法弟''。}替朕西天取经,封卿御弟三藏,如朕亲临。}

\setlength{\hangindent}{52pt}{玄奘\hspace{30pt}愿吾皇万岁。}

\setlength{\hangindent}{52pt}{李世民\hspace{20pt}平身。}

\setlength{\hangindent}{52pt}{玄奘\hspace{30pt}万万岁。}

\setlength{\hangindent}{52pt}{李世民\hspace{20pt}赐座。}

\setlength{\hangindent}{52pt}{玄奘\hspace{30pt}谢座。}

\setlength{\hangindent}{52pt}{(玄奘{\hwfs 坐大边跨椅})}

\setlength{\hangindent}{52pt}{玄奘\hspace{30pt}启奏万岁:~今乃黄道吉日,请驾发下通关文凭,即日启程。}

\setlength{\hangindent}{52pt}{李世民\hspace{20pt}内侍,文房四宝伺候。}

\setlength{\hangindent}{52pt}{李世民\hspace{20pt}【{\akai 二黄慢板}】王因为游地府许愿斋醮,超度那泾河龙重回天曹({\akai 或}:~轮回阴曹)。孤将这众高僧传旨选考({\akai 或}:~众僧人传旨选考),唯有那金山的({\akai 或}:~金山寺)玄奘法高。他情愿往西天见佛拜祷,他情愿取真经替朕代劳。孤想你往西行无穷({\akai 或}:~无数)路道,今日去何日归才得还朝?}

\setlength{\hangindent}{52pt}{玄奘\hspace{30pt}【{\akai 二黄三眼}】请吾主修文凭休迟即早({\akai 或}:~请我主写牒文休迟即早),仗吾皇洪福大何惧山遥。只要人秉诚心见佛拜祷,吾主爷何需要替僧心焦。}

\setlength{\hangindent}{52pt}{李世民\hspace{20pt}【{\akai 二黄原板}\footnote{``提龙笔''一段原来全部唱【{\akai 二黄原板}】,唱法刘曾复先生同样做了示范,词句如下:~ ``提龙笔王亲书大唐国号,命御弟唐三藏奉旨出朝。各国的众王子休挡禁道,到西天取经回替朕代劳。赐御弟锦袈裟霞光万道,孤赐你紫金钵、禅杖一条。孤赐你装经箱、毗卢僧帽,孤赐你四徒儿来把经挑。侍内臣与孤王将宝抬到,金銮殿王与你改换佛袍。''}】提龙笔王亲书大唐国号,命御弟唐三藏奉旨出朝。各国的众王子【{\footnotesize 转}{\akai 二黄三眼}】休挡禁道,到西天取经回替朕代劳。赐御弟锦袈裟霞光万道,孤赐你({\akai 或}:~赐御弟)紫金钵、禅杖一条。孤赐你({\akai 或}:~赐御弟)装经箱、毗卢僧帽,孤赐你四徒儿鞍前马后、涉水登山好把经挑({\akai 或}:~赐御弟四小童好把经挑)。内侍臣与孤王将宝抬到({\akai 或}:~替孤王将宝抬到),金銮殿王与你改换佛袍({\akai 或}:~改换法袍)。}

\setlength{\hangindent}{52pt}{(\textless{}\!{\bfseries\akai 合龙}\!\textgreater{}玄奘{\hwfs 改装},{\hwfs 下})}

\setlength{\hangindent}{52pt}{李世民\hspace{20pt}【{\akai 二黄摇板}】王传旨即便把众卿宣召,随同孤送御弟饯行沙桥。}

\setlength{\hangindent}{52pt}{(李世民{\hwfs 下})}

\vspace{3pt}{\centerline{{[}{\hei 第三场}{]}}}\vspace{5pt}

\setlength{\hangindent}{52pt}{(大太监{\hwfs 上})}

\setlength{\hangindent}{52pt}{太监\hspace{30pt}({\akai 念})朝朝随驾走,时时伴龙行。除了当今主,咱家第一人。}

\setlength{\hangindent}{52pt}{太监\hspace{30pt}咱家,大唐天子驾前,掌朝内监是也。奉了万岁爷的旨意,在沙桥备酒,与三藏法师饯行。酒宴备齐,等候圣驾与众家国公前来。正是:~}

\setlength{\hangindent}{52pt}{太监\hspace{30pt}({\akai 念})吾主许下诚心愿,就有高僧往西天。}

(徐勣、殷开山、程咬金、尉迟恭{\hwfs 上})

\setlength{\hangindent}{52pt}{李世民\hspace{20pt}({\akai 内})【{\akai 西皮导板}】出午门到沙桥王下车辇,}

\setlength{\hangindent}{52pt}{(李世民{\hwfs 搀}玄奘{\hwfs 同上})}

\setlength{\hangindent}{52pt}{太监\hspace{30pt}奴俾接驾。}

\setlength{\hangindent}{52pt}{李世民\hspace{20pt}【{\akai 西皮原板}】叫一声贤御弟细听王言:~孤想你数万里路途崎险,孤愁你何日里得到西天。但愿你此一去早把佛见,但愿你路途上免带愁颜。但愿你见佛祖取经回转,百里外排銮驾接到殿前。}

\setlength{\hangindent}{52pt}{玄奘\hspace{30pt}【{\akai 西皮慢板}】万岁爷休得要将臣怜念,容为臣一一地细奏根源({\akai 或}:~细表根源):~僧的父蒙恩赐七品正县,上任去遇刘贼劫了官船。将僧父用绳捆丢在水面({\akai 或}:~抛在水面),那贼子霸官亲就印为官。贤德母怀小僧十月孕满,想自尽又恐怕绝了后传。那一天生下僧时乖运\emph{蹇},刘洪贼他一见怒气冲天。霎时间({\akai 或}:~顷刻间)要将臣一刀两断,贤德母跪尘埃才得保全。用匣装写血书【{\footnotesize 转}{\akai 西皮二六}】抛在水面({\akai 或}:~丢在水面),取名字江流儿性命由天。金山寺老禅师道法非浅,算定臣不该死({\akai 或}:~算就臣不该死)救至在山前。取法名唤玄奘苦把经念,看破了红尘路世事不贪:~【{\footnotesize 转}{\akai 西皮快板}】一不贪富与贵做官为宦,二不贪妻共子游玩清闲。三不贪吃珍馐五荤三厌,四不贪走花街观看红颜。五不贪住龙楼凤阁温暖,六不贪五花马銮驾旌幡。七不贪用奴仆随身使唤,八不贪出门庭拥后呼前。九不贪红颜女把酒来献,十不贪穿龙袍受王官衔。但愿得见佛祖取经回转,保唐室国泰民安万万年。}

\setlength{\hangindent}{52pt}{李世民\hspace{20pt}【{\akai 西皮二六}】内侍臣看过了皇封御宴,孤爱你道德好十事不贪。孤愿你此一去无灾无难,孤愿你足生云({\akai 或}:~孤愿你足登云)早到佛前。孤赐你饯行酒金杯玉盏,太平去吉日归\footnote{夏行涛{\scriptsize 君}建议作``即日归''。}孤谢上天。}

\setlength{\hangindent}{52pt}{玄奘\hspace{30pt}【{\akai 西皮快板}】谢吾皇饯行酒金杯玉盏,怎敢当吾的主龙恩海宽。转身来对苍天把酒来奠,祝告了天和地日月星官。}

\setlength{\hangindent}{52pt}{徐勣\hspace{30pt}【{\akai 西皮摇板}】领王命在沙桥把行来饯,尊一声大法师细听吾言:~受老朽这一礼非为别干,替吾主取经回大大相烦。}

\setlength{\hangindent}{52pt}{玄奘\hspace{30pt}【{\akai 西皮快板}】小贫僧有何能怎敢领饯,公本是国王师八卦先天。三贤府盖过了臣救君难,保圣驾坐长安万万余年。}

\setlength{\hangindent}{52pt}{尉迟恭\hspace{20pt}【{\akai 西皮摇板}】老尉迟敬酒宴遮住了英雄脸,手捧着紫金杯躬身向前。但愿得此一去取经回转,某在那百里外接进朝班。}

\setlength{\hangindent}{52pt}{玄奘\hspace{30pt}【{\akai 西皮快板}】老国公家住在麻邑贵县,抢三关、夺八寨好不威严。征辽东挂帅印威风八面,访白袍保唐室万古名传。}

\setlength{\hangindent}{52pt}{程咬金\hspace{20pt}【{\akai 西皮快板}】程咬金平日里讲理不惯,尊法师休怪我粗鲁之言。烦公公你与我将酒斟满,}

\setlength{\hangindent}{52pt}{程咬金\hspace{20pt}【{\akai 西皮摇板}】见佛祖取真经早早回还。}

\setlength{\hangindent}{52pt}{玄奘\hspace{30pt}【{\akai 西皮快板}】程千岁可算得忠心赤胆,在瓦岗聚英雄人闻胆寒。弃暗地投明主官高爵显,但愿你寿延年快乐清闲。}

\setlength{\hangindent}{52pt}{殷开山\hspace{20pt}酒来。}

\setlength{\hangindent}{52pt}{殷开山\hspace{20pt}【{\akai 西皮散板}】论国法本应当国师称唤,论家法你本是老夫孙男。在长亭替你母把行来饯,我的孙何日里才得回还。}

\setlength{\hangindent}{52pt}{玄奘\hspace{30pt}【{\akai 西皮散板}】见外公不由我心中凄惨,烦外公拜儿母不孝之男。就说儿奉王旨不敢迟慢,多拜上贤德母少来问安。}

\setlength{\hangindent}{52pt}{玄奘\hspace{30pt}【{\akai 西皮散板}】在沙桥抬头看红日西转,请万岁、众国公驾回朝班。}

\setlength{\hangindent}{52pt}{李世民\hspace{20pt}【{\akai 西皮散板}】内侍臣带龙驹孤把缰挽,叫御弟跨金镫早奔阳关。}

\setlength{\hangindent}{52pt}{玄奘\hspace{30pt}哎呀!}

\setlength{\hangindent}{52pt}{玄奘\hspace{30pt}【{\akai 西皮散板}】君带马与臣骑世间稀罕,阻王驾休咒臣忙把王拦。}

\setlength{\hangindent}{52pt}{玄奘\hspace{30pt}将马带过!}

\setlength{\hangindent}{52pt}{玄奘\hspace{30pt}【{\akai 西皮散板}】辞王驾别国公忙把路趱,到西天拜佛祖取经回还。}

\setlength{\hangindent}{52pt}{(玄奘{\hwfs 下},{\hwfs 众}太监、内侍、徐勣、殷开山、程咬金、尉迟恭{\hwfs 引}李世民{\hwfs 同下})}

}

