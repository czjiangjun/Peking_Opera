\newpage\hspace{30pt}~

{%

\subsubsection{\large\hei {大保国~{\small 之}~

杨波}}

{({\akai 内})阻------诏}\footnote{段公平君作``住------喳''。}!

{({\akai 内})}兵部杨波。\hspace{20pt}~

{({\akai 内})}不但不押,还要上殿面奏。

{({\akai 内})}千岁慢走!\hspace{20pt}~

{({\akai 内})}兵部杨波。\hspace{20pt}~

{({\akai 内})}来也!\hspace{30pt}~

\setlength{\hangindent}{56pt}{【{\akai 二黄摇板}】太师朝房用奸谋,传旨国太让龙楼。本当上殿把本奏, }

唉!\hspace{40pt}~

\setlength{\hangindent}{56pt}{【{\akai 二黄摇板}】怎奈我官职小啊不敢出头。 }

参见千岁。\hspace{40pt}~

国太要将江山让与太师。千岁可曾画押?

学生敢莫是吃了熊心豹胆,焉敢与那贼同谋?

官卑职小,难以出头。\hspace{20pt}~

只要千岁作主,拚着一腔热血洒落金阶,我也要落一个青史名标。

遵命!\hspace{40pt}~

\setlength{\hangindent}{56pt}{【{\akai 二黄原板}】为的是大明朝锦绣家邦。 }

\setlength{\hangindent}{56pt}{【{\akai 二黄原板}】殿角下坐定了谋朝篡位奸贼李良。 }

\setlength{\hangindent}{56pt}{【{\akai 二黄原板}】老王爷赐铜锤上打昏君、下打谗臣、压定了满朝的文武、哪一个不尊定国王开国的忠良。 }

呵哈哈哈$\cdots{}\cdots{}$({\hwfs 笑}{\hwfs 介})

\setlength{\hangindent}{56pt}{【{\akai 二黄原板}】臣愿国太福寿绵长。 }

\setlength{\hangindent}{56pt}{【{\akai 二黄原板}】三跪九叩谢皇娘。\hspace{10pt}~ }

\setlength{\hangindent}{56pt}{【{\akai 二黄原板}】大明江山共作商量。 }

臣等不押。\hspace{40pt}~

(徐延昭\hspace{40pt}~

\setlength{\hangindent}{56pt}{【{\akai 二黄原板}】唐僖宗坐江山天心不顺,他驾前有一臣梁王朱温。臣弑君子弑父弟霸兄嫂, }

【垛板}】君不君、臣不臣、父不父、他子不子、就败坏人伦。$\cdots{}\cdots{}$)

(徐延昭\hspace{40pt}~

\setlength{\hangindent}{56pt}{【{\akai 二黄原板}】$\cdots{}\cdots{}$北海内现铜桥渡过旗人。到如今才能得干戈宁静,为什么将社稷让与他人。) }

千岁。\hspace{40pt}~

遵命。\hspace{40pt}~

臣兵部侍郎杨波,有一道太平表章,可容臣启奏?

容奏:~\hspace{40pt}~

({\akai 念})忆自元朝居华地,世上多少古今奇。山崩地裂江河啸,风起云会星斗移。

容奏:~\hspace{40pt}~

\setlength{\hangindent}{56pt}{【{\akai 二黄原板}\footnote{这段唱腔,目前通行唱法是依据李适可传承的 }

【{\akai 二黄快三眼}】,刘曾复先生示范的词句如下:~ 

【{\akai 二黄快三眼}】臣不奏前三皇后代五帝,奏的是我大明一段机密:~太祖爷在南京称孤立帝,各路的烟尘起来夺华夷。四川省明玉珍把兵来起,领人马从蜀东杀到蜀西。方国珍在浙江自立为帝,苏州城张士诚僭位登极。湖广的陈友谅兴兵起义,南京城夺取那采石矶。玉山城设下了诓君之计,在鄱阳湖边火莲炎焰血染征衣。只杀得有庄有田无人耕地,只杀得贸易经商客旅稀。只杀得弟唤兄来兄不能顾弟,只杀得父在东来子在西。先皇爷坐江山并非容易,十八年改国号臣不能全知。  又,据李楠君告知,``火莲炎焰''四字他学的是``火焰连连''。}】臣不奏前朝中历代帝君,臣启奏机密事出在大明:~太祖爷晏了驾龙归海境,大明朝无一人执掌龙庭。满朝中文武臣袖手不问,马皇后扶建文立帝为君。普天下众宗室无有议论,唯有那四太子燕山发兵。姚广孝为军师暗传将令,六月天冻黄河兵困金陵。直逼得马皇后火焚丧命,直逼得那建文帝削发为僧。亲叔侄尚且来争竞,何况那李呃太师是外姓之人。

臣不能全知。\hspace{30pt}~

徐千岁开国元勋,必然知晓。\hspace{10pt}~

谢国太!\hspace{40pt}~

\setlength{\hangindent}{56pt}{【{\akai 二黄原板}】汉高皇曾起义({\akai 或}:~初起义)路过那硭砀山,偶遇白蟒把路拦。执宝剑将蟒斩为两段,兴人马灭秦楚一统河山。到后来出王莽又出苏献,松棚会害平帝命染黄泉。夺玉玺搜宫院王莽谋篡,把一个王国母逼死井前。前朝的事迹当为鉴,也免得学国母跌足怨天。 }

田子裕在你府中常来常往,算得是心腹之人。

算得。\hspace{40pt}~

\setlength{\hangindent}{56pt}{【{\akai 二黄原板}】残唐五代乱纲常,各路诸侯霸一方。宋太祖陈桥披黄蟒,十八载马上为帝王。三下河东基业创,归来染病在龙床。御弟进宫将兄望,烛影摇红祸起萧墙。亲手足尚且这等样, }

\setlength{\hangindent}{56pt}{【{\akai 二黄摇板}】何况太师与娘娘。\hspace{10pt}~ }

你乃皇亲国戚,焉能发得高墙?\hspace{10pt}~

呃,发不得。\hspace{30pt}~

\setlength{\hangindent}{56pt}{【{\akai 二黄散板}】劝国太江山休要让。 }

唉呀!\hspace{40pt}~

\setlength{\hangindent}{56pt}{【{\akai 二黄散板}】国太金殿把贼宠,徐、杨保本一场空。回头忙把千岁请,倚老卖老打奸臣。 }

千岁虎老雄心在,撒得疯,打得动奸贼。

打得动。\hspace{40pt}~

功劳簿在此!\hspace{30pt}~

功劳簿并无奸贼的名字。

呃!\hspace{40pt}~

且慢,自从盘古以来,哪有臣打君的道理?

上殿请罪。\hspace{40pt}~

臣启国太:~徐、杨有欺君之罪,国太降旨。

谢国太!\hspace{40pt}~

国太传下旨意:~从今以后,朝中有事无事,不与徐、杨二大奸党相干。

敢夺社稷!\hspace{40pt}~
