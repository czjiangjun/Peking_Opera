\newpage
\subsubsection{\large\hei {鱼肠剑~{\small 之}~伍员}}
\addcontentsline{toc}{subsection}{\hei 鱼肠剑~\small{之}~伍员}

\hangafter=1                   %2. 设置从第1⾏之后开始悬挂缩进  %
\setlength{\parindent}{0pt}{
{\centerline{{[}{\hei 第一场}{]}}}\vspace{5pt}

({\akai 内})马来!

\setlength{\hangindent}{56pt}{【{\akai 西皮摇板}】单人匹马({\akai 或}:~单枪匹马)弃楚樊, }

\setlength{\hangindent}{56pt}{【{\akai 西皮快板}】龙奔沧海虎奔山。历阳伏匿七夜晚,须似银条({\akai 或}:~发似银条)过昭关。 }

俺,伍员。幸喜逃出楚地,来此离吴都不远,马上加鞭。({\akai 或}:~且喜逃出楚地,看前面离吴都不远,就此马上加鞭。)

\setlength{\hangindent}{56pt}{【{\akai 西皮原板}】一事无成两鬓斑,叹光阴一去不回还呐。日月轮流催晓箭,青山绿水呀常在眼前。恨平王无道纲常乱,信宠无极狗奸谗。他害我满门呐真悲惨,我与奸贼不共戴天。实指望到吴国 }

【{\footnotesize 转}{\akai 西皮快板}】借兵转,行至在昭关有阻拦。单人匹马常遮掩,在历阳山下({\akai 或}:~在历阳山前)遇高贤。设计救出了昭关险,马到长江无渡船。多蒙渔父行方便,他为我投江实可怜。浣纱女,心好善,一饭之恩前世缘。眼望吴城路不{\footnotesize 呃}远,

\setlength{\hangindent}{56pt}{【{\akai 西皮散板}】报仇心急马加鞭。 }

\vspace{3pt}{\centerline{{[}{\hei 第二场}{]}}}\vspace{5pt}

{老丈请转。}

{方才那一大汉与人争斗,见一妇人手执拐杖,一唤即回,是何故也({\akai 或}:~是何缘故)?}

{原来如此。}

{本不是({\akai 或}:~原非)此地人氏。}

{楚国监利人也。}

{这$\cdots{}\cdots{}$}

{寄居在此。}

{改日领教。}

{请------}

{(哎呀且住,)适才听老丈之言,专诸孝义双全,胁力}\footnote{ ``胁''本意为两膀。夏行涛{\scriptsize 君}认为,因``胁''繁体作``脅'',``胁力''可能是``膂力''之误。}{无比({\akai 或}:~力大无比)。我不免与他结交,以为膀臂({\akai 或}:~日后定是膀臂)。}

{正是:~({\akai 念})交友当交真君子,求人须求大丈夫。({\akai 或}:~交友须交奇男子,求人当求大丈夫。)}

{来此已是。}

{专兄开门来。}

{愚下来访。}

{请。}

{有座。}

{愚下姓伍名员字子胥,(乃)楚国监利人也。}

{岂敢。}

{唉,专兄啊!}

\setlength{\hangindent}{60pt} {【}西皮快板{】平王无道乱楚邦,父纳子媳({\akai 或}:~父纳子妻)乱纲常。特到贵邦借兵将,奈无机会见吴王。}

{(奈无机缘,不敢造次。)}

{若得如此,乃伍员之幸也。}

{闻得专兄孝义双全,愚下愿与专兄结为金兰({\akai 或}:~愚意欲与专兄结为金兰之好)。幸勿见却。}

{愚下真心实意,专兄休得谦逊({\akai 或}:~休得过谦)。}

{理当拜见({\akai 或}:~原要拜见)。}

{(请!)}

{请呐。}

\setlength{\hangindent}{60pt} {【}西皮摇板{】孝义双全人钦仰,}

\setlength{\hangindent}{60pt} {【}西皮摇板{】报仇之事全仰仗。}

{(专诸\hspace{30pt}~

\setlength{\hangindent}{56pt}{【}西皮摇板{】$\cdots{}\cdots{}$见吴王。)} }

\vspace{3pt}{\centerline{{[}{\hei 第三场}{]}}}\vspace{5pt}

{唉!}

\setlength{\hangindent}{60pt} {【{\akai 西皮快板}】行过东来又转西,举目无亲独自依。落魄的英雄谁怜惜({\akai 或}:~衣衫褴褛谁赒济),吹箫怎能({\akai 或}:~焉能)充得饥。}

{唉!俺伍员自到吴地求见姬光,奈无机会。({\akai 或}:~自与专诸结拜,吴王难见,)行囊典尽,衣履全无,只落得吹箫讨乞!}

\setlength{\hangindent}{56pt}{【{\akai 西皮原板}】姜太公不仕啊隐磻溪呀,运败时衰鬼神欺。周姬昌治西岐({\akai 或}:~梦飞熊)呀在灵台殿里,渭水河访贤保社稷。东迁雒邑\footnote{ ``雒邑''亦作``洛邑'',据明张岱所著《夜航船》\upcite{Zhang_Yehangchuan}载:~周公筑雒邑二城,后即为洛阳。汉光武定都洛邑。汉以火德王,忌水,故去水而加隹,改洛为雒;后魏以土德王,以水得土,而流土得水而柔,故又除隹加水。}【{\footnotesize 转}{\akai 西皮二六}】王纲{\footnotesize 呃}坠,各国诸侯把心离。背盟毁约失信{\footnotesize 呃}义,图霸争强各自为。吴子寿梦立王位,力压诸侯服四夷。某单人独骑弃楚{\footnotesize 呃}地,要见姬光恨无机。困苦的英雄似蝼蚁,眼见得含冤化灰泥。落魄天涯谁赒济,只落得吹箫暂充饥。}

\setlength{\hangindent}{56pt}{【{\akai 西皮快板}】正在街前闲站立,见一位官长({\akai 或}:~见一位君官)相貌奇。头戴珠冠双凤翅,身穿一件衮龙衣。莫非他是姬太子,特地前来({\akai 或}:~有意前来)访子胥。本当向前去见礼,帽破衣残不整齐。眉头一皱心生计,将我的冤仇提一提。 }

\textless{}\!{\bfseries\akai 三叫头}\!\textgreater{}爹爹!母亲!唉!兄长啊!

\setlength{\hangindent}{66pt}{【{\akai 反西皮散板}】子胥阀阅门楣第,到如今({\akai 或}:~俺好似)凤褪翎毛怎能飞。我本是英雄不得第,落魄天涯有谁知。可叹我父母的冤仇沉海底呀,空负我堂堂七尺躯啊。伍子胥呀,伍盟府哇,父母冤不能报,爹娘啊! }

来了!

\setlength{\hangindent}{56pt}{【{\akai 西皮快板}】听说一声唤子胥,愁人脸上笑微微。走向前,施一礼,愿王福寿与天齐。 }

千岁!

\setlength{\hangindent}{56pt}{【{\akai 西皮二六}】富贵穷通不由己,也是我的时衰命运低。我本是楚国的功臣家住在监利,姓伍名员字子胥。恨平王无道宠无极,败坏纲常父纳子媳。我的父谏奏剑下死,一家满门血染衣。闻千岁招贤纳士多仁义,我是特地前来借兵报冤屈。孤身到了吴国地,还望纳员\footnote{ 李楠{\scriptsize 君}认为此处当作``纳允'',夏行涛{\scriptsize 君}则认为此处当作``拿云''。}【{\footnotesize 转}{\akai 西皮快板}】把难人提。伍子胥一朝得了第,知恩报德不敢移。}

\vspace{3pt}{\centerline{{[}{\hei 第四场}{]}}}\vspace{5pt}

(谢座。)

(二位)先生请坐。

\setlength{\hangindent}{56pt}{【{\akai 西皮快板}】含悲忍泪叫贤弟,有苦之人把话提。特地借兵到此地,不杀平王气怎息。 }

(谢座。)

(启禀千岁:~)臣有一结义之弟({\akai 或}:~结拜义弟),名唤专诸。此人胁力无比({\akai 或}:~此人英勇双全),堪当此任。

姑苏城外({\akai 或}:~姑苏城内),一唤即至。

领旨。({\akai 或}:~遵命。)

\setlength{\hangindent}{56pt}{【{\akai 西皮摇板}】辞别千岁奉聘礼, }

\setlength{\hangindent}{60pt} {【}西皮摇板{】}风吹呀云散现虹霓。
}
