\hypertarget{ux540e-ux8bb0-ux4e0e-ux81f4-ux8c22}{%
	\section{\hei{后~记~与~致~谢}}\label{ux540e-ux8bb0-ux4e0e-ux81f4-ux8c22}}

刘曾复教授(1914-2012)是我国前辈生理学家,毕生致力于生理学的教学与科研工作,怹对京剧也有深入、系统的研究。刘曾老学戏,早年师从王荣山先生,后又从王凤卿、刘砚芳、贯大元、钱宝森、侯喜瑞等诸家问艺。刘老不仅对京剧老生艺术有深厚的造诣,对京剧把子、脸谱亦无一不精,这在业余京剧爱好者中是极罕见的。

2004年夏,在樊百乐兄的引荐下,我有幸拜识刘曾复先生,此后的八年里,无论在学术研究还是戏曲欣赏方面,都曾得到刘老的热心提点,自感受益匪浅。现在由百乐兄整理的刘曾复先生的说戏剧本集即将面世,我由衷地感到高兴!因我也曾参与过其中一部分工作,华东师范大学钟锦副教授嘱我写一篇后记,我想借此机会,谈一下这本书的由来。

2009年9月,受吴小如先生(1922-2014)委托,我将吴先生珍藏的刘曾复先生为中国戏曲学院录制的百余出说戏磁带翻录整理成数字格式的音频文件,供吴先生脑梗后养病期间消遣。籍此机缘,我又将能找到的刘老在不同场合说戏录音汇集,编撰成``刘曾复教授说戏录音系列''光盘。记得当年11月初,我冒雪把制作的光盘送到刘曾老府上时,老先生非常高兴,跟我说:``没想到我还能见到这套录音。''

2011年6月间,我有幸拜读到钟锦先生根据这套录音整理的刘老说戏的唱词文稿(不含念白),钟老师告诉我,因当初的录音设备简陋,加之磁带放置的年代较久,有些录音听起来不很清晰,文稿中的一些词句记录不够准确。出于忠实保存前辈艺术的考虑,也为弥补我``点金成铁''的拙劣翻录技术,我萌生了将先生说戏录音整理成文的想法,可巧此间百乐兄受刘曾老之托,已着手启动这个``工程'',我因此参与了一部分工作。从2012年3月到2013年1月,经过我们的努力,终于完成了初稿。

在文稿整理过程中,我们注意到,刘曾老在不同的场合的说戏,即使是同一出戏,个别词句也略有出入。为完整纪录先生的说戏内容,对这类剧目,我们首先选定一个底本\protect\hyperlink{fn678}{\textsuperscript{678}},在此基础上,整合刘老不同场合的说戏录音,完善成最后的剧本。凡唱、念词句与底本有出入的地方,尽可能作了标注。我们认为,这些不同处理的唱、念,也是刘老活用``三级韵''法则的示范,值得保留;希望文中的标注不至于影响读者阅读的顺畅。此外,在我们的知识范围内,对一些生僻的典故、词汇作了简要注解。而一些传统戏曲习惯用词,如``辅保''与``扶保''、``做甚''与``则甚''、``叫人''与``教人''等,整理时均未作统一。

书中有关剧目中的把子,主要摘录自《京剧新序》和《京剧老生把子见闻录》一文记录的开打和舞台调度。\textbf{除了《太平桥》等廿出剧目,其余剧目的场次安排则主要参考了《京剧汇编~(1-109集)》、《传统剧目汇编》、《京剧丛刊~(1-50集)》和``中国京剧戏考''网站上的相应的剧目的安排,个别剧目的词句也参考了``中国京剧老唱片''网站上载的老唱片戏词。剧目顺序基本按照剧中人物活动年代排列,个别剧目的年代排序参考了《京剧大戏考》}和《京剧知识词典(增订版)》\textbf{中的顺序。}

特别需要说明的是,各剧中唱腔、板式的标注,主要沿用钟锦老师提供的唱词文稿的纪录,此外也部分参考了相关剧本资料。另一方面,由于京剧发展史上``散板''和``摇板''一度互易,刘曾复先生的各类戏本中不少地方仍保留了这种痕迹,对此我们遵从原作,不作专门的统一。如果读者想要准确掌握相关剧目的唱腔、板式,建议以刘老的说戏录音为准。

本书初稿完成之际,承吴小如先生审阅了全部文稿,怹提出了非常详尽、细致的修改指导意见;刘曾老手录钞本中,《群英会》、《美良川》的复印件是刘老生前提供的,\textbf{《柴桑口》、《平五路》、《七星灯》、《铁笼山·迷当发兵》、《三击掌》、《龙虎斗》、《审头刺汤》}复印件则是刘老仙逝后由怹的女儿\textbf{祖敬阿姨在整理刘老遗物时提供的;北京的}段公平博士、\textbf{台湾中央大学}李元皓教授、上海的夏行涛先生分别为我们仔细校对了全部文稿,并修正了文稿的大量错误;\textbf{追随刘先生多年的}陈超老师、吴焕老师、娄悦老师、刘新阳老师、\textbf{何毅老师、美国芝加哥大学徐芃博士、北京市顺义区医院的樊剑医师、北京城市学院李楠博士以及北方工业大学钱盛博士、加拿大的网友``小豆子''老师、《健康时报》上海采访部的尹薇女士、北京大学的郝以鑫同学为我们提供和考订了很多准确、可靠的戏词;美国纽约梨园社青年团的马玢先生在此书出版后陆续为我们指正了诸多板式标注的舛误。诸位师长、同好的细致、无私的付出和帮助,为本书增色无数。}

\textbf{书稿完成过程中,我个人还曾得到台湾新竹交通大学邵锦昌教授、台湾大学王安祈教授以及首都医科大学李效义教授、上海广播电视台《绝版赏析》节目组柴俊为先生、复旦大学姜鹏博士和好友肖阳、刘鹏等的鼓励,也感谢我曾经的同事董栋博士、朱元慧博士。诸位师友的关心和肯定,给了我很大的动力。此外特别需要感谢爱妻高飞女士,她的宽容、理解和支持,为我能在业余时间参与有关工作创造了条件。}

\textbf{整理、出版此书的过程中,刘曾复先生和吴小如先生先后辞世,使我们更深刻地体会到腹笥渊博的前辈对于京剧的传承的重要性。保存、整理、抢救他们的艺术财富已是时不我待了。对于京剧我们都是外行,能做的也只是一点资料保存工作。本书所整理辑录的,只是我们所找到刘曾复先生的说戏资料,远非刘老所掌握剧目的全部。一些先生们生前提到的传统戏,像《清河桥》、《太行山》、《磐河战》、《葭萌关》、《凤鸣关》、《汜水关(锤换带)》等,因为没有找到相关的唱腔、唱词记录资料,只能付诸阙如;有些戏(如《柴桑口》、《平五路》、《美良川》、《取金陵》等)虽然刘老留下了钞本,但因为没有找到先生的说戏录音或说戏录音不完整,不少存疑处也无从请教。书稿虽已经同好多方指谬,但}书中存在的错误(特别是唱腔、板式的标注方面)肯定还有不少,诚恳期待京剧内行、对京剧研究有素的专家、学者和爱好京剧的读者不吝批评赐正(具体可发送电子邮件到
\href{mailto:czjiangjun@yeah.net}{{czjiangjun@yeah.net}})。

\textbf{樊百乐兄随侍刘曾老多年,尽得先生的真传。为本书的整理、出版,百乐兄耗费了大量的心血和精力,厥功至伟;本书能顺利出版,也得益于钟锦老师的默默付出。没有他们的努力,本书是不可能问世的。今年恰逢先生百周年诞辰,此书的出版想来也是对先生在天之灵很好的告慰。}

\textbf{时近甲午冬至,谨以本文表达对刘曾复先生和吴小如先生的深切缅怀!}

\begin{flushright}
\textbf{后学 姜骏 谨记}

\textbf{2014-12-18 定稿}

\textbf{2019-05-28 增补}
\end{flushright}
