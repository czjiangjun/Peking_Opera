\newpage
\phantomsection %实现目录的正确跳转
\section*{\large\hei 连营寨~\protect\footnote{陈超老师介绍:~《连营寨》前半出西皮,后半出昆腔,是这出戏的特点,陆逊先唱\textless{}\!{\bfseries\akai 粉蝶儿}\!\textgreater{}、\textless{}\!{\bfseries\akai 醉太平}\!\textgreater{},然后再唱九支曲子。{陆逊唱北曲},{其他人唱南曲}。}~{\small 之}~刘备}
\addcontentsline{toc}{section}{\hei 连营寨~{\small 之}~刘备}

\hangafter=1                   %2. 设置从第1⾏之后开始悬挂缩进  %
\setlength{\parindent}{0pt}{

\vspace{3pt}{\centerline{{[}{\hei 第一场}{]}}}\vspace{5pt}

(\setlength{\hangindent}{52pt}{诸葛瑾\hspace{20pt}【{\akai 西皮摇板}】奉王命行路程不耽时候,此一番到蜀营讲和罢休。但愿得此一去旗偃歌奏\footnote{段公平{\scriptsize 君}建议作``旗偃戈收''。},免生灵遭涂炭民死蜉蝣。) }

\vspace{3pt}{\centerline{{[}{\hei 第二场}{]}}}\vspace{5pt}

\setlength{\hangindent}{56pt}{【{\akai 西皮摇板}】孙仲谋与孤王结成仇寇,只杀得他兵和将尸堆山丘。望空中二贤弟神灵保佑,灭却了东吴贼方肯罢休。 }

有请。

平身。

请坐。

到此乃依\footnote{夏行涛{\scriptsize 君}建议作``乃一''。}({\akai 或}:~乃是)客位,有话叙谈,哪有({\akai 或}:~焉有)不坐之理?~(请坐。)

子瑜远来,有何事故?

哼,汝东吴现在危急,故命汝以巧言来说和。

住口!

汝东吴不仁,杀弟之仇,不共戴天。欲朕罢兵,哼哼,({\akai 或}:~汝东吴诡谋,损孤二弟,此仇不共戴天,欲孤罢兵,)除死方休!

不看我家丞相之面,先斩汝首({\akai 或}:~定斩汝首)。今且放汝回去,说与孙权,洗颈就戮。({\akai 或}:~今且放你回去,说与孙权,教他洗颈待戮。)

去罢!

\setlength{\hangindent}{52pt}{(诸葛瑾\hspace{20pt}【{\akai 西皮摇板}】适才间在蜀营申述利害,见主公定良谋好把兵排。) }

\vspace{3pt}{\centerline{{[}{\hei 第三场}{]}}}\vspace{5pt}

关兴、张苞,传令吩咐:~满营大小将官,俱穿孝服。将你父等灵牌请在灵堂,一概仇人绑好。({\akai 或}:~关兴、张苞,打扫灵堂,安放灵位。满营将官,俱穿孝服。将一干人犯绑至灵堂。)为伯亲自祭奠$\cdots{}\cdots{}$({\hwfs 哭介})

摆驾!

\setlength{\hangindent}{56pt}{【{\akai 西皮摇板}】想当年结桃园同天发咒,愿同年同月日({\akai 或}:~同日月)同刻罢休。到如今一旦间死别分手,孤岂肯独一人乐享无忧。 }

\setlength{\hangindent}{56pt}{【{\akai 西皮导板}】白盔白甲白旗号,\hspace{10pt}~ }

\textless{}\!{\bfseries\akai 哭头}\!\textgreater{}二弟呀,三弟呀!啊$\cdots{}\cdots{}$

\setlength{\hangindent}{56pt}{【{\akai 回龙}】孤的好兄弟!\hspace{20pt}~ }

\setlength{\hangindent}{56pt}{【{\akai 西皮原板}】满营将官哭嚎啕。孤王兴兵把仇报,扫灭了东吴恨方消。请过了神牌怀中抱, }

\setlength{\hangindent}{66pt}{【{\akai 反西皮二六}】点点珠泪往下抛。当年桃园结义好哇,胜似一母共同胞。不幸徐州失散了,万般无奈暂且降曹。那曹操待你的情义好,上马金银也曾赠过你锦袍({\akai 或}:~赐过你锦袍)。美女十名你不要,挂印封金辞奸曹。匹马单刀保皇嫂,过五关斩六将擂鼓三通把蔡阳的首级枭,可算得盖世的英豪。华容道上放曹操,大仁大义志量高({\akai 或}:~亘古流表\footnote{夏行涛{\scriptsize 君}建议作``亘古流标''。})。单刀赴会天下晓,英雄美名亘古标({\akai 或}:~志量高)。可恨(那)孙权行计巧,害孤二弟归天曹。愚兄兴兵把仇报,扫平了东吴气才消。还望二弟神灵保, }

\setlength{\hangindent}{56pt}{【{\akai 西皮散板}】神灵{\footnotesize 呐}保, }

\textless{}\!{\bfseries\akai 哭头}\!\textgreater{}孤的好兄弟呀,({\akai 或}:~二弟呀,)

\setlength{\hangindent}{56pt}{【{\akai 西皮摇板}】不灭孙权不回朝({\akai 或}:~不还朝)。 }

\setlength{\hangindent}{56pt}{【{\akai 西皮摇板}】非是为伯伤心泪掉,孤与你父({\akai 或}:~我与你父)生死交。哭罢了二弟把三弟叫, }

\textless{}\!{\bfseries\akai 哭头}\!\textgreater{}翼德(弟)呀,桓侯哇,啊,孤的好兄弟呀,

\setlength{\hangindent}{66pt}{【{\akai 反西皮二六}】叫声三弟听根苗:~大破黄巾天下晓,敌人见你望风逃。虎牢关曾把吕布的发冠挑,长坂坡前喝断当阳桥({\akai 或}:~喝断灞桥)。夜战马超胆气好,义释严颜颇有略韬。可恨那范疆、张达两个贼强盗,谋害英雄二贼脱逃。愚兄兴兵与你把仇报,只杀得孙权魄散魂消。情愿罢兵写降表,同心合意共灭奸曹。锦绣山河孤不要,一心与你把仇消({\akai 或}:~一心只想把仇消)。哭哑了咽喉把三弟叫,把三弟\textless{}\!{\bfseries\akai 哭头}\!\textgreater{}叫,豹头环眼的三弟呀, }

\setlength{\hangindent}{56pt}{【{\akai 西皮摇板}】拿住孙权两开销。\hspace{10pt}~ }

看酒来,待孤亲自祭奠。

(两弟受孤一拜。)

儿要多拜几拜。({\akai 或}:~儿等多拜几拜。呃$\cdots{}\cdots{}$({\hwfs 哭介}))

一概仇人({\akai 或}:~将一干仇人),拿去开刀。

啊?

此贼为何不斩?

哦------剑来!

\setlength{\hangindent}{56pt}{【{\akai 西皮散板}】到此时({\akai 或}:~到如今)还讲什么郎舅之分,献荆州贼有何亲戚之情。三尺剑正国法又报仇恨,死眼前看贼子你埋怨何人。 }

关兴、张苞,吩咐文武官员({\akai 或}:~传孤旨意,满营将官),歇兵三日,兵发东吴。

正是:~({\akai 念})满腔怒气冲昊天,誓把({\akai 或}:~要把)东吴踏平川。

\vspace{3pt}{\centerline{{[}{\hei 第四场}{]}}}\vspace{5pt}

({\akai 念})起居梦寐恨吴寇,不报冤仇誓不休。

孤自兴兵以来,(势如破竹,)吾国({\akai 或}:~我国)人马屡屡得胜,他邦兵将({\akai 或}:~他国兵将)节节败溃。那些吴狗们望风而逃,此乃诸将之功也。

(众\hspace{40pt}主公妙计,臣等何功之有?)

诸将俱各有功。

(马良\hspace{30pt}启奏主公,闻得孙权又拜陆逊为都督,兵扎猇亭。)

坐下。

(马良\hspace{30pt}谢座。)

陆逊何路人也?

(马良\hspace{30pt}乃九江太守陆骏\footnote{据《三国志·吴书》载,陆逊是九江都尉陆骏之子。}之后。)

哼,懦弱书生,统领人马({\akai 或}:~担此重任),岂不贻笑大方?

(马良\hspace{30pt}那陆逊虽是一介书生年幼,前番吕蒙白衣渡江,暗取荆州,乃此人之计也。)

哦!竖子诡谋({\akai 或}:~孺子诡谋),损孤二弟,今当擒之!

关兴、张苞,传令进兵。

(马良\hspace{30pt}且慢!主公请息龙怒。)

为何拦阻?

(马良\hspace{30pt}陆逊智胜周郎,不可轻敌。)

唉!孤用兵老矣!岂反不如一黄口孺子么?

(马良\hspace{30pt}如今陆逊不战不退,莫非有何诡计?)

哼!黄口孺子,有多大能为?既敢当此重任,就该领兵前来,与孤对敌。战又不战,退又不退,其情可恼!

(马良\hspace{30pt}倘若陆逊以逸待劳,如之奈何?)

孤兵精粮足,与他对守何惧({\akai 或}:~对垒何惧)?

(马良\hspace{30pt}堪堪天气炎热,暑气难当,兵扎离火之中,汲水不便。又恐将士多生疾病。)

不妨,孤将营寨,移于茂林深处({\akai 或}:~孤将人马,移至茂林深处),待(等)过夏到秋,并力进兵,东吴自然休矣({\akai 或}:~吴国可图也)。

(马良\hspace{30pt}我若兵动,倘陆逊踏营,如何是好?)

孤命关兴、张苞各带人马({\akai 或}:~带领精兵),埋伏山谷之中。倘陆逊来击,引兵突出,孺子可擒也。({\akai 或}:~倘陆逊劫营,我军伏兵杀出,一鼓擒之。)

(众\hspace{40pt}主公妙计,臣等不及也。)

(马良\hspace{30pt}主公要移营寨,可画成地图,问过丞相?)

孤亦颇知兵法,此事何必又问({\akai 或}:~再问)丞相?

(马良\hspace{30pt}古语云:兼听则明,偏听则暗。望陛下详之。)

也罢。卿可自去各营,画成四至八道\footnote{``四至八道''是旧时标志土地界域的用语。表示四面八方所到之处及通往的道路。}图本,亲到东川,去问丞相。若有不便,即来回奏。孤再作裁处。({\akai 或}:~如此就命你将山势、营盘画成图本,去至东川,送与丞相观看。倘有不到,急速回来,孤再作裁处。)

(马良\hspace{30pt}领旨。)

关兴、张苞,就此移营者({\akai 或}:~择吉移营者)。

正是:~({\akai 念})龙麟启祚\footnote{据李元皓{\scriptsize 君}告知,``启祚'',是发祥、开创帝业之意。``中国京剧戏考''网站《戏考》第六册本作``起坐'',似非。}如反掌,干戈霸业定太平。

\vspace{3pt}{\centerline{{[}{\hei 第五场}{]}}}\vspace{5pt}

({\akai 念})月当空乌鸦嘶叫,帅字旗无风自摇({\akai 或}:~无风自飘)。

(报子\hspace{30pt}报!东吴人与马缓缓移动。)

再探。

(此乃疑兵。)

啊,沙摩柯听令。

(沙摩柯\hspace{20pt}在。)

带领蛮兵女将前去探看虚实,相机击之。

(沙摩柯\hspace{20pt}得令。)

此乃疑兵,何足道哉?

(报子\hspace{30pt}报!江北营中火起。)

再探。

关兴前去救火。({\akai 或}:~关兴营救。)

(关兴\hspace{30pt}得令。)

江北营中火起,(此乃)我军自不小心。

(报子\hspace{30pt}报!两岸火起。)

再探。张苞去救。({\akai 或}:~张苞急救。)

两岸火起,我军大不利也。

(报子\hspace{30pt}报!满营火起。)

(再探。)

不、不、不$\cdots{}\cdots{}$不好了!

({\akai 念})\textless{}\!{\bfseries\akai 蛮牌令}\!\textgreater{}看、看、看,看呐,风助火威狂,火乘猛风飏。满天飞烈焰,遍地闪金光({\akai 或}:~撒金光)。祸从天降,祸从天降呃。寻不出路当央\footnote{段公平{\scriptsize 君}建议作``路当阳''。},寻不出路当央。快带丝缰,快带丝缰。

\vspace{3pt}{\centerline{{[}{\hei 第六场}{]}}}\vspace{5pt}

(赵云\hspace{30pt}赵云接驾。)

哎呀四弟呀!你看孤被他们烧得乌焦巴弓了。({\akai 或}:~四弟你来了,杀出重围。)

(赵云\hspace{30pt}主公保重。)

(四弟,)孤命休矣!

杀呀。

(赵云\hspace{30pt}杀呀。)

(赵云\hspace{30pt}赵云救驾来迟,死罪呀死罪呀!)

唉,孤虽得脱,诸将奈何~!

(赵云\hspace{30pt}由臣断后。)

前面什么所在?

(赵云\hspace{30pt}乃是白帝城。)

兵撤白帝城。

带马。

\textless{}\!{\bfseries\akai 三叫头}\!\textgreater{}二弟,三弟,唉!兄弟啊({\akai 或}:~贤弟呀)$\cdots{}\cdots{}$({\hwfs 哭介})

(罢!)

(赵云\hspace{30pt}马僮,抬枪带马。)

}
