\newpage
\subsubsection{\large\hei {武家坡~{\small 之}~薛平贵}}
\addcontentsline{toc}{subsection}{\hei 武家坡~{\small 之}~薛平贵}

\hangafter=1                   %2. 设置从第1⾏之后开始悬挂缩进  %
\setlength{\parindent}{0pt}{

{\vspace{3pt}{\centerline{{[}{\hei 第一场}{]}}}\vspace{5pt}}

{({\akai 内})【{\akai 西皮导板}】一马离了西凉界}\footnote{吴焕老师整理的剧本(经刘曾复先生审订)注``汪派此处唱`一马离了三关界({\akai 或}:~三关境)'''。},

\setlength{\hangindent}{56pt}{【{\akai 西皮原板}】不由人一阵阵泪洒胸怀。青的山绿是水花花世界,薛平贵好一似孤雁归来。那王允在朝中官居太宰,他把我贫穷人哪放在心怀。恨魏虎是内亲将我谋害,苦害我薛平贵所为何来。柳林下拴战马武家坡外,}

\setlength{\hangindent}{56pt}{【{\akai 西皮摇板}】见了那众大嫂细问开怀。}

{列位大嫂请了。}

{并非失迷路途,我乃找名问姓的。}

{(提起此人,大大有名,)王丞相之女,薛平贵之妻,王氏宝钏。}

{为何({\akai 或}:~怎么)不凑巧?}

{如今呢?}

{烦劳大嫂转达一声,就说他丈夫(与她)带来万金家书,教她前来接取。}

{有劳了。}

\setlength{\hangindent}{56pt}{【{\akai 西皮原板}】这大嫂去送信【{\footnotesize 转}{\akai 西皮快板}】太也迟慢,武家坡站得我两腿酸。下得坡来用目看,见一位大嫂把菜剜。前影好似王三姐,后影儿又像妻宝钏。本当上前把妻唤,错认了民妻礼不端。

{大嫂请了。({\akai 或}:~大嫂请来见礼。)}

{并非失迷路途,我乃找名问姓的。}

{提起此人,大大有名,就是那王丞相之女,薛平贵之妻,王氏宝钏。}

{非亲。}

{非故。}

{(大嫂有所不知,)我与她丈夫同营吃粮,与她带来万金家书,故而动问。}

{(我那薛大哥言道,书信么,要面交本人。)}

{(原书带回。)}

{请便。}

{这哑迷么,略知一二。}

{远在天边,不能相见。}

{哦,莫非你就是薛大嫂么?}

{哎呀呀,问来问去,问到本人的头上来了。}

{来来来,重见一礼呀。}

{礼多人不怪呀。}

{大嫂请稍待。}

{哎呀且住,想我平贵离家一十八载,不知她光景(到底)如何?}

{嗯,嗯,嗯$\cdots{}\cdots{}$我自有道理!}

\setlength{\hangindent}{56pt}{【{\akai 西皮快板}】洞宾曾把牡丹戏,庄子也曾戏过妻。秋胡戏过了罗敷女,薛平贵调戏自己妻。弓韔袋}\footnote{通常作``弓靫袋''。段公平{\scriptsize 君}注:~{韔(音{\textrm ch\`ang}):~弓袋,如《秦风$\cdot$小戎》``虎韔镂膺'',``交韔二弓''。亦谓将弓放入弓袋,如《小雅$\cdot$采绿》``之子于狩,言韔其弓''。}}{中摸一把,}

{哎呀!}

\setlength{\hangindent}{56pt}{【{\akai 西皮快板}】我把大嫂的书信失。}

{失落了。}

{弓韔袋中。}

{正是紧要的所在啊。}

{呃,呃,呃$\cdots{}\cdots{}$想是我前村抽弓打雁------}

{打雁充饥呀。}

{诶------一封书信,能值几何,你怎么开口伤人({\akai 或}:~出口伤人)呐?}

{哎呀呀,到底是丞相之女,出口便是文呐({\akai 或}:~出口成文)。}

{啊大嫂,你不要着急呀,这书信上的言语,呃,我还记得几句。}

{明白何来?}

{诶,(不是哟,)私看人家的书信是有罪名的呀。}

{呃,我那薛大哥修书的时节,我在一旁打点行李,我偷看了几句,倒是有的。}

{我若有心呐,还不失落你的书信呢。}

{呵呵哈哈哈$\cdots{}\cdots{}$({\hwfs 笑介})}

\setlength{\hangindent}{56pt}{【{\akai 西皮导板}】八月十五月光明呐,}

{军营中苦得很呐,哪有许多灯火。}\footnote{吴焕老师整理的剧本注:~``谭派没有`皓月当空'的词句。''}

\setlength{\hangindent}{56pt}{【{\akai 西皮原板}】薛大哥在月下修书文呐。}

{\setlength{\hangindent}{56pt}{(王宝钏\hspace{20pt}【{\akai 西皮原板}】我问他好来,)} }

\setlength{\hangindent}{56pt}{【{\footnotesize 接}{\akai 西皮原板}】他倒好,}

{\setlength{\hangindent}{56pt}{(王宝钏\hspace{20pt}【{\akai 西皮原板}】再问他安宁,)} }

\setlength{\hangindent}{56pt}{【{\footnotesize 接}{\akai 西皮原板}】倒也安宁。}

{\setlength{\hangindent}{56pt}{(王宝钏\hspace{20pt}【{\akai 西皮原板}】三餐茶饭,)} }

\setlength{\hangindent}{56pt}{【{\footnotesize 接}{\akai 西皮原板}】小军造,}

{\setlength{\hangindent}{56pt}{(王宝钏\hspace{20pt}【{\akai 西皮原板}】衣服破了)} }

\setlength{\hangindent}{56pt}{【{\footnotesize 接}{\akai 西皮原板}】自己补缝。}

\setlength{\hangindent}{56pt}{【{\akai 西皮原板}】薛大哥这几年运不通,在西凉军营中受了酷刑}\footnote{吴焕老师整理的剧本记作``受了苦刑''。}{。}

{呃,(不错)正是挨了打呀。}

{一捆四十。}

{大嫂不要啼哭,这苦哇------}

{还在后头呢。}

\setlength{\hangindent}{56pt}{【{\akai 西皮原板}\footnote{吴焕老师整理的剧本注:~``此句汪派唱【{\akai 西皮快板}】''。}{】在营中失落了一骑马,}

{自然是官马呀。}

{哼,哪怕他不赔。}

{自然有哇------}

\setlength{\hangindent}{56pt}{【{\akai 西皮原板}】为赔马借了我十两纹银。}

{一份。}

{也是一份。}

{大嫂你有所不知呀,我那薛大哥啊,原先么,本是个好人呐。}

{后来他学坏了。交了些无业的游民,吃喝嫖赌({\akai 或}:~浪荡逍遥),呃,无所不为,把一份钱粮俱都花费。不怕大嫂你笑话,为军的我乃是个贫寒出身呐,从来不晓得什么叫作花钱(呐),积攒下几两银子,都借与他赔马了。}

{怎么不对呢?}

{哦,我那薛大哥也是个贫寒出身?}

{哎呀呀,薛大哥呀薛大哥,我今日才晓得你也是个贫寒出身呐。}

{呵呵哈哈哈$\cdots{}\cdots{}$({\hwfs 笑}{\hwfs 介})}

\setlength{\hangindent}{56pt}{【{\akai 西皮原板}】本利算来二十两,并不曾还我半毫分。}

{无有也是枉然。}

{(岂不伤了朋友的和气?)}

{防身宝剑,你问它则甚?}

{诶,清平世界({\akai 或}:~青天白日),朗朗乾坤,杀人(岂不)是要偿命的呀。}

{唉,有道是:~善财难舍呀。}

\setlength{\hangindent}{56pt}{【{\akai 西皮原板}】那一日过营去将账讨,他言说长安城有一个王氏宝钏。}

{不该。}

{不欠。}

{大嫂,(我来问你,)有道是:~这父债------}

{夫债呢?}

{妻,妻$\cdots{}\cdots{}$妻怎么样?}

{呵呵,你倒推得个干净呐。}

{(呃,)有道是:~这汗呐,要出在病人的身上哦。}

\setlength{\hangindent}{56pt}{【{\akai 西皮原板}】他无钱便把妻来卖,将大嫂卖与了当军的人呐。}

{喏喏喏,就是在下。({\akai 或}:~不才,在下。)}

{呃呃呃,呃,我有婚书为证呐!}

呃呃,你慢来慢来,我看大嫂变脸变色,婚书诓至手中,三把五把扯碎,为军的岂不落一个人财两空么?

呃,你我去至前村,大户人家,请上三老四少,同拆同观。

当真。

哪个骗你呀?

{呵呵,她倒骂起来了。}

{\setlength{\hangindent}{56pt}{(王宝钏\hspace{20pt}【{\akai 西皮二六}】$\cdots{}\cdots{}$主婚的人呐。)} }

\setlength{\hangindent}{56pt}{【{\akai 西皮快板}】苏龙魏虎为媒证,那王丞相是我的主婚人呐。}

\setlength{\hangindent}{56pt}{【{\akai 西皮快板}】他三人与我有仇恨,咬定牙关就不认承。}

\setlength{\hangindent}{56pt}{【{\akai 西皮快板}】西凉川四十单八站,为军的要人不要钱。}

\setlength{\hangindent}{56pt}{【{\akai 西皮快板}】大嫂休得巧言辩,为军哪怕到官前?衙里衙外我打点,管保大嫂断与咱。}

\setlength{\hangindent}{56pt}{【{\akai 西皮快板}】好一个贞节王宝钏,百般调戏也枉然。(自古道青酒红人面,动人心,财帛金银钱。)(在)腰中取出银一锭,将银放在地平川。这锭银(子)三两三,送与大嫂作养奁({\akai 或}:~做妆奁)。买绫罗、做衣衫,打首饰、置簪环,做一个少年的夫妻就过几年呐。}

\setlength{\hangindent}{56pt}{【{\akai 西皮快板}】是烈女就该在闺房,缘何来在大路旁。为军的起下不良意,}

\setlength{\hangindent}{56pt}{【{\akai 西皮摇板}】来来来一马双跨到西凉呃。}

{走走走,上马呀。}

{在哪里?}

{诶呀!}

{呵呵哈哈哈$\cdots{}\cdots{}$({\hwfs 笑介})}

\setlength{\hangindent}{56pt}{【{\akai 西皮摇板}】一见宝钏回窑转,果然为我受熬煎({\akai 或}:~一十八载受熬煎)。不上马来步下赶,回到窑中两团圆。}

{\vspace{3pt}{\centerline{{[}{\hei 第二场}{]}}}\vspace{5pt}}

\setlength{\hangindent}{56pt}{【{\akai 西皮摇板}】后面跟随平贵男。}

\setlength{\hangindent}{56pt}{【{\akai 西皮摇板}】将为丈夫关至在(这)窑外边。}

{妻呀!}

\setlength{\hangindent}{56pt}{【{\akai 西皮导板}】想起当年泪不干呐,}

\setlength{\hangindent}{56pt}{【{\akai 西皮原板}】夫妻们在寒窑受尽了熬煎。自从我降了红鬃马,唐主爷驾前去讨官。官封我后军都督府哇,你的父上殿把本参。自从盘古【{\footnotesize 转}{\akai 西皮快板}】立地天,哪有岳父把婿参。西凉国,造了反,薛平贵倒做了先行官。两军阵前遇代战,将我擒过了马雕鞍。多蒙老王不肯斩,反把公主配良缘。西凉的老王把驾晏,(众)文武保我坐银安。那一日驾坐银安殿,宾鸿大雁口吐人言。手执金弓银弹打,打下了半幅血罗衫。展开罗衫从头看,才知道寒窑受苦的王宝钏。不分昼夜往前趱,为的是回家夫妻团圆。三姐不信从头算,连来带呃去十八年。}

\setlength{\hangindent}{56pt}{【{\akai 西皮摇板}】水流千遭归大海,原物交还本人观。}

\setlength{\hangindent}{56pt}{【{\akai 西皮摇板}】少年子弟江湖老,红粉佳人两鬓斑。三姐不信菱花看,容颜不似当年在彩楼前。}

{水盆里面。}

{话已说明,开门相见才是。({\akai 或}:~话已说明,快快开门,夫妻相见。)}

{哦,退一步。}

{哦,又退了一步。({\akai 或}:~哦,再退后一步。)}

{唉呀妻呀,后面无有路了。}

{唉!}

\setlength{\hangindent}{56pt}{【{\akai 西皮摇板}】三姐不必寻短见,为丈夫跪在地平川。}

{\setlength{\hangindent}{56pt}{(王宝钏\hspace{20pt}【{\akai 西皮摇板}】$\cdots{}\cdots{}$什么官。)} }

{进得窑来,不问饥寒,开口便是官。难道说还吃官、穿官不成么?}

{(呃,)我临行的时节({\akai 或}:~临行之时),也曾(与你)留下安家度用啊。}

{十担干柴,八斗老米。}

{就该去借。}

{相府去借呀。}

{哦,你不曾去过相府({\akai 或}:~你不曾进过相府)?}

{呵({\akai 或}:~好),有志气。告便。}

{去至相府,与你那爹爹算这一十八载的老米账啊。}

{哦,他病了?得何病症呐({\akai 或}:~他得的什么病症呐)?}

{呵呵({\akai 或}:~哦),他见不得我?}

{难道说,我还见不得他?}

{有朝一日,孤王({\akai 或}:~我)得了唐室天下,他与我牵马坠镫,(呃,)我还嫌他老呢。}

{不曾睡着。}

{(句句实言。)}

{有道是:~龙行有宝。}

{无宝呢?}

{三姐观宝。}

\setlength{\hangindent}{56pt}{【{\akai 西皮快板}】在头上整整沿毡帽,避尘珠金光照满窑。用手取出番王宝,三姐拿去仔细呀瞧。}

{下跪何人?}

{跪在孤王({\akai 或}:~我)的面前则甚呐?}

{(王宝钏\hspace{20pt}讨封。)}

{方才你在武家坡前骂得我好苦。(呃,)我是不能封的了啊。}

{若是知道,必然是不骂的了啊。({\akai 或}:~哦,倘若知道是我呢?)}

{呃,越发的不封了。}

{(当真不封。)}

{(果然不封。)}

{呃,慢来慢来,焉有不封之理。}

{三姐听封------}

\setlength{\hangindent}{56pt}{【{\akai 西皮快板}】非是孤不把你来封,有一个缘故在其中({\akai 或}:~在内中)。西凉国有个代------}

\setlength{\hangindent}{56pt}{【{\akai 西皮快板}】西凉国有一个({\akai 或}:~西凉川有个)代战女,她保孤王立大功。}

{\setlength{\hangindent}{56pt}{(王宝钏\hspace{20pt}【{\akai 西皮快板}】$\cdots{}\cdots{}$,她为正来我为偏。)} }

\setlength{\hangindent}{56pt}{【{\akai 西皮快板}】讲什么正来论什么偏,你我结发在她前({\akai 或}:~在她先)。有朝一日登宝殿({\akai 或}:~登龙殿),封你昭阳掌正权。}

\setlength{\hangindent}{56pt}{【{\akai 西皮摇板}】平贵离家十八年。}

{\setlength{\hangindent}{56pt}{(王宝钏\hspace{20pt}【{\akai 西皮摇板}】$\cdots{}\cdots{}$王宝钏。)} }

\setlength{\hangindent}{56pt}{【{\akai 西皮摇板}】今日夫妻重相见。}

{\setlength{\hangindent}{56pt}{(王宝钏\hspace{20pt}【{\akai 西皮摇板}】$\cdots{}\cdots{}$在梦间。)} }

三姐,你看红日当头,不是做梦啊。

(不是做梦。)

三姐。

来了。

呵呵哈哈哈$\cdots{}\cdots{}$({\hwfs 笑介})

}
