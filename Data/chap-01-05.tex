\newpage
\subsubsection{\large\hei 焚绵山~\protect\footnote{《京剧汇编》第七十四集作``焚棉山''。剧中介子推的词句部分参考了吴焕老师记录的刘曾复先生说戏录音文稿。}{\small 之}介子推~\protect\footnote{介子推亦作介之推。}、介母、解张}
\addcontentsline{toc}{subsection}{\hei 渭水河}

\hangafter=1                   %2. 设置从第1⾏之后开始悬挂缩进  %
\setlength{\parindent}{0pt}{
\vspace{3pt}{\centerline{{[}{\hei 第一场}{]}}}\vspace{5pt}

介子推\hspace{20pt}{[}{\akai 引子}{]}弃官离朝({\akai 或}: 弃职离朝),名利抛,侍奉年高。

介子推\hspace{20pt}({\akai 念})见机逃出是非地,落得清闲自在身。冷眼识破君王意,功成身退奉{\footnotesize 呃}娘亲。
 
\setlength{\hangindent}{56pt}{介子推\hspace{20pt} 卑人,介(子)推。晋献公驾前为臣,官居谏议大夫(之职)。只因吾主({\akai 或}: 只因君王无道,)听信谗言,毒害大臣,害死申生太子。是我与狐毛、狐偃、颠颉、魏犨等九人,保定(公子)重耳逃出罗网,周游列国一十九载。归来共渡黄河,是俺看破其意({\akai 或}: 是我识破其意),弃职归林。正是:~({\akai 念})奉君何足乐,还是孝{\footnotesize 呃}当先。}

\setlength{\hangindent}{56pt}{
介子推\hspace{20pt}【{\akai 西皮慢板}】介子推坐草堂前思后想,想起了晋国事好不凄凉。晋献公听信那谗言毁谤,宠骊姬害申生命赴黄粱。我十人保重耳呐 }【{\footnotesize 转}{\akai 西皮原板}】逃出罗网,朝同行夜同寝({\akai 或}: 夜共寝)伴随君旁。我也曾在荒郊觅食取浆,我也曾割股肉奉献君王。渡黄河一个个争呃功邀赏啊,又谁知出恶言暗把人伤。既这等做什么忠呃臣良将,因此上怀戒心弃职还乡。回家来织呐履舄\footnote{ 古代单底鞋称履,复底鞋称舄,故以``履舄''泛称鞋。}心宽【{\akai 回龙}】意爽,

介子推\hspace{20pt}【{\akai 西皮摇板}】斩断了名利锁侍奉老娘。

解张\hspace{30pt}【{\akai 西皮摇板}】晋公子登龙位各官封赠,相劝那介子推前去面君。

解张\hspace{30pt}来此已是,介兄在家么?

介子推\hspace{20pt}是哪位?

介子推\hspace{20pt}哦,原来是解兄,请进!

解张\hspace{30pt}请,这厢有礼。

介子推\hspace{20pt}还礼,请坐!

解张\hspace{30pt}有座。

介子推\hspace{20pt}解兄到此({\akai 或}: 来此)何事?

解张\hspace{30pt}介兄有所不知,今有重耳公子回朝犒赏功臣。你乃有功之臣,何不前去请功受赏?

介子推\hspace{20pt}解兄这些言语,依我看来,尽都是荒谬之言({\akai 或}: 依弟看来,尽是些荒谬之言)。

解张\hspace{30pt}何谓荒谬之言?

\setlength{\hangindent}{56pt}{
	介子推\hspace{20pt}既是公子重耳回朝({\akai 或}: 还朝复位),犒赏功臣,就该有旨前来,召我入朝({\akai 或}: 还朝)。今无旨意到此({\akai 或}: 今无圣命),岂不是({\akai 或}: 岂非)荒谬之言?}

解张\hspace{30pt}介兄啊!

\setlength{\hangindent}{56pt}{ 
解张\hspace{30pt}【{\akai 西皮原板}】莫道榜文是虚谎,老汉言来听端详: 昔日有个姜吕望,八十三岁遇文王。他也曾保主江山创,他也曾领兵去伐商。既是重耳加封赏,就该前去见君王。 }

(介子推\hspace{20pt}解兄。)

\setlength{\hangindent}{56pt}{
介子推\hspace{20pt}【{\akai 西皮原板}】解兄不必说比方,弟今言来听心旁: 讲什么兴周姜吕望,讲什么伐纣周武王。我不啊为官身闲荡,散淡逍遥侍奉高堂({\akai 或}: 侍奉萱堂)。 }

\setlength{\hangindent}{56pt}{ 
解张\hspace{30pt}【{\akai 西皮原板}】介兄不必【{\footnotesize 转}{\akai 西皮二六}】性刚强,你本盖世一忠良。周游列国随驾往,割股之功天下扬。}

\setlength{\hangindent}{56pt}{介子推\hspace{20pt}【{\akai 西皮二六}】你道为官把名扬,哪知为官无下场。一十九载【{\footnotesize 转}{\akai 西皮快板}】远飘荡,鞍前马后伴君王。重耳为君无度量,弃旧迎新理不当。既是有功该受赏,三冬的梅花自然香。冷眼识破君行状,不在山林伴虎狼。蟒袍玉带我不想,侍奉萱堂在故乡。}

\setlength{\hangindent}{56pt}{解张\hspace{30pt}【{\akai 西皮摇板}】介兄生来秉性刚,必定不肯入朝堂。施罢一礼出草堂, }

解张\hspace{30pt}告辞了。

介子推\hspace{20pt}少送。

解张\hspace{30pt}【{\akai 西皮摇板}】休怪老汉语癫狂。 

解张\hspace{30pt}哈哈哈$\cdots{}\cdots{}$({\hwfs 笑}{\hwfs 介})

\setlength{\hangindent}{56pt}{介子推\hspace{20pt}【{\akai 西皮摇板}】好一个邻舍老解张({\akai 或}: 好一个仁义老解张),絮絮滔滔语言长({\akai 或}: 话绵长)。他劝我入朝请封赏,怎知({\akai 或}: 哪知)我不愿奉君王({\akai 或}: 伴君王)。闷恹恹且坐草堂上, }

介母\hspace{30pt} 【{\akai 西皮摇板}】母子寂寞苦度时光。 

介子推\hspace{20pt}孩儿拜揖!  

介母\hspace{30pt}罢了,一旁坐下!

介子推\hspace{20pt}谢母亲!({\akai 或}: 谢座。)

介母\hspace{30pt}儿啊,方才何人到此?

介子推\hspace{20pt}邻舍解张到此。

介母\hspace{30pt}到此何事?

介子推\hspace{20pt}(是)他言道,今有公子重耳回朝({\akai 或}: 还朝复位),犒赏功臣,(他)教孩儿前去请封受赏。

介母\hspace{30pt}哦,既然如此,我儿就该前去才是。

\setlength{\hangindent}{56pt}{
	介子推\hspace{20pt}母亲,重耳既然犒赏功臣,就该有旨召我入朝({\akai 或}: 就该有旨前来,将孩儿召回朝去)。今无圣旨到来({\akai 或}: 今无圣命),岂不是把孩儿看成朽木一般了么({\akai 或}: 岂非将儿看作朽木一般)?}

\setlength{\hangindent}{56pt}{介子推\hspace{20pt}【{\akai 西皮原板}】晋君复位坐朝堂,犒赏功臣举栋梁。我昔年有功他全忘,做一个({\akai 或}: 落一个)清闲自在行孝儿郎。 }

\setlength{\hangindent}{56pt}{介母\hspace{30pt}【{\akai 西皮原板}】我的儿说话欠思量,且听为娘说端详:~既是重耳论功赏,我的儿前去又有何妨。 }

(介子推\hspace{20pt} 母亲!)

\setlength{\hangindent}{56pt}{介子推\hspace{20pt}【{\akai 西皮二六}】四四方方一垛墙,许多的迷人内中藏。有人跳出是非网,才能得蓬莱不老方。 }

\setlength{\hangindent}{56pt}{介母\hspace{30pt}【{\akai 西皮摇板}】子推生来性情刚,执意不肯回朝廊。晋君回朝行封赏,还是回朝讨风光。 }

\setlength{\hangindent}{56pt}{介子推\hspace{20pt}【{\akai 西皮摇板}】古来({\akai 或}: 自古)多少忠良将,哪个忠良有下场: 比干谏奏({\akai 或}: 比干丞相)把命丧,微子见机先逃亡哇。越思越想心火上,儿誓死({\akai 或}: 儿至死)不愿回朝廊({\akai 或}: 回朝堂)。 }

\setlength{\hangindent}{56pt}{介母\hspace{30pt}【{\akai 西皮摇板}】我的儿不愿去请功受赏,母子们坐草堂细作商量。 }

介母\hspace{30pt}我儿不去请功受赏,唉,也罢,我母子就该隐居起来才是。

介子推\hspace{20pt}啊母亲,此处有一绵山,高山峻岭,倒可安身。

介母\hspace{30pt}待为娘收拾包裹衣服。就此前往。

介子推\hspace{20pt}(唉!)好个贤德老母!

\setlength{\hangindent}{56pt}{介子推\hspace{20pt}【{\akai 西皮摇板}】老母贤德世无双({\akai 或}: 老母年迈六旬上),助儿埋名实贤良\footnote{ 此句吴焕老师整理的剧本记作``故而埋名是贤良''。}。脱衣去巾\footnote{ 吴焕老师整理的剧本记作``脱衣去襟''。}呐草堂上,老母到此({\akai 或}: 老母到来)离村庄。 }

\setlength{\hangindent}{56pt}{介母\hspace{30pt}【{\akai 西皮摇板}】收拾包裹与行囊,母子一同逃外乡。母子们双双跪草堂,祖先爷呀,拜别祖先泪汪汪。 }

\setlength{\hangindent}{56pt}{介母\hspace{30pt}【{\akai 西皮摇板}】用手拨开名利网, }

\setlength{\hangindent}{56pt}{介子推\hspace{20pt}【{\akai 西皮摇板}】翻身跳出是非墙。 }

介母\hspace{30pt}儿啊,咱们的家园$\cdots{}\cdots{}$

介子推\hspace{20pt}唉!不要了。

介母\hspace{30pt}唉,喂呀$\cdots{}\cdots{}$({\hwfs 哭}{\hwfs 介})

\vspace{3pt}{\centerline{{[}{\hei 第二场}{]}}}\vspace{5pt}

\setlength{\hangindent}{56pt}{介母\hspace{30pt}【{\akai 西皮摇板}】这几年奔走在天涯, }

\setlength{\hangindent}{56pt}{介子推\hspace{20pt}【{\akai 西皮摇板}】撇却({\akai 或}: 不贪)富贵与荣华。 }

介子推\hspace{20pt}母亲,来此已是绵山。

介母\hspace{30pt}高山峻岭,叫为娘怎样上去?

介子推\hspace{20pt}老娘!

\setlength{\hangindent}{56pt}{介子推\hspace{20pt}【{\akai 西皮二六}】绵山峻险多峰岬\footnote{ 吴焕老师整理的剧本记作``峰峡''。},四壁巍峨景物呃佳。云环峻岭({\akai 或}: 云环翠岭)雁难下,那涧下(的)清泉照眼花。猿猴、麋鹿衔枝耍,喜鹊依枝({\akai 或}: 喜鹊争鸣)叫喧哗。手攀藤条上山崖, }

\setlength{\hangindent}{56pt}{介子推\hspace{20pt}【{\akai 西皮摇板}】古树森森({\akai 或}: 古树幽森)可为家。 }

介子推\hspace{20pt}此处可好安身?

介母\hspace{30pt}唉,正好安身,只是难以度日呀!

介子推\hspace{20pt}老娘!

\setlength{\hangindent}{56pt}{介子推\hspace{20pt}【{\akai 西皮二六}】老娘亲休得要来嗟呀,草衣木食度年华。无是无非多潇洒, }

\setlength{\hangindent}{56pt}{介子推\hspace{20pt}【{\akai 西皮摇板}】胜似蓬莱第一家。 }

\setlength{\hangindent}{56pt}{介母\hspace{30pt}【{\akai 西皮摇板}】我的儿说的是哪里话,瓜果怎能度日华。母子且把山岗下, }

\setlength{\hangindent}{56pt}{介子推\hspace{20pt}【{\akai 西皮摇板}】我把那名利({\akai 或}: 我把那功劳)二字哇付与尘沙({\akai 或}: 付与流沙)。 }

\vspace{3pt}{\centerline{{[}{\hei 第三场}{]}}}\vspace{5pt}

介子推\hspace{20pt}({\akai 内})【{\akai 西皮导板}】春草青青隐翠微呀,

\setlength{\hangindent}{56pt}{介子推\hspace{20pt}【{\akai 西皮原板}】老母叮咛结草衣\footnote{ 《京剧汇编》第七十四集作``结草依''。}。山高也有长流水呀,杜鹃不住花前啼。晋重耳归国登龙位,割股功劳({\akai 或}: 割股之功)全不提。劝世人莫贪名和利,朝东暮西却为谁({\akai 或}: 为了谁)。纵然是争得呀三公位,难免荒郊坟土堆。我好比鱼儿惊钩起,我好比杨花信风吹({\akai 或}: 随风吹)。我好比孤凤丹山立\footnote{ 吴焕老师整理的剧本记作``孤凤单山立''。},我好比鸿雁隐山栖({\akai 或}: 飞雁隐山栖)。 }

\setlength{\hangindent}{56pt}{介子推\hspace{20pt}【{\akai 西皮散板}】霎时遍地({\akai 或}: 霎时一阵)风沙起,雀鸟不住往空飞\footnote{ 作``望空飞''似亦通。}。 }

\setlength{\hangindent}{56pt}{介子推\hspace{20pt}【{\akai 西皮散板}】金鼓呐喊听耳底,教人心中费猜疑({\akai 或}: 起猜疑)。 }

\setlength{\hangindent}{56pt}{介子推\hspace{20pt}【{\akai 西皮散板}】站立山头用目觑呀:  }

\setlength{\hangindent}{56pt}{介子推\hspace{20pt}【{\akai 西皮快板}】满山人马似云飞\footnote{ 吴焕老师整理的剧本记作``似影飞''。}。五色旌旗空中立,刀枪剑戟摆得齐。见几个头戴双凤翅,见几个身穿衮龙衣。见几个怀抱双环镋\footnote{ 吴焕老师整理的剧本记作``双环档''。},见几个怀抱打将锤。莫不是哪国烟尘起,莫不是重耳把兵提。莫不是来把绵山洗\footnote{ 吴焕老师整理的剧本记作``来把绵山袭''。},莫不是来访介子推。越思越想心火起呀, }
\setlength{\hangindent}{56pt}{介子推\hspace{20pt}【{\akai 西皮快板}】一腔怒气往上提。我也曾对天发宏誓,永不还朝挂紫衣。任你搜来任你洗,稳坐绵山永不离。 }

\vspace{3pt}{\centerline{{[}{\hei 第四场}{]}}}\vspace{5pt}

\setlength{\hangindent}{56pt}{介子推\hspace{20pt}【{\akai 西皮散板}】四下人马齐围困,重耳带兵搜山林。回头便把母亲请({\akai 或}: 忙把老娘请), }

\setlength{\hangindent}{56pt}{介母\hspace{30pt}【{\akai 西皮散板}】我儿为何着了惊({\akai 或}: $\cdots{}\cdots{}$为何情)。 }

\setlength{\hangindent}{56pt}{介子推\hspace{20pt}【{\akai 西皮散板}】老母({\akai 或}: 老娘)有所不知情,重耳入山将儿寻。 }

\setlength{\hangindent}{56pt}{介母\hspace{30pt}【{\akai 西皮散板}】既是重耳把你请,我儿就该去见君。 }

\setlength{\hangindent}{56pt}{介子推\hspace{20pt}【{\akai 西皮散板}】母亲说话欠思忖,孩儿立誓不回程。({\akai 或}: 曾对苍天发誓盟,至死也不转回程。)哪怕人马重重紧,教儿下山万不能({\akai 或}: 想儿下山万不能)。 }

介母\hspace{30pt}儿往哪里安身?

介子推\hspace{20pt}随儿来啊!

\vspace{3pt}{\centerline{{[}{\hei 第五场}{]}}}\vspace{5pt}


\setlength{\hangindent}{56pt}{介子推\hspace{20pt}【{\akai 西皮散板}】搀定老娘东山进({\akai 或}: 东山隐),隐姓埋名谁知情。 }

\setlength{\hangindent}{56pt}{介子推\hspace{20pt}【{\akai 西皮散板}】东山人马乱纷纭,母子无处把身存呐。 }

\vspace{3pt}{\centerline{{[}{\hei 第六场}{]}}}\vspace{5pt}

\setlength{\hangindent}{56pt}{介子推\hspace{20pt}【{\akai 西皮散板}】搀定老娘西山进({\akai 或}: 西山隐),西山里面({\akai 或}: 西山以内)躲朝廷。 }

\setlength{\hangindent}{56pt}{介子推\hspace{20pt}【{\akai 西皮散板}】西山人马似麻林,倒教子推无计行。 }

介母\hspace{30pt}为娘不耐烦了。

\setlength{\hangindent}{56pt}{介子推\hspace{20pt}【{\akai 西皮导板}】劝老娘要耐烦随儿投奔({\akai 或}: 随儿逃奔)。 }

\vspace{3pt}{\centerline{{[}{\hei 第七场}{]}}}\vspace{5pt}

介子推\hspace{20pt}哎呀!

\setlength{\hangindent}{56pt}{介子推\hspace{20pt}【{\akai 西皮散板}】只见四下烈火升呐,重耳放火烧山林。回头再呀把({\akai 或}: 回头忙把;急忙再把)老娘请, }

\setlength{\hangindent}{56pt}{介母\hspace{30pt}【{\akai 西皮散板}】我儿着急为何情。 }

\setlength{\hangindent}{56pt}{介子推\hspace{20pt}【{\akai 西皮散板}】重耳做事心太狠,不该举火绵山焚。 }

\setlength{\hangindent}{56pt}{介母\hspace{30pt}【{\akai 西皮散板}】重耳放火烧山林,快背为娘去见君。 }

\setlength{\hangindent}{56pt}{介子推\hspace{20pt}【{\akai 西皮散板}】任把绵山火焚尽({\akai 或}: 俱焚尽),情愿一死不回程({\akai 或}: 至死也不去见君)。 }

介母\hspace{30pt}哎呀!

\setlength{\hangindent}{56pt}{介母\hspace{30pt}【{\akai 西皮散板}】绵山好比酆都城,要想活命万不能。 }

\setlength{\hangindent}{56pt}{介子推\hspace{20pt}【{\akai 西皮散板}】搀定老娘东山隐({\akai 或}: 东山临\footnote{ 吴焕老师整理的剧本记作``东山岭'';《京剧汇编》第七十四集作``东山进''。}), }

\setlength{\hangindent}{56pt}{介子推\hspace{20pt}【{\akai 西皮散板}】火光四起吓煞人。 }

\setlength{\hangindent}{56pt}{介子推\hspace{20pt}【{\akai 西皮散板}】搀定老娘西山岭, }

\setlength{\hangindent}{56pt}{介子推\hspace{20pt}【{\akai 西皮散板}】四下烈火难存身。 }

\setlength{\hangindent}{56pt}{介子推\hspace{20pt}【{\akai 西皮散板}】搀定老娘({\akai 或}: 背定老娘)上山呐岭,\footnote{ 陈超老师注: 此时台上摆放横场桌,两把椅子。老生从小边上椅子,老旦滑下来头冲里躺地下,老生上桌发现老旦落山,在桌子上甩发``屁股坐子''甩发盖脸,挡脸,蹬椅子吊毛下桌。} }

\setlength{\hangindent}{56pt}{介子推\hspace{20pt}【{\akai 西皮散板}】儿的老娘啊! }
}
