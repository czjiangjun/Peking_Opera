\newpage
\subsubsection{\large\hei{审头刺汤}~\protect\footnote{根据刘曾复先生钞本整理。}}
\addcontentsline{toc}{subsection}{\hei 审头刺汤}

\hangafter=1                   %2. 设置从第1⾏之后开始悬挂缩进  %
\setlength{\parindent}{0pt}{

\vspace{3pt}{\centerline{{[}{\hei 第一场}{]}}}\vspace{5pt}

\setlength{\hangindent}{56pt}{陆炳\hspace{30pt}仁兄啊!}

\setlength{\hangindent}{56pt}{陆炳\hspace{30pt}【{\akai 四平调}\footnote{刘曾复教授钞本作``二黄平板''。}】在金殿领了万岁命,总理天下冤枉情。最可叹仁兄死得苦,天网恢恢不差毫分。}

\setlength{\hangindent}{56pt}{陆炳\hspace{30pt}老夫陆炳,官居锦衣卫正堂({\akai 或}:~官居锦衣卫大堂之职)。适才朝罢而归,圣上命我审问莫怀古的人头,我想这个人头分明是真,若是问成假的({\akai 或}:~若是断成假的),我那故友死在九泉(之下),也是不能甘心({\akai 或}:~不能瞑目)。思想此事,好不教我为难也!}

\setlength{\hangindent}{56pt}{内\hspace{40pt}汤老爷到!}

\setlength{\hangindent}{56pt}{门子\hspace{30pt}启老爷:~汤老爷到。}

\setlength{\hangindent}{56pt}{陆炳\hspace{30pt}且住,我正在为难之际,听说汤老爷驾到。我想汤勤乃是严府的耳目,他此番前来,我必须要留心在意。来,}

\setlength{\hangindent}{56pt}{门子\hspace{30pt}有。}

\setlength{\hangindent}{56pt}{陆炳\hspace{30pt}传话下去:~说老夫有王命在身,二堂不能叙话,请汤老爷到大堂一叙。}

\setlength{\hangindent}{56pt}{门子\hspace{30pt}是。}

\setlength{\hangindent}{56pt}{陆炳\hspace{30pt}吩咐开门!}

\setlength{\hangindent}{56pt}{(陆炳{\hwfs 下})}

\setlength{\hangindent}{56pt}{门子\hspace{30pt}大人传话下来:~有王命在身,二堂不能叙话,请汤老爷到大堂一叙。}

\setlength{\hangindent}{56pt}{内\hspace{40pt}啊!}

\setlength{\hangindent}{56pt}{门子\hspace{30pt}开门。}

\setlength{\hangindent}{56pt}{(门子{\hwfs 下})}

\vspace{3pt}{\centerline{{[}{\hei 第二场}{]}}}\vspace{5pt}

\setlength{\hangindent}{56pt}{({\hwfs 四红}文堂、刽子手{\hwfs 两边上},{\hwfs 门子}上,陆炳{\hwfs 上},{\hwfs 入座})}

\setlength{\hangindent}{56pt}{陆炳\hspace{30pt}来,有请汤老爷!}

\setlength{\hangindent}{56pt}{门子\hspace{30pt}有请汤老爷。}

\setlength{\hangindent}{56pt}{汤勤\hspace{30pt}({\akai 念})心中只想美佳人({\akai 或}:~心中只为美佳人),费尽三毫七孔心。但愿她心称我意\footnote{刘曾复教授钞本作``趁我意''。},人头是假也是真。}

\setlength{\hangindent}{56pt}{汤勤\hspace{30pt}小官告进。}

\setlength{\hangindent}{56pt}{汤勤\hspace{30pt}小官参见({\akai 或}:~小官叩见)老大人。}

\setlength{\hangindent}{56pt}{陆炳\hspace{30pt}汤老爷过衙来了。}

\setlength{\hangindent}{56pt}{汤勤\hspace{30pt}(是,)小官过衙来了。}

\setlength{\hangindent}{56pt}{陆炳\hspace{30pt}啊?!清早过得衙来,莫非拿老夫的什么弊病来了么?}

\setlength{\hangindent}{56pt}{汤勤\hspace{30pt}小官告退。}

\setlength{\hangindent}{56pt}{陆炳\hspace{30pt}转来。({\akai 或}:~回来。)}

\setlength{\hangindent}{56pt}{汤勤\hspace{30pt}在。}

\setlength{\hangindent}{56pt}{陆炳\hspace{30pt}为何去心太急({\akai 或}:~为何去心忒急)?}

\setlength{\hangindent}{56pt}{汤勤\hspace{30pt}启禀老大人,小官上得堂来一言未发,老大人就说``弊病''二字,小官吃罪不起。}

\setlength{\hangindent}{56pt}{陆炳\hspace{30pt}老夫乃是笑谈。}

\setlength{\hangindent}{56pt}{汤勤\hspace{30pt}哦,老大人乃是笑谈?小官我吃了一大惊呐!}

\setlength{\hangindent}{56pt}{陆炳\hspace{30pt}如此汤老爷请上坐。}

\setlength{\hangindent}{56pt}{汤勤\hspace{30pt}呃,且慢,乃是老大人的法堂,小官不敢坐。}

\setlength{\hangindent}{56pt}{陆炳\hspace{30pt}哦,你也晓得法堂?}

\setlength{\hangindent}{56pt}{汤勤\hspace{30pt}怎么不晓得?}

\setlength{\hangindent}{56pt}{陆炳\hspace{30pt}如此旁设一座。}

\setlength{\hangindent}{56pt}{汤勤\hspace{30pt}多谢老大人。}

\setlength{\hangindent}{56pt}{陆炳\hspace{30pt}呃,往上些,呃,往上些。}

\setlength{\hangindent}{56pt}{汤勤\hspace{30pt}是是是。}

\setlength{\hangindent}{56pt}{陆炳\hspace{30pt}汤老爷,过衙何事?}

\setlength{\hangindent}{56pt}{汤勤\hspace{30pt}奉了严大人之命,前来会审人头。}

\setlength{\hangindent}{56pt}{陆炳\hspace{30pt}汤老爷来得好哇!若是不来,弟具一名帖,请驾到此,会审人头。}

\setlength{\hangindent}{56pt}{汤勤\hspace{30pt}小官不,不敢当啊!}

\setlength{\hangindent}{56pt}{陆炳\hspace{30pt}来,}

\setlength{\hangindent}{56pt}{门子\hspace{30pt}有。}

\setlength{\hangindent}{56pt}{陆炳\hspace{30pt}带人犯。}

\setlength{\hangindent}{56pt}{门子\hspace{30pt}带人犯!}

\setlength{\hangindent}{56pt}{(张龙、郭义、戚继光、雪艳{\hwfs 上})}

\setlength{\hangindent}{56pt}{张龙、郭义、戚继光、雪艳\hspace{10pt}叩见大人。}

\setlength{\hangindent}{56pt}{门子\hspace{30pt}张龙、郭义,}

\setlength{\hangindent}{56pt}{张龙、郭义\hspace{30pt}有。}

\setlength{\hangindent}{56pt}{门子\hspace{30pt}戚继光,}

\setlength{\hangindent}{56pt}{戚继光\hspace{40pt}有。}

\setlength{\hangindent}{56pt}{门子\hspace{30pt}雪艳。}

\setlength{\hangindent}{56pt}{雪艳\hspace{30pt}有。}

\setlength{\hangindent}{56pt}{陆炳\hspace{30pt}戚继光、雪艳下去。}

\setlength{\hangindent}{56pt}{戚继光、雪艳\hspace{30pt}是。}

\setlength{\hangindent}{56pt}{(戚继光、雪艳分下)}

\setlength{\hangindent}{56pt}{陆炳\hspace{30pt}张龙、郭义。}

\setlength{\hangindent}{56pt}{张龙、郭义\hspace{30pt}有。}

\setlength{\hangindent}{56pt}{陆炳\hspace{30pt}(想)莫怀古夫妇,(还)是蓟州的兵丁拿获的,还是尔等拿获的?}

\setlength{\hangindent}{56pt}{张龙、郭义\hspace{30pt}乃是小人们拿获的。}

\setlength{\hangindent}{56pt}{陆炳\hspace{30pt}在哪里拿住的({\akai 或}:~在哪里拿获的)?}

\setlength{\hangindent}{56pt}{张龙、郭义\hspace{30pt}在蓟州西门之外({\akai 或}:~在蓟州西门以外),柳林之下。}

\setlength{\hangindent}{56pt}{陆炳\hspace{30pt}什么时候?}

\setlength{\hangindent}{56pt}{张龙、郭义\hspace{30pt}黄昏时候。}

\setlength{\hangindent}{56pt}{陆炳\hspace{30pt}怎么进城({\akai 或}:~怎样进城)?}

\setlength{\hangindent}{56pt}{张龙、郭义\hspace{30pt}叫开城门,批了劄子,击了戚大人堂鼓,才见得戚大人。}

\setlength{\hangindent}{56pt}{陆炳\hspace{30pt}戚大人怎样吩咐?}

\setlength{\hangindent}{56pt}{张龙、郭义\hspace{30pt}大人言道:~此事大了,要作两家担待({\akai 或}:~要和我等两家担待)。}

\setlength{\hangindent}{56pt}{陆炳\hspace{30pt}何谓``两家担待''?}

\setlength{\hangindent}{56pt}{张龙、郭义\hspace{30pt}``头门以里、仪门以外,有一军牢小房,将我等并锁一处({\akai 或}:~将我等锁在一处),里面有灯有火,外面有封有锁({\akai 或}:~外面有锁有封),锁上加封,等到五鼓天明,看着绑,看着斩,人头打入木桶,回覆严大人。''}

\setlength{\hangindent}{56pt}{陆炳\hspace{30pt}可是真情?}

\setlength{\hangindent}{56pt}{张龙、郭义\hspace{30pt}俱是真情。}

\setlength{\hangindent}{56pt}{陆炳\hspace{30pt}带下去。}

\setlength{\hangindent}{56pt}{门子\hspace{30pt}下去。}

\setlength{\hangindent}{56pt}{陆炳\hspace{30pt}带雪艳。}

\setlength{\hangindent}{56pt}{门子\hspace{30pt}带雪艳。}

\setlength{\hangindent}{56pt}{(雪艳上,跪)}

\setlength{\hangindent}{56pt}{雪艳\hspace{30pt}雪艳叩见大人。}

\setlength{\hangindent}{56pt}{陆炳\hspace{30pt}雪艳。}

\setlength{\hangindent}{56pt}{雪艳\hspace{30pt}有。}

\setlength{\hangindent}{56pt}{陆炳\hspace{30pt}你夫妇({\akai 或}:~你夫妻),还是蓟州的兵丁拿获的,还是严府校尉({\akai 或}:~严府人役)拿获的?}

\setlength{\hangindent}{56pt}{雪艳\hspace{30pt}乃是严府校尉拿获的({\akai 或}:~乃是严府人役拿获的)。}

\setlength{\hangindent}{56pt}{陆炳\hspace{30pt}在哪里拿获的({\akai 或}:~在哪里拿住的)?}

\setlength{\hangindent}{56pt}{雪艳\hspace{30pt}在蓟州西门以外,柳林之下。}

\setlength{\hangindent}{56pt}{陆炳\hspace{30pt}什么时候?}

\setlength{\hangindent}{56pt}{雪艳\hspace{30pt}黄昏时候。}

\setlength{\hangindent}{56pt}{陆炳\hspace{30pt}怎样进城?}

\setlength{\hangindent}{56pt}{雪艳\hspace{30pt}叫开城门,批了劄子,击了戚大人堂鼓,才(得)见戚大人。}

\setlength{\hangindent}{56pt}{陆炳\hspace{30pt}戚大人是怎样吩咐?}

\setlength{\hangindent}{56pt}{雪艳\hspace{30pt}戚大人言道:~此事大了,(必须)要作两家担待。}

\setlength{\hangindent}{56pt}{陆炳\hspace{30pt}何谓``两家担待''?}

\setlength{\hangindent}{56pt}{雪艳\hspace{30pt}头门以里、仪门以外,有一军牢小房,将我等并锁(在)一处,里面有灯有火,外面有封有锁({\akai 或}:~外面有锁有封),锁上加封,等到五鼓天明,看着绑,看着斩,人头打入木桶,严府校尉回复严大人去了。}

\setlength{\hangindent}{56pt}{陆炳\hspace{30pt}可是真情?}

\setlength{\hangindent}{56pt}{雪艳\hspace{30pt}俱是真情。}

\setlength{\hangindent}{56pt}{陆炳\hspace{30pt}带下去。}

\setlength{\hangindent}{56pt}{门子\hspace{30pt}下去。}

\setlength{\hangindent}{56pt}{雪艳\hspace{30pt}是。}

\setlength{\hangindent}{56pt}{(雪艳{\hwfs 下})}

\setlength{\hangindent}{56pt}{陆炳\hspace{30pt}汤老爷,汤老爷。哎!}

\setlength{\hangindent}{56pt}{(汤勤{\hwfs 看}雪艳)}

\setlength{\hangindent}{56pt}{汤勤\hspace{30pt}呃,老大人,老大人。}

\setlength{\hangindent}{56pt}{陆炳\hspace{30pt}呃,呃,我在这里哟。}

\setlength{\hangindent}{56pt}{汤勤\hspace{30pt}哎呀,老大人。}

\setlength{\hangindent}{56pt}{陆炳\hspace{30pt}汤老爷。我想戚继光乃是阃外总兵,他犯法于朝廷,不能犯法于你我,我有意赏他一个矮座,你意下如何?}

\setlength{\hangindent}{56pt}{汤勤\hspace{30pt}但凭老大人。}

\setlength{\hangindent}{56pt}{陆炳\hspace{30pt}来,}

\setlength{\hangindent}{56pt}{门子\hspace{30pt}有。}

\setlength{\hangindent}{56pt}{陆炳\hspace{30pt}带戚继光。}

\setlength{\hangindent}{56pt}{门子\hspace{30pt}带戚继光。}

\setlength{\hangindent}{56pt}{(戚继光{\hwfs 上})}

\setlength{\hangindent}{56pt}{戚继光\hspace{40pt}犯官参见大人。}

\setlength{\hangindent}{56pt}{陆炳\hspace{30pt}汤老爷赏你一矮座,(还不)过去谢谢汤老爷。}

\setlength{\hangindent}{56pt}{戚继光\hspace{40pt}多谢汤老爷。}

\setlength{\hangindent}{56pt}{(汤勤\hspace{30pt}不消。)}

\setlength{\hangindent}{56pt}{陆炳\hspace{30pt}戚继光,莫怀古夫妇,(是)在哪里拿获的?}

\setlength{\hangindent}{56pt}{戚继光\hspace{40pt}严府校尉在蓟州西门以外,柳林之下拿获莫怀古夫妇的。}

\setlength{\hangindent}{56pt}{陆炳\hspace{30pt}什么时候?}

\setlength{\hangindent}{56pt}{戚继光\hspace{40pt}黄昏时候。}

\setlength{\hangindent}{56pt}{陆炳\hspace{30pt}怎样进城?}

\setlength{\hangindent}{56pt}{戚继光\hspace{40pt}叫开城门,劈开劄子({\akai 或}:~批了劄子),击动了犯官的堂鼓,才见犯官。}

\setlength{\hangindent}{56pt}{陆炳\hspace{30pt}你是怎样吩咐?}

\setlength{\hangindent}{56pt}{戚继光\hspace{40pt}犯官(也曾)言道:~此事重大,要作两家担待({\akai 或}:~必须要两家担待)。}

\setlength{\hangindent}{56pt}{陆炳\hspace{30pt}何谓\footnote{刘曾复教授钞本作``何为''。}``两家担待''?}

\setlength{\hangindent}{56pt}{戚继光\hspace{40pt}头门以里、仪门以外,有一军牢小房,将他等并锁一处({\akai 或}:~将他们四人锁在一处),里面有灯有火,外面有封有锁({\akai 或}:~有锁有封),锁上加封,等到五鼓天明,犯官看着绑,看着斩,人头打入木桶,回复严大人。}

\setlength{\hangindent}{56pt}{陆炳\hspace{30pt}可是真情?}

\setlength{\hangindent}{56pt}{戚继光\hspace{40pt}俱是真情。}

\setlength{\hangindent}{56pt}{陆炳\hspace{30pt}带下去。}

\setlength{\hangindent}{56pt}{门子\hspace{30pt}下去。}

\setlength{\hangindent}{56pt}{戚继光\hspace{40pt}是。}

\setlength{\hangindent}{56pt}{(戚继光{\hwfs 下})}

\setlength{\hangindent}{56pt}{陆炳\hspace{30pt}汤老爷,这个人头是真的了。}

\setlength{\hangindent}{56pt}{汤勤\hspace{30pt}呃,怎见得是真的呢?}

\setlength{\hangindent}{56pt}{陆炳\hspace{30pt}他四人的口供俱是一样,怎么不是真的了啊?}

\setlength{\hangindent}{56pt}{汤勤\hspace{30pt}唉呀,老大人,想这个人头,唔,定是假的。}

\setlength{\hangindent}{56pt}{陆炳\hspace{30pt}怎么是假的?}

\setlength{\hangindent}{56pt}{汤勤\hspace{30pt}想他四人一路同行,同宿旅店,串通口供,蒙哄老大人。}

\setlength{\hangindent}{56pt}{陆炳\hspace{30pt}哦,是蒙混老夫的么?}

\setlength{\hangindent}{56pt}{汤勤\hspace{30pt}不错,是蒙混老大人。}

\setlength{\hangindent}{56pt}{陆炳\hspace{30pt}汤老爷,老夫,呃,我有个决断。}

\setlength{\hangindent}{56pt}{汤勤\hspace{30pt}老大人有何高才({\akai 或}:~老大人必有高才。)}

\setlength{\hangindent}{56pt}{陆炳\hspace{30pt}岂敢。想我日前斩了几个人头,不曾示众,我意欲将那几个人头排在堂口({\akai 或}:~摆在堂口),连莫怀古的人头也摆在其内,教那雪艳前去相认({\akai 或}:~上前相认)。她认真便真,认假便假。汤老爷,意下如何?}

\setlength{\hangindent}{56pt}{汤勤\hspace{30pt}呃,乃是老大人的高才。但凭老大人。}

\setlength{\hangindent}{56pt}{陆炳\hspace{30pt}来,听吩咐:~将我日前斩得几个人头,俱都排在堂口({\akai 或}:~俱都摆在堂口),连莫怀古的人头也排在其内,排了起来。}

\setlength{\hangindent}{56pt}{({\hwfs 四}青袍{\hwfs 两边上},{\hwfs 拿人头排在堂口介})}

\setlength{\hangindent}{56pt}{陆炳\hspace{30pt}排齐了?}

\setlength{\hangindent}{56pt}{四青袍\hspace{40pt}排齐了。}

\setlength{\hangindent}{56pt}{陆炳\hspace{30pt}起过。}

\setlength{\hangindent}{56pt}{四青袍\hspace{40pt}是。}

\setlength{\hangindent}{56pt}{陆炳\hspace{30pt}来,带雪艳。}

\setlength{\hangindent}{56pt}{门子\hspace{30pt}带雪艳。}

\setlength{\hangindent}{56pt}{(雪艳{\hwfs 上})}

\setlength{\hangindent}{56pt}{雪艳\hspace{30pt}叩见大人。}

\setlength{\hangindent}{56pt}{陆炳\hspace{30pt}雪艳。}

\setlength{\hangindent}{56pt}{雪艳\hspace{30pt}有。}

\setlength{\hangindent}{56pt}{陆炳\hspace{30pt}我道人头是真,汤老爷说这个人头是假。老夫如今有这个决断。}

\setlength{\hangindent}{56pt}{雪艳\hspace{30pt}大人有何高才?}

\setlength{\hangindent}{56pt}{陆炳\hspace{30pt}想我日前斩了几个人头,不曾示众,排在堂口,连你丈夫的人头也摆在其内,命你前去相认。你认真便真,你认假便假。哪个是你丈夫的人头------抱来见我!}

\setlength{\hangindent}{56pt}{雪艳\hspace{30pt}谢大人!}

\setlength{\hangindent}{56pt}{雪艳\hspace{30pt}【{\akai 二黄散板}】今日领了大人命,堂口之上认夫君。走向前,把夫认,}

\setlength{\hangindent}{56pt}{雪艳\hspace{30pt}喂呀$\cdots{}\cdots{}$({\hwfs 哭介})}

\setlength{\hangindent}{56pt}{雪艳\hspace{30pt}【{\akai 二黄散板}】血淋淋人头认不真。这厢不是那厢认,}

\setlength{\hangindent}{56pt}{雪艳\hspace{30pt}喂呀$\cdots{}\cdots{}$({\hwfs 哭介})}

\setlength{\hangindent}{56pt}{雪艳\hspace{30pt}【{\akai 二黄散板}】手捧人头见大人。}

\setlength{\hangindent}{56pt}{陆炳\hspace{30pt}可是你丈夫的人头?}

\setlength{\hangindent}{56pt}{雪艳\hspace{30pt}(正)是。}

\setlength{\hangindent}{56pt}{陆炳\hspace{30pt}下去。}

\setlength{\hangindent}{56pt}{雪艳\hspace{30pt}是。喂呀$\cdots{}\cdots{}$({\hwfs 哭介})}

\setlength{\hangindent}{56pt}{(雪艳{\hwfs 下},汤勤{\hwfs 看介})}

\setlength{\hangindent}{56pt}{陆炳\hspace{30pt}衙役们,将人头搬下堂口。}

\setlength{\hangindent}{56pt}{四青袍\hspace{40pt}是。}

\setlength{\hangindent}{56pt}{({\hwfs 四}青袍{\hwfs 搬下头介},{\hwfs 下})}

\setlength{\hangindent}{56pt}{陆炳\hspace{30pt}汤老爷。}

\setlength{\hangindent}{56pt}{汤勤\hspace{30pt}呃呃,老大人。}

\setlength{\hangindent}{56pt}{陆炳\hspace{30pt}汤老爷,这个人头可是真的了么?}

\setlength{\hangindent}{56pt}{汤勤\hspace{30pt}怎么,是真的?}

\setlength{\hangindent}{56pt}{陆炳\hspace{30pt}你看那雪艳抱着他丈夫莫怀古的人头痛哭流泪,岂不是真的么?}

\setlength{\hangindent}{56pt}{汤勤\hspace{30pt}哎呀老大人,我把那雪艳好有一比呀。}

\setlength{\hangindent}{56pt}{陆炳\hspace{30pt}比作何来?}

\setlength{\hangindent}{56pt}{汤勤\hspace{30pt}``猫儿哭耗子------''}

\setlength{\hangindent}{56pt}{陆炳\hspace{30pt}此话怎讲?}

\setlength{\hangindent}{56pt}{汤勤\hspace{30pt}假慈悲。}

\setlength{\hangindent}{56pt}{(众手下{\hwfs 哭介})}

\setlength{\hangindent}{56pt}{众\hspace{40pt}喂诶$\cdots{}\cdots{}$({\hwfs 哭介})}

\setlength{\hangindent}{56pt}{陆炳\hspace{30pt}啊,汤老爷,你(来)看这两班人役({\akai 或}:~这两班的衙役),他们都掉起泪来了。}

\setlength{\hangindent}{56pt}{汤勤\hspace{30pt}我把他们也好有一比呀。}

\setlength{\hangindent}{56pt}{陆炳\hspace{30pt}比作何来?}

\setlength{\hangindent}{56pt}{汤勤\hspace{30pt}``看兵书,哼,掉眼泪------''}

\setlength{\hangindent}{56pt}{陆炳\hspace{30pt}此话怎讲?}

\setlength{\hangindent}{56pt}{汤勤\hspace{30pt}替古人担忧哇。}

\setlength{\hangindent}{56pt}{(陆炳\hspace{30pt}怎么讲?)}

\setlength{\hangindent}{56pt}{(汤勤\hspace{30pt}替古人担忧。)}

\setlength{\hangindent}{56pt}{陆炳\hspace{30pt}啊,他们是替古人担忧?}

\setlength{\hangindent}{56pt}{汤勤\hspace{30pt}是。}

\setlength{\hangindent}{56pt}{陆炳\hspace{30pt}汤老爷,你怎么不担忧呢?}

\setlength{\hangindent}{56pt}{汤勤\hspace{30pt}(唉,)老大人说哪里话来,小官与他一不沾亲,二不带故。呃,我担的什么忧啊?}

\setlength{\hangindent}{56pt}{陆炳\hspace{30pt}汤老爷,你苦苦道那莫怀古的人头是假,难道说他的人头还有什么质对\footnote{刘曾复先生钞本作``执对''。}么?}

\setlength{\hangindent}{56pt}{汤勤\hspace{30pt}哼,大大的有质对!}

\setlength{\hangindent}{56pt}{陆炳\hspace{30pt}哦,我今日单单地与你要个质对,与我讲!}

\setlength{\hangindent}{56pt}{汤勤\hspace{30pt}我那莫大老爷前有梅花额\footnote{《京剧汇编》第三十九集~潘侠风~藏本作``梅花痦'',此处从刘曾复先生钞本,下同。},后有三台骨。}

\setlength{\hangindent}{56pt}{陆炳\hspace{30pt}汤老爷,我且问你,想这梅花额,生在面前,你可以常常得见;想这三台骨,长在暗处,你是怎能得见的哟!}

\setlength{\hangindent}{56pt}{汤勤\hspace{30pt}哎呀老大人(你)有所不知呀,小官乃是钱塘人氏,与那莫大老爷乃是同乡。小官不得时的时节,乃是卖字画为生呐。那天小官在陈桥打睡,我那莫大老爷拜客回来({\akai 或}:~拜客而归),将我唤醒,看我的字(是)真、草、隶、篆,看我的画,水墨丹青,将我让到他家,做了一个同窗的幕宾。我与那莫大老爷清晨起来({\akai 或}:~清早起来)同盆净面,同席吃饭,到晚来同榻而眠。这三台骨啊,是常常得见的很哦!}

\setlength{\hangindent}{56pt}{陆炳\hspace{30pt}你虽然是常常得见,呃,你可晓得(当初莫大老爷待你是好是不好啊?)这人死则变!}

\setlength{\hangindent}{56pt}{汤勤\hspace{30pt}呃,老大人,这梅花额烂去了,这三台骨,嗯,是变不了啊。}

\setlength{\hangindent}{56pt}{陆炳\hspace{30pt}哦,他的骨头变不了?!}

\setlength{\hangindent}{56pt}{汤勤\hspace{30pt}骨头变不了。}

\setlength{\hangindent}{56pt}{陆炳\hspace{30pt}汤老爷,那莫大老爷待你如何?}

\setlength{\hangindent}{56pt}{汤勤\hspace{30pt}那莫大老爷待我是恩重如山。}

\setlength{\hangindent}{56pt}{陆炳\hspace{30pt}我看将起来令人可恨!}

\setlength{\hangindent}{56pt}{汤勤\hspace{30pt}老大人,敢莫是恨着小官不成么?}

\setlength{\hangindent}{56pt}{陆炳\hspace{30pt}哎呀呀,我怎敢恨着汤老爷呀。}

\setlength{\hangindent}{56pt}{汤勤\hspace{30pt}老大人,(你)恨着哪个?}

\setlength{\hangindent}{56pt}{陆炳\hspace{30pt}我恨(只恨)那莫大老爷,他虽然是个进士出身,他就大大地失了眼力呀。}

\setlength{\hangindent}{56pt}{汤勤\hspace{30pt}怎么他大大地失了眼力?}

\setlength{\hangindent}{56pt}{陆炳\hspace{30pt}(不是哟,)想汤老爷当初不得第的时节,在钱塘卖画,莫大老爷得中了进士公,在长街拜客而归,看见汤老爷在那里卖画,看你的画,乃是水墨丹青;看你的字(是)真、草、隶、篆,他乃是个读书之人,心中焉能不喜爱字画呀,故而带你进府。他(纵然)带你进府,就(不该)带你进京;(他纵然)带你进京,就(不该)将你荐与严府,严大人这才重用于你。你就做了官,你就得了经历司({\akai 或}:~你就做了经历司)。常言道得好:~({\akai 念})不是渔夫{\akai 引},怎得({\akai 或}:~怎能)见波涛。当初莫大老爷就不该带你进府({\akai 或}:~想当初莫大老爷将你带进府去),(他就)不该将你带进京来;他纵然带将你带进京来,他就不该将你荐与严府。如今也(就)用不着你这个铁板的干证呐!依我看起来,哼,人头是真也是真,是假也是真,就是这样的落案!}

\setlength{\hangindent}{56pt}{汤勤\hspace{30pt}告辞!}

\setlength{\hangindent}{56pt}{陆炳\hspace{30pt}回来。}

\setlength{\hangindent}{56pt}{汤勤\hspace{30pt}在。}

\setlength{\hangindent}{56pt}{陆炳\hspace{30pt}你(往)哪里去?}

\setlength{\hangindent}{56pt}{汤勤\hspace{30pt}回覆严大人。}

\setlength{\hangindent}{56pt}{陆炳\hspace{30pt}你见了严大人,讲些什么?}

\setlength{\hangindent}{56pt}{汤勤\hspace{30pt}我就说陆大人糊里糊涂地就是这样落了案。}

\setlength{\hangindent}{56pt}{陆炳\hspace{30pt}啊?!那严大人是狼?}

\setlength{\hangindent}{56pt}{汤勤\hspace{30pt}嗯,不是狼。}

\setlength{\hangindent}{56pt}{陆炳\hspace{30pt}是虎?}

\setlength{\hangindent}{56pt}{汤勤\hspace{30pt}嗯,不是虎。}

\setlength{\hangindent}{56pt}{陆炳\hspace{30pt}他(又)不是狼,又不是虎,难道他还要吞吃我陆炳不成么?}

\setlength{\hangindent}{56pt}{汤勤\hspace{30pt}吞吃不了老大人。}

\setlength{\hangindent}{56pt}{陆炳\hspace{30pt}哈哈,哈哈,啊呵呵哈哈哈$\cdots{}\cdots{}$(陆炳三笑{\hwfs 介})}

\setlength{\hangindent}{56pt}{(陆炳\hspace{30pt}呵呵呵$\cdots{}\cdots{}$(冷笑{\hwfs 介}))}

\setlength{\hangindent}{56pt}{汤勤\hspace{30pt}呃,老大人为何发笑啊?}

\setlength{\hangindent}{56pt}{陆炳\hspace{30pt}我笑你这两句话癫而又狂,尊而又大。}

\setlength{\hangindent}{56pt}{汤勤\hspace{30pt}小官怎么({\akai 或}:~小官怎见得)癫而又狂,尊而又大?}

\setlength{\hangindent}{56pt}{陆炳\hspace{30pt}我问道汤老爷:~那严府是狼?你讲道他不是狼;我问他是虎?你又讲道他不是虎。汤老爷,慢说他不是狼,不是虎。纵然他是狼是虎,我也有打狼的汉子,还有擒虎的英雄!我陆炳做官,一不欺君,
二不傲上({\akai 或}:~二不枉上),三不贪赃,四不无理以公报私({\akai 或}:~四不无理,以公报公),毫无私弊呀!我做官,做的是嘉靖皇上的官,我没有做(他)严府的官,我又不是(他)严府家人、小子,我又不是他严府使用的奴才({\akai 或}:~严府的使用奴才)!我又怕他何来?!我也曾中过皇榜,我乃二甲进士出身,我是个科举({\akai 或}:~科甲)出身呐!我奉了天子之命,在此审问莫怀古的人头。你不过是奉了严爷({\akai 或}:~你不过是奉了严大人)的一句话呀,过得衙来,照看而已。我与那严大人一殿为臣,不能无礼,这才赏了你一个座位,这叫作``敬其上而爱其下'',才有此举呀。你就该过得衙来,规规矩矩,坐在一旁,耳闻目睹,听其自然,才是你的道理;你不规规矩矩,坐在一旁,反倒飞扬浮躁,信口乱言({\akai 或}:~信口胡言),忽然人头是真,又讲人头是假,真假不定,反复无常,出乎反乎,真乃是个无耻的小人!我又不买你的字画,你倒要做什么啊?({\akai 或}:~我可不买你的字画呀,你到此来则甚么?)哼,好大的一个汤老爷!}

\setlength{\hangindent}{56pt}{陆炳\hspace{30pt}来,将座位与我撤了!}

\setlength{\hangindent}{56pt}{汤勤\hspace{30pt}哎呀呀,你看这个陆大人,虽然年迈,嗯,倒有些傲性。嗯,待我讲上几句好话。}

\setlength{\hangindent}{56pt}{汤勤\hspace{30pt}嗯,呃呵呵呵$\cdots{}\cdots{}$哎呀$\cdots{}\cdots{}$老大人,小官今早吃了几杯糟酒,诶,一言冒犯,诶,老大人,小官与老大人叩头赔礼了。}

\setlength{\hangindent}{56pt}{陆炳\hspace{30pt}诶呵呵$\cdots{}\cdots{}$汤老爷请起,请坐。}

\setlength{\hangindent}{56pt}{汤勤\hspace{30pt}多谢老大人。}

\setlength{\hangindent}{56pt}{陆炳\hspace{30pt}汤老爷,虽然是笑谈,与这命案大有关系({\akai 或}:~与这人头是大有关系)。}

\setlength{\hangindent}{56pt}{汤勤\hspace{30pt}是是是。}

\setlength{\hangindent}{56pt}{陆炳\hspace{30pt}啊汤老爷,你看此案如何发落({\akai 或}:~你有何高见呐?)}

\setlength{\hangindent}{56pt}{汤勤\hspace{30pt}呃呃呃,老大人,有道是:~``抄手问贼贼不招,用棒打犬犬必逃''。不动大刑,料他们不招。}

\setlength{\hangindent}{56pt}{(陆炳\hspace{30pt}哦,不动大刑,料他们不招?)}

\setlength{\hangindent}{56pt}{陆炳\hspace{30pt}汤老爷,(你看)这上------}

\setlength{\hangindent}{56pt}{汤勤\hspace{30pt}青天。}

\setlength{\hangindent}{56pt}{陆炳\hspace{30pt}这下------}

\setlength{\hangindent}{56pt}{汤勤\hspace{30pt}浮土。}

\setlength{\hangindent}{56pt}{陆炳\hspace{30pt}你我为官的------}

\setlength{\hangindent}{56pt}{汤勤\hspace{30pt}全凭``良心''二字。}

\setlength{\hangindent}{56pt}{陆炳\hspace{30pt}若是无有良心呢?}

\setlength{\hangindent}{56pt}{汤勤\hspace{30pt}呃,呃,教天狗吃了他们。}

\setlength{\hangindent}{56pt}{陆炳\hspace{30pt}好个``天狗吃了他们''。想这无有良心的事,别人做得出来,难道我陆炳就做不出来么?}

\setlength{\hangindent}{56pt}{陆炳\hspace{30pt}来,将严府二公差与我带上来!}

\setlength{\hangindent}{56pt}{门子\hspace{30pt}带张龙、郭义。}

\setlength{\hangindent}{56pt}{(张龙、郭义{\hwfs 上})}

\setlength{\hangindent}{56pt}{张龙、郭义\hspace{30pt}叩见大人!}

\setlength{\hangindent}{56pt}{陆炳\hspace{30pt}唗!我把你这大胆的狗才,奉了严大人一张票,就是这样遮天盖地而来,不知耽误人家({\akai 或}:~误了人家)多少好事,误了人家({\akai 或}:~人家耽误)多少功名?我今天({\akai 或}:~我如今)若不打你们几下,又恐怕惯坏了你们的下次!}

\setlength{\hangindent}{56pt}{陆炳\hspace{30pt}来,扯下去打!}

\setlength{\hangindent}{56pt}{汤勤\hspace{30pt}哎呀呵,且慢,且慢。哎呀老大人,他二人打不得。}

\setlength{\hangindent}{56pt}{陆炳\hspace{30pt}难道说我就打不得他们?}

\setlength{\hangindent}{56pt}{汤勤\hspace{30pt}不是哦,他二人乃是({\akai 或}:~他们乃是)牵连在内。}

\setlength{\hangindent}{56pt}{陆炳\hspace{30pt}什么牵连在内,分明(是)与他二人讲情!}

\setlength{\hangindent}{56pt}{汤勤\hspace{30pt}呵,小官不敢,老大人开恩。}

\setlength{\hangindent}{56pt}{陆炳\hspace{30pt}哼!本当要打你们几下,汤老爷讲情,恐怕打了尔的腿,伤了汤老爷的脸面({\akai 或}:~面子)。记打,记责。下去!}

\setlength{\hangindent}{56pt}{张龙、郭义\hspace{30pt}是,多谢大人。}

\setlength{\hangindent}{56pt}{(张龙、郭义{\hwfs 下})}

\setlength{\hangindent}{56pt}{陆炳\hspace{30pt}来,带戚继光!}

\setlength{\hangindent}{56pt}{门子\hspace{30pt}带戚继光。}

\setlength{\hangindent}{56pt}{(戚继光{\hwfs 上})}

\setlength{\hangindent}{56pt}{戚继光\hspace{40pt}犯官参见大人!}

\setlength{\hangindent}{56pt}{陆炳\hspace{30pt}唗!身为八台总兵,斩了个人头不清不白,难免自羞自愧。来,扯下去打!}

\setlength{\hangindent}{56pt}{朝官\hspace{40pt}({\akai 内})黑诏下。}

\setlength{\hangindent}{56pt}{门子\hspace{30pt}启大人:~黑诏下。}

\setlength{\hangindent}{56pt}{陆炳\hspace{30pt}带下去。}

\setlength{\hangindent}{56pt}{门子\hspace{30pt}是。}

\setlength{\hangindent}{56pt}{(戚继光下)}

\setlength{\hangindent}{56pt}{陆炳\hspace{30pt}汤老爷,我正要用刑,耳闻黑诏下。还是审头事大,还是接诏事大呀?}

\setlength{\hangindent}{56pt}{汤勤\hspace{30pt}自然(是)接诏事大。}

\setlength{\hangindent}{56pt}{陆炳\hspace{30pt}(汤老爷)请到下面待茶。}

\setlength{\hangindent}{56pt}{汤勤\hspace{30pt}小官告退。}

\setlength{\hangindent}{56pt}{陆炳\hspace{30pt}请。}

\setlength{\hangindent}{56pt}{汤勤\hspace{30pt}请。}

\setlength{\hangindent}{56pt}{(汤勤{\hwfs 下})}

\setlength{\hangindent}{56pt}{陆炳\hspace{30pt}来,请诏。}

\setlength{\hangindent}{56pt}{({\hwfs 四}青袍、{\hwfs 一}朝官{\hwfs 上})}

\setlength{\hangindent}{56pt}{朝官\hspace{40pt}黑诏到,下跪!}

\setlength{\hangindent}{56pt}{陆炳\hspace{30pt}万岁。}

\setlength{\hangindent}{56pt}{朝官\hspace{40pt}听宣读,诏曰:~``今有圣上发下犯官四名,一十三名江洋大盗,命刑部正堂监斩,刑部正堂染病在床,命锦衣卫陆炳到平则门外监斩。''黑诏读罢,望诏谢恩!}

\setlength{\hangindent}{56pt}{陆炳\hspace{30pt}万万岁!香案供奉!}

\setlength{\hangindent}{56pt}{门子\hspace{30pt}是。}

\setlength{\hangindent}{56pt}{陆炳\hspace{30pt}有劳大人请诏前来,后堂留宴。}

\setlength{\hangindent}{56pt}{朝官\hspace{40pt}有朝命在身({\akai 或}:~有王命在身),不敢久停。告辞!}

\setlength{\hangindent}{56pt}{({\hwfs 四}青袍{\hwfs 带马下})}

\setlength{\hangindent}{56pt}{陆炳\hspace{30pt}送大人。}

\setlength{\hangindent}{56pt}{朝官\hspace{40pt}请------}

\setlength{\hangindent}{56pt}{(朝官{\hwfs 下})}

\setlength{\hangindent}{56pt}{(汤勤{\hwfs 上})}

\setlength{\hangindent}{56pt}{汤勤\hspace{30pt}老大人,方才黑诏到来,但不知为了何事?}

\setlength{\hangindent}{56pt}{陆炳\hspace{30pt}方才黑诏到来,有四员({\akai 或}:~有四名)犯官,一十三名江洋大盗,命刑部监斩,刑部染病在床,命老夫监斩。啊,还是斩头事大,还是审头事大?}

\setlength{\hangindent}{56pt}{汤勤\hspace{30pt}自然是斩头事大。}

\setlength{\hangindent}{56pt}{陆炳\hspace{30pt}王命为尊。}

\setlength{\hangindent}{56pt}{汤勤\hspace{30pt}是是是。}

\setlength{\hangindent}{56pt}{陆炳\hspace{30pt}汤老爷,老夫奉旨斩头,意欲将雪艳吊在西廊,委派汤老爷,审问她的口供,(汤老爷)意下如何?}

\setlength{\hangindent}{56pt}{汤勤\hspace{30pt}哦,小官如何审得?}

\setlength{\hangindent}{56pt}{陆炳\hspace{30pt}诶,你是奉了严大人的委派前来的,怎么审不得。}

\setlength{\hangindent}{56pt}{汤勤\hspace{30pt}哦,是是是。小官当得效劳。}

\setlength{\hangindent}{56pt}{陆炳\hspace{30pt}来,带雪艳。}

\setlength{\hangindent}{56pt}{门子\hspace{30pt}雪艳。}

\setlength{\hangindent}{56pt}{(雪艳{\hwfs 上})}

\setlength{\hangindent}{56pt}{雪艳\hspace{30pt}叩见大人。}

\setlength{\hangindent}{56pt}{陆炳\hspace{30pt}雪艳,老夫奉旨斩头,将你吊在西廊,委派汤老爷审问你的口供,若是翻供,吃罪不起。}

\setlength{\hangindent}{56pt}{(雪艳\hspace{30pt}是。)}

\setlength{\hangindent}{56pt}{陆炳\hspace{30pt}来,将她吊在西廊。}

\setlength{\hangindent}{56pt}{(众{\hwfs 吊}雪艳{\hwfs 介})}

\setlength{\hangindent}{56pt}{(雪艳\hspace{30pt}喂呀$\cdots{}\cdots{}$({\hwfs 哭介}))}

\setlength{\hangindent}{56pt}{陆炳\hspace{30pt}来,带张龙、郭义。}

\setlength{\hangindent}{56pt}{(张龙、郭义{\hwfs 上})}

\setlength{\hangindent}{56pt}{张龙、郭义\hspace{30pt}叩见大人。}

\setlength{\hangindent}{56pt}{陆炳\hspace{30pt}命你二人,看守雪艳。若是卖、放,打折两腿。起过!}

\setlength{\hangindent}{56pt}{张龙、郭义\hspace{30pt}是。}

\setlength{\hangindent}{56pt}{陆炳\hspace{30pt}来,带戚继光。}

\setlength{\hangindent}{56pt}{门子\hspace{30pt}带戚继光。}

\setlength{\hangindent}{56pt}{(戚继光{\hwfs 上})}

\setlength{\hangindent}{56pt}{戚继光\hspace{40pt}犯官叩见大人({\akai 或}:~犯官参见大人)!}

\setlength{\hangindent}{56pt}{陆炳\hspace{30pt}身为八台总兵,斩了(几)个人头,不明不白({\akai 或}:~不清不白)。用小轿一乘,随定我的轿后。待我斩两个人头,与你看上一看。起过!}

\setlength{\hangindent}{56pt}{戚继光\hspace{40pt}是。}

\setlength{\hangindent}{56pt}{陆炳\hspace{30pt}汤老爷,}

\setlength{\hangindent}{56pt}{汤勤\hspace{30pt}老大人。}

\setlength{\hangindent}{56pt}{陆炳\hspace{30pt}这个雪艳可是交与你了!}

\setlength{\hangindent}{56pt}{汤勤\hspace{30pt}哦,小官代劳。}

\setlength{\hangindent}{56pt}{陆炳\hspace{30pt}多多地辛苦了。}

\setlength{\hangindent}{56pt}{汤勤\hspace{30pt}是岂敢,呵,岂敢,。}

\setlength{\hangindent}{56pt}{陆炳\hspace{30pt}来,外厢开道。}

\setlength{\hangindent}{56pt}{(众{\hwfs 倒领由上场门下})}

\setlength{\hangindent}{56pt}{汤勤\hspace{30pt}哎呀妙啊妙哇,想这雪艳一案,(命我代审。)我就代审代审。}

\setlength{\hangindent}{56pt}{张龙、郭义\hspace{30pt}汤老爷,}

\setlength{\hangindent}{56pt}{汤勤\hspace{30pt}哎呀,呃,二位,适才多有受惊了({\akai 或}:~方才多受惊了)。}

\setlength{\hangindent}{56pt}{张龙、郭义\hspace{30pt}多谢汤老爷讲情。}

\setlength{\hangindent}{56pt}{汤勤\hspace{30pt}岂敢岂敢,二位在此则甚?}

\setlength{\hangindent}{56pt}{张龙、郭义\hspace{30pt}我二人奉了陆大人之命,看守雪艳。}

\setlength{\hangindent}{56pt}{汤勤\hspace{30pt}二位歇息去罢。}

\setlength{\hangindent}{56pt}{张龙、郭义\hspace{30pt}哎呀,我二人不敢远离。}

\setlength{\hangindent}{56pt}{汤勤\hspace{30pt}呀呀哎,(这)分明是(跟我)要银子啊。}

\setlength{\hangindent}{56pt}{汤勤\hspace{30pt}哎呀二位,我这里有点小意思,呃,带着吃杯茶(去)罢。}

\setlength{\hangindent}{56pt}{张龙、郭义\hspace{30pt}呃------我二人不敢收呀。}

\setlength{\hangindent}{56pt}{汤勤\hspace{30pt}只管收下,料也无妨\footnote{刘曾复先生钞本作``略无妨碍''。}。}

\setlength{\hangindent}{56pt}{张龙\hspace{40pt}(如此)多谢汤老爷。}

\setlength{\hangindent}{56pt}{郭义\hspace{40pt}哎呀,看来汤老爷是个好人呐。({\akai 或}:~哎呀汤老爷,哼,看起来你是好人呐!)}

\setlength{\hangindent}{56pt}{汤勤\hspace{30pt}哎呀,本来(的)是个好人。}

\setlength{\hangindent}{56pt}{张龙\hspace{40pt}啊伙计,这个人头原本({\akai 或}:~这个人头原来)是真的,也不知哪个坏种、嘎杂子,他(偏偏)说是假的。伙计,你在里面访,我在外面寻({\akai 或}:~我在外面访),访着此人,将他吊在树梢儿上(面),揭开他的天灵盖,里面装上火药,安上捻子,当徽州炮放这个嘎杂子啊!汤老爷!你是个好人呐。呵呵哈哈哈$\cdots{}\cdots{}$({\hwfs 笑介})}

\setlength{\hangindent}{56pt}{汤勤\hspace{30pt}($\cdots{}\cdots{}$什么,)哎呀,这两个狗头!}

\setlength{\hangindent}{56pt}{汤勤\hspace{30pt}哎呀,雪娘子,陆大人命我背审,我说人头是真的,定是真的;我说是假的,定是假的。哎呀雪娘子,你把那心要放明白了呀!({\akai 或}:~哎呀,雪娘子,陆大人命我审问人头,我说人头是真,就是真;我说是假的,定是假的。雪娘子,诶呀,你的心呀,哎呀,要放明白呀!)}

\setlength{\hangindent}{56pt}{雪艳\hspace{30pt}(暗)哎呀且住,听贼之言有戏奴之意,此时不应儿夫冤仇何日得报?!哦,有了,待我假意应下就是。}

\setlength{\hangindent}{56pt}{雪艳\hspace{30pt}哎呀汤老爷({\akai 或}:~啊汤老爷),自那日上船之时,我这心中就有了你了$\cdots{}\cdots{}$}

\setlength{\hangindent}{56pt}{汤勤\hspace{30pt}哎呀,我的亲娘------}

\setlength{\hangindent}{56pt}{内\hspace{40pt}噢------}

\setlength{\hangindent}{56pt}{(汤勤{\hwfs 急下})}

\vspace{3pt}{\centerline{{[}{\hei 第三场}{]}}}\vspace{5pt}

\setlength{\hangindent}{56pt}{(陆炳人{\hwfs 同上})}

\setlength{\hangindent}{56pt}{陆炳\hspace{30pt}【{\akai 二黄散板}】大炮一声响({\akai 或}:~号炮一声响)人头落,世人休犯({\akai 或}:~世人莫犯;为人莫犯)律萧何。}

\setlength{\hangindent}{56pt}{(陆炳{\hwfs 下轿入位},张龙、郭义{\hwfs 暗上},汤勤{\hwfs 上})}

\setlength{\hangindent}{56pt}{汤勤\hspace{30pt}老大人一路之上,多有辛苦({\akai 或}:~多受辛苦)了哇。}

\setlength{\hangindent}{56pt}{陆炳\hspace{30pt}为国勤劳,何言``辛苦''二字?}

\setlength{\hangindent}{56pt}{汤勤\hspace{30pt}老大人适才监斩,但不知斩的(都)是什么案件?}

\setlength{\hangindent}{56pt}{陆炳\hspace{30pt}有(一)十三名江洋大盗,呃,问他们什么罪过({\akai 或}:~问他个什么罪过)?}

\setlength{\hangindent}{56pt}{汤勤\hspace{30pt}呃,当问斩罪。}

\setlength{\hangindent}{56pt}{陆炳\hspace{30pt}不错的。汤老爷,我有一事不明,要在汤老爷面前领教。}

\setlength{\hangindent}{56pt}{汤勤\hspace{30pt}呃,老大人有话请讲,何言``领教''二字?}

\setlength{\hangindent}{56pt}{陆炳\hspace{30pt}呃,有一官员({\akai 或}:~有一员犯官),他奉旨领兵,不料他临阵脱逃,呃,该当问(他个)什么罪过?}

\setlength{\hangindent}{56pt}{汤勤\hspace{30pt}呃呃,也问他(个)斩罪。}

\setlength{\hangindent}{56pt}{陆炳\hspace{30pt}嗯,斩也不亏他。}

\setlength{\hangindent}{56pt}{汤勤\hspace{30pt}呃老大人,呃,还有什么案件呢?}

\setlength{\hangindent}{56pt}{陆炳\hspace{30pt}呃------呃呃,还有这么一员小官({\akai 或}:~还有这么一名小官)呐,当初他不得第(的时节),多亏他的恩主,将他提拔起来了,他在大官面前搬动是非,害死他的恩主,一家四散。呃,像这样的人呐,该当问他(个)什么罪过?呃,我要领教,汤老爷。}

\setlength{\hangindent}{56pt}{汤勤\hspace{30pt}哎呀,老大人,此案要犯在小官手内么({\akai 或}:~此案若犯在小官手内么),嗯嗯,把他提上堂来({\akai 或}:~将他提上堂来),打他五个手简子\footnote{刘曾复先生钞本作``手剪子'',下同。},不要惯(坏)了他的下次,诶,也就够他受用的了哇。}

\setlength{\hangindent}{56pt}{陆炳\hspace{30pt}什么({\akai 或}:~怎么),打他几个手简子?
({\akai 或}:~啊?打他五个手简子?)}

\setlength{\hangindent}{56pt}{汤勤\hspace{30pt}打他五个手简子。}

\setlength{\hangindent}{56pt}{陆炳\hspace{30pt}怎么,依我看将起来,定要将他问一个凌迟碎剐!}

\setlength{\hangindent}{56pt}{汤勤\hspace{30pt}哎呀,老大人,太重了哇({\akai 或}:~忒重了哇)。}

\setlength{\hangindent}{56pt}{陆炳\hspace{30pt}他是舌剑------杀人,最是可恶({\akai 或}:~最是可恨)的呀。({\akai 或}:~舌剑杀人,最是可恶的了哇。)}

\setlength{\hangindent}{56pt}{汤勤\hspace{30pt}重了,重了。}

\setlength{\hangindent}{56pt}{陆炳\hspace{30pt}重了?}

\setlength{\hangindent}{56pt}{汤勤\hspace{30pt}重了。}

\setlength{\hangindent}{56pt}{陆炳\hspace{30pt}啊,呵呵哈哈哈$\cdots{}\cdots{}$({\hwfs 笑介})啊汤老爷,你可(曾)背审雪艳呐,人头是真是假呐?}

\setlength{\hangindent}{56pt}{汤勤\hspace{30pt}这个人头么?是真的。}

\setlength{\hangindent}{56pt}{陆炳\hspace{30pt}是真的?}

\setlength{\hangindent}{56pt}{汤勤\hspace{30pt}呃,(是)真的呀。}

\setlength{\hangindent}{56pt}{陆炳\hspace{30pt}汤老爷,你上得堂来,就说了这么一句有良心的话呀。}

\setlength{\hangindent}{56pt}{汤勤\hspace{30pt}诶,小官最是有良心的({\akai 或}:~小官我是最有良心的)。}

\setlength{\hangindent}{56pt}{陆炳\hspace{30pt}张龙、郭义?}

\setlength{\hangindent}{56pt}{汤勤\hspace{30pt}销票无事。}

\setlength{\hangindent}{56pt}{陆炳\hspace{30pt}戚继光?}

\setlength{\hangindent}{56pt}{汤勤\hspace{30pt}原任蓟州八台。}

\setlength{\hangindent}{56pt}{陆炳\hspace{30pt}这雪艳------}

\setlength{\hangindent}{56pt}{汤勤\hspace{30pt}但凭老大人。}

\setlength{\hangindent}{56pt}{陆炳\hspace{30pt}哦,但凭老夫$\cdots{}\cdots{}$将她发往钱塘。}

\setlength{\hangindent}{56pt}{汤勤\hspace{30pt}路远。}

\setlength{\hangindent}{56pt}{陆炳\hspace{30pt}将她送到蓟州。}

\setlength{\hangindent}{56pt}{汤勤\hspace{30pt}无有亲人。}

\setlength{\hangindent}{56pt}{陆炳\hspace{30pt}呃,这样罢,就在老夫(的)衙门({\akai 或}:~就在老夫衙内)暂住,你意如何?}

\setlength{\hangindent}{56pt}{汤勤\hspace{30pt}哦,将雪艳要暂住老大人(的)衙内?}

\setlength{\hangindent}{56pt}{陆炳\hspace{30pt}诶,暂且寄住而已({\akai 或}:~暂住而已)。}

\setlength{\hangindent}{56pt}{汤勤\hspace{30pt}人头定是假的,还要再审再审。}

\setlength{\hangindent}{56pt}{(汤勤{\hwfs 下})}

\setlength{\hangindent}{56pt}{陆炳\hspace{30pt}汤勤{\footnotesize 呐},好奸贼!我听他之言,有要纳雪艳为妾之心({\akai 或}:~分明要纳雪艳为妾)。我若将雪艳断与那贼为妻({\akai 或}:~我若将雪艳断与那汤勤为妾),慢说(是)满朝文武道我无才,就是我这两班的衙役,也是(要)道我无才!~(这这这$\cdots{}\cdots{}$)好不难坏我也$\cdots{}\cdots{}$}

\setlength{\hangindent}{56pt}{雪艳\hspace{30pt}唉呀,好一位({\akai 或}:~好一个)不明白的陆大人呐!({\hwfs 哭介})}

\setlength{\hangindent}{56pt}{陆炳\hspace{30pt}且住!老夫正在为难之际,雪艳言道:~``好个不明白的陆大人!''哦哦哦,是了!~我那莫仁兄也曾对我讲过,他讲道({\akai 或}:~我那莫仁兄也曾对我言讲):~雪艳(虽然是个妓者出身,倒)有些仁义节烈,她莫非是有心,与我那莫仁兄报仇?!}

\setlength{\hangindent}{56pt}{陆炳\hspace{30pt}\textless{}\!{\bfseries\akai 叫头}\!\textgreater{}雪娘子啊,莫仁嫂!}

\setlength{\hangindent}{56pt}{陆炳\hspace{30pt}你若是有心与我那莫仁兄报仇,拚着我陆炳这顶乌纱不要,我就与你担待、担待。正是:~({\akai 念})清官暂把赃官做,聪明反做懵懂人。}

\setlength{\hangindent}{56pt}{(汤勤{\hwfs 上})}

\setlength{\hangindent}{56pt}{汤勤\hspace{30pt}老大人为何在背地沉吟?}

\setlength{\hangindent}{56pt}{陆炳\hspace{30pt}非是老夫背地沉吟。方才又问了雪艳一遍,(这个)人头本来是真的。}

\setlength{\hangindent}{56pt}{汤勤\hspace{30pt}哦,是真的,是真的。}

\setlength{\hangindent}{56pt}{陆炳\hspace{30pt}张龙、郭义,销票无事;戚继光,原任总兵({\akai 或}:~原任八台);仔细想来,将雪艳送回钱塘({\akai 或}:~将雪艳送往钱塘),钱塘路远呐。}

\setlength{\hangindent}{56pt}{汤勤\hspace{30pt}本来路远。}

\setlength{\hangindent}{56pt}{陆炳\hspace{30pt}送到蓟州({\akai 或}:~送往蓟州),无有亲人。}

\setlength{\hangindent}{56pt}{汤勤\hspace{30pt}(是)无有亲人。}

\setlength{\hangindent}{56pt}{陆炳\hspace{30pt}寄在老夫(的)衙内({\akai 或}:~寄住老夫的衙内),出入有些不便呐。}

\setlength{\hangindent}{56pt}{汤勤\hspace{30pt}是啊,有些不便,诶,二来,于老大人的名气,有些不好啊。}

\setlength{\hangindent}{56pt}{陆炳\hspace{30pt}(唉,)莫若将雪艳寄在汤老爷的衙内?}

\setlength{\hangindent}{56pt}{汤勤\hspace{30pt}老大人说哪里话来,雪艳她又不是什么货物,今日寄在东家,明日寄在西家。老大人要办么,就办一个``水落石出''。}

\setlength{\hangindent}{56pt}{陆炳\hspace{30pt}(哦,)要办个``水落石出''({\akai 或}:~要断个``水落石出'')?}

\setlength{\hangindent}{56pt}{汤勤\hspace{30pt}嗯,办个``水落石出''。}

\setlength{\hangindent}{56pt}{陆炳\hspace{30pt}汤老爷,你可有宝眷呐?}

\setlength{\hangindent}{56pt}{汤勤\hspace{30pt}呃,(我)无有家眷。}

\setlength{\hangindent}{56pt}{陆炳\hspace{30pt}老夫作主,将雪艳断与汤老爷(为妾)。}

\setlength{\hangindent}{56pt}{汤勤\hspace{30pt}哪个为媒?}

\setlength{\hangindent}{56pt}{陆炳\hspace{30pt}老夫为媒。}

\setlength{\hangindent}{56pt}{汤勤\hspace{30pt}哦,老大人为媒?哎呀,犹如我重生父母,再造的爹娘。我这里多谢老大人,多谢$\cdots{}\cdots{}$。}

\setlength{\hangindent}{56pt}{陆炳\hspace{30pt}呃呃,起来,起来。}

\setlength{\hangindent}{56pt}{汤勤\hspace{30pt}多谢老大人。}

\setlength{\hangindent}{56pt}{陆炳\hspace{30pt}请坐。}

\setlength{\hangindent}{56pt}{汤勤\hspace{30pt}呃,谢座。}

\setlength{\hangindent}{56pt}{陆炳\hspace{30pt}呃,我还有话问你,}

\setlength{\hangindent}{56pt}{汤勤\hspace{30pt}呃,老大人有话请讲。}

\setlength{\hangindent}{56pt}{陆炳\hspace{30pt}人头是真的?}

\setlength{\hangindent}{56pt}{汤勤\hspace{30pt}呃,真的。}

\setlength{\hangindent}{56pt}{陆炳\hspace{30pt}张龙、郭义?}

\setlength{\hangindent}{56pt}{汤勤\hspace{30pt}销票无事。}

\setlength{\hangindent}{56pt}{陆炳\hspace{30pt}戚继光?}

\setlength{\hangindent}{56pt}{汤勤\hspace{30pt}原任八台总镇。}

\setlength{\hangindent}{56pt}{陆炳\hspace{30pt}严大人见罪?}

\setlength{\hangindent}{56pt}{汤勤\hspace{30pt}有小官担待。}

\setlength{\hangindent}{56pt}{陆炳\hspace{30pt}原是要你担待。}

\setlength{\hangindent}{56pt}{陆炳\hspace{30pt}来,带张龙、郭义。}

\setlength{\hangindent}{56pt}{(张龙、郭义{\hwfs 上})}

\setlength{\hangindent}{56pt}{张龙、郭义\hspace{30pt}叩见大人。}

\setlength{\hangindent}{56pt}{陆炳\hspace{30pt}人头是真,有文书一轴回覆严大人,外有手本,问候金安,下去!}

\setlength{\hangindent}{56pt}{张龙、郭义\hspace{30pt}谢大人。}

\setlength{\hangindent}{56pt}{(张龙、郭义{\hwfs 下})}

\setlength{\hangindent}{56pt}{陆炳\hspace{30pt}来,将雪艳放了下来。}

\setlength{\hangindent}{56pt}{(公差{\hwfs 放}雪艳{\hwfs 介})}

\setlength{\hangindent}{56pt}{雪艳\hspace{30pt}叩见大人。({\akai 或}:~谢大人!)}

\setlength{\hangindent}{56pt}{陆炳\hspace{30pt}雪艳,老夫为媒,将你断与汤老爷。汤老爷可比不得莫大老爷,你必须要殷勤早晚伺(谐音:~刺)(陆炳{\hwfs 用扇遮挡介})------伺候!~伺候!}

\setlength{\hangindent}{56pt}{雪艳\hspace{30pt}多谢大人!}

\setlength{\hangindent}{56pt}{雪艳\hspace{30pt}【{\akai 二黄散板}】好一位大人断得妙,奴与汤勤配缘交。耐等三更时分到,难躲奴家这一刀。}

\setlength{\hangindent}{56pt}{雪艳\hspace{30pt}喂呀$\cdots{}\cdots{}$({\hwfs 哭介})}

\setlength{\hangindent}{56pt}{(雪艳{\hwfs 下})}

\setlength{\hangindent}{56pt}{陆炳\hspace{30pt}汤老爷,请回衙理事。}

\setlength{\hangindent}{56pt}{汤勤\hspace{30pt}多谢大人!小官告辞了哇。}

\setlength{\hangindent}{56pt}{汤勤\hspace{30pt}【{\akai 二黄摇板}】辞别大人下大堂,汤勤今晚做新郎。}

\setlength{\hangindent}{56pt}{汤勤\hspace{30pt}哈哈哈$\cdots{}\cdots{}$({\hwfs 笑介})}

\setlength{\hangindent}{56pt}{(汤勤{\hwfs 下})}

\setlength{\hangindent}{56pt}{陆炳\hspace{30pt}转堂!}

\setlength{\hangindent}{56pt}{(起\textless{}\!{\bfseries\akai 牌子}\!\textgreater{})}

\setlength{\hangindent}{56pt}{陆炳\hspace{30pt}来,有请戚大人。}

\setlength{\hangindent}{56pt}{门子\hspace{30pt}有请戚大人。}

\setlength{\hangindent}{56pt}{(戚继光{\hwfs 上})}

\setlength{\hangindent}{56pt}{戚继光\hspace{40pt}【{\akai 二黄摇板}】忽听仁兄一声请,来在二堂问详情。}

\setlength{\hangindent}{56pt}{陆炳\hspace{30pt}贤弟请坐。}

\setlength{\hangindent}{56pt}{戚继光\hspace{40pt}有座。}

\setlength{\hangindent}{56pt}{陆炳\hspace{30pt}贤弟(你)受惊了哇。}

\setlength{\hangindent}{56pt}{戚继光\hspace{40pt}有劳仁兄挂心了。}

\setlength{\hangindent}{56pt}{陆炳\hspace{30pt}岂敢。}

\setlength{\hangindent}{56pt}{陆炳\hspace{30pt}恭喜贤弟,贺喜贤弟。}

\setlength{\hangindent}{56pt}{戚继光\hspace{40pt}喜从何来?}

\setlength{\hangindent}{56pt}{陆炳\hspace{30pt}贤弟还是八台原任。}

\setlength{\hangindent}{56pt}{戚继光\hspace{40pt}此乃仁兄提拔。}

\setlength{\hangindent}{56pt}{陆炳\hspace{30pt}岂敢。}

\setlength{\hangindent}{56pt}{戚继光\hspace{40pt}那雪艳怎样落案({\akai 或}:~怎样发落)?}

\setlength{\hangindent}{56pt}{陆炳\hspace{30pt}将那雪艳断与汤勤了。}

\setlength{\hangindent}{56pt}{戚继光\hspace{40pt}唉,仁兄你真真(的)大大无才!}

\setlength{\hangindent}{56pt}{陆炳\hspace{30pt}贤弟,我这叫作``不得已而为之''!}

\setlength{\hangindent}{56pt}{陆炳\hspace{30pt}【{\akai 四平调}】自古道人亏天不亏,过往神灵饶过谁。请贤弟暂且回衙去,最可叹仁兄死得苦,三日之内自有信回。({\akai 或}:~戚贤弟暂且回衙去,三日之后自有信回。)}

\setlength{\hangindent}{56pt}{戚继光\hspace{40pt}【{\akai 四平调}】仁兄做事大有才,胸中韬略弟怎解开。辞别仁兄出府外,三日等候报马来。}

\setlength{\hangindent}{56pt}{(戚继光{\hwfs 下})}

\setlength{\hangindent}{56pt}{陆炳\hspace{30pt}来,衙役们进见。}

\setlength{\hangindent}{56pt}{(门子\hspace{30pt}是。)}

\setlength{\hangindent}{56pt}{(衙役{\hwfs 上})}

\setlength{\hangindent}{56pt}{众\hspace{40pt}参见大人。}

\setlength{\hangindent}{56pt}{陆炳\hspace{30pt}衙役们({\akai 或}:~衙役的),听我吩咐。每人领银三分({\akai 或}:~每人用银三分),庆贺汤勤。在洞房之内,尔等(们)只管劝酒,闹出祸来({\akai 或}:~闯出祸来),有老夫担待,尔等记下了。}

\setlength{\hangindent}{56pt}{众\hspace{40pt}遵命。}

\setlength{\hangindent}{56pt}{(衙役{\hwfs 下})}

\setlength{\hangindent}{56pt}{陆炳\hspace{30pt}来,吩咐掩门。}

\setlength{\hangindent}{56pt}{门子\hspace{30pt}掩门。}

\setlength{\hangindent}{56pt}{(陆炳{\hwfs 下})}

\vspace{3pt}{\centerline{{[}{\hei 第四场}{]}}}\vspace{5pt}

\setlength{\hangindent}{56pt}{雪艳\hspace{30pt}({\akai 内})苦哇$\cdots{}\cdots{}$}

\setlength{\hangindent}{56pt}{雪艳\hspace{30pt}【{\akai 二黄导板}】听谯楼打罢了初更时候,}

\setlength{\hangindent}{56pt}{(雪艳{\hwfs 上})}

\setlength{\hangindent}{56pt}{雪艳\hspace{30pt}\textless{}\!{\bfseries\akai 叫头}\!\textgreater{}老爷,夫君!喂呀$\cdots{}\cdots{}$({\hwfs 哭介})}

\setlength{\hangindent}{56pt}{雪艳\hspace{30pt}【{\akai 二黄正板}】上房内来了我雪艳女流。想当年身落在烟花巷口,多亏了莫老爷将我收留。我二人在钱塘荣华不受,一心心要进京去把官求。那日里拜客时大街行走,汤勤贼卖字画十字街头。我老爷喜字画心中爱就,他那时与汤勤骨肉相投。严世藩闻此言双眉起皱\footnote{此处刘曾复先生钞本作``双眉皱''。},他那里带校尉来把杯搜。我老爷闻此言弃官逃走,西门外柳林下将奴锁收。叹莫成替主死蓟州堂口,叹老爷去他乡不能回头。叹夫人在钱塘不能得够,叹莫家叹得我两泪交流。听谯楼打罢了二更时候,等候了贼子到好报冤仇。}

\setlength{\hangindent}{56pt}{汤勤\hspace{30pt}({\akai 内})走哇。}

\setlength{\hangindent}{56pt}{(丑皂隶{\hwfs 同上})}

\setlength{\hangindent}{56pt}{汤勤\hspace{30pt}【{\akai 二黄摇板}】人得喜事精神爽,月到中秋分外光。}

\setlength{\hangindent}{56pt}{皂隶\hspace{40pt}汤老爷啊,脑袋掉了。}

\setlength{\hangindent}{56pt}{汤勤\hspace{30pt}咳,这是怎么讲话,纱帽掉了。}

\setlength{\hangindent}{56pt}{皂隶\hspace{40pt}咳,纱帽不在脑袋上戴着么不是的。}

\setlength{\hangindent}{56pt}{汤勤\hspace{30pt}咳,捡起来。}

\setlength{\hangindent}{56pt}{皂隶\hspace{40pt}哦,捡起来。}

\setlength{\hangindent}{56pt}{汤勤\hspace{30pt}你前面带路。}

\setlength{\hangindent}{56pt}{皂隶\hspace{40pt}慢着,我不是跟着你的么,理应你在头里,我在后头。}

\setlength{\hangindent}{56pt}{汤勤\hspace{30pt}咳,你在头里。}

\setlength{\hangindent}{56pt}{皂隶\hspace{40pt}我在头里?}

\setlength{\hangindent}{56pt}{汤勤\hspace{30pt}还是你头里。}

\setlength{\hangindent}{56pt}{皂隶\hspace{40pt}我要是在头里,那岂不是你成了跟着我的么?}

\setlength{\hangindent}{56pt}{汤勤\hspace{30pt}咳,你在头里,喝道拦挡闲人。}

\setlength{\hangindent}{56pt}{皂隶\hspace{40pt}我在头里,就是这么办。}

\setlength{\hangindent}{56pt}{汤勤\hspace{30pt}前面带路。}

\setlength{\hangindent}{56pt}{皂隶\hspace{40pt}咳,我在头里。屎来啦,屎来啦。}

\setlength{\hangindent}{56pt}{汤勤\hspace{30pt}狗才,你怎么说屎来啦。}

\setlength{\hangindent}{56pt}{皂隶\hspace{40pt}人家的官大,咱们的官小,挡住道儿不教咱们过去,这一闻见屎来啦,齁臭\footnote{刘曾复先生钞本作``乎臭的'',下同。}的,人家自然而然的就``躲开罢'',``躲开他罢''$\cdots{}\cdots{}$今晚上你那洞房花烛,叩门回事\footnote{此处刘曾复先生钞本文字不确认,段公平{\scriptsize 君}建议作``那么回事''。}齁臭的,不吉祥。}

\setlength{\hangindent}{56pt}{汤勤\hspace{30pt}咳,讲得有理,屎就屎,再去喝道。}

\setlength{\hangindent}{56pt}{皂隶\hspace{40pt}屎蛋来啦。}

\setlength{\hangindent}{56pt}{汤勤\hspace{30pt}你怎么又喝道说屎蛋来啦?}

\setlength{\hangindent}{56pt}{皂隶\hspace{40pt}您{\footnotesize 呐}不知道,这屎蛋是接年\footnote{``接年'',老北京话,即``隔年''的意思。李楠{\scriptsize 君}按:~``接''字的正字应是``间'',因音近讹。}的了,比屎还臭,那人家闻见,说``比屎还臭,快快的躲开罢。''今儿晚上洞房花烛,哪个不来贺喜?}

\setlength{\hangindent}{56pt}{汤勤\hspace{30pt}讲得有理,唉,屎蛋就屎蛋。}

\setlength{\hangindent}{56pt}{皂隶\hspace{40pt}屎蛋不是?哎,屎蛋来------屎蛋来了。}

\setlength{\hangindent}{56pt}{(皂隶、汤勤{\hwfs 原场})}

\setlength{\hangindent}{56pt}{汤勤\hspace{30pt}哎呀,我走不动了。}

\setlength{\hangindent}{56pt}{皂隶\hspace{40pt}走不动了就算到了罢。}

\setlength{\hangindent}{56pt}{汤勤\hspace{30pt}你去叫开。}

\setlength{\hangindent}{56pt}{皂隶\hspace{40pt}哦,我去叫门------开门来。}

\setlength{\hangindent}{56pt}{雪艳\hspace{30pt}是哪个叫门?}

\setlength{\hangindent}{56pt}{皂隶\hspace{40pt}我再问问去。汤老爷,里面问何人叫门。}

\setlength{\hangindent}{56pt}{汤勤\hspace{30pt}你就说汤老爷到了。}

\setlength{\hangindent}{56pt}{皂隶\hspace{40pt}哎是啦,开门来。}

\setlength{\hangindent}{56pt}{雪艳\hspace{30pt}是哪个叫门?}

\setlength{\hangindent}{56pt}{皂隶\hspace{40pt}汤老爷到了,快来开门。}

\setlength{\hangindent}{56pt}{雪艳\hspace{30pt}这里不认得什么汤老爷。}

\setlength{\hangindent}{56pt}{皂隶\hspace{40pt}我再问问去。汤老爷,里面说了,不认得什么汤老爷。}

\setlength{\hangindent}{56pt}{汤勤\hspace{30pt}你说是汤勤汤老爷到了,快来开门罢。}

\setlength{\hangindent}{56pt}{皂隶\hspace{40pt}开门来,汤勤汤老爷到了,快快开门罢。}

\setlength{\hangindent}{56pt}{雪艳\hspace{30pt}我这里不知道什么汤老爷。}

\setlength{\hangindent}{56pt}{皂隶\hspace{40pt}哎呀,这个麻烦着,汤老爷,里头又说了,她不知道什么汤老爷。}

\setlength{\hangindent}{56pt}{汤勤\hspace{30pt}你再去言道,说是汤勤汤老爷、裱褙的汤老爷、裱字画的汤老爷。}

\setlength{\hangindent}{56pt}{皂隶\hspace{40pt}哎,这么些个啰哩啰嗦,里头听着:~汤勤汤老爷、裱褙的汤老爷,裱字画的、婊子下的汤老爷。}

\setlength{\hangindent}{56pt}{雪艳\hspace{30pt}我全都不认识。}

\setlength{\hangindent}{56pt}{皂隶\hspace{40pt}哎呀汤老爷,里头说了,她全都不认识。您呐自己去罢。}

\setlength{\hangindent}{56pt}{汤勤\hspace{30pt}待我去叫门。啊雪娘子开门,汤勤汤老爷到了,快快地开门来。}

\setlength{\hangindent}{56pt}{雪艳\hspace{30pt}这里无有什么汤老爷。}

\setlength{\hangindent}{56pt}{汤勤\hspace{30pt}哎呀,她不开门如何是好?}

\setlength{\hangindent}{56pt}{皂隶\hspace{40pt}昨日我舅舅死了,我舅舅下销\footnote{``下销'',指死人入殓后,在棺材上钉入销钉(销钉一般是七根,俗称``子孙钉'')。}落下一把斧子,她不开门,我就劈。}

\setlength{\hangindent}{56pt}{汤勤\hspace{30pt}前去劈门。}

\setlength{\hangindent}{56pt}{皂隶\hspace{40pt}劈你们家的坟。}

\setlength{\hangindent}{56pt}{汤勤\hspace{30pt}咳,劈门。}

\setlength{\hangindent}{56pt}{皂隶\hspace{40pt}劈门。呔,里面听着:~汤老爷说了,若不开门,可要劈你们的棺材板啦。}

\setlength{\hangindent}{56pt}{雪艳\hspace{30pt}待我开门就是了------有请汤老爷。}

\setlength{\hangindent}{56pt}{皂隶\hspace{40pt}里面有请汤老爷。}

\setlength{\hangindent}{56pt}{汤勤\hspace{30pt}哎呀。}

\setlength{\hangindent}{56pt}{(汤勤{\hwfs 醉介},{\hwfs 坐})}

\setlength{\hangindent}{56pt}{皂隶\hspace{40pt}汤老爷醉死了。}

\setlength{\hangindent}{56pt}{汤勤\hspace{30pt}咳,大喜了,大喜了。}

\setlength{\hangindent}{56pt}{皂隶\hspace{40pt}大喜了,小人讨赏。汤老爷您呐赏给我几吊罢。}

\setlength{\hangindent}{56pt}{汤勤\hspace{30pt}几吊钱?连给几百钱都没有。}

\setlength{\hangindent}{56pt}{皂隶\hspace{40pt}得了汤老爷,不论多少您呐赏我俩钱罢。}

\setlength{\hangindent}{56pt}{汤勤\hspace{30pt}哎,太唠叨了,连一个大钱都没有!}

\setlength{\hangindent}{56pt}{皂隶\hspace{40pt}我不要了,你留着钱买棺材罢。大爷走了。}

\setlength{\hangindent}{56pt}{(皂隶{\hwfs 下})}

\setlength{\hangindent}{56pt}{汤勤\hspace{30pt}这个狗才什么东西,胡说八道的,真真岂有此理话啊!}

\setlength{\hangindent}{56pt}{(丑报丧{\hwfs 上})}

\setlength{\hangindent}{56pt}{报丧人\hspace{40pt}唉呀,汤老爷,我姥姥死了汤老爷。}

\setlength{\hangindent}{56pt}{汤勤\hspace{30pt}哎呀,我这里乃是大喜的事情,你怎么跪在我这里报丧来了。这是哪里说起?滚出去。}

\setlength{\hangindent}{56pt}{报丧人\hspace{40pt}汤老爷你呐赏口棺材罢。}

\setlength{\hangindent}{56pt}{汤勤\hspace{30pt}咳,哪里来的棺材,赏你一口``狗碰头''\footnote{``狗碰头''是北京俗话,形容棺材非常简易,野狗用头就能撞开棺材板子,把尸体掏出来吃掉。}去罢。}

\setlength{\hangindent}{56pt}{报丧人\hspace{40pt}``狗碰头''留着装你自己罢。}

\setlength{\hangindent}{56pt}{(丑报丧{\hwfs 下})}

\setlength{\hangindent}{56pt}{({\hwfs 四}青袍、丑书吏{\hwfs 上})}

\setlength{\hangindent}{56pt}{众\hspace{40pt}走哇,走哇。}

\setlength{\hangindent}{56pt}{书吏\hspace{30pt}列位请了。}

\setlength{\hangindent}{56pt}{众\hspace{40pt}请了。}

\setlength{\hangindent}{56pt}{书吏\hspace{30pt}你我奉了陆大人之命,与汤勤贺喜,教我们大家将他灌醉,大家走哇。}

\setlength{\hangindent}{56pt}{(众{\hwfs 走原场})}

\setlength{\hangindent}{56pt}{书吏\hspace{30pt}到了,大家进去啊。}

\setlength{\hangindent}{56pt}{书吏\hspace{30pt}汤老爷在哪里。}

\setlength{\hangindent}{56pt}{汤勤\hspace{30pt}列位来了。}

\setlength{\hangindent}{56pt}{众\hspace{40pt}来了。汤老爷醉死了。}

\setlength{\hangindent}{56pt}{汤勤\hspace{30pt}咳,该死的,是什么话,乃是大喜了。}

\setlength{\hangindent}{56pt}{书吏\hspace{30pt}不错,大喜了。}

\setlength{\hangindent}{56pt}{汤勤\hspace{30pt}你们是哪里来的?}

\setlength{\hangindent}{56pt}{书吏\hspace{30pt}我等奉了陆大人之命前来道喜。}

\setlength{\hangindent}{56pt}{汤勤\hspace{30pt}有劳众位了。}

\setlength{\hangindent}{56pt}{书吏\hspace{30pt}汤老爷是要喝个喜酒儿。}

\setlength{\hangindent}{56pt}{汤勤\hspace{30pt}是要吃的啊。}

\setlength{\hangindent}{56pt}{书吏\hspace{30pt}待我(来)把敬把敬。}

\setlength{\hangindent}{56pt}{汤勤\hspace{30pt}不敢当了。}

\setlength{\hangindent}{56pt}{书吏\hspace{30pt}使得的。汤老爷,你吃了我这酒一杯,死后变乌龟。}

\setlength{\hangindent}{56pt}{汤勤\hspace{30pt}诶,这是什么讲话({\akai 或}:~这是怎么讲话)?咳,后来大富贵。}

\setlength{\hangindent}{56pt}{书吏\hspace{30pt}不错,(后来)大富贵。}

\setlength{\hangindent}{56pt}{书吏\hspace{30pt}汤老爷,你吃了这二杯酒,死了变黄狗({\akai 或}:~死后变黄狗)。}

\setlength{\hangindent}{56pt}{汤勤\hspace{30pt}诶,越发的不像话了。呃,后来子孙有。}

\setlength{\hangindent}{56pt}{书吏\hspace{30pt}不错的,后来子孙有。}

\setlength{\hangindent}{56pt}{书吏\hspace{30pt}汤老爷,这三杯酒儿入肚肠,死了之后见阎王。}

\setlength{\hangindent}{56pt}{汤勤\hspace{30pt}诶,这是怎么讲话?后来产生状元郎。}

\setlength{\hangindent}{56pt}{书吏\hspace{30pt}诶,状元郎。}

\setlength{\hangindent}{56pt}{书吏\hspace{30pt}汤老爷,还要饮三大杯({\akai 或}:~再饮三大杯)。}

\setlength{\hangindent}{56pt}{汤勤\hspace{30pt}诶,我这酒吃不得了哇$\cdots{}\cdots{}$呵呵,$\cdots{}\cdots{}$(足)够了。}

\setlength{\hangindent}{56pt}{书吏\hspace{30pt}诶,可以再饮三杯,连中三元,连生贵子。}

\setlength{\hangindent}{56pt}{汤勤\hspace{30pt}哦,要连生贵子。呃,再饮再饮。}

\setlength{\hangindent}{56pt}{(汤勤{\hwfs 醉介})}

\setlength{\hangindent}{56pt}{汤勤\hspace{30pt}呃,呜呜呜。}

\setlength{\hangindent}{56pt}{书吏\hspace{30pt}列位,汤勤醉了,你我走罢。}

\setlength{\hangindent}{56pt}{众\hspace{40pt}走哇。}

\setlength{\hangindent}{56pt}{书吏\hspace{30pt}如此走哇。}

\setlength{\hangindent}{56pt}{书吏\hspace{30pt}汤老爷天不早了,我们要回去了哇。}

\setlength{\hangindent}{56pt}{汤勤\hspace{30pt}哎,列位请回来罢。}

\setlength{\hangindent}{56pt}{书吏\hspace{30pt}啊雪娘子,汤勤是醉了,与你丈夫报仇也在你,不报仇也在你。我们大家走了。}

\setlength{\hangindent}{56pt}{(书吏{\hwfs 锁门})}

\setlength{\hangindent}{56pt}{众\hspace{40pt}走哇。}

\setlength{\hangindent}{56pt}{(众{\hwfs 下})}

\setlength{\hangindent}{56pt}{雪艳\hspace{30pt}汤老爷夜已深了,请安歇去罢。}

\setlength{\hangindent}{56pt}{汤勤\hspace{30pt}啊雪娘子,你我安歇了罢啊。}

\setlength{\hangindent}{56pt}{(汤勤{\hwfs 进帐},{\hwfs 脱衣介})}

\setlength{\hangindent}{56pt}{雪艳\hspace{30pt}汤老爷,汤老爷$\cdots{}\cdots{}$}

\setlength{\hangindent}{56pt}{雪艳\hspace{30pt}\textless{}\!{\bfseries\akai 叫头}\!\textgreater{}且住!}

\setlength{\hangindent}{56pt}{雪艳\hspace{30pt}看贼子已睡,此时不下手待等何时?待我动起手来罢。}

\setlength{\hangindent}{56pt}{雪艳\hspace{30pt}【{\akai 二黄散板}】见贼子不由我心中怒恼,蓟州城害夫君心如火烧。我身旁暗藏刀把夫仇来报,管教这狗奸贼命赴阴曹。}

\setlength{\hangindent}{56pt}{(雪艳{\hwfs 进帐刺}汤勤{\hwfs 介})}

\setlength{\hangindent}{56pt}{汤勤\hspace{30pt}唉呀!}

\setlength{\hangindent}{56pt}{(\textless{}\!{\bfseries\akai 扑灯蛾}\!\textgreater{}汤勤{\hwfs 死介})}

\setlength{\hangindent}{56pt}{雪艳\hspace{30pt}且住,看贼子已死,不免逃走了罢。}

\setlength{\hangindent}{56pt}{雪艳\hspace{30pt}唉呀且住,想我女流之辈,往哪里逃走?也罢,不免行个自尽了罢。}

\setlength{\hangindent}{56pt}{雪艳\hspace{30pt}喂呀$\cdots{}\cdots{}$({\hwfs 哭介})}

\setlength{\hangindent}{56pt}{雪艳\hspace{30pt}【{\akai 二黄散板}】眼望钱塘忙拜定,拜谢老爷收留恩。一把宝剑拿在手,}

\setlength{\hangindent}{56pt}{雪艳\hspace{30pt}罢!}

\setlength{\hangindent}{56pt}{雪艳\hspace{30pt}【{\akai 二黄散板}】不如一死赴幽冥。}

\setlength{\hangindent}{56pt}{(雪艳{\hwfs 自刎下})}

\vspace{3pt}{\centerline{{[}{\hei 第五场}{]}}}\vspace{5pt}

\setlength{\hangindent}{56pt}{({\hwfs 四青袍}、书吏{\hwfs 上})}

\setlength{\hangindent}{56pt}{众\hspace{40pt}天已亮了,大家开开门儿看一看。怎么样了哇?}

\setlength{\hangindent}{56pt}{(开门,众进门)}

\setlength{\hangindent}{56pt}{书吏\hspace{30pt}哦哟------列位,汤勤被雪艳刺死了。她乃女流之辈。四面俱是高墙,她可往哪里逃走。走是走不了的,她手拿着宝剑,她就这样自刎了哇。唉,真乃是为丈夫报仇,尽节!她死而无怨也。呃,呃,呃$\cdots{}\cdots{}$}

\setlength{\hangindent}{56pt}{(书吏{\hwfs 自刎介})}

\setlength{\hangindent}{56pt}{众\hspace{40pt}得,他也死了。唉呀,咱们大家将他抬下去罢!}

\setlength{\hangindent}{56pt}{({\hwfs 四}青袍{\hwfs 抬}书吏{\hwfs 下})}

}
