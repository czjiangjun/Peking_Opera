\newpage
\subsubsection{\large\hei {临江会~{\small 之}~刘备}}
\addcontentsline{toc}{subsection}{\hei 临江会~\small{之}~刘备}

\hangafter=1                   %2. 设置从第1⾏之后开始悬挂缩进  %
\setlength{\parindent}{0pt}{

{\centerline{{[}{\hei 第一场}{]}}}\vspace{5pt}

{[}{\akai 引子}{]}奸雄并立,起戈矛。怎能够,中原尽扫。

({\akai 念})临难仁心存百姓,登舟挥泪动三军。至今凭吊襄江口,父老犹然忆使君。

孤,刘备,自败当阳,兵屯夏口,只因孔明先生去往东吴,共议破曹之策。只是未见音信回来,教孤放心不下。

来,唤糜竺进帐。

罢了。

只因孔明先生去往东吴,一去渺无音信。意欲命你备下礼物,去往东吴。名为犒军,暗探先生。听我吩咐。

\setlength{\hangindent}{56pt}{【{\akai 西皮摇板}】过江去探孔明虚实动静,必须要一同回免孤忧心。 }

\setlength{\hangindent}{56pt}{【{\akai 西皮摇板}】叹只叹汉刘备未得天运,似困龙何日里平步登云。 }

\setlength{\hangindent}{56pt}{【{\akai 西皮摇板}】恨曹瞒勒逼我困于江夏,每日里操兵将习练战法。 }

\vspace{3pt}{\centerline{{[}{\hei 第二场}{]}}}\vspace{5pt}

\setlength{\hangindent}{56pt}{【{\akai 西皮摇板}】糜子仲往江东探其虚诈,待等那先生回细问根芽。 }

罢了,孔明先生何在?

哦,想是周瑜已定破曹之策,邀我面议。如此准备船只,即刻便行。

二弟少礼,请坐。

兄欲往江东赴会,二弟为何阻拦?

这$\cdots{}\cdots{}$

无有哇。

啊二弟,如今孙、刘结盟,共破曹操。周瑜相请,必有大事商议。今若不去,则两下猜疑,事不谐矣。

二弟同去,兄无忧矣。

就命三弟、四弟守寨,你我弟兄即刻过江。

\vspace{3pt}{\centerline{{[}{\hei 第三场}{]}}}\vspace{5pt}

看,江水波涛,水天一色。好一派江景也。

\setlength{\hangindent}{56pt}{【{\akai 西皮原板}】汉阳江上把船开,波涛滚滚风云来。两旁排列旌旗摆,临江会上逞英才。 }

二弟。孔明自往江东,渺无音信。今周郎邀我临江赴会,难免有诈,你我弟兄须要留心一二。

\vspace{3pt}{\centerline{{[}{\hei 第四场}{]}}}\vspace{5pt}

都督,备过江来了。

不敢,都督请。

啊,呵呵哈$\cdots{}\cdots{}$({\hwfs 陪笑介})

(周瑜\hspace{30pt}$\cdots{}\cdots{}$参拜。)

不敢,都督名闻天下,备不才无学,怎当将军重礼。

来,就分宾主而坐。

请坐。

岂敢,都督身挂金印,备少来恭贺,望祈海涵。

今日孙刘结盟,共破曹贼,乃天下之幸也。

不敢,摆下就是。

都督这是何意?

不敢,请起。

都督请!

\setlength{\hangindent}{56pt}{【{\akai 西皮原板}】多蒙美意礼相邀,临江会上似琼瑶。孙、刘两家结盟好,同心协力破奸曹。 }

\setlength{\hangindent}{56pt}{【{\akai 西皮原板}】都督英名天下晓,}

\setlength{\hangindent}{56pt}{【{\akai 西皮摇板}】定有妙计展雄韬。}

多谢都督。

干。

乃备二弟云长。

呃,乃他昔年之事,何足都督挂齿。

岂敢呐岂敢。

请便。

请问都督,帐下多少人马?何计破曹?

如此待等都督大功成就,备专当叩贺。

告别了。

\setlength{\hangindent}{56pt}{【{\akai 西皮摇板}】临江会上备讨扰。}

\vspace{3pt}{\centerline{{[}{\hei 第五场}{]}}}\vspace{5pt}

\setlength{\hangindent}{56pt}{【{\akai 西皮散板}】此来不见诸葛亮,倒教刘备挂心肠。弟兄同回江夏往。 }

哎呀,先生呐,你想煞我也。

呃,不知呀。

哎呀,险呐。

先生,同回夏口去罢!

\setlength{\hangindent}{56pt}{【{\akai 西皮散板}】诸葛亮果奇才世间少有,准备着迎他回整顿貔貅。 }

}
