\newpage
\subsubsection{\large\hei {战樊城}}
\addcontentsline{toc}{subsection}{\hei 战樊城}

\hangafter=1
\setlength{\parindent}{0pt}{
{\centerline{{[}{\hei 第一场}{]}}}\vspace{5pt}

伍尚\hspace{30pt}({\akai 念})边外狼烟净,

伍员\hspace{30pt}({\akai 念})共享太平春。

伍尚\hspace{30pt}贤弟请坐。

伍员\hspace{30pt}有座({\akai 或}:~请坐)。

伍尚\hspace{30pt}唉!

伍员\hspace{30pt}兄长自到樊城,为何终日忧闷?

伍尚\hspace{30pt}你我弟兄镇守樊城,不知双亲在京安否,令人悬念。

伍员\hspace{30pt}吉人自有天相,兄长何必多虑?

伍尚\hspace{30pt}但愿如此。

鄢将师\hspace{20pt}({\akai 念})离了京城地,来此是棠邑。

鄢将师\hspace{20pt}门上哪位在?

(家院\hspace{25pt}什么人?)

鄢将师\hspace{20pt}京城下书人求见。

伍尚、\\
伍员\hspace{30pt}\raisebox{5pt}{两厢伺候({\akai 或}:~外厢伺候)。}

伍尚、\\
伍员\hspace{30pt}\raisebox{5pt}{吩咐书先进,人落后。}

鄢将师\hspace{20pt}是。

(家院\hspace{25pt}书信呈上。)

伍尚\hspace{30pt}呈上来。

伍尚\hspace{30pt}贤弟,爹娘有书信到来,贤弟请看。

伍员\hspace{30pt}兄长请看。

伍尚\hspace{30pt}一同观看。

伍尚、\\
伍员\hspace{30pt}\raisebox{5pt}{爹娘在上,恕儿等不孝罪也。}

\setlength{\hangindent}{56pt}{伍尚\hspace{30pt}【{\akai 西皮原板}】未曾拆书泪先淋,纸上相逢父子情。平王思念临潼会,伍尚、伍员快回京。``外加走之''书后遁,骏马``十疋''莫留停。看罢书信喜不胜,}

伍员\hspace{30pt}哦!

\setlength{\hangindent}{56pt}{伍员\hspace{30pt}【{\akai 西皮散板}】伍员呐心中自沉吟。 }

伍尚\hspace{30pt}贤弟你再仔细观看。

伍员\hspace{30pt}不必观看,书信上言语,兄长可解?

伍尚\hspace{30pt}愚兄不解。

\setlength{\hangindent}{56pt}{伍员\hspace{30pt}既是调我弟兄进京,加官授爵,书信之上为何有``逃走''二字?~``外''加``走之''是``迯'',骏马``十疋''是``走''。分明是``迯走''二字啊,令人难解。({\akai 或}:~``外''加``走之''是``迯'',骏马``十疋''是``走''。分明是``迯走''二字。既是调我弟兄进京,加官授爵,为何有``逃走''二字?)}

伍尚\hspace{30pt}传下书人一问。

伍员\hspace{30pt}传下书人({\akai 或}:~唤下书人)。

(家院\hspace{25pt}下书人。)

鄢将师\hspace{20pt}在。

鄢将师\hspace{20pt}参见二位少老爷。

伍尚、\\
伍员\hspace{30pt}\raisebox{5pt}{罢了。}

伍尚、\\
伍员\hspace{30pt}\raisebox{5pt}{圣上驾安?}

鄢将师\hspace{20pt}我主驾安。

伍尚、\\
伍员\hspace{30pt}\raisebox{5pt}{太老爷?}

鄢将师\hspace{20pt}安泰。

伍尚、\\
伍员\hspace{30pt}\raisebox{5pt}{太夫人安泰?({\akai 或}:~太夫人?)}

鄢将师\hspace{20pt}福寿康宁。

伍尚、\\
伍员\hspace{30pt}\raisebox{5pt}{你叫什么名字?}

鄢将师\hspace{20pt}小人名叫鄢将师。

伍尚、\\
伍员\hspace{30pt}\raisebox{5pt}{你是新进相府还是久在相府?}

鄢将师\hspace{20pt}乃是新进相府。

伍尚、\\
伍员\hspace{30pt}\raisebox{5pt}{相府书信怎样({\akai 或}:~相府书信何人)交付与你?}

鄢将师\hspace{20pt}里封外传。

伍尚、\\
伍员\hspace{30pt}\raisebox{5pt}{什么时候?}

鄢将师\hspace{20pt}黄昏时候。

伍尚、\\
伍员\hspace{30pt}\raisebox{5pt}{圣上调我弟兄进京何事?}

鄢将师\hspace{20pt}这$\cdots{}\cdots{}$

伍尚、\\
伍员\hspace{30pt}\raisebox{5pt}{讲!}

鄢将师\hspace{20pt}不过是------呃,加官进爵而已。

伍员\hspace{30pt}呵!呵!呵呵呵$\cdots{}\cdots{}$({\hwfs 冷笑介})

伍尚、\\
伍员\hspace{30pt}\raisebox{5pt}{下去!}

鄢将师\hspace{20pt}是,是,是$\cdots{}\cdots{}$

鄢将师\hspace{20pt}好一个仔细的二老爷!险呐!

伍尚\hspace{30pt}贤弟你看如何?

伍员\hspace{30pt}下书人言语吱唔,凶多吉少,去之无益。

伍尚\hspace{30pt}既是爹娘亲笔书信,焉有不去之理?

伍员\hspace{30pt}有道是:~将在外,君命有所不受哇。

伍尚\hspace{30pt}唉!贤弟呀------

\setlength{\hangindent}{56pt}{伍尚\hspace{30pt}【{\akai 西皮原板}】昔日里有个商纣君,囚禁文王整七春。伯邑考许父丧了命,留得美名万古存。 }

伍员\hspace{30pt}兄长!

\setlength{\hangindent}{56pt}{伍员\hspace{30pt}【{\akai 西皮原板}】兄长说话欠思论,休把今人比古人。那文王被囚天注定,伯邑考粉身命里生成。既是平王【{\footnotesize 转}{\akai 西皮二六}】加官赠,就该有圣旨到樊城。若是爹娘修书信,为什么有``迯走''二字在书后存。怕的是失足罹陷阱\footnote{ 据李楠{\scriptsize 君}告知,余叔岩唱片套封此句记作``失足遗陷阱''。},那时节插翅亦难腾。我一心坐定樊城镇,愿作个不忠不孝人。}

\setlength{\hangindent}{56pt}{伍尚\hspace{30pt}【{\akai 西皮快板}】听他不肯进京城,背转身来自思忖:~长子就该遵父命,是好是歹走一程。 }

伍尚\hspace{30pt}贤弟,你是不进京的了?

伍员\hspace{30pt}凶多吉少,去之无益。

伍尚\hspace{30pt}待愚兄一人前去。

伍员\hspace{30pt}兄长一人前去({\akai 或}:~兄长要去),弟放心不下。可命家将跟随,一路之上也好侍奉鞍马。

伍尚\hspace{30pt}言得极是,贤弟安排。

伍员\hspace{30pt}唤家将。

伍尚、\\
伍员\hspace{30pt}\raisebox{5pt}{罢了!}

伍尚\hspace{30pt}二老爷有差。

伍员\hspace{30pt}命你跟随大老爷进京,一路之上侍奉鞍马。倘有事故,速报我知。

伍员\hspace{30pt}外厢备马。

伍尚\hspace{30pt}看衣改换。

\setlength{\hangindent}{56pt}{伍尚\hspace{30pt}【{\akai 西皮原板}】头上乌纱来摘定,紫绶官衣脱离身。教家将备马府门等,你老爷即刻就要登程。 }

伍员\hspace{30pt}看酒!

\setlength{\hangindent}{56pt}{伍员\hspace{30pt}【{\akai 西皮原板}】一封书信到樊城,拆散我弟兄两离分。叫家院看过酒一樽,弟与兄长【{\footnotesize 转}{\akai 西皮二六}】来饯行:~登山涉水多安稳,披星戴月奔都呃城。若是阖家同欢庆,在爹娘台前问安宁。倘若是家门遭不幸,报仇之事有弟伍员。非是小弟不从命,为的是``迯走''二字{\footnotesize 呃}解不明。兄长饮干杯中酒,一路平安早到京。}

\setlength{\hangindent}{56pt}{伍尚\hspace{30pt}【{\akai 西皮快板}】用手接过酒一樽,背转身来谢神灵。回头再对贤弟论,愚兄言来听分明:~倘若此去遭不幸,你是伍家报仇人。 }

\setlength{\hangindent}{56pt}{伍尚\hspace{30pt}【{\akai 西皮散板}】含悲忍泪跨金镫,不分昼夜奔都城。 }

(伍员\hspace{30pt}呃!)

\setlength{\hangindent}{56pt}{伍员\hspace{30pt}【{\akai 西皮快板}】兄长上马珠泪淋,教人难舍又难分。流泪眼观流泪眼,断肠人送断肠人。倘若家门遭不幸,杀上天子午朝门。吉凶二字未分定,稳坐樊城等信音。 }

\vspace{3pt}{\centerline{{[}{\hei 第二场}{]}}}\vspace{5pt}

\setlength{\hangindent}{56pt}{伍尚\hspace{30pt}【{\akai 西皮快板}】伍尚跨马奔帝京,眼观绿水青山行。无心观看路旁景,但愿早见二双亲。 }

\vspace{3pt}{\centerline{{[}{\hei 第三场}{]}}}\vspace{5pt}

费无极\hspace{20pt}({\akai 念})量小非君子,无毒不丈夫。

费无极\hspace{20pt}老夫------费无极。也曾命鄢将师诓那伍氏兄弟,未见到来。左右------伺候了。

鄢将师\hspace{20pt}({\akai 念})诓来棠公,回复将令。

鄢将师\hspace{20pt}参见相爷。

费无极\hspace{20pt}罢了。伍氏弟兄可曾诓到?

鄢将师\hspace{20pt}伍员不肯进京,伍尚现已诓到。

鄢将师\hspace{20pt}监中探父。

费无极\hspace{20pt}下面歇息。

鄢将师\hspace{20pt}谢相爷。

费无极\hspace{20pt}且住。$\cdots{}\cdots{}$伍尚诓到,不免上朝启奏,来------打轿上朝。

费无极\hspace{20pt}臣,费无极见驾,吾主千岁。

费无极\hspace{20pt}伍奢拿到。

楚平王\hspace{20pt}将他父子绑至金阶斩首,不得有误,领旨下殿。

费无极\hspace{20pt}领旨。

费无极\hspace{20pt}校尉等------打道法场!

\vspace{3pt}{\centerline{{[}{\hei 第四场}{]}}}\vspace{5pt}

费无极\hspace{20pt}来,将伍奢、伍尚父子绑了上来!

费无极\hspace{20pt}拿去开刀!

伍尚\hspace{30pt}唉!爹爹呀!

伍奢\hspace{30pt}蠢材!

\setlength{\hangindent}{56pt}{伍奢\hspace{30pt}【{\akai 西皮快板}】一见我儿绑金阶,骂声伍尚太无才:~枉读诗书为相宰,为父言语解不开。 }

\setlength{\hangindent}{56pt}{伍尚\hspace{30pt}【{\akai 西皮快板}】老爹爹休得把儿怪,书信一到怎不来?父子们犯了何条戒,因何捆绑在金阶? }

\setlength{\hangindent}{56pt}{伍奢\hspace{30pt}【{\akai 西皮快板}】平王无道纲常坏,父纳子媳礼不该。金顶轿换了银顶轿,为父谏奏惹祸灾。 }

\setlength{\hangindent}{56pt}{伍尚\hspace{30pt}【{\akai 西皮散板}】听一言来恼心怀,骂声奸佞如狼豺。恨不得将尔来踏坏------ }

\setlength{\hangindent}{56pt}{伍尚\hspace{30pt}【{\akai 西皮散板}】阴曹地府等尔来。 }

\setlength{\hangindent}{56pt}{伍奢\hspace{30pt}【{\akai 西皮散板}】父子们辜负了幽冥界! }

费无极\hspace{20pt}起过。

\setlength{\hangindent}{56pt}{
费无极\hspace{20pt}且住!伍奢、伍尚虽已斩首,今有伍员坐镇樊城,若不铲除,终是后患。不免二次上朝启奏。}

费无极\hspace{20pt}来!~$\cdots{}\cdots{}$上朝。

费无极\hspace{20pt}臣费无极二次上殿,吾主万岁。

楚平王\hspace{20pt}二次上殿,有何启奏?

费无极\hspace{20pt}伍奢、伍尚已然斩首,今有伍家满门老小尚在,伍员抗旨不遵,坐镇樊城,请我主裁处。

\setlength{\hangindent}{56pt}{
	楚平王\hspace{20pt}就请卿家带领人役,抄杀伍府满门大小。命武城黑带领三千人马,去至樊城,捉拿伍员进京问罪,领旨下殿。}

费无极\hspace{20pt}校尉等------抄杀伍府去者!

\vspace{3pt}{\centerline{{[}{\hei 第五场}{]}}}\vspace{5pt}

\setlength{\hangindent}{56pt}{太夫人\hspace{20pt}【{\akai 二黄摇板}】伍家世代忠良臣,平王无道掌乾坤。父纳子妻行不正,尽忠谏责显忠诚。 }

丫鬟\hspace{30pt}参见太夫人,大事不好了!

太夫人\hspace{20pt}何事惊慌?

丫鬟\hspace{30pt}太老爷与大老爷不知身犯何罪,斩首金阶。

太夫人\hspace{20pt}哎呀!

丫鬟\hspace{30pt}太夫人醒来!

\setlength{\hangindent}{56pt}{太夫人\hspace{20pt}【{\akai 二黄摇板}】听一言来冷汗淋,吓得七魂掉三魂。哭一声老相夫丧了命,{\textless{}\!{\bfseries\akai 哭头}\!\textgreater{}}老爷呀, }

\setlength{\hangindent}{56pt}{太夫人\hspace{20pt}【{\akai 二黄摇板}】昏王无道斩忠臣。回头我对家将论:~你快到樊城报信音。 }

家将\hspace{30pt}遵命!

(费无极{\hwfs 挎宝剑上})

家院\hspace{30pt}太夫人,大事不好了!

太夫人\hspace{20pt}何事惊慌?

家院\hspace{30pt}那费无极带领校尉抄杀满门来了!

太夫人\hspace{20pt}哎呀!

\setlength{\hangindent}{56pt}{太夫人\hspace{20pt}【{\akai 二黄摇板}】未觉乌鸦叫声震,今日大祸降临门。耳旁听得人声震,定是奸贼到来临。 }

太夫人\hspace{20pt}好贼!

\setlength{\hangindent}{56pt}{太夫人\hspace{20pt}【{\akai 二黄摇板}】手指奸贼骂高声:~祸国欺君害忠臣,拚着一死抛性命,或生或死一路行。 }

校尉\hspace{30pt}抄杀已毕。

费无极\hspace{20pt}上殿交旨!

(\textless{}\!{\bfseries\akai 急急风}\!\textgreater{}{\hwfs 下})

\vspace{3pt}{\centerline{{[}{\hei 第六场}{]}}}\vspace{5pt}

伍员\hspace{30pt}({\akai 念})独坐({\akai 或}:~闷坐)樊城心忧闷,吉凶二字未分明。

伍员\hspace{30pt}什么大事?

伍员\hspace{30pt}你才怎讲({\akai 或}:~你待怎讲)?!

伍员\hspace{30pt}\textless{}\!{\bfseries\akai 叫头}\!\textgreater{}爹娘!兄长!

伍员\hspace{30pt}哎呀!

\setlength{\hangindent}{56pt}{伍员\hspace{30pt}【{\akai 西皮导板}】听说爹娘丧了命呐, }

伍员\hspace{30pt}\textless{}\!{\bfseries\akai 三叫头}\!\textgreater{}爹爹!母亲!唉!兄长呃!

\setlength{\hangindent}{56pt}{伍员\hspace{30pt}【{\akai 西皮散板}】珠泪点点洒前胸。忍泪含悲家将问, }

伍员\hspace{30pt}家将!

伍员\hspace{30pt}\setlength{\hangindent}{56pt}{ 【{\akai 西皮散板}】犯罪的情由({\akai 或}:~被害的情由)说分明。 }

(伍员\hspace{30pt}好贼!)

\setlength{\hangindent}{56pt}{伍员\hspace{30pt}【{\akai 西皮散板}】骂声无极贼奸佞,无道昏君灭人伦。我若人马来点动,不忠的名儿万古闻({\akai 或}:~天下闻)。 }

伍员\hspace{30pt}再探!

伍员\hspace{30pt}\textless{}\!{\bfseries\akai 叫头}\!\textgreater{}家将!

伍员\hspace{30pt}今有武城黑带兵围困樊城,如何是好?

伍员\hspace{30pt}反得的?

伍员\hspace{30pt}如此------反、反、反呐!

\setlength{\hangindent}{56pt}{伍员\hspace{30pt}【{\akai 西皮散板}】兵来将挡({\akai 或}:~兵临将挡)自古论,水来自有土来屯。 }

\setlength{\hangindent}{56pt}{伍员\hspace{30pt}【{\akai 西皮散板}】家将备马传将令。 }

\setlength{\hangindent}{56pt}{伍员\hspace{30pt}【{\akai 西皮散板}】玲珑铠甲放光明({\akai 或}:~闪光明),三军与爷开城禁({\akai 或}:~人来与爷城开定)。 }

\vspace{3pt}{\centerline{{[}{\hei 第七场}{]}}}\vspace{5pt}

伍员\hspace{30pt}马前来的敢是武城黑?

伍员\hspace{30pt}武城黑!

伍员\hspace{30pt}带兵意欲何往?

伍员\hspace{30pt}一派胡言!放马过来!

(伍子胥{\hwfs 大边}({\hwfs 念完}``{放马过来}''{\hwfs 一合},{\hwfs 两合},{\hwfs 架住}),{\hwfs 钻完烟筒},{\hwfs 一扯两扯},{\hwfs 半个剜萝卜归小边},{\hwfs 幺二三},{\hwfs 把兜一别},{\hwfs 撤枪刺耳被勾走马腰封到大边},{\hwfs 往里一盖转身},{\hwfs 打前蓬头}、{\hwfs 后蓬头到小边},{\hwfs 转身接前蓬头}、{\hwfs 后蓬头又归大边},{\hwfs 转身},{\hwfs 掣肘},{\hwfs 拉转身},{\hwfs 幺二三},{\hwfs 把兜左转身走里边},{\hwfs 再一个把兜左转身过到小边},{\hwfs 外边打一个腰封},{\hwfs 两个腰封},{\hwfs 被盖右转身到里边}、{\hwfs 接一个腰封}、{\hwfs 两个腰封},{\hwfs 回身掣肘},{\hwfs 拉转身},{\hwfs 幺二三},{\hwfs 里外各一二三绕},{\hwfs 外里外一二三压},({\hwfs 往}){\hwfs 里外各一盖两盖},{\hwfs 打腰封},武{\hwfs 左转身},伍{\hwfs 往里漫}武{\hwfs 头}、{\hwfs 扎脖},武{\hwfs 右手推出去},{\hwfs 亮住}。\footnote{ {《战樊城》是楚平王派武城黑去捉拿伍子胥},{与伍会战是殊死之战},{开打不能太少}。{余叔岩打的这套小快枪由钱金福、余叔岩一起设计。}})

(武{\hwfs 下}伍{\hwfs 接耍下场}:~{\hwfs 出枪提枪花转身},{\hwfs 三个提枪花},{\hwfs 出枪右手转枪鐏},{\hwfs 反手扶枪把},{\hwfs 左手在上正手扶枪杆},{\hwfs 双手抱枪},{\hwfs 右腿抬},{\hwfs 左腿单腿站},{\hwfs 枪鐏放在右腿膝盖后马面上},{\hwfs 枪头向上矗住},{\hwfs 右手打鐏将枪打出去},{\hwfs 右手抄枪杆下端接住再扔枪右转身面向下场门左手接枪杆下端},{\hwfs 在左腰间平端},{\hwfs 右手举拳过顶}\footnote{ ``顶''字在《京剧老生把子见闻录》\upcite{XQYS1-32_1983}一文中误作``项'',据《京剧新序》一书更正。},{\hwfs 左腿抬},{\hwfs 右腿单腿站亮住},{\hwfs 下场门下}。)

\setlength{\hangindent}{56pt}{伍员\hspace{30pt}且住,武城黑来得厉害,不免伤他一箭~!\footnote{ 李元皓{\scriptsize 君}认为作``赏他一箭''。据陈超老师告知,之后的开打与目前舞台上的很不一样。{\bfseries\akai 夹鞭}的身段也不一样:~是{\hwfs 先出鞭},{\hwfs 再抬腿同时转腰},{\hwfs 不抬弓}。武{\hwfs 也不被射死},{\hwfs 搂扑虎后},{\hwfs 起身}({\hwfs 这个时间伍刚好完成夹鞭身段}),{\hwfs 挡脸下。}}}

\setlength{\hangindent}{56pt}{伍员\hspace{30pt}【{\akai 西皮散板}】张弓布矢威风凛({\akai 或}:~本帅开弓放雕翎),武城黑鼠窜逃了生({\akai 或}:~带箭逃了生)。本帅逃出呃棠邑郡\footnote{ 《战樊城》别名《出棠邑》。}({\akai 或}:~天罗境)呐,可叹我的家将丧残生。 }

伍员\hspace{30pt}\textless{}\!{\bfseries\akai 三叫头}\!\textgreater{}爹娘!兄长!哎!家将啊,呃$\cdots{}\cdots{}$({\hwfs 哭介})
}
