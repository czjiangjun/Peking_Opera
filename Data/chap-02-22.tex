\newpage
\subsubsection{\large \hei 七星灯~\protect\footnote{陈超老师介绍:~刘曾复先生所传的是贾丽川家的路子。}}
\addcontentsline{toc}{subsection}{\hei 七星灯}

\hangafter=1                   %2. 设置从第1⾏之后开始悬挂缩进  %
\setlength{\parindent}{0pt}{

\vspace{3pt}{\centerline{{[}{\hei 第一场}{]}}}\vspace{5pt}

\setlength{\hangindent}{56pt}{诸葛亮\hspace{20pt}({\akai 念})万事不由人作主,一心难与命争衡。}

\setlength{\hangindent}{56pt}{旗牌\hspace{30pt}参见丞相。}

\setlength{\hangindent}{56pt}{诸葛亮\hspace{20pt}你回来了?}

\setlength{\hangindent}{56pt}{旗牌\hspace{30pt}回来了。}

\setlength{\hangindent}{56pt}{诸葛亮\hspace{20pt}司马看了衣服、书信有何举动?}

\setlength{\hangindent}{56pt}{旗牌\hspace{30pt}司马受了巾帼女衣、看了书信,并不嗔怒,呃,反请小人饮宴,席中只问丞相寝食及事之烦简如何,饭间未提军旅之事。}

\setlength{\hangindent}{56pt}{诸葛亮\hspace{20pt}尔如何抗对?}

\setlength{\hangindent}{56pt}{旗牌\hspace{30pt}呃,小人言道:~``丞相夙兴夜寐,罚二十以上皆亲览焉。所啖之食,不过数升。''}

\setlength{\hangindent}{56pt}{诸葛亮\hspace{20pt}司马何言?}

\setlength{\hangindent}{56pt}{旗牌\hspace{30pt}呃,司马并无言语。}

\setlength{\hangindent}{56pt}{诸葛亮\hspace{20pt}呃------下去。}

\setlength{\hangindent}{56pt}{诸葛亮\hspace{20pt}唉,司马深知吾也!}

\setlength{\hangindent}{56pt}{诸葛亮\hspace{20pt}【{\akai 二黄原板}】仰面朝天【{\footnotesize 转}{\akai 二黄慢板}】自己嗟叹,司马懿可算得将中魁元。送脂粉和钗裙不恼不怨,反与那旗牌官酒食来餐。有刚有柔是好汉,我诸葛比司马难上加难。先帝爷下南阳君臣相见,感恩}深重要扭转汉室河山。博望坡烧夏侯({\akai 或}:~博望坡烧曹兵;博望坡烧贼兵)初次交战,借东风助周郎火烧战船。烧藤甲用火攻孟获丧胆({\akai 或}:~蛮夷丧胆),谁不知诸葛亮智能扭天。到如今与司马两下会战,葫芦峪设地雷安排机关。我料他父子们必遭此难,又谁知天不遂也是枉然。一时里心血涌浑身是汗\footnote{刘曾复先生钞本作``遍身是汗''。},

\setlength{\hangindent}{56pt}{诸葛亮\hspace{20pt}【{\akai 二黄摇板}】传姜维和魏延速到帐前。

\setlength{\hangindent}{56pt}{旗牌\hspace{30pt}姜维、魏延进帐。}

姜维\\魏延\raisebox{5pt}{\hspace{30pt}来也!}

\setlength{\hangindent}{56pt}{魏延\hspace{30pt}【{\akai 二黄摇板}】帐上一声唤,}

\setlength{\hangindent}{56pt}{姜维\hspace{30pt}【{\akai 二黄摇板}】上前问根源。}

姜维\\魏延\raisebox{5pt}{\hspace{30pt} 参见丞相,有何将令?}

\setlength{\hangindent}{56pt}{诸葛亮\hspace{20pt}魏延。}

\setlength{\hangindent}{56pt}{魏延\hspace{30pt}在!}

\setlength{\hangindent}{56pt}{诸葛亮\hspace{20pt}命你巡营瞭哨,司马叫阵,不可出兵,紧守大营,不得有误!}

\setlength{\hangindent}{56pt}{魏延\hspace{30pt}得令。}

\setlength{\hangindent}{56pt}{诸葛亮\hspace{20pt}姜维。}

\setlength{\hangindent}{56pt}{姜维\hspace{30pt}在。}

\setlength{\hangindent}{56pt}{诸葛亮\hspace{20pt}命你随我后帐安排七星祭坛,一干人等,勿得入内。}

\setlength{\hangindent}{56pt}{姜维\hspace{30pt}遵令}({\akai 或}:~{遵命)。}

\setlength{\hangindent}{56pt}{诸葛亮\hspace{20pt}唉!}

\setlength{\hangindent}{56pt}{诸葛亮\hspace{20pt}【{\akai 二黄摇板}】设坛拜星求北斗,但愿天意早回头。三寸气在千般用,一旦无常万事休。扫荡中原难回首,}

\setlength{\hangindent}{56pt}{诸葛亮\hspace{20pt}【{\akai 二黄摇板}】怕的是天意不遂不自由。}

\vspace{3pt}{\centerline{{[}{\hei 第二场}{]}}}\vspace{5pt}

\setlength{\hangindent}{56pt}{司马懿\hspace{20pt}【{\akai 二黄导板}】谯楼鼓打罢了初更时分,}

\setlength{\hangindent}{56pt}{司马懿\hspace{20pt}【{\akai 回龙}】静悄悄出魏营观看天星。}

\setlength{\hangindent}{56pt}{司马懿\hspace{20pt}【{\akai 二黄原板}】叫人来前引路高岗来进,司马懿观天象细算详情:~观东方甲乙木木能生火,观南方丙丁火火能克金。正西方庚辛金金能生水,观北方壬癸水水遇土屯。佔中央戊己土仔细看定,}

\setlength{\hangindent}{56pt}{司马懿\hspace{20pt}【{\akai 二黄摇板}】五行生克观不清。}

\setlength{\hangindent}{56pt}{司马懿\hspace{20pt}【{\akai 二黄摇板}】北斗星垣来观看,主星暗淡光不明。看罢天象心拿稳,}

\setlength{\hangindent}{56pt}{司马懿\hspace{20pt}回营!}

\setlength{\hangindent}{56pt}{司马懿\hspace{20pt}【{\akai 二黄摇板}】安排巧计擒孔明。}

\vspace{3pt}{\centerline{{[}{\hei 第三场}\protect\footnote{陈超老师介绍了这一场相关的舞台布局及调度。}{]}}\vspace{5pt}

\setlength{\hangindent}{56pt}{姜维\hspace{30pt}有请丞相!}

\setlength{\hangindent}{56pt}{诸葛亮\hspace{20pt}({\akai 内})先王呀!}

\setlength{\hangindent}{56pt}{({\hwfs 正场小座},``{\hwfs 七星灯}''{\hwfs 不摆在正场桌},{\hwfs 而是摆在下场门斜场},诸葛亮{\hwfs 拄宝剑上})}

\setlength{\hangindent}{56pt}{诸葛亮\hspace{20pt}【{\akai 二黄慢板}】为汉家把我的心血用尽,都只为先帝爷托孤之恩。执法剑进祭坛({\akai 或}:~执宝剑上坛台)实难扎挣,}

\setlength{\hangindent}{56pt}{诸葛亮\hspace{20pt}【{\akai 二黄原板}】险些儿把老夫跌倒埃尘。}

\setlength{\hangindent}{56pt}{诸葛亮\hspace{20pt}({\akai 念})亮,谨书尺素,上告穹苍:~伏望天慈,俯垂鉴听:~亮生于乱世,甘老林泉;承昭烈皇帝三顾之恩,托孤之重,誓讨国贼,永延汉祚}\footnote{《三国演义》原文为``{永延汉}祀''。}{。上求北斗,曲延臣算,非敢妄祈,实由------唉------情切。诶,呃$\cdots{}\cdots{}$({\hwfs 哭介})}

\setlength{\hangindent}{56pt}{(\textless{}\!{\bfseries\akai 小开门}\!\textgreater{}{\hwfs 烧符箓})}

\setlength{\hangindent}{56pt}{诸葛亮\hspace{20pt}上苍呐!}

\setlength{\hangindent}{56pt}{诸葛亮\hspace{20pt}【{\akai 二黄原板}】诸葛亮不敢扭天行,为的是我主锦乾坤。拜南斗和北斗}\footnote{刘曾复先生钞本作``南斗合北斗''。}{赐我阳寿,掌簿官执笔吏留下人情。佔中央戊己土深深拜定,}

\setlength{\hangindent}{56pt}{(诸葛亮{\hwfs 叩头},{\hwfs 拜后下坛台},{\hwfs 踱步至上场门},{\hwfs 回身看星灯})}

\setlength{\hangindent}{56pt}{诸葛亮\hspace{20pt}【{\akai 二黄摇板}】见将星比往常显见光明。}

\setlength{\hangindent}{56pt}{诸葛亮\hspace{20pt}【{\akai 二黄摇板}】虽然是星明亮吉凶未定,}

\setlength{\hangindent}{56pt}{(诸葛亮{\hwfs 归小座})}

\setlength{\hangindent}{56pt}{诸葛亮\hspace{20pt}【{\akai 二黄散板}】怕的是({\akai 或}:~怕只怕)天意难违大事难成。}

\setlength{\hangindent}{56pt}{魏延\hspace{30pt}【{\akai 二黄摇板}】司马来踏营,近前说分明。}

\setlength{\hangindent}{56pt}{诸葛亮\hspace{20pt}【{\akai 二黄摇板}】这是我大限有一定,魏延扑熄我的本命灯。将本命灯撇在尘埃地,}

\setlength{\hangindent}{56pt}{姜维\hspace{30pt}【{\akai 二黄摇板}】丞相发怒为何情。}

\setlength{\hangindent}{56pt}{诸葛亮\hspace{20pt}【{\akai 二黄摇板}】我拜斗今日六天整,堪堪}\footnote{刘曾复先生钞本作``看看''。}{七天大功成。恨魏延他把我本命灯扑熄,我性命就要哇一旦倾。}

\setlength{\hangindent}{56pt}{姜维\hspace{30pt}啊?!}

\setlength{\hangindent}{56pt}{姜维\hspace{30pt}【{\akai 二黄摇板}】听一言来怒气生,魏延贼子起反心。手执宝剑将尔斩,}

\setlength{\hangindent}{56pt}{魏延\hspace{30pt}你要斩哪个?}

\setlength{\hangindent}{56pt}{姜维\hspace{30pt}要杀你。}

\setlength{\hangindent}{56pt}{魏延\hspace{30pt}你杀不得。}

\setlength{\hangindent}{56pt}{诸葛亮\hspace{20pt}将军!}

\setlength{\hangindent}{56pt}{诸葛亮\hspace{20pt}【{\akai 二黄摇板}】将军息怒且消停。}

\setlength{\hangindent}{56pt}{诸葛亮\hspace{20pt}魏延。}

\setlength{\hangindent}{56pt}{魏延\hspace{30pt}在。}

\setlength{\hangindent}{56pt}{诸葛亮\hspace{20pt}莫非司马前来踏营?}

\setlength{\hangindent}{56pt}{魏延\hspace{30pt}正是。}

\setlength{\hangindent}{56pt}{诸葛亮\hspace{20pt}前去抵挡,出帐去罢!}

\setlength{\hangindent}{56pt}{魏延\hspace{30pt}遵命!

\setlength{\hangindent}{56pt}{魏延\hspace{30pt}({\akai 念})堪堪孔明不长久,管教蜀营众将休!}

\setlength{\hangindent}{56pt}{魏延\hspace{30pt}哼!}

\setlength{\hangindent}{56pt}{诸葛亮\hspace{20pt}姜维搀我出坛!}

\setlength{\hangindent}{56pt}{诸葛亮\hspace{20pt}【{\akai 二黄摇板}】姜维后营一声请,快快请出李大人。}

\setlength{\hangindent}{56pt}{姜维\hspace{30pt}有请李大人!}

\setlength{\hangindent}{56pt}{李福\hspace{30pt}({\akai 内})来也!}

\setlength{\hangindent}{56pt}{李福\hspace{30pt}【{\akai 二黄摇板}】忽听帐上一声请,急忙进帐看分明。}

\setlength{\hangindent}{56pt}{李福\hspace{30pt}参见丞相!}

\setlength{\hangindent}{56pt}{诸葛亮\hspace{20pt}李大人。}

\setlength{\hangindent}{56pt}{李福\hspace{30pt}在。}

\setlength{\hangindent}{56pt}{诸葛亮\hspace{20pt}这有表章一轴,连夜送往成都,替吾转奏,请吾主龙目御览。}

\setlength{\hangindent}{56pt}{李福\hspace{30pt}遵命。}

\setlength{\hangindent}{56pt}{诸葛亮\hspace{20pt}搀扶!}

\setlength{\hangindent}{56pt}{诸葛亮\hspace{20pt}【{\akai 二黄摇板}】远望成都忙跪定,拜谢我主爵禄恩。羞愧难见刘先主,李大人速速转奏快快登程。}

\setlength{\hangindent}{56pt}{李福\hspace{30pt}【{\akai 二黄摇板}】辞别丞相忙登程,不分昼夜奔都城。}

\setlength{\hangindent}{56pt}{诸葛亮\hspace{20pt}姜维!}

\setlength{\hangindent}{56pt}{姜维\hspace{30pt}在!}

\setlength{\hangindent}{56pt}{诸葛亮\hspace{20pt}听我吩咐!}

\setlength{\hangindent}{56pt}{姜维\hspace{30pt}啊!}

\setlength{\hangindent}{56pt}{诸葛亮\hspace{20pt}【{\akai 二黄碰板三眼}】我和你虽为将帅倒有那师徒之义,}

\setlength{\hangindent}{56pt}{诸葛亮\hspace{20pt}【{\akai 二黄原板}】必须要秉忠心扶保华夷。一封锦囊交与你,内藏着妙算与神机。我死后三件大事托与你,一桩桩一件件莫要泄机:~第一件我死后休得挂孝,第二件必须要缓缓移营。第三件我死后那魏延必反,}

\setlength{\hangindent}{56pt}{姜维\hspace{30pt}啊?!}

\setlength{\hangindent}{56pt}{诸葛亮\hspace{20pt}【{\akai 二黄散板}】我自有妙计除此人。我将这奇门遁甲传授你,阵阵不离此图形。这一弩能发十条箭,九伐中原你担承。将军与我传将令,快传那杨仪、马岱与王平。}

\setlength{\hangindent}{56pt}{姜维\hspace{30pt}杨仪、马岱、王平速速进帐!}

杨仪\\马岱\hspace{30pt}【{\akai 二黄摇板}】丞相帐中传将令,一同上前看分明。\\王平

杨仪\\马岱\hspace{30pt}丞相醒来!\\王平

\setlength{\hangindent}{56pt}{诸葛亮\hspace{20pt}【{\akai 二黄摇板}】指望}\footnote{刘曾复先生钞本作``只望''。}{霸业兴炎汉,谁知半途不周全。猛然睁开昏花眼,又只见众将官站立面前。}

\setlength{\hangindent}{56pt}{诸葛亮\hspace{20pt}杨仪!}

\setlength{\hangindent}{56pt}{杨仪\hspace{30pt}在。(\textless{}\!{\bfseries\akai 小拉子}\!\textgreater{})}

\setlength{\hangindent}{56pt}{诸葛亮\hspace{20pt}【{\akai 二黄摇板}】我死后军师大印你掌管,事事谨慎要周全。我今与你这小柬,我死之后再来观。}

\setlength{\hangindent}{56pt}{杨仪\hspace{30pt}遵命。(\textless{}\!{\bfseries\akai 住头}\!\textgreater{})}

\setlength{\hangindent}{56pt}{诸葛亮\hspace{20pt}子均!}

\setlength{\hangindent}{56pt}{王平\hspace{30pt}在。(\textless{}\!{\bfseries\akai 小拉子}\!\textgreater{})}

\setlength{\hangindent}{56pt}{诸葛亮\hspace{20pt}【{\akai 二黄摇板}】王子均近前听召唤,一封小柬带身边。事到头来}\footnote{刘曾复先生钞本作``事到临头''。}{再观看,内有如此与这般。}

\setlength{\hangindent}{56pt}{王平\hspace{30pt}遵命。}

\setlength{\hangindent}{56pt}{诸葛亮\hspace{20pt}马岱!}

\setlength{\hangindent}{56pt}{马岱\hspace{30pt}在。(\textless{}\!{\bfseries\akai 小拉子}\!\textgreater{})}

\setlength{\hangindent}{56pt}{诸葛亮\hspace{20pt}【{\akai 二黄摇板}】西凉马岱听我言,我有言来记心间。倘若是魏延来造反,这封小柬临阵观。}

\setlength{\hangindent}{56pt}{马岱\hspace{30pt}遵命。}

\setlength{\hangindent}{56pt}{诸葛亮\hspace{20pt}【{\akai 二黄摇板}】众将官搀扶我吾主叩见,诸葛亮在营中拜别龙颜。叩罢头抽身起心血上泛,}

\setlength{\hangindent}{56pt}{诸葛亮\hspace{20pt}呜$\cdots{}\cdots{}$({\hwfs 吐血介})}

\setlength{\hangindent}{56pt}{诸葛亮\hspace{20pt}【{\akai 二黄摇板}】我面前站定了庞统士元。}

\setlength{\hangindent}{56pt}{诸葛亮\hspace{20pt}【{\akai 二黄摇板}】在荆州对把八字算,我二人各有不周全。我算他落凤坡前身带箭,他算我难逃五丈原。霎时间胸内痛({\akai 或}:~霎时间心内痛)心血上泛,}

\setlength{\hangindent}{56pt}{诸葛亮\hspace{20pt}呜$\cdots{}\cdots{}$({\hwfs 吐血介})}

\setlength{\hangindent}{56pt}{诸葛亮\hspace{20pt}【{\akai 二黄摇板}】昏沉沉一旦间命归九泉。}

\setlength{\hangindent}{56pt}{众\hspace{40pt}丞相啊!}

\setlength{\hangindent}{56pt}{李福\hspace{30pt}丞相钧体如何?}

\setlength{\hangindent}{56pt}{众\hspace{40pt}已归仙境。}

\setlength{\hangindent}{56pt}{李福\hspace{30pt}哎呀,误了吾主大事了!}

\setlength{\hangindent}{56pt}{诸葛亮\hspace{20pt}嗯哼$\cdots{}\cdots{}$}

\setlength{\hangindent}{56pt}{姜维\hspace{30pt}哦,丞相醒转!大人有何圣谕,快快禀来!}

\setlength{\hangindent}{56pt}{李福\hspace{30pt}启禀丞相:~万岁问道:~丞相之后,何人接替。}

\setlength{\hangindent}{56pt}{诸葛亮\hspace{20pt}蒋公琰。}

\setlength{\hangindent}{56pt}{李福\hspace{30pt}公琰之后?}

\setlength{\hangindent}{56pt}{诸葛亮\hspace{20pt}费文伟。}

\setlength{\hangindent}{56pt}{李福\hspace{30pt}文伟之后?}

\setlength{\hangindent}{56pt}{诸葛亮\hspace{20pt}三国归于$\cdots{}\cdots{}$}

\setlength{\hangindent}{56pt}{众\hspace{40pt}丞相啊$\cdots{}\cdots{}$}

\setlength{\hangindent}{56pt}{姜维\hspace{30pt}列公且免悲泪,待我打开丞相钧谕观看。}

\setlength{\hangindent}{56pt}{姜维\hspace{30pt}原来如此。}

\setlength{\hangindent}{56pt}{姜维\hspace{30pt}(丞相命我等)用沉香木塑成钧体,安放四轮车上。倘若司马踏营,将车推至阵前,司马必然不战自退。}

\setlength{\hangindent}{56pt}{众\hspace{40pt}原来如此。}

\setlength{\hangindent}{56pt}{姜维\hspace{30pt}你我后营安排,准备一切便了!}

\setlength{\hangindent}{56pt}{众\hspace{40pt}请呐!}

}
