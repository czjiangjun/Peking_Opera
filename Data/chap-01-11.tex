\newpage
\subsubsection{\large\hei {浣纱记~{\small 之}~伍员}}
\addcontentsline{toc}{subsection}{\hei 浣纱记~\small{之}~伍员}

\hangafter=1                   %2. 设置从第1⾏之后开始悬挂缩进  %
\setlength{\parindent}{0pt}{
{\centerline{{[}{\hei 第一场}{]}}}\vspace{5pt}
且住!前有大江,后有追兵,如何是好?

远望扁舟一叶,待我呼唤:小舟过来!

渡我过江,自当酬谢。

哦,原来是老丈。

我乃落难之人,望求方便。

有劳了!

老丈作歌何意?

原来如此。

这$\cdots{}\cdots{}$

我若说出名姓,又恐连累老丈。

愚下姓伍名员字子胥,楚国人也。

岂敢。

多谢老丈。

啊,老丈,这有祖传宝剑,上有七星,价值连城。赠与老丈,以表谢意。

多承美意。

老丈尊姓大名,日后也好图报。

如此老丈------

渔丈人。

啊------呵呵哈哈哈$\cdots{}\cdots{}$({\hwfs 笑介})

告辞了。

\setlength{\hangindent}{56pt}{【{\akai 西皮摇板}】老丈渡我过江河,千金谢仪不为多。辞别老丈忙走却, }

\setlength{\hangindent}{56pt}{【{\akai 西皮摇板}】还有一事再相托。 }

啊,老丈。我虽过江,后面追兵甚急。倘若到来,老丈千万莫提你我之事。

在哪里?

哎呀!({\akai 念})\textless{}\!{\bfseries\akai 扑灯蛾}\!\textgreater{}老丈投江河,投江河,不由人,珠泪落。世上多少英雄汉,好教我,难话说。

且住!老丈投江,我急急走去呀!

\vspace{3pt}{\centerline{{[}{\hei 第二场}{]}}}\vspace{5pt}

\setlength{\hangindent}{56pt}{【{\akai 西皮导板}】豪杰打马奔吴国, }

\setlength{\hangindent}{56pt}{【{\akai 西皮快板}】龙离沧海虎离窝。樊城一呼人百诺,令出山岳不敢挪\footnote{ 樊百乐{\scriptsize 君}介绍,刘曾复先生曾说明:~``令出山摇''是``令出山岳动,言发鬼神惊''的讹误。因昆腔北曲``岳''念``要(\textrm{yào})''音,后误作``山摇'',因袭至今,此处从俗。其余剧目中亦同。}。力举千斤伍盟府,各国不敢动干戈。天下的诸侯皆服我,秦邦惧怕求讲和。也是我当初做事错,大不该秦楚临潼做媒妁。可叹我一家无有结果,负仇含冤奔吴国。秋半蓉花溪边落,见一娘行浣纱罗。(貌似三月桃花朵,柳眉杏眼似秋波。\footnote{ 陈超老师介绍,这两句是贾丽川的词句,台上一般不唱,刘曾复先生为保留贾丽川的词句而唱了这两句。})一路行来腹中饿,她篮中有饭又有馍。上前求她赒济\footnote{ 赒济:~接济、救助之意。}我, }

\setlength{\hangindent}{56pt}{【{\akai 西皮摇板}】自觉惭愧难定夺。 }

娘行有礼。我乃落难之人,穷途无食,还望娘行赒济。

唉!娘行听了!

\setlength{\hangindent}{56pt}{【{\akai 西皮二六}】未曾开言心难过,两眼不住泪如梭。家住在楚国监利玉皇阁,我父伍相扶保山河。伍子胥,就是我,临潼会斗宝压倒万国。恨平王无道纲常错,父纳子媳礼不合。我的父谏奏反遭大祸,可怜我的满门一家大小见阎罗。只剩下伍员人一个,弃走楚樊逃奔吴国。昭关赴险得逃脱,幸遇着渔父 }

【{\footnotesize 转}{\akai 西皮快板}】渡江河。一路行来腹中饿,只见篮中饭与馍。望求娘行赒济我,此生不忘这恩德。

\setlength{\hangindent}{56pt}{【{\akai 西皮摇板}】这也是苍天怜惜我,凭空降下女娇娥。忙将饭食来取过,千金谢礼不为多。有朝伍员时转过,不忘娘行这恩德。 }

\setlength{\hangindent}{56pt}{【{\akai 西皮摇板}】娘行一言提醒我,男女交谈礼不合。伍员溪边忙走过, }

\setlength{\hangindent}{56pt}{【{\akai 西皮摇板}】再把娘行来嘱托。 }

我今此去,倘有楚国人马至此,千万莫说我打此经过。

在哪里?

哎呀!

\setlength{\hangindent}{56pt}{【{\akai 西皮散板}】一见------浣纱女子投江河。可叹你为我自尽\textless{}\!{\bfseries\akai 哭头}\!\textgreater{}死,娘行啊, }

\setlength{\hangindent}{56pt}{【{\akai 西皮散板}】留得清白万古播。 }

({\akai 念})尔浣纱,我行乞。我腹饱,尔身溺\footnote{ ``我腹饱,尔身溺''这一句,陈超老师跟刘曾复先生学的是``{我腹果},{尔身溺}''。}。十年之后,千金报德,

千金报德。

且住!事已至此,我当急行投吴去也。
}
