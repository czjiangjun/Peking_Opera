\newpage
\phantomsection %实现目录的正确跳转
\section*{\large\hei {青石山~{\small 之}~吕祖、关帝}}
\addcontentsline{toc}{section}{\hei 青石山~{\small 之}~吕祖、关帝}

\hangafter=1                   %2. 设置从第1⾏之后开始悬挂缩进  %
\setlength{\parindent}{0pt}{

\vspace{3pt}{\centerline{{[}{\hei 第一场}{]}}}\vspace{5pt}
\setlength{\hangindent}{52pt}{吕祖\hspace{30pt}({\akai 内})【{\akai 二黄导板}】赴蟠桃辞王母离了仙境,}

\setlength{\hangindent}{52pt}{吕祖\hspace{30pt}【{\akai 回龙}】迈步儿出洞府散淡精神。}

\setlength{\hangindent}{52pt}{吕祖\hspace{30pt}【{\akai 二黄原板}】青是山({\akai 或}:~青的山)绿是水一派美景,有苍松和翠柏密密层层。莲池内鱼戏水甚是沉静({\akai 或}:~看来沉静),有糜鹿衔灵草倒也安宁。尘世里好一似蓬莱仙境({\akai 或}:~胜过了蓬莱仙境),看此处正好作养性修身。}

\setlength{\hangindent}{52pt}{吕祖\hspace{30pt}抬头观看。}

\setlength{\hangindent}{52pt}{吕祖\hspace{30pt}为何这等模样? }

\setlength{\hangindent}{52pt}{吕祖\hspace{30pt}你的法力呢? }

\setlength{\hangindent}{52pt}{吕祖\hspace{30pt}不消。({\akai 或}:~罢了。)}

\setlength{\hangindent}{52pt}{吕祖\hspace{30pt}好大的妖气。}

\setlength{\hangindent}{52pt}{吕祖\hspace{30pt}此乃九尾玄狐。}

\setlength{\hangindent}{52pt}{吕祖\hspace{30pt}清香一枝,法鼓三通。}

\setlength{\hangindent}{52pt}{吕祖\hspace{30pt}善哉呀善哉。}

\setlength{\hangindent}{52pt}{吕祖\hspace{30pt}【{\akai 二黄原板}】稳坐在法坛上三光照定,提羊毫写牒文上达天庭。都只为青石山妖狐狂狞,害得那小周生不得安宁。望神圣发慈悲神兵遣定,灭却了这妖狐({\akai 或}:~这妖魔)黎民太平。}

\setlength{\hangindent}{52pt}{吕祖\hspace{30pt}一祭,天清;~二祭,地靖;~三祭,百宁。}

\setlength{\hangindent}{52pt}{吕祖\hspace{30pt}值日功曹何在? }

\setlength{\hangindent}{52pt}{吕祖\hspace{30pt}牒文一道,烦劳尊神({\akai 或}:~有劳尊神)南天门送达。}

\vspace{3pt}{\centerline{{[}{\hei 第二场}{]}}}\vspace{5pt}

\setlength{\hangindent}{52pt}{关帝\hspace{30pt}【{\akai 唢呐二黄导板}】想当年破黄巾威风缭绕,}

\setlength{\hangindent}{52pt}{关帝\hspace{30pt}【{\akai 回龙}】扶保我大兄王锦绣龙朝。}

\setlength{\hangindent}{52pt}{关帝\hspace{30pt}【{\akai 唢呐二黄原板}】都只为吕法师牒文来到,因此上统神将下了九霄。\footnote{据朱家溍先生的演出录像\upcite{Recoder_Qingshishan},此处不念的那四句为``蚕眉凤目美髯飘,手中青龙偃月刀。胯下赤兔千里马,一片丹心保汉朝。''\\《京剧汇编》第五十二集~李万春~藏本中此四句为``蚕眉凤目美髯飘,红光照耀偃月刀。胯下赤兔追风马,一片忠心保天曹。''其中``胯下''作``跨下''。}}

\setlength{\hangindent}{52pt}{关帝\hspace{30pt}吕法师牒文到来。}

\setlength{\hangindent}{52pt}{关帝\hspace{30pt}众神将! }

\setlength{\hangindent}{52pt}{关帝\hspace{30pt}驾起祥云者。}

\vspace{3pt}{\centerline{{[}{\hei 第三场}{]}}}\vspace{5pt}

\setlength{\hangindent}{52pt}{吕祖\hspace{30pt}尊神请了! }

\setlength{\hangindent}{52pt}{关帝\hspace{30pt}法师请了!~相召有何见谕? }

\setlength{\hangindent}{52pt}{吕祖\hspace{30pt}今有青石山妖狐作乱,还望尊神收服({\akai 或}:~降伏者)。}

\setlength{\hangindent}{52pt}{关帝\hspace{30pt}法师稳坐坛台,看吾神降妖者。}

\setlength{\hangindent}{52pt}{吕祖\hspace{30pt}圣寿无疆。}

\setlength{\hangindent}{52pt}{关帝\hspace{30pt}众神将! }

\setlength{\hangindent}{52pt}{(众\hspace{40pt}有!)}

\setlength{\hangindent}{52pt}{关帝\hspace{30pt}撒下天罗地网。}

\setlength{\hangindent}{52pt}{(众\hspace{40pt}啊!)}

\setlength{\hangindent}{52pt}{关帝\hspace{30pt}关平、仓将,前去降妖者。}

\vspace{3pt}{\centerline{{[}{\hei 第四场}{]}}}\vspace{5pt}

\setlength{\hangindent}{52pt}{吕祖\hspace{30pt}且住。九尾玄狐十分狂狞({\akai 或}:~妖狐十分狂狞),天灵灵,地灵灵。}

\setlength{\hangindent}{52pt}{吕祖\hspace{30pt}周元丰\footnote{按《三国演义》,周仓无字。《关羽戏集:~李洪春演出本》\upcite{GuanYu_Opera}中载,周仓字元福;~此处从《京剧汇编》第五十二集~李万春~藏本。}何在?}

