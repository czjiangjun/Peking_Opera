\newpage
\subsubsection{\large\hei {洪羊洞\protect\footnote{剧本参照《刘曾复京剧文存》\upcite{Liu-Wencun}中收录的《\textless{}\!\,洪羊洞\,\!\textgreater{}说戏誊稿》并结合刘曾复先生说戏录音整理。}~{\small 之}~杨延昭、老令公}}
\addcontentsline{toc}{subsection}{\hei 洪羊洞~{\small 之}~杨延昭、老令公}

\hangafter=1                   %2. 设置从第1⾏之后开始悬挂缩进  %}}
\setlength{\parindent}{0pt}{

{\vspace{3pt}{\centerline{{[}{\hei 第一场}{]}}}\vspace{5pt}}

(\textless{}\!{\bfseries\akai 冲头}\!\textgreater{}{\hwfs 切住},{\hwfs 起更},{\hwfs 要稍快},用\textless{}\!{\bfseries\akai 咚哐}\!\textgreater{}{\hwfs 收},{\hwfs 一场}\textless{}\!{\bfseries\akai 大锣打上}\!\textgreater{},{\hwfs 四}鬼卒{\hwfs 引}老令公魂子{\hwfs 上},鬼卒{\hwfs 站门},令公{\hwfs 到台口},\textless{}\!{\bfseries\akai 大锣归位}\!\textgreater{})

\setlength{\hangindent}{56pt}{老令公\hspace{20pt}({\akai 念})生前为大将,死后做忠魂。(\textless{}\!{\bfseries\akai 住头}\!\textgreater{})}

\setlength{\hangindent}{56pt}{老令公\hspace{20pt}吾乃继业灵魂({\akai 或}:~阴魂;鬼魂)是也。(\textless{}\!{\bfseries\akai 住头}\!\textgreater{})}

\setlength{\hangindent}{56pt}{老令公\hspace{20pt}今当三星归位之期,我不免去至天波杨府托兆一番({\akai 或}:~托梦一回;托梦一番)便了。}

\setlength{\hangindent}{56pt}{老令公\hspace{20pt}众鬼卒,}

\setlength{\hangindent}{56pt}{(鬼卒{\hwfs 应}``呜''。)}

\setlength{\hangindent}{56pt}{老令公\hspace{20pt}驾起阴风,天波杨府去者。}

(\textless{}\!{\bfseries\akai 帽子头}\!\textgreater{}\textless{}{\!\bfseries\akai 哆啰}\!\textgreater{}{\hwfs 起带头子}【{\akai 二黄原板}】)

\setlength{\hangindent}{56pt}{老令公\hspace{20pt}【{\akai 二黄原板}】风萧萧\footnote{刘曾复先生为戏曲学院说戏录音中近于``{风飘飘}'',此处从《说戏誊稿》。}冷飕飕星稀月淡,荡悠悠飘渺渺来到人间。教鬼卒前引路风旗辗转({\akai 或}:~风旗拨转\footnote{段公平{\scriptsize 君}建议作``风起魄转''。}),}

\setlength{\hangindent}{56pt}{{(\textless{}\!{\bfseries\akai 大锣抽头}\!\textgreater{}鬼卒领下)}}

\setlength{\hangindent}{56pt}{{老令公\hspace{20pt}【{\akai 二黄原板}】此一去见六郎细说根源。}}

\setlength{\hangindent}{56pt}{{(\textless{}\!{\bfseries\akai 大锣抽头}\!\textgreater{}老令公下)}}

\vspace{3pt}{\centerline{{[}{\hei 第二场}{]}}}\vspace{5pt}

({\textless{}\!{\bfseries\akai 大锣抽头}\!\textgreater{}{\hwfs 转}\textless{}\!{\bfseries\akai 小锣四反正}\!\textgreater{}家院{\hwfs 打灯笼}{\hwfs 引}杨延昭{\hwfs 上}\footnote{据《京剧谈往录四编》\upcite{Jingju-Tanwanglu-4}载刘曾复先生著《我所见过的一些京剧配角老生演员》介绍,``\textless{}\!{\bfseries\akai 小锣四反正}\!\textgreater{}由\textless{}\!{\bfseries\akai 抽头}\!\textgreater{}\textless{}\!{\bfseries\akai 反带锣}\!\textgreater{}\textless{}\!{\bfseries\akai 硬四击}\!\textgreater{}\textless{}\!{\bfseries\akai 夺头}\!\textgreater{}四个小锣鼓点连接合成的。''{\hei 此处的表演细节为}:~``家院{\hwfs 在}\textless{}\!{\bfseries\akai 抽头}\!\textgreater{}{\hwfs 中掌灯上场},{\hwfs 走到中场右转身举灯向上场门一照},{\hwfs 这就是交代},表示让\textless{}\!{\bfseries\akai 抽头}\!\textgreater{}收住,转\textless{}\!{\bfseries\akai 反带锣}\!\textgreater{};~杨延昭{\hwfs 这才好在}\textless{}\!{\bfseries\akai 反带锣}\!\textgreater{}{\hwfs 中上场},{\hwfs 在}\textless{}\!{\bfseries\akai 硬四击}\!\textgreater{}{\hwfs 中一亮},{\hwfs 双投袖叫}\textless{}\!{\bfseries\akai 夺头}\!\textgreater{}{\hwfs 起}【{\akai 二黄原板}】{\hwfs 听更}{\akai 起唱}。''},起【{\akai 二黄原板}】{\hwfs 二更})

\setlength{\hangindent}{56pt}{杨延昭\hspace{20pt}【{\akai 二黄原板}】为国家哪何曾半日闲空,我也曾平服了塞北西东。官封到节度使啊皇王恩重,}

\setlength{\hangindent}{56pt}{(家院{\hwfs 带门下},杨延昭{\hwfs 归大座})}

\setlength{\hangindent}{56pt}{杨延昭\hspace{20pt}【{\akai 二黄原板}】身不爽不由人瞌睡朦胧啊。}

(杨延昭{\hwfs 睡介},{鬼卒{\hwfs 上站}``一条边''},老令公{\hwfs 上},{\hwfs 到小边台},{\akai 唱}【{\akai 二黄原板}】{\hwfs 三更})

\setlength{\hangindent}{56pt}{{老令公\hspace{20pt}【{\akai 二黄原板}】黑暗暗雾沉沉人烟息静,惨戚戚悲切切来到家门。静悄悄沉寂寂天波府进,}}

(众{\hwfs 挖门},\textless{}\!{\bfseries\akai 大锣抽头}\!\textgreater{},鬼卒{\hwfs 站门},老令公{\hwfs 到大边})

\setlength{\hangindent}{56pt}{{老令公\hspace{20pt}【{\akai 二黄原板}】又只见六郎儿瞌睡沉沉。我这里将他的灵魂唤醒,}}

({\hwfs 叫散},杨延昭{\hwfs 醒},\textless{}\!{\bfseries\akai 乱锤}\!\textgreater{}{\hwfs 站},{\hwfs 一望},{\hwfs 桌面上向}老令公{\hwfs 双投袖},\textless{}{\!\bfseries\akai 撕边}\!\textgreater{}\textless{}{\!\bfseries\akai 凤点头}\!\textgreater{},{\hwfs 起}【{\akai 二黄摇板}】)

\setlength{\hangindent}{56pt}{杨延昭\hspace{20pt}【{\akai 二黄摇板}】猛抬头又只见我父令公。({\textless{}\!{\bfseries\akai 仓}\!\textgreater{}})}

\setlength{\hangindent}{56pt}{杨延昭\hspace{20pt}【{\akai 二黄摇板}】曾记得在两狼父归仙境,哪有个人故后又能复逢。({\textless{}\!{\bfseries\akai 仓}\!\textgreater{}})}

\setlength{\hangindent}{56pt}{杨延昭\hspace{20pt}【{\akai 二黄摇板}】我这里({\textless{}\!{\bfseries\akai 仓}\!\textgreater{}})}

\setlength{\hangindent}{56pt}{杨延昭\hspace{20pt}【{\footnotesize 接}{\akai 二黄摇板}】下位去({\textless{}\!{\bfseries\akai 顷仓}\!\textgreater{}})}

\setlength{\hangindent}{56pt}{杨延昭\hspace{20pt}【{\footnotesize 接}{\akai 二黄摇板}】实难呐({\textless{}{\!\bfseries\akai 仓仓仓仓仓才仓}\!\textgreater{})}}

\setlength{\hangindent}{56pt}{杨延昭\hspace{20pt}【{\footnotesize 接}{\akai 二黄摇板}】转动。}

(杨延昭{\hwfs 坐},\textless{}\!{\bfseries\akai 撕边一锣}\!\textgreater{},\textless{}{\!\bfseries\akai 哆啰}\!\textgreater{},{\hwfs 起}【{\akai 回龙}】)

\setlength{\hangindent}{56pt}{{老令公\hspace{20pt}【{\akai 回龙}】我的儿休贪睡父有话云:~}}

(杨延昭{\hwfs 睡介},\textless{}\!{\bfseries\akai 夺头}\!\textgreater{},{\hwfs 起}【{\akai 二黄原板}】{\hwfs 四更})

\setlength{\hangindent}{56pt}{{老令公\hspace{20pt}【{\akai 二黄原板}】儿前番命孟良骸骨搬运,那乃是萧天佐以假为真({\akai 或}:~弄假为真)。真骸骨现在({\akai 或}:~真骸骨藏在)洪羊洞,望乡台上第三层。叮咛的言语牢牢记紧,}}

({\hwfs 叫散},鬼卒{\hwfs 领下},老令公{\hwfs 归大边外角},\textless{}\!{\bfseries\akai 扭丝}\!\textgreater{}{\hwfs 切住},{\hwfs 五更}\textless{}{\!\bfseries\akai 凤点头}\!\textgreater{})

\setlength{\hangindent}{56pt}{{老令公\hspace{20pt}【{\akai 二黄散板}】待等儿临危时}\footnote{夏行涛{\scriptsize 君}建议作``临位时'',即``归位''之时。此处从《说戏誊稿》。}父再来临。}

(\textless{}\!{\bfseries\akai 大锣打下}\!\textgreater{}老令公{\hwfs 下},{\hwfs 亮更},家院{\hwfs 上},{\hwfs 推门}、{\hwfs 进门},{\hwfs 挖到小边})

\setlength{\hangindent}{56pt}{{(家院\hspace{30pt}元帅醒来。)}}

\setlength{\hangindent}{56pt}{杨延昭\hspace{20pt}【{\akai 二黄导板}】方才老元戎呐前来托梦,}

(杨延昭{\hwfs 望介},{\textless{}\!{\bfseries\akai 嘟仓}\!\textgreater{},}{\hwfs 立},{\hwfs 出位},\textless{}\!{\bfseries\akai 快扭丝}\!\textgreater{}{\hwfs 到中场})

\setlength{\hangindent}{56pt}{杨延昭\hspace{20pt}【{\akai 二黄散板}】醒来时不由人珠泪满胸。({\textless{}\!{\bfseries\akai 住头}\!\textgreater{}})}

\setlength{\hangindent}{56pt}{(杨延昭{\hwfs 小座})

\setlength{\hangindent}{56pt}{杨延昭\hspace{20pt}有请孟二爷。}

\setlength{\hangindent}{56pt}{(家院\hspace{30pt}有请孟二爷。)}

\setlength{\hangindent}{56pt}{({家院{\hwfs 下})

\setlength{\hangindent}{56pt}{(孟良\hspace{30pt}({\akai 内})嗯喷。)}

(\textless{}\!{\bfseries\akai 小锣打上}\!\textgreater{}孟良{\hwfs 上})

\setlength{\hangindent}{56pt}{(孟良\hspace{30pt}({\akai 念})不听皇王三诏宣,单听杨家一令传。)}

\setlength{\hangindent}{56pt}{(孟良{\hwfs 进门})}

\setlength{\hangindent}{56pt}{(孟良\hspace{30pt}参见元帅。)}

\setlength{\hangindent}{56pt}{杨延昭\hspace{20pt}贤弟少礼,请坐。}

\setlength{\hangindent}{56pt}{(孟良\hspace{30pt}谢座。)}

\setlength{\hangindent}{56pt}{(\textless{}\!{\bfseries\akai 台}\!\textgreater{}孟良{\hwfs 坐大边})}

\setlength{\hangindent}{56pt}{(孟良\hspace{30pt}唤末将前来有何军事议论?)}

\setlength{\hangindent}{56pt}{杨延昭\hspace{20pt}贤弟哪里知道,昨晚三更时分,老元戎前来托梦,言道:~前番盗骨,乃是假的。}

\setlength{\hangindent}{56pt}{(孟良\hspace{30pt}真的呢?)}

\setlength{\hangindent}{56pt}{杨延昭\hspace{20pt}现在北国洪羊洞,望乡台第三层之上。愚兄意欲,命贤弟二下番营,盗取骸骨,不知贤弟意下如何?}

\setlength{\hangindent}{56pt}{(孟良\hspace{30pt}元帅说哪里话来,末将好比元帅胯下之驹,扬鞭就走,勒缰即止。就请元帅传令。)}

\setlength{\hangindent}{56pt}{杨延昭\hspace{20pt}如此贤弟听令:~}

\setlength{\hangindent}{56pt}{(杨延昭{\hwfs 站},孟良{\hwfs 站})}

\setlength{\hangindent}{56pt}{(孟良\hspace{30pt}在。)}

\setlength{\hangindent}{56pt}{(杨延昭{\hwfs 拿令旗})}

\setlength{\hangindent}{56pt}{杨延昭\hspace{20pt}({\akai 念})本帅帐中把令传,}

\setlength{\hangindent}{56pt}{(杨延昭{\hwfs 令旗交}孟良,孟良{\hwfs 接令旗})}

\setlength{\hangindent}{56pt}{(孟良\hspace{30pt}({\akai 念})此去哪怕路艰难。)}

\setlength{\hangindent}{56pt}{(杨延昭{\hwfs 边过大边边念})}

\setlength{\hangindent}{56pt}{杨延昭\hspace{20pt}({\akai 念})但愿盗得尸骸转,}

\setlength{\hangindent}{56pt}{(孟良{\hwfs 同时过小边})}

\setlength{\hangindent}{56pt}{(孟良\hspace{30pt}({\akai 念})凌烟阁上美名传。)}

\setlength{\hangindent}{56pt}{杨延昭\hspace{20pt}小心。}

\setlength{\hangindent}{56pt}{(孟良\hspace{30pt}得令。)}

(杨延昭{\hwfs 由下场门下},孟良{\hwfs 上场门下},{\textless{}\!{\bfseries\akai 小锣打下}\!\textgreater{}})

\vspace{7pt}{\centerline{({\hei 以下{[}{\hei 第三场}{]}到{[}{\hei 第七场}{]}是孟良、焦赞二人盗骨,与他们二人之死。这几场唱【{\akai 西皮}】,从略})}}\vspace{7pt}

\setlength{\hangindent}{56pt}{(程宣{\hwfs 下}\textless{}\!{\bfseries\akai 小锣打下}\!\textgreater{})}

\vspace{3pt}{\centerline{[{}{\hei 第八场}{]}}}\vspace{5pt}

\setlength{\hangindent}{56pt}{(\textless{}\!{\bfseries\akai 小锣抽头}\!\textgreater{},杨延昭{\hwfs 上},{\hwfs 站})}

\setlength{\hangindent}{56pt}{杨延昭\hspace{20pt}【{\akai 二黄摇板}】孟良盗骨无音信,倒教本帅挂在心。}

(杨延昭{\hwfs 坐小座},\textless{}\!{\bfseries\akai 小锣五击头}\!\textgreater{},家院{\hwfs 领}程宣{\hwfs 上到小边台口})

\setlength{\hangindent}{56pt}{(家院\hspace{30pt}候着。)}

\setlength{\hangindent}{56pt}{(家院{\hwfs 进门归大边})}

\setlength{\hangindent}{56pt}{(家院\hspace{30pt}启禀元帅,小番求见。)}

\setlength{\hangindent}{56pt}{杨延昭\hspace{20pt}传。}

\setlength{\hangindent}{56pt}{(家院{\hwfs 出门})}

\setlength{\hangindent}{56pt}{(家院\hspace{30pt}元帅传你,需要小心。)}

\setlength{\hangindent}{56pt}{(程宣\hspace{30pt}是。)}

\setlength{\hangindent}{56pt}{(家院、程宣{\hwfs 进门},家院{\hwfs 大边},程宣{\hwfs 边挖到小边边念})}

\setlength{\hangindent}{56pt}{(程宣\hspace{30pt}元帅在哪里,元帅在哪里。)}

\setlength{\hangindent}{56pt}{杨延昭\hspace{20pt}嗯------({\textless{}\!{\bfseries\akai 台}\!\textgreater{}})}

\setlength{\hangindent}{56pt}{杨延昭\hspace{20pt}胆大小番,头顶何物?见了本帅,大胆不跪?}

\setlength{\hangindent}{56pt}{(程宣\hspace{30pt}来人言过:~见了元帅,去掉头上匣儿,方可下跪。)}

\setlength{\hangindent}{56pt}{杨延昭\hspace{20pt}将匣儿取去。({\akai 或}:~来,将匣儿取过。)}

(家院{\hwfs 取匣},\textless{}\!{\bfseries\akai 台}\!\textgreater{},{\hwfs 大边端匣请}杨延昭{\hwfs 看})

\setlength{\hangindent}{56pt}{杨延昭\hspace{20pt}呈上来。}

\setlength{\hangindent}{56pt}{(杨延昭{\hwfs 接匣},{\hwfs 看})

\setlength{\hangindent}{56pt}{杨延昭\hspace{20pt}令公骸------(\textless{}\!{\bfseries\akai 仓}\!\textgreater{})}

\setlength{\hangindent}{56pt}{杨延昭\hspace{20pt}唉呀!}

\setlength{\hangindent}{56pt}{(\textless{}\!{\bfseries\akai 快扭丝}\!\textgreater{},杨延昭{\hwfs 台口跪})}

\setlength{\hangindent}{56pt}{杨延昭\hspace{20pt}【{\akai 二黄散板}】见骸骨哇不由人泪双流,({\textless{}\!{\bfseries\akai 仓}\!\textgreater{}})}

\setlength{\hangindent}{56pt}{杨延昭\hspace{20pt}【{\akai 二黄散板}】如今才见亲骨肉哇。({\textless{}\!{\bfseries\akai 仓}\!\textgreater{}})}

\setlength{\hangindent}{56pt}{杨延昭\hspace{20pt}【{\akai 二黄散板}】家院供奉二堂后,}

\setlength{\hangindent}{56pt}{(杨延昭{\hwfs 起身},\textless{}\!{\bfseries\akai 扭丝}\!\textgreater{},{\hwfs 匣交}家院{\hwfs 拿},{\hwfs 放堂桌上})}

\setlength{\hangindent}{56pt}{杨延昭\hspace{20pt}【{\akai 二黄散板}】再与老军说从头。}

\setlength{\hangindent}{56pt}{(杨延昭{\hwfs 小座},\textless{}\!{\bfseries\akai 住头}\!\textgreater{})}

\setlength{\hangindent}{56pt}{(程宣\hspace{30pt}叩见元帅。)}

\setlength{\hangindent}{56pt}{(程宣{\hwfs 叩介})

\setlength{\hangindent}{56pt}{杨延昭\hspace{20pt}罢了,起来。}

\setlength{\hangindent}{56pt}{(程宣\hspace{30pt}谢元帅。)}

\setlength{\hangindent}{56pt}{(程宣{\hwfs 起身归小边})}

\setlength{\hangindent}{56pt}{杨延昭\hspace{20pt}你奉何人所差?}

\setlength{\hangindent}{56pt}{(程宣\hspace{30pt}孟良孟二爷所差。)}

\setlength{\hangindent}{56pt}{杨延昭\hspace{20pt}有何为证?}

\setlength{\hangindent}{56pt}{(程宣\hspace{30pt}板斧为证。)}

\setlength{\hangindent}{56pt}{杨延昭\hspace{20pt}呈上来。}

\setlength{\hangindent}{56pt}{(家院{\hwfs 拿斧呈}杨延昭{\hwfs 看},\textless{}\!{\bfseries\akai 台}\!\textgreater{})}

\setlength{\hangindent}{56pt}{杨延昭\hspace{20pt}收过。}

\setlength{\hangindent}{56pt}{(家院{\hwfs 拿斧放堂桌上})}

\setlength{\hangindent}{56pt}{杨延昭\hspace{20pt}你叫什么名字?}

\setlength{\hangindent}{56pt}{(程宣\hspace{30pt}小人名叫程宣。)}

\setlength{\hangindent}{56pt}{杨延昭\hspace{20pt}程宣,你孟二爷他往哪里去了?}

\setlength{\hangindent}{56pt}{(程宣\hspace{30pt}哎呀,元帅呀!\textless{}\!{\bfseries\akai 台台令令台}\!\textgreater{})}

\setlength{\hangindent}{56pt}{(程宣\hspace{30pt}孟二爷前去盗骨,焦二爷暗地跟随。孟二爷一时失手,将焦二爷劈死了!~\textless{}\!{\bfseries\akai 仓}\!\textgreater{})}

\setlength{\hangindent}{56pt}{杨延昭\hspace{20pt}怎么讲?!}

\setlength{\hangindent}{56pt}{(程宣\hspace{30pt}将焦二爷劈死了!~\textless{}\!{\bfseries\akai 仓}\!\textgreater{})}

\setlength{\hangindent}{56pt}{杨延昭\hspace{20pt}(唉!)贤弟呀$\cdots{}\cdots{}$({\hwfs 哭介})}

\setlength{\hangindent}{56pt}{(\textless{}\!{\bfseries\akai 快扭丝}\!\textgreater{},杨延昭{\hwfs 站})}

\setlength{\hangindent}{56pt}{杨延昭\hspace{20pt}【{\akai 二黄散板}】听罢言来泪双淋,(\textless{}\!{\bfseries\akai 仓}\!\textgreater{})}

\setlength{\hangindent}{56pt}{杨延昭\hspace{20pt}【{\akai 二黄散板}】可叹你为杨家命丧番营。}

\setlength{\hangindent}{56pt}{杨延昭\hspace{20pt}(唉!)贤弟呀,呃$\cdots{}\cdots{}$({\hwfs 哭介})}

\setlength{\hangindent}{56pt}{(\textless{}\!{\bfseries\akai 住头}\!\textgreater{},杨延昭{\hwfs 坐})}

\setlength{\hangindent}{56pt}{杨延昭\hspace{20pt}焦二爷已死,那孟二爷也该来见我哇。}

\setlength{\hangindent}{56pt}{(程宣\hspace{30pt}哎呀,元帅呀!\textless{}\!{\bfseries\akai 台台令令台}\!\textgreater{})}

\setlength{\hangindent}{56pt}{(程宣\hspace{30pt}孟二爷将焦二爷劈死了,不愿回来,他就自刎在洪羊洞!)}

\setlength{\hangindent}{56pt}{(\textless{}\!{\bfseries\akai 快冲头}\!\textgreater{},杨延昭{\hwfs 拉}程宣{\hwfs 到台口})}

\setlength{\hangindent}{56pt}{杨延昭\hspace{20pt}怎么讲?!}

\setlength{\hangindent}{56pt}{(程宣\hspace{30pt}自刎在洪羊洞!)}

\setlength{\hangindent}{56pt}{(\textless{}\!{\bfseries\akai 快冲头}\!\textgreater{},\textless{}\!{\bfseries\akai 双叫头}\!\textgreater{})}

\setlength{\hangindent}{56pt}{杨延昭\hspace{20pt}孟良!~(\textless{}\!{\bfseries\akai 顷仓}\!\textgreater{})}

\setlength{\hangindent}{56pt}{杨延昭\hspace{20pt}焦赞!~(\textless{}\!{\bfseries\akai 仓才仓}\!\textgreater{})}

\setlength{\hangindent}{56pt}{杨延昭\hspace{20pt}唉呀!}

\setlength{\hangindent}{56pt}{(杨延昭{\hwfs 台口气椅坐},\textless{}\!{\bfseries\akai 快冲头}\!\textgreater{}家院、程宣{\hwfs 挡})}

$\bigg( \begin{aligned} &\mbox{家院}\\&\mbox{程宣}\mbox{\raisebox{5pt}{\hspace{24pt}元帅醒来。}} \end{aligned}\bigg)$

\setlength{\hangindent}{56pt}{杨延昭\hspace{20pt}【{\akai 二黄导板}】听说是二将双双丧命呐,}

\setlength{\hangindent}{56pt}{(杨延昭{\hwfs 站},\textless{}\!{\bfseries\akai 冲头}\!\textgreater{},\textless{}\!{\bfseries\akai 双叫头}\!\textgreater{})}

\setlength{\hangindent}{56pt}{杨延昭\hspace{20pt}焦赞!~(\textless{}\!{\bfseries\akai 顷仓}\!\textgreater{})}

\setlength{\hangindent}{56pt}{杨延昭\hspace{20pt}孟良!~(\textless{}\!{\bfseries\akai 仓才仓}\!\textgreater{})}

\setlength{\hangindent}{56pt}{杨延昭\hspace{20pt}唉,贤弟呀,呃$\cdots{}\cdots{}$({\hwfs 哭介})}

\setlength{\hangindent}{56pt}{(\textless{}\!{\bfseries\akai 扭丝}\!\textgreater{})}

\setlength{\hangindent}{56pt}{杨延昭\hspace{20pt}【{\akai 二黄散板}】去掉我左右膀难以飞行。教老军(\textless{}\!{\bfseries\akai 仓}\!\textgreater{})}

\setlength{\hangindent}{56pt}{杨延昭\hspace{20pt}【{\footnotesize 接}{\akai 二黄散板}】到番营尸骸搬运,}

\setlength{\hangindent}{56pt}{(\textless{}\!{\bfseries\akai 扭丝}\!\textgreater{},程宣{\hwfs 出门},{\hwfs 上场门下})}

\setlength{\hangindent}{56pt}{杨延昭\hspace{20pt}【{\akai 二黄散板}】待本帅奏圣上超度阴魂。(\textless{}\!{\bfseries\akai 仓}\!\textgreater{})}

\setlength{\hangindent}{56pt}{杨延昭\hspace{20pt}唉呀!}

(\textless{}\!{\bfseries\akai 乱锤}\!\textgreater{},杨延昭{\hwfs 胸疼介},家院{\hwfs 搀},\textless{}\!{\bfseries\akai 凤点头}\!\textgreater{})

\setlength{\hangindent}{56pt}{杨延昭\hspace{20pt}【{\akai 二黄散板}】霎时间(\textless{}\!{\bfseries\akai 仓}\!\textgreater{})}

\setlength{\hangindent}{56pt}{杨延昭\hspace{20pt}【{\footnotesize 接}{\akai 二黄散板}】心内痛啊({\akai 或}:~腹内痛啊)(\textless{}\!{\bfseries\akai 顷仓}\!\textgreater{})}

\setlength{\hangindent}{56pt}{杨延昭\hspace{20pt}【{\footnotesize 接}{\akai 二黄散板}】鲜血({\akai 或}:~心血)上(\textless{}\!{\bfseries\akai 仓仓仓仓才仓}\!\textgreater{})}

\setlength{\hangindent}{56pt}{杨延昭\hspace{20pt}【{\footnotesize 接}{\akai 二黄散板}】涌啊,}

\setlength{\hangindent}{56pt}{(家院{\hwfs 搀},\textless{}\!{\bfseries\akai 乱锤}\!\textgreater{},杨延昭{\hwfs 吐介})}

\setlength{\hangindent}{56pt}{呜$\cdots{}\cdots{}$(\textless{}\!{\bfseries\akai 顷仓}\!\textgreater{})呜$\cdots{}\cdots{}$(\textless{}\!{\bfseries\akai 顷仓}\!\textgreater{}\textless{}\!{\bfseries\akai 叭嗒仓}\!\textgreater{})呜$\cdots{}\cdots{}$(\textless{}\!{\bfseries\akai 乱锤}\!\textgreater{})}

(\textless{}\!{\bfseries\akai 扭丝}\!\textgreater{}\textless{}{\!\bfseries\akai 凤点头}\!\textgreater{},家院{\hwfs 搀}杨延昭。杨延昭{\hwfs 倒左右手}、{\hwfs 左右两转身到大边})

\setlength{\hangindent}{56pt}{杨延昭\hspace{20pt}【{\akai 二黄散板}】休得要惊动年迈的太君。}

(杨延昭{\hwfs 左转身},{\hwfs 右手扶}家院{\hwfs 手},{\textless{}\!{\bfseries\akai 大锣打下}\!\textgreater{},\textless{}\!{\bfseries\akai 撤锣}\!\textgreater{},杨延昭、家院{\hwfs 下})

\vspace{3pt}{\centerline{{[}{\hei 第九场}{]}}}\vspace{5pt}

(\textless{}\!{\bfseries\akai 小锣打上}\!\textgreater{},{\hwfs 四}太监、大太监{\hwfs 上},{\hwfs 站门},赵德芳{\hwfs 上},{\hwfs 到台口})

\setlength{\hangindent}{56pt}{(赵德芳\hspace{20pt}{[}{\akai 引子}{]}一片丹心,保叔王,锦绣龙庭。)}

\setlength{\hangindent}{56pt}{(\textless{}\!{\bfseries\akai 小锣归位}\!\textgreater{},赵德芳{\hwfs 坐小座})}

\setlength{\hangindent}{56pt}{(赵德芳\hspace{20pt}({\akai 念})紫金冠凤翅双飘,蟒龙袍玉带围腰。上金殿扬尘舞蹈,凹面锏压定群僚。)}

\setlength{\hangindent}{56pt}{(\textless{}\!{\bfseries\akai 小锣住头}\!\textgreater{})}

\setlength{\hangindent}{56pt}{(赵德芳\hspace{20pt}本御赵德芳。\textless{}\!{\bfseries\akai 台}\!\textgreater{})}

\setlength{\hangindent}{56pt}{(赵德芳\hspace{20pt}下朝回宫,内侍报道:~御妹夫身染重病。本御放心不下,亲去探望。内侍。)}

\setlength{\hangindent}{56pt}{(内侍{\hwfs 应}``有'')}

\setlength{\hangindent}{56pt}{(赵德芳\hspace{20pt}御林军走上。)}

\setlength{\hangindent}{56pt}{(内侍\hspace{30pt}御林军走上。)}

\setlength{\hangindent}{56pt}{(\textless{}\!{\bfseries\akai 冲头}\!\textgreater{},{\hwfs 四}御林军{\hwfs 两边上},{\hwfs 合龙})}

\setlength{\hangindent}{56pt}{(御林军\hspace{20pt}参见贤爷。)}

\setlength{\hangindent}{56pt}{(赵德芳\hspace{20pt}罢了。)}

\setlength{\hangindent}{56pt}{({\hwfs 四}御林军{\hwfs 分站两边},\textless{}\!{\bfseries\akai 住头}\!\textgreater{})}

\setlength{\hangindent}{56pt}{(赵德芳\hspace{20pt}外厢开道,天波杨府去者。)}

\setlength{\hangindent}{56pt}{(众{\hwfs 应}``啊'')}

\setlength{\hangindent}{56pt}{(赵德芳\hspace{20pt}带马。)}

(赵德芳{\hwfs 上马},{\hwfs 起}\textless{}\!{\bfseries\akai 大锣长锤}\!\textgreater{},赵德芳{\akai 唱}【{\akai 二黄原板}】{\hwfs 中}众``{\hwfs 扯四门}'')

\setlength{\hangindent}{56pt}{(赵德芳\hspace{20pt}【{\akai 二黄原板}】我本是金枝体大宋根本,秉忠心保叔王锦绣龙庭。内侍报御妹夫身染重病,因此上为王我御驾亲临。御林军忙摆驾前把路引,)}

({\hwfs 叫散},\textless{}\!{\bfseries\akai 大锣扭丝}\!\textgreater{},众{\hwfs 站小边}。虎形{\hwfs 下场门上},{\hwfs 到大边台口},{\hwfs 风声},{\hwfs 跳介},\textless{}{\!\bfseries\akai 凤点头}\!\textgreater{})

\setlength{\hangindent}{56pt}{(赵德芳\hspace{20pt}【{\akai 二黄散板}】只见猛虎下山林。)}

\setlength{\hangindent}{56pt}{(赵德芳\hspace{20pt}弓箭伺候。)}

\setlength{\hangindent}{56pt}{(\textless{}{\!\bfseries\akai 凤点头}\!\textgreater{})}

\setlength{\hangindent}{56pt}{(赵德芳\hspace{20pt}【{\akai 二黄散板}】手挽弓又搭箭将虎射定。)}

\setlength{\hangindent}{56pt}{(\textless{}\!{\bfseries\akai 扫头}\!\textgreater{},虎形{\hwfs 下},众{\hwfs 下},\textless{}\!{\bfseries\akai 撤锣}\!\textgreater{})}

\vspace{3pt}{\centerline{{[}{\hei 第十场}{]}}}\vspace{5pt}

\setlength{\hangindent}{56pt}{杨延昭\hspace{20pt}({\akai 内})搀扶!}

({\textless{}\!{\bfseries\akai 铙钹夺头}\!\textgreater{},{\hwfs 起}【{\akai 二黄慢板}】{\textless{}\!{\bfseries\akai 才}\!\textgreater{},杨宗保{\hwfs 在右},{\hwfs 搀}杨延昭{\hwfs 手},{\hwfs 上})

\setlength{\hangindent}{56pt}{杨延昭\hspace{20pt}【{\akai 二黄慢板}】叹杨家投宋主啊心血用尽,最可叹({\akai 或}:~真可叹)焦、孟将命丧番营。宗保儿搀为父病房来进({\akai 或}:~床榻靠枕;~软榻靠枕),}

({\hwfs 铙钹}\textless{}\!{\bfseries\akai 搓锤}\!\textgreater{},杨延昭{\hwfs 把}杨宗保{\hwfs 从小边带到大边},杨延昭{\hwfs 面向堂桌正面},{\hwfs 走},{\hwfs 到了堂桌},{\hwfs 右手扶桌},{\hwfs 一滑},{\hwfs 左转身},{\hwfs 甩髯口},{\hwfs 右手托身后},{\hwfs 向右扶桌},{\hwfs 左手在脸前从右往左画圈招手},{\hwfs 叫}杨宗保,杨宗保{\hwfs 双手扶}杨延昭{\hwfs 双手},\textless{}\!{\bfseries\akai 抽头}\!\textgreater{},{\hwfs 推磨},杨延昭{\hwfs 进大座},杨宗保{\hwfs 归小边堂桌旁边},杨延昭{\hwfs 坐})

\setlength{\hangindent}{56pt}{杨延昭\hspace{20pt}【{\akai 二黄原板}】怕只怕难捱过({\akai 或}:~熬不过)尺寸光阴。}

(\textless{}\!{\bfseries\akai 小锣抽头}\!\textgreater{},赵德芳众{\hwfs 上},御林军``{\hwfs 一条边}'')

\setlength{\hangindent}{56pt}{(赵德芳\hspace{20pt}【{\akai 二黄散板}】来至在府门外王下金镫,)}

\setlength{\hangindent}{56pt}{(赵德芳{\hwfs 下马},众{\hwfs 下}。杨宗保{\hwfs 出门},{\hwfs 作揖})}

\setlength{\hangindent}{56pt}{(杨宗保\hspace{20pt}迎接千岁。)}

\setlength{\hangindent}{56pt}{(赵德芳\hspace{20pt}【{\akai 二黄散板}】宗保儿免礼你且平身。你父帅身染病何处安顿?)}

\setlength{\hangindent}{56pt}{(杨宗保\hspace{20pt}现在病房。)}

\setlength{\hangindent}{56pt}{(赵德芳\hspace{20pt}带路。)}

\setlength{\hangindent}{56pt}{(\textless{}\!{\bfseries\akai 小锣抽头}\!\textgreater{},赵德芳{\hwfs 挖到大边},杨宗保{\hwfs 归小边})}

\setlength{\hangindent}{56pt}{(赵德芳\hspace{20pt}【{\akai 二黄散板}】又只见御妹夫瞌睡沉沉。)}

\setlength{\hangindent}{56pt}{(赵德芳\hspace{20pt}醒来。)}

\setlength{\hangindent}{56pt}{杨延昭\hspace{20pt}【{\akai 二黄导板}\footnote{《说戏誊稿》记作【{\akai 大锣导板}】【{\akai 散板}】,因刘曾复先生在文稿中注明这是``三个导板之一'',此处记作【{\akai 导板}】。}】方才郊外闲游散闷,(\textless{}\!{\bfseries\akai 仓}\!\textgreater{})}

\setlength{\hangindent}{56pt}{杨延昭\hspace{20pt}【{\akai 二黄散板}】见一官长放雕翎呐。(\textless{}\!{\bfseries\akai 仓}\!\textgreater{})}

\setlength{\hangindent}{56pt}{杨延昭\hspace{20pt}【{\akai 二黄散板}】对我胸前射一箭,(\textless{}\!{\bfseries\akai 仓}\!\textgreater{})}

\setlength{\hangindent}{56pt}{杨延昭\hspace{20pt}【{\akai 二黄散板}】险些儿丧了命残生{\footnotesize 呐}。(\textless{}\!{\bfseries\akai 仓}\!\textgreater{})}

\setlength{\hangindent}{56pt}{杨延昭\hspace{20pt}【{\akai 二黄散板}】猛然睁开昏花眼呐,(\textless{}\!{\bfseries\akai 仓}\!\textgreater{})}

\setlength{\hangindent}{56pt}{(杨延昭{\hwfs 一望}赵德芳)

\setlength{\hangindent}{56pt}{杨延昭\hspace{20pt}哎呀!}

\setlength{\hangindent}{56pt}{(杨延昭{\hwfs 站到桌左内侧},{\hwfs 扶杨宗保},\textless{}{\!\bfseries\akai 凤点头}\!\textgreater{})}

\setlength{\hangindent}{56pt}{杨延昭\hspace{20pt}【{\akai 二黄散板}】面前站定放箭之人。(\textless{}\!{\bfseries\akai 仓}\!\textgreater{})}

\setlength{\hangindent}{56pt}{杨延昭\hspace{20pt}【{\akai 二黄散板}】我和你一无冤仇,二无怨恨,你,你,你却缘何放雕翎呐射我前心?}

\setlength{\hangindent}{56pt}{(杨延昭{\hwfs 又坐睡},\textless{}{\!\bfseries\akai 凤点头}\!\textgreater{})}

\setlength{\hangindent}{56pt}{(赵德芳\hspace{20pt}【{\akai 二黄散板}】听罢言来才知情,\textless{}\!{\bfseries\akai 仓}\!\textgreater{})}

\setlength{\hangindent}{56pt}{(赵德芳\hspace{20pt}【{\akai 二黄散板}】白虎是他本命星。\textless{}\!{\bfseries\akai 仓}\!\textgreater{})}

\setlength{\hangindent}{56pt}{(赵德芳\hspace{20pt}【{\akai 二黄散板}】走向前来把话论:~\textless{}\!{\bfseries\akai 仓}\!\textgreater{})}

\setlength{\hangindent}{56pt}{(赵德芳\hspace{20pt}【{\akai 二黄散板}】休把我当作了放箭之人。)}

\setlength{\hangindent}{56pt}{(杨延昭{\hwfs 睡介})

\setlength{\hangindent}{56pt}{(杨宗保\hspace{20pt}贤爷驾到。)}

\setlength{\hangindent}{56pt}{杨延昭\hspace{20pt}哦!}

\setlength{\hangindent}{56pt}{(赵德芳{\hwfs 坐堂桌大边侧},\textless{}{\!\bfseries\akai 凤点头}\!\textgreater{})}

\setlength{\hangindent}{56pt}{杨延昭\hspace{20pt}【{\akai 二黄散板}】听说贤爷驾到临,(\textless{}\!{\bfseries\akai 仓}\!\textgreater{})}

\setlength{\hangindent}{56pt}{杨延昭\hspace{20pt}【{\akai 二黄散板}】宗保儿替为父赔罪负荆。}

\setlength{\hangindent}{56pt}{(杨延昭{\hwfs 醒},\textless{}\!{\bfseries\akai 住头}\!\textgreater{})}

\setlength{\hangindent}{56pt}{(杨宗保\hspace{20pt}千岁恕罪。)}

\setlength{\hangindent}{56pt}{(赵德芳\hspace{20pt}平身,赐座。)}

\setlength{\hangindent}{56pt}{(杨宗保\hspace{20pt}谢千岁。)}

\setlength{\hangindent}{56pt}{(杨宗保{\hwfs 起身},{\hwfs 坐大边},\textless{}\!{\bfseries\akai 台}\!\textgreater{})}

\setlength{\hangindent}{56pt}{(赵德芳\hspace{20pt}御妹夫此病从何而起?)}

\setlength{\hangindent}{56pt}{杨延昭\hspace{20pt}\textless{}\!{\bfseries\akai 小锣叫头}\!\textgreater{}唉!~(\textless{}\!{\bfseries\akai 台}\!\textgreater{})}

\setlength{\hangindent}{56pt}{杨延昭\hspace{20pt}贤爷呀,呃$\cdots{}\cdots{}$({\hwfs 哭介})}

\setlength{\hangindent}{56pt}{杨延昭\hspace{20pt}【{\akai 二黄快三眼}】自那日朝罢归身罹疾病({\akai 或}:~身染重病),三更时梦呃见了年迈爹尊{\footnotesize 呐}。臣前番({\akai 或}:~我前番)命孟良骸骨搬请,那乃是萧天佐以假为真({\akai 或}:~弄假为真)。真骸骨伊藏在\footnote{《说戏誊稿》作``已藏在'',据樊百乐{\scriptsize 君}告知,刘曾复先生作``伊藏在''。作``匿藏在''似亦可。}({\akai 或}:~藏至在)洪羊洞,望乡台第三层那才是真。二次里命孟良番营来进,又谁知焦克明他私自后跟呐。老军报他二人在洪羊洞丧命,去掉我左右膀难以飞行。为此事终日里忧愁急窘({\akai 或}:~忧成疾病),因此上臣的病重加十分。千岁爷呀!}

\setlength{\hangindent}{56pt}{(赵德芳\hspace{20pt}【{\footnotesize 接}{\akai 二黄原板}】御妹夫休得要心中烦闷,焦、孟将他二人难以复生。宗保儿近前来听王命:~)}

\setlength{\hangindent}{56pt}{(杨宗保{\hwfs 站})}

\setlength{\hangindent}{56pt}{(赵德芳\hspace{20pt}【{\akai 二黄原板}】后堂内快请出儿祖母、娘亲。)}

\setlength{\hangindent}{56pt}{(杨宗保{\hwfs 向外})

\setlength{\hangindent}{56pt}{(杨宗保\hspace{20pt}有请祖母、娘亲。)}

(赵德芳{\hwfs 下场门下},\textless{}撞金钟\textgreater{},佘太君、柴夫人{\hwfs 上场门上})

\setlength{\hangindent}{56pt}{(佘太君\hspace{20pt}【{\akai 二黄摇板}】忽听宗保一声请,)}

\setlength{\hangindent}{56pt}{(柴夫人\hspace{20pt}【{\akai 二黄摇板}】急忙前来问分明。)}

(佘太君、柴夫人{\hwfs 挖进去},{\hwfs 站小边},杨宗保{\hwfs 站大边})

\setlength{\hangindent}{56pt}{(佘太君\hspace{20pt}醒来。)}

\setlength{\hangindent}{56pt}{杨延昭\hspace{20pt}【{\akai 二黄导板}\footnote{《说戏誊稿》记作{【{\akai 导板}】【{\akai 散板}】,刘曾复先生在文稿中注明这是``三个导板之一'',此处记作【{\akai 导板}】。}}】我方才朦胧荏苒\footnote{``荏苒''是逡巡、一刹那的意思。}动啊,(\textless{}\!{\bfseries\akai 仓}\!\textgreater{})}

\setlength{\hangindent}{56pt}{杨延昭\hspace{20pt}【{\akai 二黄散板}】耳旁又听有人声。(\textless{}\!{\bfseries\akai 仓}\!\textgreater{})}

\setlength{\hangindent}{56pt}{杨延昭\hspace{20pt}【{\akai 二黄散板}】睁开了昏花眼难以扎挣,(\textless{}\!{\bfseries\akai 仓}\!\textgreater{})}

\setlength{\hangindent}{56pt}{杨延昭\hspace{20pt}哎呀!}

\setlength{\hangindent}{56pt}{(\textless{}{\!\bfseries\akai 凤点头}\!\textgreater{})}

\setlength{\hangindent}{56pt}{杨延昭\hspace{20pt}【{\akai 二黄散板}】抬头只见儿的老娘亲{\footnotesize 呐}。(\textless{}\!{\bfseries\akai 仓}\!\textgreater{})}

\setlength{\hangindent}{56pt}{杨延昭\hspace{20pt}【{\akai 二黄散板}】生下了孩儿人七个,到如今白发人反送了黑发人{\footnotesize 呐},儿的娘啊!~(\textless{}\!{\bfseries\akai 顷仓}\!\textgreater{})}

\setlength{\hangindent}{56pt}{杨延昭\hspace{20pt}【{\footnotesize 接}{\akai 二黄散板}】好不伤情。}

\setlength{\hangindent}{56pt}{(众{\hwfs 哭},\textless{}{\!\bfseries\akai 凤点头}\!\textgreater{})}

\setlength{\hangindent}{56pt}{杨延昭\hspace{20pt}【{\akai 二黄散板}】舍不得宗保儿无人教训,(\textless{}\!{\bfseries\akai 仓}\!\textgreater{})}

\setlength{\hangindent}{56pt}{杨延昭\hspace{20pt}【{\akai 二黄散板}】实难舍柴夫人结发之情{\footnotesize 呐}。(\textless{}\!{\bfseries\akai 仓}\!\textgreater{})}

\setlength{\hangindent}{56pt}{杨延昭\hspace{20pt}【{\akai 二黄散板}】宗保儿{\footnotesize 喏}、(\textless{}\!{\bfseries\akai 仓}\!\textgreater{})}

\setlength{\hangindent}{56pt}{杨延昭\hspace{20pt}【{\footnotesize 接}{\akai 二黄散板}】柴夫人{\footnotesize 呐}(\textless{}\!{\bfseries\akai 顷仓}\!\textgreater{})}

\setlength{\hangindent}{56pt}{杨延昭\hspace{20pt}【{\footnotesize 接}{\akai 二黄散板}】将我(\textless{}\!{\bfseries\akai 仓仓仓仓仓才仓}\!\textgreater{})}

\setlength{\hangindent}{56pt}{杨延昭\hspace{20pt}【{\footnotesize 接}{\akai 二黄散板}】搀定,}

(检场{\hwfs 撤桌}。宗保、柴夫人{\hwfs 搀}杨延昭{\hwfs 站},{\hwfs 走到台口},佘太君{\hwfs 站小边},杨延昭{\hwfs 等跪})

\setlength{\hangindent}{56pt}{杨延昭\hspace{20pt}【{\akai 二黄散板}】一家人跪埃尘叩谢圣恩呐。(\textless{}\!{\bfseries\akai 仓}\!\textgreater{})}

\setlength{\hangindent}{56pt}{杨延昭\hspace{20pt}【{\akai 二黄散板}】恕为臣\footnote{``为臣''作``微臣''似亦可。}再不能社稷重整,恕为臣再不能扶保乾坤。(\textless{}\!{\bfseries\akai 仓}\!\textgreater{})}

\setlength{\hangindent}{56pt}{杨延昭\hspace{20pt}【{\akai 二黄散板}】霎时间(\textless{}\!{\bfseries\akai 仓}\!\textgreater{})}

\setlength{\hangindent}{56pt}{杨延昭\hspace{20pt}【{\footnotesize 接}{\akai 二黄散板}】心内痛{\footnotesize 啊}({\akai 或}:~腹内痛啊)(\textless{}\!{\bfseries\akai 顷仓}\!\textgreater{})}

\setlength{\hangindent}{56pt}{杨延昭\hspace{20pt}【{\footnotesize 接}{\akai 二黄散板}】鲜血({\akai 或}:~心血)上(\textless{}\!{\bfseries\akai 仓仓仓仓仓才仓}\!\textgreater{})}

(众{\hwfs 起身},{\hwfs 四}鬼魂{\hwfs 两边上})

\setlength{\hangindent}{56pt}{杨延昭\hspace{20pt}【{\footnotesize 接}{\akai 二黄散板}】涌{\footnotesize 啊},}

\setlength{\hangindent}{56pt}{(杨延昭{\hwfs 吐介})}

\setlength{\hangindent}{56pt}{呜$\cdots{}\cdots{}$(\textless{}\!{\bfseries\akai 乱锤}\!\textgreater{})}

\setlength{\hangindent}{56pt}{{(\textless{}{\!\bfseries\akai 凤点头}\!\textgreater{})}}

\setlength{\hangindent}{56pt}{杨延昭\hspace{20pt}【{\akai 二黄散板}】我面前站定了许多鬼魂:~}

(焦赞{\hwfs 外}、岳胜{\hwfs 里小边},孟良{\hwfs 外}、老令公{\hwfs 里大边})

\setlength{\hangindent}{56pt}{杨延昭\hspace{20pt}【{\akai 二黄散板}】焦克明气昂昂他的心、心怀不忿,那(、那)孟、孟佩苍他那里拱手相迎。}

(焦赞、岳胜{\hwfs 换位},{\textless{}{\!\bfseries\akai 凤点头}\!\textgreater{}})

\setlength{\hangindent}{56pt}{杨延昭\hspace{20pt}【{\akai 二黄散板}】这一旁站定了勇将岳胜,(\textless{}\!{\bfseries\akai 仓}\!\textgreater{})}

\setlength{\hangindent}{56pt}{(孟良、老令公{\hwfs 换位})

\setlength{\hangindent}{56pt}{杨延昭\hspace{20pt}唉呀!}

(\textless{}\!{\bfseries\akai 乱锤}\!\textgreater{},杨延昭{\hwfs 跪}老令公{\hwfs 前},\textless{}{\!\bfseries\akai 凤点头}\!\textgreater{})

\setlength{\hangindent}{56pt}{杨延昭\hspace{20pt}【{\akai 二黄散板}】抬头只见老严亲呐。(\textless{}\!{\bfseries\akai 仓}\!\textgreater{})}

\setlength{\hangindent}{56pt}{杨延昭\hspace{20pt}【{\akai 二黄散板}】哭一声(\textless{}\!{\bfseries\akai 仓}\!\textgreater{})}

\setlength{\hangindent}{56pt}{杨延昭\hspace{20pt}【{\footnotesize 接}{\akai 二黄散板}】老爹尊({\akai 或}:~老爹爹)(\textless{}\!{\bfseries\akai 顷仓}\!\textgreater{})}

\setlength{\hangindent}{56pt}{杨延昭\hspace{20pt}【{\footnotesize 接}{\akai 二黄散板}】黄泉路{\footnotesize 呃}(\textless{}\!{\bfseries\akai 仓仓仓仓仓才仓}\!\textgreater{})}

\setlength{\hangindent}{56pt}{杨延昭\hspace{20pt}【{\footnotesize 接}{\akai 二黄散板}】等啊,}

(\textless{}\!{\bfseries\akai 乱锤}\!\textgreater{}杨延昭{\hwfs 洒},杨延昭众{\hwfs 站台口正面},杨延昭{\hwfs 吐介})

\setlength{\hangindent}{56pt}{呜$\cdots{}\cdots{}$(\textless{}\!{\bfseries\akai 顷仓}\!\textgreater{})呜$\cdots{}\cdots{}$(\textless{}\!{\bfseries\akai 顷仓}\!\textgreater{}\textless{}\!{\bfseries\akai 叭嗒仓}\!\textgreater{})呜$\cdots{}\cdots{}$}}

\setlength{\hangindent}{56pt}{(\textless{}\!{\bfseries\akai 凤点头}\!\textgreater{})}

\setlength{\hangindent}{56pt}{杨延昭\hspace{20pt}【{\akai 二黄散板}】无常到万事休去见先人。}

(\textless{}\!{\bfseries\akai 嘟仓}\!\textgreater{},杨延昭{\hwfs 死介},{\hwfs 倒坐在台口椅上},{\bfseries\akai 唢呐}{[}{\bfseries\akai 吹打}{]}\textless{}\!{\bfseries\akai 牌子}\!\textgreater{},{\hwfs 四}鬼魂{\hwfs 合拢挡介},杨延昭{\hwfs 解头上绸条和腰包},{\hwfs 腰包搭椅背上}、{\hwfs 绸条折成两条搭在腰包上},{\hwfs 四}鬼魂{\hwfs 领}杨延昭{\hwfs 下},孟良、老令公、杨延昭、岳胜、焦赞``{\hwfs 搭轿}''{\hwfs 下})

\setlength{\hangindent}{56pt}{({\bfseries\akai 唢呐}{\hwfs 停},赵德芳{\hwfs 上},{\hwfs 站大边},众{\hwfs 哭},\textless{}\!{\bfseries\akai 凤点头}\!\textgreater{})}

\setlength{\hangindent}{56pt}{(佘太君\hspace{20pt}【{\akai 二黄散板}】见此情不由人心中酸痛,)}

\setlength{\hangindent}{56pt}{(柴夫人\hspace{20pt}【{\akai 二黄散板}】撇下了母子们好不伤情。)}

\setlength{\hangindent}{56pt}{({\textless{}{\!\bfseries\akai 凤点头}\!\textgreater{}})}

\setlength{\hangindent}{56pt}{(赵德芳\hspace{20pt}【{\akai 二黄散板}】劝太君和御妹须要珍重,宗保儿随为王金殿面君。)}

(赵德芳{\hwfs 出门},杨宗保{\hwfs 出门},赵德芳{\hwfs 下场门下},杨宗保{\hwfs 跟下},{\hwfs 同时}佘太君{\hwfs 上场门下},柴夫人{\hwfs 托腰包和绸条随下})

(柴夫人{\hwfs 取腰包}、{\hwfs 绸条时检场撤台口椅},{\hwfs 在}\textless{}\!{\bfseries\akai 尾声}\!\textgreater{}{\hwfs 中下场})

}

\vspace{15pt}

\setlength{\hangindent}{56pt}{\hei 附:~关于《洪羊洞》阴魂人物扮相:~}

\setlength{\hangindent}{56pt}{四鬼卒:~龙套扮演,本脸,小鬼发、青袍、卒坎、黑风旗。}

\setlength{\hangindent}{56pt}{老令公:~金大镫、白满、白蟒、玉带;}

\setlength{\hangindent}{56pt}{岳胜:~忠纱、黑三、绿蟒、玉带;}

\setlength{\hangindent}{56pt}{孟良:~硬扎巾、黪红紥、红蟒、玉带;}

\setlength{\hangindent}{56pt}{焦赞:~硬扎巾、黪紥、黑蟒、玉带。}

\setlength{\hangindent}{56pt}{老令公、岳胜、孟良、焦赞{四人都拿云帚};~岳胜、孟良、焦赞~戴黑纱,老令公~不戴黑纱。}

\setlength{\hangindent}{56pt}{杨延昭:~ 死后解去绸条、腰包,剩下员外巾、古铜褶子,不戴黑纱,空手,左手拉老令公云帚尾下。}

