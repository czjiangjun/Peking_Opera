\newpage\hspace{30pt}~
\subsubsection{\large\hei {珠帘寨~{\small 之}~李克用}}
\addcontentsline{toc}{subsection}{\hei 珠帘寨~{\small 之}~李克用}

\hangafter=1                   %2. 设置从第1⾏之后开始悬挂缩进  %
\setlength{\parindent}{0pt}{
\vspace{3pt}{\centerline{{[}{\hei 第一场}{]}}}\vspace{5pt}

\textless{}\!{\bfseries\akai 点绛唇}\!\textgreater{}荆棘铜驼\footnote{``铜驼''即铜铸的骆驼,古代置于宫门外。借指京城、宫廷。同时也比喻朝代兴亡。},唐室残破。离朝阁,自立山河,沙陀全归我。

({\akai 念})太白斗酒诗百篇,长安市上酒家眠。摔死国舅段文楚,唐王一怒贬北番。

孤,李克用呃,祖居沙陀,先父朱(姓,讳)国昌\footnote{据史料载,李克用之父李国昌,本名朱邪赤心,唐末沙陀部落首领,唐懿宗时因镇压庞勋起义之功,被赐名``李国昌''。``朱邪''姓亦作``朱耶'',艺人不识,误作朱姓;``国''字系入声字,此处保留湖北方言念法。},归顺唐室。讨贼有功,因赐国姓({\akai 或}:~只因屡建奇功,唐王见喜,恩赐国姓)。唐王见孤左眼小比龙,右眼大比虎,生就龙虎之姿,认孤为螟蛉义子殿下({\akai 或}:~认孤为义儿殿下),赐名``鸦儿''\footnote{据史料载,李克用别名``李鸦儿'',一目失明,其主力部队因穿黑衣服而以``鸦儿军''闻名。}啊。

只因那年,孤王得胜还朝,唐王见喜,在五凤楼前恩赐御宴,文武百官庆贺千秋,({\akai 或}:~只因那年,孤王得胜还朝,唐王在五凤楼前恩赐御宴,以贺千秋,)内有国舅段文楚,笑孤坐席不正,礼貌不周。怒恼孤家,隔席抓过,(我)就摔呃------摔在丹墀,那贼就口吐鲜血而亡了。唐王大怒,将孤推出午门斩首,多亏恩官程敬思连保数本,唐王赦了死罪,将孤削职,贬回故土。({\akai 或}:~唐王死罪已免,活罪难饶,将孤谪贬沙陀为民。)

来到沙陀({\akai 或}:~是孤来在沙陀),众家王子,顶盔贯甲,拦住孤的马头,俱要与孤(王)比试。那时孤哪有什么闲情逸致与他们玩耍,是孤稳坐雕鞍,心生一计,将孤的九九八十一斤定唐宝刀------哗喇喇------耍上数路,众家王子一个个拜服马前,尊孤为首。一路之上,收了({\akai 或}:~收下)二位皇娘,一十一家太保。来到沙陀({\akai 或}:~来在沙陀),风调雨顺,国泰民安。朝朝饮宴,夜夜笙歌。好不洒乐人也!正是:~({\akai 念})红尘一点不到处,

太保,回来了?

打来({\akai 或}:~打了)多少飞禽走兽?

(哦,)什么新闻?

嚯------胆大黄巢,欺我唐室无人。

太保(听令,)传令(下去):~二位皇娘挂帅,众家太保以为前站先行,带领({\akai 或}:~发动)沙陀国四十五万番汉兵将,前去兴唐灭巢!

慢,慢$\cdots{}\cdots{}$慢着!唐王无道,将孤谪贬({\akai 或}:~当年唐王将孤谪贬),哪有人马与他解围。

太保,原令追回!

太保,你是怎样知晓?

哦,程恩官来了。

他乃孤王({\akai 或}:~孤家)活命恩人,(必须迎接于他。)太保------

吩咐摆队相迎。

\vspace{3pt}{\centerline{{[}{\hei 第二场}{]}}}\vspace{5pt}

久违了。({\akai 或}:~啊,恩官。)

你也皓然\footnote{夏行涛君建议作``皓髯''。}了哇。

一样。\hspace{20pt}~

啊------呵呵呵哈哈哈$\cdots{}\cdots{}$({\hwfs 笑}{\hwfs 介})

恩官到此,乃是客位。

恩官请。\hspace{10pt}~

(如此)你我挽手而行。

\vspace{3pt}{\centerline{{[}{\hei 第三场}{]}}}\vspace{5pt}

且慢,你乃孤活命恩人,受孤一拜。

太保,见过儿的程叔父({\akai 或}:~拜见程叔父)。

唐王驾安?\hspace{10pt}~

满朝文武可好?

有劳他们。

请坐。\hspace{20pt}~

不知恩官驾到,未曾远迎,当面恕罪。

岂敢。\hspace{20pt}~

看酒来,待孤把盏。

太保代敬。

恩官请。\hspace{10pt}~

干!\hspace{40pt}~

正是:~({\akai 念})忆昔五凤楼,相隔有数秋({\akai 或}:~相隔数十秋)。

好哇------好一个``叙叙旧根由''。

\setlength{\hangindent}{56pt}{【{\akai 西皮导板}】太保传令把队收, }

干!\hspace{40pt}~

\setlength{\hangindent}{56pt}{【{\akai 西皮原板}】孤与贤弟叙一叙旧根由。 }

\setlength{\hangindent}{56pt}{【{\akai 西皮原板}】忆昔当年五凤楼,文武百官庆贺千秋。内有个文楚段国舅,他笑孤王坐席不正、礼貌不周。怒恼了孤王气冲牛斗,隔席抓过摔死龙楼。摔死了国舅段文楚,唐主爷一怒要斩头。自从那年离朝后,今日里相逢在北州。 }

\setlength{\hangindent}{56pt}{(程敬思\footnote{据《新五代史·唐本纪第四》载``黄巢已陷京师,中和元年,代北起军使陈景思发沙陀先所降者,与吐浑、安庆等万人赴京师,行至绛州,沙陀军乱,大掠而还。景思念沙陀非克用不可将,乃以诏书召克用于鞑靼,承制以为代州刺史、雁门以北行营节度使。率蕃汉万人出石岭关$\cdots{}\cdots{}$二年十一月,景思、克用复以步骑万七千赴京师。''戏中程敬思的原型即为陈景思。}

【{\akai 西皮原板}】自从千岁离朝后,满朝中文武泪双流。为千岁懒把朝房走,为千岁懒观五凤楼。山遥路远少来问候,望千岁恕学生礼貌不周。) }

\setlength{\hangindent}{56pt}{【{\akai 西皮导板}】太保推杯换大斗, }

\setlength{\hangindent}{56pt}{【{\akai 西皮快板}】李克用跪席前脸带惭羞。当初不该打死国舅,怒恼了唐王要斩人头。如不是恩官把本奏,孤王焉有活命留。天高地厚恩少有,这一斗水酒你要饮下喉。 }

\setlength{\hangindent}{56pt}{(程敬思\hspace{20pt}【{\akai 西皮快板}】用手儿接过梨花盏,学生大胆把话言:~甲子年,开科选,山东来了一生员。家住曹州并曹县,姓黄名巢字巨天\footnote{段公平{\scriptsize 君}注:~黄巢的字,于史无载。《残唐五代史演义》作``巨天'',此处从之。  吴小如先生早年曾撰文\upcite{Wu_Wenlu-I}指出,此两句原作``家住曹州定陶县,姓黄名巢字霸天''。``并曹县''是``定陶县''的讹传;旧时艺人文化程度低,将``霸天''记作``垻天''字,以致讹成``具天''。\\姜骏按:~据《新唐书》载,黄巢是``山东曹州冤句人'',据史料推测,冤句在曹县、定陶一带(具体方位有争议)。因此``曹州并曹县''中``并''理解为衬字亦可。}。三篇文章作得好,试官点他为状元。跨马三日游宫苑,宫娥、彩嫔笑连天。唐王嫌他容貌丑,斩了试官革状元。斩了试官不要紧,革了状元起祸端。祥梅寺\footnote{《残唐五代史演义》作``藏梅寺''。},造了反,将我主驾逼在西祁\footnote{《京剧汇编》第四十六集~郝寿臣~藏本作``西岐'',此处从《残唐五代史演义》,下同。}美良川。学生到此无别干,一来搬兵二问安。) }

\setlength{\hangindent}{56pt}{【{\akai 西皮快板}】听说黄巢造了反,不由得孤王笑连天\footnote{``笑连天''亦可作``笑颜添''。}。贤弟饮宴且饮宴,提起了唐王孤不耐烦。 }

\setlength{\hangindent}{56pt}{(程敬思\hspace{20pt}【{\akai 西皮快板}】我这里提起唐天子,这老儿一旁不耐烦。是是是,明白了,老儿是个爱宝男。叫人来将宝搭上殿,特请千岁把宝观。) }

\setlength{\hangindent}{56pt}{【{\akai 西皮快板}】一见珠宝帐前摆,不由得孤王笑颜开。上有蟒袍和玉带,凤冠头上插金钗。明明知道佯不解,假意儿上前问开怀。你做清官数十载,此宝打从何处来。 }

\setlength{\hangindent}{56pt}{(程敬思\hspace{20pt}【{\akai 西皮快板}】此宝出在山海外,三年五载进宝来。唐王爱将恩似海,特命学生进宝来。\nolinebreak) }

\setlength{\hangindent}{56pt}{【{\akai 西皮快板}】贤弟进宝因何故, }

\setlength{\hangindent}{56pt}{(程敬思\hspace{20pt}【{\akai 西皮快板}】特请千岁把兵排。) }

\setlength{\hangindent}{56pt}{【{\akai 西皮快板}】年纪迈,血气衰,难作国家的栋梁才。 }

\setlength{\hangindent}{56pt}{(程敬思\hspace{20pt}【{\akai 西皮快板}】千岁爷虎老雄心在,黄巢闻名他不敢来。) }

\setlength{\hangindent}{56pt}{【{\akai 西皮快板}】贤弟休得把孤抬,有一辈古人说上来:~昔日有个姜吕望,稳坐钓鱼台他不下来。 }

\setlength{\hangindent}{56pt}{(程敬思\hspace{20pt}【{\akai 西皮快板}】钓鱼台,不下来,他保周朝八百载。千岁不发人和马,黄巢笑你老无才。\nolinebreak) }

\setlength{\hangindent}{56pt}{【{\akai 西皮快板}】笑只笑唐天子,他笑孤王为何来。中军帐,挂了帅,众家太保两边排。一马儿踏入唐室界,万里的乾坤扭转来。 }

\setlength{\hangindent}{56pt}{(程敬思\hspace{20pt}【{\akai 西皮快板}】说此话就该发人马,) }

\setlength{\hangindent}{56pt}{【{\akai 西皮摇板}】唐王晏驾你再来。 }

\setlength{\hangindent}{56pt}{(程敬思\hspace{20pt}【{\akai 西皮摇板}】问千岁此宝爱不爱?) }

\setlength{\hangindent}{56pt}{【{\akai 西皮摇板}】孤念你千里迢迢路远来,却之不恭呃,受之有愧,来来来,一体全收哇往后抬。 }

\setlength{\hangindent}{56pt}{(程敬思\hspace{20pt}【{\akai 西皮快板}】这老儿做事不公平,收了宝物不发兵。用手取出唐王旨,我奉圣旨来调兵。\nolinebreak) }

(程敬思\hspace{20pt}圣旨下。)

呃!\hspace{40pt}~

\setlength{\hangindent}{56pt}{【{\akai 西皮快板}】程敬思做事太无情,不该圣旨欺寡人。用手拿过({\akai 或}:~接过)皇王旨({\akai 或}:~唐王旨),回手压下帝王(的)文。哪一个再提发兵事,定斩沙陀不徇情。 }

\setlength{\hangindent}{56pt}{(程敬思\hspace{20pt}【{\akai 西皮快板}】一见千岁变了脸,回头埋怨李嗣源。我在松林寻短见,不该救我活命还。) }

\setlength{\hangindent}{56pt}{【{\akai 西皮快板}】奴才做事真胆大,胡言乱语少家法({\akai 或}:~把话答)。({\akai 或}:~我与恩官来讲话,大胆奴才把话答。)吩咐两旁武士手({\akai 或}:~刀斧手),推出帐去({\akai 或}:~推出午门)把头杀。 }

\setlength{\hangindent}{56pt}{(程敬思\hspace{20pt}【{\akai 西皮摇板}】千岁要斩把学生斩,快快赦回太保还。) }

\setlength{\hangindent}{56pt}{【{\akai 西皮快板}】我与恩公({\akai 或}:~恩官)来讲话,奴才一旁({\akai 或}:~奴才竟敢)把话答。恩公若回长安转,耻笑孤王无家法({\akai 或}:~少家法)。 }

\setlength{\hangindent}{56pt}{(程敬思\hspace{20pt}【{\akai 西皮摇板}】有家法来无家法,看学生薄面绕过他。) }

\setlength{\hangindent}{56pt}{【{\akai 西皮摇板}】贤弟({\akai 或}:~恩官)不必礼恭敬,帐外赦回太保身({\akai 或}:~午门赦回小畜生)。 }

\setlength{\hangindent}{56pt}{【{\akai 西皮摇板}】一足将儿踏帐下({\akai 或}:~恨不得一足将儿踏), }

\setlength{\hangindent}{56pt}{【{\akai 西皮摇板}】程恩官讲情儿要谢过他。 }

\setlength{\hangindent}{56pt}{【{\akai 西皮导板}】昔日有个三大贤, }

\setlength{\hangindent}{56pt}{【{\akai 西皮原板}】刘、关、张结义在桃园。弟兄们徐州曾失散,古城相逢又团圆。关二爷马上呼三弟,张翼德在城楼怒发冲冠。你既然降了奸曹操,看来是无义反桃园({\akai 或}:~负义反桃园)。耳边厢又听【{\footnotesize 转}{\akai 西皮快板}】人呐喊,老蔡阳的人马来到了古城边。城楼上助你三通鼓,日月旌旗壮壮威严。哗喇喇打罢了头通鼓,关二爷提刀跨雕鞍。哗喇喇喇打罢了二通鼓,人有精神马又欢。哗喇喇打罢了三通鼓,蔡阳的人头落在马前。一来是老儿该丧命,二来弟兄得团圆。贤弟休回长安转,就在这沙陀过几年,落得个清闲。}

\vspace{3pt}{\centerline{{[}{\hei 第四场}{]}}}\vspace{5pt}

\setlength{\hangindent}{56pt}{(程敬思\hspace{20pt}【{\akai 西皮快板}】过了一天又一天,心中好似滚油煎。眼望长安难回转,不知唐王驾可安。) }

\setlength{\hangindent}{56pt}{【{\akai 西皮摇板}】贤弟不必({\akai 或}:~休得)想唐朝,长安哪有({\akai 或}:~焉有)此地高。沙陀国有你的乌纱帽,沙陀国有你紫罗袍。 }

\setlength{\hangindent}{56pt}{【{\akai 西皮导板}】贤弟随孤哇来观瞧, }

\setlength{\hangindent}{56pt}{【{\akai 西皮快板}】队队旌旗空中飘。(在后营有的是粮和草。众家太保杀气高:~)大太保亚赛温侯貌,二太保上阵似白袍。三太保上山擒虎豹,四太保下海斩龙蛟。五太保惯使开山斧,六太保手持丈八矛。七太保金枪({\akai 或}:~银枪)真奥妙,八太保手持青龙偃月刀。九太保双锏({\akai 或}:~金锏)耍得好,亚赛秦叔宝,十太保钢鞭逞英豪({\akai 或}:~鞭插马鞍桥)。还有个十一小太保,他的武艺好,双手能打火龙镖。哪怕黄巢兵来到,孤与他枪对枪来刀对刀。 }

\setlength{\hangindent}{56pt}{(程敬思\hspace{20pt}【{\akai 西皮摇板}】众家太保武艺好,你不发兵我心焦。) }

\setlength{\hangindent}{56pt}{【{\akai 西皮摇板}】贤弟休得心内焦({\akai 或}:~心烦恼),当饮酒时且逍遥。来来来,吃几杯解烦恼, }

\setlength{\hangindent}{56pt}{(程敬思\hspace{20pt}【{\akai 西皮摇板}】程敬思一旁闷无聊。) }

慌什么?\hspace{20pt}~

对她二人({\akai 或}:~对她们)言讲,现有长安贵客在此,少时({\akai 或}:~少刻)退帐再来传见。

又慌什么?\hspace{10pt}~

方才言过,退帐(再来)传见,为什么又来啰唣。

呃,忒以的啰嗦了。

哎呀,得罪了(,得罪了)。

罢了,你(们)二人进帐何事?

唐王无道({\akai 或}:~当初唐王将孤谪贬),哪有人马与他解围。

那是送与孤家的,提它则甚? ({\akai 或}:~呃,呃------俱都入了库了。)

(这凤冠霞帔么,)呃------(也)一齐入了库了。

孤心已定,休得多言({\akai 或}:~不必多奏)。

嗯------孤就是不发兵。({\akai 或}:~孤心已定,就是不发兵呐。)

(这做什么?)

啊?慢说是三声,(就是)三十声、三百声,又有何妨({\akai 或}:~又待何妨)啊?

呃,(呃,呃,我)一个不发兵。

嗯,(呃,呃,这)两个不发兵。

呃,你不必前来插足啊。({\akai 或}:~我们之间的事体,与台驾无关呐。)

呃,(诶------)我就是不发兵。

(诶呀!)\hspace{20pt}~

哦,倘若来迟呢?

(啊)太保,我们商量商量。

呃,我们商量商量。

唉!\hspace{40pt}~

\setlength{\hangindent}{56pt}{【{\akai 西皮摇板}】大太保本是惹祸精呐,到后宫搬来了两个夜叉妇人呐。顺水推舟我把人情送,我为你点动了({\akai 或}:~我为你发动了)番汉兵。 }

\setlength{\hangindent}{56pt}{(程敬思\hspace{20pt}【{\akai 西皮摇板}】千岁休得人情送,学生心内明如灯,皇娘人马来点动,程敬思不领你这空头情。) }

\setlength{\hangindent}{56pt}{【{\akai 西皮摇板}】贤弟休得笑盈盈,休笑愚兄({\akai 或}:~孤王)我怕,(我)怕,(我$\cdots{}\cdots{}$)怕妇人呐。沙陀国内访一访你再问一问,怕老婆的人儿我是第一名。 }

\vspace{3pt}{\centerline{{[}{\hei 第五场}{]}}}\vspace{5pt}

({\akai 念})白发白须似银条,胸中韬略智谋高。也是黄巢气数到呃,试试------孤的定唐刀。

唉!只因黄巢造反,勒逼唐王驾幸西祁美良。程恩官解押珠宝来到沙陀,借兵解围({\akai 或}:~搬兵解围)。本当不发人马({\akai 或}:~兵马),可笑我那两个无知的妇人,一个要发兵,一个要挂帅。发兵也罢,挂帅也罢,({\akai 或}:~是一个要什么挂帅,一个要什么发兵;唉,挂帅也罢,发兵也罢,)也不知怎么({\akai 或}:~怎样)糊里糊涂地,把一个({\akai 或}:~这个)前站先行,弄到孤家的头上来了。

本当不遵,怎奈她们的家法,实在地厉害呀({\akai 或}:~怎奈是她的家法十分地厉害)。为此紧急料理宫廷善后之事,全身披挂,校场听点。({\akai 或}:~为此急忙料理完毕宫廷善后之事,急忙辕门听点。)

来(来来),带马带马。

呃呃呃,你二人为何争论起来。

哦,你不是看守宫殿的老军么?

你(前)来则甚呐?

诶------两军阵前,刀枪无眼呐,倘有伤损({\akai 或}:~倘有差错),那还了得?(你呀,)还是养养你这老命吧!

哦,看你不出,倒有一片爱国的精神。({\akai 或}:~听你之言,倒有一片爱国的精神呐。)

嗯,就命你与孤带马。({\akai 或}:~好好好,我就教你带马。)

\vspace{3pt}{\centerline{{[}{\hei 第六场}{]}}}\vspace{5pt}

\setlength{\hangindent}{56pt}{【{\akai 西皮快板}】又听辕门({\akai 或}:~耳听辕门;又听营门)放号炮,众家太保杀气高({\akai 或}:~众家儿郎逞英豪;{\akai 或}:~儿郎个个杀气高)。来在辕门下鞍鞒, }

呵嘿!\hspace{30pt}~

\setlength{\hangindent}{56pt}{【{\akai 西皮摇板}】误卯牌悬挂({\akai 或}:~误卯牌高挂)要糟糕。 }

呃,来了!\hspace{10pt}~

我早就来了,怎么(说)误了呢?

传旨进去,就说孤王驾到,教她们快快迎接。({\akai 或}:~传话进去,就说孤王到了,教她们下位迎接于我。)

呃,我们是夫妇顺呐。

怎么讲?({\akai 或}:~啊?!)

呀呸!不来(下位)迎接,还则罢了,反教孤王({\akai 或}:~反教我)报门而进。

哼哼,呵呵,(这人马是孤家的,)我不干了,另请高明罢!

反了哇,反了哇!

\setlength{\hangindent}{56pt}{【{\akai 西皮摇板}】大太保传令理不通,不由孤王怒气生。({\akai 或}:~太保传令山摇震\footnote{此处``山摇震''或``山摇动''从俗,方与``胆战惊''对,下同。},不由孤王胆战惊。) }

\setlength{\hangindent}{56pt}{【{\akai 西皮摇板}】本当进帐({\akai 或}:~本当与她)来争论,怎奈是她的家法比国法还要狠十分。孤若不遵她的令,到晚来不教孤进孤的({\akai 或}:~她的)卧室门呐。 }

\setlength{\hangindent}{56pt}{【{\akai 西皮摇板}】东宫不留把西宫进, }

\setlength{\hangindent}{56pt}{【{\akai 西皮摇板}】西宫也是照样行------关门熄了灯。 }

\setlength{\hangindent}{56pt}{【{\akai 西皮摇板}】闹得孤({\akai 或}:~恼得孤)黑夜里无投奔,银安殿上把闷气生。 }

\setlength{\hangindent}{56pt}{【{\akai 西皮摇板}】孤王一生好把酒来饮({\akai 或}:~也是孤王好心性),好酒贪杯惯坏了她们呐。 }

\setlength{\hangindent}{56pt}{【{\akai 西皮摇板}】沙陀国内访一访,你再问一问,家家有本难念的经,个个观世音。叫老军与孤王 }

【{\akai 回龙}】你就报门进,

\setlength{\hangindent}{56pt}{【{\akai 西皮摇板}】上面坐定两个夜叉精。她二人狼狈为奸端了一个稳,她那里不语({\akai 或}:~不言)我也不作声。 }

({\hwfs 小嗓})不错,是我啊!({\akai 或}:~是啊,来啦!)

我这个先行,不是花钱买来的,也不是运动来的。乃是你们({\akai 或}:~你二人)亲自委派的。

孤王料理宫廷善后之事,一步来迟,何必(这样的)大惊小怪呀。

哎呀,糟了$\cdots{}\cdots{}$

(呃呃呃,还好还好啊。)

(哎呀呀,)这还了得?!哼!

这才是({\akai 或}:~这就是)父子亲呐。

哦,不用我了?这就好了。({\akai 或}:~怎么,你不用我了?)

劳您驾------

在。\hspace{30pt}~

得令啊!\hspace{20pt}~

古来无有的事,如今都有了!古来无有的事,如今都有了!

呃,(古来就有,)你且讲来。

看你不出,你还知道这么多的历史故事啊({\akai 或}:~倒晓得这么些个历史的知识啊)。

你讲得不错。

呃,孤(王)就是,呃,有这样({\akai 或}:~这么)一点点的短处。

呃,这就好了,这就好了。

呃,你不曾听见吗?

皇娘传令,赐孤({\akai 或}:~赐我)五千名虎卫军,压住后队。

嗯,这倒是一个美差。

哦,难道孤听错了?({\akai 或}:~怎么,难道孤听错了?)

呃------你这个人,怎么这样势利眼呐?

那些太保,你们看来一个个如狼似虎,他们只能在围场之上,行围射猎;在沙场之上,交锋对垒,还要看孤家。({\akai 或}:~呃,你不要看那些太保们,一个个如狼似虎,他们只能拿强捕盗,行围射猎;战场交锋,还要看孤家的。)

那个自然,带马!

\vspace{3pt}{\centerline{{[}{\hei 第七场}{]}}}\vspace{5pt}

\setlength{\hangindent}{56pt}{【{\akai 西皮快板}】将令一出山摇动({\akai 或}:~山摇震),儿郎个个胆战惊。来在营门下金镫({\akai 或}:~来在辕门下能行;{\akai 或}:~催马来在辕门近), }

\setlength{\hangindent}{56pt}{【{\akai 西皮摇板}】这样的紧急为何情呐。 }

哦,恩官。({\akai 或}:~哦哦哦,请坐请坐。)

哦,恩官。(呃呃呃,哎呀,得罪了得罪了。)

呃,(那)我的座位呢?

(哦,谢坐谢坐。)

呃,家无常礼呀。

调我前来则甚呐?({\akai 或}:~调孤前来有何军情议论?)

哦,谈谈心?

呃,(那)我们就谈谈心呐。

想是那周德威呀?

无名小辈,草莽贼寇,({\akai 或}:~草莽贼寇,无名小辈,)何足道哉?

会他一会么?(又有何妨啊?)

(哦,)今天不耐烦。

不伺候。\hspace{10pt}~

呃,什么叫作好处?你说将出来,我听上一听。({\akai 或}:~哦,我倒不晓得有什么好处,呃,有什么好处呢?)

(哦哦哦$\cdots{}\cdots{}$)

噫------(哈哈哈$\cdots{}\cdots{}$({\hwfs 笑}{\hwfs 介}))你不是骗了我一次了!我再也不上当了。({\akai 或}:~我再也不信了!你骗了我不是一次了!)

你说将出来,教大家听上一听({\akai 或}:~看上一看),看看使得使不得。

打仗也吃酒,不打仗也吃酒。这酒么------不稀罕({\akai 或}:~呃,吃酒不稀罕)。

呃,我们是家务事啊,不劳台驾呀。({\akai 或}:~诶,这是我们家务事,你不要前来插足哇。)

(你说)哪个老了?

(哦,你说孤王老了?)

(孤王)我老只老头上发,项下须,胸中韬略却还不老!

有道是:~({\akai 念})虎老雄心在,这------年迈呀------力刚强。

你(呀,)拿过来吧!

\setlength{\hangindent}{56pt}{【{\akai 西皮二六}】老只老孤的须发老,胸中的韬略比人高。非是孤王不服老,上阵全凭马和刀。草莽的贼寇何足道,教他来试一试孤的九九八十一斤定唐刀。 }

\setlength{\hangindent}{56pt}{【{\akai 西皮快板}】你把酒宴安排好,得胜回来贺贺功劳。叫老军与爷带马到, }

\setlength{\hangindent}{56pt}{【{\akai 西皮散板}】会一会山寇小儿曹。 }

\vspace{3pt}{\centerline{{[}{\hei 第八场}{]}}}\vspace{5pt}

({\hwfs 堂鼓轻击},龙套{\hwfs 在九龙口左右站斜旗门},\textless{}\!{\bfseries\akai 撕边}\!\textgreater{}李克用、周德威{\hwfs 左右旗门冲出},{\hwfs 双出门},李{\hwfs 从台中间左转身回到台中间漫}周{\hwfs 头},{\hwfs 打}周{\hwfs 鼻子}(周{\hwfs 与}李{\hwfs 相反走},\textless{}\!{\bfseries\akai 撕边}\!\textgreater{}李、周{\hwfs 分别左右转身回到旗门},{\hwfs 同时抱刀亮}、龙套{\hwfs 领起来分站左右},李、周{\hwfs 出来架住})

({\akai 念})呔!马前来的敢是周德威?

周德威,看你相貌堂堂,为何失身落草?

({\akai 或}:~我看你相貌堂堂,文韬武略,不该落草为寇。)依孤相劝,归顺孤家,封你以为一家太保,你且三思。

呜哙呀,他还惦记孤(家)的珠宝哩!

唉,全都烧光了。

哼,你若胜得过孤(王的)这定唐宝刀,(孤)将珠宝与你留下。

你若不胜?\hspace{10pt}~

丈夫一言------

(你我各传一令!)

众将官,压住阵脚。

\setlength{\hangindent}{56pt}{【{\akai 西皮导板}】叫三军与爷战鼓伐, }

({\hwfs 堂鼓轻击},{\hwfs 钻烟筒},\textless{}\!{\bfseries\akai 冲头}\!\textgreater{}{\hwfs 一合两合},{\hwfs 搕开},\textless{}\!{\bfseries\akai 紧锤}\!\textgreater{})

\setlength{\hangindent}{56pt}{【{\akai 西皮快板}】马前闪出年少娃。量儿本领有多大,敢与老夫动杀法伐。 }

({\hwfs 一合两合},李{\hwfs 接上下左右},{\hwfs 刀头拉转身},李{\hwfs 被漫头过去到小边},{\hwfs 向里回头打}周{\hwfs 后蓬头},{\hwfs 拉肚转身},李{\hwfs 被勾走马腰封到大边},{\hwfs 绞起来}李{\hwfs 压}周{\hwfs 刀},{\hwfs 往里一盖两盖},{\hwfs 往外一盖两盖},{\hwfs 起大刀花蹦子转身剁}周{\hwfs 头},{\hwfs 亮住}(李{\hwfs 平端刀指}周)。\textless{}\!{\bfseries\akai 香柳娘}\!\textgreater{}{\hwfs 亮收对面互看},{\hwfs 夸奖},{\hwfs 双出门分别左右转身对面拉开},{\hwfs 搕},{\hwfs 里面拉开}({\hwfs 不要正冠}),{\hwfs 搕},李{\hwfs 回花转身到大边里面抱刀单腿立亮相}(周{\hwfs 小边外面矮相}),李{\hwfs 向外边大刀花转身到大边外面斜横刀弓箭步矮相}(周{\hwfs 小边里面亮相}),李{\hwfs 向上场门出刀转身砍过去到中间里边},{\hwfs 面外抱刀亮相}(周{\hwfs 里面矮相}),{\hwfs 拉转身到小边},{\hwfs 对面拉开},{\hwfs 搕},{\hwfs 拉开},{\hwfs 搕},{\hwfs 回花转身到小边外面斜横刀矮相},{\hwfs 向里边大刀花转身到小边里面},{\hwfs 倒手外边面里斜横刀矮相},{\hwfs 拉},{\hwfs 回到大边}(\textless{}\!{\bfseries\akai 牌子}\!\textgreater{}{\hwfs 停}),{\hwfs 搭},{\hwfs 拉到大边里角},{\hwfs 一合到小边外角},{\hwfs 从小边外角经小边内角向大边外角退},{\hwfs 倒提柳到大边外角},{\hwfs 横着一个大刀花过合},{\hwfs 两个大刀花过合},{\hwfs 又到大边},{\hwfs 打}周{\hwfs 上下左右},{\hwfs 勾}周{\hwfs 走马腰封到大边},李{\hwfs 归小边},{\hwfs 原地被勾刀转身向里接鼻子},{\hwfs 在原地勾周刀左转身向外打}周{\hwfs 鼻子},{\hwfs 刀鐏盖}周{\hwfs 刀再打一个鼻子},{\hwfs 转身切刀亮住},{\hwfs 接耍下场},{\hwfs 普通大刀下场下},{\hwfs 但在劈马}、{\hwfs 正花转身面外亮住}、{\hwfs 串腕转身之后},{\hwfs 再来一个正花转身面外串腕},{\hwfs 弓箭步拿刀杆中间向外亮},{\hwfs 缓刀掠刀追下})\footnote{{\hei 这是谭鑫培、钱金福在《珠帘寨》李克用与周德威对刀的``刀架子'',余叔岩演《珠帘寨》也打这一套刀架子。}这套刀架子,特别是头子,非常有内容,{\hwfs 一上来周德威显得很冲,使人有李克用要招架不住之感,但是几下之后李就轻巧地控制了周。}这套把子层次分明,很有戏。}

\setlength{\hangindent}{56pt}{【{\akai 西皮散板}】接过雕翎箭一条。 }

\setlength{\hangindent}{56pt}{【{\akai 西皮散板}】这样的射法不算好,放箭哪有接箭高。 }

\setlength{\hangindent}{56pt}{【{\akai 西皮散板}】接过雕翎箭二根。 }

\setlength{\hangindent}{56pt}{【{\akai 西皮散板}】这样的射法不算准,孔夫子门前你卖的什么文。 }

\setlength{\hangindent}{56pt}{【{\akai 西皮散板}】勒马停蹄战场等,停箭不射为何情? }

周德威,战又不战,降又不降,又在那里弄的什么诡计?

你的箭法呀,哼,孤(王)方才领教过了。

(要)怎样比试?

但不知哪家先射?

孤若先射,就无有你的份了({\akai 或}:~孤王先射,就无有尔的份了!让你先射)!

站定了!\hspace{20pt}~

\setlength{\hangindent}{56pt}{【{\akai 西皮散板}】量尔不是汉李广,养由基再世又何妨。 }

\setlength{\hangindent}{56pt}{【{\akai 西皮散板}】满满搭上朱红扣, }

(诶,)你把我闹糊涂了!

\setlength{\hangindent}{56pt}{【{\akai 西皮散板}】观不见金钱在何方。 }

\setlength{\hangindent}{56pt}{【{\akai 西皮散板}】低下头来暗思想, }

啊?!\hspace{40pt}~

\setlength{\hangindent}{56pt}{【{\akai 西皮散板}】忽然一计上胸膛。 }

(周德威\hspace{20pt}李克用,$\cdots{}\cdots{}$停箭不射?)

非是孤王停箭不射,想这金钱({\akai 或}:~想那金钱)乃是一面死物,纵然射中,也不足为奇。

抬头观看------

空中飞的何物?

好哇------孤今({\akai 或}:~孤王一箭上去)要射它(一)个双雕落地。

丈夫一言------

(太保,)站定了。

\setlength{\hangindent}{56pt}{【{\akai 西皮散板}】克用暗地告上苍,祝告天地日月光、过往的神灵听端详:~我若有福收此将,箭射双雕落平阳。 }

(哪里走。)\hspace{10pt}~

\setlength{\hangindent}{56pt}{【{\akai 西皮散板}】德威可算英雄将({\akai 或}:~忠良将),封你太保在朝堂({\akai 或}:~在朝廊)。({\akai 或}:~你若真心把孤保,封你太保辅大唐。) }

(罢了,)见过众家哥弟。

不必查点。

转至御营({\akai 或}:~同至御营),见过你二位皇娘。

}

\vspace{20pt}
{\hei 据陈超老师介绍,《珠帘寨》有``烧宫''一场,刘曾复先生有特别传授,兹照录如下:~}

程敬思\hspace{20pt}({\akai 唱})李克用依记前仇恨,且喜皇娘发救兵。

程敬思\hspace{20pt}是我解宝到此搬兵,不想李克用记恨前仇不肯出兵,多蒙二位皇娘发动倾国人马,李克用以为先锋,兵马已在四十里外扎营。是我悄悄折回,将李克用的宫殿焚毁,教他此去难回也。

程敬思\hspace{20pt}({\akai 唱})非是我程敬思忒心狠,为保我主锦乾坤。

李克用\hspace{20pt}({\akai 内})【{\akai 西皮导板}】王宫火光冲天境,

(李克用、老军{\hwfs 上})

老军\hspace{30pt}老千岁,您可小心这点儿。

李克用\hspace{20pt}({\akai 唱})急忙转回看分明,顾不得即出征军情紧(圆场、下马)

老军\hspace{30pt}好大火啊(李克用扑火)。都烧成这样了,您就别忙活啦!

李克用\hspace{20pt}嘿嘿,

李克用\hspace{20pt}({\akai 唱})宫殿全已被火焚。

老军\hspace{30pt}回来您再盖新的吧。

李克用\hspace{20pt}唉!({\akai 唱})孤多年以来积攒的奇珍异宝------

老军\hspace{30pt}我的被窝褥子------

李克用\hspace{20pt}({\akai 唱})顷刻之间就化为飞灰,叫孤怎不心疼啊。

老军\hspace{30pt}老千岁您不错啦!我连被窝褥子都没啦!(\textless{}\!{\bfseries\akai 三鼓}\!\textgreater{})

老军\hspace{30pt}完了!聚将鼓响了,咱们再不走可来不及啦!

李克用\hspace{20pt}({\akai 唱})不住战鼓来催命,

李克用\hspace{20pt}马来!

李克用\hspace{20pt}({\akai 唱})此一去定回归重建宫廷。
