\newpage
\subsubsection{\large\hei {柴桑口}\protect\footnote{根据刘曾复先生钞录本整理。刘曾复先生钞本注``马少山本,民廿九 1940  王存''。段公平{\scriptsize 君}按:~据何毅老师介绍,刘曾复先生本有意为此剧本润色文辞,后因言派《卧龙吊孝》已流行,``我就别招这个讨厌了'',竟未完成。因此抄本中讹误脱漏较多,且多有刘曾复先生修改和原文并存痕迹。这次整理,在文辞通顺的基础上拟对这些痕迹适当保留。刘曾老所改录在正文,抄本原文录在脚注中。抄本整理过程中,幸得``戏考''网``小豆子''老师惠赐《柴桑口》余胜荪~藏本扫描件,和此抄本同质性很高,故多用为参考。}}
\addcontentsline{toc}{subsection}{\hei 柴桑口}

\hangafter=1                   %2. 设置从第1⾏之后开始悬挂缩进  %
\setlength{\parindent}{0pt}{

\vspace{3pt}{\centerline{{[}{\hei 第一场}{]}}}\vspace{5pt}
\setlength{\hangindent}{56pt}{({\hwfs 四}文堂,刘备、诸葛亮{\hwfs 上})}

\setlength{\hangindent}{56pt}{刘备\hspace{30pt}{[}{\akai 引子}{]}祸福惟天造,岂在人谋计巧。}

\setlength{\hangindent}{56pt}{诸葛亮\hspace{20pt}{[}{\akai 引子}{]}秦、崔玉童\footnote{秦即地狱秦广王,专司人间夭寿;~崔即判官崔钰,掌生死簿。刘曾复先生钞本中,``童''字不确认,疑``章''字;~玉童,即仙童。}到,从此知音稀少。}

\setlength{\hangindent}{56pt}{诸葛亮\hspace{20pt}参见主公。}

\setlength{\hangindent}{56pt}{刘备\hspace{30pt}先生少礼,请坐。}

\setlength{\hangindent}{56pt}{诸葛亮\hspace{20pt}谢座。}

\setlength{\hangindent}{56pt}{刘备\hspace{30pt}({\akai 念})周战场中祸难平,岂知转福结$\square$~$\square$。\footnote{刘曾复先生钞本``周''字不确认,亦欠通;~``结''字不确认,后二字缺。段公平{\scriptsize 君}按:~据余胜荪~藏本,此四句诗为``干戈有时化玉帛,蜀吴修好结姻亲。炎汉正统有天相,不须人谋定隆兴。''和此本中的诗应该颇相关。因此推断第二句所缺两字也是``姻亲''之类。第一句``难''字不能确认,或疑为``虽'',结合文意,作``难''字似乎较合理。}}

\setlength{\hangindent}{56pt}{诸葛亮\hspace{20pt}({\akai 念})自古辛勤有天下,不在人谋定相星。}

\setlength{\hangindent}{56pt}{刘备\hspace{30pt}孤,刘备。}

\setlength{\hangindent}{56pt}{诸葛亮\hspace{20pt}山人诸葛亮。}

\setlength{\hangindent}{56pt}{刘备\hspace{30pt}先生。}

\setlength{\hangindent}{56pt}{诸葛亮\hspace{20pt}主公。}

\setlength{\hangindent}{56pt}{刘备\hspace{30pt}日前周郎设下``假途灭虢''之计,被先生奇谋,只气得他喷血坠马\footnote{段公平{\scriptsize 君}按:~刘曾复先生钞本作``赞马'',似欠通,今据文意及《京剧汇编》第九十三集~余胜荪~藏本改。},此时未闻周郎吉凶如何。}

\setlength{\hangindent}{56pt}{诸葛亮\hspace{20pt}亮夜观天象,见将星坠落。我料周郎刻下必死无疑也。}

\setlength{\hangindent}{56pt}{刘备\hspace{30pt}此话难料。}

\setlength{\hangindent}{56pt}{旗牌\hspace{30pt}({\akai 内})走哇。}

(旗牌{\hwfs 上})

\setlength{\hangindent}{56pt}{旗牌\hspace{30pt}启上主公,昨日命小人持书递进吴营,周瑜拆开一观,忽然呕吐气绝身亡。今将灵柩移至柴桑去了。}

\setlength{\hangindent}{56pt}{诸葛亮\hspace{20pt}如何。}

\setlength{\hangindent}{56pt}{刘备\hspace{30pt}知道了。}

\setlength{\hangindent}{56pt}{诸葛亮\hspace{20pt}退下。}

(旗牌{\hwfs 下})

\setlength{\hangindent}{56pt}{刘备\hspace{30pt}果然不出先生所料,周郎已死,还当如何?}

\setlength{\hangindent}{56pt}{诸葛亮\hspace{20pt}我料代周郎之权必是鲁肃。亮观天象,见将星聚在东吴,我当以过江吊孝为由,好觅贤士辅佐\footnote{刘曾复先生钞本作``扶佐''。}主公。}

\setlength{\hangindent}{56pt}{刘备\hspace{30pt}且慢,东吴将士恨先生如入骨髓,此番前去,必遭其害。且莫做那披麻救火,自惹其祸。}

\setlength{\hangindent}{56pt}{诸葛亮\hspace{20pt}周瑜在日,亮犹不惧,今瑜(已)死\footnote{刘曾复先生钞本作``今瑜死``,此处从《三国演义》原文。},何足惧哉?}

\setlength{\hangindent}{56pt}{刘备\hspace{30pt}去得的?}

\setlength{\hangindent}{56pt}{诸葛亮\hspace{20pt}去得的。}

\setlength{\hangindent}{56pt}{刘备\hspace{30pt}无妨事?}

\setlength{\hangindent}{56pt}{诸葛亮\hspace{20pt}无妨事。}

\setlength{\hangindent}{56pt}{刘备\hspace{30pt}孤放心不下,可命三弟带领铁骑一万,战船千只,跟随先生前往。孤也好放心。}

\setlength{\hangindent}{56pt}{诸葛亮\hspace{20pt}既蒙主公垂念,就命四将军子龙随我前往,可命三将军在江南岸上等候。山人吊祭已毕,过江之时,用羽扇一招,前来接应便了。}

\setlength{\hangindent}{56pt}{刘备\hspace{30pt}先生呐。}

\setlength{\hangindent}{56pt}{刘备\hspace{30pt}【{\akai 西皮散板}\footnote{刘曾复先生钞本未注明西皮或二黄板式,今据后文及《京剧汇编》第九十三集~余胜荪~藏本推测。}】非是孤王人安顿,东吴尽是豺狼虎群。此番前去稍有伤损,岂不\textcolor{red}{教}孤\textcolor{red}{两}离分\footnote{刘曾复先生钞本``教''、``两''二字不确认,据文意推断。}。}

\setlength{\hangindent}{56pt}{诸葛亮\hspace{20pt}【{\akai 西皮散板}\footnote{刘曾复先生钞本注``散(板)转快(板)或二六''。}】时不到兴与衰天心造定,当治乱自有那一辈时人。【{\footnotesize 转}{\akai 西皮\textcolor{red}{快板}}】天生来周公瑾吴邦英俊,偏又有诸葛亮汉室称臣。三江口协力时同把曹併,彼爱我我爱彼各无异心。他起意三次里害我性命,心未遂气得他丧了残生。在帐中施一礼主心安稳,}

\setlength{\hangindent}{56pt}{刘备\hspace{30pt}先生保重了。}

\setlength{\hangindent}{56pt}{诸葛亮\hspace{20pt}【{\akai 西皮散板}】但看我一叶飘如风送云。}

\setlength{\hangindent}{56pt}{刘备\hspace{30pt}【{\akai 西皮散板}】他那里坦然不虑心安稳,孤心中不定胆战心惊。但愿得此一去吉星照定,且待他无凶险也好放心。}

\setlength{\hangindent}{56pt}{({\hwfs 同下})}

\vspace{3pt}{\centerline{{[}{\hei 第二场}{]}}}\vspace{5pt}

({\hwfs 四白}文堂、{\hwfs 二}旗牌、鲁肃{\hwfs 上})

\setlength{\hangindent}{56pt}{鲁肃\hspace{30pt}【{\akai \textcolor{red}{西皮摇板}}\footnote{刘曾复先生钞本未注明板式,今据《京剧汇编》第九十三集~余胜荪~藏本添。}】伤我那擎天柱一旦早丧,眼见得我东吴谁是栋梁。蒙主恩虽与我兵权执掌,愧匪才怎做得治国安邦。}

\setlength{\hangindent}{56pt}{鲁肃\hspace{30pt}下官鲁肃,不料孔明用计,竟将公瑾气死。蒙公瑾生前已奏表章,举我统领兵权。唉,我自愧匪才,焉能掌此大权,无奈主公再三\footnote{刘曾复先生钞本原文如此,文意欠通。},难以推却,只得勉力而为。今将公瑾灵柩移至柴桑,等候他子周循\footnote{刘曾复先生钞本作``周巡'',此处从《三国演义》原文,作``周循''。}到来,成服\footnote{旧时死者入殓后,其亲属穿着符合各自身分的丧服,称为``成服''。}丧葬也。}

\setlength{\hangindent}{56pt}{四将\hspace{30pt}({\akai 内})走哇。}

\setlength{\hangindent}{56pt}{(四将{\hwfs 上})}

\setlength{\hangindent}{56pt}{四将\hspace{30pt}参见都督。}

\setlength{\hangindent}{56pt}{鲁肃\hspace{30pt}列公少礼。}

\setlength{\hangindent}{56pt}{(鲁肃{\hwfs 看介})}

\setlength{\hangindent}{56pt}{鲁肃\hspace{30pt}列公怒气不息,进帐何事?}

\setlength{\hangindent}{56pt}{四将\hspace{30pt}启禀都督,今有孔明带了祭礼前来祭奠先帅,故而进帐请教都督:~还是将孔明杀了后祭,还是祭了后杀呢?}

\setlength{\hangindent}{56pt}{鲁肃\hspace{30pt}哦,那孔明竟敢前来吊祭先帅么?}

\setlength{\hangindent}{56pt}{四将\hspace{30pt}正是。}

\setlength{\hangindent}{56pt}{鲁肃\hspace{30pt}他带了多少人马?}

\setlength{\hangindent}{56pt}{四将\hspace{30pt}只有一叶扁舟,并无人马。}

\setlength{\hangindent}{56pt}{鲁肃\hspace{30pt}只有一叶扁舟,并无人马?}

\setlength{\hangindent}{56pt}{四将\hspace{30pt}正是。}

\setlength{\hangindent}{56pt}{鲁肃\hspace{30pt}({\hwfs 冷笑介})哼,呵呵$\cdots{}\cdots{}$这厮又来作怪。我东吴将士恨不得食尔之肉,他偏偏驾一叶扁舟而来。孔明啊孔明,你今前来岂不是羊入虎口。不知列公心意如何?}

\setlength{\hangindent}{56pt}{四将\hspace{30pt}吾家先帅被他气死,恨不得手刃此贼\footnote{刘曾复先生钞本作``千刃此贼'',此处据《京剧汇编》第九十三集~余胜荪~藏本改。}。他今自寻前来,都督传令可将他剖腹挖心\footnote{刘曾复先生钞本作``破腹挖心''。},活祭先帅,以报此仇。}

\setlength{\hangindent}{56pt}{鲁肃\hspace{30pt}不可。他今前来,定有一番道理。若凶惧而杀之,天下人道我东吴无容人之量。且待他祭奠之后,先用言语责他,然后治死也还不迟。不知列公意下如何?}

\setlength{\hangindent}{56pt}{四将\hspace{30pt}只是教那贼多活一时。}

\setlength{\hangindent}{56pt}{赵云\hspace{30pt}({\akai 内})孔明先生到。}

\setlength{\hangindent}{56pt}{鲁肃\hspace{30pt}列公不可造次,当遵吾令。}

\setlength{\hangindent}{56pt}{四将\hspace{30pt}遵命。}

\setlength{\hangindent}{56pt}{鲁肃\hspace{30pt}有请。}

\setlength{\hangindent}{56pt}{四将\hspace{30pt}有请。}

\setlength{\hangindent}{56pt}{(赵云、童儿、诸葛亮{\hwfs 上})}

\setlength{\hangindent}{56pt}{鲁肃\hspace{30pt}先生。}

\setlength{\hangindent}{56pt}{诸葛亮\hspace{20pt}都督。}

\setlength{\hangindent}{56pt}{鲁肃\hspace{30pt}呀,先生一别有年,使人梦想。}

\setlength{\hangindent}{56pt}{诸葛亮\hspace{20pt}{久未晤面}({\akai 或}:~久违教益),如有所亡\footnote{刘曾复先生钞本原录``如有所望'',后改``如有所忘'',段公平{\scriptsize 君}注:~``如有所忘''文意欠通。应为``如有所亡'',即``如有所失''意。}。}

\setlength{\hangindent}{56pt}{鲁肃\hspace{30pt}荷蒙\footnote{荷蒙,犹``承蒙''之意。}足下远来吊祭,足见多情。}

\setlength{\hangindent}{56pt}{诸葛亮\hspace{20pt}故交之谊,聊此一行,以表寸意。}

\setlength{\hangindent}{56pt}{鲁肃\hspace{30pt}多谢了。}

\setlength{\hangindent}{56pt}{诸葛亮\hspace{20pt}先灵供在何处?}

\setlength{\hangindent}{56pt}{鲁肃\hspace{30pt}现在后帐。待下引路。}

\setlength{\hangindent}{56pt}{诸葛亮\hspace{20pt}有劳了。}

({\hwfs 同}下)

\vspace{3pt}{\centerline{{[}{\hei 第三场}{]}}}\vspace{5pt}

(又{\hwfs 上})

\setlength{\hangindent}{56pt}{诸葛亮\hspace{20pt}\textless{}\!{\bfseries\akai 三叫头}\!\textgreater{}公瑾,先生,唉,都督哇,呃$\cdots{}\cdots{}$({\hwfs 哭介})}

\setlength{\hangindent}{56pt}{诸葛亮\hspace{20pt}【{\akai 二黄导板}\footnote{刘曾复先生钞本注``导(板)或散(板)''}】{见陵寝}({\akai 或}:~见灵寝)不由人泪如雨降,想俊容不由人痛断肝肠。}

\setlength{\hangindent}{56pt}{诸葛亮\hspace{20pt}\textless{}\!{\bfseries\akai 三叫头}\!\textgreater{}公瑾,先生,唉,都督哇,呃$\cdots{}\cdots{}$({\hwfs 哭介})}

\setlength{\hangindent}{56pt}{诸葛亮\hspace{20pt}【{\akai 二黄散板}】可惜你钟山秀{春年正旺}({\akai 或}:~春华正旺),}

\setlength{\hangindent}{56pt}{诸葛亮\hspace{20pt}【{\akai 反二黄原板}\footnote{刘曾复先生钞本注``以下散(板)或反二黄原板''}】可惜你美英才一旦夭亡。可惜你空碌碌一生容让,可惜你兢业业半世奔忙。实指望併曹瞒你我安享,\textless{}\!{\bfseries\akai 哭头}\!\textgreater{}都督哇$\cdots{}\cdots{}$}

\setlength{\hangindent}{56pt}{诸葛亮\hspace{20pt}【{\akai 反二黄原板}】又谁知黄粱梦\footnote{刘曾复先生钞本作``大数到'',此处据《京剧汇编》第九十三集~余胜荪~藏本改。}昙花一场。}

\setlength{\hangindent}{56pt}{旗牌\hspace{30pt}进位,上香,鞠躬。跪,叩首,二叩首,三叩首。兴,鞠躬。}

\setlength{\hangindent}{56pt}{赵云\hspace{30pt}({\akai 念})呜呼公瑾,不幸夭亡!寿短固天,人岂不伤!我心实痛,酬酒一觞;~君其有灵,想我衷肠。}

\setlength{\hangindent}{56pt}{诸葛亮\hspace{20pt}公瑾,想你英雄盖世,一代风流。贯精忠于日月,秉赤胆与东吴。不幸一旦身故,未遂你胸中之志,好不遗恨人也。}

\setlength{\hangindent}{56pt}{诸葛亮\hspace{20pt}【{\akai 反二黄\textcolor{red}{慢板}}\footnote{刘曾复先生钞本仅注反二黄,今据《京剧汇编》第九十三集~余胜荪~藏本添。}】你是个霸业的忠贞良将,你是个振东吴豪杰贤良。谁似你天生来高智雅量,谁似你文武略器宇轩昂。谁似你青年人兵权执掌,谁似你定霸业扶弱抑强。奸曹贼统雄兵如风似浪\footnote{刘曾复先生钞本原录``暗兵机如风波浪'',先生改为``统雄兵如风似浪''。},只吓得江南士束手要降。若非你怀大志陈兵相抗\footnote{刘曾复先生钞本原录``沉兵相挡'',先生改为``陈兵相抗''。},运机谋烧得他抛甲弃枪。到今日稍得遂太平景象\footnote{刘曾复先生钞本原录``烧得谁太平青浪'',先生改为``稍得遂太平景象''。},转瞬间\footnote{刘曾复先生钞本原录``话未了'',先生改为``转瞬间''。}天不佑大厦断梁。抛得我故人儿将谁依傍,\textless{}\!{\bfseries\akai 哭头}\!\textgreater{}公瑾呐$\cdots{}\cdots{}$}

\setlength{\hangindent}{56pt}{诸葛亮\hspace{20pt}【{\akai 反二黄\textcolor{red}{原板}}】闪得我前后事与谁商量。今日里原比作那伯仲情况,我与你又好比鸡黍范、张\footnote{``范张鸡黍''指范式、张劭一起喝酒食鸡。比喻朋友之间情义与深情。刘曾复先生钞本记作``稷黍范张''。}。{望阴灵鉴吾这虔诚祭享,虔诚祭享,泪纷纷捧玉樽享受烝尝。}({\akai 或}:~望阴灵鉴之我祭奠,知我祭奠,诸葛亮泪纷纷痛断肝肠。\footnote{刘曾复先生钞本原有此两句,后改用前句。})}

\setlength{\hangindent}{56pt}{旗牌\hspace{30pt}进位,上香,退,鞠躬。}

\setlength{\hangindent}{56pt}{赵云\hspace{30pt}({\akai 念})呜呼公瑾!生死永别!朴守其真,冥冥灭灭,君如有灵,乞见我心:~从此天下,更无知音!呜呼痛哉,伏惟上享。}

\setlength{\hangindent}{56pt}{诸葛亮\hspace{20pt}公瑾{\footnotesize 呐}$\cdots{}\cdots{}$}

\setlength{\hangindent}{56pt}{诸葛亮\hspace{20pt}【{\akai 反二黄\textcolor{red}{散板}}】你非是妒贤辈胸怀愚量,都只是各为主不得不防。到今日奸曹在你命身丧,\textless{}\!{\bfseries\akai 哭头}\!\textgreater{}都督{\footnotesize 哇}$\cdots{}\cdots{}$}

\setlength{\hangindent}{56pt}{诸葛亮\hspace{20pt}【{\akai 反二黄\textcolor{red}{散板}}】闷得我诸葛亮心意彷徨。思至此哭得我{咽喉气颡}({\akai 或}:~咽喉难让),}

\setlength{\hangindent}{56pt}{众\hspace{40pt}【{\akai 反二黄\textcolor{red}{散板}}】只哭得满营中泪洒千行。\footnote{刘曾复先生钞本注``全体同唱,鲁(肃)末句''。}}

\setlength{\hangindent}{56pt}{鲁肃\hspace{30pt}先生且免悲伤,还当同心破曹要紧。}

\setlength{\hangindent}{56pt}{众\hspace{40pt}是呀,同心破曹,全仗先生。}

\setlength{\hangindent}{56pt}{诸葛亮\hspace{20pt}亮纵有千言万语,一时难以尽诉。列公以奸曹为念,亮当佩服,与公同心破曹,就此告辞了。}

\setlength{\hangindent}{56pt}{鲁肃\hspace{30pt}且慢,备得水酒,聊表地主之情。}

\setlength{\hangindent}{56pt}{诸葛亮\hspace{20pt}本当领受,怎奈有公务在身,告辞了。}

\setlength{\hangindent}{56pt}{鲁肃\hspace{30pt}有慢了。}

\setlength{\hangindent}{56pt}{诸葛亮\hspace{20pt}\textless{}\!{\bfseries\akai 叫头}\!\textgreater{}公瑾!}

\setlength{\hangindent}{56pt}{诸葛亮\hspace{20pt}你若有灵,须见我心{\footnotesize 呐}!}

\setlength{\hangindent}{56pt}{诸葛亮\hspace{20pt}【{\akai \textcolor{red}{反二黄}散板}】心问口、口问心牢骚千状,有万篇写不尽我心哀伤。}

(诸葛亮{\hwfs 出门})

\setlength{\hangindent}{56pt}{众\hspace{40pt}送先生。}

\setlength{\hangindent}{56pt}{诸葛亮\hspace{20pt}【{\akai \textcolor{red}{反二黄}散板}】送千里终须别何须谦让,}

\setlength{\hangindent}{56pt}{鲁肃\hspace{30pt}恕不远送了。}

(赵云{\hwfs 下})

\setlength{\hangindent}{56pt}{诸葛亮\hspace{20pt}【{\akai \textcolor{red}{反二黄}散板}】试看我一帆风雨洒康庄\footnote{刘曾复先生钞本原录``你看我一阵风如在康庄'',先生改。}。}

(诸葛亮{\hwfs 下})

\setlength{\hangindent}{56pt}{鲁肃\hspace{30pt}【{\akai \textcolor{red}{反二黄}散板}】他那里诚恳恳哀泣模样,为朋友可算得古道热肠。}

\setlength{\hangindent}{56pt}{鲁肃\hspace{30pt}列公,人言先帅与孔明不睦,今日一见真乃伤情。}

\setlength{\hangindent}{56pt}{众\hspace{40pt}看将起来,诸葛先生乃是大大的好人。}

\setlength{\hangindent}{56pt}{二旗牌\hspace{20pt}({\akai 内})走哇。}

({\hwfs 二}旗牌{\hwfs 上})

\setlength{\hangindent}{56pt}{二旗牌\hspace{20pt}公子到。}

\setlength{\hangindent}{56pt}{鲁肃\hspace{30pt}有请。}

\setlength{\hangindent}{56pt}{众\hspace{40pt}有请。}

(周循{\hwfs 上})

\setlength{\hangindent}{56pt}{周循\hspace{30pt}伯父。}

\setlength{\hangindent}{56pt}{鲁肃\hspace{30pt}公子,不知公子驾到,未曾远迎,面前恕罪。}

\setlength{\hangindent}{56pt}{周循\hspace{30pt}岂敢。小侄来的鲁莽,伯父、众位伯父恕罪。}

\setlength{\hangindent}{56pt}{众\hspace{40pt}岂敢。}

\setlength{\hangindent}{56pt}{周循\hspace{30pt}我父灵堂今在何处?}

\setlength{\hangindent}{56pt}{鲁肃\hspace{30pt}现在后帐,随我来。}

\setlength{\hangindent}{56pt}{周循\hspace{30pt}有劳伯父。}

({\hwfs 同一翻两翻},周循{\hwfs 看介}、{\hwfs 哭介})

\setlength{\hangindent}{56pt}{周循\hspace{30pt}唉,爹爹呀$\cdots{}\cdots{}$({\hwfs 哭介})}

\setlength{\hangindent}{56pt}{周循\hspace{30pt}【{\akai \textcolor{red}{二黄}散板}】在灵位不由我死去又醒,}

\setlength{\hangindent}{56pt}{周循\hspace{30pt}\textless{}\!{\bfseries\akai 三叫头}\!\textgreater{}爹爹,我父,唉,爹爹呀$\cdots{}\cdots{}$({\hwfs 哭介})}

\setlength{\hangindent}{56pt}{周循\hspace{30pt}【{\akai \textcolor{red}{二黄}散板}】想今生要见面万万不能。为国家受尽了千般苦衷,谁信那青史上万载标名。}

\setlength{\hangindent}{56pt}{鲁肃\hspace{30pt}啊公子,先帅已死,不能复生,请自保重要紧。}

\setlength{\hangindent}{56pt}{众\hspace{40pt}是啊,请自保重要紧。}

\setlength{\hangindent}{56pt}{周循\hspace{30pt}诚领众位伯父之教,小侄敢不从命。啊伯父,小侄有一言不知可听否?}

\setlength{\hangindent}{56pt}{鲁肃\hspace{30pt}公子有话请讲。}

\setlength{\hangindent}{56pt}{周循\hspace{30pt}先父执掌兵权有年,不料被孔明三气而死,我与他不共戴天之仇,望伯父助小侄一膀之力,杀往荆(州\footnote{此处据《京剧汇编》第九十三集~余胜荪~藏本添加。}),生擒孔明,与我父报仇雪恨。伯父料无推辞的了。}

\setlength{\hangindent}{56pt}{鲁肃\hspace{30pt}啊公子,那孔明乃是个好人。}

\setlength{\hangindent}{56pt}{周循\hspace{30pt}啊?怎见得他是好人?}

\setlength{\hangindent}{56pt}{鲁肃\hspace{30pt}他闻先帅已死,过得江来哭了又祭,祭了又哭,岂不是个好人?}

\setlength{\hangindent}{56pt}{周循\hspace{30pt}啊,那孔明几时来的?}

\setlength{\hangindent}{56pt}{鲁肃\hspace{30pt}方才在此。}

\setlength{\hangindent}{56pt}{周循\hspace{30pt}如今何在?}

\setlength{\hangindent}{56pt}{鲁肃\hspace{30pt}回往荆州去了。}

\setlength{\hangindent}{56pt}{周循\hspace{30pt}啊伯父,你要与我统领人马追杀孔明,如若不然,我就碰死在灵前。}

\setlength{\hangindent}{56pt}{众\hspace{40pt}公子不必如此,大家追杀孔明便了。}

\setlength{\hangindent}{56pt}{周循\hspace{30pt}好,快快追赶。}

\setlength{\hangindent}{56pt}{鲁肃\hspace{30pt}去不得。}

({\hwfs 同下})

\vspace{3pt}{\centerline{{[}{\hei 第四场}{]}}}\vspace{5pt}

(赵云、童儿、诸葛亮{\hwfs 上})

\setlength{\hangindent}{56pt}{诸葛亮\hspace{20pt}【{\akai 西皮散板}】非是我笑他们无有志量,怎知我袖儿内暗有行藏。{遇不着智谋人心中惆怅}({\akai 或}:~将身儿来至在江边岸上),}

(庞统{\hwfs 上})

\setlength{\hangindent}{56pt}{庞统\hspace{30pt}呔,你往哪里走。}

\setlength{\hangindent}{56pt}{庞统\hspace{30pt}【{\akai 西皮散板}】你纵有托天手({\akai 或}:~托天胆)难逃罗网。}

\setlength{\hangindent}{56pt}{庞统\hspace{30pt}孔明呐孔明,你用计将周郎三气而死,又假意过江吊祭,分明笑我东吴无有能人。来来来,我与你较量。}

\setlength{\hangindent}{56pt}{诸葛亮\hspace{20pt}原来是凤雏先生。先生平生大才,今日出此不经之言,故意吓我。}

\raisebox{0pt}[22pt][16pt]{\raisebox{8pt}{诸葛亮}\raisebox{-8pt}{\hspace{-32pt}{庞统}}\raisebox{0pt}{\hspace{30pt}哈哈,哈哈,啊哈哈哈$\cdots{}\cdots{}$(诸葛亮、庞统{\hwfs 对笑介})}}

\setlength{\hangindent}{56pt}{诸葛亮\hspace{20pt}啊先生,我此来实为吾兄,我料仲谋必不能重用足下,玄德公宽仁厚德,(不负\footnote{据``中国京剧戏考''网站载《戏考》本添加。})公生平所学,我有草书一封,趁便即往\footnote{刘曾复先生钞本原录``趁乘谢之时,去往'',欠通,后改为``趁便即往''。}荆州共扶汉室,名垂千古,岂不美哉?}

\setlength{\hangindent}{56pt}{庞统\hspace{30pt}承蒙美意,自得遵教\footnote{刘曾复先生钞本作``尊教''。}。}

(起\textless{}\!{\bfseries\akai 鼓架子}\!\textgreater{},诸葛亮、庞统{\hwfs 两望})

\setlength{\hangindent}{56pt}{庞统\hspace{30pt}啊先生,看那旁人声呐喊,必是周循追赶前来。}

\setlength{\hangindent}{56pt}{诸葛亮\hspace{20pt}公且自回避,亮要登舟去了。}

\setlength{\hangindent}{56pt}{庞统\hspace{30pt}后会有期。}

\setlength{\hangindent}{56pt}{诸葛亮\hspace{20pt}请呐。}

\setlength{\hangindent}{56pt}{诸葛亮\hspace{20pt}【{\akai 西皮散板}】此时节说不尽话言惆怅,}

\setlength{\hangindent}{56pt}{庞统\hspace{30pt}【{\akai 西皮散板}】暂分别改日里再会荆襄。}

(庞统{\hwfs 下})

\setlength{\hangindent}{56pt}{诸葛亮\hspace{20pt}哈哈哈$\cdots{}\cdots{}$({\hwfs 笑介})}

\setlength{\hangindent}{56pt}{诸葛亮\hspace{20pt}【{\akai 西皮散板}】想人生荣与枯得失难量,际风云{显奇谋}({\akai 或}:~显奇能)姓字名扬。望一派{白茫茫}({\akai 或}:~白亮亮){翻江波浪}({\akai 或}:~滔天波浪),}

\setlength{\hangindent}{56pt}{张飞\hspace{30pt}({\hwfs 下场门}{\akai 内})嘚,开船。}

\setlength{\hangindent}{56pt}{({\hwfs 四黑}龙套、{\hwfs 二}船夫、张飞{\hwfs 上})}

\setlength{\hangindent}{56pt}{张飞\hspace{30pt}【{\akai 西皮散板}】张翼德接先生来到长江。}

\setlength{\hangindent}{56pt}{张飞\hspace{30pt}先生,搭了扶手。}

(诸葛亮、赵云、童儿{\hwfs 上船介};~{\hwfs 四白}文堂、四将、周循{\hwfs 上})

\setlength{\hangindent}{56pt}{周循\hspace{30pt}呔,船头之上,可是诸葛亮?}

\setlength{\hangindent}{56pt}{诸葛亮\hspace{20pt}然也,来者何人?}

\setlength{\hangindent}{56pt}{周循\hspace{30pt}俺乃公瑾之子周循是也。}

\setlength{\hangindent}{56pt}{诸葛亮\hspace{20pt}哦,原来是公子到了,敢莫是与父谢孝的么?}

\setlength{\hangindent}{56pt}{周循\hspace{30pt}正是。}

\setlength{\hangindent}{56pt}{诸葛亮\hspace{20pt}为何持戈相向,是何理也?}

\setlength{\hangindent}{56pt}{周循\hspace{30pt}请先生上岸,周循有话言讲。}

\setlength{\hangindent}{56pt}{诸葛亮\hspace{20pt}哼呵呵呵$\cdots{}\cdots{}$({\hwfs 冷笑介})我若上岸,只恐你那小性命必随儿父去也。}

\setlength{\hangindent}{56pt}{周循\hspace{30pt}呔,孔明你若不上岸,休怪周循无礼了。}

\setlength{\hangindent}{56pt}{张飞\hspace{30pt}呔,我把你这不孝的乳臭小儿,汝父既死,儿不居守灵帐,执持器械,何以成孝?你这不忠不孝、不仁不义之人,要儿何用!先生闪开,待咱老张将他射死也。}

\setlength{\hangindent}{56pt}{诸葛亮\hspace{20pt}不可,饶他这条小命去罢。}

\setlength{\hangindent}{56pt}{张飞\hspace{30pt}也罢,念尔有重孝在身,暂且饶儿不死。嘚,开船!}

(张飞{\hwfs 三笑},诸葛亮众{\hwfs 下})

\setlength{\hangindent}{56pt}{周循\hspace{30pt}苍天{\footnotesize 呐}苍天,}

\setlength{\hangindent}{56pt}{周循\hspace{30pt}【{\akai \textcolor{red}{西皮摇板}}\footnote{刘曾复先生钞本未注明板式,今据《京剧汇编》第九十三集~余胜荪~藏本添。}】满江洒下青丝网,怎奈鱼儿又脱缰。}

\setlength{\hangindent}{56pt}{周循\hspace{30pt}罢!}

(周循{\hwfs 跳水介},{\hwfs 四}将{\hwfs 拦介})

\setlength{\hangindent}{56pt}{四将\hspace{30pt}公子不必如此,驾船追杀孔明便了。}

\setlength{\hangindent}{56pt}{周循\hspace{30pt}好。驾船追者!}

({\hwfs 同下})

}

\vspace{20pt}
{\hei 注:~钞本中诸葛亮祭奠周瑜的祭文,个别词句与《三国演义》原文音同字异。今将《三国演义》中诸葛亮祭文附后:~}

\vspace{15pt}

\hspace*{55pt}{呜呼公瑾,不幸夭亡!~修短故天,人岂不伤?}

\hspace*{55pt}{我心实痛,酹酒一觞;~君其有灵,享我烝尝!}

\hspace*{55pt}{吊君幼学,以交伯符;~仗义疏财,让舍以民。}

\hspace*{55pt}{吊君弱冠,万里鹏抟;~定建霸业,割据江南。}

\hspace*{55pt}{吊君壮力,远镇巴丘;~景升怀虑,讨逆无忧。}

\hspace*{55pt}{吊君丰度,佳配小乔;~汉臣之婿,不愧当朝,}

\hspace*{55pt}{吊君气概,谏阻纳质;~始不垂翅,终能奋翼。}

\hspace*{55pt}{吊君鄱阳,蒋干来说;~挥洒自如,雅量高志。}

\hspace*{55pt}{吊君弘才,文武筹略;~火攻破敌,挽强为弱。}

\hspace*{55pt}{想君当年,雄姿英发;~哭君早逝,俯地流血。}

\hspace*{55pt}{忠义之心,英灵之气;~命终三纪,名垂百世;}
 
\hspace*{55pt}{哀君情切,愁肠千结;~惟我肝胆,悲无断绝。}

\hspace*{55pt}{昊天昏暗,三军怆然;~主为哀泣,友为泪涟。}

\hspace*{55pt}{亮也不才,丐计求谋;~助吴拒曹,辅汉安刘;}

\hspace*{55pt}{掎角之援,首尾相俦;~若存若亡,何虑何忧?}

\hspace*{55pt}{呜呼公瑾!~生死永别!}

\hspace*{55pt}{朴守其贞,冥冥灭灭;~魂如有灵,以鉴我心:~从此天下,更无知音!}

\hspace*{55pt}{呜呼痛哉!~伏惟尚飨。}
