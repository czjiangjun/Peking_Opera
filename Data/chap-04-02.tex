\newpage
\phantomsection %实现目录的正确跳转
\section*{\large\hei {下河东~{\small 之}~呼延寿廷}}
\addcontentsline{toc}{section}{\hei 下河东~{\small 之}~呼延寿廷}

\hangafter=1                   %2. 设置从第1⾏之后开始悬挂缩进  %}
\setlength{\parindent}{0pt}{

{\vspace{3pt}{\centerline{{[}{\hei 第一场}{]}}}\vspace{5pt}}

{领旨!}

{({\akai 念})怀揣忠义胆,保主锦江山。}

{臣呼延寿廷见驾,吾皇万岁!}

{万万岁。}

{宣臣上殿,有何国事议论?}

{欧相乃是文职官员,焉能挂得武将帅印?}

{臣与欧相有打牙仇恨,此番到了河东,犹恐}\footnote{段公平{\scriptsize 君}建议作``又恐''。}{以公报私。}

{万岁作主。}

{万万岁。}

{参见元帅!}

{身为大将,焉能不晓军令?}

{一捆四十。}

{两捆八十。}

{这三卯------}

{呵呵呵呵$\cdots{}\cdots{}$({\hwfs 冷笑介})}

{也不过就是项上的人头。}

{贼呀,贼!}

{({\akai 念})身居矮檐下,怎敢不低头。}

{哎!}

\vspace{3pt}{\centerline{{[}{\hei 第二场}{]}}}\vspace{5pt}

{回府。}

{可恼!}

{今有河东打来连环战表,要我主御驾亲征。万岁命欧相挂帅,下官以为前战先行。}

{欧相奏道:~幼年习文,中年习武。({\akai 念})习就文共武,扶保帝王都。}

{好个有道明君,言道:~待等平定河东回来,与我两家解和。}

{如此有劳夫人。}

{正是:~({\akai 念})青龙背上屯军马。}

\setlength{\hangindent}{56pt}{【{\akai 二黄导板}】这几年未出征干戈宁静,}

\setlength{\hangindent}{56pt}{【{\akai 二黄散板}】玲珑铠甲挂灰尘。}

\setlength{\hangindent}{56pt}{【{\akai 二黄散板}】迈步且把二堂进,有劳夫人点雄兵。}

\setlength{\hangindent}{56pt}{【{\akai 二黄散板}】接过夫人得胜饮,背转身来谢神灵。回头再对夫人论,下官言来你试听}\footnote{段公平{\scriptsize 君}建议作``你是听''。}{:~倘若河东遭不幸,这是呼家报仇人。}

\setlength{\hangindent}{56pt}{【{\akai 二黄散板}】辞别夫人跨金镫,}

\setlength{\hangindent}{56pt}{【{\akai 二黄散板}】但愿此去奏凯回程。}

\vspace{3pt}{\centerline{{[}{\hei 第三场}{]}}}\vspace{5pt}

{参见元帅。}

{这$\cdots{}\cdots{}$披挂来迟,元帅恕罪。}

{谢元帅!}

{圣驾到!}

{在。}

{得令。}

{令出:~圣上有旨,元帅有令:~文武百官免送。人马打从德胜门}\footnote{《京剧汇编》第十六集~苏连汉~藏本作``得胜门''。}{而出,就此响炮离京。}

{\vspace{3pt}{\centerline{{[}{\hei 第四场}{]}}}\vspace{5pt}}

{前站为何不行?}

{候令!}

{启禀元帅:~前面已到河东地界。}

{在,}

{得令。}

{令出:~下面听者:~圣上有旨,元帅有令:~就在此地,择一平阳所在,靠山近水,安营扎寨。歇兵三日,再与河东鏖战呐!}

{传令已毕。}

{\vspace{3pt}{\centerline{{[}{\hei 第五场}{]}}}\vspace{5pt}}

{({\akai 念})来到河东地,昼夜费心机。}

{带马。}

{(呼延寿廷{\hwfs 从上手接枪上马},{\hwfs 左转身到上场门},{\hwfs 右手出枪亮},{\hwfs 小绕到小边台口双手托枪},{\hwfs 走到下场门扎出去跺泥右手收回来举枪亮},{\hwfs 回来到台中间右转身面向外出枪提枪花},{\hwfs 转身},{\hwfs 三个提枪花},{\hwfs 枪上右手膀子},{\hwfs 枪头向左},{\hwfs 右手在胸前平端枪},{\hwfs 左手向左伸出接枪杆},{\hwfs 跨左腿},{\hwfs 踢右腿},{\hwfs 右脚落地},{\hwfs 右手在脸前画圈拿枪杆下端},{\hwfs 向右翻身面向下场门},{\hwfs 枪交左手拿枪杆下端在左腰间平端},{\hwfs 右手画圈握拳勒马},{\hwfs 弓箭步亮住},{\hwfs 下场门下})}\footnote{{\hei 这是杨小楼晚年最常用的一个下场},不仅《下河东》、《阳平关》、《战宛城》等戏用它,《挑华车》、《霸王别姬》大铲枪下场也用它,《铁笼山》姜维与司马师,《贾家楼》唐璧与来护,一前一后,{\hwfs 提枪花},{\hwfs 并马双下场}也用它。\\{\hei 《下河东》呼延寿廷耍两个枪下场。头一个下场是呼延闻报后急于出马救欧阳芳,所以上马后只耍一个小下场表示急忙上阵。第二个下场是呼延杀败白龙太子,追击一程,用大下场。}}

{\vspace{3pt}{\centerline{{[}{\hei 第六场}{]}}}\vspace{5pt}}

{元帅受惊了,元帅受惊了!}

{你与白龙交战,堪堪落马,多亏末将一马当先,将你救下。来来来,请上功劳簿{\footnotesize 哇}。}

{这$\cdots{}\cdots{}$无有。}

{也无有。}

{谢元帅责!}

{打的------呃,不公!}

{也不是。}

{有罪不敢抬头。}

{谢元帅。}

{是是是。}

\setlength{\hangindent}{56pt}{【{\akai 二黄散板}】奸贼做事太欺情,有功不赏反加刑。}

\setlength{\hangindent}{56pt}{【{\akai 二黄散板}】人来搀我小营进,快请姑娘到此营。}

{贤妹哪里知道,老贼与白龙交战,堪堪落马,多亏愚兄将他救下。}

{有功劳不赏,反将愚兄------唉,重责。呃$\cdots{}\cdots{}$({\hwfs 哭介})}

{贤妹呀!}

\setlength{\hangindent}{56pt}{【{\akai 二黄散板}】随王驾来愿王兴,食王爵禄当报恩。军权落在奸贼手哇,得自闲来且自清。}

{\vspace{3pt}{\centerline{{[}{\hei 第七场}{]}}}\vspace{5pt}}

{臣在暗地保驾。}

}
