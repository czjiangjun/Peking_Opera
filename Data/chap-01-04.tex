\newpage
\hypertarget{ux9a6cux978dux5c71}{%
\subsection{马鞍山}\label{ux9a6cux978dux5c71}}

(李舒遗作 根据刘曾复先生手书原稿抄录)

\textbf{{[}第一场{]}}

(童儿、俞伯牙小锣打上)

俞伯牙 {[}引子{]}为访贤友,涉水登舟。

俞伯牙
{[}诗{]}青溪流过碧山头,空水澄鲜一色秋。隔断红尘三十里,白云红叶两悠悠。\protect\hyperlink{fn31}{\textsuperscript{31}}

俞伯牙
下官姓俞名瑞字伯牙,(乃)鲁国人氏,晋国为官。只因去岁往各国催贡,船行马鞍山前,偶遇钟子期,我二人共谈琴律,情意相投,结为金兰之好,临行(之时,)赠他黄金二笏,约定今岁中秋还在马鞍山前相会。来此不见贤弟到来,昨晚琴音缭乱\protect\hyperlink{fn32}{\textsuperscript{32}},不知是何缘故,我不免去往集贤村寻访于他便了。

俞伯牙 童儿。

(童儿 有。)

俞伯牙 看衣改换。

(\textless{}\textbf{小开门}\textgreater{},下,再上)

俞伯牙 带了瑶琴,(随我往)集贤村去者!

(一翻两翻,半个\textless{}\textbf{扯四门}\textgreater{},小童一直站大边)

俞伯牙
【二黄三眼】我二人在山前金兰结好,今此来他不到所为哪条(或:今此来不见他所为哪条)。换冠裳我亲自义友寻找,此一去集贤村访见故交。

(\textless{}\textbf{小锣打下}\textgreater{})

\textbf{{[}第二场{]}}

钟元普\protect\hyperlink{fn33}{\textsuperscript{33}} (内白)走哇。

(提篮子,\textless{}\textbf{小锣抽头}\textgreater{}上)

钟元普 唉!

钟元普 【二黄摇板】屋漏偏遭连阴雨,破船又遇当头风。

钟元普
老汉钟元普。吾儿(或:亡儿)名唤子期。只因去岁,中秋在马鞍山前砍柴,偶遇一位晋国大夫俞伯牙大人,他二人共谈琴律,情意相投,结为金兰之好。临别(或:临行)赠我儿黄金二笏,约定今岁中秋还在马鞍山前相会。谁想我儿回得家来,白日砍柴,夜晚攻书,朝暮积劳,染成疾病。他就此一命身亡了\ldots{}\ldots{}(钟元普哭介)

钟元普
咳,今当吾儿百日之期(或:今当亡儿百日之期),为此备了几陌纸钱,去往坟前烧化。天呐,天,(念)家有万贯终何用,老来无子一场空。

钟元普
【二黄原板】老眼昏花路难行,又闻得(或:又听得)松林内百鸟喧声。乌鸦倒有反哺意,羊羔也有跪乳情。似乌云遮住了天边月,似狂风吹散了满天云。这才是黄梅已老青梅落,白发人反送了黑发儿的身。我的儿呀!

(\textless{}\textbf{小锣抽头}\textgreater{}下)

(俞伯牙接\textless{}\textbf{小锣抽头}\textgreater{}上)

俞伯牙
【二黄摇板】昨夜晚抚瑶琴暗藏悲调,看起来这内中事有蹊跷。移步儿来至在双阳岔道,(\textless{}\textbf{小锣抽头}\textgreater{}圆场)寻不着集贤村路走哪条?

俞伯牙 哎呀且住,来此已是双阳岔道,但不知这集贤村往哪条道路而走。

钟元普 (内嗽)嗯哼。

俞伯牙 看那旁来一老丈。等他到来问明再走。

(钟元普上)

钟元普
【二黄摇板】曲弯弯行过了溪边小道,哪有个父与子把纸来烧(或:把纸化烧)。(过大边)

俞伯牙 老丈请转。

钟元普 呃,原来是一位先生,这位先生可是失迷路途?

俞伯牙 正是。

钟元普 但不知问的是何所在?

俞伯牙 我问的是集贤村。

钟元普
先生,你来看:这东去十里也是集贤村,西去十里也是集贤村。但不知是哪个集贤村呢?

俞伯牙
这\ldots{}\ldots{}哎呀,贤弟呀,现有两个集贤村(或:既有两个集贤村),为何不对愚兄说明,如今叫我作难了。

钟元普 啊先生,敢是指路不明?

俞伯牙 呃呃,久住三五载,

钟元普 无处不亲连。

俞伯牙 正是。

俞伯牙、钟元普 啊哈哈哈哈。

钟元普 但不知问的是哪一家?

俞伯牙 我问的是钟子期。

钟元普 哦,钟子期。

俞伯牙 正是。

钟元普 咳,儿呀。

钟元普 【二黄摇板】相逢未说几句话,不由老汉泪如麻。(哭介)

钟元普 先生你来迟了。

俞伯牙 (老丈)何言来迟?

钟元普
老汉钟元普。吾儿子期(或:亡儿子期),只因去岁中秋与俞大人结拜(或:与俞大人结为金兰之好),分别之后回到家中,他白日砍柴,夜晚攻书,积劳成疾(或:积劳成病),百日前(他)一命身亡了\ldots{}\ldots{}

俞伯牙 你待怎讲?

钟元普 一命身亡了。

俞伯牙 哎呀!(\textless{}\textbf{崩登仓冲头}\textgreater{})

(俞伯牙昏介)

钟元普 这是何人?

童儿 这就是俞大人。

钟元普 哦哦(或:哎呀),大人醒来。

俞伯牙 【二黄导板】听说是钟贤弟一命丧了,

俞伯牙 \textless{}\textbf{三叫头}\textgreater{}贤弟! 子期!哎贤弟呀。

俞伯牙 【二黄散板】此一番好一似马行断桥。他的父是尊长急忙拜倒,

钟元普 【二黄散板】请大人莫折煞年迈山樵。

(俞伯牙 老伯。)

俞伯牙 【二黄散板】我就是俞伯牙伯父知晓,贤弟死留何言细说根苗。

钟元普 【二黄散板】我的儿临危时也曾言道:葬埋在马鞍山候驾来瞧。

俞伯牙 【二黄散板】烦伯父你与我坟台引道,

(\textless{}\textbf{扭丝}\textgreater{},钟元普、俞伯牙同走圆场)

钟元普 【二黄散板】这就是新坟土尚挂纸标。

(俞伯牙 哎呀!)

俞伯牙 【二黄散板】见坟台不由我双膝跪倒,呼不应、唤不醒生死故交。

俞伯牙 贤弟呀\ldots{}\ldots{}(哭介)

俞伯牙 (啊,)老伯,那旁有一石台,老伯稍坐一时,待侄儿一祭。

(俞伯牙哭介)

钟元普 有劳大人。

俞伯牙 童儿。

(童儿 有。)

俞伯牙 将我瑶琴摆在坟前。

(此处上渔、樵)

俞伯牙
唉!(念)此来空枉费,人琴付东流\protect\hyperlink{fn34}{\textsuperscript{34}}。灵魂渺茫去呀,可叹一土丘。

俞伯牙
\textless{}\textbf{帽子头}\textgreater{}【二黄慢板】想去岁中秋节论琴交好,今日里见坟台不见故交。来时喜去时悲愁云渺渺,又只见秋风起黄叶飘飘。为贤弟我不爱黄金荣耀,为贤弟我不爱玉带紫袍。为贤弟二双亲少行孝道,为贤弟辞王驾亲走这遭。为贤弟终日里梦魂颠倒,为贤弟千里迢迢,涉水登山,枉费徒劳。实指望与贤弟同饮香醪,实指望与贤弟共论琴操。实指望与贤弟朝夕欢笑,实指望与贤弟春游芳草,夏赏荷香,秋饮菊酒,冬藏梅阁,散淡逍遥。在坟台抚瑶琴以为祭吊,

(俞伯牙抚琴介)

(俞伯牙 唉!)

俞伯牙 【二黄散板】子期死少知音琴对谁调。我这里将瑶琴摔碎不要,

(俞伯牙摔琴介\textless{}\textbf{乱锤}\textgreater{})

钟元普 【二黄散板】问大人摔瑶琴所为哪条?

俞伯牙 老伯,

俞伯牙
(念)摔碎瑶琴凤尾寒,子期不在向谁弹?春风满面皆朋友,要会知音难上难。(俞伯牙哭介)

俞伯牙 【二黄散板】问伯父贤弟死家有何靠,

钟元普 【二黄散板】隐居在集贤村倒也逍遥(或:倒还逍遥)。

俞伯牙
【二黄散板】这黄金与伯父甘旨\protect\hyperlink{fn35}{\textsuperscript{35}}养老,且待我迎接你替他代劳。

俞伯牙 老伯,侄儿去后伯父不要思他。

钟元普 我不思他。

俞伯牙 不要想他。

钟元普 我也不想他。(或:呃,我不想他。)

俞伯牙 子期是我。

钟元普 (呃,)不敢。

俞伯牙 我是子期。

钟元普 实实不敢(或:唉,越发地不敢呐)。

俞伯牙 小侄告辞了。

俞伯牙 【二黄散板】辞伯父别坟墓扬长就道,

钟元普 【二黄散板】虽异姓似手足犹如同胞。

俞伯牙 【二黄散板】伯牙在\ldots{}\ldots{}

钟元普 【二黄散板】子期死(啊)\ldots{}\ldots{}

俞伯牙 (接唱)【二黄散板】知音缺少,摔瑶琴谢知音不负故交。

俞伯牙 \textless{}\textbf{三叫头}\textgreater{}老伯,子期,唉贤弟呀。

(钟元普 \textless{}\textbf{三叫头}\textgreater{}大人,我儿,唉儿呀。)

俞伯牙 罢!

(俞伯牙下,小童同下)

(钟元普 唉!)

钟元普
【二黄散板】似这等金兰友如同管鲍,转身来见坟台不见儿曹。猛抬头见红日西山落了,回家去与老妻细说根苗。

钟元普 儿呀!

(小锣打下\textless{}\textbf{尾声}\textgreater{})

\textbf{本戏人物扮相}:

俞伯牙 纱帽,蓝帔,黑三,高方巾,宝蓝褶子,绦子。

钟元普 白氈帽,白满,白老斗衣,腰包,鞋。

童儿 抓髻,白花褶子,鞋。

\textbf{道具}:

小篮一只,内装纸钱。

石台,即倒椅两把。

\textbf{附}:

渔、樵词 (念法很多,此为其中的一种)

两个小花脸,一老一少扮相如渔、樵。

(俞伯牙念``将摇琴摆在坟前\ldots{}\ldots{}'',渔、樵内``啊哈''\textless{}\textbf{小锣五击}\textgreater{}上)

渔 (念)渔翁夜傍西岩宿,

樵 (念)更殚余力付樵苏。

渔 伙计,你说什么呐?

樵 我这念诗呐。

渔 你这长相还会念诗。

樵 就算我不会,那你干什么呐?

渔 我这可是念诗呐。

樵 许你念就不许我念。

渔 咱俩别吵,我是道听途说。

樵 我也是胡说八道。

渔 哎,你看这些人在这干什么呐?

樵 我看是上坟的。

渔 走累了,咱俩一边一个靠着树坐会儿。

樵 坐着坐着。

(俞伯牙弹完琴\ldots{}\ldots{})

渔 伙计,你听他们干什么呐?

樵 八成是弹棉花的。\\
渔 没事憩会儿好不好,跑这坟圈子里弹棉花干什么。

樵 吃饱了在这凉快凉快。

渔 别挨骂了。正是:(念)兰浦秋来烟雨深,

樵 (念)几多情思在琴心。

渔 又拽上了。 别听弹棉花的了。

樵 回家睡大觉去喽,哈\ldots{}\ldots{}

(渔、樵同下)

\textbf{注:}

\begin{enumerate}
\def\labelenumi{\arabic{enumi}.}
\item
  上渔、樵,俞、钟、童三人面向里。上渔、樵无非是不懂琴音,俞伯牙无知音,实际无此必要,故后来删掉。
\item
  《马鞍山》是乔玉林传下来的路子,传统、规矩,是学徒的唱法,中华戏曲专科学校的唱法略同与此,时慧宝的唱法与此有出入。此戏在过去是前三出的大路戏。
\end{enumerate}

