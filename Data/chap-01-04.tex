\setlength{\hangindent}{60pt}{\newpage}

\setlength{\hangindent}{60pt}

\setlength{\hangindent}{60pt}{%\subsection{马鞍山}\label{ux9a6cux978dux5c71}}}

\setlength{\hangindent}{60pt}{\subsubsection{{\hei\large 马鞍山}}%\protect\hyperlink{fn8}{\textsuperscript{8}}}

\setlength{\hangindent}{60pt}{\addcontentsline{toc}{subsection}{\hei 马鞍山}}

\setlength{\hangindent}{60pt}

\setlength{\hangindent}{60pt}{{\centerline{(李舒~遗作~~根据刘曾复先生手书原稿抄录)}}}

\setlength{\hangindent}{60pt}{\hangafter=1\hspace{20pt}~ %2. 设置从第1⾏之后开始悬挂缩进}

\setlength{\hangindent}{60pt}{\setlength{\parindent}{0pt}{

\setlength{\hangindent}{60pt}{{\centerline{{\bfseries\akai {[}\hei 第一场{]}}}}}

\setlength{\hangindent}{60pt}{\vspace{5pt}}

\setlength{\hangindent}{60pt}{(童儿、俞伯牙\textless{}\!{\bfseries\akai 小锣打上}\!\textgreater{})}

\setlength{\hangindent}{60pt}{\spacept{俞伯牙}{20pt} {[}{\akai 引子}{]}为访贤友,涉水登舟。}

\setlength{\hangindent}{60pt}{\spacept{俞伯牙}{20pt} {[}{\akai 诗}{]}青溪流过碧山头,空水澄鲜一色秋。隔断红尘三十里,白云红叶两悠悠。\footnote{该定场诗用的是北宋程颢的七绝《秋月》诗句,李舒先生钞录稿末句作``白云鸿雁两悠悠''。}}

\setlength{\hangindent}{60pt}{\spacept{俞伯牙}{20pt} 下官姓俞名瑞字伯牙,(乃)鲁国人氏,晋国为官。只因去岁往各国催贡,船行马鞍山前,偶遇钟子期,我二人共谈琴律,情意相投,结为金兰之好,临行(之时,)赠他黄金二笏,约定今岁中秋还在马鞍山前相会。来此不见贤弟到来,昨晚琴音缭乱\footnote{李元皓{\scriptsize 君}建议作``撩乱''。},不知是何缘故,我不免去往集贤村寻访于他便了。}

\setlength{\hangindent}{60pt}{\spacept{俞伯牙}{20pt} 童儿。}

\setlength{\hangindent}{60pt}{(童儿\hspace{30pt}~ 有。) }

\setlength{\hangindent}{60pt}{\spacept{俞伯牙}{20pt} 看衣改换。}

\setlength{\hangindent}{60pt}{(\textless{}\!{\bfseries\akai 小开门}\!\textgreater{},下,再上)}

\setlength{\hangindent}{60pt}{\spacept{俞伯牙}{20pt} 带了瑶琴,(随我往)集贤村去者!}

\setlength{\hangindent}{60pt}{(一翻两翻,半个\textless{}\!{\bfseries\akai 扯四门}\!\textgreater{},小童一直站大边)}

\setlength{\hangindent}{60pt}{\spacept{俞伯牙}{20pt} 【{\akai 二黄三眼}】我二人在山前金兰结好,今此来他不到所为哪条({\akai 或}:~今此来不见他所为哪条)。换冠裳我亲自义友寻找,此一去集贤村访见故交。}

\setlength{\hangindent}{60pt}{(\textless{}\!{\bfseries\akai 小锣打下}\!\textgreater{})}

\setlength{\hangindent}{60pt}{\vspace{3pt}{\centerline{\textrm{{[}{\hei 第二场}{]}}}}\vspace{5pt}}

\setlength{\hangindent}{60pt}{钟元普\footnote{钟元普亦作钟元甫,此处从李舒先生钞录稿。}} ({\akai 内白})走哇。}

\setlength{\hangindent}{60pt}{(提篮子,\textless{}\!{\bfseries\akai 小锣抽头}\!\textgreater{}{\hwfs 上})}

\setlength{\hangindent}{60pt}{\spacept{钟元普}{20pt} 唉!}

\setlength{\hangindent}{60pt}{\spacept{钟元普}{20pt} 【{\akai 二黄摇板}】屋漏偏遭连阴雨,破船又遇当头风。}

\setlength{\hangindent}{60pt}{\spacept{钟元普}{20pt} 老汉钟元普。吾儿({\akai 或}:~亡儿)名唤子期。只因去岁,中秋在马鞍山前砍柴,偶遇一位晋国大夫俞伯牙大人,他二人共谈琴律,情意相投,结为金兰之好。临别({\akai 或}:~临行)赠我儿黄金二笏,约定今岁中秋还在马鞍山前相会。谁想我儿回得家来,白日砍柴,夜晚攻书,朝暮积劳,染成疾病。他就此一命身亡了$\cdots${}$\cdots${}(钟元普{\hwfs 哭介})}

\setlength{\hangindent}{60pt}{\spacept{钟元普}{20pt} 咳,今当吾儿百日之期({\akai 或}:~今当亡儿百日之期),为此备了几陌纸钱,去往坟前烧化。天呐,天,({\akai 念})家有万贯终何用,老来无子一场空。}

\setlength{\hangindent}{60pt}{\spacept{钟元普}{20pt} 【{\akai 二黄原板}】老眼昏花路难行,又闻得({\akai 或}:~又听得)松林内百鸟喧声。乌鸦倒有反哺意,羊羔也有跪乳情。似乌云遮住了天边月,似狂风吹散了满天云。这才是黄梅已老青梅落,白发人反送了黑发儿的身。我的儿呀!}

\setlength{\hangindent}{60pt}{(\textless{}\!{\bfseries\akai 小锣抽头}\!\textgreater{}{\hwfs 下})}

\setlength{\hangindent}{60pt}{(俞伯牙{\hwfs 接}\textless{}\!{\bfseries\akai 小锣抽头}\!\textgreater{}{\hwfs 上})}

\setlength{\hangindent}{60pt}{\spacept{俞伯牙}{20pt} 【{\akai 二黄摇板}】昨夜晚抚瑶琴暗藏悲调,看起来这内中事有蹊跷。移步儿来至在双阳岔道,(\textless{}\!{\bfseries\akai 小锣抽头}\!\textgreater{}圆场)寻不着集贤村路走哪条?}

\setlength{\hangindent}{60pt}{\spacept{俞伯牙}{20pt} 哎呀且住,来此已是双阳岔道,但不知这集贤村往哪条道路而走。}

\setlength{\hangindent}{60pt}{\spacept{钟元普}{20pt} ({\akai 内嗽})嗯哼。}

\setlength{\hangindent}{60pt}{\spacept{俞伯牙}{20pt} 看那旁来一老丈。等他到来问明再走。}

\setlength{\hangindent}{60pt}{(钟元普{\hwfs 上})}

\setlength{\hangindent}{60pt}{\spacept{钟元普}{20pt} 【{\akai 二黄摇板}】曲弯弯行过了溪边小道,哪有个父与子把纸来烧({\akai 或}:~把纸化烧)。(过大边)}

\setlength{\hangindent}{60pt}{\spacept{俞伯牙}{20pt} 老丈请转。}

\setlength{\hangindent}{60pt}{\spacept{钟元普}{20pt} 呃,原来是一位先生,这位先生可是失迷路途?}

\setlength{\hangindent}{60pt}{\spacept{俞伯牙}{20pt} 正是。}

\setlength{\hangindent}{60pt}{\spacept{钟元普}{20pt} 但不知问的是何所在?}

\setlength{\hangindent}{60pt}{\spacept{俞伯牙}{20pt} 我问的是集贤村。}

\setlength{\hangindent}{60pt}{\spacept{钟元普}{20pt} 先生,你来看:这东去十里也是集贤村,西去十里也是集贤村。但不知是哪个集贤村呢?}

\setlength{\hangindent}{60pt}{\spacept{俞伯牙}{20pt} 这$\cdots${}$\cdots${}哎呀,贤弟呀,现有两个集贤村({\akai 或}:~既有两个集贤村),为何不对愚兄说明,如今叫我作难了。}

\setlength{\hangindent}{60pt}{\spacept{钟元普}{20pt} 啊先生,敢是指路不明?}

\setlength{\hangindent}{60pt}{\spacept{俞伯牙}{20pt} 呃呃,久住三五载,}

\setlength{\hangindent}{60pt}{\spacept{钟元普}{20pt} 无处不亲联。\footnote{李舒先生钞录稿作``无处不亲连。'',此处据樊百乐{\scriptsize 君}转述刘曾复先生确认文字。在表示通婚结成姻亲关系时,``联姻''通``连姻''。}

\setlength{\hangindent}{60pt}{\spacept{俞伯牙}{20pt} 正是。}

俞伯牙、\\
\spacept{钟元普}{20pt} \raisebox{5pt}{啊哈哈哈哈。}

\setlength{\hangindent}{60pt}{\spacept{钟元普}{20pt} 但不知问的是哪一家?}

\setlength{\hangindent}{60pt}{\spacept{俞伯牙}{20pt} 我问的是钟子期。}

\setlength{\hangindent}{60pt}{\spacept{钟元普}{20pt} 哦,钟子期。}

\setlength{\hangindent}{60pt}{\spacept{俞伯牙}{20pt} 正是。}

\setlength{\hangindent}{60pt}{\spacept{钟元普}{20pt} 咳,儿呀。}

\setlength{\hangindent}{60pt}{\spacept{钟元普}{20pt} 【{\akai 二黄摇板}】相逢未说几句话,不由老汉泪如麻。({\akai 哭介})}

\setlength{\hangindent}{60pt}{\spacept{钟元普}{20pt} 先生你来迟了。}

\setlength{\hangindent}{60pt}{\spacept{俞伯牙}{20pt} (老丈)何言来迟?}

\setlength{\hangindent}{60pt}{\spacept{钟元普}{20pt} 老汉钟元普。吾儿子期({\akai 或}:~亡儿子期),只因去岁中秋与俞大人结拜({\akai 或}:~与俞大人结为金兰之好),分别之后回到家中,他白日砍柴,夜晚攻书,积劳成疾({\akai 或}:~积劳成病),百日前(他)一命身亡了$\cdots${}$\cdots${}}

\setlength{\hangindent}{60pt}{\spacept{俞伯牙}{20pt} 你待怎讲?}

\setlength{\hangindent}{60pt}{\spacept{钟元普}{20pt} 一命身亡了。}

\setlength{\hangindent}{60pt}{\spacept{俞伯牙}{20pt} 哎呀!~(\textless{}\!{\bfseries\akai 崩登仓冲头}\!\textgreater{})}

\setlength{\hangindent}{60pt}{(俞伯牙{\hwfs 昏介})}

\setlength{\hangindent}{60pt}{\spacept{钟元普}{20pt} 这是何人?}

\setlength{\hangindent}{60pt}{童儿\hspace{30pt}~ 这就是俞大人。 }

\setlength{\hangindent}{60pt}{\spacept{钟元普}{20pt} 哦哦({\akai 或}:~哎呀),大人醒来。}

\setlength{\hangindent}{60pt}{\spacept{俞伯牙}{20pt} 【{\akai 二黄导板}】听说是钟贤弟一命丧了,}

\setlength{\hangindent}{60pt}{\spacept{俞伯牙}{20pt} \textless{}\!{\bfseries\akai 三叫头}\!\textgreater{}贤弟! 子期!哎贤弟呀。}

\setlength{\hangindent}{60pt}{\spacept{俞伯牙}{20pt} 【{\akai 二黄散板}】此一番好一似马行断桥。他的父是尊长急忙拜倒,}

\setlength{\hangindent}{60pt}{\spacept{钟元普}{20pt} 【{\akai 二黄散板}】请大人莫折煞年迈山樵。}

\setlength{\hangindent}{60pt}{(\spacept{俞伯牙}{20pt} 老伯。)}

\setlength{\hangindent}{60pt}{\spacept{俞伯牙}{20pt} 【{\akai 二黄散板}】我就是俞伯牙伯父知晓,贤弟死留何言细说根苗。}

\setlength{\hangindent}{60pt}{\spacept{钟元普}{20pt} 【{\akai 二黄散板}】我的儿临危时也曾言道:葬埋在马鞍山候驾来瞧。}

\setlength{\hangindent}{60pt}{\spacept{俞伯牙}{20pt} 【{\akai 二黄散板}】烦伯父你与我坟台引道,}

\setlength{\hangindent}{60pt}{(\textless{}\!{\bfseries\akai 扭丝}\!\textgreater{},钟元普、俞伯牙{\hwfs 同走圆场})}

\setlength{\hangindent}{60pt}{\spacept{钟元普}{20pt} 【{\akai 二黄散板}】这就是新坟土尚挂纸标。}

\setlength{\hangindent}{60pt}{(\spacept{俞伯牙}{20pt} 哎呀!)}

\setlength{\hangindent}{60pt}{\spacept{俞伯牙}{20pt} 【{\akai 二黄散板}】见坟台不由我双膝跪倒,呼不应、唤不醒生死故交。}

\setlength{\hangindent}{60pt}{\spacept{俞伯牙}{20pt} 贤弟呀$\cdots${}$\cdots${}({\hwfs 哭介})}

\setlength{\hangindent}{60pt}{\spacept{俞伯牙}{20pt} (啊,)老伯,那旁有一石台,老伯稍坐一时,待侄儿一祭。}

\setlength{\hangindent}{60pt}{(俞伯牙{\hwfs 哭介})}

\setlength{\hangindent}{60pt}{\spacept{钟元普}{20pt} 有劳大人。}

\setlength{\hangindent}{60pt}{\spacept{俞伯牙}{20pt} 童儿。}

\setlength{\hangindent}{60pt}{(童儿\hspace{30pt}~ 有。) }

\setlength{\hangindent}{60pt}{\spacept{俞伯牙}{20pt} 将我瑶琴摆在坟前。}

\setlength{\hangindent}{60pt}{({\hwfs 此处上渔}、{\hwfs 樵})}

\setlength{\hangindent}{60pt}{\spacept{俞伯牙}{20pt} 唉!~({\akai 念})此来空枉费,人琴付东流\footnote{李舒先生钞录稿作``人情付东流'',似非。}。灵魂渺茫去呀,可叹一土丘。}

\setlength{\hangindent}{60pt}{\spacept{俞伯牙}{20pt} \textless{}\!{\bfseries\akai 帽子头}\!\textgreater{}【{\akai 二黄慢板}】想去岁中秋节论琴交好,今日里见坟台不见故交。来时喜去时悲愁云渺渺,又只见秋风起黄叶飘飘。为贤弟我不爱黄金荣耀,为贤弟我不爱玉带紫袍。为贤弟二双亲少行孝道,为贤弟辞王驾亲走这遭。为贤弟终日里梦魂颠倒,为贤弟千里迢迢,涉水登山,枉费徒劳。实指望与贤弟同饮香醪,实指望与贤弟共论琴操。实指望与贤弟朝夕欢笑,实指望与贤弟春游芳草,夏赏荷香,秋饮菊酒,冬藏梅阁,散淡逍遥。在坟台抚瑶琴以为祭吊,}

\setlength{\hangindent}{60pt}{(俞伯牙{\hwfs 抚琴介})}

\setlength{\hangindent}{60pt}{(\spacept{俞伯牙}{20pt} 唉!)}

\setlength{\hangindent}{60pt}{\spacept{俞伯牙}{20pt} 【{\akai 二黄散板}】子期死少知音琴对谁调。我这里将瑶琴摔碎不要,}

\setlength{\hangindent}{60pt}{(俞伯牙{\hwfs 摔琴介}\textless{}\!{\bfseries\akai 乱锤}\!\textgreater{})}

\setlength{\hangindent}{60pt}{\spacept{钟元普}{20pt} 【{\akai 二黄散板}】问大人摔瑶琴所为哪条?}

\setlength{\hangindent}{60pt}{\spacept{俞伯牙}{20pt} 老伯,}

\setlength{\hangindent}{60pt}{\spacept{俞伯牙}{20pt} ({\akai 念})摔碎瑶琴凤尾寒,子期不在向谁弹?春风满面皆朋友,要会知音难上难。(俞伯牙{\akai 哭介})}

\setlength{\hangindent}{60pt}{\spacept{俞伯牙}{20pt} 【{\akai 二黄散板}】问伯父贤弟死家有何靠,}

\setlength{\hangindent}{60pt}{\spacept{钟元普}{20pt} 【{\akai 二黄散板}】隐居在集贤村倒也逍遥({\akai 或}:~倒还逍遥)。}

\setlength{\hangindent}{60pt}{\spacept{俞伯牙}{20pt} 【{\akai 二黄散板}】这黄金与伯父甘旨\footnote{甘旨,原意是美味的食品。引申为对双亲的奉养。}养老,且待我迎接你替他代劳。}

\setlength{\hangindent}{60pt}{\spacept{俞伯牙}{20pt} 老伯,侄儿去后伯父不要思他。}

\setlength{\hangindent}{60pt}{\spacept{钟元普}{20pt} 我不思他。}

\setlength{\hangindent}{60pt}{\spacept{俞伯牙}{20pt} 不要想他。}

\setlength{\hangindent}{60pt}{\spacept{钟元普}{20pt} 我也不想他。({\akai 或}:~呃,我不想他。)}

\setlength{\hangindent}{60pt}{\spacept{俞伯牙}{20pt} 子期是我。}

\setlength{\hangindent}{60pt}{\spacept{钟元普}{20pt} (呃,)不敢。}

\setlength{\hangindent}{60pt}{\spacept{俞伯牙}{20pt} 我是子期。}

\setlength{\hangindent}{60pt}{\spacept{钟元普}{20pt} 实实不敢({\akai 或}:~唉,越发地不敢呐)。}

\setlength{\hangindent}{60pt}{\spacept{俞伯牙}{20pt} 小侄告辞了。}

\setlength{\hangindent}{60pt}{\spacept{俞伯牙}{20pt} 【{\akai 二黄散板}】辞伯父别坟墓扬长就道,}

\setlength{\hangindent}{60pt}{\spacept{钟元普}{20pt} 【{\akai 二黄散板}】虽异姓似手足犹如同胞。}

\setlength{\hangindent}{60pt}{\spacept{俞伯牙}{20pt} 【{\akai 二黄散板}】伯牙在$\cdots${}$\cdots${}}

\setlength{\hangindent}{60pt}{\spacept{钟元普}{20pt} 【{\akai 二黄散板}】子期死(啊)$\cdots${}$\cdots${}}

\setlength{\hangindent}{60pt}{\spacept{俞伯牙}{20pt} ({\akai 接唱})【{\akai 二黄散板}】知音缺少,摔瑶琴谢知音不负故交。}

\setlength{\hangindent}{60pt}{\spacept{俞伯牙}{20pt} \textless{}\!{\bfseries\akai 三叫头}\!\textgreater{}老伯,子期,唉贤弟呀。}

\setlength{\hangindent}{60pt}{(\spacept{钟元普}{20pt} \textless{}\!{\bfseries\akai 三叫头}\!\textgreater{}大人,我儿,唉儿呀。)}

\setlength{\hangindent}{60pt}{\spacept{俞伯牙}{20pt} 罢!}

\setlength{\hangindent}{60pt}{(俞伯牙{\hwfs 下},小童{\hwfs 同下})}

\setlength{\hangindent}{60pt}{(\spacept{钟元普}{20pt} 唉!)}

\setlength{\hangindent}{60pt}{\spacept{钟元普}{20pt} 【{\akai 二黄散板}】似这等金兰友如同管鲍,转身来见坟台不见儿曹。猛抬头见红日西山落了,回家去与老妻细说根苗。}

\setlength{\hangindent}{60pt}{\spacept{钟元普}{20pt} 儿呀!}

\setlength{\hangindent}{60pt}{(\textless{}\!{\bfseries\akai 小锣打下}\!\textgreater{}\textless{}\!{\bfseries\akai 尾声}\!\textgreater{})}

\vspace{20pt}
\setlength{\hangindent}{60pt}{{\bfseries 本戏人物扮相}:~ }

\setlength{\hangindent}{60pt}{\spacept{俞伯牙}{20pt}~ 纱帽,蓝帔,黑三,高方巾,宝蓝褶子,绦子。}

\setlength{\hangindent}{60pt}{\spacept{钟元普}{20pt}~ 白氈帽,白满,白老斗衣,腰包,鞋。}

\setlength{\hangindent}{60pt}{童儿\hspace{30pt}~ 抓髻,白花褶子,鞋。 }

\vspace{25pt}
\setlength{\hangindent}{60pt}{{\hei 道具}:~ }

\setlength{\hangindent}{60pt}{小篮一只,内装纸钱。}

\setlength{\hangindent}{60pt}{石台,即倒椅两把。}

\vspace{25pt}
\setlength{\hangindent}{60pt}{{\bfseries\hei 附:~  渔、樵词}~ (念法很多,此为其中的一种) 

\setlength{\hangindent}{60pt}{两个小花脸,一老一少扮相如渔、樵。}

\setlength{\hangindent}{60pt}{(俞伯牙{\akai 念}``将摇琴摆在坟前$\cdots${}$\cdots${}'',渔、樵{\akai 内}``啊哈''\textless{}\!{\bfseries\akai 小锣五击}\!\textgreater{}{\hwfs 上})}

\setlength{\hangindent}{60pt}{渔\hspace{40pt}~ ({\akai 念})渔翁夜傍西岩宿,}

\setlength{\hangindent}{60pt}{樵\hspace{40pt}~ ({\akai 念})更殚余力付樵苏。}

\setlength{\hangindent}{60pt}{渔\hspace{40pt}~ 伙计,你说什么呐? }

\setlength{\hangindent}{60pt}{樵\hspace{40pt}~ 我这念诗呐。 }

\setlength{\hangindent}{60pt}{渔\hspace{40pt}~ 你这长相还会念诗。 }

\setlength{\hangindent}{60pt}{樵\hspace{40pt}~ 就算我不会,那你干什么呐? }

\setlength{\hangindent}{60pt}{渔\hspace{40pt}~ 我这可是念诗呐。 }

\setlength{\hangindent}{60pt}{樵\hspace{40pt}~ 许你念就不许我念。 }

\setlength{\hangindent}{60pt}{渔\hspace{40pt}~ 咱俩别吵,我是道听途说。 }

\setlength{\hangindent}{60pt}{樵\hspace{40pt}~ 我也是胡说八道。 }

\setlength{\hangindent}{60pt}{渔\hspace{40pt}~ 哎,你看这些人在这干什么呐? }

\setlength{\hangindent}{60pt}{樵\hspace{40pt}~ 我看是上坟的。 }

\setlength{\hangindent}{60pt}{渔\hspace{40pt}~ 走累了,咱俩一边一个靠着树坐会儿。 }

\setlength{\hangindent}{60pt}{樵\hspace{40pt}~ 坐着坐着。 }

\setlength{\hangindent}{60pt}{(俞伯牙{\hwfs 弹完琴}$\cdots${}$\cdots${})}

\setlength{\hangindent}{60pt}{渔\hspace{40pt}~ 伙计,你听他们干什么呐? }

\setlength{\hangindent}{60pt}{樵\hspace{40pt}~ 八成是弹棉花的。 }

\setlength{\hangindent}{60pt}{渔\hspace{40pt}~ 没事憩会儿好不好,跑这坟圈子里弹棉花干什么。 }

\setlength{\hangindent}{60pt}{樵\hspace{40pt}~ 吃饱了在这凉快凉快。 }

\setlength{\hangindent}{60pt}{渔\hspace{40pt}~ 别挨骂了。正是:~({\akai 念})兰浦秋来烟雨深,}

\setlength{\hangindent}{60pt}{樵\hspace{40pt}~ ({\akai 念})几多情思在琴心。}

\setlength{\hangindent}{60pt}{渔\hspace{40pt}~ 又拽上了。 别听弹棉花的了。}

\setlength{\hangindent}{60pt}{樵\hspace{40pt}~ 回家睡大觉去喽,哈$\cdots${}$\cdots${} }

\setlength{\hangindent}{60pt}{(渔、樵{\hwfs 同下})}

\vspace{25pt}
\bfseries\akai\hspace{10pt}~ 注:~}
\begin{enumerate}
	\item 上~渔、樵,俞、钟、童三人面向里。上渔、樵无非是不懂琴音,俞伯牙无知音,实际无此必要,故后来删掉。
	\item 《马鞍山》是乔玉林传下来的路子,传统、规矩,是学徒的唱法,中华戏曲专科学校的唱法略同与此,时慧宝的唱法与此有出入。此戏在过去是前三出的大路戏。
\end{enumerate}
}
