\newpage
\phantomsection %实现目录的正确跳转
\section*{\large\hei {硃砂痣~{\small 之}~韩廷凤}~({\hwl 谭派})}
\addcontentsline{toc}{section}{\hei 硃砂痣~{\small 之}~韩廷凤}

\hangafter=1                   %2. 设置从第1⾏之后开始悬挂缩进  %}
\setlength{\parindent}{0pt}{

{\vspace{3pt}{\centerline{{[}{\hei 第一场}{]}}}\vspace{5pt}}

{唉!}

\setlength{\hangindent}{56pt}{【{\akai 二黄摇板}】为续弦前后厅灯光明亮,梦不想今夜晚再做新郎。}

{抬上堂来({\akai 或}:~}搭上堂来)\footnote{以下括号中的念白据吴焕老师整理的剧本(经刘曾复先生审订)添加。}{。}

{每人赏银二分({\akai 或}:~}每人赏钱四个){。}

{带他们下面用饭。}

{待我观看一回。}

\setlength{\hangindent}{56pt}{【{\akai 二黄慢板}】我这里借灯光用目观望,我看她({\akai 或}:~只见她)与前妻一样风光。}

\setlength{\hangindent}{56pt}{【{\akai 二黄慢板}】为什么({\akai 或}:~因何故)皱眉间泪带面上({\akai 或}:~泪带脸上),莫不是嫌年迈难配鸾凰。}

\setlength{\hangindent}{56pt}{【{\akai 二黄慢板}】要穿衣锦绣衫任你选样,}

\setlength{\hangindent}{56pt}{【{\akai 二黄慢板}】要用饭现有那谷米陈仓。}

\setlength{\hangindent}{56pt}{【{\akai 二黄慢板}】这不是那不是难以猜想,尊娘行({\akai 或}:~问娘行)因何事珠泪汪汪,为的是哪桩,你何妨({\akai 或}:~又何妨)细说端详?}

\setlength{\hangindent}{56pt}{【{\akai 二黄摇板}】听她言这婚姻({\akai 或}:~好一似)冰结霜降,一时里惹动我烦恼愁肠。看起来断不可此事勾当,自情愿伴孤灯独守空房。}

{韩福过来。}

{命你同媒婆送这位大娘子回去,对他丈夫言讲({\akai 或}:~与他丈夫言讲):~前番(那)一百两银子不要,再送他一百两银子,教他好好将养疾病,将大娘子送回,成全他夫妻恩义。就此去罢。}

{啊大娘子,还有这婚书呢。}

{唉,就在灯前焚化了罢。}

\setlength{\hangindent}{56pt}{【{\akai 二黄摇板}】伊言道她丈夫病卧床上,没奈何卖妻子暂度时光。将自己比旁人俱是({\akai 或}:~皆是)一样,我岂肯拆散他恩爱鸳鸯。({\akai 或}:~善与恶自有那天理昭彰。)}

{\vspace{3pt}{\centerline{{[}{\hei 第二场}{]}}}\vspace{5pt}}

\setlength{\hangindent}{63pt}{【{\akai 二黄摇板}\footnote{据吴焕老师整理的剧本注:~汪派在``想当年''前面还有两句,词句是``劝世人一个个须要学好,皆因是自有那天理昭彰。''}{】想当年为太守何等荣耀,遇兵荒妻和子无有下梢。多亏了陈太尉将我来保,才能得归田园自在逍遥。}

\setlength{\hangindent}{56pt}{【{\akai 二黄摇板}】忙迫中呃挽定了把礼还到,一时里好教我难解根苗。}

{呜,你不是昨晚送回去的大娘子么?}

{既然将你送回,你又来则甚呐?}

{哦,(原来是)吴相公来了。}

{哎呀,请坐请坐。}

{呃,还有话讲啊。}

{(有话叙谈,}哪有不坐之理,请坐请坐。{)}

{啊大娘子,你也坐下。}

{啊吴相公,昨日听得大娘子说到家中一番苦楚,甚是凄惨。相公有恙在身,改日再来叙谈,为何带病前来?({\akai 或}:~大相公,昨日听得大娘子说到家中一番苦楚。相公你有病在床,病体好了,再来不迟。)}

{哦,你见了银子,出了一身的通汗,这病就好了么?}

{哎呀呀,吴相公啊,看将起来,这银子啊,是好物件!}

\setlength{\hangindent}{76pt}{【{\akai 二黄碰板垛板}】我救你的急,救你的难,救你的贫困;~全尔的节,全尔的义,全尔的婚{\footnotesize 呐}姻。纵有妻不生子前生造定,我岂肯拆婚姻\footnote{此处刘曾复先生说戏录音近似``错婚姻'',吴小如先生从夏山楼主学的是``拆婚姻'',此处从吴小如先生。李楠{\scriptsize 君}认为,此处系刘曾复先生将``拆''字按上口字处理。}落下了骂名。}

{啊,啊,呵呵呵哈哈哈$\cdots{}\cdots{}$({\hwfs 笑介})}

{请坐请坐。}

{哎呀呀吴相公啊,我当日也是恩爱夫妻,只因兵荒马乱,中途失散,却无子嗣。若有一子传宗接代,我也就不续弦再娶的了哇。}

{这$\cdots{}\cdots{}$子孙前世所修,再续么,也就不必了。({\akai 或}:~}子孙之事,前生所修。再娶么,唉,也就不必了。{)}

{言得极是,只是本处孩儿多有不便呐。}

{本当如此,奈无机会。}

{这只好凭天机、遇时宜了。}

{吴相公慢些走啊!}

\setlength{\hangindent}{56pt}{【{\akai 二黄摇板}】他夫妻进门来双双拜倒({\akai 或}:~双双跪倒),口声声叫恩人泪似呃雨抛。非是我用银钱假意行好哇,韩廷凤全仁义一片心苗。}

{\vspace{3pt}{\centerline{{[}{\hei 第三场}{]}}}\vspace{5pt}}

\setlength{\hangindent}{46pt}{【{\akai 四平调}】叹光阴去不归无限烦闷,不觉已老两鬓如银。读古书难解我心头烦闷,饮香醪怎畅我衷肠凄清。}

{哦,你是吴相公,呃,请坐请坐。}

{你往成都收取账目,呃,必定是发了财了哇。({\akai 或}:~}闻你往成都,收取账目,一定发财的了。{)}

{请便。}

{哦,这是何人?}

{哦,罢了(罢了),一旁坐下。}

{呃,不妨不妨,只管地坐下。}

{呵呵呵哈哈哈$\cdots{}\cdots{}$({\hwfs 笑介})}

{啊吴相公,你看这小小的孩童,呃,也(很)知大体呀。}

{是啊,小孩子原要(他)爹娘教导哇。}

{啊吴相公,你买他前来,还是为子啊,还是为仆呢?}

{哦,这等说来({\akai 或}:~}如此说来){,是送与我的?}

{哎呀呀吴相公啊,你真是({\akai 或}:~}你真乃){信实人也。}

\setlength{\hangindent}{56pt}{【{\akai 四平调}】吴官人你真真言而有信,你与我谋后代不惜辛勤。感谢你这好意情深义尽,\textless{}\!{\bfseries\akai 行弦}\!\textgreater{}}

{吴大哥,你请来上坐。({\akai 或}:~这边坐,这边坐。)}

{请坐,请坐。}

\setlength{\hangindent}{56pt}{【{\akai 四平调}】退一日自当另有条陈。}

{呵呵哈哈哈$\cdots{}\cdots{}$({\hwfs 笑介})}

{哎呀吴官人呐,我如今有了儿子就不愁了。}

{是啊,吴官人离家日久,我也不便相留({\akai 或}:~不便强留),改日我父子要登门叩谢。}

{儿啊,送过你吴大爷。}

{(啊,吴官人何事?)}

{请来吃酒,请来吃酒。}

{啊$\cdots{}\cdots{}$呃,吴官人,请转请转。}

{改日我父子,呃,要登门再谢。}

{呃一定要去,一定要去。}

{啊$\cdots{}\cdots{}$呃,无有了,请便请便。({\akai 或}:~吴官人,请呐请呐。)}

{哦------哈哈哈$\cdots{}\cdots{}$({\hwfs 笑介})}

{儿啊,随为父的进来。}

{一旁坐下,待我细看一回({\akai 或}:~待我}细观一回{)。}

\setlength{\hangindent}{56pt}{【{\akai 二黄原板}】我的儿须从容端然坐定,看形象并非是平等之人。细观他各部位五官端正,这两鬓齐开朗目秀眉清。儿在家可读过圣贤书本,一一地对为父细说分明。}

\setlength{\hangindent}{56pt}{【{\akai 二黄原板}】他说话有分寸智慧聪明,倒像个宦门后不差毫分。可记得是何年月日生辰,说出来将八字细与儿评。}

\setlength{\hangindent}{56pt}{【{\akai 二黄原板}】这小娃言语中隐藏暗景,再问他亲父母便知真情。儿父母年多少在不在,因何故图银钱卖与他人。}

{儿今年多大年纪了({\akai 或}:~}儿今年几岁了{) ?}

{儿父?}

{死了五载。}

{儿母?}

{七旬有余。}

{(惊{\hwfs 介})儿有一十三岁({\akai 或}:~}儿才一十三岁{)。}

{(惊{\hwfs 介})嗯$\cdots{}\cdots{}$}

\setlength{\hangindent}{56pt}{【{\akai 二黄原板}】这其间又盘出}\footnote{吴焕老师整理的剧本记作``又攀出''。}{奇情种种,哪有个花甲年又产娇生。}

{儿啊------}

\setlength{\hangindent}{56pt}{【{\akai 二黄原板}】必然是那老娘将儿蒙混,这内中另有个生儿的娘亲。}

{哦,儿是捡来的么?}

\setlength{\hangindent}{56pt}{【{\akai 二黄原板}】细盘问这来由日月推论,仔细想当年事越加是真:~宣和年四月里成都调任,行至在青州府路遇贼兵。亲生儿在娘怀无有踪影,实可怜贤德妻命赴幽冥。}

\setlength{\hangindent}{56pt}{【{\akai 二黄散板}】这形象好一似韩门真种,举动间与老夫骨肉有情。我这里取菱花照照相品呐,}

\setlength{\hangindent}{56pt}{【{\akai 二黄散板}】半像我半像妻不差毫分。}

\setlength{\hangindent}{56pt}{【{\akai 二黄散板}】亲生子再相逢三生有幸,这才是天地意弄假成真。}

{不对了,不对了!}

{(韩玉印\hspace{20pt}怎么不对呢?)}

{我那亲生的孩儿,落生下来,左足心上有硃砂红痣。你无有,不是我的亲生儿子啊。}

{怎么,儿也有?}

{为父的不信呐,待我看来。}

{(唉,儿啊------)}

\setlength{\hangindent}{56pt}{【{\akai 二黄散板}】你是我亲生的儿呀名唤玉印,遇兵荒遭失散十有二春。盼娇儿盼得我身染重病,盼娇儿盼得我昼夜不宁。盼娇儿啊不做官告归故呃井,喂呀我的儿呀,梦不想天保佑枯木哇逢春。}

{你母命丧东平,也曾命人搬尸去了$\cdots{}\cdots{}$哇,呃$\cdots{}\cdots{}$({\hwfs 哭介})}

{言得极是,派人接她前来就是。}

{正是:~({\akai 念})北转南来西复东,今朝骨肉又重逢。父子再把菱花照,}

{(儿啊------)}

{({\akai 念})只怕相逢在梦中。}

{哦,不是做梦?}

{玉印,我儿,啊------啊------哈哈哈$\cdots{}\cdots{}$({\hwfs 笑介})}

{随我来。}
