\newpage\hspace{30pt}~

{%

\subsubsection{\large\hei {忠孝全~{\small 之}~

秦洪}}

{\vspace{3pt}{\centerline{{[}{\hei 第一场}{]}}}\vspace{5pt}}

{(王振\hspace{40pt}~

下跪可是福建的解粮官么?)}

{正是。}\hspace{40pt}~

{(王振\hspace{40pt}~

$\cdots{}\cdots{}$,讲!)}

{一路之上风雨阻隔,故而来迟。}

{(王振\hspace{40pt}~

$\cdots{}\cdots{}$看粮、豆如何?)}

{(众\hspace{40pt}~

$\cdots{}\cdots{}$干豆。)}

{(王振\hspace{40pt}~

$\cdots{}\cdots{}$干豆,$\cdots{}\cdots{}$却是为何?)}

{乃是卑府一片孝心!}

{(王振\hspace{40pt}~

二府?)}

{升去。}\hspace{40pt}~

{(王振\hspace{40pt}~

三府?)}

{告老。}\hspace{40pt}~

{(王振\hspace{40pt}~

$\cdots{}\cdots{}$为何不抬起头来?)}

{有罪不敢抬头。}\hspace{20pt}~

{(王振\hspace{40pt}~

赦你无罪。)}

{谢千岁!}\hspace{30pt}~

{(王振\hspace{40pt}~

唗!)}

{罢了哇罢了!}\hspace{30pt}~

{\vspace{3pt}{\centerline{{[}{\hei 第二场}{]}}}\vspace{5pt}}

\setlength{\hangindent}{56pt}{【{\akai 西皮导板}】犯官受刑身无主,}

\setlength{\hangindent}{56pt}{【{\akai 西皮原板}】二目圆睁怕煞人。中军帐好一比森罗殿,王公公好一比五殿阎君。刽子手好一比催命鬼,我好比屈死一鬼魂。再不能宛平为正印,再不能福建管黎民。再不能一家团圆庆,再不能满门享太平。忍泪含悲法场进,}

\setlength{\hangindent}{56pt}{【{\akai 西皮摇板}】咬定牙关等时辰。}

{(秦继龙\hspace{40pt}~

$\cdots{}\cdots{}$家住哪里?

$\cdots{}\cdots{}$监斩一毕,也好收你的尸首。)}

{监斩爷容禀!}\hspace{30pt}~

\setlength{\hangindent}{56pt}{【{\akai 西皮原板}】未曾开言泪先淋,}

{(秦继龙\hspace{40pt}~

不要啼哭,慢慢讲来。)}

\setlength{\hangindent}{56pt}{【{\akai 西皮原板}】尊一声监斩爷细听分明:~}

{(秦继龙\hspace{40pt}~

家住哪里?)}

\setlength{\hangindent}{56pt}{【{\akai 西皮原板}】家住山东莱州郡,}

{(秦继龙\hspace{40pt}~

哪里家门?)}

\setlength{\hangindent}{56pt}{【{\akai 西皮原板}】即墨县内有门庭。}

{(秦继龙\hspace{40pt}~

可曾得中?)}

\setlength{\hangindent}{56pt}{【{\akai 西皮原板}】甲子年间得中举,}

{(秦继龙\hspace{40pt}~

可曾上进?)}

\setlength{\hangindent}{56pt}{【{\akai 西皮原板}】连科得会进士名。}

{(秦继龙\hspace{40pt}~

初任哪里为官?)}

\setlength{\hangindent}{56pt}{【{\akai 西皮原板}】初任宛平为正印,}

{(秦继龙\hspace{40pt}~

可曾升任?)}

\setlength{\hangindent}{56pt}{【{\akai 西皮原板}】钦命福建管黎民。}

{(秦继龙\hspace{40pt}~

身犯何罪?)}

\setlength{\hangindent}{56pt}{【{\akai 西皮原板}】都只为金鳌叛边境,误粮不到问典刑。}

{(秦继龙\hspace{40pt}~

膝下有几个儿子?)}

\setlength{\hangindent}{56pt}{【{\akai 西皮原板}】犯官膝下有三子,}

{(秦继龙\hspace{40pt}~

亲生还是螟蛉?)}

\setlength{\hangindent}{56pt}{【{\akai 西皮原板}】二子亲生一子螟蛉。}

{(秦继龙\hspace{40pt}~

亲生之子叫何名字?)}

\setlength{\hangindent}{56pt}{【{\akai 西皮原板}】长子继美、次继远,}\footnote{继美、继远的名字从《传统京剧汇编》第八集~范叔年~藏本。}

{(秦继龙\hspace{40pt}~

螟蛉何名?)}

\setlength{\hangindent}{56pt}{【{\akai 西皮原板}】螟蛉叫作秦继龙。}

{(秦继龙\hspace{40pt}~

秦继龙哪里去了?)}

\setlength{\hangindent}{56pt}{【{\akai 西皮原板}】都只为小娇儿不听教训,一家四口赶出了门庭呐。}

{(秦继龙\hspace{40pt}~

你如今可思想于他?)}

\setlength{\hangindent}{56pt}{【{\akai 西皮原板}】眼前若有继龙啊\textless{}\!{\bfseries\akai 哭头}\!\textgreater{}子,}

\setlength{\hangindent}{56pt}{【{\akai 西皮摇板}】纵死黄泉也甘心。}

{(秦继龙\hspace{40pt}~

你叫何名字?)}

\setlength{\hangindent}{56pt}{【{\akai 西皮摇板}】监斩爷问我的名和姓,红旗上现有我犯官姓名。}

{(秦继龙\hspace{40pt}~

\setlength{\hangindent}{56pt}{【{\akai 西皮摇板}】$\cdots{}\cdots{}$我担承。)} }

\setlength{\hangindent}{56pt}{【{\akai 西皮摇板}】法场上绑得我昏迷不醒,}

{(秦继龙\hspace{40pt}~

起来。)}

{唉呀!}\hspace{40pt}~

\setlength{\hangindent}{56pt}{【{\akai 西皮摇板}】问一声监斩爷你是何人。}

{(秦继龙\hspace{40pt}~

爹爹。)}

{(秦继龙\hspace{40pt}~

\setlength{\hangindent}{56pt}{【{\akai 西皮摇板}】$\cdots{}\cdots{}$把身存。)} }

{儿是继龙?}\hspace{30pt}~

{\textless{}\!{\bfseries\akai 哭头}\!\textgreater{}啊,我的儿啊!}

\setlength{\hangindent}{56pt}{【{\akai 西皮散板}】不正不正父不正,不该将儿赶出门。只说儿天涯逃性呐\textless{}\!{\bfseries\akai 哭头}\!\textgreater{}命,我的儿啊,}

\setlength{\hangindent}{56pt}{【{\akai 西皮散板}】法场收父死尸灵。}

{(秦继龙\hspace{40pt}~

\setlength{\hangindent}{56pt}{【{\akai 西皮散板}】$\cdots{}\cdots{}$老天伦。)} }

{儿啊,此番将父松绑,可有圣上旨意?}

{(秦继龙\hspace{40pt}~

无有。)}

{元帅将令}\hspace{30pt}~

{(秦继龙\hspace{40pt}~

也无有。)}

{唉呀,不好了!}\hspace{20pt}~

\setlength{\hangindent}{56pt}{【{\akai 西皮散板}】私解法绳犯军令,知法犯法罪非轻。为父年迈当尽呐命,我的儿啊,怎肯连累小娇生。}

{(秦继龙\hspace{40pt}~

爹爹!)}

{(秦继龙\hspace{40pt}~

\setlength{\hangindent}{56pt}{【{\akai 西皮散板}】$\cdots{}\cdots{}$走一程。)} }

{(王振\hspace{40pt}~

你们都起来吧!)}

{谢千岁!}\hspace{30pt}~

{(王振\hspace{40pt}~

$\cdots{}\cdots{}$下去。)}

{儿啊,离家日久,哪里来的这身荣耀?}

{(秦继龙\hspace{40pt}~

$\cdots{}\cdots{}$)}

{(王振\hspace{40pt}~

圣旨下,跪!)}

{万岁!}\hspace{40pt}~

{(王振\hspace{40pt}~

$\cdots{}\cdots{}$养老太师!)}

{谢主隆恩!}\hspace{30pt}~

{(王振\hspace{40pt}~

$\cdots{}\cdots{}$谢恩呐!)}

{万万岁。}\hspace{30pt}~

{(王振\hspace{40pt}~

恭喜老太师。)}

{谢千岁提拔之恩。}\hspace{20pt}~

{(王振\hspace{40pt}~

$\cdots{}\cdots{}$谁来搭救咱?)}

{呃,千岁若不嫌弃,将小儿拜在千岁名下,以为螟蛉义子,不知千岁意下如何?}

{(王振\hspace{40pt}~

那可使不得呀!)}

{使得的,儿啊,快快拜见你的义父。}

{(王振\hspace{40pt}~

$\cdots{}\cdots{}$忒轻呃!)}

{千岁!啊------哈哈哈哈$\cdots{}\cdots{}$({\hwfs 笑}{\hwfs 介})}

{(王振\hspace{40pt}~

$\cdots{}\cdots{}$,老太师哪里为官?)}

{原任为官。}\hspace{30pt}~

{(王振\hspace{40pt}~

几时启程?)}

{即刻启程。}\hspace{30pt}~

{(王振\hspace{40pt}~

恕不远送。)}

{告辞了。}\hspace{30pt}~

{(王振\hspace{40pt}~

请------)}

{免。}\hspace{40pt}~
