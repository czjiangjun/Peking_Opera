\setlength{\hangindent}{56pt}{
\newpage\hspace{30pt}~
{%
\subsubsection{\large\hei {一捧雪}}
}

\setlength{\hangindent}{56pt}{
\vspace{3pt}{\centerline{{[}{\hei 第一场}{]}}}\vspace{5pt}\hspace{20pt}~
}

\setlength{\hangindent}{56pt}{
(龙套、校尉、严世藩上,汤勤迎,归坐)
}

\setlength{\hangindent}{56pt}{
严世藩\hspace{40pt}~
可恼哇可恼!
}

\setlength{\hangindent}{56pt}{
汤勤\hspace{40pt}~
大人下朝为何这等烦恼?
}

\setlength{\hangindent}{56pt}{
严世藩\hspace{40pt}~
适才在金殿与夏言老儿顶本,被他一本参倒,你道恼是不恼?
}

\setlength{\hangindent}{56pt}{
汤勤\hspace{40pt}~
这有何难,明日早朝老太师上殿奏本,稍带他几句,也就够他受用的了。
}

\setlength{\hangindent}{56pt}{
严世藩\hspace{40pt}~
言之有理,这几日不在府中,你往哪里去了?
}

\setlength{\hangindent}{56pt}{
汤勤,小人过衙谢官。\hspace{20pt}~
}

\setlength{\hangindent}{56pt}{
严世藩\hspace{40pt}~
看起来你倒是个有良心的。
}

\setlength{\hangindent}{56pt}{
汤勤\hspace{40pt}~
小人本来是有良心的,眼前有一人他无有良心呐。
}

\setlength{\hangindent}{56pt}{
严世藩\hspace{40pt}~
哪个无有良心?
}

\setlength{\hangindent}{56pt}{
汤勤\hspace{40pt}~
就是那莫大老爷。
}

\setlength{\hangindent}{56pt}{
严世藩\hspace{40pt}~
我那莫仁兄怎见得无有良心?
}

\setlength{\hangindent}{56pt}{
汤勤\hspace{40pt}~
那日莫大老爷献与大人的玉杯,是真是假?
}

\setlength{\hangindent}{56pt}{
严世藩\hspace{40pt}~
自然是真杯,焉有假杯之理。
}

\setlength{\hangindent}{56pt}{
汤勤\hspace{40pt}~
乃是假的。
}

\setlength{\hangindent}{56pt}{
严世藩\hspace{40pt}~
怎见得?
}

\setlength{\hangindent}{56pt}{
汤勤\hspace{40pt}~
那日酒席筵前,他将真杯显露出来,被小官看见了。
}

\setlength{\hangindent}{56pt}{
严世藩\hspace{40pt}~
依你之见。
}

\setlength{\hangindent}{56pt}{
汤勤\hspace{40pt}~
大人过府搜杯。
}

\setlength{\hangindent}{56pt}{
严世藩\hspace{40pt}~
搜杯得出。
}

\setlength{\hangindent}{56pt}{
汤勤\hspace{40pt}~
莫大老爷之罪。
}

\setlength{\hangindent}{56pt}{
严世藩\hspace{40pt}~
搜杯不出。
}

\setlength{\hangindent}{56pt}{
汤勤\hspace{40pt}~
小人之罪。
}

\setlength{\hangindent}{56pt}{
严世藩\hspace{40pt}~
起过了。
}

\setlength{\hangindent}{56pt}{
(汤勤下,严世藩立)\hspace{20pt}~
}

\setlength{\hangindent}{56pt}{
严世藩\hspace{40pt}~
【{\akai 二黄散板}】汤勤说话如刀切,舌尖杀人哪见血。人来顺轿把队列,
}

\setlength{\hangindent}{56pt}{
(严上轿,众领下)\hspace{20pt}~
}

\setlength{\hangindent}{56pt}{
严世藩\hspace{40pt}~
({\akai 接唱})过府去搜一捧雪。
}

\setlength{\hangindent}{56pt}{
\vspace{3pt}{\centerline{{[}{\hei 第二场}{]}}}\vspace{5pt}\hspace{20pt}~
}

\setlength{\hangindent}{56pt}{
(莫怀古、雪艳上)\hspace{20pt}~
}

\setlength{\hangindent}{56pt}{
莫怀古\hspace{40pt}~
【{\akai 二黄散板}】昨夜一梦大不祥,
}

\setlength{\hangindent}{56pt}{
雪艳\hspace{40pt}~
({\akai 接唱})老爷言来妾身详。
}

\setlength{\hangindent}{56pt}{
(八字小座)\hspace{30pt}~
}

\setlength{\hangindent}{56pt}{
莫成\hspace{40pt}~
({\akai 内}白)走哇,
}

\setlength{\hangindent}{56pt}{
(莫成\textless{}\!{\bfseries\akai 水底鱼}\!\textgreater{}上,进门,立小边禀报)
}

\setlength{\hangindent}{56pt}{
莫成\hspace{40pt}~
禀老爷:~严爷到。
}

\setlength{\hangindent}{56pt}{
莫怀古\hspace{40pt}~
知道了,外厢伺候。
}

\setlength{\hangindent}{56pt}{
(莫怀古、雪艳下,莫成出门立大边外场,严众上,校尉小边一字)
}

\setlength{\hangindent}{56pt}{
严世藩\hspace{40pt}~
【{\akai 二黄散板}】来在莫府下了轿,
}

\setlength{\hangindent}{56pt}{
(严世藩下轿)\hspace{30pt}~
}

\setlength{\hangindent}{56pt}{
严世藩\hspace{40pt}~
({\akai 接唱})会会当年故旧交。
}

\setlength{\hangindent}{56pt}{
莫成\hspace{40pt}~
禀老爷,严爷到。({\akai 或}:~有请老爷,严爷下轿。)
}

\setlength{\hangindent}{56pt}{
(莫成下场门上)\hspace{30pt}~
}

\setlength{\hangindent}{56pt}{
莫怀古\hspace{40pt}~
【{\akai 二黄散板}】听说严爷到府门,整整衣冠礼相迎。
}

\setlength{\hangindent}{56pt}{
(莫怀古出门迎,严世藩众挖门进站门,严立大边,莫挖进,莫成跟进,站小边)
}

\setlength{\hangindent}{56pt}{
莫怀古\hspace{40pt}~
({\akai 接唱})莫不是升官少谢忱?
}

\setlength{\hangindent}{56pt}{
严世藩\hspace{40pt}~
你做\footnote{《京剧新序》中``做''字误为``坐''。}的是嘉靖皇上的官,谢我何来?
}

\setlength{\hangindent}{56pt}{
莫怀古\hspace{40pt}~
({\akai 接唱})大人发怒为何因?
}

\setlength{\hangindent}{56pt}{
莫怀古\hspace{40pt}~
大人怒气不息,为着谁来?
}

\setlength{\hangindent}{56pt}{
严世藩\hspace{40pt}~
就为你来。
}

\setlength{\hangindent}{56pt}{
莫怀古\hspace{40pt}~
为下官何来?
}

\setlength{\hangindent}{56pt}{
严世藩\hspace{40pt}~
一捧雪进与不进但凭于你,为何假杯哄我?
}

\setlength{\hangindent}{56pt}{
莫怀古\hspace{40pt}~
杯子就有一只进与大人并无第二。
}

\setlength{\hangindent}{56pt}{
莫成\hspace{40pt}~
着哇。
}

\setlength{\hangindent}{56pt}{
严世藩\hspace{40pt}~
住了。
}

\setlength{\hangindent}{56pt}{
(莫成上场门暗下)\hspace{20pt}~
}

\setlength{\hangindent}{56pt}{
严世藩\hspace{40pt}~
【{\akai 二黄散板}】听一言来怒气生,骂声怀古太欺心。进京未满一月整,保你太常寺正卿。人来与我忙搜定,
}

\setlength{\hangindent}{56pt}{
(校尉两抄,莫成上场门上到小边台口又下)
}

\setlength{\hangindent}{56pt}{
严世藩\hspace{40pt}~
({\akai 接唱})掘土三尺再搜寻。
}

\setlength{\hangindent}{56pt}{
(又两抄,莫成下场门上到大边台口又下)
}

\setlength{\hangindent}{56pt}{
严校尉\hspace{40pt}~
玉杯无有。
}

\setlength{\hangindent}{56pt}{
严世藩\hspace{40pt}~
起过了。
}

\setlength{\hangindent}{56pt}{
严世藩\hspace{40pt}~
({\akai 接唱})搜杯不出面带红,失却当年故旧情。
}

\setlength{\hangindent}{56pt}{
(雪艳上场门暗上,挡脸)\hspace{10pt}~
}

\setlength{\hangindent}{56pt}{
严世藩\hspace{40pt}~
仁兄身后何人?
}

\setlength{\hangindent}{56pt}{
莫怀古\hspace{40pt}~
贱妾雪艳。
}

\setlength{\hangindent}{56pt}{
严世藩\hspace{40pt}~
请来相见。
}

\setlength{\hangindent}{56pt}{
莫怀古\hspace{40pt}~
夫人见过严爷。
}

\setlength{\hangindent}{56pt}{
(雪艳见礼)\hspace{30pt}~
}

\setlength{\hangindent}{56pt}{
雪艳\hspace{40pt}~
参见大人。
}

\setlength{\hangindent}{56pt}{
严世藩\hspace{40pt}~
仁嫂。
}

\setlength{\hangindent}{56pt}{
莫怀古\hspace{40pt}~
回避。
}

\setlength{\hangindent}{56pt}{
(雪艳上场门下)\hspace{30pt}~
}

\setlength{\hangindent}{56pt}{
严世藩\hspace{40pt}~
仁兄,真杯也罢,假杯也罢,拿将出来小弟一观,不要你的就是。
}

\setlength{\hangindent}{56pt}{
莫怀古\hspace{40pt}~
方才也曾言过,杯子一只进与大人并无第二。
}

\setlength{\hangindent}{56pt}{
严世藩\hspace{40pt}~
有人得见。
}

\setlength{\hangindent}{56pt}{
莫怀古\hspace{40pt}~
何人得见?
}

\setlength{\hangindent}{56pt}{
严世藩\hspace{40pt}~
汤勤得见。
}

\setlength{\hangindent}{56pt}{
莫怀古\hspace{40pt}~
汤勤,是了,那日汤勤过府谢官,在酒席宴前得罪于他也是有的,他在大人台前搬动是非,有道是傍耳之言不可深信。
}

\setlength{\hangindent}{56pt}{
严世藩\hspace{40pt}~
你待怎讲?
}

\setlength{\hangindent}{56pt}{
莫怀古\hspace{40pt}~
不可深信。
}

\setlength{\hangindent}{56pt}{
严世藩\hspace{40pt}~
住口!
}

\setlength{\hangindent}{56pt}{
严世藩\hspace{40pt}~
【{\akai 二黄散板}】听罢言来怒气生,我有一言听分明。朝里朝外问一问,严家岂是省油灯。人来与爷把轿顺,
}

\setlength{\hangindent}{56pt}{
(严世藩众出门、严上轿,莫怀古随出门送)
}

\setlength{\hangindent}{56pt}{
严世藩\hspace{40pt}~
({\akai 接唱})三日定要灭满门。
}

\setlength{\hangindent}{56pt}{
(严世藩下,莫怀古进门归大边立,雪艳上立小边)
}

\setlength{\hangindent}{56pt}{
莫怀古\hspace{40pt}~
夫人,莫成这个奴才往哪里去了?
}

\setlength{\hangindent}{56pt}{
雪艳\hspace{40pt}~
妾身不知。
}

\setlength{\hangindent}{56pt}{
莫怀古\hspace{40pt}~
两厢唤来。
}

\setlength{\hangindent}{56pt}{
(莫怀古、雪艳抄过、莫小边、雪大边,向里叫)
}

\setlength{\hangindent}{56pt}{
莫怀古\hspace{40pt}~
莫成。
}

\setlength{\hangindent}{56pt}{
雪艳\hspace{40pt}~
掌家。
}

\setlength{\hangindent}{56pt}{
(莫怀古、雪艳回身,莫正场、雪大边立)
}

\setlength{\hangindent}{56pt}{
莫成\hspace{40pt}~
({\akai 内}白)走哇!
}

\setlength{\hangindent}{56pt}{
(莫成\textless{}\!{\bfseries\akai 水底鱼}\!\textgreater{}上,进门,立小边侧)
}

\setlength{\hangindent}{56pt}{
莫成\hspace{40pt}~
老爷受惊了。
}

\setlength{\hangindent}{56pt}{
莫怀古\hspace{40pt}~
我受的什么惊?
}

\setlength{\hangindent}{56pt}{
莫成\hspace{40pt}~
哎呀老爷呀,可记得前日吃酒之事么?
}

\setlength{\hangindent}{56pt}{
莫怀古\hspace{40pt}~
动不动就是你老爷的酒,你敢戒你老爷的酒,待我打死你这个奴才。
}

\setlength{\hangindent}{56pt}{
(莫怀古打、莫成遮、雪艳拦)\hspace{10pt}~
}

\setlength{\hangindent}{56pt}{
莫成\hspace{40pt}~
老爷,纵然打死小人,可容小人讲个明白。
}

\setlength{\hangindent}{56pt}{
莫怀古\hspace{40pt}~
快快讲来。
}

\setlength{\hangindent}{56pt}{
莫成\hspace{40pt}~
容禀:~小人方才见严爷下轿之时,气色有些不正,想必为的是一捧雪而来。是小人去至上房,扭开箱锁,揣了一捧雪,打从前门而走,
}

\setlength{\hangindent}{56pt}{
(双手外指)\hspace{30pt}~
}

\setlength{\hangindent}{56pt}{
莫成\hspace{40pt}~
不想前门有兵丁把守(双手阻{\hwfs 介}),小人也只得打从后门而逃(双手内指),又有严府校尉阻拦(双手阻{\hwfs 介})。小人无奈,打从犬洞逃出。在鬻食棚躲避,见严爷上轿已走,才得回来。进得府门,老爷不问青红皂白,开口就骂,举手就打,看起来,像我这为奴的呀,唉,好难办的事呀!({\hwfs 哭}{\hwfs 介})
}

\setlength{\hangindent}{56pt}{
莫怀古\hspace{40pt}~
有了一捧雪还则罢了,若无一捧雪,夫人不必阻拦,待我打死这个奴才。
}

\setlength{\hangindent}{56pt}{
(莫怀古打,莫成遮,雪艳拦)\hspace{10pt}~
}

\setlength{\hangindent}{56pt}{
雪艳\hspace{40pt}~
待妾身向前,掌家!
}

\setlength{\hangindent}{56pt}{
莫成\hspace{40pt}~
夫人。
}

\setlength{\hangindent}{56pt}{
雪艳\hspace{40pt}~
你可知你家老爷发怒为了何事?
}

\setlength{\hangindent}{56pt}{
莫成\hspace{40pt}~
小人不知。
}

\setlength{\hangindent}{56pt}{
雪艳\hspace{40pt}~
就是为那一捧雪。
}

\setlength{\hangindent}{56pt}{
莫成\hspace{40pt}~
啊,一捧雪,在、在$\cdots{}\cdots{}$这里。
}

\setlength{\hangindent}{56pt}{
(摸左袖,摸右袖,摸胸掏出杯,双手举杯呈献,雪艳指杯,莫怀古看)
}

\setlength{\hangindent}{56pt}{
莫怀古\hspace{40pt}~
哦。
}

\setlength{\hangindent}{56pt}{
莫怀古\hspace{40pt}~
【{\akai 二黄散板}】一见玉杯果是真,好个伶俐小莫成。走向前来与掌家论,
}

\setlength{\hangindent}{56pt}{
莫成\hspace{40pt}~
老爷,小人挨不起了哇。({\hwfs 哭}{\hwfs 介})
}

\setlength{\hangindent}{56pt}{
莫怀古\hspace{40pt}~
({\akai 接唱})错打几下莫挂心。
}

\setlength{\hangindent}{56pt}{
莫怀古\hspace{40pt}~
将玉杯收好。
}

\setlength{\hangindent}{56pt}{
莫成\hspace{40pt}~
是。
}

\setlength{\hangindent}{56pt}{
(揣杯)\hspace{40pt}~
}

\setlength{\hangindent}{56pt}{
莫怀古\hspace{40pt}~
有了一捧雪,拿稳作官,怕他何来。
}

\setlength{\hangindent}{56pt}{
莫成\hspace{40pt}~
着哇!
有了一捧雪,拿稳作官,还怕他何来。啊老爷,但不知严爷上轿之时({\akai 或}:~临行之时),讲些什么?
}

\setlength{\hangindent}{56pt}{
莫怀古\hspace{40pt}~
讲了两句淡话。
}

\setlength{\hangindent}{56pt}{
莫成\hspace{40pt}~
哪两句话?
}

\setlength{\hangindent}{56pt}{
莫怀古\hspace{40pt}~
三日定要灭满门。
}

\setlength{\hangindent}{56pt}{
莫成\hspace{40pt}~
三日要灭满门({\akai 或}:~``三日之后,灭却尔满门''),哎呀老爷呀,这三日要灭满门({\akai 或}:~这三日之后),是灭夫人的满门,还是灭小人的满门,一定是灭老爷的满门呐。
}

\setlength{\hangindent}{56pt}{
莫怀古\hspace{40pt}~
怎么讲?
}

\setlength{\hangindent}{56pt}{
莫成\hspace{40pt}~
老爷的满门呐。
}

\setlength{\hangindent}{56pt}{
莫怀古\hspace{40pt}~
哪个?
}

\setlength{\hangindent}{56pt}{
莫成\hspace{40pt}~
呵,(是)老爷呀。
}

\setlength{\hangindent}{56pt}{
莫怀古\hspace{40pt}~
哎呀!
}

\setlength{\hangindent}{56pt}{
(昏坐,莫成、雪艳左右)\hspace{10pt}~
}

\setlength{\hangindent}{56pt}{
莫成\hspace{40pt}~
老爷醒来。
}

\setlength{\hangindent}{56pt}{
雪艳\hspace{40pt}~
老爷醒来。(同莫成)
}

\setlength{\hangindent}{56pt}{
莫怀古\hspace{40pt}~
【{\akai 二黄导板}】听说三日灭满门,
}

\setlength{\hangindent}{56pt}{
(立,\textless{}\!{\bfseries\akai 三叫头}\!\textgreater{})
}

\setlength{\hangindent}{56pt}{
莫怀古\hspace{40pt}~
夫人。
}

\setlength{\hangindent}{56pt}{
雪艳\hspace{40pt}~
老爷。
}

\setlength{\hangindent}{56pt}{
莫怀古\hspace{40pt}~
莫成。
}

\setlength{\hangindent}{56pt}{
莫成\hspace{40pt}~
老爷。
}

\setlength{\hangindent}{56pt}{
莫怀古\hspace{40pt}~
夫人呐!
}

\setlength{\hangindent}{56pt}{
雪艳\hspace{40pt}~
老爷呀!
}

\setlength{\hangindent}{56pt}{
莫怀古\hspace{40pt}~
【{\akai 二黄散板}】吓得三魂少二魂。向前忙对掌家论,快想良谋救我身。
}

\setlength{\hangindent}{56pt}{
莫成\hspace{40pt}~
这,有道是不做他人官,不受他人管,不如弃官走({\akai 或}:~弃官逃走)了罢。
}

\setlength{\hangindent}{56pt}{
莫怀古\hspace{40pt}~
走得的。
}

\setlength{\hangindent}{56pt}{
莫成\hspace{40pt}~
啊,走得的。
}

\setlength{\hangindent}{56pt}{
莫怀古\hspace{40pt}~
如此回转钱塘。
}

\setlength{\hangindent}{56pt}{
莫成\hspace{40pt}~
且慢,哪个不知老爷是钱塘人氏,他必然往钱塘追赶。
}

\setlength{\hangindent}{56pt}{
莫怀古\hspace{40pt}~
那往何处而去。
}

\setlength{\hangindent}{56pt}{
莫成\hspace{40pt}~
小人跟随老爷进京之时打从海岱门前经过,遇着一位穿红袍的官员他姓什么戚。
}

\setlength{\hangindent}{56pt}{
莫怀古\hspace{40pt}~
敢是那戚继光?
}

\setlength{\hangindent}{56pt}{
莫成\hspace{40pt}~
不错,正是那戚继光戚大老爷,他今在何处为官?
}

\setlength{\hangindent}{56pt}{
莫怀古\hspace{40pt}~
蓟州总镇。
}

\setlength{\hangindent}{56pt}{
莫成\hspace{40pt}~
你我主仆(就)往蓟州而逃。
}

\setlength{\hangindent}{56pt}{
莫怀古\hspace{40pt}~
如此吩咐外厢车马伺候。
}

\setlength{\hangindent}{56pt}{
莫成\hspace{40pt}~
哎呀老爷呀,事到如今还要什么车马呀,必须换了亵衣小帽,混出城去再作道理。
}

\setlength{\hangindent}{56pt}{
雪艳\hspace{40pt}~
喂呀$\cdots{}\cdots{}$
}

\setlength{\hangindent}{56pt}{
莫怀古\hspace{40pt}~
下官连累你了。
}

\setlength{\hangindent}{56pt}{
(\textless{}\!{\bfseries\akai 尾声}\!\textgreater{}前段,雪艳下)
}

\setlength{\hangindent}{56pt}{
莫怀古\hspace{40pt}~
\textless{}\!{\bfseries\akai 叫头}\!\textgreater{}莫成你老爷进京未满一月,这身荣耀怎能舍得。
}

\setlength{\hangindent}{56pt}{
莫成\hspace{40pt}~
\textless{}\!{\bfseries\akai 叫头}\!\textgreater{}哎呀老爷呀,事到如今舍不得也要(翻袖内转身,莫怀古一望)舍,丢不得也要(翻袖外转身,莫怀古两望)丢,舍、丢了、请。(一轰两轰,拱手,莫怀古下)咳!
}

\setlength{\hangindent}{56pt}{
(投袖、抓袖,\textless{}尾声合头\textgreater{}莫成下)
}

\setlength{\hangindent}{56pt}{
\vspace{3pt}{\centerline{{[}{\hei 第三场}{]}}}\vspace{5pt}\hspace{20pt}~
}

\setlength{\hangindent}{56pt}{
(严世藩众上、汤勤迎,严下轿,进门,严坐中间,汤勤小边立,校尉站门)
}

\setlength{\hangindent}{56pt}{
严世藩\hspace{40pt}~
将汤勤绑了。
}

\setlength{\hangindent}{56pt}{
汤勤\hspace{40pt}~
且慢!启大人,他若是真杯献与大人必然拿稳做官,他若假杯献与大人必然弃官逃走,且听校尉一报,再绑不迟。
}

\setlength{\hangindent}{56pt}{
严世藩\hspace{40pt}~
且听校尉一报。
}

\setlength{\hangindent}{56pt}{
(张龙、郭义进门,站拜,立大边)
}

\setlength{\hangindent}{56pt}{
张龙、郭义\hspace{40pt}~
参见大人,莫怀古弃官逃走。
}

\setlength{\hangindent}{56pt}{
汤勤\hspace{40pt}~
大人如何?
}

\setlength{\hangindent}{56pt}{
(严世藩立)\hspace{30pt}~
}

\setlength{\hangindent}{56pt}{
严世藩\hspace{40pt}~
莫仁兄呐,真杯也罢,假杯也罢,只管拿稳做官,不该弃官逃走,来,顺轿。
}

\setlength{\hangindent}{56pt}{
汤勤\hspace{40pt}~
大人哪里去?
}

\setlength{\hangindent}{56pt}{
严世藩\hspace{40pt}~
追赶莫仁兄回来做官。
}

\setlength{\hangindent}{56pt}{
汤勤\hspace{40pt}~
他如今做不得官了。
}

\setlength{\hangindent}{56pt}{
严世藩\hspace{40pt}~
依你之见。
}

\setlength{\hangindent}{56pt}{
汤勤\hspace{40pt}~
必须行文各处将他捉拿,大小治他一个罪名。
}

\setlength{\hangindent}{56pt}{
严世藩\hspace{40pt}~
待我修文。
}

\setlength{\hangindent}{56pt}{
汤勤\hspace{40pt}~
小人溶墨。
}

\setlength{\hangindent}{56pt}{
(严世藩大坐修文,汤勤立小边磨墨)
}

\setlength{\hangindent}{56pt}{
严世藩\hspace{40pt}~
太子少保兵部左侍郎严,票行阃外事,为犯官一名莫怀古怀带皇家印信弃官逃走,有欺君误国之罪,命马上二校尉沿途追赶,不论文武大小衙门拿获者$\cdots{}\cdots{}$
}

\setlength{\hangindent}{56pt}{
汤勤\hspace{40pt}~
斩头解京。
}

\setlength{\hangindent}{56pt}{
严世藩\hspace{40pt}~
我那莫仁兄哪有这么大的罪过?
}

\setlength{\hangindent}{56pt}{
汤勤\hspace{40pt}~
这是他自作自受,哪个混账王八羔子害他不成。
}

\setlength{\hangindent}{56pt}{
严世藩\hspace{40pt}~
咳,斩头解京。来,公文一角,沿途追赶,不得有误。
}

\setlength{\hangindent}{56pt}{
汤勤\hspace{40pt}~
你们必须往蓟州追赶。
}

\setlength{\hangindent}{56pt}{
张龙、郭义\hspace{40pt}~
遵命。
}

\setlength{\hangindent}{56pt}{
(张龙、郭义出门)\hspace{20pt}~
}

\setlength{\hangindent}{56pt}{
严世藩\hspace{40pt}~
转来。
}

\setlength{\hangindent}{56pt}{
(张龙、郭义回来)\hspace{20pt}~
}

\setlength{\hangindent}{56pt}{
严世藩\hspace{40pt}~
莫怀古事小,一捧雪事大。
}

\setlength{\hangindent}{56pt}{
(张龙、郭义出门,汤勤跟过去)
}

\setlength{\hangindent}{56pt}{
汤勤\hspace{40pt}~
二位上差,一捧雪事小,雪娘子事大。
}

\setlength{\hangindent}{56pt}{
张龙、郭义\hspace{40pt}~
哼!
}

\setlength{\hangindent}{56pt}{
(张龙、郭义下,汤勤回来)\hspace{10pt}~
}

\setlength{\hangindent}{56pt}{
严世藩\hspace{40pt}~
汤勤随我饮酒来。
}

\setlength{\hangindent}{56pt}{
(严世藩、汤勤同下)\hspace{20pt}~
}

\setlength{\hangindent}{56pt}{
\vspace{3pt}{\centerline{{[}{\hei 第四场}{]}}}\vspace{5pt}\hspace{20pt}~
}

\setlength{\hangindent}{56pt}{
(\textless{}\!{\bfseries\akai 水底鱼}\!\textgreater{}张龙、郭义上)
}

\setlength{\hangindent}{56pt}{
张龙、郭义\hspace{40pt}~
俺。
}

\setlength{\hangindent}{56pt}{
张龙\hspace{40pt}~
张龙。
}

\setlength{\hangindent}{56pt}{
郭义\hspace{40pt}~
郭义。
}

\setlength{\hangindent}{56pt}{
张龙\hspace{40pt}~
请了。
}

\setlength{\hangindent}{56pt}{
郭义\hspace{40pt}~
请了。
}

\setlength{\hangindent}{56pt}{
张龙\hspace{40pt}~
奉了严爷之命,追赶莫怀古夫妇,就此马上加鞭。
}

\setlength{\hangindent}{56pt}{
(\textless{}\!{\bfseries\akai 撤锣}\!\textgreater{}同下)
}

\setlength{\hangindent}{56pt}{
\vspace{3pt}{\centerline{{[}{\hei 第五场}{]}}}\vspace{5pt}\hspace{20pt}~
}

\setlength{\hangindent}{56pt}{
莫成\hspace{40pt}~
({\akai 内}白)趱行。
}

\setlength{\hangindent}{56pt}{
({小锣}\textless{}\!{\bfseries\akai 柳青娘}\!\textgreater{}莫成背包袱上扎犄角,莫怀古、雪艳跟上,九龙口雪艳哭,坐下)
}

\setlength{\hangindent}{56pt}{
雪艳\hspace{40pt}~
喂呀$\cdots{}\cdots{}$
}

\setlength{\hangindent}{56pt}{
莫怀古\hspace{40pt}~
夫人为何不走?
}

\setlength{\hangindent}{56pt}{
雪艳\hspace{40pt}~
两足疼痛难以行走。
}

\setlength{\hangindent}{56pt}{
莫怀古\hspace{40pt}~
莫成。
}

\setlength{\hangindent}{56pt}{
(莫成回身)\hspace{30pt}~
}

\setlength{\hangindent}{56pt}{
莫成\hspace{40pt}~
老爷何事?
}

\setlength{\hangindent}{56pt}{
莫怀古\hspace{40pt}~
夫人两腿疼痛难以行走,如何是好?
}

\setlength{\hangindent}{56pt}{
(莫成望)\hspace{40pt}~
}

\setlength{\hangindent}{56pt}{
莫成\hspace{40pt}~
此地离蓟州西门不远,待小人去至前面,雇乘小轿前来
}

\setlength{\hangindent}{56pt}{
(莫成望)\hspace{40pt}~
}

\setlength{\hangindent}{56pt}{
莫成\hspace{40pt}~
老爷事要小心呐。
}

\setlength{\hangindent}{56pt}{
(莫成包袱交莫怀古,莫怀古望)
}

\setlength{\hangindent}{56pt}{
莫怀古\hspace{40pt}~
你要小心呐,柳林相会。
}

\setlength{\hangindent}{56pt}{
莫成\hspace{40pt}~
是。
}

\setlength{\hangindent}{56pt}{
(莫成下)\hspace{40pt}~
}

\setlength{\hangindent}{56pt}{
莫怀古\hspace{40pt}~
夫人,掌家前去雇轿,待下官搀扶于你,柳林躲藏。
}

\setlength{\hangindent}{56pt}{
(莫怀古搀雪艳到大边里面,雪艳坐,莫怀古立,张龙、郭义上小边立)
}

\setlength{\hangindent}{56pt}{
张龙\hspace{40pt}~
他们在前面走,我们在后面赶,赶至此地为何不见?
}

\setlength{\hangindent}{56pt}{
郭义\hspace{40pt}~
想必柳林躲藏,你我冒叫一声。
}

\setlength{\hangindent}{56pt}{
张龙、郭义\hspace{40pt}~
(同叫)里面可是莫大人?
}

\setlength{\hangindent}{56pt}{
雪艳\hspace{40pt}~
老爷外面有人唤你。
}

\setlength{\hangindent}{56pt}{
莫怀古\hspace{40pt}~
是哪一位?
}

\setlength{\hangindent}{56pt}{
张龙、郭义\hspace{40pt}~
你是莫怀古,锁了。
}

\setlength{\hangindent}{56pt}{
(张龙、郭义押莫怀古、雪艳下,郭义拿莫怀古包袱)
}

\setlength{\hangindent}{56pt}{
{[}连场}{]}}}\vspace{5pt}\hspace{30pt}~
}

\setlength{\hangindent}{56pt}{
(张龙、郭义、莫怀古、雪艳上叫城同下,张龙上击堂鼓,四役旗牌{\akai 引}戚继光上)
}

\setlength{\hangindent}{56pt}{
戚继光\hspace{40pt}~
辕门鼓角声高,想是公文来到。
}

\setlength{\hangindent}{56pt}{
(戚继光上高台,两层桌,桌后两椅背对背,戚跨椅上站桌后,张龙立大边)
}

\setlength{\hangindent}{56pt}{
张龙\hspace{40pt}~
大人请了。
}

\setlength{\hangindent}{56pt}{
戚继光\hspace{40pt}~
请了。
}

\setlength{\hangindent}{56pt}{
张龙\hspace{40pt}~
上司行文大人观看。(递公文)
}

\setlength{\hangindent}{56pt}{
戚继光\hspace{40pt}~
当堂拆封,人犯可曾带齐?
}

\setlength{\hangindent}{56pt}{
张龙\hspace{40pt}~
人犯带上。
}

\setlength{\hangindent}{56pt}{
(郭义、莫怀古、雪艳上,莫怀古大边,郭义、雪艳小边立)
}

\setlength{\hangindent}{56pt}{
雪艳\hspace{40pt}~
喂呀$\cdots{}\cdots{}$
}

\setlength{\hangindent}{56pt}{
莫怀古\hspace{40pt}~
夫人不必害怕,来此戚贤弟衙料也无妨,上面敢是戚$\cdots{}\cdots{}$
}

\setlength{\hangindent}{56pt}{
戚继光\hspace{40pt}~
嗯,本镇点名哪怕你们不齐,听点。
}

\setlength{\hangindent}{56pt}{
旗牌\hspace{40pt}~
犯官一名莫怀古。
}

\setlength{\hangindent}{56pt}{
莫怀古\hspace{40pt}~
有。
}

\setlength{\hangindent}{56pt}{
旗牌\hspace{40pt}~
女犯无名。
}

\setlength{\hangindent}{56pt}{
戚继光\hspace{40pt}~
带下去。
}

\setlength{\hangindent}{56pt}{
莫怀古\hspace{40pt}~
夫人,事到如今连戚贤弟也不认你我了。
}

\setlength{\hangindent}{56pt}{
雪艳\hspace{40pt}~
这都是你交的好朋友。
}

\setlength{\hangindent}{56pt}{
(莫怀古、雪艳同下)\hspace{20pt}~
}

\setlength{\hangindent}{56pt}{
戚继光\hspace{40pt}~
他二人哪里拿获的?(戚继光记录)
}

\setlength{\hangindent}{56pt}{
张龙\hspace{40pt}~
西门以外柳林之下。
}

\setlength{\hangindent}{56pt}{
戚继光\hspace{40pt}~
什么时候?
}

\setlength{\hangindent}{56pt}{
张龙\hspace{40pt}~
黄昏时候。
}

\setlength{\hangindent}{56pt}{
戚继光\hspace{40pt}~
怎样进城?
}

\setlength{\hangindent}{56pt}{
张龙\hspace{40pt}~
叫开城门,批了箚子,来见大人。
}

\setlength{\hangindent}{56pt}{
戚继光\hspace{40pt}~
本镇看来,此事重大,必须做一个两下担待。
}

\setlength{\hangindent}{56pt}{
张龙\hspace{40pt}~
何谓两下担待?
}

\setlength{\hangindent}{56pt}{
戚继光\hspace{40pt}~
头门以里,仪门以外,有军牢小房,里面有火,外面有锁,锁上加封,将你等四人锁在一起,待等五鼓天明,看着绑,看着斩,人头打入木桶,回复严爷。
}

\setlength{\hangindent}{56pt}{
张龙\hspace{40pt}~
好却好,只是我等二人辛苦。
}

\setlength{\hangindent}{56pt}{
戚继光\hspace{40pt}~
自有你二人的下程。
}

\setlength{\hangindent}{56pt}{
(张龙、郭义四役同下)
}

\setlength{\hangindent}{56pt}{
戚继光\hspace{40pt}~
且住,想我那莫年兄不知为了何事冒犯严府,想他有一掌家名叫莫成,颇能办事,为何不跟随前来,是了,想是人烟众多挨挤不上也未可知,不免去至大街之上寻找便了,来,掌灯。(更)。
}

\setlength{\hangindent}{56pt}{
戚继光\hspace{40pt}~
【{\akai 二黄散板}】人来掌灯大街进,前去寻找小莫成。
}

\setlength{\hangindent}{56pt}{
(\textless{}\!{\bfseries\akai 撤锣}\!\textgreater{}旗牌、戚继光下,更夫上)
}

\setlength{\hangindent}{56pt}{
更夫\hspace{40pt}~
为人莫当差,当差不自在。风里也得去,雨里也得来。我更夫便是。只因夜间拿住犯官莫怀古,五更天明就要开刀,街前严禁。就此巡更去者。
}

\setlength{\hangindent}{56pt}{
(\textless{}\!{\bfseries\akai 水底鱼}\!\textgreater{}更夫走,莫成上相撞)
}

\setlength{\hangindent}{56pt}{
莫成\hspace{40pt}~
({\akai 内}白)走哇。
}

\setlength{\hangindent}{56pt}{
更夫  拿住了。\hspace{30pt}~
}

\setlength{\hangindent}{56pt}{
莫成\hspace{40pt}~
拿住了什么?
}

\setlength{\hangindent}{56pt}{
更夫\hspace{40pt}~
拿住犯夜的了。
}

\setlength{\hangindent}{56pt}{
莫成\hspace{40pt}~
我不是犯夜的呀。
}

\setlength{\hangindent}{56pt}{
更夫\hspace{40pt}~
你不是犯夜的,你是谁呀?
}

\setlength{\hangindent}{56pt}{
莫成\hspace{40pt}~
我是乡下人呐。
}

\setlength{\hangindent}{56pt}{
更夫\hspace{40pt}~
乡下人不犯夜。难道说城里的人才犯夜吗?
}

\setlength{\hangindent}{56pt}{
莫成\hspace{40pt}~
呵大哥,我是前来交粮米的呀。
}

\setlength{\hangindent}{56pt}{
更夫\hspace{40pt}~
交钱粮的,文官衙门去交,你怎么跑这儿来拉?
}

\setlength{\hangindent}{56pt}{
莫成\hspace{40pt}~
这是哪个衙门?
}

\setlength{\hangindent}{56pt}{
更夫\hspace{40pt}~
这是戚大人的衙门。
}

\setlength{\hangindent}{56pt}{
莫成\hspace{40pt}~
为何这样的热闹?
}

\setlength{\hangindent}{56pt}{
更夫\hspace{40pt}~
你不知道,今夜拿住犯官一名叫莫怀古,五更天明就要开刀问斩。
}

\setlength{\hangindent}{56pt}{
莫成\hspace{40pt}~
咳,老爷呀。({\hwfs 哭}{\hwfs 介})
}

\setlength{\hangindent}{56pt}{
更夫\hspace{40pt}~
你哭哪一门子呀?
}

\setlength{\hangindent}{56pt}{
莫成\hspace{40pt}~
人人都说这位莫大老爷为官清正,我故而心中酸痛呐({\akai 或}:~在此叹息呀)。
}

\setlength{\hangindent}{56pt}{
更夫\hspace{40pt}~
你这不是看兵书落泪,替古人担忧吗。
}

\setlength{\hangindent}{56pt}{
莫成\hspace{40pt}~
本来是清官呐。
}

\setlength{\hangindent}{56pt}{
更夫\hspace{40pt}~
你有住处吗?
}

\setlength{\hangindent}{56pt}{
莫成\hspace{40pt}~
无有哇。
}

\setlength{\hangindent}{56pt}{
更夫\hspace{40pt}~
这么办吧,你先到我更房里去,到天明亮了你再去交钱粮,哎呀可是我的地方窄小哇。
}

\setlength{\hangindent}{56pt}{
莫成\hspace{40pt}~
方便方便吧。
}

\setlength{\hangindent}{56pt}{
(二人摸)\hspace{40pt}~
}

\setlength{\hangindent}{56pt}{
更夫\hspace{40pt}~
你在哪里呢?
}

\setlength{\hangindent}{56pt}{
莫成\hspace{40pt}~
呵呵呵,这里来,(我)在这里。({\akai 或}:~呃呃呃,在这里,这里来,呃,在这里。)
}

\setlength{\hangindent}{56pt}{
更夫\hspace{40pt}~
你多大岁数了?
}

\setlength{\hangindent}{56pt}{
莫成\hspace{40pt}~
我四十六岁了。
}

\setlength{\hangindent}{56pt}{
更夫\hspace{40pt}~
老头子呀,你那边去,我这边,(进更棚坐)替我听着点儿更呐。
}

\setlength{\hangindent}{56pt}{
(二更\footnote{《京剧新序》中``更''字误为``桌''。},旗牌、戚继光上)
}

\setlength{\hangindent}{56pt}{
戚继光\hspace{40pt}~
【{\akai 二黄散板}】听谯楼打罢了二更时分,八台官倒做了巡更之人。
}

\setlength{\hangindent}{56pt}{
莫成\hspace{40pt}~
(唉,)老爷呀$\cdots{}\cdots{}$
}

\setlength{\hangindent}{56pt}{
戚继光\hspace{40pt}~
({\akai 接唱})啼哭之人哪一个?
}

\setlength{\hangindent}{56pt}{
(莫成出来)\hspace{30pt}~
}

\setlength{\hangindent}{56pt}{
莫成\hspace{40pt}~
我是莫$\cdots{}\cdots{}$
}

\setlength{\hangindent}{56pt}{
戚继光\hspace{40pt}~
噤声。(拉莫成下)
}

\setlength{\hangindent}{56pt}{
更夫\hspace{40pt}~
到时辰了,哎!。打更锣锤不见了,拿脑袋撞,不怕不怕。
}

\setlength{\hangindent}{56pt}{
(更夫下,旗牌、戚继光、莫成上)
}

\setlength{\hangindent}{56pt}{
戚继光\hspace{40pt}~
({\akai 接唱})来在二堂问分明
}

\setlength{\hangindent}{56pt}{
莫成\hspace{40pt}~
叩见大人。
}

\setlength{\hangindent}{56pt}{
(莫成跪)\hspace{40pt}~
}

\setlength{\hangindent}{56pt}{
戚继光\hspace{40pt}~
罢了,你家老爷来了。
}

\setlength{\hangindent}{56pt}{
莫成\hspace{40pt}~
怎么我家老爷也来了么,可容我主仆一见?
}

\setlength{\hangindent}{56pt}{
戚继光\hspace{40pt}~
容你主仆一见。
}

\setlength{\hangindent}{56pt}{
莫成\hspace{40pt}~
谢大人。
}

\setlength{\hangindent}{56pt}{
戚继光\hspace{40pt}~
下面伺候。
}

\setlength{\hangindent}{56pt}{
莫成\hspace{40pt}~
是,这就好了。(莫成下)
}

\setlength{\hangindent}{56pt}{
戚继光\hspace{40pt}~
来。
}

\setlength{\hangindent}{56pt}{
旗牌\hspace{40pt}~
在。
}

\setlength{\hangindent}{56pt}{
戚继光\hspace{40pt}~
看看严府校尉可曾睡着?
}

\setlength{\hangindent}{56pt}{
(旗牌望上场门,三更)
}

\setlength{\hangindent}{56pt}{
旗牌\hspace{40pt}~
睡着了。
}

\setlength{\hangindent}{56pt}{
戚继光\hspace{40pt}~
悄悄启开封锁,有请莫大老爷。
}

\setlength{\hangindent}{56pt}{
(旗牌开门)\hspace{30pt}~
}

\setlength{\hangindent}{56pt}{
旗牌\hspace{40pt}~
有请莫大老爷。
}

\setlength{\hangindent}{56pt}{
(【小拉子】莫怀古、雪艳上,旗牌带门,雪艳哭)
}

\setlength{\hangindent}{56pt}{
莫怀古\hspace{40pt}~
【{\akai 反二黄散板}】夫人啼哭莫高声,休要惊动严府人。悲切切且把二堂进。
}

\setlength{\hangindent}{56pt}{
(莫怀古、雪艳进门,戚继光、旗牌大边,莫中间,雪小边,均立)
}

\setlength{\hangindent}{56pt}{
戚继光\hspace{40pt}~
({\akai 接唱})
【{\akai 二黄散板}】披枷带锁为何情。(白)仁兄你掌家莫成来了。
}

\setlength{\hangindent}{56pt}{
莫怀古\hspace{40pt}~
在哪里?
}

\setlength{\hangindent}{56pt}{
戚继光\hspace{40pt}~
请莫掌家。
}

\setlength{\hangindent}{56pt}{
旗牌\hspace{40pt}~
莫掌家,你家老爷来了。
}

\setlength{\hangindent}{56pt}{
莫成\hspace{40pt}~
来了。(上,进门小边立)
}

\setlength{\hangindent}{56pt}{
莫成  老爷在哪里,老爷受惊了。
}

\setlength{\hangindent}{56pt}{
莫怀古\hspace{40pt}~
你这奴才办得好事呀。
}

\setlength{\hangindent}{56pt}{
莫成\hspace{40pt}~
事到如今埋怨小人可也是枉然(的)了哇。
}

\setlength{\hangindent}{56pt}{
戚继光\hspace{40pt}~
是呀,事到如今
掌家也是枉然了。但不知仁兄因何冒犯严府?
}

\setlength{\hangindent}{56pt}{
莫怀古\hspace{40pt}~
就为的是一捧雪。
}

\setlength{\hangindent}{56pt}{
戚继光\hspace{40pt}~
一捧雪乃是小事,为何有紧急公文到来?
}

\setlength{\hangindent}{56pt}{
莫成\hspace{40pt}~
怎么,这紧急公文({\akai 或}:~这行事的公文也)到了么?
}

\setlength{\hangindent}{56pt}{
莫怀古\hspace{40pt}~
来得好快,待我一观。
}

\setlength{\hangindent}{56pt}{
戚继光\hspace{40pt}~
不看也罢。
}

\setlength{\hangindent}{56pt}{
莫怀古\hspace{40pt}~
看了也好做一准备。
}

\setlength{\hangindent}{56pt}{
莫成\hspace{40pt}~
是呀,看了({\akai 或}:~看过)也好做一准备呀。
}

\setlength{\hangindent}{56pt}{
戚继光\hspace{40pt}~
旗牌掌灯,仁兄请看。。
}

\setlength{\hangindent}{56pt}{
(旗牌掌灯,众台中间前边看文)
}

\setlength{\hangindent}{56pt}{
莫怀古\hspace{40pt}~
太子少保兵部左侍郎严,票行阃外事,为犯官莫怀古怀带皇家印信,弃官逃走有欺君误国之罪,命马上校尉沿途追赶,不论文武大小衙门拿获者$\cdots{}\cdots{}$
}

\setlength{\hangindent}{56pt}{
(戚继光抢文)\hspace{30pt}~
}

\setlength{\hangindent}{56pt}{
莫怀古\hspace{40pt}~
为何不教我看了?
}

\setlength{\hangindent}{56pt}{
莫成\hspace{40pt}~
呵大人,为何不教我家老爷观看呐?
}

\setlength{\hangindent}{56pt}{
戚继光\hspace{40pt}~
恐怕仁兄看了心惊。
}

\setlength{\hangindent}{56pt}{
莫怀古\hspace{40pt}~
看了也好做一准备。
}

\setlength{\hangindent}{56pt}{
莫成\hspace{40pt}~
是呀,看了也好做一准备呀。
}

\setlength{\hangindent}{56pt}{
戚继光\hspace{40pt}~
如此仁兄看来。
}

\setlength{\hangindent}{56pt}{
莫怀古\hspace{40pt}~
拿获者斩头解京。哎呀!
}

\setlength{\hangindent}{56pt}{
莫成\hspace{40pt}~
哎呀!
}

\setlength{\hangindent}{56pt}{
(戚继光拿公文交旗牌,旗牌下场门下,莫怀古吓昏,气椅)
}

\setlength{\hangindent}{56pt}{
戚继光\hspace{40pt}~
仁兄醒来。
}

\setlength{\hangindent}{56pt}{
莫成\hspace{40pt}~
老爷醒来。(同戚)
}

\setlength{\hangindent}{56pt}{
莫怀古\hspace{40pt}~
【{\akai 二黄导板}】听说斩头要解京,
}

\setlength{\hangindent}{56pt}{
(众哭{\hwfs 介})\hspace{40pt}~
}

\setlength{\hangindent}{56pt}{
莫怀古\hspace{40pt}~
({\akai 接唱})【{\akai 二黄散板}】好似钢刀刺我心。回头再对贤弟论,
}

\setlength{\hangindent}{56pt}{
(莫怀古揖\textless{}\!{\bfseries\akai 乱锤}\!\textgreater{},戚继光扶)
}

\setlength{\hangindent}{56pt}{
莫怀古\hspace{40pt}~
({\akai 接唱})快想良谋救我生。
}

\setlength{\hangindent}{56pt}{
戚继光\hspace{40pt}~
仁兄,有道是不做他人官,不受他人管,不如弃官逃走了罢。
}

\setlength{\hangindent}{56pt}{
莫成\hspace{40pt}~
哦,走得的(么)。
}

\setlength{\hangindent}{56pt}{
戚继光\hspace{40pt}~
走得的。
}

\setlength{\hangindent}{56pt}{
莫成\hspace{40pt}~
如此说来,(我们)走哇!
}

\setlength{\hangindent}{56pt}{
(莫成、莫怀古、雪艳、戚继光出门一串往下场门走,站住,莫成拦)
}

\setlength{\hangindent}{56pt}{
莫成\hspace{40pt}~
走不得,走$\cdots{}\cdots{}$不得。
}

\setlength{\hangindent}{56pt}{
(莫成翻回来领众进门,站原地)
}

\setlength{\hangindent}{56pt}{
莫成\hspace{40pt}~
哎呀大人呀!想我家老爷(就是)为的弃官逃走才惹下这场杀身大祸,如今又要弃官逃走,岂不连累戚、戚$\cdots{}\cdots{}$大人,走、走$\cdots{}\cdots{}$不得。
}

\setlength{\hangindent}{56pt}{
戚继光\hspace{40pt}~
\textless{}\!{\bfseries\akai 叫头}\!\textgreater{}也罢,不如你我点动人马反了罢!
}

\setlength{\hangindent}{56pt}{
莫成\hspace{40pt}~
反得的。
}

\setlength{\hangindent}{56pt}{
戚继光\hspace{40pt}~
反得的。
}

\setlength{\hangindent}{56pt}{
莫成\hspace{40pt}~
如此就反呐。
}

\setlength{\hangindent}{56pt}{
(莫成举拳领众往上场门走,站住,莫成拦)
}

\setlength{\hangindent}{56pt}{
莫成\hspace{40pt}~
反不得,反、反$\cdots{}\cdots{}$不得。
}

\setlength{\hangindent}{56pt}{
戚继光\hspace{40pt}~
怎么反不得?
}

\setlength{\hangindent}{56pt}{
莫成\hspace{40pt}~
请问大人,这蓟州有多少人马?
}

\setlength{\hangindent}{56pt}{
戚继光\hspace{40pt}~
三千人马,五百守城军。
}

\setlength{\hangindent}{56pt}{
莫成\hspace{40pt}~
着哇,这三千人马,五百守城军,离乱年间尚可抵挡一阵,这太平年间慢说是交锋打仗,就是垫马蹄也是不、不$\cdots{}\cdots{}$够哇,反、反$\cdots{}\cdots{}$反不得。
}

\setlength{\hangindent}{56pt}{
戚继光\hspace{40pt}~
咳!
【{\akai 二黄散板}】叫你反来你不反,叫你逃走你不行。待等五鼓天明亮,我坐法堂你受刑。
}

\setlength{\hangindent}{56pt}{
雪艳\hspace{40pt}~
({\hwfs 哭}{\hwfs 介})喂呀$\cdots{}\cdots{}$
}

\setlength{\hangindent}{56pt}{
(莫怀古、戚继光、雪艳分坐中,大、小两边)
}

\setlength{\hangindent}{56pt}{
莫成\hspace{40pt}~
\textless{}\!{\bfseries\akai 叫头}\!\textgreater{}老爷,大人,哎夫人呐!
}

\setlength{\hangindent}{56pt}{
(面里正中)【{\akai 二黄导板}】一家人只哭得如酒哇
}

\setlength{\hangindent}{56pt}{
(撕领子,转身向外亮住)\hspace{10pt}~
}

\setlength{\hangindent}{56pt}{
莫成\hspace{40pt}~
({\akai 接唱})醉呀。
}

\setlength{\hangindent}{56pt}{
莫成\hspace{40pt}~
\textless{}\!{\bfseries\akai 叫头}\!\textgreater{}老爷,大人,哎夫人呐!
}

\setlength{\hangindent}{56pt}{
(\textless{}\!{\bfseries\akai 四击头}、{撕边一锣}\!\textgreater{}莫成转身到大边台口弓箭步,右手推胡子望雪艳,雪艳哭)
}

\setlength{\hangindent}{56pt}{
莫成\hspace{40pt}~
({\akai 接唱})【{\akai 二黄原板}】那一边哭坏了雪氏夫人。
}

\setlength{\hangindent}{56pt}{
(``那一边''往里一指,到唱``夫人''时画圈里指外甩胡须,三番,\textless{}\!{\bfseries\akai 大锣搓锤}\!\textgreater{},三搓手边向小边走,摆袖、投袖、到小边台口翻袖望戚继光)
}

\setlength{\hangindent}{56pt}{
莫成\hspace{40pt}~
({\akai 接唱})
【{\akai 二黄原板}】戚大人八台官救不了家主爷的命,家主爷的命,老爷呀,实实的难坏了小莫成。
}

\setlength{\hangindent}{56pt}{
莫成\hspace{40pt}~
(白)且住!事到如今我倒想起一桩心事({\akai 或}:~一桩事故)来了,曾记得我家老爷进京之时,大夫人在钱塘府中,手举({\akai 或}:~手执)斗酒叫道:~莫成呐掌家,此番跟随你家老爷进京,
事要正办,酒要少饮,要办上他几桩好事小心回来,慢说是你家老爷要另眼看待于你,就是夫人在钱塘也要好好照看你那文禄孩儿喏$\cdots{}\cdots{}$如今我家老爷进京,得罪了汤勤狗男女,害得我家老爷有这样的杀身大祸,事到如今,难道说教我这为奴的呀,看水流舟不成。哎呀!
}

\setlength{\hangindent}{56pt}{
\begin{quote}\hspace{10pt}~
(\textless{}\!{\bfseries\akai 乱锤}\!\textgreater{}手比画往里半转身)咳!(再起\textless{}\!{\bfseries\akai 乱锤}\!\textgreater{}搓手,想)事到如今,我又想起一桩心事来了,曾记得跟随我家老爷,去到长街拜客,遇着一位相面的先生与我家老爷看了一相,然后又与我觑了一觑,那(相面的)先生说道:~莫大哥哇莫掌家,你的好贵相呵,这相,你有({\akai 或}:~倒有)你(家)老爷之相,可惜无有老爷之福哇,老爷日后有一桩大事,必然应在掌家你的身上。呜哙呀,想那相面的先生他说之无意,我倒是听之有记呀,莫非今日就应在这蓟$\cdots{}\cdots{}$州堂上?(看莫怀古,托胡子看自己,\textless{}\!{\bfseries\akai 乱锤}\!\textgreater{},走到莫怀古座左边单腿跪托莫怀古胡子看,右手挡向外托自己胡子看,握着双手回到台口小边,\textless{}\!{\bfseries\akai 叫头}\!\textgreater{})哎呀!想我莫成生在世上,无非是奴仆而已,怎能(或:~焉能)有个出头之日,不如今晚({\akai 或}:~今日)在这蓟州堂上替我家老爷一死,日后也落得个青史名标,(这)万古流芳,我就是这个主意呀,(右拳击左掌)哦,我$\cdots{}\cdots{}$就是这个主$\cdots{}\cdots{}$意呀$\cdots{}\cdots{}$
}

\setlength{\hangindent}{56pt}{
【{\akai 二黄散板}】走向前来忙跪定,(走过去,跪戚继光面前偏右些)
\end{quote}
}

\setlength{\hangindent}{56pt}{
莫成\hspace{40pt}~
({\akai 接唱})【{\akai 二黄散板}】我家老爷有救星。(白)禀大人,我家老爷有救。
}

\setlength{\hangindent}{56pt}{
戚继光\hspace{40pt}~
救在哪里?
}

\setlength{\hangindent}{56pt}{
莫成\hspace{40pt}~
只要大人开天高地厚之恩,我家老爷就有救。
}

\setlength{\hangindent}{56pt}{
戚继光\hspace{40pt}~
起来。
}

\setlength{\hangindent}{56pt}{
莫成\hspace{40pt}~
谢大人。
}

\setlength{\hangindent}{56pt}{
(起回小边,戚继光向莫怀古)\hspace{10pt}~
}

\setlength{\hangindent}{56pt}{
戚继光\hspace{40pt}~
仁兄醒来。
}

\setlength{\hangindent}{56pt}{
(莫怀古、雪艳站起,莫怀古中间,戚继光、雪艳左右)
}

\setlength{\hangindent}{56pt}{
莫怀古\hspace{40pt}~
贤弟何事?
}

\setlength{\hangindent}{56pt}{
戚继光\hspace{40pt}~
仁兄有救了。
}

\setlength{\hangindent}{56pt}{
莫怀古\hspace{40pt}~
救在哪里?
}

\setlength{\hangindent}{56pt}{
戚继光\hspace{40pt}~
莫成言道仁兄有救。
}

\setlength{\hangindent}{56pt}{
(莫怀古向莫成)\hspace{30pt}~
}

\setlength{\hangindent}{56pt}{
莫怀古\hspace{40pt}~
莫成,你家老爷救在哪里?
}

\setlength{\hangindent}{56pt}{
莫成\hspace{40pt}~
哎呀老爷呀!事到如今还有什么救哇,小人情愿替老爷一死。
}

\setlength{\hangindent}{56pt}{
莫怀古\hspace{40pt}~
呵,想世上哪有人替人死之理,你有这两句话可也就够了。
}

\setlength{\hangindent}{56pt}{
莫成\hspace{40pt}~
老爷若是不信,小人有一辈古人说与老爷、大人、夫人听。
}

\setlength{\hangindent}{56pt}{
莫怀古、戚继光、雪艳\hspace{20pt}~
你且讲来。
}

\setlength{\hangindent}{56pt}{
莫成\hspace{40pt}~
容禀:~昔日有一杨生最好育犬,一日杨生酒醉睡卧在荒山,(右手枕状,身侧卧状)彼时有一牧童他不晓得事务,他就放火(双翻袖过大边,面斜向外举双手,右先左后轰袖)烧荒,看看那火就烧在杨生的身上,(双手下指,边指边回小边)那黄犬是慌忙无计({\akai 或}:~忙中无计),只得跳下涧去呀,(双手胸前平向右侧画圈指)滚草\footnote{《京剧新序》中``草''字误为``叫''。}救火({\akai 或}:~湿透毛衣,滚火救主;或:~湿透毛衣,滚草救火)(右手左侧身边投袖,左手右侧身边投袖,双投)。那杨生醒来,只见那黄犬就累死在身旁(双手下指),彼时杨生捶胸顿足({\akai 或}:~捶胸跌足)(比画),仰天叹曰({\akai 或}:~对天叹曰),马有垂缰之德,羊有跪乳之恩,乌鸦有反哺之义,犬有救主之心,何况小人乃是人乎,老爷今日不教小人替死,我就碰(台口翻右袖挡脸,三人拦)死在这蓟州堂上。
}

\setlength{\hangindent}{56pt}{
莫怀古、戚继光、雪艳\hspace{20pt}~
不必如此。
}

\setlength{\hangindent}{56pt}{
(换巾子,莫成带链子,扶莫成里边正坐昏介,莫怀古中、戚继光、雪艳左右三人跪朝里拜)
}

\setlength{\hangindent}{56pt}{
莫怀古、戚继光、雪艳\hspace{20pt}~
【{\akai 二黄导板}】莫成请上礼恭敬,拜你如同拜先人。
}

\setlength{\hangindent}{56pt}{
莫成\hspace{40pt}~
【{\akai 二黄导板}】未曾犯法先上刑,
}

\setlength{\hangindent}{56pt}{
(莫成立,推莫怀古正坐,成到小边台口面外)
}

\setlength{\hangindent}{56pt}{
莫成\hspace{40pt}~
({\akai 接唱})【{\akai 二黄散板}】我、我$\cdots{}\cdots{}$犹如去到({\akai 或}:~犹如来到)哇那枉死城。眼望呵,钱塘哭文禄,喂呀我的儿啊!苦命的娇儿无靠承。走向前来把话禀({\akai 或}:~走向前来忙告禀),(往里走到莫怀古左侧跪)恕小人有言要禀明({\akai 或}:~有话要禀明)。
}

\setlength{\hangindent}{56pt}{
莫怀古\hspace{40pt}~
你且讲来。
}

\setlength{\hangindent}{56pt}{
莫成\hspace{40pt}~
哎呀老爷呀!小人今日替老爷一死,并无牵挂,只有一个文禄孩儿在钱塘伺候大相公在学中攻书,那大相公是性情不好哇,比不得老爷待小人这样恩厚,倘有不到之处,那大相公开口就骂,举手就打({\akai 或}:~举手就打,开口就骂),想我那文禄孩儿他是三岁丧母,今日小人替老爷一死。看看他是七岁丧父哇。望求老爷另眼照看我那薄命的孩儿,小人纵死九泉,唉,也就瞑目了哇。
}

\setlength{\hangindent}{56pt}{
莫怀古\hspace{40pt}~
莫成,日后我若错待你那孩儿,叫我天诛地灭。(发誓一跪)
}

\setlength{\hangindent}{56pt}{
莫成\hspace{40pt}~
谢老爷。(立)【{\akai 二黄散板}】文禄孩儿有了靠,纵死九泉也甘心呐。水流,千遭,归大海呀\textless{}\!{\bfseries\akai 哭头}\!\textgreater{},(掏杯)原物交还旧主人。
}

\setlength{\hangindent}{56pt}{
(交杯,莫怀古接)\hspace{20pt}~
}

\setlength{\hangindent}{56pt}{
莫怀古\hspace{40pt}~
({\akai 接唱})【{\akai 二黄散板}】玉杯本是起货根,为你伤了小莫成。一怒将杯来倾碎,
}

\setlength{\hangindent}{56pt}{
(莫怀古摔杯,戚继光拦,接过杯)
}

\setlength{\hangindent}{56pt}{
戚继光\hspace{40pt}~
({\akai 接唱})【{\akai 二黄散板}】倾杯犹如欺先人。
}

\setlength{\hangindent}{56pt}{
莫怀古\hspace{40pt}~
将此杯寄在贤弟衙内,日后见杯如见愚兄一般。
}

\setlength{\hangindent}{56pt}{
雪艳\hspace{40pt}~
喂呀$\cdots{}\cdots{}$
}

\setlength{\hangindent}{56pt}{
莫怀古\hspace{40pt}~
贤弟请上受兄一拜。
}

\setlength{\hangindent}{56pt}{
戚继光\hspace{40pt}~
施礼为何?
}

\setlength{\hangindent}{56pt}{
莫怀古\hspace{40pt}~
将贱妾雪艳寄在贤弟衙内,休当贱妾看待,当作使女丫鬟。
}

\setlength{\hangindent}{56pt}{
戚继光\hspace{40pt}~
不敢,仁嫂看待。
}

\setlength{\hangindent}{56pt}{
莫怀古\hspace{40pt}~
多谢贤弟。
}

\setlength{\hangindent}{56pt}{
莫成\hspace{40pt}~
老爷,还有小人呢。({\hwfs 哭}{\hwfs 介})
}

\setlength{\hangindent}{56pt}{
莫怀古\hspace{40pt}~
贤弟受愚兄全礼。
}

\setlength{\hangindent}{56pt}{
戚继光\hspace{40pt}~
全礼为何?
}

\setlength{\hangindent}{56pt}{
莫怀古\hspace{40pt}~
待到五更天明将我恩人斩首,赏下棺木一口,将他成殓起来,埋在西门以外柳林之下,立一碑碣上写明故太常寺正卿莫公之墓,日后我那儿孙也好与他烧钱化纸。
}

\setlength{\hangindent}{56pt}{
莫成\hspace{40pt}~
谢老爷。
}

\setlength{\hangindent}{56pt}{
莫怀古\hspace{40pt}~
【{\akai 二黄散板}】三件大事托付你,
}

\setlength{\hangindent}{56pt}{
(打四更,莫成数)\hspace{20pt}~
}

\setlength{\hangindent}{56pt}{
莫成\hspace{40pt}~
({\akai 接唱})【{\akai 二黄散板}】又听得谯楼打四更。(白)哎呀老爷呀!谯楼已打四更({\akai 或}:~谯楼已打四鼓),看看五鼓天明,难道说这蓟州堂上还有两个莫怀古不成。
}

\setlength{\hangindent}{56pt}{
戚继光\hspace{40pt}~
\textless{}\!{\bfseries\akai 乱锤}\!\textgreater{}\textless{}\!{\bfseries\akai 叫头}\!\textgreater{}仁兄!小弟有一好友在古北为官,待弟修书一封,仁兄去往那里躲避躲避。
}

\setlength{\hangindent}{56pt}{
莫怀古\hspace{40pt}~
多谢修书。
}

\setlength{\hangindent}{56pt}{
戚继光\hspace{40pt}~
仁兄更衣。
}

\setlength{\hangindent}{56pt}{
(莫成\hspace{40pt}~
老爷后面更衣。)
}

\setlength{\hangindent}{56pt}{
(莫怀古、雪艳下)\hspace{20pt}~
}

\setlength{\hangindent}{56pt}{
莫成\hspace{40pt}~
小人磨墨。
}

\setlength{\hangindent}{56pt}{
(莫成磨墨,戚继光立桌大边修书)
}

\setlength{\hangindent}{56pt}{
戚继光\hspace{40pt}~
【{\akai 二黄散板}】上写顿首三顿首,拜上古北魏参谋。怀古本是我好友,还望仁兄好收留。一封书信忙修就,仁兄快快离蓟州。
}

\setlength{\hangindent}{56pt}{
(莫怀古换装,雪艳同上)\hspace{10pt}~
}

\setlength{\hangindent}{56pt}{
(莫怀古下。戚继光写书信。排子。莫怀古上,戚继光交书信。)
}

\setlength{\hangindent}{56pt}{
莫怀古\hspace{40pt}~
多谢了。【{\akai 二黄散板}】多谢贤弟施恻隐,搭救愚兄命残生。回头便对夫人论,下官言来听分明。五鼓天明时刻到,你向莫成叫夫君。
}

\setlength{\hangindent}{56pt}{
莫成\hspace{40pt}~
小人不敢。
}

\setlength{\hangindent}{56pt}{
莫怀古\hspace{40pt}~
({\akai 接唱})辞别贤弟足踏镫,
}

\setlength{\hangindent}{56pt}{
(旗牌带马,莫怀古扶马)\hspace{10pt}~
}

\setlength{\hangindent}{56pt}{
莫成\hspace{40pt}~
老爷请转呐$\cdots{}\cdots{}$
}

\setlength{\hangindent}{56pt}{
莫怀古\hspace{40pt}~
({\akai 接唱})莫成起下追悔心?
}

\setlength{\hangindent}{56pt}{
(莫怀古回来,旗牌带马一边)\hspace{10pt}~
}

\setlength{\hangindent}{56pt}{
莫怀古\hspace{40pt}~
莫成敢是有追悔之意,来来来将刑具与我戴上。
}

\setlength{\hangindent}{56pt}{
莫成\hspace{40pt}~
哎呀老爷呀!小人焉有追悔之意,(想)老爷此番去至古北({\akai 或}:~去至湖广),事要正办,酒要少饮,当交的交上他几个,不要像那汤勤狗男女,害得老爷这样(的)杀身大祸。今日在这蓟州堂上,有我这样一个不怕死的莫成替老爷一死,老爷日后,再有这样的祸事,再想这第二个莫成呐,只怕是无有了。
}

\setlength{\hangindent}{56pt}{
莫怀古\hspace{40pt}~
话是好话,可惜你讲迟了。
}

\setlength{\hangindent}{56pt}{
莫成\hspace{40pt}~
也还不迟,老爷快快上马去吧。
}

\setlength{\hangindent}{56pt}{
(旗牌带马,莫怀古上马)\hspace{10pt}~
}

\setlength{\hangindent}{56pt}{
(同\textless{}\!{\bfseries\akai 三叫头}\!\textgreater{},莫怀古下,旗牌下,戚继光、雪艳、莫成进门)
}

\setlength{\hangindent}{56pt}{
雪艳\hspace{40pt}~
({\hwfs 哭})喂呀!
}

\setlength{\hangindent}{56pt}{
戚继光\hspace{40pt}~
【{\akai 二黄散板}】仁嫂休要珠泪汪。
}

\setlength{\hangindent}{56pt}{
雪艳\hspace{40pt}~
【{\akai 二黄散板}】全凭大人作主张。
}

\setlength{\hangindent}{56pt}{
(雪艳下)\hspace{40pt}~
}

\setlength{\hangindent}{56pt}{
莫成\hspace{40pt}~
大事全仗戚总镇呐,
}

\setlength{\hangindent}{56pt}{
戚继光\hspace{40pt}~
(白)莫成,
}

\setlength{\hangindent}{56pt}{
戚继光\hspace{40pt}~
【{\akai 二黄散板}】你的美名天下扬。(白)莫成,少时五鼓天明,去到法场,不要胡言,不要乱语,你老爷性命,本镇前程,俱在你身上。
}

\setlength{\hangindent}{56pt}{
莫成\hspace{40pt}~
大人,少时五鼓天明,去至法场,小人我一不、不敢胡言,二不、不敢乱语,只求大人与小人一个快。
}

\setlength{\hangindent}{56pt}{
(跪求,戚继光搀,退下,莫成转身面里中间)
}

\setlength{\hangindent}{56pt}{
莫成\hspace{40pt}~
\textless{}\!{\bfseries\akai 叫头}\!\textgreater{}文禄,我儿,儿呀!(转身面外)儿在钱塘,今日也盼为父回来,明日也盼为父回去,盼来盼去将为父盼到枉死城中来了哇$\cdots{}\cdots{}$\textless{}\!{\bfseries\akai 叫头}\!\textgreater{}文禄,我儿,难得见儿呀!({\hwfs 哭}{\hwfs 介})呀呀呸!想我莫成今日替我家老爷一死,日后纵然不能青史名标,也会落得个流传百世,此乃是一桩喜事呀,我是痛的什么,诶,我$\cdots{}\cdots{}$又哭$\cdots{}\cdots{}$的什么,(或:~我是哭的什么,诶,我$\cdots{}\cdots{}$又痛$\cdots{}\cdots{}$的是什么?)我$\cdots{}\cdots{}$必须要笑哇,嗯哼,我是要笑哇。\textless{}\!{\bfseries\akai 叫头}\!\textgreater{}哈$\cdots{}\cdots{}$,哈$\cdots{}\cdots{}$呵$\cdots{}\cdots{}$儿呀!(打五更,比势做右手扶脖,坐子,起来)罢。
}

\setlength{\hangindent}{56pt}{
(下)\hspace{40pt}~
}

\setlength{\hangindent}{56pt}{
{[}连场}{]}}}\vspace{5pt}\hspace{30pt}~
}

\setlength{\hangindent}{56pt}{
(起牌子\textless{}\!{\bfseries\akai 吹打}\!\textgreater{}四龙套,张龙、郭义骑马,禁卒拿桶,旗牌,戚继光骑马过场。四校尉,二刀斧手,莫成披红官衣、纱帽,招子绑上,扎犄角,回来{\akai 引}雪艳上,张龙、郭义、戚继光同上高台,牌子停)
}

\setlength{\hangindent}{56pt}{
戚继光\hspace{40pt}~
({\akai 念})本镇坐法堂,摆下杀人场,若有冤枉事,全仗一炉香。
}

\setlength{\hangindent}{56pt}{
\begin{quote}\hspace{10pt}~
本镇,戚继光。奉了严大人之命,监斩莫怀古,刀斧手,将莫怀古绑上来,
\end{quote}
}

\setlength{\hangindent}{56pt}{
(莫成原人上,成面里)
}

\setlength{\hangindent}{56pt}{
戚继光\hspace{40pt}~
二位看得清,验得明。
}

\setlength{\hangindent}{56pt}{
(戚继光点招子,刀斧手去莫成乌纱、官衣,成面外)
}

\setlength{\hangindent}{56pt}{
莫成\hspace{40pt}~
天呐天,想我莫$\cdots{}\cdots{}$
}

\setlength{\hangindent}{56pt}{
戚继光\hspace{40pt}~
刀斧手,将莫怀古绑好了!
}

\setlength{\hangindent}{56pt}{
雪艳\hspace{40pt}~
老爷你那心中要放明白些呀!
}

\setlength{\hangindent}{56pt}{
莫成\hspace{40pt}~
莫$\cdots{}\cdots{}$怀古,死得好不瞑目哇$\cdots{}\cdots{}$
}

\setlength{\hangindent}{56pt}{
(起\textless{}千秋岁\textgreater{}圆场,莫成蹉步、雪艳跪蹉,押成下,牌子完。二刀斧手上,头交雪,雪哭、咬头毁容,禁卒抢头、打入木桶交张龙、郭义,雪、禁下)
}

\setlength{\hangindent}{56pt}{
戚继光\hspace{40pt}~
二位公差,此案已毕,文书一轴回复严爷,外有手本问候大人金安,二位辛苦。
}

\setlength{\hangindent}{56pt}{
张龙、郭义\hspace{40pt}~
谢大人。(张龙、郭义下)
}

\setlength{\hangindent}{56pt}{
戚继光\hspace{40pt}~
军士们,回衙门。
}

\setlength{\hangindent}{56pt}{
(\textless{}\!{\bfseries\akai 尾声}\!\textgreater{}同下)
}
