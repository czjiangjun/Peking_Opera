\newpage
\phantomsection %实现目录的正确跳转
\section*{\large\hei {清官册~{\small 之}~寇准}}
\addcontentsline{toc}{section}{\hei 清官册~{\small 之}~寇准}

\hangafter=1                   %2. 设置从第1⾏之后开始悬挂缩进  %}
\setlength{\parindent}{0pt}{

{\vspace{3pt}{\centerline{{[}{\hei 第一场}{]}}}\vspace{5pt}}

{{[}{\akai 引子}{]}做清官民之父母,积阴功留与儿孙。}

{({\akai 念})读诗书智广才高,中皇榜青史名标。三杯御酒加封号,被权臣一本参掉。}

{下官寇准,陕西华州人氏。蒙圣恩得中一甲一名,不想被权臣参掉。是我在吏部效力三载,蒙八千岁提拔,才得职授霞峪县正堂。自到任以来,地方安定。今当三、六、九日,放告之期。左右,}

{将放告牌抬出。}

{有请。}

{万岁,}

{万万岁。}

{({\akai 念})即刻便登程。}

{转堂。}

{有请夫人。}

{夫人,请坐。}

{金牌调我连夜进京,不知为了何事。}

{但愿如此。}

{即刻启程。}

{有劳夫人。}

\setlength{\hangindent}{56pt}{【{\akai 二黄原板}】接过了夫人酒一樽,背转身来谢神灵。贤夫人请上受一礼,下官言来你试听:~高堂老母要你孝敬,早晚侍奉要殷勤。弓开难留弦上箭,}

\setlength{\hangindent}{56pt}{【{\akai 二黄摇板}】舟急哪顾岸上人。}

\vspace{3pt}{\centerline{{[}{\hei 第二场}{]}}}\vspace{5pt}

\setlength{\hangindent}{56pt}{【{\akai 二黄散板}】一路马踏芳草尽,日落西山小桃红。}

{罢了。}

{前途俱已用过。}

{今晚小心更鼓。}

{家院,四更时分,冠带伺候。}

{想我寇准,职授霞峪县令,为官以来,上不负君,下不亏民。圣上金牌调我连夜进京,不知为了何事。今晚独宿馆驿,好不愁闷人也------}

\setlength{\hangindent}{56pt}{【{\akai 二黄慢板}】一轮明月正东升,想起了高堂上老娘亲。伴君犹如羊伴虎,尽得忠啊来难把孝行。}

\setlength{\hangindent}{56pt}{【{\akai 二黄原板}】移星换斗二更时分,想起当年一举成名。八贤爷奏一本领凭上任,来至在霞峪县主管}\footnote{夏行涛{\scriptsize 君}建议作``治管''或``执管''。}{万民。早堂接状早堂审,午堂接状要审明。到晚来接下了无头冤状,一对红灯审到了天明。}

\setlength{\hangindent}{56pt}{【{\akai 二黄原板}】听谯楼打三更人烟静,一轮明月照街心。霞峪县我不曾亏负百姓,金牌调我所为何情。}

{看衣更换。}

\setlength{\hangindent}{56pt}{【{\akai 二黄原板}】顶冠束带四更尽,忙把家院叫一声:~我命你回衙报一信,你就说平安到了都城。倘若是太夫人将你问,你就说你老爷进都城、平步登云往上升,切莫要挂心。}

\setlength{\hangindent}{56pt}{【{\akai 二黄原板}】朝臣待漏五更冷,铁甲将军夜宿津。朝房鼓不住地嗵嗵打,文武百官列朝门。东华门前文官走,西华门前武将行。}

\setlength{\hangindent}{56pt}{【{\akai 二黄原板}】我寇准打从这东华门进,两旁文武着了惊。都看我七品小县令,小小前程也来见君。有才不在官职小,无才枉受爵禄恩。整装敛容丹墀进,}

\setlength{\hangindent}{56pt}{【{\akai 二黄散板}】品级台前见当今。}

{臣寇准见驾,吾皇万岁!}

{调臣进京,不知有何圣命?}

{臣启万岁,潘、杨两家,一家是当朝太师,一家是皇家郡马,臣官卑职小,难以审问。}

{谢主隆恩!}

{({\akai 念})捧旨下龙庭。}

{({\akai 念})叩见八贤君。}

{千岁在上。恕臣有王命在身,不能全礼。}

{贤爷千岁!}

{进京来了。}

{调臣进京,审问潘、杨两家之事。}

{蒙圣恩,七品县令升为西台御史。}

{千岁提拔。}

{臣却不知。}

{({\hwfs 惊介})有这等事?}

{待臣回复圣命。}

{多谢千岁!}

\setlength{\hangindent}{56pt}{【{\akai 二黄散板}】八千岁做了主大胆审问,哪怕那潘洪贼国戚皇亲。}

\vspace{3pt}{\centerline{{[}{\hei 第三场}{]}}}\vspace{5pt}

\setlength{\hangindent}{56pt}{【{\akai 二黄摇板}】一枝杏花香十里,状元归来马如飞。}

{供奉圣旨。}

{有请!}

{公公!}

{喜从何来?}

{公公提拔。}

{在敝衙审问。}

{({\hwfs 惊介})好一份厚礼呀!}

{啊,公公此礼为何?}

{王法森严,必须按律而断!}

{无功不受禄啊!}

{不敢,收下。}

{({\akai 念})王法不徇情!}

{且住!正要升堂理事,后宫潘娘娘送来一份厚礼,与老贼讲情。我若收下此礼,岂不学了前任刘御史;我若不收此礼,后宫娘娘降罪如何事好?!哎呀,这、这、这$\cdots{}\cdots{}$}

{有了,下殿之时,八千岁言道:~若有为难之处,可至南清宫领教。}

{左右,打道南清宫。}

\vspace{3pt}{\centerline{{[}{\hei 第四场}{]}}}\vspace{5pt}

\setlength{\hangindent}{56pt}{【{\akai 二黄散板}】急忙来在宫闱境,心有疑难问圣明。}

{来此宫门,待我叩环。}

{烦劳通禀:~寇准求见。}

{领旨!}

{臣寇准见驾,贤爷千岁。}

{谢座。}

{臣正要升堂理事,后宫潘娘娘送来一份厚礼,现有礼单在此,贤爷请看。}

{臣若收了此礼,岂不学了前任刘御史之故。}

{也罢,就暂寄南清宫,候事完毕,再作定夺。}

{千岁何出此言?}

{哎呀,臣要按律而断!}

{谢千岁}

{臣乃步行而来。}

{谢千岁。}

{呃,千岁在此,多有不便,将马往下带。}

{呃,方才言过,贤爷在此,多有不便,往下带,往下带。}

{哎呀!}

\setlength{\hangindent}{56pt}{【{\akai 二黄散板}】自盘古哇哪有君与臣带马呀,}

\setlength{\hangindent}{56pt}{【{\akai 二黄散板}】臣大胆跨龙驹足踏金镫,}

\setlength{\hangindent}{56pt}{【{\akai 二黄散板}】得意洋洋发笑声。}

{呵呵哈哈哈$\cdots{}\cdots{}$({\hwfs 笑介})}

{({\hwfs 掩口介})}

\vspace{3pt}{\centerline{{[}{\hei 第五场}{]}}}\vspace{5pt}

\setlength{\hangindent}{56pt}{【{\akai 二黄散板}】御史衙前下金镫,升坐大堂鬼神惊。}

{来,升堂。}

{今日升堂理事,刑具俱要齐备。}

{潘洪到此,教他报门而进。}

{潘洪,见了本御史为何不跪?}

{呵呵呵呵$\cdots{}\cdots{}$({\hwfs 冷笑介})}

{你欺我官卑职小。来,请过圣命!}

{潘洪:~圣旨在上,本御史在此,你怎样私通北国,苦害杨家,从实招来一一讲!}

{怎么讲?}

{潘洪,你这卖国的奸贼!}

{自古道:~({\akai 念})君待臣以礼,臣事君以忠。}

{想你身为当朝太师,一人之下,万万人之上,你是何等侥幸?谁想你这老贼心怀叵测:~命你子潘豹在天齐庙前摆下百日擂台,要将天下的英雄一网打尽,你这老贼也好谋篡社稷。}

{也是那杨老将军他的家规不严呐,那杨七将军私出府门,行至在天齐庙前,见你子潘豹在擂台之上是洋洋得意。那杨七将军性如烈火,焉能容得?上得擂台,三拳两足,将你子潘豹打死。}

{你这老贼就与那杨老将军抓袍掳带,面见当今。好一个有道明君,不加罪过,在龙楼之上,与你两家解和。谁想,你这老贼怀恨在心,修书一封,私通北国胡儿,教他们打来连环战表,夺取宋室天下。你这老贼在金殿之上,讨下帅印,单单就要那杨老将军以为前站先行。那杨老将军上殿,连辞数本,万岁不准。只得在金殿之上讨一名保官。圣上就命呼延老将军做了杨家的保官。你这老贼也要讨一名保官,想这满朝文武,谁来保你?呵,偏偏那贺朝进,与你这贼狼狈为奸!那杨老将军见势不祥,只得去到瓦桥三关,调他两个孩儿回营,共灭胡儿。}

{你这老贼,兵到雁门,升帐点卯。天气炎热,误了你的卯期,可也是有之啊。怎么,你这老贼就要将杨老将军斩首。那呼延老将军进帐讲情,你这老贼假意准情,又命人报道:~营中缺粮。想你这为元帅者,岂不知:~兵马未动,粮草先行。营中焉能缺粮?你故意命那呼延老将军催解粮草。想那呼延老将军乃是他杨家的保官,岂能替你这老贼前去催粮?本当不允,又恐违背你的将令。那呼延老将军出得大营,大笑了三声,气堵胸膛,就口吐鲜血而亡了!}

{那杨老将军见呼延老将军一死,犹如断了他杨家的命脉,就带他两子,怒出大营,不听你的调遣。你这老贼就命白牌请过了尚方宝剑,追赶他父子回营。那杨七将军性如烈火,打碎了白牌,扭断了令箭。那杨老将军,乃是知罪的臣子啊,就命他六子回营请罪。你也不管他是皇家的郡马,就一捆四十,叉出了大营。}

{黄道日期,你不准他父子出兵;黑道日期,反命他父子出马。偏偏他父子又是得胜而归,你就该大开城门,迎接他父子进城,才是你做元帅的道理呀。怎么,你反命那贺朝进带领五百名雁翎刀手,把守在雁门关,对那杨老将军言道:~必须将北国胡儿斩尽杀绝,方许进城。想那北国的胡儿犹如潮水一般,一时焉能斩得尽,杀得绝?他父子万般无奈,就杀一阵、败一阵,败一阵、杀一阵,败至在这两狼山下!}

{他父子被困在两狼山,那杨老将军就命那杨七将军回转雁门,搬兵取救。不想你这老贼想起了打子的仇恨,将他诓下马来,用酒灌醉,绑在法标之上,射了他一百单三箭!将他射死,你这打子的仇恨也就报了。怎么,你还是按兵不动呢?}

{那杨老将军只为放心不下,又命杨六将军杀出重围,探听下落。那杨老将军被困在两狼山,({\akai 念})盼兵兵不到,

望子子不归。白日受饥饿,夜晚受风吹。万般无奈,就碰死在李陵碑下!}

{那杨六将军闻得他父已死,进京告下御状。圣上命刘御史审问你这老贼,审得是不清不明,被八千岁金锏打死。万岁又发金牌连夜调本御史进京,审问你这老贼。你这老贼,为臣不能尽忠,为子不能尽孝。似你这等不忠不孝、卖国欺君,国法岂能容得?!}

\setlength{\hangindent}{56pt}{【{\akai 二黄散板}】老贼不信抬头看,本御史非比前任官。}

{来,打!}

{呀呸!}

\setlength{\hangindent}{56pt}{【{\akai 二黄散板}】皇亲国戚我不打,打的谋朝篡位臣。}

{打!}

{潘洪!万岁在这里问你,你是怎样私通北国,苦害杨家,速速招来!}

{呸!}

\setlength{\hangindent}{56pt}{【{\akai 二黄散板}】人来看过铜夹棍,看他招承不招承。}

{问他有招无招。}

{收!}

{松刑。}

{潘洪,万岁又在那里问你,你是怎样苦害杨家,按兵不动,谁与同谋?}

{呀呸。}

{(【{\akai 二黄散板}】人来看过红铁链,不招即刻赴幽冥。)}

{(哎呀!)}

{(且住,五刑用过,老贼并无半点口供,竟而气绝身亡,这、这、这$\cdots{}\cdots{}$)}

{(快些取来!)}

{(嗯------)}

{啊太师,不必如此,待下官将此事推在杨郡马身上,与太师无干就是。}

{搀了下去。}

{哎呀且住,五刑用尽,老贼并无半点口供,不免再去南清宫商议。}

{来,带马南清宫去者!}

\vspace{3pt}{\centerline{{[}{\hei 第六场}{]}}}\vspace{5pt}

{千岁不必惊慌,为臣这里还有------一张!}
