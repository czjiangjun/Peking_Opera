\newpage
\subsubsection{\large\hei 芦花河~{\small 之}~薛丁山~\protect\footnote{本剧本中有关人物台上地方由段公平{\scriptsize 君}协助整理。}}
\addcontentsline{toc}{subsection}{\hei 芦花河~{\small 之}~薛丁山}

\hangafter=1                   %2. 设置从第1⾏之后开始悬挂缩进  %
\setlength{\parindent}{0pt}{

({\akai 内})马来!

(薛丁山{\hwfs 上})

\setlength{\hangindent}{56pt}{【{\akai 西皮二六}】奉主旨意往西征,数年铠甲未离身。先父当年挂帅印,在白虎关前命归天庭。多亏了智勇樊夫人,她也能提调众三军。来至在辕门下金镫, }

(薛丁山{\hwfs 下马})

啊?!

\setlength{\hangindent}{56pt}{【{\akai 西皮快板}】辕门外绑定薛应龙。我儿犯了何条令,缘何捆绑问典刑({\akai 或}:~问斩刑)? }

(醒来。)

哦!

\setlength{\hangindent}{56pt}{【{\akai 西皮摇板}】我道是犯了那皇王的军令,却原来为的是这临阵招亲------ }

儿啊!

\setlength{\hangindent}{56pt}{【{\akai 西皮快板}】我的儿只管心放稳,为父进帐讲人情。进帐只用三两语,管教你母饶儿身。本帅撩袍宝帐进------ }

哦!

\setlength{\hangindent}{56pt}{【{\akai 西皮快板}】王法条条不徇情。我若讲情她不允({\akai 或}:~我若讲情她不准),把娇儿反送在这枉死城。 }

\setlength{\hangindent}{56pt}{【{\akai 西皮快板}】进帐去先行周公礼,必然念在夫妻的情。 }

\setlength{\hangindent}{56pt}{【{\akai 西皮摇板}】秦、窦二将往上禀({\akai 或}:~秦、窦二将一声禀),你就说二路的元帅转回大营。 }

(薛丁山{\hwfs 下})

$\bigg( \begin{aligned} &\mbox{秦汉}\\&\mbox{窦一虎}\mbox{\raisebox{5pt}{\hspace{18pt}二路元帅到!}} \end{aligned}\bigg)$

\setlength{\hangindent}{56pt}{(樊梨花\hspace{20pt}【{\akai 西皮导板}】$\cdots{}\cdots{}$) }

(薛丁山{\hwfs 上},{\hwfs 进帐},樊梨花{\hwfs 出帐},{\hwfs 二}人{\hwfs 撞肩膀})

(樊梨花\hspace{20pt}王爷!)

夫人!

(樊梨花\hspace{20pt}王爷请!)

夫人请!

夫人请坐!

({\hwfs 二}人{\hwfs 换位},薛丁山{\hwfs 大边}、樊梨花{\hwfs 小边},{\hwfs 坐})

\setlength{\hangindent}{56pt}{(樊梨花\hspace{20pt}【{\akai 西皮原板}】迎接元帅进大营,$\cdots{}\cdots{}$打听得哪路发来兵?) }

\setlength{\hangindent}{56pt}{【{\akai 西皮原板}】一来是夫人威名盛,各国闻名不敢动兵。 }

\setlength{\hangindent}{56pt}{(樊梨花\hspace{20pt}【{\akai 西皮原板}】$\cdots{}\cdots{}$闲事情。) }

呀!({\akai 或}:~哦!)

\setlength{\hangindent}{56pt}{【{\akai 西皮快板}】樊夫人她倒有({\akai 或}:~樊夫人她倒能)隔山照镜,就知本帅讲人情。未曾开言把罪请呐。 }

\setlength{\hangindent}{56pt}{(樊梨花\hspace{20pt}【{\akai 西皮原板}】问王爷施礼为何情?) }

\setlength{\hangindent}{56pt}{【{\akai 西皮摇板}】应龙儿犯了({\akai 或}:~身犯)何条令,缘何捆绑问典刑({\akai 或}:~缘何捆绑在辕门)? }

(樊梨花\hspace{20pt}王爷问的是他?)

正是!

(樊梨花\hspace{20pt}王爷呀!)

\setlength{\hangindent}{56pt}{(樊梨花\hspace{20pt}【{\akai 西皮二六}】$\cdots{}\cdots{}$问斩刑?) }

(哦。)

\setlength{\hangindent}{56pt}{【{\akai 西皮摇板}】我道是犯了那皇王(的)军令,却原来为的是这临阵招亲。提起来招亲的事,话也难尽,难道说贤夫人你心不明。想当年大战在那寒------ }

(樊梨花\hspace{20pt}噤声。)

掩门。

\setlength{\hangindent}{56pt}{【{\akai 西皮二六}】寒江岭,寒江关前动刀兵。我与夫人来会阵,夫人与我来提亲。 }

(薛丁山{\hwfs 拉}樊梨花,樊梨花{\hwfs 羞介},{\hwfs 二}人{\hwfs 换位})

\setlength{\hangindent}{56pt}{【{\footnotesize 接}{\akai 西皮二六}】本帅再三不应允,夫人又把巧计生。使下了({\akai 或}:~设下了)移山倒海阵, }

({\hwfs 二}人{\hwfs 换位})

\setlength{\hangindent}{64pt}{【{\footnotesize 接}{\akai 西皮二六}】将本帅吊在那({\akai 或}:~吊至在)半空存。那时我唤天,天不应;我待入地,地又无门。万般无奈才应允,夫妻双双进唐营。若论这临阵招亲,是你我先来做定,常言道前人开路,这后人行呐。 }

\setlength{\hangindent}{56pt}{(樊梨花\hspace{20pt}【{\akai 西皮快板}】王爷说话$\cdots{}\cdots{}$,军无私来就法无情。) }

\setlength{\hangindent}{56pt}{【{\akai 西皮摇板}】应龙犯罪理当斩, }

(樊梨花\hspace{20pt}谢王爷!)

且慢!

\setlength{\hangindent}{56pt}{【{\akai 西皮摇板}】还要看他的年纪轻。 }

\setlength{\hangindent}{56pt}{(樊梨花\hspace{20pt}【{\akai 西皮快板}】  $\cdots{}\cdots{}$不是娘生?)}

\setlength{\hangindent}{56pt}{【{\akai 西皮快板}】本帅与你讲人情,哪个和你比古人。大夫人生下麒麟子,二夫人也有后代根。唯独夫人无有后,收下应龙作螟蛉。到如今夫人有了梦熊信\footnote{古人以梦中见熊罴为生男的征兆。后以``梦熊''作生男的颂语。语本《诗·小雅·斯干》:~``吉梦维何?维熊维罴。''又:~``大人占之,维熊维罴,男子之祥。''  郑玄笺注:~``熊罴在山,阳之祥也,故为生男。''},便把应龙当外人。倘若是娇儿有伤损,旁人道你两样心。 }

\setlength{\hangindent}{56pt}{【{\akai 西皮快板}】你若是赦了应龙子,唐王降罪我担承。 }

(【{\akai 西皮快板}】你今赦了应龙子,满营将官我也担承。)

\setlength{\hangindent}{56pt}{【{\akai 西皮快板}】不能不能万不能呐。 }

\setlength{\hangindent}{56pt}{(樊梨花\hspace{20pt}【{\akai 西皮摇板}】  你把你$\cdots{}\cdots{}$看大了。)}

\setlength{\hangindent}{56pt}{【{\akai 西皮摇板}】威宁侯啊,也不放在本帅心。 }

哎呀!

\setlength{\hangindent}{56pt}{【{\akai 西皮快板}】一见宝剑挂营门,吓得三魂少二魂。眼望娇儿无救\textless{}\!{\bfseries\akai 哭头}\!\textgreater{}应,我的儿啊, }

\setlength{\hangindent}{56pt}{【{\akai 西皮摇板}】父子们做鬼一路行。 }

\textless{}\!{\bfseries\akai 哭头}\!\textgreater{}薛应龙,我的儿啊({\akai 或}:~小娇儿啊),啊$\cdots{}\cdots{}$

夫人,

\textless{}\!{\bfseries\akai 哭头}\!\textgreater{}我的儿啊!

夫人你看,众将皆服了({\akai 或}:~满营将官皆服了)。

\setlength{\hangindent}{56pt}{(樊梨花\hspace{20pt}【{\akai 西皮摇板}】$\cdots{}\cdots{}$平身。) }

解下桩来。

(樊梨花{\hwfs 欲踹}薛应龙{\hwfs 介})

夫人方才赦过了。

出帐去罢。\footnote{段公平{\scriptsize 君}注:~若不带``{观阵}''一场,此处可按如下处理:~

\hspace*{-2pt}(樊梨花\hspace{18pt}仙师有何吩咐?)

薛丁山\hspace{20pt}仙师赐你丹药二料,一要戴之胸膛,二要用清水服下。$\cdots{}\cdots{}$,料无妨碍。

薛丁山\hspace{20pt}夫人请至后帐。

\hspace*{-2pt}(薛丁山、樊梨花{\hwfs 下}) }

(探子\hspace{30pt}$\cdots{}\cdots{}$讨战。)

再探~!

夫人,贼人摆下阵势,你我夫妻敌楼一观------({\akai 或}:~军士们,带马城头去者。)

(樊梨花\hspace{20pt}带马。)

\setlength{\hangindent}{56pt}{【{\akai 西皮散板}】适才探马报一声,芦花河贼子发来兵。 }

\setlength{\hangindent}{56pt}{【{\akai 西皮散板}】下得马来敌楼进,观看贼阵是何名({\akai 或}:~观看贼子发来兵)。 }

夫人,贼人摆的是何阵势?

(樊梨花\hspace{20pt}此乃是金------)

噤声!

\setlength{\hangindent}{56pt}{【{\akai 西皮散板}】叫声夫人莫高声, }

(薛丁山、樊梨花{\hwfs 下城})

\setlength{\hangindent}{56pt}{【{\akai 西皮散板}】休要惊动那贼兵({\akai 或}:~休要惊动这贼兵)。 }

\setlength{\hangindent}{56pt}{【{\akai 西皮散板}】下得马来宝帐进({\akai 或}:~下得马来大营进), }

\setlength{\hangindent}{56pt}{【{\akai 西皮散板}】再与夫人把话云({\akai 或}:~夫妻对坐论军情)。 }

夫人方才讲的金什么? ({\akai 或}:~适才摆的什么阵势?)

(樊梨花\hspace{20pt}乃是金光大阵。)

可有破法?

如此(待)本帅二次回转仙山,哀求师父,求来法宝,再破此阵。

即刻启程,我有一言,夫人听了:~

(薛丁山{\hwfs 拉}樊梨花{\hwfs 到台口})

\setlength{\hangindent}{56pt}{【{\akai 西皮快板}】手挽手,站营门,尊声梨花樊夫人。芦花河摆下金光阵,莫教应龙去出征。倘若娇儿有伤损,那时失了夫妻情。辞别夫人上马行, }

\setlength{\hangindent}{56pt}{【{\akai 西皮摇板}】我嘱咐你言语呀,你(要)谨记在心呐。 }

}
