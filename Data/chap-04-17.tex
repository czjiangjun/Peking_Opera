\newpage 
\phantomsection %实现目录的正确跳转
\section*{\large\hei 打鱼杀家~\protect\footnote{此戏是《庆顶珠》中的一部分,据有关资料载\upcite{WSZS2-78_2000},原《庆顶珠》全剧有``得宝''、``庆珠''、``比武''、``珠聘''、``打鱼''、``恶讨''、``屈责''、``献珠''、``杀家''、``投亲''、``劫牢''、``珠圆''等场次。一般习惯从``打鱼''演到``杀家'',故称《打鱼杀家》;~旧时戏单``鱼''、``渔''混用,亦作``打渔杀家''。据吴小如先生告,{\textrm 1949}年后因田汉建议,戏名统一为《打渔杀家》。}~{\small 之}~萧恩}
\addcontentsline{toc}{section}{\hei 打鱼杀家~{\small 之}~萧恩}

\hangafter=1                   %2. 设置从第1⾏之后开始悬挂缩进  %}
\setlength{\parindent}{0pt}{
{\vspace{3pt}{\centerline{{[}{\hei 第一场}{]}}}\vspace{5pt}}

{开船呐!}

{儿啊------}

\setlength{\hangindent}{56pt}{【{\akai 西皮摇板}】父女打鱼在河下,家贫哪怕人笑咱。拉住篷索父把网撒,}

\setlength{\hangindent}{56pt}{【{\akai 西皮散板}】年纪衰迈气力不佳。}

{唉,本当不做这河下生意,你我父女何以度日呀?}

{儿啊,不要啼哭,将船湾在柳荫之下,凉爽凉爽。}

{儿啊,将几尾鲜鱼烹煮好了,少时为父还要饮酒。}

{有人唤我。}

{是哪一位?}

{哦,原来是李贤弟。}

{敢莫要到舟中走走?}

{待我搭了扶手。}

{此位是------?}

{({\hwfs 惊介})这做什么?}

{老了,不中用了。}

{儿啊,出舱来见过二位叔父。}

{小女桂英。}

{一十六岁,痴长啊。}

{且慢,方才打得几尾鲜鱼,就在船头之上,你我弟兄畅快饮一回。}

{儿啊,看酒来。}

{啊,二位贤弟,愚兄有个酒令儿。}

愚兄做的是河下生意,忌的是``干''、``旱''二字。

不敢说罚,要敬酒三杯。

请------

呃,敬你三杯。

哦,待我看来。

诶!做什么的?

问的是哪一家呀?

你来看,就在前面,八字粉墙,合脊门楼,那就是丁$\cdots{}\cdots{}$

诶,哼,放肆!

量他也不敢呐。

请------

哦,又有人唤我。

二位贤弟再饮几杯。

哦,原来是丁郎儿,到此何事啊?

你来看,这几日天旱水浅,鱼不上网。改日有了银钱,送上府去。

哦,是是是。

放他去罢,

教他去罢。

(李俊、倪荣 啊,萧兄为何这等$\cdots{}\cdots{}$。)

他们的人多。

他们的势力大。

这就难讲话了。

本当不做河下生意,怎奈这囊中------唉,惭愧。

哪位贤弟送来?

当面谢过。

有了人家了。

花荣之子,名唤花逢春。

请------

二位贤弟慢走,愚兄不能远送了。

哦,来了。

儿问的是他?

儿啊------

\setlength{\hangindent}{56pt}{【{\akai 西皮摇板}】他本江湖二豪侠,倪荣、李俊就是他。蟒袍、玉带不愿挂,弟兄双双走天涯。}

\setlength{\hangindent}{56pt}{【{\akai 西皮散板}】猛抬头见红日坠落西斜。}

{儿啊,天色不早,我们回去了吧。}

{正是:~({\akai 念})父女打鱼在江下,}

{({\akai 念})堪堪不觉红日落,}

\vspace{3pt}{\centerline{{[}{\hei 第二场}{]}}}\vspace{5pt}

\setlength{\hangindent}{66pt}{【{\akai 西皮快三眼}】昨夜晚吃酒醉和衣而卧,稼场鸡惊醒了梦里南柯。二贤弟在河下相劝与我,他教我把打鱼的事啊一旦丢却。我本当不打鱼啊关门闲坐,怎奈我家贫穷无计奈何。清早起开柴扉乌鸦叫过,飞过来叫过去【{\footnotesize 转}{\akai 西皮二六}】却是为何。将身儿来至在草堂内坐,桂英儿取茶来为父解渴。}

{不教儿渔家打扮,怎么偏偏要渔家打扮?}

{呃------不听父言,就为不孝哇。}

{这便才是!}

{是哪个?}

{你们是哪里来的?}

{哦,原来是丁府上的教师爷。}

{哼!}

{做什么来了?}

{这几日天旱水浅,鱼不上网。改日有钱,送上府去,何必你来!}

{旁人来了无有,教师爷你来了么,}

{哼哼,越发的无有了!}

{朝廷王法,要它何用?}

{哼!}

{哼,尔要锁?}

{当真要锁?}

{果然要锁?}

{如此你就------锁!}

{不教锁。}

{哼!}

{话么,倒是两句好话呀,可惜呀可惜,}

{可惜你二大爷无有工夫啊。}

{尔要讲打?}

{诶呀!老汉幼年间,听说打架,如同小孩子穿新鞋、过新年的一般;如今呐------老了,打不动了啊!}

{尔当真要打?}

{果然要打?!}

{娃娃!待老汉将衣帽留在家中,打个样儿与你们见识见识。}

\setlength{\hangindent}{56pt}{【{\akai 西皮导板}】听一言不由我七窍冒火,}

\setlength{\hangindent}{56pt}{【{\akai 西皮摇板}】不由我年迈人咬碎牙车}\footnote{据樊百乐{\scriptsize 君}告知,刘曾复先生强调,``牙车''是``牙床''的意思。吴小如先生曾撰文指出,天津的王庾生先生此句唱作``这才是闭门坐平地生波''。}{。江湖上叫萧恩不才是我,}

\setlength{\hangindent}{56pt}{【{\akai 西皮摇板}】大战场、小战场见过许多。爷本是出山虎独自一个,}

\setlength{\hangindent}{56pt}{【{\akai 西皮摇板}】尔好比看家犬一群一窝。你本是奴下奴敢来欺我。}

{慢说是三``羊头'',就是尔这三``狗头'',二大爷何惧!}

{你是丁府上的教师爷?}

{你好本领呐!老汉要领教领教。}

{一定要领教。}

{这是大十八般武艺?}

{这小十八样兵器?}

{这拳脚是?}

{软硬功夫?}

{呃,这叫什么?}

{不好哇。}

{这叫什么?}

{呃,不好。}

{呃,越发的不好啊。}

{方才撞了老汉三``羊头'',如今我要打你三拳头,放你过去。}

{哪里有功夫。}

{着打!}

{着打!}

{打得好,只恐打出祸来了!}

{那贼回去,定不甘休。待为父赶至县衙,抢他一个原告!}

{不要儿管,取为父的衣帽过来。}

{好好看守门户,为父去去就来。}\footnote{陈超老师注:~刘曾复先生曾说明:~此处萧恩\textless{}\!{\bfseries\akai 扫头}\!\textgreater{}{\hwfs 下场},{\hwfs 不踢大带}:~{\hwfs 出门},{\hwfs 褶子倒手},{\hwfs 右手向上一指},{\hwfs 弹髯下}。这个\textless{}\!{\bfseries\akai 扫头}\!\textgreater{}扫的是一个``对儿'',``闭门家中坐,祸从天上来'',因此这一指表示``祸从天上来''。}

\vspace{3pt}{\centerline{{[}{\hei 第三场}{]}}}\vspace{5pt}

{\setlength{\hangindent}{52pt}{(萧桂英\hspace{20pt}【{\akai 西皮散板}】老爹爹清晨出前去出首,)}\footnote{刘曾复先生为樊百乐{\scriptsize 君}说《审刺客》一戏时,顺便说了``萧恩挨打''部分的内容。} }

{(众\hspace{40pt}({\akai 内})一十!)}

{(萧桂英\hspace{20pt}【{\akai 西皮散板}】倒叫我桂英儿挂在心头。)}

{(众\hspace{40pt}({\akai 内})二十!)}

{(萧桂英\hspace{20pt}【{\akai 西皮散板}】将身儿来至在草堂门口,)}

{(众\hspace{40pt}({\akai 内})三十!)}

{(萧桂英\hspace{20pt}【{\akai 西皮散板}】只等得爹爹回细问根由。)}

{(众\hspace{40pt}({\akai 内})四十打完!)}

{(吕子秋\hspace{20pt}({\akai 内})赶下堂去!)}

{好贼!}

\setlength{\hangindent}{56pt}{【{\akai 西皮散板}】恼恨那吕子秋为官不正,仗势力欺压我受苦的良民呐。上堂来他那里一言不问,责打我四十板呐赶出了头门。}

\setlength{\hangindent}{56pt}{【{\akai 西皮散板}】我这里咬牙关忙往家奔,叫一声桂英儿你快来开门{\footnotesize 呐}。}

{为父上得堂去,那贼一言不问,将为父重责。}

{这还不算受屈。他教为父连夜过江与那老贼赔礼,那才算受屈呢。}

{哎呀------我恨不得飞过江去,我就杀$\cdots{}\cdots{}$}

{我要杀贼的满门,方消我心头之恨!}

{小小年纪,懂得什么?!不用多管,取为父衣帽、戒刀过来。}

{不用儿管,快些取来。}

{为父去也。}

{何事?}

{小小年纪,去之无益。}

{有此胆量?}

{快将儿的衣服、兵刃收拾好了。}

{壮胆量也是好的。}

{走哇$\cdots{}\cdots{}$}

{做什么?}

{这门么,关也罢,不关也罢呀。}

{又做什么?}

{唉!门都不要了,还要什么家具呀?}

{唉,不明白的冤家呀,呃$\cdots{}\cdots{}$({\hwfs 哭介})}

{不要啼哭,随为父的走哇。}

{儿啊,此番前去,教儿骂,儿就骂,教儿杀,儿就杀,不要害怕。}

{走哇。}

{儿啊,夜晚行船,比不得白日,儿要掌稳了舵!}

\setlength{\hangindent}{56pt}{【{\akai 西皮快板}】这件事不由我心头冒火,今夜晚过江去将他杀却。恨不得插双翅江河越过,}

\setlength{\hangindent}{56pt}{【{\akai 西皮散板}】我的儿因何故放了篷索。}

{杀人还有什么假的不成?}

{呀呸!为父在家中不教儿前来,儿是偏偏地要来。这船行半江之中,儿又要回去------也罢!待为父拨转船头,送儿回去。}

{\textless{}\!{\bfseries\akai 哭头}\!\textgreater{}啊,桂英,我的儿呀!}

{少时还在此处上船,儿记下了?}

{儿啊,庆顶珠\footnote{吴小如先生告,萧恩对萧桂英提及``庆顶珠''时,念白作``聘礼珠子'',``庆顶珠''留作戏名。}可在身旁?}

{此番前去,倘有不测,儿自带庆顶珠逃往花家去吧。}

{我么------儿就不用管了哇。}

{不要啼哭,随我走哇。}

{来此已是。}

{且慢。}

{收拾好了。}

{有人么?走出一个来呀。}

{过府赔罪来了。}

{哼!}

{请了。}

{我来问你:~这鱼税银子,可有圣上旨意?}

{户部公文?}

{凭着何来?}

{敢是那吕子秋?!}

{哼!}

\setlength{\hangindent}{56pt}{【{\akai 西皮摇板}】这税银我不纳是我的本分,你不该差人役打上我门。}

{儿啊,骂呀!}

{且慢,我父女有好心献上。}

{打鱼之时,得来一宗宝贝,名唤庆顶珠。}

{这$\cdots{}\cdots{}$耳目甚多。}

{看刀!}

{儿啊,随为父的杀呀!}

}
