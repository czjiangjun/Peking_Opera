\newpage
\phantomsection %实现目录的正确跳转
\section*{\large\hei {伐东吴}}
\addcontentsline{toc}{section}{\hei 伐东吴}

\hangafter=1                   %2. 设置从第1⾏之后开始悬挂缩进  %}
\setlength{\parindent}{0pt}{
\vspace{3pt}{\centerline{{[}{\hei 第一场}{]}}}\vspace{5pt}

\setlength{\hangindent}{52pt}{黄忠\hspace{30pt}({\akai 念})英雄回首忆长沙,百战威名成虎牙\footnote{据《三国志·蜀书》载:~``黄忠、赵云强挚壮猛,并作爪牙,其灌、滕之徒欤?'',据樊百乐{\scriptsize 君}告,``虎牙''是古代将军的名号,喻其猛锐。《汉书$\cdot$匈奴传上》:``云中太守田顺为虎牙将军,三万馀骑出五原。''《后汉书$\cdot$盖延传》:``光武即位,以延为虎牙将军。''。姜骏建议作``逞虎牙''似亦通。 陈超老师按:~ 《伐东吴》如果带``小桃园'',则是黄忠与吴班同上念此对儿。 \\陈超老师介绍带``小桃园''演法如下:~ 刘备{\hwfs 封}黄忠``以为随军副帅'',{\hwfs 封}吴班``前站先行'',{\hwfs 二}人{\hwfs 领旨下};~ 刘备{\hwfs 再传令}``哪位将军愿领副先锋?''关兴、张苞{\hwfs 争功},{\hwfs 比武}、{\hwfs 折箭后},{\hwfs 再接}黄忠``忆昔当年''。  一般演出都不带``小桃园``。}。}

\setlength{\hangindent}{52pt}{黄忠\hspace{30pt}圣上宣召。}

\setlength{\hangindent}{52pt}{黄忠\hspace{30pt}一同进帐。}

\setlength{\hangindent}{52pt}{黄忠\hspace{30pt}臣等见驾。}

\setlength{\hangindent}{52pt}{黄忠\hspace{30pt}愿吾皇万岁,(万岁,)万万岁。}

\setlength{\hangindent}{52pt}{黄忠\hspace{30pt}宣臣等进帐有何旨意?}

\setlength{\hangindent}{52pt}{黄忠\hspace{30pt}领旨。}

\vspace{3pt}{\centerline{{[}{\hei 第二场}{]}}}\vspace{5pt}

\setlength{\hangindent}{52pt}{刘备\hspace{30pt}【{\akai 西皮原板}】风吹旌旗山岳动,关兴、张苞出御营。未知此去可得胜。举首翘望心不宁。}

\setlength{\hangindent}{52pt}{黄忠\hspace{30pt}【{\akai 西皮原板}】忆昔当年长沙镇,算来不觉有数春({\akai 或}:~转眼不觉数十春)。荆襄、阆中遭不幸,一心要把东吴平({\akai 或}:~吾主爷要把东吴平)。黄汉升撩袍御营进,}

\setlength{\hangindent}{52pt}{刘备\hspace{30pt}【{\akai 西皮原板}】老将军免礼且平身。暂陪朕坐消愁闷,}

\setlength{\hangindent}{52pt}{黄忠\hspace{30pt}【{\akai 西皮原板}】行兵不必泪伤心({\akai 或}:~兴兵不必泪常涔)。}

\setlength{\hangindent}{52pt}{张苞\hspace{30pt}【{\akai 西皮摇板}】斩将擒贼破敌阵,}

\setlength{\hangindent}{52pt}{关兴\hspace{30pt}【{\akai 西皮摇板}】弟兄御前显奇能。}

\setlength{\hangindent}{52pt}{张苞\hspace{30pt}启禀皇伯,儿臣出阵,不料谭雄暗放雕翎,射死战马;幸得关兴赶到,不然性命难保。}

\setlength{\hangindent}{52pt}{关兴\hspace{30pt}儿臣见张苞兄长落马,赶到阵前,刀劈谢旌,活捉谭雄,特来交令。}

\setlength{\hangindent}{52pt}{刘备\hspace{30pt}快将谭雄绑了上来!}

\setlength{\hangindent}{52pt}{刘备\hspace{30pt}好吴狗!}

\setlength{\hangindent}{52pt}{刘备\hspace{30pt}【{\akai 西皮散板}】四百年来争汉鼎,东吴不君也不臣。鼠窃犬偷真堪恨,快斩逆贼立即行。}

\setlength{\hangindent}{52pt}{黄忠\hspace{30pt}号令辕门。}

\setlength{\hangindent}{52pt}{刘备\hspace{30pt}将这厮首级祭奠二千岁灵前;洒下热血,以祭死马。搭了下去。}

\setlength{\hangindent}{52pt}{众\hspace{40pt}啊------}

\setlength{\hangindent}{52pt}{刘备\hspace{30pt}朕今兵发东吴,与二位贤弟报仇,幸得二虎侄头阵取胜,惊破吴人之胆。}

\setlength{\hangindent}{52pt}{刘备\hspace{30pt}左右,看酒。与二位皇侄贺功。}

\setlength{\hangindent}{52pt}{内侍\hspace{30pt}是。}

\setlength{\hangindent}{52pt}{黄忠\hspace{30pt}嗯哼。(黄忠{\hwfs 痰嗽介})}

\setlength{\hangindent}{52pt}{刘备\hspace{30pt}哦------老将军你也来呀。}

\setlength{\hangindent}{52pt}{黄忠\hspace{30pt}臣领旨。}

\setlength{\hangindent}{52pt}{刘备\hspace{30pt}【{\akai 西皮原板}】庆贺功劳把酒饮,想起了当年破黄巾。战贼挽手威风凛,虎牢关前显威名。}

\setlength{\hangindent}{52pt}{刘备\hspace{30pt}想当年与尔父等桃园结义之后,破黄巾、得徐州、收襄阳、入西川,皆尔父等之力也,今他们一旦去世啊,所有当年之将,尽是些老迈无用。}

\setlength{\hangindent}{52pt}{刘备\hspace{30pt}幸有二皇侄斩将破敌,如此英勇,何愁东吴不平。}

\setlength{\hangindent}{52pt}{刘备\hspace{30pt}看酒来,朕亲为二皇侄贺功。}

\setlength{\hangindent}{52pt}{刘备\hspace{30pt}【{\akai 西皮摇板}】幸喜皇侄多英俊,此酒酬劳庆功勋。}

\setlength{\hangindent}{52pt}{黄忠\hspace{30pt}老了哇,老了哇!}

\setlength{\hangindent}{52pt}{黄忠\hspace{30pt}【{\akai 西皮散板}】主公说话不思忖,他道老将便无能。({\akai 或}:~主公说话欠思忖,怎知老将便无能。)}

\setlength{\hangindent}{52pt}{黄忠\hspace{30pt}且住,关兴、张苞子侄之辈,阵前擒来谭雄,无非是些许功劳。主公帐中({\akai 或}:~主公隆宠),夸了又夸,讲了又讲。反讲当年五虎上将尽是些老迈无用。这这这$\cdots{}\cdots{}$}

\setlength{\hangindent}{52pt}{黄忠\hspace{30pt}也罢!我不免({\akai 或}:~俺不免)去至两军阵前,斩那东吴八员上将,看看俺黄忠老是不老。}

\setlength{\hangindent}{52pt}{黄忠\hspace{30pt}【{\akai 西皮快板}】太公八十方交运,廉颇七旬挡秦军。黄忠年迈有本领,}

\setlength{\hangindent}{52pt}{黄忠\hspace{30pt}【{\akai 西皮摇板}】再学走马取定军。}

\setlength{\hangindent}{52pt}{报子\hspace{30pt}黄老将军私自出营,人向东而去。}

\setlength{\hangindent}{52pt}{刘备\hspace{30pt}快去打探。}

\setlength{\hangindent}{52pt}{探子\hspace{30pt}得令。}

\setlength{\hangindent}{52pt}{刘备\hspace{30pt}哎呀且住。黄汉升绝非叛逆之人,想是适才朕言老将无能,故而一怒出营,意在斩将显能尔。既然如此,诚恐有失。}

\setlength{\hangindent}{52pt}{刘备\hspace{30pt}关兴、张苞,}

%关兴\\张苞\raisebox{5pt}{\hspace{30pt}在}
\raisebox{0pt}[22pt][16pt]{\raisebox{8pt}{关兴}\raisebox{-8pt}{\hspace{-22pt}{张苞}}\raisebox{0pt}{\hspace{30pt}在,}}

\setlength{\hangindent}{52pt}{刘备\hspace{30pt}命你二人急速前去保护。倘若老将军得胜,劝他回营,不得有误。}

%关兴\\张苞\raisebox{5pt}{\hspace{30pt}得令。}
\raisebox{0pt}[22pt][16pt]{\raisebox{8pt}{关兴}\raisebox{-8pt}{\hspace{-22pt}{张苞}}\raisebox{0pt}{\hspace{30pt}得令。}}

\setlength{\hangindent}{52pt}{刘备\hspace{30pt}将宴撤去。}

\setlength{\hangindent}{52pt}{刘备\hspace{30pt}【{\akai 西皮摇板}】得意忘形错是朕,激怒老将黄汉升。但愿他马到成功呃早得胜,平安无事转回程。}

\vspace{3pt}{\centerline{{[}{\hei 第三场}{]}}}\vspace{5pt}

\setlength{\hangindent}{52pt}{黄忠\hspace{30pt}【{\akai 西皮导板}】黄忠马上呵呵笑,}

\setlength{\hangindent}{52pt}{黄忠\hspace{30pt}哈哈,哈哈,啊呵呵哈哈$\cdots{}\cdots{}$({\hwfs 笑介})}

\setlength{\hangindent}{52pt}{黄忠\hspace{30pt}【{\akai 西皮快板}】主公帐中论英豪。溺爱不明夸年少,反道老将无略韬。只要杀人胆量好,哪怕胡须似银条。催马来在阳关道,}

%关兴\\张苞\raisebox{5pt}{\hspace{30pt}老将军慢走。}
\raisebox{0pt}[22pt][16pt]{\raisebox{8pt}{关兴}\raisebox{-8pt}{\hspace{-22pt}{张苞}}\raisebox{0pt}{\hspace{30pt}老将军慢走。}}

\setlength{\hangindent}{52pt}{黄忠\hspace{30pt}【{\akai 西皮摇板}】二小将赶来为哪条。}

%关兴\\张苞\raisebox{5pt}{\hspace{30pt}老将军且慢。}
\raisebox{0pt}[22pt][16pt]{\raisebox{8pt}{关兴}\raisebox{-8pt}{\hspace{-22pt}{张苞}}\raisebox{0pt}{\hspace{30pt}老将军且慢。}}

\setlength{\hangindent}{52pt}{黄忠\hspace{30pt}二位小将赶来则甚?}

%关兴\\张苞\raisebox{5pt}{\hspace{30pt}(我等奉了)皇伯之命,请老将军回营,(诚恐)年迈有失。\footnote{此处据《京剧汇编》第一百零一集~马连良~藏本增补。}}
\raisebox{0pt}[22pt][16pt]{\raisebox{8pt}{关兴}\raisebox{-8pt}{\hspace{-22pt}{张苞}}\raisebox{0pt}{\hspace{30pt}(我等奉了)皇伯之命,请老将军回营,(诚恐)年迈有失。\footnote{此处据《京剧汇编》第一百零一集~马连良~藏本增补。}}}

\setlength{\hangindent}{52pt}{黄忠\hspace{30pt}呀呸!}

\setlength{\hangindent}{52pt}{黄忠\hspace{30pt}【{\akai 西皮快板}】二小将把话讲差了,讲什么阵前把命抛。我也不图凌烟标,恢复汉室锦皇朝。见了主公好言告,你就说年迈的黄忠要立功劳。}

\setlength{\hangindent}{52pt}{张苞\hspace{30pt}【{\akai 西皮摇板}】黄忠年迈性情傲,}

\setlength{\hangindent}{52pt}{关兴\hspace{30pt}【{\akai 西皮摇板}】(相随)保护莫辞劳。\footnote{此处据《京剧汇编》第一百零一集~马连良~藏本增补。}}

\vspace{3pt}{\centerline{{[}{\hei 第四场}{]}}}\vspace{5pt}

\setlength{\hangindent}{52pt}{吴班\hspace{30pt}【{\akai 西皮摇板}】大将出川把贼剿,挂印先行不辞劳。连营下寨恐非妙,见机而行稳重高。}

\setlength{\hangindent}{52pt}{报子\hspace{30pt}报!}

\setlength{\hangindent}{52pt}{报子\hspace{30pt}黄老将军到。}

\setlength{\hangindent}{52pt}{吴班\hspace{30pt}有请。}

\setlength{\hangindent}{52pt}{吴班\hspace{30pt}啊,老将军。}

\setlength{\hangindent}{52pt}{黄忠\hspace{30pt}哼!}

\setlength{\hangindent}{52pt}{吴班\hspace{30pt}黄老将军,怒气不息,为着何来?}

\setlength{\hangindent}{52pt}{黄忠\hspace{30pt}呃!}

\setlength{\hangindent}{52pt}{黄忠\hspace{30pt}\textless{}\!{\bfseries\akai 叫头}\!\textgreater{}吴将军,}

\setlength{\hangindent}{52pt}{黄忠\hspace{30pt}想那关兴、张苞乃子侄之辈,阵前擒来谭雄,无非是些许功劳。主公帐中({\akai 或}:~主公隆宠),夸了又夸,讲了又讲。反讲当年五虎上将尽是些老迈无用。你道恼是不恼?}

\setlength{\hangindent}{52pt}{吴班\hspace{30pt}哎,本来的老了哇。}

\setlength{\hangindent}{52pt}{黄忠\hspace{30pt}啊?({\akai 或}:~呀呸!)}

\setlength{\hangindent}{52pt}{黄忠\hspace{30pt}【{\akai 西皮摇板}】为什么人人道我老哇,}

\setlength{\hangindent}{52pt}{吴班\hspace{30pt}唉,本来是老了。}

\setlength{\hangindent}{52pt}{黄忠\hspace{30pt}呀呸!}

\setlength{\hangindent}{52pt}{黄忠\hspace{30pt}【{\akai 西皮快板}】不由怒气上眉梢。吾十岁({\akai 或}:~某十岁)弓马颇知晓,十三、十四使宝刀。交锋对垒有多少,数十年未离马鞍鞒。战长沙已然须发皓,取东川谁不道我是英豪。我也曾天荡、定军一齐扫,夏侯渊一命赴阴曹。到如今八十三岁何曾老,我是哪些儿老,}

\setlength{\hangindent}{52pt}{吴班\hspace{30pt}老将军本来是老了啊。}

\setlength{\hangindent}{52pt}{黄忠\hspace{30pt}呃!}

\setlength{\hangindent}{52pt}{黄忠\hspace{30pt}【{\akai 西皮快板}】年迈也要逞英豪。来来来与爷带马到,斩几个人头你瞧一瞧。}

\setlength{\hangindent}{52pt}{吴班\hspace{30pt}【{\akai 西皮摇板}】老将人老心不老,}

\setlength{\hangindent}{52pt}{吴班\hspace{30pt}带马。}

\setlength{\hangindent}{52pt}{吴班\hspace{30pt}【{\akai 西皮摇板}】暗地保护走一遭。}

\vspace{3pt}{\centerline{{[}{\hei 第五场}{]}}}\vspace{5pt}

崔禹\\史蹟\raisebox{5pt}{\hspace{30pt}俺,东吴大将------}
\raisebox{0pt}[22pt][16pt]{\raisebox{8pt}{崔禹}\raisebox{-8pt}{\hspace{-22pt}{史蹟}}\raisebox{0pt}{\hspace{30pt}俺,东吴大将------}}

\setlength{\hangindent}{52pt}{崔禹\hspace{30pt}崔禹。}

\setlength{\hangindent}{52pt}{史蹟\hspace{30pt}史蹟。}

\setlength{\hangindent}{52pt}{崔禹\hspace{30pt}我等奉了吴侯旨意,镇守猇亭。探子报道,黄忠前来讨战,你我二人前去会他一会。}

\setlength{\hangindent}{52pt}{史蹟\hspace{30pt}请。}

\setlength{\hangindent}{52pt}{黄忠\hspace{30pt}来将通名!}

\setlength{\hangindent}{52pt}{崔禹\hspace{30pt}哼,连你家老爷东吴大将崔禹全不认识?}

\setlength{\hangindent}{52pt}{黄忠\hspace{30pt}通上名来。}

\setlength{\hangindent}{52pt}{崔禹\hspace{30pt}某乃东吴大将崔禹。}

\setlength{\hangindent}{52pt}{史蹟\hspace{30pt}俺乃史蹟。哇呀呀呀$\cdots{}\cdots{}$你还不跑?}

\setlength{\hangindent}{52pt}{黄忠\hspace{30pt}诶呀!我道是东吴八员上将,原来是两个无名的狗头。}

%崔禹\\史蹟\raisebox{5pt}{\hspace{30pt}什么狗头,这是人头。}
\raisebox{0pt}[22pt][16pt]{\raisebox{8pt}{崔禹}\raisebox{-8pt}{\hspace{-22pt}{史蹟}}\raisebox{0pt}{\hspace{30pt}什么狗头,这是人头。}}

\setlength{\hangindent}{52pt}{黄忠\hspace{30pt}饶尔等不死,去罢!}

%崔禹\\史蹟\raisebox{5pt}{\hspace{30pt}什么人头、狗头的,你这老头儿叫什么名字?}
\raisebox{0pt}[22pt][16pt]{\raisebox{8pt}{崔禹}\raisebox{-8pt}{\hspace{-22pt}{史蹟}}\raisebox{0pt}{\hspace{30pt}什么人头、狗头的,你这老头儿叫什么名字?}}

\setlength{\hangindent}{52pt}{黄忠\hspace{30pt}老夫黄忠。}

\setlength{\hangindent}{52pt}{崔禹\hspace{30pt}咦咦咦,}

\setlength{\hangindent}{52pt}{史蹟\hspace{30pt}哇啊啊------}

%崔禹\\史蹟\raisebox{5pt}{\hspace{30pt}呵呵哈哈哈$\cdots{}\cdots{}$({\hwfs 笑介})}
\raisebox{0pt}[22pt][16pt]{\raisebox{8pt}{崔禹}\raisebox{-8pt}{\hspace{-22pt}{史蹟}}\raisebox{0pt}{\hspace{30pt}呵呵哈哈哈$\cdots{}\cdots{}$({\hwfs 笑介})}}

\setlength{\hangindent}{52pt}{黄忠\hspace{30pt}尔为何发笑?}

%崔禹\\史蹟\raisebox{5pt}{\hspace{30pt}呵呵黄忠啊,我道是天神下界,原来是一个老倭瓜。}
\raisebox{0pt}[22pt][16pt]{\raisebox{8pt}{崔禹}\raisebox{-8pt}{\hspace{-22pt}{史蹟}}\raisebox{0pt}{\hspace{30pt}呵呵黄忠啊,我道是天神下界,原来是一个老倭瓜。}}

\setlength{\hangindent}{52pt}{黄忠\hspace{30pt}休得胡言,快(快)教那潘璋出马,饶尔等不死。去罢!}

\setlength{\hangindent}{52pt}{崔禹\hspace{30pt}我这里待我耍个``提枪花'',摘就一个老倭瓜。}

%崔禹\\史蹟\raisebox{5pt}{\hspace{30pt}呵呵将军{\footnotesize 呐}。}
\raisebox{0pt}[22pt][16pt]{\raisebox{8pt}{崔禹}\raisebox{-8pt}{\hspace{-22pt}{史蹟}}\raisebox{0pt}{\hspace{30pt}呵呵将军{\footnotesize 呐}。}}

\setlength{\hangindent}{52pt}{(崔禹\hspace{30pt}人道黄忠乃是好将,未战两个回合,他为何败下阵去。)\footnote{此处据《京剧汇编》第一百零一集~马连良~藏本增补。}}

\setlength{\hangindent}{52pt}{史蹟\hspace{30pt}想是不忍杀害于你。}

\setlength{\hangindent}{52pt}{崔禹\hspace{30pt}哼,休得胡言,你我赶上前去。定然死在他手。}

\vspace{3pt}{\centerline{{[}{\hei 第六场}{]}}}\vspace{5pt}

\setlength{\hangindent}{52pt}{黄忠\hspace{30pt}啊?!}

\setlength{\hangindent}{52pt}{吴班\hspace{30pt}老将军刀劈崔禹、史蹟,就是莫大之功,可以回营交令了。}

\setlength{\hangindent}{52pt}{黄忠\hspace{30pt}俺要去至吴营,斩那东吴八员上将,看看我黄忠老是不老({\akai 或}:~俺要去至吴营,斩那东吴八员上将,方显我黄忠不老)。}

\setlength{\hangindent}{52pt}{吴班\hspace{30pt}唉,老将军呐,}

\setlength{\hangindent}{52pt}{吴班\hspace{30pt}【{\akai 西皮摇板}】老将军威风谁不晓,破敌须防战马劳。\footnote{此处据《京剧汇编》第一百零一集~马连良~藏本作``吴班有言来禀告,破敌须防战马劳。老将军威风谁不晓,何妨饶他这一遭。''}}

\setlength{\hangindent}{52pt}{黄忠\hspace{30pt}吴将军,}

\setlength{\hangindent}{52pt}{黄忠\hspace{30pt}【{\akai 西皮摇板}】这几句话儿讲得好,黄忠的怒气一半消。回营报功呃休取笑,暂且饶他这一宵。}

\setlength{\hangindent}{52pt}{吴班\hspace{30pt}老将军不如请暂回师。\footnote{此处刘曾复先生只念``暂回师'',据上下文增补。}}

\setlength{\hangindent}{52pt}{黄忠\hspace{30pt}(啊)吴将军,你我今日暂回大营({\akai 或}:~你我暂且回营),教那些吴狗们多活上一夜。}

\setlength{\hangindent}{52pt}{吴班\hspace{30pt}是啊,教他们多活上一夜。}

\setlength{\hangindent}{52pt}{黄忠\hspace{30pt}便宜了他们。}

\setlength{\hangindent}{52pt}{吴班\hspace{30pt}呃,便宜了他们。}

\setlength{\hangindent}{52pt}{黄忠\hspace{30pt}啊吴将军,你看我黄忠老是不老?}

\setlength{\hangindent}{52pt}{吴班\hspace{30pt}呃,将军么------嗯,不老。}

\setlength{\hangindent}{52pt}{黄忠\hspace{30pt}嗯,不老?}

\setlength{\hangindent}{52pt}{吴班\hspace{30pt}呃,不老。}

%黄忠\\吴班\raisebox{5pt}{\hspace{30pt}啊,呵呵哈哈哈$\cdots{}\cdots{}$({\hwfs 笑介})}
\raisebox{0pt}[22pt][16pt]{\raisebox{8pt}{黄忠}\raisebox{-8pt}{\hspace{-22pt}{吴班}}\raisebox{0pt}{\hspace{30pt}啊,呵呵哈哈哈$\cdots{}\cdots{}$({\hwfs 笑介})}}

\vspace{3pt}{\centerline{{[}{\hei 第七场}{]}}}\vspace{5pt}

\setlength{\hangindent}{52pt}{潘璋\hspace{30pt}【{\akai 西皮摇板}】探马不住急来报,黄忠斩我两英豪。\footnote{此处刘曾复先生唱的是``{$\cdots{}\cdots{}$}斩将论英豪'',似欠通,此处从《京剧汇编》第一百零一集~马连良~藏本。}}

\setlength{\hangindent}{52pt}{潘璋\hspace{30pt}俺,潘璋。前者同吕蒙定计袭取荆州\footnote{此处据《京剧汇编》第一百零一集~马连良~藏本增补。},我主大喜,将关羽刀、马赐俺,赤兔马不食草料而死;青龙刀虽在我手,却未斩一将。适才探子报道,黄忠踏营,岂肯容他张狂,待俺擒他便了。}

\setlength{\hangindent}{52pt}{黄忠\hspace{30pt}来将通名。}

\setlength{\hangindent}{52pt}{潘璋\hspace{30pt}东吴大将潘------}

\setlength{\hangindent}{52pt}{黄忠\hspace{30pt}潘什么?}

\setlength{\hangindent}{52pt}{潘璋\hspace{30pt}潘璋。}

\setlength{\hangindent}{52pt}{黄忠\hspace{30pt}啊!}

\setlength{\hangindent}{52pt}{黄忠\hspace{30pt}【{\akai 西皮快板}】一见潘璋把牙咬,手持青龙偃月刀。怎不教人珠泪掉,斩尔的狗头马后捎。}

\vspace{3pt}{\centerline{{[}{\hei 第八场}{]}}}\vspace{5pt}

\setlength{\hangindent}{52pt}{马忠\hspace{30pt}({\akai 念})旌旗飞龙影,干戈耀日月。\footnote{《京剧汇编》第一百零一集~马连良~藏本此处作``旌旗飞龙影,干戈耀日明''。}}

\setlength{\hangindent}{52pt}{马忠\hspace{30pt}俺,马忠。只因潘璋出营,大战黄忠,不知胜负如何,俺且出营一望。}

\setlength{\hangindent}{52pt}{马忠\hspace{30pt}将军胜负如何?}

\setlength{\hangindent}{52pt}{潘璋\hspace{30pt}黄忠十分骁勇,难以取胜。}

\setlength{\hangindent}{52pt}{马忠\hspace{30pt}将军且退后阵,待俺前去会他。}

\setlength{\hangindent}{52pt}{潘璋\hspace{30pt}多加小心。}

\vspace{3pt}{\centerline{{[}{\hei 第九场}{]}}}\vspace{5pt}

\setlength{\hangindent}{52pt}{潘璋\hspace{30pt}将军。}

\setlength{\hangindent}{52pt}{马忠\hspace{30pt}将军。}

\setlength{\hangindent}{52pt}{潘璋\hspace{30pt}你我被黄忠杀败,主公降罪如何是好?}

\setlength{\hangindent}{52pt}{马忠\hspace{30pt}黄忠虽然骁勇,潘将军你且与他交战,待俺暗放一箭。}

\setlength{\hangindent}{52pt}{潘璋\hspace{30pt}黄忠善射,百步穿杨,若是射他不中,只恐你``画虎不成反类其犬''。}

\setlength{\hangindent}{52pt}{马忠\hspace{30pt}岂不知``会家不防''?}

\setlength{\hangindent}{52pt}{潘璋\hspace{30pt}既然如此,待俺再会他一阵。}

\setlength{\hangindent}{52pt}{马忠\hspace{30pt}须要小心。}

\vspace{3pt}{\centerline{{[}{\hei 第十场}{]}}}\vspace{5pt}

\setlength{\hangindent}{52pt}{黄忠\hspace{30pt}【{\akai 西皮导板}】黄忠今日遭圈套,}

\setlength{\hangindent}{52pt}{黄忠\hspace{30pt}【{\akai 西皮快板}】中了奸贼计笼牢({\akai 或}:~谅我插翅也难逃)。}

\setlength{\hangindent}{52pt}{黄忠\hspace{30pt}【{\akai 西皮快板}】大将临危有神保,}

%关兴\\张苞\raisebox{5pt}{\hspace{30pt}【{\akai 西皮快板}】来了关兴和张苞。}
\raisebox{0pt}[22pt][16pt]{\raisebox{8pt}{关兴}\raisebox{-8pt}{\hspace{-22pt}{张苞}}\raisebox{0pt}{\hspace{30pt}【{\akai 西皮快板}】来了关兴和张苞。}}

\setlength{\hangindent}{52pt}{(潘璋\hspace{30pt}黄忠带箭,被二小将救出重围,你我速速追赶。)\footnote{此处据《京剧汇编》第一百零一集~马连良~藏本增补。}}

\setlength{\hangindent}{52pt}{马忠\hspace{30pt}赶上前去。}

\vspace{3pt}{\centerline{{[}{\hei 第十一场}{]}}}\vspace{5pt}

\setlength{\hangindent}{52pt}{刘备\hspace{30pt}【{\akai 西皮摇板}】黄忠性傲见识浅,不该匹马去争先。张苞、关兴料难劝,但愿平安得胜还。}

\setlength{\hangindent}{52pt}{黄忠\hspace{30pt}({\akai 念})\textless{}\!{\bfseries\akai 金钱花}\!\textgreater{}渭城朝雨清尘、清尘;轮台古月黄云、黄云。催花羯鼓去从军。枕头上,别情人;刀头上,做功臣。}

\setlength{\hangindent}{52pt}{刘备\hspace{30pt}【{\akai 西皮散板}】一见老将身带箭,霎时胆落百丈渊。早知出兵遭凶呃险,}

\setlength{\hangindent}{52pt}{刘备\hspace{30pt}\textless{}\!{\bfseries\akai 哭头}\!\textgreater{}将军呐------}

\setlength{\hangindent}{52pt}{刘备\hspace{30pt}【{\akai 西皮摇板}】朕悔一时错出言。}

\setlength{\hangindent}{52pt}{黄忠\hspace{30pt}【{\akai 西皮散板}】精神恍惚四肢软,耳旁又听有人言。大骂潘璋休弄呃险,}

\setlength{\hangindent}{52pt}{刘备\hspace{30pt}老将军!}

\setlength{\hangindent}{52pt}{黄忠\hspace{30pt}【{\akai 西皮散板}】只见主公在眼前。急忙叩谢龙恩典,黄忠的性命难保全。}

\setlength{\hangindent}{52pt}{刘备\hspace{30pt}唉呀老将军呐,朕一言之错,使你怒出大营,如今带箭而归,教朕痛断肝肠了哇啊$\cdots{}\cdots{}$}

\setlength{\hangindent}{52pt}{黄忠\hspace{30pt}哎呀主公啊,老臣出马,刀劈史蹟、崔禹------}

\setlength{\hangindent}{52pt}{刘备\hspace{30pt}就该回营。}

\setlength{\hangindent}{52pt}{黄忠\hspace{30pt}因见吴狗潘璋手持二君侯青龙宝刀,老臣一见,肝胆俱裂。正要擒贼下马,不想贼营暗放冷箭,中臣肩窝。}

\setlength{\hangindent}{52pt}{刘备\hspace{30pt}啊------}

\setlength{\hangindent}{52pt}{刘备\hspace{30pt}啊,老将军乃是善射的能手,为何不防?}

\setlength{\hangindent}{52pt}{黄忠\hspace{30pt}哎呀陛下呀。}

\setlength{\hangindent}{52pt}{黄忠\hspace{30pt}【{\akai 西皮散板}】老臣智不如王翦,临阵怎敢不当先。况且仇人两相见,心急哪顾听弓弦({\akai 或}:~哪有闲心听弓弦)。}

\setlength{\hangindent}{52pt}{黄忠\hspace{30pt}此乃老臣自不小心。}

\setlength{\hangindent}{52pt}{刘备\hspace{30pt}【{\akai 西皮散板}】真是风云不测变,空将血泪洒胸前。回头便把小将怨:~年轻无知小儿男。}

%关兴\\张苞\raisebox{5pt}{\hspace{30pt}儿臣等知罪。}
\raisebox{0pt}[22pt][16pt]{\raisebox{8pt}{关兴}\raisebox{-8pt}{\hspace{-22pt}{张苞}}\raisebox{0pt}{\hspace{30pt}儿臣等知罪。}}

\setlength{\hangindent}{52pt}{黄忠\hspace{30pt}陛下,此乃臣自不小心,休要埋怨二位小将军。}

\setlength{\hangindent}{52pt}{刘备\hspace{30pt}既然如此,待朕与老将军起箭。}

\setlength{\hangindent}{52pt}{黄忠\hspace{30pt}哎呀万岁呀,这箭上有药,箭在------臣在,这箭去------臣亡。}

\setlength{\hangindent}{52pt}{刘备\hspace{30pt}老将军带箭不起,那敢是怕痛?\footnote{此句刘曾复先生录音不清楚,据文意添加。存疑。}}

\setlength{\hangindent}{52pt}{黄忠\hspace{30pt}老臣死且不惧,焉能畏痛?一言永别,伏乞圣听:~}

\setlength{\hangindent}{52pt}{黄忠\hspace{30pt}【{\akai 反西皮二六}】平生今洒泪几点,回首功名八十年。主上待臣恩非浅,粉身碎骨理当然。幸得全尸已无怨({\akai 或}:~幸得全身已无怨),叩谢圣恩归九泉。万岁须当谋虑远\footnote{陈超老师介绍,他跟刘曾复先生学的此句``谋虑远''唱,``平吴不及定中原''一句【{\akai 散板}】。}({\akai 或}:~主上须当韬略远;主上须当\textless{}\!{\bfseries\akai 哭头}\!\textgreater{}谋略远),}

\setlength{\hangindent}{52pt}{黄忠\hspace{30pt}【{\akai 西皮散板}】平吴不及定中原。}

\setlength{\hangindent}{52pt}{刘备\hspace{30pt}【{\akai 西皮散板}】老将军休得心惊战,起箭医疗早愈痊。康复之后功臣宴,愿你康宁寿百年。}

\setlength{\hangindent}{52pt}{黄忠\hspace{30pt}【{\akai 西皮散板}】见主公说话({\akai 或}:~见主公只哭得)泪满面,关兴、张苞哭两边。大丈夫一死终难免,强打精神假流连。}

\setlength{\hangindent}{52pt}{刘备\hspace{30pt}【{\akai 西皮散板}】事到临头难挽转,张苞、关兴听朕言:~}

\setlength{\hangindent}{52pt}{刘备\hspace{30pt}关兴、张苞,搀扶老将军,待朕与老将军起箭。}

\setlength{\hangindent}{52pt}{黄忠\hspace{30pt}且慢呐,大将取箭,不用人搀,待老臣自取。}

\setlength{\hangindent}{52pt}{黄忠\hspace{30pt}闪开了!}

\setlength{\hangindent}{52pt}{黄忠\hspace{30pt}唉呀------}

\setlength{\hangindent}{52pt}{刘备\hspace{30pt}【{\akai 西皮散板}】一见老将归九天,冷水浇头落空潭。从今何处再相见,\footnote{此处至本剧结尾,刘曾复先生只是大致示范,只能听清个别词句。因此剧本中词句据《京剧汇编》第一百零一集~马连良~藏本增补。}}

\setlength{\hangindent}{52pt}{刘备\hspace{30pt}\textless{}\!{\bfseries\akai 哭头}\!\textgreater{}老将军{\footnotesize 呐}------}

\setlength{\hangindent}{52pt}{刘备\hspace{30pt}【{\akai 西皮散板}】热泪行行洒征衫。}

\setlength{\hangindent}{52pt}{张苞\hspace{30pt}【{\akai 西皮散板}】大将尸全世少见,}

\setlength{\hangindent}{52pt}{关兴\hspace{30pt}【{\akai 西皮散板}】皇伯不必损龙颜。}

\setlength{\hangindent}{52pt}{张苞\hspace{30pt}【{\akai 西皮散板}】尸首后帐好收殓,}

\setlength{\hangindent}{52pt}{关兴\hspace{30pt}【{\akai 西皮散板}】准备灭吴报仇冤。}

\setlength{\hangindent}{52pt}{刘备\hspace{30pt}【{\akai 西皮散板}】五虎大将三不见{\footnotesize 呐},}

\setlength{\hangindent}{52pt}{(刘备\hspace{30pt}\textless{}\!{\bfseries\akai 三叫头}\!\textgreater{}汉升!二弟,三弟呀!)}

\setlength{\hangindent}{52pt}{刘备\hspace{30pt}【{\akai 西皮散板}】休想古城再团圆。黄忠有灵当应显,踏平东吴在眼前。}

\setlength{\hangindent}{52pt}{刘备\hspace{30pt}【{\akai 西皮散板}】张苞、关兴传令箭,}

\setlength{\hangindent}{52pt}{刘备\hspace{30pt}拿潘璋------}

\setlength{\hangindent}{52pt}{刘备\hspace{30pt}【{\akai 西皮散板}】刀出鞘来弓上弦。}

}
\vspace{15pt}
$\ast${\bfseries\hwfs 王荣山教《定军山》、《阳平关》和《伐东吴》大刀把子}

王荣山说《定军山》、《阳平关》和《伐东吴》三出黄忠戏戏情不同,大刀把子不同,很明显的是《阳平关》有大战,打挡棒攒,其它两出没有,实际上许多处都不一样。

《定军山》有个小``三股档'',用在黄忠见韩浩、夏侯尚那一场。这场是韩、夏侯正式奉令出马,黄也是正式迎战交锋。黄忠抖擞精神把韩和夏侯杀得大败而逃。这场开打不能大又不能小,太大显不出韩、夏侯弱,太小显不出黄忠勇。用这个``三股档''黄把韩和夏侯拨拉过来,拨拉过去,两合就连削带抓他们打下去,紧接着黄来一个大刀花下场,表示奋勇追击。

《阳平关》黄忠见张郃、杜袭一场,也是三个人,但情节与《定军山》黄见韩、夏侯不同。张、杜是曹营名将,非韩、夏侯可及,黄忠夜半劫营,他们仓猝迎战,溃散之下,投奔曹兵主力开始大战,这场开打如果用《定军山》的三股档就不太合适,可以打``硬三枪''头子或其它套子。

《伐东吴》是黄忠一肚子气,拼了老命要斩东吴八员上将,潘璋虽勇但招架不住,因此潘、黄开打不要多,但黄要耍三个下场以示其奋勇冲杀深入重围。

《定军山》三股档(谭派打法)

{\hei 刘砚芳介绍}:~

韩浩、夏侯尚在大边台口,韩里,夏侯外,黄上场门上,一指,向大边台口漫夏侯头,夏侯过小边外边,黄用刀鐏勾韩腰到大边外边(即一肘),再从下场门向小边外边漫夏侯头,夏侯又过大边,黄到小边,回身与韩穿肚左转身回来打夏侯鼻子,右转身削韩头,打夏侯靠旗,亮,接大刀花下场。

《阳平关》见张郃、杜袭硬三枪头子({\bfseries\hwfs 王凤卿打法}):~

黄众人搭轿上到台口站齐,黄上时左手抱刀右手扶刀,到台口刀交右手平出刀,举左手数更,({\akai 白})``放起火来'',往里一砍,张著上手下,下手张、杜两边抄过合,黄从中间到小边,张郃留在大边,余者分下,一扯,两扯,剜萝卜黄到大边,拉转身,一枪、两枪、三枪,打张鼻子,转到里面打腰封,接背躬。向外漫张头,向里两穿,向外两盖,打张鼻子,用鐏勾杜袭上,从中间向外起大刀花在台中间被压住,搅起来,用鐏勾杜腰过小边(即一肘),黄归大边,在二人中间一合到小边,拉肚转身,鼻子,削头,抓靠旗,看左拳,看马后,左手指,耍下场。看拳和马后是看擒着人没有,指是指张、杜跑了,耍下场是猛追。

{\bfseries\hwfs 王荣山说这一场也可以少打些},打法是:~

剜萝卜黄归大边后,张刺黄一压,打张腰封,勾杜上,一压,搅起来,用鐏勾杜腰过小边(即一肘),黄归大边,中间过合,穿肚往里转,鼻子,削头,抓靠旗,亮,耍下场下。

《阳平关》挡棒攒头子({\bfseries\hwfs 王凤卿打法}):~

第一场曹操上,唱,上桌唱毕,黄上场门上出刀被徐晃漫头过小边,一滑,打上下左右,鐏一盖,打后蓬头。拉转身,搕,回花转身,大刀花转身搕,下叉,从里面漫头过去到大边,打鼻子,勾王平上,下接打棒攒,亮下场门下。

第二场曹又唱毕,黄下场门上,出刀,被徐晃漫头,搕反抄勾王平上,下接反攒,亮上场门下。

第三场,曹唱毕,黄内导(/{\textcolor{red}{倒})板,边唱上场门上,一亮,一趱子,台口刀头朝后洒,拨拉众上,到小边,一、二大扯,推到中间架住,唱。

{\bfseries\hwfs 王荣山说前边不打硬三枪},{\bfseries\hwfs 则接攒第一场可用硬三枪},打法是:~

黄上场门上,晃漫头上黄归小边,一兜往里转身,一枪、两枪、三枪,绕刀鐏一盖前蓬头,一盖后蓬头,拉肚转身,漫头过去到大边,晃刺,黄压打腰封,左转身勾王平上,下略。

二场是黄下场门上,晃漫头,一勾打晃腰封,反勾王平上,下略。

三场,唱上一亮,一趱子,三甩胡,出刀拨拉众上,领起来由大边到小边,一指,左转身向里一合,向外两合,向里架住,唱。

《伐东吴》三个大刀花下场(王荣山演此戏用):~

{\hwfs 第一个}:~即常用的大刀花下场,出刀,三个正花转身过大边,串腕回花转身,正花转身,大刀花,劈马,正花转身,面外,亮住,串腕转身,出刀,转身背刀,弓箭步,向里亮下。

{\hwfs 第二个}:~开始同第一个,到亮住,串腕转身,出刀,用右手背垫刀反手接刀,脸前绕大圈,平着往左齐腰横砍,撤右脚,向外正面正花转向大边,亮住,串腕右转身,向外正面出刀交左手反蹦子转身弓箭步左手握刀杆中间向外亮,下。

{\hwfs 第三个}:~开始同第一个,到劈马,再耍三个正花转身到小边,大刀花劈马,正花转身到大边,面向小边左右两涮刀,右转身右手持刀撕开,刀上膀子左转身,刀交左手串腕,在大边弓箭步左手握刀杆中间向外亮,大绕下场门下。

{\bfseries\hwfs 陈超老师说明}:~

刘曾复先生传授的贾洪林配演刘备的词很讲究,而且很重要,兹举两例:~

刘备{\akai 念}``昔年随朕开基创业之将,死的死了,亡的亡了'',而黄忠会错意,认为``开基创业之将''就是``五虎上将''。刘备没有针对,因此不念``昔年五虎上将,死的死了,亡的亡了''也不至于失言至此。

刘备见谭雄不{\akai }唱``孤与孙权冤仇深$\cdots{}\cdots{}$''而是``四百年来争汉鼎,东吴不君也不臣。''

一句道出了刘备伐吴的真正原因,以报仇为借口,灭吴统一。

潘璋是东吴八员上将,黄忠再勇也不至于一个``扫头''就落荒而逃。因此有些演法是黄忠、潘璋对刀。

谭鑫培、余叔岩认为不能对刀的原因是:~孙权将青龙刀赏赐潘璋,而潘璋并不会使青龙偃月刀。特别值得一提的是,钱金福为潘璋设计的开打,总是用刀鐏杵,不用刀头砍。加之黄忠勇武,因此一个\textless{}\!{\bfseries\akai 扫头}\!\textgreater{},潘璋落花流水。

