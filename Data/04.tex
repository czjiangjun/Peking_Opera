\addcontentsline{toc}{section}{\hfill[\hei 隋·唐]\hfill}
\newpage
\chead{隋·唐} % 页眉中间位置内容
\textbf{当锏卖马}\protect\hyperlink{fn307}{\textsuperscript{307}}

{[}第一场{]}

王老好 (内)啊哈!

(\textless{}\textbf{小锣打上}\textgreater{})

王老好
【数板】不赊不欠不算店,赊赊欠欠不见面。他在前街走,我在后街转。二人见了面,他说不方便。改日再见,改日再见。(白)开的是店,卖的是饭。一个人吃半斤,三人\protect\hyperlink{fn308}{\textsuperscript{308}}吃斤半。我王老好。怎么叫这个名儿呢,我在这天堂地面,开了一座小小的店房,有那南来北往的客人,有钱儿没钱儿的吃了就走,人们就给我送了个绰号,叫王老好,这且不言。前些日子,我这店里来了个山东好汉,名叫秦琼,住了这么多天啦,一个大钱儿也没给,人的饭,马的料,我能老垫着吗,今儿个把他请出来,商量商量,就是这个主意。

(站,向上场门)

王老好 二爷起来了没有,请出来凉快凉快罢。

秦琼 (内)嗯喷。

(小锣上,揉手揉眼,伸懒腰,到小边台口\textless{}\textbf{哭相思}尾\textgreater{})好汉英雄困天堂,不知何日回故乡。

(指,转身望王老好)

王老好 二爷。

(秦琼拱手,让)

秦琼 店主东,请到里面,请坐。

(进门,中间小坐,王老好大边旁坐)

秦琼
(啊,)店主东,将你二爷请了出来,是饮酒哇(或:还是吃酒哇),还是吃饭呐?

(左手做拿杯,双手做吃饭状)

王老好
出来就是吃喝,二爷酒也要喝,饭也要用,今儿小店家有两句话,不知当讲不当讲?

秦琼 店主东有话请讲当面。

(撩鸾带、左腿搭右腿上,鸾带搭在左腿上、左手下,右手放在左手上,一块儿放在膝上,望王老好)

王老好
没别的,二爷您在我这儿的店里,日子可也不少啦,人的饭食,马的草料,我有点儿垫不起啦。

(秦琼做睡状)

王老好 你瞧,着啦,二爷你醒醒,别睡。

秦琼 你讲你的。

(睡着说)

王老好 那么您呐。

秦琼 我哇,我睡我的呀。

(仍睡状)

王老好 那我说给谁听呐?

秦琼 你讲我听得见呐。

王老好 二爷您还是别睡啦。

秦琼 好,我不睡就是。

(醒听王老好讲,放下腿)

王老好
我说二爷,您在我这儿店里日子可也不少啦,人的饭食,马的草料,我可有点垫不起啦,您是有银子有钱,拿出来,我好垫补着花。

秦琼 听你之言是要钱呐?

王老好 不敢说要,跟您借俩钱使唤使唤。

秦琼 (店主东,)我进店的时节,(我)也曾对你讲过哇。

王老好 真格的,您说什么来着?

秦琼
只因我押解了一十八名江洋大盗,(天气炎热,)损伤一名。那蔡大老爷不与我批票回文,故而我在此等候(或:故而被困在此)。待等批票发下,那时节再算还你的店钱不迟,(或:待等那蔡大老爷发下路费银两,算还你的饭钱也还不迟,)哦,你何必这样着急呀。

(略转右,不听状)

王老好 哦,是我着急,比方这么说吧 ,这批票回文要是一个月发不下来?

秦琼 你就等上他一个月。

(看一下王老好,再转过来)

王老好 一个月三十天不多,好等,那么一年不下来呢?

秦琼 何妨等上他一年呐。

(身子放右转)

王老好 哦,要是一辈子发不下来呢?

秦琼 (哦,这一辈子么,)那就算你倒了运了。

(右转身,右腿架左腿上,右臂架椅背上,不理状)

王老好 嘿,您别着急,我再跟您说,人吃五谷杂粮,
可没有不生病的,瞧您这样儿,要是您死在我这个店里呢。

秦琼 怎么?

(站椅右侧)

秦琼
(我)若是死在你的店(中)么,哈哈哈\ldots{}\ldots{}那你就大大的发了财了。

(坐下)

王老好 哦哦哦,我这财是怎么个发法呢?

秦琼
等你二爷死后,你必须买上寿衣、寿帽,大大的一口棺木(或:棺材),将你二爷成殓起来,(比势)灵前立一牌位,上写山东好汉秦琼之位。(王应)(啊,)店主东那时节你可就不要这样打扮了。

王老好
对,我发啦财\protect\hyperlink{fn309}{\textsuperscript{309}},得捯饬捯饬。

秦琼
你必须头戴麻冠(王老好应),身穿重孝(王老好应),手拿哭丧棒,再与你二爷摔丧盆子,(拿棒状,摔盆状,伸右手,切手,点两下)然后你再大大的请上(他)一个份子,岂不就发了财了么?(双手比势)

王老好 啊哟,照你这么一说,我不就成了你的儿子了吗?

秦琼
哎呀呀,(我不敢呐,)我没有那样的造化呀。(立椅右侧,双手摇,托胡子向王老好扔,坐)

王老好
好哇,你吃饭不给钱,还转着弯儿骂人,你身上没穿树叶儿,我今儿要剥你。

秦琼 怎么,你要剥(上口)你二爷?(哼,量你也不敢呐。)

王老好
上口也得剥你,说剥就剥。(王上前剥状,秦琼绕王老好右手抓着一拧,往后拉住一按,王转身弯腰叫,秦推王手推开王,王起甩胳膊揉)

王老好 瞧这瘦样,怎么这么大劲,不信我打不过,我呀,喊叫臊你的皮。

秦琼 (怎么,你要喊叫?)任凭你去喊叫。

王老好
说叫就叫,街坊、邻舍,我这来了个山东好汉秦琼,白吃不给钱,还要打人。

(王老好出门,向上场门叫,回身,秦琼摊右手,出门,右手堵王嘴,放手)

王老好 你要堵死我哇?

秦琼 我有策划呀。

王老好 要钱不给,还要拆毁。

秦琼 就是商量商量呐,进来进来。

(秦琼进门归中间,王老好跟进归大边,立)

秦琼 槽头之上,二爷的黄骠马,牵到市上卖了银钱,(算)还你的店饭钱就是。

王老好 就是那匹马呀,瘦得成马灯啦,没人儿要!

秦琼 你是不懂呐,这货哇要与识家呀。(右手在上边一画,再指眼)

王老好 对,我外行。

秦琼 店主东。

王老好 怎么着?

秦琼 牵马呀\ldots{}\ldots{}

王老好 嘿,别哭哇。

(王老好过小边拉马回大边)

秦琼 【西皮慢板】(王老好唱中夹白)店主东带过了黄骠马(王老好
\textbf{给您带过来了}),不由得秦叔宝两泪如麻(王老好
\textbf{您怎么哭啦})。提起了此马来头大(王老好
\textbf{怎么个来头呢}),兵部堂皇甫爷相赠与咱(王老好
\textbf{不该卖呐})。遭不幸困至在天堂下,还你的店饭钱无奈何只得来卖它。摆一摆手儿你就牵去了吧,(秦唱中比势黄骠马,握右拳牵马状,``泪如麻''右手擦眼泪,``来头大''抬右大指,``与咱''轻拍右腹部,``卖它''左手捋胡子右手指,``去了吧''右手两画圈挥手叫王牵马走,下场门下,秦跟出门望下场门,回身进门,转身面外接唱末句)但不知此黄骠落于谁家(或:但不知此马落在谁家)。

(秦琼上步转身走下)

{[}第二场{]}

(\textless{}\textbf{抽头}\textgreater{}接\textless{}\textbf{快长锤}\textgreater{},四青袍引单雄信上,站九龙口唱)

单雄信 【西皮摇板】自幼闯荡江湖下,

(众归正)

单雄信 (接唱)【西皮摇板】人人道来是豪家。闲来无事大街耍,

(单雄信众圆场,王老好牵马从外边抄回去到大边台口牵马亮,单转到台小边台口,上步背扇一望,王牵马下,单归中间)

单雄信 (接唱)【西皮摇板】只见黄骠人爱煞。人来与爷忙赶下,

(众下,单雄信到下场门边回身)

单雄信
(接唱)【西皮摇板】\protect\hyperlink{fn310}{\textsuperscript{310}}不知此马是谁家?

(单雄信下)

{[}第三场{]}

(轻\textless{}\textbf{快长锤}\textgreater{}秦琼上,九龙口望,叹,到台口,\textless{}\textbf{闪锤}\textgreater{}唱)

秦琼
【西皮摇板】店主东卖黄骠不见回转,倒叫我(或:好教我)秦叔宝两眼望穿。

(坐中间小座,王急上大边栓马进门)

王老好 马给你牵回来啦,一根马毛也没短。买马的在后头,有话你们说罢。

(王老好急下,秦摊手、立,出店门立大边。单雄信众上站门,单立小边,一青袍解马牵马,单望念,\textless{}\textbf{小锣二三锣}\textgreater{})

单雄信
(念)此马是黄骠,四蹄似雪飘。浑身发金色,遍体无杂毛。胜似南山豹,亚塞浪里蛟。

\begin{quote}
好马呀,好马!
\end{quote}

(秦右手比画一挥,单雄信望秦)

秦琼 啊,连夸好马,敢是有爱马之意?

单雄信 好马人人皆爱,只是膘头忒瘦了。

秦琼 草料不佳之故。此处不是讲话之所,请到里面(或:店房一叙)。

单雄信 请呐。

(单雄信、秦琼挖门进店,众跟进站门,单大边秦小边八字。里边坐,单望秦)

单雄信 听兄台讲话不像此地人氏。

秦琼 本不是此地人氏。

单雄信 哪里人氏?

秦琼 山东历城县人氏。

单雄信 山东历城县,弟有一家好友,兄台可知。

秦琼 有名便知,无名不晓。

单雄信 提起此人大大有名。

秦琼 但不知是哪一家。

单雄信 此人姓秦名琼字叔宝。

秦琼 秦琼,此人落魄潦倒哇(或:唉,此人落魄得紧呐)。

单雄信 何出此言?

秦琼 这,愚下(或:在下)就是秦琼。

(单雄信、秦琼立,单托秦双手架住)

单雄信 你是秦二哥?

秦琼 不敢。(或:岂敢。)

单雄信 叔宝。

秦琼 正是。(或:不敢。)

单雄信 哈哈哈\ldots{}\ldots{}请来上坐。

(单雄信让秦琼,秦过大边,单过小边,同坐下)

(秦琼 哈哈哈\ldots{}\ldots{})

秦琼 听兄台讲话也不像此地人氏。

单雄信 本不是此地人氏。

秦琼 哪里人氏?

单雄信 河南二贤庄人氏。

秦琼 (河南)二贤庄弟有一(位)好友,兄台可知。

单雄信 有名便知,无名不晓。

秦琼 提起此人大大有名。

单雄信 但不知是哪一家。

秦琼 姓单名通字雄信。

单雄信 小弟就是单通。

(秦琼、单雄信立架住,秦望单)

秦琼 你(就)是单通?

单雄信 正是。

秦琼 单员外。

单雄信 不敢。

(秦琼让坐)

秦琼
请来上坐,哈哈哈\ldots{}\ldots{}(或:啊哈哈哈\ldots{}\ldots{}请来上坐。)

(秦琼单雄信换座,秦掸座、单拦,坐)

单雄信 二哥为何这等模样?

秦琼
只因愚兄押解一十八名江\ldots{}\ldots{}俱是我们绿林中的朋友哇,天气炎热,中途路上,损伤一名。那蔡大老爷不与我批票回文,故而被困在此。

单雄信 这有何难,待小弟拿我名帖讨来就是。

秦琼 有劳贤弟。

单雄信 前者伯母寿诞之期,小弟有一份薄礼可曾收到。

秦琼 但不知打从哪道而去。

单雄信 打从那黑\ldots{}\ldots{}

秦琼 收到了,当面谢过。(或:哦,当面谢过。)

(\textless{}\textbf{冲头}\textgreater{}家院上,进门站大边)

家院 启员外:大事不好了。

单雄信 何事惊慌?

家院 大员外在临潼山被李渊一箭射死。

单雄信 不好了!

(\textless{}\textbf{急三枪}\textgreater{}单雄信擦泪,秦琼摊手)

秦琼 天气炎热,就该搬灵的才是呀(或:就该搬尸的才是呀)。

单雄信 怎奈无有乘骑。

家院 这儿不有匹马吗?

单雄信 秦二爷的马焉能乘骑。

秦琼
呵贤弟,(有道是:)乘肥马,衣轻裘,与朋友共,敝之而无憾\protect\hyperlink{fn311}{\textsuperscript{311}}呐。哈哈哈\ldots{}\ldots{}

单雄信 怎么,骑得的。

秦琼 骑得的。

单雄信 带马。

(王老好暗上,\textless{}急三枪\textgreater{}单雄信秦琼出门,单上马,秦小边,单回身)

单雄信
\textless{}\textbf{叫头}\textgreater{}二哥!小弟此去多则半月,少则十天,将马匹送回,请。

(单雄信\textless{}\textbf{抽头}\textgreater{}下,秦琼过大边望)

秦琼
(啊,)贤弟慢走,(恕)愚兄不能远送了。(啊,哈哈哈\ldots{}\ldots{}(笑介))

(秦琼看王老好)

秦琼 这才是我的好朋友。(这才是我的好朋友。)

(秦琼拍右腹,伸右手大指)

王老好 这才是好桐油。

秦琼
哦,分明是好朋友,怎么说是好桐油哇。(或:呃,分明是好朋友,什么好桐油哇)

(挥右手,探手)

王老好
好朋友?我们这里,他是响马头儿。这儿三岁小孩子都认得他,他把你的马给拐走啦。

秦琼
(单通单二员外呀,)就是他。(秦琼右手弹胡子,左手捋胡子,抬左腿,左转身向下场门,右手指,弓箭步矮相)

王老好 不是他还是我?

(秦琼右手捋胡子,右转身回来,抬手切手)

秦琼
唉!【西皮摇板】骂一声秦琼瞎了眼呐,把响马当作好宾朋。我拉住了(或:我抓住了)店家撒一个赖呀。

(背供指,抓王老好领)

秦琼
好店家,你勾结响马把我的马骗了去了,快快还我的马便罢,不然呐,我就要你的老命呐。

王老好 你先等等,咱们讲个理儿。

秦琼 讲。

王老好 我问问你,你的马在槽头上拴着,是前门撬了锁啦?

秦琼 不曾。

王老好 后门挖了窟窿啦?

秦琼 也不曾。

王老好
这不结了吗。你把你的马送给你的好朋友啦,跑我这儿撒赖,别不害臊啦。(扔秦琼胳膊,秦臂画一圈垂右侧,望王老好)

秦琼
唉!(接唱)【西皮摇板】如此说我和你呀(就)两丢开。(右拳击左掌,双手分开摊,手把王老好碰倒,秦琼进门正小坐,王起来,王进门站大边)

王老好 摔着啦,两丢开两丢开,还得拿钱来。

秦琼 (唉,)还是没有钱呐。

王老好 你没钱,我照方抓药,我还给你喊叫去。

秦琼 任凭你去喊叫。

(王老好出门)

王老好 街坊、邻舍:我这儿\ldots{}\ldots{}

(王老好先向下场门,回身,秦琼出门,秦出门挡王,王往后一退)

王老好 我早预备啦。

秦琼 (呃,)我还有策划呀,进来进来。

王老好 那咱们就进去,你说罢。

(二人进门,秦琼中、王老好大边站,小边架锏)

秦琼 兵刃架上劈抡双锏,拿到市上卖了银钱,(算)还你的饭钱。

王老好 就那俩家伙儿,当通条嫌短,当火筷子太长,没人儿要。

秦琼 呵,(有道是)货卖与识家呀。

王老好 还卖呐,马都让人给识了去啦。

秦琼 惭愧。

王老好 蝉蜕?卖给药铺啦。

秦琼 你且取来。

王老好 好。(王老好过小边,拿锏拿不动)

王老好 回二爷的话,它拿我不动。

秦琼 敢是你拿它不动罢。(或:你敢是拿它不动罢。)

王老好 有那么点儿。

秦琼 闪开了。

(左右两枕,紧大带,过小边右手拿锏,交左手抱,右手山膀亮住)

王老好 我短这两手。

秦琼 【西皮摇板】(王夹白)家住山东历城县(王老好
好地方),秦琼的名儿天下传。我本是顶天立地男儿汉(王老好
拿钱来),好汉无钱到处难(或:处处难)(王老好
甭瞎充)。无奈何出店门我就卖\ldots{}\ldots{}

(秦琼右手撩带迈右脚出门,左望,退脚右手挡脸,左转身向小边)

王老好 你卖什么,我的爹。

秦琼 嗳!(接唱)【西皮摇板】卖锏呐,

(\textless{}\textbf{快长锤}\textgreater{}秦琼出门,王老好跟着,秦弹胡子,山膀,往左走圆场,王伯当\protect\hyperlink{fn312}{\textsuperscript{312}}、谢映登二人上,从外边抄过去,到大边,秦捋胡子站住,与王、谢对望亮住,王、谢下,秦弹胡子\textless{}\textbf{紧锤}\textgreater{}向下场门过去望,捋胡子转身向外)

(王伯当、谢映登同上,过场,同下。)

秦琼
(接唱)【西皮快板】两匹马跑得似雪花。分明知道(或:明明知道)是响马,无有批票不好拿。叫声店家(或:叫住店家)快来吧,还你的饭钱就是他。(秦琼一指,到台口,王老好到小边,秦往外一指,往里甩胡子,左手抱锏,由下往右往上往左外甩胡子,右手往外指台下,左腿抬站住,王跟着望,秦弹胡子转身向下场门山膀下)

王老好 您哪一位给呀?

(王老好回头一看)

王老好 跑啦,追。

(王老好下)

{[}第四场{]}

(\textless{}\textbf{水底鱼}\textgreater{}王伯当、谢映登上)

王伯当 王伯当。

谢映登 谢映登。

王伯当 贤弟,前面一座酒肆,你我歇息歇息。

谢映登 好,店家哪里?

(\textless{}\textbf{小锣}\textgreater{},店家上)

店家 (念)杏花村店酒,开门十里香。

\begin{quote}
二位是喝酒的吗?
\end{quote}

王伯当、谢映登 正是。将马带过。

(店接马拴大边,王、谢进店骑马八字坐)

店家 二位用些什么?

王伯当 好酒取来。

店家 好酒一壶!酒到。(店家拿酒摆桌上)

谢映登 唤你再来。

(店家下)

王伯当 请。

(王、谢饮酒,秦琼抱锏、王老好先后上,秦到台口)

秦琼 卖锏!

王老好 卖脸。

(秦琼看王老好)

秦琼 卖锏呐!

王老好 卖脸呐!

秦琼 (呃,)分明是卖锏,怎说是卖脸呐(或:卖的什么脸呐)?

王老好
前街走到后巷,后巷又到大街,连个搭理的也没有,这不是卖脸还卖什么呀?

秦琼 (卖锏,)还是卖锏的受听呐。

王老好 卖脸,卖脸定啦。

(秦琼转小圆场到台口)

秦琼
君子不得第(或:君子不得志),反被这小人欺\textless{}\textbf{哭相思}\textgreater{}。

王老好
店家倒了运,遇见了白吃的\textless{}\textbf{哭相思}\textgreater{}。

秦琼 哪个白吃?

王老好 就是你,你拿钱来。

(秦琼在台口望大边叫王老好)

秦琼 (好好好,)店主东你来看,那旁有两骑高头大马。

王老好 怎么,你要偷人家。

秦琼
啊,骑马之人(或:乘马之人)不是好人,定是响马,你去问问他们可要锏呐。

(秦琼右手指大边再指锏,王、秦换位)

王老好 啊,我去问问,你可别走。

(秦琼 只管地前去。)

王老好 嘿,是我亲家这儿,亲家哪儿?

店家 那是谁这么嚷呐,哟,亲家,你好哇,干什么来啦?

王老好 你们要锏不要呐?

店家 我不洗衣裳,要碱干什么?

王老好 不是,是兵器。有俩骑马的罢,你去问问他们要不要。

店家 好,我问问去,做成了二八扣。

王老好 你去吧。

店家 (店家进门)二位客官要不要锏呐,是兵器。

王伯当 叫那卖锏之人进来,当面言价。

店家 是,亲家,人家要,可是得当面讲价儿。

王老好 是,二八扣你也吹啦,我说二爷,着啦,二爷,

(秦琼打盹,听见叫,把锏推在地上)

秦琼 哎呦砸了我的脚了。

王老好 你得了罢,还上口呐,锏躺这边,会砸了你的脚,没砸着我就不错。

(随便捡起锏要递给秦琼)

秦琼 放下,你不是拿它不动吗?

(王老好赶快放下锏)

王老好 我忘啦。

(秦琼拾锏,秦、王换位)

秦琼 他们讲些什么?

王老好 要倒是要,叫你进去当面讲价。

秦琼 好,走走走。(或:呃,走走走。我们一同前去。)

(秦琼拉王老好)

王老好 我不进去,你拉着干嘛?

秦琼 我怕你跑了哇。

王老好 你该我的,我不怕你跑,你干嘛怕我跑。

秦琼 你跑了我吃哪一个哇?

王老好 去你的吧,官人响马我别掺混啦。

(秦琼 势利的小人。)

(王老好下,秦琼、店家先后挖进,秦大边,店小边,秦望二人)

秦琼 请了。

王伯当 此锏可是要卖?

秦琼 正是要卖。

王伯当 借来一观,放在桌上。

秦琼 有些沉重。

(锏放桌上)

店家 别砸坏桌子。

王伯当 有我等包赔。那一汉子你可会使?

秦琼 略知一二。

王伯当 耍来我等观看。

秦琼 这。(摸肚子)

王伯当 店家,带他前去用饭。

店家 走哇。

秦琼 做什么?

店家 吃饭去。

秦琼 吃饭呐,走哇。(一挥手,跟店家下)

王伯当 我看此人,莫非秦琼。

谢映登 少时问过。

店家 (店家上小边)二位,这位好大饭量,五十包子,外带十碗粥。

王伯当 我等开销。

秦琼 (秦琼上,拱手) 多谢二位酒饭。

王伯当 可曾用好?

秦琼 (秦琼拍腹)(无非是)充饥而已。

王伯当 耍来我等观看。

秦琼 (秦琼看地方,指)此地狭小。(或:此地窄小)

王伯当 哪里宽阔?

店家 后面宽阔。

王伯当 带路。

(秦琼抱锏中间,王伯当、谢映登、店家两边一翻两翻,秦望,王、谢归坐,店下,秦归大边站)

秦琼 献丑了。

(躬揖,回身中间站)

秦琼
【西皮摇板】站在店中用目𥋌\protect\hyperlink{fn313}{\textsuperscript{313}},

(右手弹胡子,小边外边山膀,望王伯当、谢映登,回来正面捋胡子)

秦琼
(接唱)【西皮快板】不由得叔宝怒气发。明明认得他是响马,江湖路上也曾会过他。骂一声贼子真胆大,杀人放火海走天涯。今日里相逢在潞州天堂下,无有批票不好拿(或:不敢拿)。眼前若在历城县,定要将他锁拿到公衙。板子打夹棍夹,看他犯法不犯法。减头去尾耍一耍,(倒叫二位耻笑咱。在舞台演出时此句不唱)

(耍\textbf{锏架子}\protect\hyperlink{fn314}{\textsuperscript{314}},\textless{}\textbf{扫头}\textgreater{}归中间坐,王伯当、谢映登左右两边坐,秦放锏椅侧)

王伯当 听兄台讲话不像此地人氏。

秦琼 本不是此地人氏。

王伯当 哪里人氏?

秦琼 山东历城县人氏。

王伯当 山东历城县弟有一家好友,兄台可知?

秦琼 (有名便知,无名不晓。)但不知哪一家?

王伯当 姓秦名琼字叔宝。

秦琼 愚下(或:在下)就是秦琼。

王伯当、谢映登 原来秦二哥!失敬了。

秦琼 岂敢。请问二位上姓?

王伯当 小弟王伯当。

谢映登 小弟谢映登。

秦琼 原来是二位贤弟,失敬了。

(店家暗上)

王伯当 二哥为何这等模样?

秦琼
只因愚兄解押一十八名绿林(中的)朋友,天气炎热,中途路上,损伤一名。那蔡大老爷不与我批票回文,故而被困在此。

王伯当 这有何难,店家拿我二人名帖,去到蔡大老爷那里,请他发下回文。

(秦琼 有劳了。)

(店家接帖下)

王伯当 前者伯母寿诞之期,弟等有份薄礼可曾收到?

秦琼 但不知打从哪道而去?

王伯当 打从那黑\ldots{}\ldots{}

秦琼 收到了(或:这\ldots{}\ldots{}收下了),当面谢过。

(店家上)

店家 批票领到。

(店家交王伯当,王交秦琼,店下)

王伯当 回文在此 。

(秦琼接文,谢映登取银交秦)

王伯当 弟等散碎银两,二哥收下。

秦琼 二位贤弟银两愚兄怎能用得。

王伯当 二哥不必过谦。

秦琼 (如此)愧领了。

(三人站,秦琼抱锏,三人躬揖)

秦琼
【西皮摇板】心中恼恨单雄信,不该骗我马能行。有朝犯在秦琼手,我打一锏来我要问一声。

(两手分锏举锏亮,合锏右手平划,右转整身面外,右食指向上指,手放下,王伯当、谢映登揖,秦还揖)

王伯当 看在我二人份上,

秦琼 (接唱)【西皮摇板】二贤弟只管把响马来放,

(撩带出门,弹胡子,山膀转身里走向下场门站住,王伯当、谢映登跟出站小边)

王伯当 闯出祸来?

(秦捋胡子回身)

秦琼 (接唱)【西皮摇板】闯出祸来由秦琼担承。

(秦琼拍肚子,向外甩胡子,双手揖,左手抱锏向右捋胡子,右手山膀下。王、谢请,二人挖门进去八字立)

王伯当 店家,酒饭钱在此,我等去也。

(王、谢二人出门拉马,上马\textless{}\textbf{扫头}\textgreater{}下)

\newpage
\hypertarget{ux5357ux9633ux5173}{%
\subsection{南阳关}\label{ux5357ux9633ux5173}}

{[}第一场{]}

(四红龙套,尚师徒、麻叔谋,韩擒虎大锣打上,小座)

韩擒虎
\textless{}\textbf{点绛唇}\textgreater{}奉王钦命,统领雄兵,军威盛,将勇兵精,干戈定太平。

韩擒虎
(念)堂堂男儿立帝基,巍巍武将挂铁衣。咚咚战鼓惊天地,杀气腾腾鸟难飞。

韩擒虎
(白)本帅,韩擒虎。奉了新主钦命,统领人马,捉拿伍云召进京问罪,二位将军,人马可齐?

尚师徒、麻叔谋 俱已齐备。

韩擒虎 兵发南阳。

尚师徒、麻叔谋 兵发南阳去者。

(\textless{}小\textbf{朱奴儿}\textgreater{}众领下)

{[}第二场{]}

伍保 (内白)马来。

(\textless{}\textbf{水底鱼}\textgreater{}前半,上)

伍保
俺,伍保。太老爷不知身犯何罪,敲牙割舌而亡。不免回转南阳关,报与老爷知道,就此马上加鞭。

(\textless{}\textbf{水底鱼}\textgreater{}后半,下)

{[}第三场{]}

(\textless{}发点\textgreater{}四文堂站门,伍云召上)

伍云召 {[}引子{]}威风浩荡,统雄师,镇守南阳。

(\textless{}\textbf{发点}合头\textgreater{},大座)

伍云召
(念)统雄师东征西荡(或:东杀西挡),每年间杀砍战场。食君禄哪得安享,与祖先廊庙争光。

伍云召
(白)本帅伍云召。隋帝驾前为臣,吾父官居太宰,(或:本帅武云召,吾父伍建章,文帝驾前为臣,官居当朝太宰。)本帅镇守南阳,只因夫人新生(或:只因夫人生下)一子,也曾命家将(或:也曾命家人)伍保进京,一来与圣上问安,二来与爹娘报喜,一去数日未见回报。今当操演之期,站堂军,教场去者。(或:这几日本帅心惊肉跳,不知为了何事。今当操演之期,左右,教场去者。或:这几日本帅心惊肉跳,不知为了何事。站堂军,伺候了!或:这几日本帅心惊肉跳,不知为了何事。这且不言,今当三六九日操演之期,站堂军,教场去者。)

伍保 (内白)走哇!

(上,小边下马,进门站小边)

伍保 参见老爷。

伍云召 伍保回来了?

伍保 回来了,大事不好了!

伍云召 (伍云召惊介)什么大事?(或:何事惊慌?)

伍保 太老爷、太夫人不知身犯何罪,敲牙割舌而亡。

(伍云召出位,台口拉伍保)

伍云召 你待怎讲?(或:怎么讲?!)

伍保 敲牙割舌而亡。

(伍云召拉住伍保带到大边去)

伍云召
\textless{}\textbf{叫头}\textgreater{}爹爹,母亲,哎呀!(昏坐下,伍保挡)

伍保 老爷醒来。

伍云召
【西皮导板】闻惊耗不由人魂魄掉,\textless{}\textbf{叫头}\textgreater{}爹爹(或:父亲),母亲,爹娘啊,啊!

伍云召 (接唱)【西皮散板】珠泪点点往下抛。忍泪含悲(或:本帅开言)叫伍保,

伍云召 伍保,

伍云召 (接唱)【西皮散板】被害的(或:犯罪的)情由(细)说根苗。

伍保 老爷。

伍保
(接唱)【西皮散板】杨素化及行奸巧,太宰割舌把牙敲。擒虎领兵人马到,捉拿老爷转回朝(或:转还朝)。

伍云召 好贼!(推胡子,指)

伍云召
【西皮散板】听一言来心头恼,二目圆睁似火烧。站在大堂传令号,大小三军(或:大小儿郎)听根苗:本帅有心把仇报,尔等可敢反皇朝?(翻袖指)

众 我等情愿。

(众跪,伍云召翻双袖扶众,众起立,右回身拿令旗亮住,面向大边伍保)

伍云召 (接唱)【西皮散板】伍保与爷改旗号,

(绕旗扔递给伍保,众、伍保举旗与众归小边,伍云召撩袍面向下场门、扔袍回身立大边面向台口)

伍云召 (接唱)【西皮散板】南阳关杀一个浪里蛟。

(\textless{}\textbf{四击头}\textgreater{}亮相,撩袍走到大边,撩袍抬左腿面向里亮相,走下,众跟下)

{[}第四场{]}

韩擒虎 (内)【西皮导板】南阳关前放号炮,

(韩众上站斜门,韩擒虎上)

韩擒虎 【西皮原板】对对旌旗空中飘。

(众归正场)

韩擒虎 (接唱)
【西皮原板】左先行骑的是呼雷豹,右先行稳坐在马鞍桥。大小三军齐开道,韩擒虎马上叹英豪。伍建章他本三朝元老,为国忠良无下梢。

尚师徒 (接唱) 【西皮原板】新主降下旨一道,元帅何必挂心劳?

麻叔谋 (接唱) 【西皮原板】在王驾前说王好,食王爵禄当效劳。

韩擒虎
【西皮快板】二先行\protect\hyperlink{fn315}{\textsuperscript{315}}说话志量高,不由老夫喜心稍。就将南阳齐围绕,

(韩众领下,\textless{}\textbf{纽丝}\textgreater{})

韩擒虎 (接唱)【西皮散板】捉拿云召转还朝。

(\textless{}\textbf{大锣打下}\textgreater{},起鼓)

{[}第五场(连场){]}

伍云召 (内)【西皮导板】恨杨广斩忠良谗臣当道,

(\textless{}\textbf{急急风}\textgreater{}韩擒虎众上站门,打上、韩上,站台中间,伍云召上城,\textless{}\textbf{四击头}\textgreater{}起唱,云弹胡子亮、哭``爹娘啊'')

伍云召
【西皮原板】叹双亲不由人珠泪双抛。手扶着垛口往下瞧,韩擒虎虽年迈杀气高。尚师徒胯下呼雷豹,麻叔谋使钢鞭稳坐在马鞍桥(或:麻叔谋使长枪鞭插在马鞍桥;或:麻叔谋打将鞭稳跨在马鞍桥)。左右先锋把帅保,耀武扬威逞英豪。搌干了(或:擦干了)泪痕伯父(哇)叫,【西皮二六】侄男有话禀年高:自古(道)臣尽忠来子当尽孝,方不愧人间走一遭。我的父忠心把国保,敲牙割舌为的是哪条?连四员虎将俱都斩了,我那年迈的娘也受那一刀。【西皮快板】到此时就该把气消了,兵困南阳为哪条?世代的忠良难话表,叫儿泪抛不泪抛。

韩擒虎
【西皮快板】贤侄休得珠泪掉,为伯言来听根苗:毁谤新主罪非小,随同为伯转回朝。

伍云召
【西皮快板】老伯父把话讲差了,侄儿言来听根苗:宇文化及行奸巧,杨广无道霸当朝。纵然将侄儿拿去了,绝了伍家后代根苗。既与我父同朝好,就该宽放路一条。伍家有朝把仇报,早烧香,晚唪经(或:晚点灯),供奉年高,饶是不饶?

韩擒虎
【西皮摇板】贤侄休把事看小,非是为伯不肯饶。新主降下旨一道,左右先行(或:二位先行)杀气高。

尚师徒 【西皮摇板】快将南阳城开了。

麻叔谋 【西皮摇板】枪对枪来刀对刀。

(伍云召 呀呸!)

伍云召
【西皮散板】伯父与我(或:我与伯父)好言告,匹夫竟敢逞英豪。叫伍保与爷城开了。

(扫一句,``叫伍保''时,伍保应)

(\textless{}\textbf{扫头}\textgreater{}下城\textless{}\textbf{急急风}\textgreater{}开打,韩败下,伍云召龙套追过场,云耍下场亮大边,下场门下\protect\hyperlink{fn316}{\textsuperscript{316}})

{[}第六场{]}

(伍保上,尚师徒、麻叔谋上,开打尚师徒、麻叔谋败下,韩擒虎上,伍保败下,伍云召上,开打韩败下,云耍下场追下\protect\hyperlink{fn317}{\textsuperscript{317}})

{[}第七场{]}

(韩擒虎众\textless{}\textbf{乱锤}\textgreater{}败上)

韩擒虎 伍云召杀法厉害,安营扎寨。

(韩擒虎众上)

{[}第八场{]}

(\textless{}\textbf{风入松}\textgreater{}四上手推粮车,宇文成都上)

宇文成都
某,宇文成都。奉了新主旨意,押运粮草,南阳关前听用,军士们,催军。

(\textless{}\textbf{风入松}合头\textgreater{},宇文成都,众下)

{[}第九场{]}

(四龙套站门,引韩擒虎上,小坐)

韩擒虎 眼观旌旗起,耳听好消息。

(内白:``无敌将军到。'')

韩擒虎 有请。

(\textless{}\textbf{吹打}\textgreater{}韩擒虎出门迎,宇文成都众上,挖门进,宇文、韩大小边八字坐)

韩擒虎 不知贤侄驾到未曾远迎,当面恕罪。

宇文成都 岂敢。小侄来的鲁莽,伯父海涵。

韩擒虎 岂敢。

宇文成都 可曾与那贼会过阵来?

韩擒虎 会过一阵,大败而回。

宇文成都 待侄会他。

韩擒虎 须要小心。

(韩擒虎下)

宇文成都 众将官,杀。

(宇文成都脱蟒,拿镋,众引下)

{[}第十场{]}

(伍云召、伍夫人打上,夫人抱喜神)

伍云召 (念)父母冤仇恨,

伍夫人 (念)时刻挂在心(或:常挂一片心)。

(八字坐,伍保报上站小边)

伍保 报,宇文成都讨战。

伍云召 (再探!)不,不\ldots{}\ldots{}好了!

伍云召
【西皮散板】宇文成都领兵到,娃娃的武艺比我高,眼见得冤仇不能报,爹娘呀!

(伍保拿枪)

伍保 请老爷上马。

(伍云召脱开氅,提枪上马,伍保下,伍云召回身收腿)

伍云召 (接唱)【西皮散板】老天爷助我成功劳。

(伍云召下)

伍夫人 【西皮散板】一见老爷跨金镫,倒教奴家挂在心。

(夫人下)

{[}第十一场{]}

(二龙出水,伍云召众大边、宇文成都众小边,伍、宇文一、二过合,搕开)

宇文成都 【西皮散板】一见云召心好恼,

(宇文成都打伍云召蓬头,伍保挑出架开)

宇文成都
(接唱)【西皮散板】骂声无知小儿曹。既知某家领兵到,就该一同转还回朝。

伍云召 (接唱)【西皮散板】你父不该行奸巧。

宇文成都 (接唱)【西皮散板】你父不该骂当朝。

伍云召 (接唱)【西皮散板】任凭尔是天神到,

(扫一句。剜萝卜,钻烟筒,开打伍云召败下,宇文成都追下\protect\hyperlink{fn318}{\textsuperscript{318}})

{[}第十二场{]}

(伍夫人上)

伍夫人 【西皮散板】老爷出兵去会阵,不知胜负与输赢。

(伍保上)

伍保 老爷回府。

(伍云召上亮住下马,与夫人推磨进门坐中间,昏)

伍夫人 老爷醒来。

伍云召 【西皮导板】这一阵杀得我昏迷了,(白)看枪!(立)

伍夫人 喂呀!

(伍云召 唉呀!)

伍云召
【西皮散板】不由本帅心内焦。回头忙把夫人叫(或:便把夫人叫),放我父子把命逃。

伍夫人 (接唱)【 【西皮散板】娇儿交与伍保抱,后花园中赴阴曹。

(伍保接喜神,伍夫人碰死下)

伍保 夫人自尽。

伍云召
【西皮散板】一见夫人命丧了(或:自尽了),不由本帅(或:怎不教人)泪双抛。叫伍保(将)尸首掩埋了。\textless{}\textbf{扫头}\textgreater{}

(伍保交喜神给伍云召,伍云召后场面里做束子介,伍保埋尸,带马,伍云召上马,向上场门,伍保下,宇文成都上,漫云头,伍云召过小边,勾伍云召到大边,里盖,打云蓬头,保挑开,伍云召下,宇文、保开打,打死保,宇文成都追下)

{[}第十三场{]}

(朱灿\textless{}\textbf{五击头}\textgreater{}上,站中间)

朱灿
一载干戈动,十载不太平。某朱灿。只因隋帝无道,隐居家园,今日闲暇无事不免闲游一回。(鼓架子)呀,那旁人马呐喊,待我登高一望。

(朱灿上桌望,伍云召、宇文成都龙套上,伍云召双望门,先上后下,平端枪,龙套追下)

朱灿
前面走的伍云召,后面追赶宇文成都,我想一个人怕一个人也就是了,为何苦苦追赶?待我上前打一个抱不平,又恐不是他人对手。(钟声)昔日王员外修造关帝庙宇,我不免去至庙中,借那周仓老爷大刀盔铠,前去搭救,吓死这些亡八东西。(下)

{[}第十四、五场(连场){]}

(龙套、伍云召上望门,先下再上,打马捋枪下,众再追下。朱灿下场着盔铠拿刀上椅站大边台口,伍云召\textless{}\textbf{水底鱼}\textgreater{}上,台口回头望上场门起\textless{}\textbf{乱锤}\textgreater{},到上场门再到小边台口见朱灿\protect\hyperlink{fn319}{\textsuperscript{319}},朱灿招手,伍云召下马拉马蹉步过去,站朱灿身后下马,宇文成都众上跟朱灿架住)

宇文成都 何人挡住某家去路?

朱灿 吾神周仓。

宇文成都 哎呀!

(宇文成都众下。朱灿下椅,伍云召抱喜神,朱灿、伍大边推磨,一二枕,伍大边、朱灿小边立)

(伍云召 多谢贤弟搭救。)

朱灿 仁兄为何这等模样?

伍云召
只因(或:可恨)杨广无道,将我父敲牙割舌而亡,愚兄逃出(或:杀出)南阳,不是贤弟搭救,险遭不测。

朱灿 仁兄意欲何往?

伍云召 愚兄意欲往雄阔海那里搬兵报仇。只是娇儿无人抚养。

朱灿 待小弟抱回家中抚养,日后父子自有相逢之日。

伍云召
如此贤弟请上,受我父子一拜。(或:如此有劳贤弟,请上受我父子一拜。)

(二人拜,喜神交朱灿,站)

伍云召
【西皮快板】幸喜贤弟遇得巧,救我父子命二条。娇儿付与贤弟抱,昼夜之间(多)受辛劳(或:要辛劳)。辞别贤弟跨虎豹,

(伍云召上马,回身)

伍云召 (接唱)【西皮散板】学一个伍子胥往吴国逃。

(转身\textless{}\textbf{叫头}\textgreater{})

伍云召 登科,我儿,(唉,)儿吓\ldots{}\ldots{}罢!

(伍云召下)

朱灿
仁兄已去,我不免送还周爷老爷大刀盔铠。(小孩哭)儿吓,不要啼哭,我给你买糕干去。

(朱灿下)

{[}第十六场(连场){]}

(\textless{}\textbf{乱锤}\textgreater{}宇文成都众上)

宇文成都
且住!正要擒那伍云召下马,周仓老爷显圣,不是俺马走如飞,险遭不测,不免回朝启奏。众将官,

(众应)

宇文成都 收兵回朝。

(\textless{}\textbf{尾声}\textgreater{}众下)

*王荣山关于此戏的把子研究:

《南阳关》是隋炀帝命韩擒虎挂帅到南阳去捉拿伍云召进京问罪。隋炀帝知道伍本领大,所以不会派无能之辈挂帅,韩擒虎不是没本事的人。但是戏中表出韩擒虎同情伍家冤枉,不愿意拿伍云召,与伍交战比画几下,希望伍逃走,他就算追赶不上收兵交差。可是伍不理解韩的心情,一心要报父仇,不顾一切奋勇追击,直到宇文成都到来才被迫逃亡。照这样的戏情来安排,伍与韩的开打就不能多,与宇文更不能多打,因为宇文非常厉害。与韩开打是二人都使枪,头场开城会阵\ldots{}\ldots{},二场打是追韩上\ldots{}\ldots{}这些都是表示韩擒虎节节退让,伍云召则不明此理,奋勇追杀,头场打还来个龙套追过场作衬托。后半出韩擒虎不露面,由宇文成都追伍云召,让朱灿装神把宇文蒙回来传令收兵,伍云召逃走,追赶不上的责任落在宇文身上,既合乎戏情,又满足观众愿望。

\textbf{陈超老师介绍:}

《南阳关》把子中甩发的使用是特色:

开打中有三下甩发。

见宇文成都也有三下甩发。

\textbf{陈超老师按:}

王凤卿有三出戏绝不穿马褂,《南阳关》、《战樊城》、《穆天王》,这三出现在都穿马褂了。

《南阳关》、《战樊城》余叔岩、王荣山、王又宸都不穿马褂,杨宝森学余也不穿。

\newpage
\hypertarget{ux6253ux767bux5dde-ux4e4b-ux79e6ux743c}{%
\subsection{打登州 之
秦琼}\label{ux6253ux767bux5dde-ux4e4b-ux79e6ux743c}}

{[}第一场{]}

【西皮导板】无情铁索困蛟龙,

【西皮原板】一腔怒气贯长虹。俺平生交友义气重,侠肠义胆论英雄。靠山王令出山岳动,历城县内捉拿秦琼。舍不得老娘【转西皮快板】无人侍奉,舍不得妻和子泪洒前胸。舍不得亲眷们同衙伙众,实难舍邻居们仁义宾朋。

【西皮摇板】前思后想心酸痛,可叹我闯荡江湖十数春有始无终。

【西皮散板】耳边厢又听得悲声大放,

呀!

【西皮散板】抬头只见儿的娘。不想灾祸从天降,此去恐难转还乡。

【西皮散板】老娘亲休把儿盼望,全当是未生儿一场。

哎呀,母亲呐!靠山王捉拿孩儿,定是为了大反山东之故。儿今此去,吉凶不保。望母亲静养身体,休要挂念你这苦命的------唉,孩儿啊\ldots{}\ldots{}(哭介)

哎呀,妻呀!事已至此,我纵有千言万语,一时焉能说得尽?我今此去,只恐有死无生,望你在母亲面前多多孝敬,管教我儿,长大成人也好接续香烟。倘得生还,一家还有相逢之日,母亲请上,孩儿就此叩别了!

【西皮散板】含悲忍泪拜慈亲,

(秦母 【西皮散板】\ldots{}\ldots{}顷刻两离分。)

(贾氏 【西皮散板】\ldots{}\ldots{}泪淋淋,)

【西皮散板】你休得要恨天怨地泪淋淋。

\textless{}\textbf{哭头}\textgreater{}老娘亲呐,

(秦母 我儿。)

\textless{}\textbf{哭头}\textgreater{}受苦的妻啊,啊,

\textless{}\textbf{哭头}\textgreater{}儿的娘哪!

{[}第二场{]}

【西皮摇板】历城县内上了杻,儿行千里母担忧。眼观日落西山后,望求差爷把店投。

{[}第三场{]}

这\ldots{}\ldots{}

病倒有,难道你会医治?

但不知何药为引?

记下了。

想俺秦琼,英名盖世,不想夜宿三家店中,受此苦刑,好不伤感人也\ldots{}\ldots{}(哭介)

【二黄三眼】在店中吊上杆威风难展,龙困在无水沙滩难把身翻。良马渴思饮长江水,人到了难中仗金兰。魏大哥、徐三弟难得见面,咬金、俊达,金甲、童环。鲁明星、鲁明月,伯当、国远,燕山罗成相见难。眼前若有那罗成士信,

【二黄散板】一定要搭救我免受熬煎。

【二黄散板】差官不必将我问,我与罗成姑表亲。

【二黄散板】问声差官是何人。

【二黄散板】不该不该大不该,不该将兄吊起来。

【二黄散板】你不知来我不怪,弟兄对坐叙开怀。

我若逃走,岂不连累于你?

哦,里八个,外八个,开门看看是哪个?

正是愚兄。

走不得。

现有公差在此。

杀不得。

他不是外人呐,乃是一门内亲。

这倒使得。

罗贤弟见过史贤弟。

这做什么?

呵,胆量是好的。

是好的。

我若逃走,一来连累罗表弟;二来家中还有老母、妻室,如何走得?

我两膀疼痛,难以提笔。

何人下书?

好,有劳二位贤弟!

【西皮导板】三家店内把计定,

这做什么?

胆量是好的。

【西皮散板】好似龙虎会风云。上写秦琼把首顿,

这做什么?

呃,你自己揉上一揉也就好了。

【西皮散板】拜上同盟结义人。八月十五登州进,

待我看来。

外面无人,乃是灯光照得你我弟兄的人影。

【西皮散板】搭救愚兄上山林。一封书信修齐整,

贤弟呀!

【西皮散板】有劳贤弟走一程。

杀不得。

方才言过,他乃是一门内亲。

且慢,惊动店家,只恐走不成了。

愚兄全知。

啊贤弟,史贤弟在瓦岗寨上不算第一也算第二。

真乃英雄也。

【西皮散板】待等到中秋节登州放火,

(罗周 【西皮散板】那时节做一个里应外合。)

{[}第四场{]}

参见千岁。

正是。

千岁将小人拿来,敢莫是为了长叶岭之事(或:原来为的是大反山东之事)?

(杨林 着,着,着!)

【西皮散板】长叶岭谁是谁不是?

【西皮散板】千层浪里翻身滚,百尺高竿又复生。

哎呀贤弟呀!适才教那老贼一棒将兄打死,也免得贤弟你挂心呐。

噤声!

贤弟呀!

【西皮散板】蛟龙正在沙滩困,切盼春雷响一声。待等八月中秋到,

(罗周 【西皮散板】再与老贼定输赢。)

{[}第五场{]}

【西皮散板】长叶岭前一着错\protect\hyperlink{fn320}{\textsuperscript{320}},事到头来无奈何。

正合我意。

就烦贤弟先行,打探瓦岗兄弟的消息便了。

【二黄导板】登州城闷坏了秦叔宝,

【回龙】走过来、行过去大街之上腕带杻锁倒教我好不心焦。

【二黄原板】十数载马上威风浩,英雄四路\protect\hyperlink{fn321}{\textsuperscript{321}}美名标。到如今屋漏偏遭连阴雨,船到江心失了篙。

那旁来了敢是茂\ldots{}\ldots{}

我是你二哥秦琼。

贤弟,莫非救我来了?

【二黄原板】徐茂公阴阳算得好,何不救我出笼牢。是是是来明白了,内中定有巧计高。

来的敢是尤\ldots{}\ldots{}

我是你二哥秦琼。

贤弟,敢是救我来了?

【二黄原板】尤俊达反山东劫过宝,官兵拿获难脱逃。若不亏我秦叔宝,何人放他往外逃。

那旁来的敢是雄信?

我是你二哥秦琼。

莫非念在结义之情前来救我来了?

【二黄原板】单雄信他本是江洋盗,枣阳山前逞英豪。被我一锏来打倒,我也曾饶过他性命一条。

来的敢是表弟?

我是你二哥秦琼。

想是贤弟念在姑表之亲救我来了?

啊?!

【二黄散板】罗成是我亲姑表,不认我秦琼为哪条?越思越想心头恼。

(罗周 【二黄散板】二哥为何心内焦?)

哎呀贤弟呀,适才众家兄弟俱不相认愚兄,如何是好?

有劳了!

【二黄散板】罗周待我情义好,他与秦琼似同胞。

【二黄散板】站在阳关心焦躁,

啊?!

【二黄散板】来了咬金故旧交。

来的敢是程\ldots{}\ldots{}

咬金!

我是二哥秦琼。

你我幼年相交,想是救我来了。

啊?!

【二黄散板】好个聪明的程咬金,他把救字暗藏身。急难之人心不稳。

打探瓦岗弟兄之事如何?

好哇------

【二黄散板】待等明日中秋到。

(罗周 【二黄散板】再与老贼动枪刀。)

{[}第六场{]}

参见千岁。

(杨林 为何这等白胖?)

(罗周 这个\ldots{}\ldots{})

(秦琼 海风吹得浮肿。)\protect\hyperlink{fn322}{\textsuperscript{322}}

不足千岁一观。

遵令。

【西皮二六】老杨林校场令传下,不由得秦琼怒气发。既然他疑心我通响马,到此为何不把我来杀。今日里校场试锏法,又恐其中事有差。我今既在矮檐下,

【西皮快板】生死二字何惧他。约定了瓦岗弟兄把山下,搭救秦琼把老贼拿。

【西皮快板】减头去尾耍一耍。

(秦琼耍锏\protect\hyperlink{fn323}{\textsuperscript{323}})

(杨林 不足为奇。)

小人马上武艺还好。

小人不敢!

【西皮摇板】鹞子翻身上黄骠,

【西皮快板】来了瓦岗众英豪。斜跨雕鞍高声叫,

杨林!老贼!

【西皮摇板】敢与老爷动枪刀。

多谢众位贤弟搭救。

(请呐!)

\newpage
\hypertarget{ux65adux5bc6ux6da7-ux4e4b-ux738bux4f2fux5f53}{%
\subsection{断密涧 之
王伯当}\label{ux65adux5bc6ux6da7-ux4e4b-ux738bux4f2fux5f53}}

\textbf{{[}第一场{]}}

\textbf{(念)习就百步箭穿杨,一片丹心扶瓦岗。}

\textbf{俺王勇}\protect\hyperlink{fn324}{\textsuperscript{324}}\textbf{,奉了大王之命,追赶徐勣、魏徵,赶至三岔路口,念在贾家楼结义之情,将他二人释放。不免回山懵懂启奏。}

\textbf{众好汉,回山!}

\textbf{{[}第二场{]}}

\textbf{(念)留得五湖明月在,何愁无处下金钩。}

\textbf{报------王勇告呃进!}

\textbf{参见大王,伯当交令。}

\textbf{谢座。}

\textbf{他二人去远,追赶不上,特地回山启奏。}

\textbf{大王------}

\textbf{【西皮原板】大王说话太痴迷,细听伯当把话提:三十六人曾结义,生死相交永不离。二劫皇杠把祸起,大反山东【转西皮二六】赶惹是非。王伯当呃单人【转西皮快板】赶唐璧,秦叔宝匹马取}金堤\protect\hyperlink{fn325}{\textsuperscript{325}}\textbf{。夜夺瓦岗非容易,三月三日拜帅旗。那程咬金有福登龙位,兵多将广人马齐。自从大王到此地,锦绣江山化灰泥。恨飞鼠盗去仓粮米,满山喽罗似雀飞。众家弟兄散了队,一个东来一个西。只剩下王勇来保你,反说为臣把君欺。倘若哪国刀兵起,祸到临头悔不及。}

\textbf{(李密 【西皮快板】\ldots{}\ldots{},莫做三心二意呃的。)}

\textbf{【西皮快板】说什么三心共二意,为臣怎敢把君欺。王勇保主有假意,气化清风肉化泥。}

\textbf{(李密 好哇!)}

\textbf{(李密
【西皮快板】好一个忠良王贤弟,亚赛过当年介子推。孤王若把良心昧,乱箭攒身不回归。)}

\textbf{(李密 \ldots{}\ldots{},必须重整瓦岗才是。)}

\textbf{瓦岗已散,难以重整。}

\textbf{(李密 这便如何是好?)}

\textbf{臣保定大王前去降唐。}

\textbf{(李密 \ldots{}\ldots{},只是那南牢之事?)}

\textbf{敢保大王无事。}

\textbf{且慢,这样不能前往。}

\textbf{必须换了亵衣小帽,方可前往。}

\textbf{事到如今,舍不得,也要舍!}

\textbf{改换。}

\textbf{(李密 \ldots{}\ldots{},传令。)}

\textbf{领旨。}

\textbf{下面听者:大王前去降唐,愿随者随,不愿随者,各自散去。}

\textbf{(李密 贤弟,与孤------带呃马!)}

\textbf{臣------领呐------旨。}

\textbf{(李密 【西皮快板】\ldots{}\ldots{}被犬欺。)}

\textbf{【西皮快板】大王不必长叹息,伯当言来听端的:阿房宫,今何在,铜雀楼台化灰泥。江山自古有兴废,男儿能伸又能屈。改邪归正投唐去,青史名标万古题。}

\textbf{(李密 【西皮快板】怕只怕唐童把仇记,笼中之鸟也难飞。)}

\textbf{【西皮摇板】杀身大祸臣愿替,愿保大王挂紫衣。}

\textbf{{[}第三场{]}}

\textbf{(李密 【西皮散板】昔日螳螂去捕蝉,)}

\textbf{【西皮散板】偶遇黄雀把路拦。}

\textbf{(李密 【西皮散板】黄雀又被金弹打,)}

\textbf{【西皮散板】打弹之人被虎餐。}

\textbf{(李密 【西皮散板】猛虎落在陷阱内,)}

\textbf{【西皮散板】仇报仇来冤报冤。}

\textbf{(李密 【西皮散板】勒住丝缰用目看,)}

\textbf{【西皮散板】只见箭雁落马前。}

\textbf{``西府秦王,百发百中。''}

\textbf{乃是箭雁。}

\textbf{(李密 待孤看来。)}

\textbf{大王请看。}

\textbf{且慢,有了箭雁,就有了进身}\protect\hyperlink{fn326}{\textsuperscript{326}}\textbf{之策。}

\textbf{待等唐童到此,将箭雁献上,岂不是进身之策?}

\textbf{啊------敢保大王无事。}

\textbf{松林等候。}

\textbf{请呐------}

\textbf{(李密 【西皮摇板】\ldots{}\ldots{}雕翎箭,)}

\textbf{【西皮摇板】狭路相逢天凑缘。}

\textbf{(众 \ldots{}\ldots{}挡道!)}

\textbf{王勇!}

\textbf{参见千岁!}

\textbf{(李世民 来此则甚?)}

\textbf{为臣拾得箭雁,特来献上。}

\textbf{(李世民 到此何事?)}

\textbf{臣保定一家,前来降唐。}

\textbf{(李世民 但不知哪一家?)}

\textbf{就是那西魏王李密。}

\textbf{(李世民 哦呵,请来相见。)}

\textbf{犹恐千岁记起南牢仇恨。}

\textbf{多谢千岁。}

\textbf{松林等候。}

\textbf{啊千岁,须要言而有信。}

\textbf{谢千岁。}

\textbf{【西皮快板】好一个仁义二主君,他比尧、舜强十分。松林忙把大王请,}

\textbf{(李密 【}西皮摇板\textbf{】\ldots{}\ldots{}不安宁。)}

\textbf{也曾会过。}

\textbf{请大王前去相见。}

\textbf{一概不究。}

\textbf{吾王且慢,见了千岁怎生行礼?}

\textbf{他乃一君,你乃一臣,必须下一全礼。}

\textbf{你前来做甚?}

\textbf{却又来!}

\textbf{【}西皮快板\textbf{】说什么瓦岗你为君,说什么低头不拜人。上前去施一个君臣礼,我保你头戴乌纱入朝门。}

\textbf{(李密 【}西皮摇板\textbf{】人言唐童似尧、舜,)}

\textbf{【}西皮摇板\textbf{】话不虚传果是真。}

\textbf{(李密 【}西皮摇板\textbf{】降唐事儿呃心拿稳呐,)}

\textbf{【}西皮摇板\textbf{】似狂风吹散了满天云呐。}

\textbf{{[}第四场{]}}

\textbf{(李密 (念)低头入朝门,)}

\textbf{(念)叩见圣明君。}

\textbf{(李密 臣李密,)}

\textbf{王勇,}

\textbf{(李密、王勇 见驾,万岁。)}

\textbf{臣有本启奏。}

\textbf{谢万岁!}

\textbf{{[}第五场{]}}

\textbf{正是:(念)王勇生来秉性刚,一臣不保二君王。}

\textbf{【西皮摇板】蟒袍玉带我不爱,一片丹心揣在怀。}

\textbf{{[}第六场{]}}

\textbf{【西皮摇板】将身来在宫墙外,大王慌张为何来。}

\textbf{大王慌慌张张,为了何事?}

\textbf{我却不信。}

\textbf{哎呀!}

\textbf{【西皮}摇板\textbf{】一见公主倒尘埃,怎不教人痛伤怀。}

\textbf{【西皮}摇板\textbf{】河阳公主今何在,}

\textbf{(李密 呵哈哈哈\ldots{}\ldots{}(笑介))}

\textbf{呀呸!}

\textbf{【西皮}摇板\textbf{】忘恩负义怎安排。}

\textbf{似你这样忘恩负义,谁来救你?}

\textbf{大王请起。}

\textbf{征剿山东。}

\textbf{吏部旨意可在身旁?}

\textbf{你我趁此黑夜,诈出皇城,再作道理。}

\textbf{走哇!}

\textbf{{[}第七场{]}}

\textbf{【西皮摇板】摇头摆尾再不来。}

\textbf{(李密 【西皮摇板】只身逃出天罗网,)}

\textbf{【西皮摇板】翻身跳出是非墙。}

\textbf{你我去往山后刘武周那里安身便了。}

\textbf{唐童追兵甚急,倘若赶上,那还了得。}

\textbf{又有你!}

\textbf{俺王勇的性命,断送------你手!}

\textbf{(李密 呵呵哈哈哈\ldots{}\ldots{}(笑介))}

\textbf{(李密 【西皮原板】\ldots{}\ldots{}}带惆怅?\textbf{)}

\textbf{【西皮原板】你杀那河阳公主因何故,忘恩负义所为哪桩。}

\textbf{【西皮快板】闻言怒发三千丈,太阳头上冒火光。可叹三十六员将,东逃西奔各一方。单单剩下王伯当,大胆保你来降唐。唐王待你恩德广,河阳公主招东床。谋朝篡位心妄想,顺者昌来逆者亡。}

\textbf{(李密 【西皮快板】昔日韩信谋家邦,)}

\textbf{【西皮快板】未央宫中一命亡。}

\textbf{(李密 【西皮快板】毒死平帝是王莽,)}

\textbf{【西皮快板】千刀万剐无下场。}

\textbf{(李密 【西皮快板】曹丕也曾把中原掌,)}

\textbf{【西皮快板】留得骂名天下扬。}

\textbf{(李密 【西皮快板】李渊也是臣谋主,)}

\textbf{【西皮快板】他本是真龙下天堂。}

\textbf{(李密 【西皮快板】说什么\ldots{}\ldots{},封你一字并肩王。)}

\textbf{【西皮快板】说什么一字并肩王,羞得王勇脸无光。你好比人心不足蛇吞象,你好比困龙思想上天堂。手摸胸膛想一想,你是人面兽心肠。}

\textbf{(李密
【西皮快板】\ldots{}\ldots{}君臣一路好商量,李密打马朝前闯呃,)}

\textbf{【西皮散板】伯当错保无义王。}

\textbf{{[}第八场{]}}

\textbf{【西皮摇板】大王休得心慌忙,自有王勇做主张。}

\textbf{千岁!}

\textbf{【西皮快板】稳坐雕鞍把话讲,尊声千岁听端详:王勇生来性情刚,一臣不保二君王。}

\textbf{【西皮摇板】千岁不把臣来放,情愿战死在沙场。}

\textbf{【西皮摇板】王勇前来救大王。}

\textbf{{[}第九场{]}}

\textbf{下马观看。}

\textbf{``断密涧''。}

\textbf{``断密涧''。}

\textbf{不好了!}

\newpage
\hypertarget{ux5343ux79cbux5cad}{%
\subsection{千秋岭}\label{ux5343ux79cbux5cad}}

\textbf{{[}第一场{]}}

\textbf{(罗成上,起霸)}

\textbf{罗成
(念)束发金盔显少年,气吐虹霓贯九天。银枪摆动龙戏水,战马------}

\textbf{(四下手上)}

\textbf{罗成 (念)驰驱似火焰。}

\textbf{罗成
俺、罗成。今在洛阳王世充帐下为将。闻听唐营兵马到来,岂肯容他张狂}\protect\hyperlink{fn327}{\textsuperscript{327}}\textbf{。}

\textbf{罗成 众将官!}

\textbf{众 有!}

\textbf{罗成 起兵前往!}

\textbf{众 啊!}

\textbf{(四下手带马同下)}

\textbf{{[}第二场{]}}

\textbf{(秦琼、尉迟恭上,【打上】)}

\textbf{秦琼 (念)劈抡锏盖世无双,}

\textbf{尉迟恭 (念)水磨鞭保定唐王。}

\textbf{(秦琼大边、尉迟恭小边)}

\textbf{秦琼
某、护国公}\protect\hyperlink{fn328}{\textsuperscript{328}}\textbf{秦琼。}

\textbf{尉迟恭 鄂国公敬德。}

\textbf{秦琼 请了!}

\textbf{尉迟恭 请了!}

\textbf{秦琼 二主升帐,两厢伺候!}

\textbf{尉迟恭 请!}

\textbf{(四文堂引徐勣、李世民上)}

\textbf{李世民
\textless{}点绛唇\textgreater{}\ldots{}\ldots{}儿郎虎豹,军威浩,地动山摇,要把狼烟扫。}

\textbf{(李世民归大座)}

\textbf{众人 参见主公。}

\textbf{李世民 众卿少礼。}

\textbf{李世民 (念)奉了父王令,征战洛阳城。虽然我为主,全仗众公卿。}

\textbf{李世民
小王李世民。奉了父王之命,征战洛阳王世充,可恨他战又不战,降又不降。也曾命探马前去打探,未见回报。}

\textbf{(报子上)}

\textbf{报子 罗成讨战。}

\textbf{李世民 再探!}

\textbf{报子 啊。}

\textbf{(报子下)}

\textbf{李世民 先生!}

\textbf{徐勣 主公!}

\textbf{李世民 罗成讨战,命何将出马?}

\textbf{徐勣 命马三保带领本部人马,攻打头阵。}

\textbf{李世民 先生传令。}

\textbf{徐勣 得令。}

\textbf{徐勣 下面听者!二主有令:命马三保带领本部人马攻打头阵!}

\textbf{马三保 啊!}

\textbf{徐勣 主公请至后帐,且听好音便了。}

\textbf{(众同下)}

\textbf{{[}第三场{]}}

\textbf{(马三保带四龙套上,\textless{}四边静\textgreater{}头段)}

\textbf{马三保 某、马三保。奉命攻打头阵。}

\textbf{马三保 众将官,杀!}

\textbf{(\textless{}急急风\textgreater{},罗成、四下手同上,会阵)}

\textbf{罗成 马三保,你乃久败之将,擅敢起兵到此!}

\textbf{马三保 一派胡言。放马过来!}

\textbf{(马三保败下,罗成追下)}

\textbf{{[}第四场{]}}

\textbf{(\textless{}四边静合头\textgreater{}四文堂、尉迟恭、秦琼、徐勣、李世民上,李世民归大座)}

\textbf{(报子上)}

\textbf{报子 马三保败阵!}

\textbf{秦琼 再探!}

\textbf{报子 啊!}

\textbf{(报子下)}

\textbf{李世民 先生,马三保败阵,当命何将出马?}

\textbf{徐勣 待臣思忖回话。}

\textbf{尉迟恭 嗯------}

\textbf{徐勣
呜哙呀!看此黑贼耀武扬威,我不免在二主面前搬动是非,教他出马打一败仗,也好灭灭他的火性。}

\textbf{徐勣
臣启主公:罗成骁勇,我营将士,无人对敌,将营盘暂退四十里,打本进京,奏与老王,遣来能将大战罗成!}

\textbf{李世民 先生传令!}

\textbf{徐勣 得令。}

\textbf{徐勣 令出\ldots{}\ldots{}}

\textbf{尉迟恭 且慢呐!}

\textbf{徐勣 尉迟恭为何阻令?}

\textbf{尉迟恭
先生!你道那罗成天上少有,地下难寻,眼前若有二主将令,某要生擒那罗成进帐呃。}

\textbf{徐勣 你若擒得住罗成,山人愿将军师大印,付你执掌。你呢?}

\textbf{尉迟恭 我若擒不住罗成,愿输项上的人头。}

\textbf{徐勣 空口无凭,敢与山人击掌?}

\textbf{尉迟恭 击掌?请啊!}

\textbf{徐勣 尉迟恭听令!}

\textbf{尉迟恭 在。}

\textbf{徐勣 命你带领本部人马,大战罗成!}

\textbf{尉迟恭 得令!}

\textbf{李世民 皇兄须要小心。}

\textbf{尉迟恭 二主!}

\textbf{尉迟恭
【西皮摇板】二主不必太小量,强中自有强中强。辞别二主出宝帐,}

\textbf{(尉迟恭过大边,秦琼到小边)}

\textbf{秦琼 哪里去?}

\textbf{尉迟恭 大战罗成!}

\textbf{秦琼 你不能得胜!}

\textbf{尉迟恭 你怎样知晓?}

\textbf{秦琼 他与我乃是姑表之亲。}

\textbf{尉迟恭 你们俱是一党!}

\textbf{秦琼 哼!}

\textbf{尉迟恭 【西皮摇板】战鼓嗵嗵下校场。}

\textbf{(尉迟恭下)}

\textbf{秦琼
【西皮摇板】一见尉迟出唐营,回头埋怨徐先生。明知罗成多骁勇,不该命他去出征。先生传令某出马,}

\textbf{徐勣 你也不能得胜!}

\textbf{秦琼 【西皮摇板】不得胜落一个两太平。}

\textbf{徐勣
【西皮摇板】二哥有所不知情,小弟言来听分明。自从黑贼进唐营,屡屡欺压弟兄们。教他出兵打败阵,看他逞能不逞能。}

\textbf{秦琼
【西皮摇板】好个仁义徐先生,把我弟兄看得清。辞别二主出唐营,}

\textbf{秦琼 【西皮摇板】且听探马报分明。}

\textbf{(秦琼下)}

\textbf{李世民 【西皮摇板】带过御马跨金镫,}

\textbf{(李世民上马)}

\textbf{李世民 【西皮摇板】去到阵前看分明。}

\textbf{(众人同下)}

\textbf{{[}第五场{]}}

\textbf{(四文堂、尉迟恭上)}

\textbf{尉迟恭
【西皮摇板】胯下}\protect\hyperlink{fn329}{\textsuperscript{329}}\textbf{一骑乌骓马,打将钢鞭手中拿。三军与爷催战马,大小将官听根芽。别的将官休出马,单要罗成小娃娃。}

\textbf{(尉迟恭站大边,里边)}

\textbf{四下手 (内)唐将骂阵!}

\textbf{罗成 (内)【西皮摇板】正在后帐习兵法,}

\textbf{(四下手引罗成上,站小边)}

\textbf{罗成
【西皮摇板】听得唐将叫骂咱。三军带过爷的马,管教敌将染黄沙。 }

\textbf{(尉迟恭、罗成驾住)}

\textbf{尉迟恭 来将通名!}

\textbf{罗成 少爷罗成!}

\textbf{尉迟恭 哈哈!哈哈!啊,呵哈哈\ldots{}\ldots{}(笑介)}

\textbf{罗成 黑贼为何发笑?}

\textbf{尉迟恭
罗成呐,孺子!我道你,天上少有,地下难寻;原来是个黄毛的孺子,怎当某家一战?}

\textbf{尉迟恭 三军的,报与二主,上某家大大的头功。}

\textbf{罗成 黑贼到此,一战未交,为何上尔大大头功?}

\textbf{尉迟恭 娃娃,慢说交战,提起你老爷昔年的威风,吓破尔的苦胆!}

\textbf{罗成 你要讲啊!}

\textbf{尉迟恭 你要听呐!}

\textbf{尉迟恭
【西皮快板】勒住马头慢交战,细听老爷表家园:家住山西麻邑县,聚贤村内有家园。日抢三关夺八寨,杀得唐将心胆寒。慢说与爷来交战,提起了威风吓尔还。}

\textbf{罗成
【西皮摇板】忽听黑贼表家园,吓得少爷心胆寒。不战黑贼走了罢,}

\textbf{尉迟恭 敢是怯战?}

\textbf{罗成 呸!}

\textbf{罗成
【西皮快板】细听少爷表家园:我七岁气吹檐前瓦,八九学艺在燕山。十岁姑表枪换锏,十一岁结拜在济南。十二山东放响马,十三岁威名天下传。十四夜打登州府,十五扬州夺状元。十六岁景阳曾打虎,十七染病在山前。少爷今年十八岁,}

\textbf{尉迟恭 十八岁你该死了!}

\textbf{罗成 【西皮摇板】杀得唐将丧黄泉。}

\textbf{(尉迟恭、罗成俩人一兜,合扇下)}

\textbf{{[}第六场{]}}

\textbf{(四龙套引徐勣、李世民同上)}

\textbf{李世民 【西皮摇板】君臣打马出唐营,}

\textbf{徐勣 【西皮摇板】观看两家动刀兵。}

\textbf{李世民 【西皮摇板】下得马来高坡近,}

\textbf{(李世民、徐勣下马)}

\textbf{徐勣}\protect\hyperlink{fn330}{\textsuperscript{330}}
\textbf{【西皮摇板】看是谁胜哪家赢。}

\textbf{(罗成先上,尉迟恭上。罗成把尉迟恭勾过去,把尉迟恭打败,打下)}

\textbf{李世民 【西皮摇板】尉迟好比南山豹,}

\textbf{徐勣 【西皮摇板】罗成好比浪里蛟。}

\textbf{尉迟恭 (内)【西皮导板】越杀越勇罗士信,}

\textbf{(尉迟恭上)}

\textbf{尉迟恭 哎呀!}

\textbf{尉迟恭
【西皮摇板】杀得某家少精神。只杀得襆头戴不稳,乌骓马倒退不前行。背地只把二主怨,回头埋怨徐先生。明知罗成多骁勇,不该命某来出兵。}

\textbf{尉迟恭 【西皮摇板】慈悲大士救八难,缘何不救某难中人。}

\textbf{尉迟恭 罢!}

\textbf{尉迟恭 【西皮摇板】不顾生死将他战,}

\textbf{(罗成上,压住)}

\textbf{罗成 哼,哼,哼\ldots{}\ldots{}(冷笑介)}

\textbf{尉迟恭 【西皮摇板】倒被娃娃笑一声。}

\textbf{尉迟恭 娃娃!}

\textbf{尉迟恭 【西皮摇板】你若马前来归顺,老爷收儿做螟蛉。}

\textbf{(剜萝卜,完了一躲,尉迟恭败下)}

\textbf{李世民 【西皮摇板】越杀越勇罗士信,}

\textbf{徐勣 【西皮摇板】战败我朝鄂国公。}

\textbf{李世民 罗成好将啊,好将!}

\textbf{徐勣 主公连夸好将,莫非有爱将之意?}

\textbf{李世民 小王爱他,他不归顺,也是枉然!}

\textbf{徐勣 待臣顺说他来降,不知二主待将如何?}

\textbf{李世民 皇兄啊!}

\textbf{李世民 【西皮摇板】只要他真心来归顺,我与他皇兄御弟称。}

\textbf{徐勣
【西皮摇板】好个仁义二主君,把我弟兄看得清。辞别主公跨金镫,顺说罗成降唐营。}

\textbf{(徐勣下桌子,下)}

\textbf{李世民 【西皮摇板】人来带过御马乘,}

\textbf{(李世民下桌子)}

\textbf{李世民 【西皮摇板】且候罗成降唐营。}

\textbf{(众人同下)}

\textbf{{[}第七场{]}}

\textbf{(尉迟恭、罗成上)}

\textbf{罗成 黑贼,敢是怯战?}

\textbf{尉迟恭 住了!非是你老爷怯战,我家元帅鸣金收兵,明日再战!}

\textbf{罗成 你少爷不收兵,一定要战!}

\textbf{尉迟恭 你不收兵?}

\textbf{罗成 不收兵!}

\textbf{尉迟恭 你不收兵?}

\textbf{罗成 不收兵!}

\textbf{尉迟恭 嘿!我收兵呃。}

\textbf{(尉迟恭下)}

\textbf{罗成 哼哼哼\ldots{}\ldots{}(冷笑介)}

\textbf{罗成
【西皮摇板】催命鼓来救命锣,阵前战败黑阎罗。豪杰打马过山坡,}

\textbf{徐勣 贤弟慢走!}

\textbf{罗成 【西皮摇板】那边来了徐三哥。}

\textbf{(徐勣上)}

\textbf{徐勣 【西皮摇板】扬鞭打马出唐营,见了贤弟说分明。}

\textbf{罗成 三哥!小弟有礼。}

\textbf{徐勣 贤弟!众家弟兄俱已降唐,惟有贤弟不降,是何道理?}

\textbf{罗成 小弟早有此心,只是无有引荐之人。}

\textbf{徐勣 愚兄愿作引荐之人。}

\textbf{罗成 不知二主,待将如何?}

\textbf{徐勣 贤弟呀!}

\textbf{徐勣
【西皮摇板】只要你真心来归顺,他与你皇兄御弟称。辞别贤弟把马乘,莫作三心二意人。}

\textbf{(徐勣上场门下)}

\textbf{罗成 【西皮摇板】好个仁义徐先生,顺说罗成降唐营。}

\textbf{罗成 众将官!}

\textbf{众 有}

\textbf{罗成 随爷降唐去者!}

\textbf{众 (内)啊!}

\textbf{(罗成上场门下)}

\textbf{{[}第八场{]}}

\textbf{(\textless{}扫头\textgreater{},四红文堂引徐勣、程咬金、秦琼、李世民上,尉迟恭在\textless{}扫头\textgreater{}中上)}

\textbf{尉迟恭
二主,某正要擒那罗成下马,被这牛鼻子老道,鸣金收兵。你要罪他,你要与我怪他!}

\textbf{程咬金 嘿,我说老黑呃!耀武扬威的,敢是得了胜了?}

\textbf{尉迟恭 这个------嘿!}

\textbf{程咬金
嘿,败了!哼,卖不了的秫秸,你那边戳戳}\protect\hyperlink{fn331}{\textsuperscript{331}}\textbf{吧。}

\textbf{李世民
【西皮摇板】一见尉迟败了阵,回头埋怨徐先生:明知罗成多骁勇,不该命他去出兵!}

\textbf{徐勣
【西皮摇板】二主休要罪微臣,把话说与尉迟听:你出兵好似一只虎,}

\textbf{尉迟恭 某家好比一只猛虎!}

\textbf{程咬金 你呀,嘿,墙上爬着的蝎拉虎子!}

\textbf{尉迟恭 猛虎!}

\textbf{程咬金 你呀,蝎拉虎子。}

\textbf{尉迟恭 猛虎!}

\textbf{程咬金 得得得,就算是猛虎。}

\textbf{徐勣 【西皮摇板】恨不得把罗成一口吞。}

\textbf{尉迟恭 着哇!某恨不得把那娃娃吞吃在腹内!}

\textbf{程咬金 你呀,诶,嘴大嗓子眼小,你咽不下去。}

\textbf{尉迟恭 吞吃在腹内!}

\textbf{程咬金 你咽不下去!}

\textbf{尉迟恭 吞吃腹内!}

\textbf{徐勣 【西皮摇板】只杀得襆头戴不稳,}

\textbf{尉迟恭 先生,那是风吹歪了的。}

\textbf{程咬金
哼,你呀,哼,那是我把弟爷给你挑歪了}\protect\hyperlink{fn332}{\textsuperscript{332}}\textbf{。}

\textbf{尉迟恭 风吹歪了的!}

\textbf{程咬金 枪挑歪了的。}

\textbf{徐勣 【西皮摇板】乌骓马倒退不前行。}

\textbf{尉迟恭 先生,那是罗成的马!}

\textbf{程咬金
哼,我那把弟呀,骑的是白龙马,你呀,骑的是那个------蛤蟆。}

\textbf{尉迟恭 罗成的马。}

\textbf{徐勣 【西皮摇板】慈悲大士救八难,缘何不救某难中人!}

\textbf{尉迟恭 先生,那是罗成讲的呀。}

\textbf{程咬金 嘿,你说的,我听见了!}

\textbf{尉迟恭 罗成讲的!}

\textbf{程咬金 你说的!}

\textbf{徐勣 【西皮摇板】不用杀来不用战,点点手儿唤罗成。}

\textbf{尉迟恭 某家不信。}

\textbf{徐勣 【西皮摇板】不信与我来击掌!}

\textbf{尉迟恭 击掌?请!}

\textbf{程咬金
我说老黑呀!你先前就打过赌,输了个脑袋了;你还打赌啊,别不害臊啦!}

\textbf{尉迟恭 嘿!}

\textbf{(尉迟恭坐大边)}

\textbf{徐勣
【西皮摇板】谅你不敢赌输赢。回头启奏二主君:有罪罗成降唐营。}

\textbf{徐勣 罗成降唐。}

\textbf{李世民 宣他进帐!}

\textbf{徐勣 二主有令:罗成进帐!}

\textbf{(罗成上)}

\textbf{罗成 【西皮摇板】催动坐骑到唐营,}

\textbf{秦琼 贤弟!}

\textbf{罗成
【西皮摇板】唐营中来会众宾朋。别的宾朋我不问,二哥是我姑表亲。}

\textbf{(秦琼出门)}

\textbf{罗成 【西皮摇板】二哥报门小弟进,}

\textbf{(秦琼报门)}

\textbf{秦琼 报,罗成告进!}

\textbf{(罗成挖进去,站小边)}

\textbf{罗成 【西皮摇板】有罪罗成降唐营。}

\textbf{李世民 【西皮摇板】只要你真心来归顺,封你为越国公永在朝门。}

\textbf{罗成 【西皮摇板】叩罢头来谢罢恩,}

\textbf{(罗成坐小边)}

\textbf{罗成 【西皮摇板】那旁坐的对头人。}

\textbf{尉迟恭 啊!}

\textbf{尉迟恭
【西皮摇板】一见罗成讨了封,不由某家怒气升。唐营中哪有你的座,}

\textbf{尉迟恭 站过去!}

\textbf{(\textless{}快长锤\textgreater{}里,尉迟恭拉罗成,换坐位,罗成归大边,尉迟恭归小边)}

\textbf{尉迟恭 【西皮摇板】这座让与俺鄂国公,哼!}

\textbf{程咬金 【西皮快板】程咬金怒不息,骂声敬德你把人欺!}

\textbf{尉迟恭 你才欺人!}

\textbf{程咬金 【西皮快板】曾记得美良川鞭对斧,你鞭鞭打在我胸窝里。}

\textbf{尉迟恭 某家我服了你了。}

\textbf{程咬金 你服我什么?}

\textbf{尉迟恭 我服你呀,好捱挨打哦。}

\textbf{程咬金 我也服了你!}

\textbf{尉迟恭 服我何来?}

\textbf{程咬金 服你好狠心哦!}

\textbf{尉迟恭 只是便宜了你!}

\textbf{程咬金 便宜了你!}

\textbf{程咬金 【西皮快板】不看罗成你看在我。}

\textbf{尉迟恭 唐营之中好大的一个你呃!}

\textbf{程咬金 哎,好大的一个你呃!}

\textbf{程咬金 【西皮快板】岂不知我们是把兄弟。}

\textbf{尉迟恭 哎,你们是猪兄狗弟!}

\textbf{程咬金 哎,龙兄虎弟。}

\textbf{尉迟恭 猪兄狗弟!}

\textbf{程咬金 我说二哥啊,把弟啊,咱们大伙打他。}

\textbf{程咬金 【西皮摇板】唐营中哪有你的座?}

\textbf{程咬金 你站站呐!你站站呐!}

\textbf{(程咬金揪尉迟恭,尉迟恭不起来)}

\textbf{程咬金
【西皮快板】你不起来我不依。扭回头搬是非,叫声二哥听端的:他不欺罗成欺的是你,你不骂他我不依。}

\textbf{秦琼
【西皮摇板】听一言来心好恼,大骂黑贼听根苗:曾记得美良川鞭对锏,三鞭两锏斗英豪。你三鞭打不动秦叔宝,俺两锏打得你望影而逃。唐营中哪有你的座?}

\textbf{秦琼 站过去!}

\textbf{(\textless{}快长锤\textgreater{}里,秦琼拉尉迟恭,尉迟恭与罗成换坐位,尉迟恭归大边,罗成归小边)}

\textbf{秦琼 【西皮摇板】这座让与我表弟英豪。}

\textbf{秦琼 贤弟,进得唐营,为何这等懦弱?}

\textbf{罗成 小弟初进唐营,不敢造次。}

\textbf{秦琼 待愚兄教导与你。}

\textbf{程咬金 是啊,二哥你教给他!}

\textbf{秦琼
俺罗成一不降唐,二不归顺;多蒙仁义徐先生,顺说俺来降,二主见爱,赐俺一个座位,又被廊下的匹夫占去。要杀抬枪,要打何惧!}

\textbf{罗成
唐营众将听者:俺罗成一不降唐,二不归顺;多蒙仁义徐先生,顺说俺来降。二主见爱,赐俺一个座位,又被廊下匹夫占去。要杀------抬枪,要打------何惧!}

\textbf{尉迟恭 呔!罗成你要打哪个?}

\textbf{罗成 我要打你。着打。}

\textbf{(罗成打尉迟恭)}

\textbf{尉迟恭 哎呀!}

\textbf{尉迟恭 【西皮摇板】唐营中俱是他猪兄狗弟,}

\textbf{程咬金 我们都是龙兄虎弟。二哥,把弟!咱们打他!打他!}

\textbf{尉迟恭 哎!}

\textbf{尉迟恭 【西皮摇板】惟有俺尉迟恭是个外来的。}

\textbf{程咬金 你呀,咳,你好比茄子地里长蒺藜,嘿!坏种独苗大紫包。}

\textbf{尉迟恭 哼!}

\textbf{尉迟恭 【西皮摇板】怒气不息打进去!}

\textbf{程咬金 干什么啊?}

\textbf{尉迟恭 要打你。}

\textbf{程咬金 讲打?一个打一个,哼,不算好朋友,二哥,把弟,打他!打他!}

\textbf{尉迟恭
【西皮摇板】险些逼起一窝蜂。豪杰低头进大营,把话说与二主听。}

\textbf{尉迟恭 二主!当初御果园有难,何臣保驾?}

\textbf{李世民 小王失记。}

\textbf{尉迟恭 为臣全记。}

\textbf{李世民 奏来!}

\textbf{尉迟恭 容奏呃!}

\textbf{尉迟恭
【西皮摇板】御果园中曾有难,口口声声叫尉迟皇兄。有了罗成忘了臣,有了新臣忘旧臣。怒气不息出大营,把话说与众三军。}

\textbf{尉迟恭 三军的!降唐人马,随爷反出大营!}

\textbf{(尉迟恭又坐大边,向下场门叫三军)}

\textbf{李世民 呀!}

\textbf{李世民
【西皮摇板】一见尉迟反唐营,倒教小王吃一惊。走向前来忙跪定,}

\textbf{(李世民跪在尉迟恭座位前,众齐跪)}

\textbf{李世民 【西皮摇板】俱是皇兄御弟称。}

\textbf{(众搀扶李世民起,众起)}

\textbf{众人 臣等和睦!}

\textbf{李世民 后帐摆宴,与众卿贺功!}

\textbf{众人 谢主公!}

\textbf{(\textless{}尾声\textgreater{}。众人同下)}

\textbf{(罗成、尉迟恭坐左右``虎头椅'')}

\newpage
\hypertarget{ux5babux95e8ux5e26ux5341ux9053ux672c-ux4e4b-ux674eux6e0aux891aux9042ux826f}{%
\subsection{宫门带·十道本 之
李渊、褚遂良}\label{ux5babux95e8ux5e26ux5341ux9053ux672c-ux4e4b-ux674eux6e0aux891aux9042ux826f}}

\textbf{{[}第一场{]}}

\textbf{李渊
【二黄慢板】都只为御梓童命归仙境,因此上为王的染病在身。内侍臣搀扶王龙床安定,还需要却烦虑静养精神。}

\textbf{(李世民 【二黄摇板】内侍摆驾进龙廷,父王台前问安宁。)}

\textbf{(李世民 儿臣见驾,父王万岁!)}

\textbf{李渊 皇儿平身。}

\textbf{(李世民 万万岁!)}

\textbf{李渊 赐座。}

\textbf{(李世民 谢座。)}

\textbf{李渊 皇儿进宫为了何事?}

\textbf{(李世民 儿臣在太医院,取得太平汤药,进宫与父王熬煎。)}

\textbf{李渊 我儿真乃孝道(或:孝心)。}

\textbf{(李世民 内侍,金炉伺候!)}

\textbf{(李世民
【二黄原板】父王将息龙床养,儿臣进宫煎药汤。屈膝跪在尘埃地,拜天拜地拜三光。但愿父王身无恙,焚香顶礼谢上苍。)}

\textbf{李渊
【二黄原板】儿孝心感动天和地,药下咽喉病离身。空养建成、元吉子,并不进宫问安宁。日后为父(或:为父日后)归仙境,儿就是东宫的守阙人。谯楼鼓打三更时分,}

\textbf{李渊 【二黄摇板】皇儿回避(或:暂且)出宫廷。}

\textbf{(李世民 【二黄摇板】辞别父王出宫门,)}

\textbf{(李世民 【二黄摇板】为何还有作乐声。)}

\textbf{(李世民 \ldots{}\ldots{}明白便了!)}

\textbf{(李世民 【二黄摇板】听谯楼鼓打三更尽,看是何人作乐声。)}

\textbf{{[}第二场{]}}

\textbf{李渊 【西皮摇板】宫中服药精神爽,悼念御妻神暗伤。}

\textbf{(张妃、刘妃 万岁呐!)}

\textbf{李渊 梓童为何这等模样?}

\textbf{(张妃、刘妃 今有二主秦王,二更二点进宫调戏我二人。万岁做主!)}

\textbf{李渊 世民素行仁孝,为王(或:孤王)不信。}

\textbf{(张妃、刘妃 \ldots{}\ldots{}玉带为证。)}

\textbf{李渊 呈上来。}

\textbf{(张妃、刘妃 万岁请看。)}

\textbf{李渊 哎呀!}

\textbf{李渊
【西皮摇板】一见玉带怒气生,胆大奴才乱宫廷。你二人暂且回宫禁(或:后宫进),}

\textbf{(张妃、刘妃 【西皮摇板】\ldots{}\ldots{}世民丧残生。)}

\textbf{李渊 【西皮摇板】内侍摆驾金殿进,快宣皇儿李世民。}

\textbf{(李世民 【西皮摇板】忽听父王宣世民,急忙上殿问分明。)}

\textbf{(李世民 儿臣见驾,父王万岁!)}

\textbf{李渊 儿是世民?}

\textbf{(李世民 是世民。)}

\textbf{李渊 好奴才!}

\textbf{李渊
【西皮摇板】把儿当作擎天柱,奴才竟是忤逆人。吩咐两旁武士手,推出午门问斩刑。}

\textbf{(李世民
【西皮摇板】一言未发来问斩,教我有话不敢言。因何将儿推出斩,说明儿死也心甘。)}

\textbf{李渊 【西皮摇板】奴才不必将父问,现有玉带作证凭。}

\textbf{(李世民 【西皮散板】却原来为的是联珠带,)}

\textbf{(李世民 父王,父王,父王啊,呃\ldots{}\ldots{}(哭介))}

\textbf{(李世民
【西皮散板】吓得三魂少二魂。本当说出二兄长,又恐伤了手足情。望父王饶了儿的\textless{}哭头\textgreater{}命,父王啊,还望看在父子情。)}

\textbf{李渊
【西皮散板】手摸胸膛想一想,此事可行不可行。吩咐殿前(或:吩咐两旁)武士手,推出午门问典刑。}

\textbf{(李世民 【西皮摇板】含悲忍泪下龙廷,看是何人把本升。)}

\textbf{(长孙无忌 刀下留人!)}

\textbf{(武士 啊!)}

\textbf{(长孙无忌 【西皮摇板】迈步撩袍上龙廷,品级台前臣见君。)}

\textbf{(长孙无忌 臣长孙无忌见驾,吾皇万岁!)}

\textbf{李渊 上殿有何本奏?}

\textbf{(长孙无忌 二主秦王身犯何罪,\ldots{}\ldots{}午门问斩?)}

\textbf{李渊 蠢子不正,扰乱宫廷,故而问斩。}

\textbf{(长孙无忌 想秦王有十大汗马功劳,只可一赦,不可一斩。)}

\textbf{李渊 孤王龙心已定,定斩不赦。}

\textbf{(长孙无忌 万岁呀!)}

\textbf{(长孙无忌
【西皮摇板】当年驾坐太原省,隋炀帝无道灭人伦。二主大战王世充,才保我主坐龙廷。)}

\textbf{李渊 呃!(或:唗!)}

\textbf{李渊
【西皮摇板】无忌奏本太欺情,敢在金殿藐寡人。吩咐殿前(或:两旁)武士手,他与奴才同罪名。}

\textbf{李渊 绑了下去!}

\textbf{(长孙无忌 【西皮摇板】\ldots{}\ldots{},看是何人保我生。)}

\textbf{{[}第三场{]}}

\textbf{(徐勣
【西皮摇板】一见秦王上了刑,不由徐勣心内惊。\ldots{}\ldots{}忙往龙殿奔,)}

\textbf{(秦琼、程咬金 先生慢行。)}

\textbf{(秦琼、程咬金 【西皮摇板】见了先生礼相迎。)}

\textbf{(徐勣 二公慌慌张张为了何事?)}

\textbf{(秦琼、程咬金
二主秦王不知身犯何罪,推出午门斩首。我等上殿保本。)}

\textbf{(徐勣 此本你我保不下来。有人来了,你我暂退朝房便了。)}

\textbf{(徐勣 【西皮摇板】三人一同朝房进,)}

\textbf{褚遂良 (内)先生慢走!}

\textbf{(徐勣 【西皮摇板】那旁来了褚先生。)}

\textbf{(褚遂良 反了哇,反了呃!)}

\textbf{褚遂良
【西皮散板】听说要斩二主君呐,斩断了(或:斩坏了)擎天柱一根。万岁不听(或:万岁不准)忠良本,长孙无忌问斩刑。这都是二奸妃用计狠,谁知我主假作真。}

\textbf{褚遂良 哎呀!}

\textbf{褚遂良
这,这\ldots{}\ldots{},罢!(或:这,这,这\ldots{}\ldots{}哎呀!罢!)}

\textbf{褚遂良
【西皮散板】歪戴乌纱斜插带,假装疯魔去见君。大摇大摆金殿进,}

\textbf{褚遂良 【西皮散板】与他个君不君来臣不臣。}

\textbf{褚遂良 臣,褚遂良见驾,吾主万岁,万万岁!呃呃呃,请了!}

\textbf{李渊
呃嗯------胆大褚遂良,上得殿来衣冠不整,莫非你疯了?(或:呃嗯------卿家莫非你疯了?)}

\textbf{褚遂良
呃,臣倒不曾疯啊,只恐(或:只怕)万岁你昏了。二主秦王身犯何罪,推出午门斩首?}

\textbf{李渊 奴才扰乱宫廷,因此斩首!}

\textbf{褚遂良
想二主秦王,东挡西杀,南征北剿,有十大汗马功劳。将他斩首,君心何忍,这臣心何安呐?!}

\textbf{褚遂良
【西皮快板】想当年驾坐太原郡,三搜晋阳才为君。二主大战王世充,瓦岗寨收下众英雄。美良川,收敬德,千秋岭下收罗成。大唐收了罗世信,才保我主坐龙廷。挣来的江山多安稳,为何要斩创业人。}

\textbf{李渊 呃嗯------胆大褚遂良,上殿言君之过,绑了!}

\textbf{褚遂良 万岁!臣有十道条陈,容臣奏完,(诶,)再斩不迟。}

\textbf{李渊 呈上龙案,寡人御览。}

\textbf{褚遂良 臣修本不及,乃是口奏。}

\textbf{李渊 奏来。}

\textbf{褚遂良
容奏:臣这第一道条陈奏的是夏禹王坐了一十七代,四百五十八载。后出一君,名曰桀王,宠爱一妃,名唤妹喜。那桀王听信妹喜之言,以酒为池,以肉为林,忠臣良将,俱已遭害呀。}

\textbf{褚遂良
【西皮快板】自古道有道反无道,汤王定计安黎民。南巢岭桀王丧了命,只落得江山一旦倾。}

\textbf{李渊 大胆褚遂良,毁谤孤王。武士手,绑了!}

\textbf{褚遂良
啊,万岁!臣只奏过一道,还有九道未奏啊,容臣奏完,诶,再斩不迟。}

\textbf{李渊 奏来!}

\textbf{褚遂良
容奏:臣这第二道条陈奏的是成汤王得了桀王天下,传至三十一代。后出一君,名曰纣王,宠爱一妃,名叫妲己。他驾前有两个谗臣,一名费仲,一名尤浑。那纣王听信妲己之言,盖一楼名曰``摘星(楼)''。造下炮烙之刑,糟害百姓(或:残虐百姓)。比干丞相剖心而亡,贾氏夫人坠楼而死,姜后娘娘挖目剁手(或:剜目剁手),东宫太子一旦逐出,黄家父子反出五关。到后来姜尚兴兵伐纣,可叹那纣王啊,只落得火焚摘星楼台而亡。万岁,你看他也是宠爱奸妃的无道昏君喏。}

\textbf{李渊 呃------}

\textbf{李渊
【西皮摇板】褚遂良奏本孤心恨,把孤比作无道君。寡人至德平天下,学尧舜不差半毫分。}

\textbf{李渊 再将三道条陈奏来!}

\textbf{褚遂良
容奏:臣这道条陈奏的是周朝。那周文王得了纣王天下,后出一君,名曰幽王,宠爱一妃,名曰褒姒,生得(是)面貌如花。怎奈进宫以来,
永无笑容。那幽王无计可施,他驾前有一谗臣,名叫尹球。是他奏道:万岁要娘娘发笑不难。在骊山设宴,火焚烟墩。那幽王听信尹球所奏,就在骊山设宴,火焚烟墩。
各路诸侯见烟墩火起,想必国家有难,一个个顶盔贯甲,兵临城下。观见他君妃在楼台饮酒取乐哇,一个个乘兴而来,(是)败兴而返呐;那褒姒一见是呵呵地大笑。后来犬戎作乱,那幽王又将烟墩点起。各路诸侯言道:想必(又是)他君妃(又)在那里饮酒取乐啊,你我各保汛地}\protect\hyperlink{fn333}{\textsuperscript{333}}\textbf{要紧。(一个个是按兵不动。啊)万岁,你看那幽王为褒姒一笑不值紧要啊,失落周室家邦,他还死在了乱军之中。}

\textbf{褚遂良
【西皮快板】幽王无道掌乾坤,骊山设宴焚烟墩。各路诸侯无救应,江山一旦化灰尘。}

\textbf{李渊 幽王无道,戏耍诸侯,提他则甚?再将四道条陈奏来。}

\textbf{褚遂良
容奏:臣这道条陈奏的是东周列国,周惠王驾前有一家诸侯,名曰晋献公。他(或:那晋献公)宠爱一妃名唤骊姬。前妃所生二子,长子申生,次子重耳。那骊姬在献公面前搬动是非,要害(那)申生太子一死,那献公是执意地不听呐。骊姬一计不成,又生二计。用蜂蜜擦头,到御花园观花(或:采花),命申生太子保驾采花。蜜蜂围绕头上,申生太子不解其意,在后面用扇搧开。那献公在楼台之上观见,言道:这奴才果有戏母(或:残母}\protect\hyperlink{fn334}{\textsuperscript{334}}\textbf{)之心。吩咐殿前武士,将(或:把)申生太子推出午门问斩。来在午门,众大臣拦路言道:
千岁(你)有满腹含冤,为何不奏知你父王?申生太子言道:我若奏知我父王,我父王大怒,必将骊姬斩首。斩了骊姬不关紧要哇,有日我父王思想骊姬成病,岂不是小王之罪?小王只可一死,不做那不忠不孝之人。万岁,为臣看来,二主秦王与前朝申生太子一般无二。}

\textbf{(太监 着啊!)}

\textbf{(李渊 呃嗯------)}

\textbf{李渊
【西皮摇板】晋献公本是无道君,听信谗言斩亲生(或:听信谗言斩申生)。世民本是不肖子,淫乱宫闱问斩刑。}

\textbf{李渊 再将五道条陈奏来!}

\textbf{褚遂良
呃呃,臣这道条陈奏的是楚平王在临潼斗宝,多亏伍子胥力举千斤鼎,压定各国为下邦。到后来秦、楚结亲,(那)楚平王闻得(或:楚平王闻听)无祥女生得是天姿国色,有意纳妾,怎奈儿媳不好启齿啊。他驾前有一谗臣,名叫费无极,奉旨(前)往秦国迎亲。行至在钟离山前,用金顶轿改换银顶轿,无祥女改换马昭仪。好个伍子胥,保定皇家四口反出昭关,去往吴国借兵。可叹那平王(啊,他)死后,只落得鞭尸三百有余。}

\textbf{褚遂良
【西皮快板】楚平王本是无道君,父纳子妻乱人伦。子胥后来【转西皮摇板】发人马,鞭尸三百留骂名。}

\textbf{李渊 那父纳子妻,乃(是)酒色昏王,提他则甚?再将六道条陈奏来。}

\textbf{褚遂良
呃,呃,臣这六道条陈奏的是姑苏吴王宠爱一妃,名曰(西施或:名唤西施)。他驾前有一谗臣,名唤伯嚭。那吴王听信西施、伯嚭之言,起造一台,名曰姑苏台。挑选天下出色的女子,去往楼台饮酒取乐。到后来勾践兴兵前来,只杀得那吴王有家难奔,有国难投哇。}

\textbf{李渊 呃------嗯。}

\textbf{李渊
【西皮摇板】姑苏吴王无道君,听信谗言选红裙。越王勾践发人马,吴国从此不太平。}

\textbf{李渊 再将七道条陈奏来。}

\textbf{褚遂良
容奏:臣这道条陈奏的是齐宣王在桑园射猎,收来一妃,名唤无盐,手持春秋大棒(或:春秋大棍)压定各国。只因宠爱一妃,名曰夏迎春。那无盐娘娘身怀六甲,那夏迎春讨下收生代劳的旨意,用金丝狸猫剥去皮尾。启奏大王言道:那无盐娘娘产生妖魔鬼怪。齐宣王大怒,将无盐娘娘推出斩首,多亏满朝文武保奏,打入冷宫。后来吴起伐齐,只落得跪门求救哇。}

\textbf{褚遂良
【西皮摇板】齐宣王本是无道君,宠爱奸妃夏迎春。后来吴起发人马,只落得跪门去求兵。}

\textbf{李渊 齐宣王宠妃害贤,怎比孤王?将八道条陈奏来!}

\textbf{褚遂良
容奏:臣这道条陈奏的是齐湣王宠爱一妃,名曰邹赛花。他驾前有一宦官名叫伊立。那湣王听信邹妃之言,要害东宫太子一死。后来乐毅兴兵,前来追赶湣王。赶得他有家难奔,有国难投。只落得日晒湣王,路剐邹妃(或:路卧}\protect\hyperlink{fn335}{\textsuperscript{335}}\textbf{邹妃)。万岁!这就是前朝宠妃灭子的报应喏!}

\textbf{李渊 呃------}

\textbf{李渊
【西皮摇板】宠妃灭子害忠臣,他将湣王比寡人。待等奏完十道本,定与奴才同罪名。}

\textbf{李渊 再将九道条陈奏来。}

\textbf{褚遂良
容奏:臣这道条陈奏的是前朝杨广欺娘奸妹,败坏人伦,后来亡国丧身。}

\textbf{褚遂良
【西皮摇板】杨广本是无道君,欺娘奸妹乱宫廷。五花棒奸王丧了命,才保我主坐龙廷。}

\textbf{李渊 将十道条陈奏完,孤王定要将你碎尸万段。}

\textbf{褚遂良 这个\ldots{}\ldots{}}

\textbf{(太监
我说褚先生,十道条陈奏了九道,只管奏来,自有咱家帮助于你呃。)}

\textbf{褚遂良 万岁,臣这十道条陈奏的是前朝君王与本朝皇帝一般无二。}

\textbf{李渊 唗!}

\textbf{李渊
【西皮摇板】褚遂良奏本孤心恨,道道条陈刺寡人。吩咐殿前武士手,推出午门问斩刑。}

\textbf{褚遂良 冤枉------}

\textbf{武士 褚遂良冤枉。}

\textbf{李渊 召回来!}

\textbf{武士 啊!}

\textbf{褚遂良 谢万岁不斩之恩。}

\textbf{李渊 非是寡人不斩于你,为何口喊冤枉?}

\textbf{褚遂良 臣有一事不明,要在万岁驾前领教!}

\textbf{李渊 何事不明?}

\textbf{褚遂良 二主秦王什么时候进宫?}

\textbf{李渊 一更一点。}

\textbf{褚遂良 什么时候煎汤熬药?}

\textbf{李渊 二更二点。}

\textbf{褚遂良 什么时候出宫?}

\textbf{李渊 三更三点。}

\textbf{褚遂良
二位娘娘(或:皇娘)奏道,抓袍夺带是什么时间(或:什么时候)?}

\textbf{李渊 这个\ldots{}\ldots{}}

\textbf{(太监 二更二点。)}

\textbf{褚遂良
着哇,二主秦王,一更一点进宫,二更二点煎汤熬药,三更三点才得出宫。二位娘娘(或:二位皇娘)奏的是二更二点,这不是(或:岂不是)冤枉吗?}

\textbf{李渊 现有玉带,拿去看来。}

\textbf{褚遂良 待臣看来。}

\textbf{褚遂良
万岁,我想这抓袍夺带,必定是你这一拉,我这一扯。这玉带之上并无一点伤损,岂不是大大的冤枉么?}

\textbf{李渊 寡人看来(或:待孤看来)。}

\textbf{褚遂良 万岁请看。}

\textbf{李渊 嘿嘿!}

\textbf{褚遂良 嘿嘿!}

\textbf{李渊 【西皮摇板】寡人如醉方才醒,险些错斩李世民。}

\textbf{李渊
【西皮摇板】孤王急忙下龙廷,手提羊毫写分明:一赦皇儿李世民,二赦长孙无忌卿。忙将赦旨交与你,快到法场走一程。}

\textbf{褚遂良
【西皮散板】手捧赦旨下龙廷,笑坏了两班文武臣。文班中笑坏了徐勣先生,武班中笑坏了叔宝、咬金二位(或:众位)将军。都道我褚遂良不怕死。}

\textbf{褚遂良 哈哈,哈哈,啊呵呵哈哈哈\ldots{}\ldots{}(笑介)}

\textbf{{[}第四场{]}}

\textbf{(李世民 【西皮摇板】父王传旨斩世民,)}

\textbf{(长孙无忌 【西皮摇板】听信谗言斩忠臣。)}

\textbf{(李世民 【西皮摇板】忍泪含悲法场进,)}

\textbf{(长孙无忌 【西皮摇板】两眼睁睁等时辰。)}

\textbf{褚遂良 赦旨下。}

\textbf{(武士 赦旨下。)}

\textbf{(李世民 接旨。)}

\textbf{褚遂良 圣旨下。跪!)}

\textbf{(李世民、长孙无忌 万岁!)}

\textbf{褚遂良
听宣读。诏曰:只因孤王误听谗言,错斩皇儿李世民与国舅长孙无忌。多亏褚遂良保奏,将他二人赦回金殿加封。旨意读罢(或:旨意读奏),望诏谢恩。}

\textbf{(李世民、长孙无忌 万万岁!)}

\textbf{褚遂良 请过圣命。}

\textbf{(李世民 有劳先生保奏。)}

\textbf{褚遂良 保本来迟,千岁恕罪。}

\textbf{(李世民 岂敢。)}

\textbf{褚遂良 一同上殿交旨。}

\textbf{(李世民 请!)}

\textbf{{[}第五场{]}}

\textbf{李渊 (念)可恨奸妃做事错,平白无故起风波。}

\textbf{褚遂良 (念)忙将赦旨事,启奏万岁知。}

\textbf{褚遂良 启万岁:二主千岁、长孙无忌宣到。}

\textbf{李渊 宣他二人冠带上殿!}

\textbf{褚遂良 二人冠带上殿!}

\textbf{(李世民 (念)法场得活命,)}

\textbf{(长孙无忌 (念)死而又复生。)}

\textbf{(李世民 儿臣李世民\ldots{}\ldots{})}

\textbf{(长孙无忌 臣长孙无忌,)}

\textbf{(李世民、长孙无忌 谢万岁不斩之恩。)}

\textbf{李渊 皇儿、国舅平身。赐座。}

\textbf{(李世民 谢座。)}

\textbf{李渊 长孙无忌为皇儿误受一绑,加升三级,免朝一月。下殿!}

\textbf{(长孙无忌 谢万岁!)}

\textbf{李渊 褚遂良上殿听封。}

\textbf{褚遂良 臣有本启奏。}

\textbf{李渊 奏来!}

\textbf{褚遂良 臣不愿加官封赠。}

\textbf{李渊 愿者何来?}

\textbf{褚遂良 请我主差哪部大臣,将宫中查明。}

\textbf{李渊 赐座。}

\textbf{褚遂良 谢座。}

\textbf{李渊 皇儿,你二姨母怎样害你(或:怎生害你),一一奏来!}

\textbf{(李世民 父王啊!)}

\textbf{(李世民
【二黄原板】未开言不由人珠泪滚滚,尊父王听儿臣细说分明:二皇兄与姨母行事不正,儿戏君妃乱胡行。儿本当进宫细查问,又恐失了手足情。因此上将玉带宫门挂定,这就是一桩桩一件件父王详情。)}

\textbf{李渊
【二黄原板】劝皇儿休得要【转二黄快三眼】珠泪滚滚,为父的心中明如灯:将二妃贬至在冷宫禁,她自羞自惭自丧生。为江山儿何曾略得安静,为江山东挡西除、南征北剿未享安宁。今日里儿活命实称万幸,改日里过府去酬谢先生。武德君迈虎步忙下九重,}

\textbf{(褚遂良跪)}

\textbf{李渊
【二黄快三眼】用手儿挽定了褚先生(或:搀扶起褚先生)。满朝中文武臣袖手不问,怎当得先生你赤胆忠心。为皇儿把卿家的心血用尽,为皇儿哪顾得费尽辛勤。为皇儿在朝房一番议论(或:一番争论),为皇儿可算得擎天柱一根。(为皇儿把卿家的心血用尽,为皇儿哪顾得费尽辛勤。)为皇儿假装作疯魔急病,为皇儿衣冠不整来见当今。为皇儿把君臣大礼全然不论,为皇儿哪顾得舍死忘生。为皇儿连奏过十道表本,为皇儿把夏桀与商纣、前朝后代历代的昏王一代一代比与孤听。加封你吏部大堂带管那都察院,太子少保伴君正卿(或:太子少保外加正卿;或:太子少保陪伴寡人)。再赐你尚方剑如山压定,【垛板】压定了九卿四相、满朝文武、大小的官员哪一个不遵,先斩后奏启奏寡人,你是捍国}\protect\hyperlink{fn336}{\textsuperscript{336}}\textbf{的良臣。}

\textbf{褚遂良
【二黄原板】非是臣我不愿(或:我不爱)加官封赠,为的是我主锦乾坤。从今后主休听宫闱谗本(或:主休听宫中谗本),普天下众黎民乐享太平,都道你是(海不扬波是一个)有道明君。}

\textbf{李渊
【二黄原板】好一个孝道李世民,赤胆忠心褚先生。孤的皇儿残生性命亏你救应,命皇儿与先生结为师生。侍内臣把酒宴宫中摆定,孤与那皇儿、先生来压惊。左手带定世民子,右手带定褚先生。孤的皇儿李世民,孤的爱卿褚先生,你本是皇儿的恩人、孤的爱卿。劝皇儿休流泪、免悲声,放大胆一步一步随定寡人。(或:孤的皇儿李世民,孤的爱卿褚先生,你二人一步一步随定寡人。)}

\textbf{取帅印}\protect\hyperlink{fn337}{\textsuperscript{337}}

{[}第一场{]}

(吹\textless{}\textbf{点绛唇})\textgreater{}

徐勣 (念)朝臣待漏月坠西,

尉迟恭 (念)文臣武将整朝衣。

程咬金 (念)金钟玉磬连声响,

徐勣、尉迟恭、程咬金 (同念)三跪九叩拜丹墀。

徐勣 山人徐勣。

尉迟恭 鄂国公敬德。

程咬金 鲁国公咬金。

徐勣 列公请了!

尉迟恭、程咬金 请了!

徐勣
今有张士贵,在绛州龙门,招军已满,有本回朝。少刻万岁登殿,一同启奏。

徐勣、尉迟恭、程咬金 看,香烟缭绕,圣驾临朝,分班伺候。请!

李世民 {[}引子{]}海宴河淸,喜的是,四海升平。

徐勣、尉迟恭、程咬金 臣等见驾,吾皇万岁!

李世民 平身。

徐勣、尉迟恭、程咬金 万万岁!

李世民 赐座。

徐勣、尉迟恭、程咬金 谢座!

李世民
(念)父王宴驾命归天,孤王接位掌江山。征扫北国回朝转,可恨辽东起狼烟。

李世民
孤,李世民。国号贞观在位。父王宴驾,众卿保孤登基。可恨辽东盖苏文,打来连环战表,教寡人御驾亲征。是寡人命张士贵,在绛州龙门,招军集将,王君可监造战船。二人出京,数月有余,并无本章回朝,教孤日夜忧虑也。

徐勣 臣启万岁:今有张士贵有本还朝,请我主龙目御览!

李世民 呈上来!

李世民
【西皮原板】日出扶桑万道霞,群臣歌颂帝王家。张环有本奏陛下,请主龙目细详察。\protect\hyperlink{fn338}{\textsuperscript{338}}奉王旨意招人马,英雄投效到王家。并无有仁贵投帐下,

李世民 【西皮摇板】再与先生把话答。

李世民
先生,张士贵绛州龙门招军,为查访应梦贤臣。这本章上面,并无``仁贵''二字,张环莫非有欺君之意?

徐勣 张环焉敢欺君。万岁此番征东,若无贤臣保驾,臣之罪也。

李世民 秦恩公染病在床,先生保奏何人?

徐勣 臣保尉迟恭挂帅,征伐辽东,一战成功。

程咬金 万岁,休听军师之言,尉迟恭挂不得帅印。

尉迟恭 程将军,军师保我挂帅,你为何拦阻?想我开国元勋不挂,谁人能挂?

程咬金 得了罢,动不动就是开国元勋,难道我老程就不是开国元勋吗?
\protect\hyperlink{fn339}{\textsuperscript{339}}

尉迟恭 你不能。

程咬金 哼,我不能,那你也不能啊。

李世民
且慢!二卿不必争论,帅印现在秦府,就命程皇兄前去取印回来,再作定夺。

程咬金 臣领旨。

李世民 转来!

程咬金 臣在。

李世民 他乃有病之人,必须见机而行。听孤旨下!

李世民
【西皮摇板】恩公投唐功劳大,东挡西除定邦家。虽然卧病在床榻,雄心依然保唐家。卿家要说温柔话,随机应变把印拿。

程咬金 领旨!

程咬金
【西皮摇板】万岁叮咛一席话,为臣一一转秦家。辞王别驾把殿下,背转身来自咂牙。

程咬金
哎呀且住。我想这颗帅印,乃是秦二哥执掌多年;我若取回,岂不白白地送与黑贼之手?呃呃呃,我有了!我不免午门闲游一番,急回谎奏,就说二哥染病在床,昏迷不醒,他不肯交印。也免得那个黑贼痴心妄想也!

程咬金 【西皮摇板】急回谎奏君王驾,痴心妄想不归他。

黄门官 【西皮摇板】春城无处不飞花,随王并无半日暇。

黄门官 臣黄门官见驾,吾皇万岁!

李世民 平身。

黄门官 万万岁!

李世民 上殿有何本奏?

黄门官 今有王君可,有本回朝。我主龙目御览!

李世民 呈上来,待孤观看!

李世民
【西皮摇板】奉王旨意到海下,王君可修本奏皇家。请主旨意发人马,扫平辽东定中华。看罢本章记心下,黄门官近前听根芽。吩咐众将免见驾,三日之后候旨发。

黄门官 领旨!

黄门官 【西皮摇板】辞王别驾把殿下,晓谕文武百官家。

程咬金 【西皮摇板】胸藏妙计说假话,急忙上殿我骗皇家。

程咬金 臣交旨。

李世民 赐坐。

程咬金 谢座。

李世民 秦恩公病体如何?

程咬金 照常一样,呕吐不止。

李世民 唉,恩公啊\ldots{}\ldots{}(哭介)

李世民
【西皮摇板】恩公病势不见佳,不由孤王泪如麻。东挡西除功劳大,病体缠身难挣扎。

程咬金 他乃久病之人,万岁何必忧虑。

李世民 他乃有功之臣,倘有不测,孤心不忍。

程咬金 万岁真乃有道明君。

李世民 取印之事如何?

程咬金
万岁休要提起取印之事。为臣走进了病房之间,言道:二哥,你病体如何。他说:照常一样,呕吐不止。是臣言道,万岁因你染病在床,龙心悬念,命我前来探望于你。他说真是有道明君。他又问起为臣,呃,征东一事如何。臣言:万岁今日设立早朝,张士贵有本回朝,招军已满。万岁因你染病在床,龙心未定。军师力保尉迟恭挂帅。他听说尉迟恭挂帅,哼,是一派的好埋怨\protect\hyperlink{fn340}{\textsuperscript{340}}啊。

李世民 埋怨何来?

程咬金
呃,呃,他言道:想我秦琼,自投唐以来,攻无不胜,战无不取,才挣下,呃,这颗帅印。如今染病在床,倘有不测,呃,还有我儿怀玉呀。再一说,还有咬金兄弟,也是文武双全,可以挂得帅印。想那尉迟恭,与我秦琼,并无半点瓜葛之情。况且他目不识丁,何能决胜千里之外?呵呵,他一派的好埋怨呐!

徐勣 哼,你一派说谎。

程咬金 嘿,你又不曾听见,你怎么知道是谎言呢?

程咬金 臣见他那般光景呵!

程咬金
【西皮摇板】气力不佳难讲话,病势未减只又加。为臣一见心害怕,万岁龙心细详察。

李世民
【西皮摇板】孤王闻言头低下,心中辗转泪如麻。辽东若不去征伐,定说孤王惧怕他。

李世民 先生,秦恩公昏迷不醍,不肯交印,如何是好?

徐勣 万岁明日过府,一来探病,二取帅印。

程咬金
呵,万岁,休听军师之言,取印乃是一桩小事,不论差哪部大人前去也就是了,何劳御驾亲往!

尉迟恭 程将军,君入臣门,蓬荜生辉,你为何拦阻?

徐勣 你呀,真是多口!

程咬金 哎呀!你们俩呀,嘿嘿,打成了合同了。

程咬金 臣启万岁:呃,臣好有一比。

李世民 比作何来?

程咬金 掌上的乌鸦,呵,我开不得口啊!

李世民 开口便怎样?

程咬金
开口便是祸。方才说了一句话,一个,道臣拦阻,一个,道臣多口。明日过府,呃,必须有几句言语要讲;不讲,呃,反倒得罪他们,我还是不去的为妙啊。

李世民
呃,你与秦恩公昔年结为好友,只管大胆,保孤前去。有什么祸事,寡人与你担待。

程咬金 我说三哥哟,这可是万岁教我去的。呃,我这可是奉了旨的了!

徐勣 真乃一张油口。

程咬金 哼,我又油口了。

李世民 听孤旨下!

李世民
【西皮摇板】昔日恩公走天涯,锏打杨广救全家。明日文武齐保驾,孤王亲自去看他。

程咬金
【西皮摇板】黑贼金殿夸大话,军师一旁暗保他。就是帅印归他挂,也教他口念活菩萨呀。

尉迟恭
【西皮摇板】军师金殿抬爱咱,咬金一旁把话答。若是帅印归我挂,寻一良谋摆布他。

徐勣
【西皮摇板】适才咬金一席话,蒙哄万岁弄巧牙。辞王别驾把殿下,明日保主到秦家。

李世民
【西皮摇板】龙楼凤阁紫雾霞,金殿祥光绕瑞华。内侍与孤摆銮驾,探望功臣到秦家。

{[}第二场{]}

(程咬金 (内)掌灯。)

(程家院引程咬金\textless{}\textbf{小锣打上}\textgreater{})

程咬金 【西皮摇板】无事关心心不乱,有事关心心不安。

程咬金
老夫程咬金。适才在金殿用花言巧语,蒙哄万岁。可恨那个牛鼻子老道,奏了一本,请万岁明日过府,一来探病,二来取印。哎呀,我想这件事,二哥可是一概不知呀。倘若万岁明日过府,问起情由,二哥一概不知,我岂不是有蒙君之罪么。我不免连夜过府,与二哥送上一信。倘若明日万岁问起情由,也免得二哥临时,诶,失于机变。

程咬金 家院。

程家院 有。

程咬金 掌灯秦府。

程咬金
【西皮摇板】谋事在人成事天,心中恼恨黑炭丸。一心要放暗中箭,摆布黑贼有何难。

{[}第三场{]}

(秦家院、秦怀玉搀秦琼上)

秦琼
【西皮摇板】投唐保国扶江山,东挡西除马上眠。自从扫北回朝转,疾病缠身整一年。眼观红日西山晚,

(秦琼入大座)

秦琼 【西皮摇板】心中焦躁不耐烦。

(秦琼睡介,程咬金上)

程咬金
【西皮摇板】万岁金殿把旨传,晓谕文武众两班。明日过府把病探,怕的泄漏这机关。

程家院 来在(或:来到)秦府。

程咬金 前去通禀,鲁国公求见。

程家院 门上哪位在?

秦家院 做什么的?(或:什么人?)

程家院 鲁国公求见。

秦家院 候着。

秦家院 启少爷:鲁国公求见。

秦怀玉 启爹爹:程叔父到。

秦琼 怀玉迎接!

秦怀玉 遵命。

秦怀玉 【西皮摇板】怀玉出了府门前,见了叔父礼当先。

秦怀玉 迎接叔父。

程咬金 哎,罢了,罢了。怀玉,你父病体如何?

秦怀玉 照常一样,呕吐不止。

程咬金 你父今在何处?

秦怀玉 现在病房。

程咬金 带路病房啊!

程咬金 【西皮摇板】来在(或:来到)病房用目看,

程咬金 唉(或:哎呀)!

程咬金 【西皮摇板】看见二哥病容颜。重病缠身容颜变,精神恍惚非当年。

程咬金 二哥醒来。

秦琼 【西皮摇板】适才朦胧将合眼,耳旁又听有人言。猛然睁开昏花眼,

程咬金 二哥。

秦琼 【西皮摇板】只见贤弟在眼前(或:只见贤弟在面前)。

秦琼 贤弟来了,请坐。

程咬金 嘿,谢坐哟。请问二哥,(你)病体如何(呀)?

秦琼 照常一样,呕吐不止。

程咬金 你乃久病之人,何须忧虑。

秦琼 啊,贤弟连夜过府(或:夤夜到此)何事?

程咬金
哎呀二哥呀,小弟有一件为难之事(或:要紧之事)。二哥,你要依我才是啊。

秦琼 哎,贤弟,你我自结金兰,患难相顾。今日有何为难之事,愚兄依你就是。

程咬金
我说是啊,二哥啊,你有所不知啊。张士贵绛州龙门招军已满,有本回奏(或:有本回朝)。万岁见你染病在床,无人挂帅,是龙心未定啊。可恨那个牛鼻子老道,诶,他保尉迟恭挂帅呀。

秦琼 哦,那尉迟恭么\ldots{}\ldots{}

程咬金 嘿,是哦。

秦琼 哼,一介村夫,况且目不识丁,焉能(够)掌得帅印。

程咬金
是啊。小弟就是因为此事,与黑贼跟------诶,军师争论了几句,圣上命我前来取印。我想这颗帅印,二哥你执掌了多年,我若取回(或:我要取回),岂不白白地送给黑贼之手么。那时小弟,在午门闲游了一番,急回谎奏(或:急忙谎奏)。呃,实指望说几句言语,蒙哄万岁;诶嘿嘿,谁知那个牛鼻子老道,又启奏了一本:明日过府,一来探病,二来取印。我想这些个事,二哥啊,你可是一概不知啊,我特来通报与你。二哥啊,要是圣驾过府,问起取印之事,二哥啊,你就说小弟我来过了,周全小弟无罪。想你我弟兄,自投唐以来嚯!

程咬金
【西皮摇板】算来倒有数十年,并未分首各一天。生死相交共患难,这件事儿要周全。

秦琼 贤弟。

秦琼
【西皮摇板】可笑(或:堪笑)万岁见识浅,听信军师入蜚言(或:宠信军师入蜚言)。

秦琼 贤弟,只管放心,些许小事,由我担承。

程咬金 是,谢二哥。

秦琼 怀玉,取印过来!

秦怀玉 遵命。帅印在此!

秦琼 放在床前。

秦琼 夜已经深了,贤弟回府去罢。

程咬金
二哥,明日那黑贼保驾前来,你必须用言语摆布于他。小弟的言语,你可要牢牢地紧记。我告辞啦!

程咬金 【西皮摇板】辞别二哥回身转,

秦琼 怀玉代送。

程咬金 【西皮摇板】猛然想起巧机关。

秦怀玉 送叔父!

程咬金 哎,我说怀玉,你可知你父的病体,是因何而得呢?

秦怀玉 唉,前者在金殿赌力而得。

程咬金
嘿,这可就不对啦。为叔的我要不说,你哪儿知道哇。昔年大战美良川,三鞭换两锏。你父在马鞍鞒上,气堵胸膛,故而成病,至今才发。我要是不说呀,你这一辈子也不明白(或:也不知道)。

秦怀玉 依叔父之见?

程咬金
依我之见呐,明日圣驾过府探病,那黑贼一定是保驾前来。你父用言语摆布于他,他必然叫骂你父啊,你听见之后啊,就只管地打他。

秦怀玉 哎呀,他乃是开国元勋,侄儿怎敢呐。

程咬金
他是开国元勋,那你父就不是开国元勋吗?为叔我------诶,那也不是开国元勋吗?哼。

秦怀玉 侄儿打他不过呀。

程咬金
诶,小小的年纪,就说这种软弱的话。你打不过,不碍事啊,为叔父的,诶,我帮着你。

秦怀玉 哦,侄儿遵命。

程咬金 你要记下了!

程咬金
【西皮摇板】昔年大战美良城,三鞭两锏赌输赢。明日只管将他打,打出祸来有我老程(或:打出祸来我担承)。

秦怀玉
【西皮摇板】适才叔父对我言,为的当年旧仇冤。为子当把父仇报,暗藏心机不漏言。

{[}第四场{]}

(吹\textless{}\textbf{牌子}\textgreater{},大铠等引李世民、徐勣、尉迟恭、程咬金上)

秦怀玉 怀玉接驾!

李世民 你父病体如何?

秦怀玉 照常一样,呕吐不止。

李世民 前去通禀,孤王前来探病。

秦怀玉 领旨。

秦怀玉 (念)君入臣门第,蓬荜又生辉。

(秦怀玉下)

李世民 程皇兄,吩附銮驾,府外伺候!

程咬金 銮驾府外伺候哇。

(大铠等众下)

李世民 尉迟皇兄!

尉迟恭 万岁。

李世民 少时秦恩公问起征东之事,孤必命卿挂帅。他乃有病之人,你必须忍耐。

尉迟恭 领旨。

秦怀玉 启万岁:臣父叫之不应,请驾回宫。

李世民 孤是为你父病而来,再去通禀,孤在前厅等候。

秦怀玉 领旨。

(秦怀玉下)

李世民 唉!皇兄啊!

李世民 【西皮摇板】孤摆銮驾到府门,

(李世民站起)

李世民
【西皮摇板】亭台以外柳青青。山水古画多齐整,厅前瑞草送芳馨。君臣且在前厅等,

(李世民坐下)

李世民 【西皮摇板】等候怀玉报信音。

秦怀玉 【西皮摇板】君入臣门多侥幸,龙行一步百草生。

秦怀玉 臣启万岁:臣父昏迷不醒,叫之不应,请驾回宫。

李世民 平身。

李世民 先生,秦恩公昏迷不醒,如何是好?

徐勣 万岁请至(或:请到)病房,将他唤醒。

李世民 怀玉,

秦怀玉 在。

李世民 带路病房!

秦怀玉 领旨。

李世民 【西皮摇板】恩公昏迷不得醒,去至病房看真情。

(\textless{}\textbf{大锣打下}\textgreater{})

{[}第五场{]}

(秦家院搀秦琼上)

秦琼 【西皮摇板】昨日咬金来送信,果然今日驾临门。假装昏迷睡不醒,

(秦琼入大座,睡介)

秦琼 【西皮摇板】且看万岁怎样行(或:圣上怎样行)。

(秦怀玉引李世民、徐勣、尉迟恭、程咬金上)

李世民 【西皮摇板】孤王亲自来探病,功劳打动帝王心。君臣且把病房进,

(众挖门进)

李世民 【西皮摇板】只见恩公睡沉沉。

秦琼
【西皮散板】适才朦胧将睡定(或:适才朦胧荏苒\protect\hyperlink{fn341}{\textsuperscript{341}}动),耳旁又听有人声。猛然睁开昏花眼,抬头只见圣明君。

秦琼 怀玉,圣驾到此,为何不来通报?

秦怀玉 孩儿呼唤爹爹不醒,未敢惊动。

秦琼 哼------只恐儿难免欺君之罪。

李世民 啊皇兄,此乃寡人自进,与小爱卿无干。

秦琼 若非圣上开恩,定要加罪。还不谢过万岁!

秦怀玉 谢万岁!

李世民 平身。

秦琼 万岁驾到臣门,奈臣有病,不能接驾。万岁恕罪!

李世民 皇兄,你乃有病之人,谁来怪你。

秦琼 谢万岁!

李世民 孤今前来问病,但不知你病体如何?

秦琼 照常一样,呕吐不止。

李世民 你乃久病之人,且免忧虑。

秦琼
微臣久病,日加沉重,今见万岁一面,再不能朝见的了,呃\ldots{}\ldots{}(哭介)

尉迟恭 元帅,某这几日有朝事在身,少来问候。今日保驾前来,问候元帅金安。

秦琼 有劳台驾。

程咬金
二哥,您听见了没有?他说这几天有朝事在身,少来问候。今日保驾前来,捎带着给你问个好儿。这个人情,诶,你可得领他的。

徐勣 你又来多口!

程咬金 嘿,我又------哼,多口了。

秦琼 有劳众位皇兄前来问病,请坐。

徐勣、尉迟恭、程咬金 有座。

秦琼 万岁征东一事如何?

李世民
孤王昨日设立早朝,张士贵有本回朝,招军已满;王君可海下战船造齐,二人俱有本章还朝。孤见恩公染病在床,无人挂帅,孤心未定。

秦琼 万岁征东事大,奈臣,唉,这病\ldots{}\ldots{}(哭介)

秦琼
【西皮原板】疾病缠身整一春,吉凶二字解不明。残生难逃幽冥境,再不能替主扫烟尘。

李世民
【西皮原板】孤王闻言心酸痛(或:泪双淋),好似狼牙箭攒心。恩公有病难挂印,有何人替孤领雄兵。

秦琼
【西皮原板】臣子怀玉年纪轻,文韬武略智超群。胸中颇有安邦论,可命他替主(或:挂帅)统雄兵。

李世民
【西皮原板】怀玉虽然有本领,年幼怎能压(或:年幼怎能服)老臣。况且皇兄身有病,还要膝下奉晨昏。

秦琼
【西皮原板】甘罗十二为宰臣,无志空活百岁龄。怀玉年幼难挂帅,有何人替主掌权衡。

李世民
【西皮原板】昨日金殿【转西皮快板】同议论,公保尉迟老将军。因此为王来取印,即日兴兵往东征。

秦琼 啊?!

秦琼
【西皮快板】听说尉迟挂帅印,急得我心头似火焚。尉迟恭有勇无学问,焉能挂帅统雄兵。非是为臣抗君命,军师分明你错用了人呐。

徐勣 【西皮摇板】休道万岁错用人,

徐勣
【西皮快板】病缠有力不从心(或:病缠身力不从心)。将军染病一年整,无人挂印掌权衡。尉迟恭暂挂元帅印,剿灭辽东盖苏文(或:扫平辽东盖苏文)。且等将军病安稳,再往辽东扫烟尘。我主龙心似尧、舜,岂负你开国老元勋。

尉迟恭 元帅。

尉迟恭 【西皮摇板】元帅息怒且消停,

尉迟恭
【西皮快板】末将言来听分明:你双锏打下唐社稷,(某)单鞭挣下锦乾坤。末将虽然无学问,可以挂帅领雄兵。征伐辽东干戈定,元帅大印付将军(或:元帅大印还将军)。非是某有意来夺印,元帅还要三思行。

程咬金 老黑。

程咬金
【西皮快板】黑贼休要逞舌能,我等俱是有功臣。投唐国公三十六,咬金也能我统雄兵。

尉迟恭 你不能。

程咬金 我不能啊,你也不能。

尉迟恭 你不能。

程咬金 诶,你也不能。

李世民 且慢。

李世民
【西皮快板】二位皇兄免争论,俱是开国老元勋。尉迟皇兄能挂印,程皇兄也能领雄兵。太平原是将军定,原是将军定太平。

李世民
【西皮快板】回头来把话论,尊一声恩公听分明:四海的狼烟俱扫尽,不伏辽东盖苏文。恩公不肯让帅印,征东的事儿(或:征东大事)孤去不成。

秦琼 【西皮导板】狼烟一起主亲征呐,

秦琼 【西皮摇板】怎敢违抗圣明君。

秦琼
\textless{}(\textbf{三})\textbf{叫头}\textgreater{}万岁!我主!(唉,万岁啊!)

李世明
\textless{}(\textbf{三})\textbf{叫头}\textgreater{}恩公!皇兄!(唉,恩公啊!)

秦琼
万岁此番征东,三年不知,五载不晓。主在边庭,臣在京内(或:阃内)。臣不能见君,君不能见臣。臣子怀玉,尚未授职,又无婚配;臣病入膏肓,不久于人世。(或:臣病入膏肓,不久于人世;臣子怀玉,尚未授职,又未婚配。)臣纵死九泉,唉!也是不能瞑目的了哇,呃\ldots{}\ldots{}(哭介)

李世民 哦!(或:呀!)

李世民
【西皮二六】孤王闻言心酸痛,句句言辞记龙心(或:记在心)。倘若是皇兄遭不幸,细听孤王加荣封:追封王位归正品,儿孙代代(或:子子孙孙)入朝门。孤有公主银屏女,赐与怀玉配为婚。众家国公为媒证,即日里婚配驾登程,老皇兄请放宽心。

秦琼
【西皮摇板】叔宝闻言心安稳,纵死九泉也甘心呐。在枕边谢恩把首顿,转面再谢众公卿。

秦琼 怀玉。

秦琼
【西皮二六】叫怀玉近前来听父命,万岁爷的金言要谨记心。倘若是为父遭不幸,追封王位葬至在山林。我的儿子袭父职品,银屏公主配儿为婚。列位国公为媒证,即日里婚配驾启程。这就是(或:这都是)圣上面应允,儿要三跪九叩谢王(的)恩。

秦怀玉 【西皮摇板】怀玉床前遵父命,

秦怀玉
【西皮快板】含悲忍泪谢皇恩。恩赐子婿父极品,食王爵禄当报恩。叩罢头来抽身起,

秦怀玉
【西皮快板】转面再谢众公卿。小侄在朝受皇恩,还要叔父好看承。倘有一点不到处,休怪怀玉乱胡行。

程咬金 儿啊!

程咬金
【西皮快板】我儿只管任意行,凡事有我程咬金。你父身归蓬莱境,我的儿,你是这当朝的驸马,你还怕何人呐。

秦琼 呃------

秦琼
【西皮快板】咬金不必逞舌能,不使良言\protect\hyperlink{fn342}{\textsuperscript{342}}训子孙。怀玉(或:我儿)在朝受皇恩,大事全仗徐先生。怀玉年轻须教训,念在当年结义的情。怀玉请过元帅印,

秦琼
【西皮快板】再听为父细叮咛:后堂酒宴安排整(或:安排定),先敬君来后敬臣。

秦怀玉 【西皮摇板】怀玉床前遵父命,准备酒宴好看承。

秦琼
【西皮快板】见床前(或:见床头)摆列元帅印,见物情伤好惨心(或:惨凄)。咽喉紧哽话难尽,

秦琼 【西皮摇板】再叫尉迟猛将军。

秦琼 尉迟将军!

尉迟恭 元帅。

秦琼 万岁此番征东,可是挂你为帅?

尉迟恭 正是末将为帅。

秦琼 既然挂你为帅,你可晓得为将帅之道?

尉迟恭 身为武将,焉有不知为帅之道。

秦琼 好,今当万岁金面,你且讲来!

尉迟恭
元帅容禀(或:元帅听了):为帅之道,必须盔缨灿烂,铠甲鲜明;刀枪锋利,金鼓齐鸣(或:锣鼓齐鸣);安营巩固(或:安营坚固),谨守大营;擂鼓而起(或:擂鼓而进),鸣金收兵。战马须要强壮,上阵观看动静。众将不能取胜,某就单鞭匹马------我就杀------杀入万马营中。三合九战,方可收兵。这才是为帅的之道。

秦琼 (真乃是)一派的胡言!

程咬金 嘿,简直是放屁(或:如同放屁)呀!

秦琼 跪近床前,待本帅教导于你!

尉迟恭 元帅有何金言,当面请教,何必要跪!

秦琼 一定要跪!

尉迟恭 谁来跪你!

程咬金 要我挂帅啊,嘿,跪一辈子我也跪(或:跪一辈子我也干呐)!

李世民 啊尉迟皇兄,看孤份上,你就跪他一膝(或:你且屈膝)。

尉迟恭 嗯,臣(或:某家)这里跪下了。

程咬金 嘿,我说老黑诶,你要跪,就两条腿都跪下罢,这条腿干什么呐!

(程咬金踹介,尉迟恭跪)

秦 琼
听道:为将帅者,必须饱读经纶,深通战策;运筹帷幄之中,决胜千里之外呀。行兵不可马踏青苗;将士不可扰害百姓。逢高山莫先登,
遇空城莫乱入。高防围困,低防水淹;森林防埋伏,芦苇防火攻。身为元帅,令不乱传。有功嘉赏,有过责罚。有道是:(念)朝中天子三诏宣,阃外将军令出山岳动,这言发鬼神惊\protect\hyperlink{fn343}{\textsuperscript{343}}。渴饮刀头血,困卧马上眠。受得苦中苦,方为人上人。这都是为将帅之道(或:此乃是为将帅之道),须要牢牢谨记!

(程咬金 一点儿没病。)

尉迟恭 末将全记。

秦琼 帅印在此。

尉迟恭 拿来。(或:鲁莽了。)

秦琼
唗!想这帅印,乃圣上钦赐与我,理当交还万岁(或:交还圣上)。你是甚等样人(或:尔是甚等样人),竟敢前来夺取帅印,真正是无羞无耻,匹夫之辈(或:真正是匹夫之辈,毫不知羞耻)也!

秦琼
【西皮快板】不记得当年美良城,三鞭两锏赌输赢。不看万岁、先生面,定要叉出帅府门。

尉迟恭 【西皮快板】可恨秦琼太欺心,藐视尉迟有功臣。怒气不息到前厅,

程咬金
【西皮摇板】程咬金与你个腚后跟\protect\hyperlink{fn344}{\textsuperscript{344}}。

秦怀玉 【西皮摇板】准备酒宴多齐整,特请万岁饮杯巡。

秦琼
【西皮摇板】请主后堂把宴饮,军师伴驾饮杯巡。(或:后堂酒宴安排整,请先生陪驾饮杯巡。)

(秦琼作揖,佯睡介)

李世民 【西皮摇板】辞别恩公把宴饮,

(李世民、徐勣出来,李世民指印,徐勣抱印,李世民下)

徐勣 【西皮摇板】后堂奉陪天子君。

(徐勣下,秦琼醒,出座)

秦琼
【西皮摇板】好个有道圣明君,赛似(或:亚似)尧、舜掌龙庭。可叹(或:嗟叹)秦琼身染病,不能保主\protect\hyperlink{fn345}{\textsuperscript{345}}扫烟尘。

(秦琼下)

{[}第六场{]}

程咬金 【西皮摇板】黑贼前厅怒不息,咬金正好搬是非。迈步且进二堂里,

秦怀玉 【西皮摇板】见了叔父问端的。

程咬金 怀玉你慌慌张张,是为了何事?

秦怀玉 请尉迟赴席。

程咬金 嘿嘿,是呀?他骂你的父可骂出理来了!

秦怀玉 他骂我父何来?

程咬金
嘿,他骂你父啊,病不死的老牛精。他说一颗帅印,让与不让,但凭于你,为何当着众家的国公,羞侮于我呀。从前你血气方刚,可以耀武扬威;如今你咽喉啊,就只有一口虚气喽,还待这样的性傲啊。嘿嘿,教你死在阴山背后,永不翻身。这场骂哟!

秦怀玉 叔父之言,侄儿不信。

程咬金 哎呀,为叔父这大的年纪,还跟你撒谎啊。

秦怀玉 依叔父之见?

程咬金 嘿,还是昨晚那话呀,忘了啊?打他呀!

秦怀玉 哎呀,他乃是开国元勋,侄儿不敢打呀。

程咬金
哼,他是开国元勋呐,嘿,如今晚儿,你可就是当朝驸马了。你只管地打他,打完了,嗯,还得教他给你赔个礼儿。

秦怀玉 为何与侄儿赔礼呀?

程咬金
嘿,你听我告诉你,他必然在前厅,叫骂你父。你悄悄地走在他的背后,给他来个饿虎扑食啊,劈拳就打。那时为叔父的去至后堂,把万岁给请来。你听我再咳嗽这么一声,诶,赶紧翻身在地,百般地你就喊叫。

秦怀玉 喊叫什么?

程咬金
你就说,诶,儿臣好好请尉迟恭赴宴,谁知他以大压小,将儿臣暴打了一顿。若非父王到此啊,孩儿性命可有亏。那时候为叔就一句话,他就得给你赔礼。

秦怀玉 哪一句话?

程咬金 我说老黑诶,你敢打当朝驸马,犹如欺君之罪。你说他给你赔礼不赔礼啊?

秦怀玉 自然要赔礼。

程咬金 嘿,你要记下了!

秦怀玉 遵命!

秦怀玉 【西皮摇板】叔父之言牢谨记,顷刻之间打尉迟。

程咬金 【西皮摇板】娃娃中了我的计,管教老黑儿他暗吃亏。

{[}第七场{]}

尉迟恭 走哇!

尉迟恭 哇啊啊\ldots{}\ldots{}

尉迟恭
【西皮摇板】恼恨秦琼太无礼,一阵火起\protect\hyperlink{fn346}{\textsuperscript{346}}往上提。将身且坐前厅椅,气坏当朝老尉迟。

尉迟恭 \textless{}\textbf{叫头}\textgreater{}秦琼啊!匹夫!

尉迟恭
一颗帅印让与不让,但凭于你。从先血气方刚,可以耀武扬威;如今咽喉只有一口虚气,还是这样地性傲。我把你这病不死的老牛精!

秦怀玉 着打!

(尉迟恭、秦怀玉扭打介)

程咬金 嘿嘿,老黑呀,你当着万岁,你还打他呐!

程咬金
起来,起来,起来,起来\ldots{}\ldots{}嘿,有事别哭,别哭!有什么话,你只管地讲啊。

秦怀玉 \textless{}\textbf{叫头}\textgreater{}父王!

秦怀玉
儿臣好意请尉迟恭赴席,谁想他以大压小,将儿臣暴打一顿。若非父王到此,唉,儿的性命有亏啊\ldots{}\ldots{}(哭介)

尉迟恭 哎呀万岁呀,他打了老臣了!

程咬金 怎么着?他打了你了,诶,那你怎么在上头,他在底下呢?

尉迟恭 哎呀!这个\ldots{}\ldots{}喳,喳,喳,喳\ldots{}\ldots{}

程咬金 哎呀,我说老黑诶,你敢打当朝驸马呀,这就是欺君之罪呀!

李世民 着哇!

程咬金 你摸摸,还有脑袋吗?

李世民 唗!

李世民 【西皮摇板】孤王闻言怒冲起,

李世民
【西皮快板】开言大骂黑面皮。你本是堂堂国公体,全然不知高和低。怀玉本是东床婿,打他犹如把孤欺。罚俸三载赎你罪,快与驸马把罪赔。

尉迟恭 \textless{}\textbf{叫头}\textgreater{}万岁。

尉迟恭
【西皮快板】万岁容臣本奏启,细听为臣辩是非:为臣坐在前厅椅,他背后将椅往下推。就势将臣推在地,脊背打得响如雷。他见万岁来到此,翻身在地假悲啼。花言巧语奏万岁,反说为臣把他欺。臣本是堂堂的国公体,岂与他无知少年把罪替。

程咬金 我说怀玉呀,你可别哭,有什么话,你倒是说呀。

秦怀玉 \textless{}\textbf{叫头}\textgreater{}父王。

秦怀玉
【西皮快板】父王在上容奏启,细听儿臣辩是非:我父功劳谁能比,盖世忠良属第一。临潼山,把功立,双锏保驾把名提\protect\hyperlink{fn347}{\textsuperscript{347}}。如今病在牙床里,来在帅府夺帅旗。兵权大印让与你,反来叫骂无礼仪。怀玉虽然小年纪,出山的猛虎抖毛衣。万岁招我东床婿,当今驸马谁不知。以大压小把我欺,你不赔礼我不依。

尉迟恭
【西皮快板】娃娃说话太无礼,花言巧语你骂谁。你父与我同一辈,论什么高来论什么低。他双锏打来唐社稷,某单鞭挣下锦华夷。你父临潼把功立,御果园单鞭救驾回。你父功劳谁能比,某的功劳也不亏。花言巧语就是你,要想我赔礼日出西。

程咬金 你呀,赔礼罢!

尉迟恭 我不能!

程咬金 嗨嗨\ldots{}\ldots{}打红眼了,我可不惹你!

徐勣 尉迟呀!

徐勣
【西皮快板】尉迟恭暂忍心头气,我有一言听端的:怀玉虽然得罪你,大能容小休要提。为人休得心生气,又无烦来又无非。你执意不肯去赔礼,有道是君王有命怎能违。

尉迟恭 先生。

尉迟恭
【西皮快板】先生说话大有理,背转身来自猜疑。马行夹道难回避,船到江心补漏迟。罢罢罢,暂忍心头气,保主征东挂帅旗。走向前来我忙赔礼,

程咬金
嘿,怀玉你看呐,你看瞧尉迟恭这么大的岁数,跪在你的面前,你呀,饶了他罢!

秦怀玉 呵,我饶恕于你!

程咬金 嘿,老黑诶!你瞧瞧咯,这人情可是我讲的。

尉迟恭 怎么,你讲的?诶,我这儿谢谢你喽!

尉迟恭
【西皮快板】驸马爷宽宏休要提。万岁驾前告过罪\protect\hyperlink{fn348}{\textsuperscript{348}},臣不该把他少年欺。

李世民
【西皮摇板】一见尉迟来赔礼,满天浮云一扫归。怀玉近前听旨意,安排花烛结光辉。孤王回到昭阳去,急送\protect\hyperlink{fn349}{\textsuperscript{349}}公主出宫闱。

(李世民、徐勣、尉迟恭、程咬金分下)

秦怀玉 怀玉送驾!

{[}第八场{]}

李世民 尉迟皇兄,挂你为帅,当殿谢过。

尉迟恭 领旨。

李世民 程皇兄,你为三十六路都先锋,带领八十三万人马,去至海口扎营!

尉迟恭、程咬金 谢万岁!

\newpage
\hypertarget{ux6c7eux6cb3ux6e7e-ux4e4b-ux859bux4ec1ux8d35}{%
\subsection{汾河湾 之
薛仁贵}\label{ux6c7eux6cb3ux6e7e-ux4e4b-ux859bux4ec1ux8d35}}

{[}第一场{]}

(四将起霸,发点,四文堂站门,薛仁贵上)

\textless{}\textbf{点绛唇}\textgreater{}跨海征东,英名远震(或:威名远震)。军威盛,扫荡烟尘,保主锦绣春。

(薛仁贵入大座)

(念)忆昔跨海去征东,拔山举鼎显异能。可恨张环行毒计,埋没英雄汗马勋。

本爵薛仁贵,(乃)山西绛州龙门县人氏。只因保定唐王跨海征东,立下十大汗马功劳,唐王见喜,封俺为平辽王之位(或:封我为平辽王之职)。只因我离家日久,不知妻室怎生度日,为此辞王别驾,回家探望。来此离家不远,我不免改换行装,以免惊动乡邻。

中军(或:左右),看衣更换。

(\textless{}合龙\textgreater{},吹\textless{}\textbf{牌子}\textgreater{},薛仁贵换衣,分开)

中军听令。

(中军 在。)

传令下去,(吩咐)大小三军就在此地靠山近水安营扎寨,不可踏践青苗,扰害百姓。(或:不可骚扰百姓,马踏青苗,违令者斩。)

(中军 \ldots{}\ldots{}传令已毕。)

(众下)

(与爷)带马------

(薛仁贵扯四门中唱)

【西皮原板】忆昔当年去投军,张士贵是我的对头人。打虎遇着程千岁,他带我仁贵见了当今。卖弓计打破了摩天岭,三枝神箭辽东平。\protect\hyperlink{fn350}{\textsuperscript{350}}前三日修下了辞王本,

【西皮摇板】回家探望妻迎春。

(\textless{}\textbf{撤锣}\textgreater{}薛仁贵下;王禅老祖上)

(王禅老祖 (念)\ldots{}\ldots{})

(王禅老祖唤虎童,下;盖苏文魂上,过场)

(盖苏文 【西皮摇板】\ldots{}\ldots{}驾起阴风朝前走,要报当年一箭仇。)

(盖苏文魂下,接柳迎春上)

{[}第二场{]}

马来!

【西皮摇板】催马来在汾河湾,见一顽童打弹丸。弹打,弹打南来宾鸿雁,

枪挑呃,

【西皮摇板】枪挑鱼儿水浪翻。翻身下了马雕鞍,再与顽童把话言。

那一顽童在此作何玩耍?

一弹上去,能打几雁落地?

为军的不信呐。

好,你且打来。

呜哙呀,小小年纪有此本领,我若将他带回朝去,(将来)定是大大膀臂。

我自有道理。

啊,顽童。

弹打双雁落地,不足为奇。为军的(不才,呃,)我也会打雁呐。

(薛丁山 一弹上去,能打几雁落地?)

一弹上去,我能打三雁落地。(或:呃呃,我能打三雁落地。)

(薛丁山 我却不信。)

(哦,)打来你看呐。

(薛丁山 你且打来。)

(呃------呃,)借弓弹一用。

且住!南山之上下来猛虎,有伤顽童之意。身旁带有袖箭,不免伤它一箭。

呔!顽童闪开,猛虎来了!看箭!\protect\hyperlink{fn351}{\textsuperscript{351}}

唉呀!

【西皮摇板】打虎误伤顽童命,是非之地莫久停呐。仁贵拉马朝前奔(或:朝前进)。

{[}第三场{]}

马来!

【西皮快板】适才离了(或:适才路过)汾河境,一马儿来在柳家村。勒住丝缰来观定,

【西皮快板】见一个妇人在窑门(或:见一位妇人站窑门\protect\hyperlink{fn352}{\textsuperscript{352}})。布裙荆钗容貌整,看她好像柳迎春。翻身下马来询问(或:下了马能行),

(薛仁贵下马介)

【西皮快板】躬身施礼把话云。

大嫂请了。

(柳迎春 还礼。军爷敢是失迷路途。)

正是失迷路途,请问大嫂,此处可是柳家村么?

(柳迎春
\ldots{}\ldots{}这面也是柳家村,这面也是柳家村。\ldots{}\ldots{}问的是哪个?)

此地有一柳氏迎春,大嫂可晓得?

(柳迎春 \ldots{}\ldots{}问她做甚?)

大嫂有所不知,我与他丈夫同营吃粮。与她(或:托我)带来万金家书,故而动问。

(我那薛大哥言道:书信要面交本人。)

(柳迎春 不见本人呢?)

(原书带回。)

请便。

(柳迎春 啊军爷,与你打个哑谜你可晓得?)

这哑迷么?略知一二。

(柳迎春 这远------)

远在天边,不能相见。

(柳迎春 这近------)

哦!莫非你就是薛大嫂么?

哎呀呀,问来问去,问到本人的头上来了。

来来来,重见一礼呀。

(柳迎春 见过礼了。)

礼多人不怪呀。

大嫂请稍待。

哎呀且住,想我仁贵离家一十八载,不知她光景如何?

嗯,我自有道理。

啊大嫂,我实对你说了吧(或:我实对你讲了吧):我那薛大哥,在营中已欠我二十两银子(或:在营中借了我二十两银子),将大嫂你就卖与我了。

呃,呃,呃,我有婚书为证呐。

呃,慢来慢来,我看大嫂变脸变色,婚书诓到手中,三把五把扯碎,为军的岂不落一个人财两空啊?

(柳迎春 依你只见?)

呃,你我去至前村,大户人家,请上三老四少,同拆同观。

当真。

哪个骗你呀?

(柳迎春 【西皮导板】狠心的强盗啊\ldots{}\ldots{})

呵呵,她倒骂起来了哇!

哦,在哪里?

(柳迎春关门)

哎,妻呀!

【西皮摇板】叫声贤妻快开门,我是你丈夫薛仁贵转回程。

妻呀!

【西皮导板】家住绛州县龙门,

【西皮原板】薛仁贵好命苦无亲无邻呐。幼年间父早亡母又丧命,丢下了仁贵无处身存。常言道姻缘一线引,柳家庄上招了亲。你的父嫌贫(他的)心太狠,将你我二人赶出了门庭\protect\hyperlink{fn353}{\textsuperscript{353}}。夫妻们双双【转西皮快板】无投奔,破瓦寒窑\protect\hyperlink{fn354}{\textsuperscript{354}}暂安身。每日里窑中苦难尽,无奈何立志去投军。结交了兄弟们周青等,跨海征东把贼平。幸喜得狼烟俱扫尽,保定圣驾转回京。前三日修下了辞王的本,特地回来探望柳迎春。我的妻若还不肯信,来来来,算一算,连来带去十八春。

(柳迎春 \ldots{}\ldots{}薛郎,你好啊?)

我好,你可好啊?

你也不像从前了。这就是:(念)少年子弟江湖老。

(柳迎春 (念)红粉佳人白了头。)

彼此?

一样。

啊,啊,哈哈哈\ldots{}\ldots{}(笑介)

(柳迎春 你临行之前,你还讲过什么言语你可记得?)

我讲过什么(言语),我倒是记不起来了哇。

(柳迎春 想必是做官回来了。(或:你不做官是不回来的,必定是做了官了。))

(呃,)再(也)不要提起做官呐,早去三天也好,晚去三天也好哇。

(柳迎春 \ldots{}\ldots{}刚刚凑巧。)

呃,凑巧倒还凑巧哇,

只是做了一名马头军\protect\hyperlink{fn355}{\textsuperscript{355}}。

(柳迎春 哦,马头军?)

(或:正是。)

(柳迎春 但不知你有多大的品级?)

呵,大得很呐!若论这品级台位么,呃,少不得,(少不得)也有它个七、八、十来品呐。

(柳迎春 呃,做官有七、八、十来品,但不知掌管什么?)

妻呀,为丈夫在家的时节,我管些什么?

(柳迎春 与人家看马。)

我如今呐,还是与人家牵马(或:看马)------

(柳迎春 \ldots{}\ldots{}看马。)

和从前是一样啊。

呃,有心胸,

(柳迎春 \ldots{}\ldots{}你有志气。)

有志气。

我这个志气还小吗?

(柳迎春 喂呀\ldots{}\ldots{}(哭介))

呃,我不回来,你是盼我回来。我好容易回来了,你又是这样鼻子、脸子的。

好好好,我在家中住上三五天,呃,我还是出外啊。

呃,葬埋在龙头山。

何谓马头山?

呃,还是龙头山的受听呐。

龙头山,龙头山。(或:龙头山,龙头\ldots{}\ldots{})

(柳迎春 \ldots{}\ldots{}马头山。)

啊,妻呀,我那岳父岳母百年之后,葬埋在何处啊?

呵,你看你看,到了他们家就成了凤凰山了。

依我看来,呵,不叫作凤凰山呐。

要叫作穷苦山。

你想啊,我在家的时节,你就是这样的受苦;我(如今)出外一十八载,如今回来,你还是这样的受苦(或:你还是住在这个破窑)。你爹娘生下你这受苦的女儿,呃,岂不是叫作穷苦山么?

呃,这也是你家的坟地里的风水呀。

呃,呵呵呵,穷苦山呐。

呵,穷苦山,穷\ldots{}\ldots{}

呃,这我倒不晓得呀。

哦,你是为我哇。

唉,我在外面一十八载,(省吃俭用,)受尽风霜之苦------

哦,我为的是哪个啊?

我啊,我也是为的是你呀。

我不为你,还为这座破窑不成么?

呃------我乃是受尽风霜之人,你不要呕我哇。

你不要呕我哇。

呵。

噗。

薛礼呀薛礼,你真真地岂有此理(呀)!

你今日回得家来,乃是一桩喜事,你偏偏要(来)呕她。

哎呀你看你看,把她气得这个样儿。

不妨不妨,待我取出一件东西,教她来看看,她就不生气了。

啊妻呀,为丈夫与你带回来好东西来了。

哎呀,(你也)特以地挖苦了,不是这些(或:这般)物件呐。

你拿去看来。

你看仔细。(或:仔细看来。)

你呀,拿过来吧。

这是我保定唐王跨海征东,立下十大汗马功劳,唐王见喜,封我为平辽王之位(或:之职)。这就是平辽王的虎头金印呐!

砷黄铜\protect\hyperlink{fn356}{\textsuperscript{356}}?!像这样的砷黄铜你见过几块呀。

呵呵呵,你(呀,)不开眼呐。

砷黄铜不要看了。

还要看看?

但要小心了。(或:你要看仔细。)

哎呀!

你还是拿过来吧。

你要把(或:要将)我这平辽王吞吃在腹内呀。

(柳迎春 \ldots{}\ldots{}饿怕了。)

惭愧!

啊妻呀,为丈夫一路行来,有些口渴,有什么香茶取来一用。

用些什么? (好,与我取来。)

好好好,快些取来。

【西皮摇板】在长安何曾吃白水,此水难饮泼埃尘。

不用啊。

(妻呀,为丈夫)腹中有些饥饿,有什么好酒好饭,取来一用呃。

用些什么?

何谓鱼羹?

好好好,与我取来。

【西皮摇板】用手接过鲜鱼羹,

呃,

【西皮摇板】这样腥气实难闻。

不用了啊。

鞍马劳顿,身体困倦呐。

哦,怎么还有后窑?

好,快快打扫。(或:与我打扫)

我如今回来了。

啊?

【西皮摇板】听她言来自思忖,莫非相交有情人。(或:察言观色详其情,教人心中解不明。)出得窑去观动静,

【西皮摇板】窑外并无一个人。

【西皮摇板】将马拴在柳荫下,

【西皮摇板】鞍辔放在了地埃尘。

【西皮摇板】站在窑中来观定,

【西皮摇板】这只男鞋必有因。

且住!怪道她变脸变色,原来她有了外遇了!

呀呀呸!(柳氏啊柳氏,)你在你丈夫跟前露出马脚来了。

贱人,你与我走了出来呀!

(呀呸!)你自己做的事还要问我?

你呀,你就是与我------

死呃。(或:呃!)

要赃。

要双。

呵呵,怕无有你的赃证?!这不是你的赃?这不是你的证?

你就是与我\ldots{}\ldots{}

唉!

(柳迎春 \ldots{}\ldots{}可问的这穿鞋的人\ldots{}\ldots{})

呃,我不问这穿鞋的人儿,还问我这穿靴子的人么?

(柳迎春 \ldots{}\ldots{}比你强得多啊!)

(是啊,)自然比我强啊!

我如今有了这个讨厌的东西了。

是啊,你若是靠着我,饿啊,(或:你要靠着我,这一十八载,饿啊,)也把你饿干了哇。

什么新鲜的事情?(或:哦,什么新鲜之事。)

哎呀呀,你真是无羞无耻呀!

你不死待我来碰。

(你啊,你就是与我------)

(唉!)

嗯,有的。

也是有的。

一十七岁的孩儿------不大,不小,是正穿呐。

哎呀,她倒端起来了。

哎呀呀,你拿过来吧。

妇道人家,拿刀、动杖,呃,成什么样儿啊?

薛礼呀薛礼,难为你还是个平辽王啊,做事就是这样粗鲁。

哎呀呀,这窑前窑后,也无人前来解劝呐\ldots{}\ldots{}这这这\ldots{}\ldots{}

这这这\ldots{}\ldots{}这这这\ldots{}\ldots{}这这这\ldots{}\ldots{}

唉,上前赔个笑脸可也就拉倒了。

妻呀!为丈夫的不是,喏喏喏,我这厢(与你)赔礼了。

妻呀!为丈夫的不是,我这厢又赔礼了。

唉,妻呀!为丈夫这厢(或:这里)跪下了哇。(或:俱是为丈夫的不是,我这厢跪下了。)

哎呀你这是怎么样了?(或:呃,你这做什么?)

哎呀,耍出汗来了。

(啊)妻呀,将你我的儿子唤将出来,教他看看(或:教他看一看)我这不成器的老子。

哦,他往哪里去了?

(惊介)

(啊)妻呀!我来问你,这窑前窑后,(可)还有别人家的孩儿会打雁?

贤妻你这里来啊!

你我的儿子出窑的时节,这头戴?

身穿?

左手?

右手?

唉呀!

【西皮导板】听一言来吓掉魂,

\textless{}\textbf{三叫头}\textgreater{}丁山!我儿!唉!儿啊\ldots{}\ldots{}(哭介)

(柳迎春 \ldots{}\ldots{}儿的老子。)

唉!

【西皮散板】凉水浇头怀抱冰。适才路过汾河境,见一个顽童打弹能。弹打南来宾鸿雁,枪挑鱼儿水浪分。

他不会来了\protect\hyperlink{fn357}{\textsuperscript{357}}!

【西皮摇板】本当与她实言论,又恐吓坏这受苦的人呐。

唉!

【西皮摇板】事到临头难瞒隐,咬定牙关说真情。

\textless{}\textbf{叫头}\textgreater{}唉呀妻呀!

适才为丈夫打从汾河湾前经过,观见你我的儿子在那里射雁。南山之上,下来猛虎,身旁带有袖箭,实望将虎打走,不想这一箭呐------将你我的儿子就射死了!

射死了!

唉呀,又是一条人命呐!

醒来!

汾河湾前。

随我来呀。(或:一同寻找。)

丁山\ldots{}\ldots{}

我儿\ldots{}\ldots{}

\newpage
\hypertarget{ux6c99ux6865ux996fux522b}{%
\subsection{沙桥饯别}\label{ux6c99ux6865ux996fux522b}}

{[}第一场{]}

(唐玄奘上)

玄奘 {[}引子{]}一年气象一年新,抛却红尘念佛经。

玄奘 (念)正在佛前打坐,回头观见五岳。一班俱是神像,为何欺善怕恶。

玄奘
贫僧玄奘。只因唐王天子为游地府许下大愿,要往西天拜佛,取经回朝,设立坛台,超度众魂。是我情愿替主一往。今乃黄道吉日,不免上朝,请主发下通关文凭,即日启程便了。

玄奘
【二黄慢板】有玄奘离娘怀身遭大难,蒙吾师搭救我来到金山。取法名唤玄奘苦读经卷,每日里在殿前把佛来参。因唐王游地府许下大愿,为的是斩神龙起下祸端。传旨意将众僧道法考选,我情愿替君王取经回还。

(玄奘下)

{[}第二场{]}

(徐勣、殷开山、程咬金、尉迟恭同上)

徐勣 (念)日出山高一片红,

殷开山 (念)唐王江山掌握中。

程咬金 (念)长安多少花似锦,

尉迟恭 (念)堪叹不觉白头翁。

徐勣、殷开山、程咬金、尉迟恭 老夫,

徐勣 徐茂公。

殷开山 殷开山。

程咬金 程咬金。

尉迟恭 尉迟敬德。

徐勣 列公请了。

殷开山、程咬金、尉迟恭 请了。

徐勣
只因吾主曾命金山法师,去往西天拜佛取经,今日上殿见驾领凭,同在朝房伺候。请。

(玄奘上)

玄奘 (念)离了金山寺,上殿见圣君。

玄奘 众位国公在上,贫僧稽首。

徐勣、殷开山、程咬金、尉迟恭 有礼相还。

殷开山 儿啊\ldots{}\ldots{}

徐勣、程咬金、尉迟恭 请问国公,他是何人?

殷开山 唉!乃是老朽外孙。

徐勣、程咬金、尉迟恭
原来如此。法师请在殿角伺候,圣驾临朝,我等启奏。请。

(玄奘下)

徐勣、殷开山、程咬金、尉迟恭 金钟三响,圣驾临朝,分班伺候。

(四小太监、二大太监引李世民上)

李世民 {[}引子{]}先王晏驾,龙归藏,孤掌朝堂。

徐勣、殷开山、程咬金、尉迟恭 臣等见驾,愿吾皇万岁。

李世民 平身。

徐勣、殷开山、程咬金、尉迟恭 万万岁。

李世民
(念)忆昔当年战洛阳,收得瓦岗众豪强。可叹恩公秦琼丧,寡人日夜不安康。

李世民
寡人大唐天子,贞观在位。因游地府,曾许大愿。考得僧人玄奘,道法甚高,愿替寡人西天拜佛取经。今乃黄道吉日,命他前往。徐皇兄。

徐勣 臣。

李世民 玄奘可曾宣到?

徐勣 今在殿角候旨。

李世民 宣他上殿。

徐勣 领旨。万岁有旨,玄奘上殿。

玄奘  (内)领旨。

(玄奘上)

玄奘 (念)金殿传旨宣,别驾往西天。

玄奘 玄奘见驾,愿------吾皇万岁。

李世民
法师\protect\hyperlink{fn358}{\textsuperscript{358}}替朕西天取经,封卿御弟三藏,如朕亲临。

玄奘 愿吾皇万岁。

李世民 平身。

玄奘 万万岁。

李世民 赐座。

玄奘 谢座。

(玄奘坐大边跨椅)

玄奘 启奏万岁:今乃黄道吉日,请驾发下通关文凭,即日启程。

李世民 内侍,文房四宝伺候。

李世民
【二黄慢板】王因为游地府许愿斋醮,超度那泾河龙重回天曹(或:轮回阴曹)。孤将这众高僧传旨选考(或:众僧人传旨选考),唯有那金山的(或:金山寺)玄奘法高。他情愿往西天见佛拜祷,他情愿取真经替朕代劳。孤想你往西行无穷(或:无数)路道,今日去何日归才得还朝?

玄奘
【二黄三眼】请吾主修文凭休迟即早(或:请我主写牒文休迟即早),仗吾皇洪福大何惧山遥。只要人秉诚心见佛拜祷,吾主爷何需要替僧心焦。

李世民
【二黄原板\protect\hyperlink{fn359}{\textsuperscript{359}}】提龙笔王亲书大唐国号,命御弟唐三藏奉旨出朝。各国的众王子【转二黄三眼】休挡禁道,到西天取经回替朕代劳。赐御弟锦袈裟霞光万道,孤赐你(或:赐御弟)紫金钵、禅杖一条。孤赐你(或:赐御弟)装经箱、毗卢僧帽,孤赐你四徒儿鞍前马后、涉水登山好把经挑(或:赐御弟四小童好把经挑)。内侍臣与孤王将宝抬到(或:替孤王将宝抬到),金銮殿王与你改换佛袍(或:改换法袍)。

(\textless{}合龙\textgreater{}玄奘改装,下)

李世民 【二黄摇板】王传旨即便把众卿宣召,随同孤送御弟饯行沙桥。

(李世民下)

{[}第三场{]}

(大太监上)

太监 (念)朝朝随驾走,时时伴龙行。除了当今主,咱家第一人。

太监
咱家,大唐天子驾前,掌朝内监是也。奉了万岁爷的旨意,在沙桥备酒,与三藏法师饯行。酒宴备齐,等候圣驾与众家国公前来。正是:

太监 (念)吾主许下诚心愿,就有高僧往西天。

(徐勣、殷开山、程咬金、尉迟恭上)

李世民 (内)【西皮导板】出午门到沙桥王下车辇,

(李世民搀玄奘同上)

太监 奴俾接驾。

李世民
【西皮原板】叫一声贤御弟细听王言:孤想你数万里路途崎险,孤愁你何日里得到西天。但愿你此一去早把佛见,但愿你路途上免带愁颜。但愿你见佛祖取经回转,百里外排銮驾接到殿前。

玄奘
【西皮慢板】万岁爷休得要将臣怜念,容为臣一一地细奏根源(或:细表根源):僧的父蒙恩赐七品正县,上任去遇刘贼劫了官船。将僧父用绳捆丢在水面(或:抛在水面),那贼子霸官亲就印为官。贤德母怀小僧十月孕满,想自尽又恐怕绝了后传。那一天生下僧时乖运\emph{蹇},刘洪贼他一见怒气冲天。霎时间(或:顷刻间)要将臣一刀两断,贤德母跪尘埃才得保全。用匣装写血书【转西皮二六】抛在水面(或:丢在水面),取名字江流儿性命由天。金山寺老禅师道法非浅,算定臣不该死(或:算就臣不该死)救至在山前。取法名唤玄奘苦把经念,看破了红尘路世事不贪:【转西皮快板】一不贪富与贵做官为宦,二不贪妻共子游玩清闲。三不贪吃珍馐五荤三厌,四不贪走花街观看红颜。五不贪住龙楼凤阁温暖,六不贪五花马銮驾旌幡。七不贪用奴仆随身使唤,八不贪出门庭拥后呼前。九不贪红颜女把酒来献,十不贪穿龙袍受王官衔。但愿得见佛祖取经回转,保唐室国泰民安万万年。

李世民
【西皮二六】内侍臣看过了皇封御宴,孤爱你道德好十事不贪。孤愿你此一去无灾无难,孤愿你足生云(或:孤愿你足登云)早到佛前。孤赐你饯行酒金杯玉盏,太平去吉日归\protect\hyperlink{fn360}{\textsuperscript{360}}孤谢上天。

玄奘
【西皮快板】谢吾皇饯行酒金杯玉盏,怎敢当吾的主龙恩海宽。转身来对苍天把酒来奠,祝告了天和地日月星官。

徐勣
【西皮摇板】领王命在沙桥把行来饯,尊一声大法师细听吾言:受老朽这一礼非为别干,替吾主取经回大大相烦。

玄奘
【西皮快板】小贫僧有何能怎敢领饯,公本是国王师八卦先天。三贤府盖过了臣救君难,保圣驾坐长安万万余年。

尉迟恭
【西皮摇板】老尉迟敬酒宴遮住了英雄脸,手捧着紫金杯躬身向前。但愿得此一去取经回转,某在那百里外接进朝班。

玄奘
【西皮快板】老国公家住在麻邑贵县,抢三关、夺八寨好不威严。征辽东挂帅印威风八面,访白袍保唐室万古名传。

程咬金
【西皮快板】程咬金平日里讲理不惯,尊法师休怪我粗鲁之言。烦公公你与我将酒斟满,

程咬金 【西皮摇板】见佛祖取真经早早回还。

玄奘
【西皮快板】程千岁可算得忠心赤胆,在瓦岗聚英雄人闻胆寒。弃暗地投明主官高爵显,但愿你寿延年快乐清闲。

殷开山 酒来。

殷开山
【西皮散板】论国法本应当国师称唤,论家法你本是老夫孙男。在长亭替你母把行来饯,我的孙何日里才得回还。

玄奘
【西皮散板】见外公不由我心中凄惨,烦外公拜儿母不孝之男。就说儿奉王旨不敢迟慢,多拜上贤德母少来问安。

玄奘 【西皮散板】在沙桥抬头看红日西转,请万岁、众国公驾回朝班。

李世民 【西皮散板】内侍臣带龙驹孤把缰挽,叫御弟跨金镫早奔阳关。

玄奘 哎呀!

玄奘 【西皮散板】君带马与臣骑世间稀罕,阻王驾休咒臣忙把王拦。

玄奘 将马带过!

玄奘 【西皮散板】辞王驾别国公忙把路趱,到西天拜佛祖取经回还。

(玄奘下,众太监、内侍、徐勣、殷开山、程咬金、尉迟恭引李世民同下)

(\textless{}\textbf{尾声}\textgreater{})

\newpage
\hypertarget{ux4e7eux5764ux5e26-ux4e4b-ux674eux4e16ux6c11}{%
\subsection{乾坤带 之
李世民}\label{ux4e7eux5764ux5e26-ux4e4b-ux674eux4e16ux6c11}}

(内)摆驾!

【二黄慢板】想当年老王爷带兵出征,下江南十余载才得回程。得了胜回朝来交旨复命,麒麟阁摆筵宴犒赏功臣。小杨广在席前言语不正,紫金杯打奸王惹下祸根。因此上修下了辞王表本,连夜里带家属转回故林。行至在临潼山被贼围困,多亏了秦恩公搭救满门。隋炀帝坐山河天心不顺,下扬州观琼花涂炭黎民。天降下五花棒奸王丧命,众公卿保父皇驾坐乾坤。遭不幸老王爷龙归海境,众老臣一个个辅孤王驾坐九重。恨只恨摩里沙兴兵犯境,命驸马秦怀玉前去剿平。但愿得此一去旗开得胜,但愿得此一去马到功成。内侍臣摆御驾九龙口进,又听得殿角下大放悲声。

(\textbf{或}:【二黄慢板】想当年老王爷带兵出征,下江南十余载得胜回程,得胜归回朝来交旨复命,麒麟阁摆筵宴犒赏功臣。小杨广在席前言语不正,紫金杯打奸王惹下祸根。因此上修下了辞王表本,连夜里带家属转回故林。行至在临潼山前被贼围困,多亏了秦恩公搭救满门。隋炀帝坐江山天心不顺,下扬州观琼花涂炭黎民。天降下五花棒奸王丧命,众公卿保父皇驾坐龙庭。遭不幸老王爷龙归海境,窦太后望儿楼凤驭上宾。众老臣一个个忠心耿耿,一个个辅孤王驾坐金龙。恨只恨摩里沙打来奏本,他要夺孤王的锦绣乾坤。为王的在金殿传下旨意,命驸马秦怀玉去把贼平。但愿得此一去旗开得胜,但愿得此一去马到功成。侍内臣摆御驾九龙口进,又听得后宫院大放悲声。)

梓童为何这等模样?

呜哙呀,有这等事?

梓童平身。

赐座。

内侍(或:来),宣银屏公主带子上殿。

(公主 万岁!)

皇儿,你可知罪?

这才是皇儿的道理。

平身。

殿前武士,将秦英绑上殿来。

唗!胆大秦英,前番将程雄打死,孤不降罪于你,也就是了。怎么,今日又将詹老太师打死(金水桥前),二罪归一。

殿前武士,将秦英推出午门斩首(或:斩了)。

(长孙皇后 吾皇万岁。)

御妻平身。

赐座。

(梓童上殿有和本奏?)

(长孙皇后 \ldots{}\ldots{}是哪位大臣?)

小将秦英。

(长孙皇后 \ldots{}\ldots{}所犯何罪?)

前番将程雄打死,不降罪于他。今日又将詹老太师打死金水桥前,故而推出斩首。

(这个\ldots{}\ldots{})

寡人龙心已定了,御妻不必多奏。

你又来多事了。

【西皮慢板】劝御妻休得要把本奏上,孤怎比开河运无道隋炀。孤岂肯听信那谗言毁谤,孤岂肯斩忠良绝了那秦门后香。慢说是打死了詹老丞相,就是那庶民人也要抵偿。

是啊,你母女在金殿奏得本,爱梓童,哎,连一句话都讲不得吗?

梓童你有本?当殿奏来,寡人与你作主。

梓童平身。

梓童赐座。

哼,这还了得!

呃,梓童奏来。

【西皮导板】这桩事教孤王难以发放,

(长孙皇后 儿啊\ldots{}\ldots{}(哭介))

【西皮原板】娘哭儿、女哭父好不惨伤。孤传旨斩了那秦英小将,

唉!

【西皮原板】孤皇儿在一旁两泪汪洋。孤传旨赦了那秦英小将,

\textless{}\textbf{哭头}\textgreater{}教孤好为难呐,(老皇妻呐,)

【西皮原板】爱梓童殿角下哭断肝肠。唐贞观在龙书案前思后想,

【西皮原板】爱梓童近前来【转西皮二六】细听端详:你的父并不曾欺君罔上,可怜他金水桥一命身亡。孤劝你把此事休挂心上,哪有个人死后又能还阳。孤传旨挑选那能工巧匠,孤传旨修一座忠义祠堂。孤传旨赐你父金井玉葬,孤传旨文武臣送至在山岗,王去拈香,孤的爱梓童,你那里且免愁肠。

(詹妃 【西皮摇板】\ldots{}\ldots{}母女二人。)

呃------

【西皮摇板】唐贞观亦非是懦弱(的)皇上,为的是安黎民整顿朝纲。哪一个大胆人敢来违抗?

【西皮二六】叫皇儿近前来父女商量。小秦英打死了皇亲国丈,论国法就应该叫他抵偿。念秦门昔年间东杀西挡,念秦门只有这一脉后香。金銮殿父赐儿玉液琼浆,殿角下去哀求詹妃娘娘。

【西皮摇板】好一个爱梓童宽宏大量,不由孤心内喜(或:不由孤龙心喜)暗称贤良。

【西皮摇板】为王的在金殿把旨来降,午门外快赦回秦门儿郎。

非是寡人不斩于你,詹娘娘讲情,将你饶恕。一旁谢过詹娘娘(或:上前谢过詹娘娘)。

(秦英 谢过姨姥哦!)

御妻、梓童、皇儿回避。

内侍(或:内臣),宣徐勣上殿。

平身。

赐座。

卿家上殿,有何本奏?

呈上来。

待孤(或:寡人)看来。

哎呀!原来驸马被困,卿家计将安在(或:有何良策)?

小将秦英,打死皇亲国戚(或:皇亲国丈)。

已然赦却(或:赦回)。

依卿所奏。

内侍(或:内臣),宣秦英上殿。

秦英,今有你父被困摩里沙。命你带领人马前去征剿,得胜还朝(或:得胜回朝),将功折罪。外赐乾坤宝带,以振军威。

(秦英 谢万岁!)

见过儿徐祖父。

退班。

\textbf{芦花河 之
薛丁山}\protect\hyperlink{fn361}{\textsuperscript{361}}

(内)马来!

(薛丁山上)

【西皮二六】奉主旨意往西征,数年铠甲未离身。先父当年挂帅印,在白虎关前命归天庭。多亏了智勇樊夫人,她也能提调众三军。来至在辕门下金镫,

(薛丁山下马)

啊?!

【西皮快板】辕门外绑定薛应龙。我儿犯了何条令,缘何捆绑问典刑(或:问斩刑)?

(醒来。)

哦!

【西皮摇板】我道是犯了那皇王的军令,却原来为的是这临阵招亲------

儿啊!

【西皮快板】我的儿只管心放稳,为父进帐讲人情。进帐只用三两语,管教你母饶儿身。本帅撩袍宝帐进------

哦!

【西皮快板】王法条条不徇情。我若讲情她不允(或:我若讲情她不准),把娇儿反送在这枉死城。

【西皮快板】进帐去先行周公礼,必然念在夫妻的情。

【西皮摇板】秦、窦二将往上禀(或:秦、窦二将一声禀),你就说二路的元帅转回大营。

(薛丁山下)

(秦汉、窦一虎 二路元帅到!)

(樊梨花 【西皮导板】\ldots{}\ldots{})

(薛丁山上,进帐,樊梨花出帐,二人撞肩膀)

(樊梨花 王爷!)

夫人!

(樊梨花 王爷请!)

夫人请!

夫人请坐!

(二人换位,薛丁山大边、樊梨花小边,坐)

(樊梨花 【西皮原板】迎接元帅进大营,\ldots{}\ldots{}打听得哪路发来兵?)

【西皮原板】一来是夫人威名盛,各国闻名不敢动兵。

(樊梨花 【西皮原板】\ldots{}\ldots{}闲事情。)

呀!(或:哦!)

【西皮快板】樊夫人她倒有(或:樊夫人她倒能)隔山照镜,就知本帅讲人情。未曾开言把罪请呐。

(樊梨花 【西皮原板】问王爷施礼为何情?)

【西皮摇板】应龙儿犯了(或:身犯)何条令,缘何捆绑问典刑(或:缘何捆绑在辕门)?

(樊梨花 王爷问的是他?)

正是!

(樊梨花 王爷呀!)

(樊梨花 【西皮二六】\ldots{}\ldots{}问斩刑?)

(哦。)

【西皮摇板】我道是犯了那皇王(的)军令,却原来为的是这临阵招亲。提起来招亲的事,话也难尽,难道说贤夫人你心不明。想当年大战在那寒------

(樊梨花 噤声。)

掩门。

【西皮二六】寒江岭,寒江关前动刀兵。我与夫人来会阵,夫人与我来提亲。

(薛丁山拉樊梨花,樊梨花羞介,二人换位)

【接西皮二六】本帅再三不应允,夫人又把巧计生。使下了(或:设下了)移山倒海阵,

(二人换位)

【接西皮二六】将本帅吊在那(或:吊至在)半空存。那时我唤天,天不应;我待入地,地又无门。万般无奈才应允,夫妻双双进唐营。若论这临阵招亲,是你我先来做定,常言道前人开路,这后人行呐。

(樊梨花 【西皮快板】王爷说话\ldots{}\ldots{},军无私来就法无情。)

【西皮摇板】应龙犯罪理当斩,

(樊梨花 谢王爷!)

且慢!

【西皮摇板】还要看他的年纪轻。

(樊梨花 【西皮快板】 \ldots{}\ldots{}不是娘生?)

【西皮快板】本帅与你讲人情,哪个和你比古人。大夫人生下麒麟子,二夫人也有后代根。唯独夫人无有后,收下应龙作螟蛉。到如今夫人有了梦熊信\protect\hyperlink{fn362}{\textsuperscript{362}},便把应龙当外人。倘若是娇儿有伤损,旁人道你两样心。

【西皮快板】你若是赦了应龙子,唐王降罪我担承。

(【西皮快板】你今赦了应龙子,满营将官我也担承。)

【西皮快板】不能不能万不能呐。

(樊梨花 【西皮摇板】 你把你\ldots{}\ldots{}看大了。)

【西皮摇板】威宁侯啊,也不放在本帅心。

哎呀!

【西皮快板】一见宝剑挂营门,吓得三魂少二魂。眼望娇儿无救\textless{}\textbf{哭头}\textgreater{}应,我的儿啊,

【西皮摇板】父子们做鬼一路行。

\textless{}\textbf{哭头}\textgreater{}薛应龙,我的儿啊(或:小娇儿啊),啊\ldots{}\ldots{}

夫人,

\textless{}\textbf{哭头}\textgreater{}我的儿啊!

夫人你看,众将皆服了(或:满营将官皆服了)。

(樊梨花 【西皮摇板】\ldots{}\ldots{}平身。)

解下桩来。

(樊梨花欲踹薛应龙介)

夫人方才赦过了。

出帐去罢。\protect\hyperlink{fn363}{\textsuperscript{363}}

(探子 讨战。)

再探!

夫人,贼人摆下阵势,你我夫妻敌楼一观------(或:军士们,带马城头去者。)

(樊梨花 带马。)

【西皮散板】适才探马报一声,芦花河贼子发来兵。

【西皮散板】下得马来敌楼进,观看贼阵是何名(或:观看贼子发来兵)。

夫人,贼人摆的是何阵势?

(樊梨花 此乃是金------)

噤声!

【西皮散板】叫声夫人莫高声,

(薛丁山、樊梨花下城)

【西皮散板】休要惊动那贼兵(或:休要惊动这贼兵)。

【西皮散板】下得马来宝帐进(或:下得马来大营进),

【西皮散板】再与夫人把话云(或:夫妻对坐论军情)。

夫人方才讲的金什么? (或:适才摆的什么阵势?)

(樊梨花 乃是金光大阵。)

可有破法?

如此(待)本帅二次回转仙山,哀求师父,求来法宝,再破此阵。

即刻启程,我有一言,夫人听了:

(薛丁山拉樊梨花到台口)

【西皮快板】手挽手,站营门,尊声梨花樊夫人。芦花河摆下金光阵,莫教应龙去出征。倘若娇儿有伤损,那时失了夫妻情。辞别夫人上马行,

【西皮摇板】我嘱咐你言语呀,你(要)谨记在心呐。

\newpage
\hypertarget{ux54edux5c38}{%
\subsection{哭尸}\label{ux54edux5c38}}

樊梨花 (念)眼跳心惊,未知吉凶。

秦汉、窦一虎 元帅,大事不好了。

樊梨花 什么大事?

窦一虎 小本官芦花河阵中身亡。

樊梨花 不好了!

樊梨花 尸首可曾带回?

秦汉、窦一虎 带回来了。

樊梨花 搭了上来!

樊梨花
【西皮快板】一见娇儿丧了命,不由为娘痛在心。千言万语说不\textless{}\textbf{哭头}\textgreater{}尽,喂呀我的儿啊,

樊梨花 【西皮快板】哪个教你去出兵?

薛丁山 (内)马来!

薛丁山 【西皮摇板】深山奉了师尊命,回到大营说分明。

樊梨花 \textless{}\textbf{叫头}\textgreater{}元帅!

樊梨花 大事不好了!

薛丁山 何事惊慌?

樊梨花 应龙丧命。

薛丁山 在哪里?

樊梨花 在这里。

薛丁山 哎呀!

薛丁山 【西皮摇板】一见娇儿丧了命,

樊梨花 【西皮摇板】怎不教我痛在心。

薛丁山 【西皮摇板】你在那玉泉山何等安静呐,

樊梨花 【西皮摇板】哪个教你去出兵?

樊梨花 【西皮摇板】千言呐------

薛丁山 【西皮摇板】万语啊------

薛丁山、樊梨花 【西皮摇板】说不尽\ldots{}\ldots{}

薛丁山
【西皮快板】把话说与夫人听:芦花河摆下金光阵。休教应龙去出兵,今日我儿丧了命,看你心疼不心疼?

樊梨花 \textless{}\textbf{哭头}\textgreater{}薛应龙,

薛丁山 \textless{}\textbf{哭头}\textgreater{}小娇儿啊\ldots{}\ldots{}

樊梨花
\textless{}\textbf{哭头}\textgreater{}喂呀,我的儿啊\ldots{}\ldots{}

樊梨花
【西皮快板】元帅息怒容我禀,为妻言来你试听\protect\hyperlink{fn364}{\textsuperscript{364}}:我在营中传将令,不知他私自去出兵。如今娇儿丧了命,难道我不痛在心?

薛丁山 \textless{}\textbf{哭头}\textgreater{}薛应龙,

樊梨花 \textless{}\textbf{哭头}\textgreater{}小娇儿啊,

薛丁山 \textless{}\textbf{哭头}\textgreater{}啊, 我的儿啊!

薛丁山
【西皮快板】大夫人生下麒麟子,二夫人也有后代根。唯独夫人无有后,你把应龙当亲生。今日夫人心不稳,有意绝我后代根。

樊梨花 \textless{}\textbf{哭头}\textgreater{}薛应龙,

薛丁山 \textless{}\textbf{哭头}\textgreater{}小娇儿啊,啊------

樊梨花 \textless{}\textbf{哭头}\textgreater{}我的儿啊!

樊梨花
【西皮快板】元帅有所不知情,为妻言来听分明:先禁自己后禁人,怕的三军心不平。

薛丁山 \textless{}\textbf{哭头}\textgreater{}薛应龙,

樊梨花 \textless{}\textbf{哭头}\textgreater{}小娇儿啊,

薛丁山 \textless{}\textbf{哭头}\textgreater{}啊------我的儿啊!

薛丁山 呀呸!

薛丁山
【西皮快板】自古道青竹蛇儿口,自古道黄蜂尾上针。自古道万般皆由命,自古道最毒妇人心。

樊梨花 \textless{}\textbf{哭头}\textgreater{}薛应龙,

薛丁山 \textless{}\textbf{哭头}\textgreater{}小娇儿啊,啊------

樊梨花 \textless{}\textbf{哭头}\textgreater{}喂呀, 我的儿啊!

樊梨花
【西皮快板】元帅道我两样心,对着苍天把誓盟:梨花待子有假意,死在千军万马营。

薛丁山 言重了!

薛丁山
【西皮摇板】听一言来才知情,本帅错怪樊夫人。叫人来抬尸首后营进呐,父子们相逢万不能。

樊梨花 \textless{}\textbf{叫头}\textgreater{}薛应龙,

薛丁山 \textless{}\textbf{叫头}\textgreater{}小娇儿!

薛丁山、樊梨花 唉!儿啊\ldots{}\ldots{}(哭介)

\textbf{法场换子}\protect\hyperlink{fn365}{\textsuperscript{365}}

{[}第一场{]}

夫人 (念)夫受皇家爵,妻沾雨露恩。

徐策 (内)开道!

(\textless{}\textbf{小锣六幺令}前段\textgreater{},徐策下轿,\textless{}\textbf{小锣原场}\textgreater{},进门)\protect\hyperlink{fn366}{\textsuperscript{366}}

徐策 唉!

夫人 相爷今日下得朝来,为何这等长叹?

徐策
夫人有所不知,(或:夫人呐------听道:)今日早朝,可恨张泰奸贼将薛猛夫
妻调进京来,要害他二人一死,(唉,)倒也罢了哇。最可叹未满三月小薛蛟,
也要受皇家一刀之苦。怎不令人长叹呐!

夫人 就该寻一计策,搭救忠良才是。

徐策 (下官正为此事回来,与夫人商议。)计策倒有哇,只是要应在夫人的身上。

夫人 难道说教妾身替他不成?

徐策
不是哟。我看金斗孩儿,面带七煞\protect\hyperlink{fn367}{\textsuperscript{367}},终难抚养。意欲带到法场,将薛蛟调换
下来,以接薛门宗嗣。不知夫人你的意下如何?

夫人
相爷说哪里话来,想你我夫妻,年将半百,只有此子,若是替人------万万不能。
\protect\hyperlink{fn368}{\textsuperscript{368}}

徐策 唉!夫人呐,呃,唉\ldots{}\ldots{}(哭介)

夫人 万万不得能够!

徐策 唉!夫人呐,呃\ldots{}\ldots{}(哭介)

徐策
【二黄快三眼】恨薛刚小奴才不如禽兽,吃醉了酒全不顾满面惭羞。闯下了滔天祸一人逃走,连累他二爹娘不能到头。把一个两辽王午门斩首,樊夫人拔宝剑自刎人头。眼见得忠良臣乏嗣无后,可怜他斩草除根、寸草不留、天地含忧,怎教我看水流舟,夫人呐!

夫人
【二黄原板】老相爷说此话情理不周,听妾身把此事再说从头:张泰贼与薛家结成仇扣,满朝中文武臣不敢出头。怕的是画虎不成反类狗,那时节船到江心倒做了逆水行舟。

徐策
【二黄散板】贤夫人舍不得娇儿金斗,眼见得小薛蛟一命罢休。为忠良我只得屈膝叩首哇,

夫人 【二黄散板】老相爷跪埃尘情理不周。

夫人 相爷不必如此,妾身应允就是。

徐策 多谢夫人。

徐策 家院(过来)。

家院 有。

徐策 将你家少公爷放在食盒之内,抬到法场。再拿我名帖,去见张泰,就说老夫
要亲自祭奠。

(家院 是。)

(家院招丫鬟抱小孩(喜神)同下)

徐策 附耳上来(,记下了)。

家院 遵命。

夫人 啊相爷,妾身也要跟随前去。

徐策 法场之上,耳目甚众,去之无益。

夫人 妾身要去。

徐策 夫人要去?到了法场,看下官眼色行事。

夫人 遵命。

徐策 (如此)夫人请。

夫人 相爷请。

徐策 正是:(念)可叹薛家世代贤,

夫人 (念)忠良无故把刀餐(或:忠良无辜被刀餐)。

徐策 (念)苍天有灵睁开眼,

夫人 (念)仇报仇来冤报冤。

徐策 着哇!好一个``仇报仇来冤报冤''。

徐策 夫人,

夫人 相爷。

徐策 随我来。

{[}第二场{]}

(\textless{}\textbf{大锣六幺令}后段\textgreater{},张泰上)

张泰
(念)树大遮天盖地,根深哪怕风狂。(任他皇亲国戚,一本斩草除根。)\protect\hyperlink{fn369}{\textsuperscript{369}}

张泰 老夫张泰,奉圣命监斩薛猛夫妻。刀斧手,将薛猛夫妻押了上来。

马氏
哎吓老爷呀,你我夫妻一死,不值要紧,可叹三月孩儿,也要受皇家一刀之苦
哇\ldots{}\ldots{}(哭介)

马氏
【二黄散板】叫你反来你不反,叫你行来你不行(或:叫你行来逃生你不行)。
你我一死不要紧,可怜那娇儿也受酷刑。\protect\hyperlink{fn370}{\textsuperscript{370}}

薛猛 夫人呐!

薛猛
【二黄散板】薛家世代忠良后,\ldots{}\ldots{}怎做那叛逆臣。回头再把张泰论:苦害薛
家为何情?恨不得一足将尔踏,阴曹地府勾尔魂。\protect\hyperlink{fn371}{\textsuperscript{371}}

张泰
校尉等,将他夫妻绑上法标。有人讨祭,报我知道。\protect\hyperlink{fn372}{\textsuperscript{372}}

家院
(念)奉了相爷命,法场走一程。\protect\hyperlink{fn373}{\textsuperscript{373}}

家院 法场之上哪位听事。\protect\hyperlink{fn374}{\textsuperscript{374}}

校尉 做什么的?

家院 徐老相爷有名帖奉上,前来法场祭奠。

校尉 候着。

校尉 启相爷,徐相爷有帖拜上。

张泰 呈上来。

张泰 呜哙呀,这老儿又来多事。

(校尉 他的夫人也来了。)\protect\hyperlink{fn375}{\textsuperscript{375}}

张泰
(哦,夫人爷来了。)\protect\hyperlink{fn376}{\textsuperscript{376}}命他一祭(或:容他一祭),时辰一到,速报我知。

校尉 容你们一祭。

家院 祭礼走上。

(丫鬟随家院、四青袍上,中间两个青袍抬盒,家院站台中间,四青袍脸朝里,开盒,丫鬟取出小孩与马氏怀中小孩交换。家院令众人下,青袍领下,丫鬟随青袍后,右手抱小孩、用左袖盖小孩随下,家院留场上)

家院 有请相爷夫人。

徐策 (内)夫人,随我来!

(家院下)

徐策 【二黄散板】夫妻双双到法场呃,

夫人 【二黄散板】不见忠良在哪厢。

徐策 \textless{}\textbf{叫头}\textgreater{}薛猛!

夫人 \textless{}\textbf{叫头}\textgreater{}马氏。

徐策 唉!儿啊\ldots{}\ldots{}(哭介)

徐策 【二黄散板】他夫妻好比一张弓,

夫人 【二黄散板】万马营中抖威风。

徐策
【二黄散板】未把箭放弦又\textless{}\textbf{哭头}\textgreater{}断,我的儿啊,

夫人 【二黄散板】一到法场一场空。

徐策 夫人,天色不早,先回府去吧。

(夫人过大边,徐策过小边,夫人在大边面向小边)

夫人 待我辞别辞别。

夫人 \textless{}\textbf{叫头}\textgreater{}薛猛!

夫人
\textless{}\textbf{叫头}\textgreater{}马氏------我那金\ldots{}\ldots{}

(徐策面向大边,阻拦介)

徐策 噤声!

夫人 今生今世难得见的\ldots{}\ldots{}亲儿啊\ldots{}\ldots{}(哭介)

(夫人下场门下)

徐策
正是:(念)法场之上冷嗖嗖,绳拿索绑不自由。盖世忠良遭毒手,(或:法鼓嗵
嗵打,西山月影斜。黄泉无客店,)

徐策 \textless{}\textbf{叫头}\textgreater{}薛猛!

徐策 \textless{}\textbf{叫头}\textgreater{}马氏!

徐策
(念)花开花落(或:花开花谢)籽未丢哇。(或:今晚宿谁家。)啊,呃\ldots{}\ldots{}(哭介)

徐策
【反二黄慢板】见夫人哭出了席棚以外,可怜她抛撇下十月怀胎。催命鼓响嗵嗵魂飞天界,勾命锣仓啷响魄散泉台。这壁厢绑的是薛猛元帅,那壁厢绑的是马氏裙钗。马夫人使双刀名扬四海,女将中可算得出色英才。你夫妻原本是镇守边塞,为什么一心心闯进京来。儿好无才,我的儿啊!

徐策
【反二黄三眼】千不该万不该是儿不该,大不该命薛刚私出府来。那奴才寿堂上把寿来拜,二爹娘一见娇儿,溺爱不明,把酒戒来开。三杯酒下咽喉劣性还在,酒壮胆、胆包天闯下祸来。驸马爷张登荣被他踢呃坏,太子爷紫金冠也打落尘埃。保驾的官、文武臣一齐打坏,最不该持香炉去打张泰。张泰贼奏一本将你来害,将儿的一家人捆绑御街。你夫妻尽了忠留名后代(或:你夫妻尽了忠留名四海;你夫妻双双死命里所在),【垛板】最可叹,断送了未满三月小婴孩,捆绑到御街,刀下赴泉台,儿好无才!冤哉冤哉,令人悲哀,好不伤怀,我的儿啊!

徐策
【反二黄原板】我也曾送儿的信,儿怎生不解,书信中藏密语儿解之不开。我教儿领人马反出了边塞,儿为何一心心闯进网来(或:闯入网来)。老徐策见此情无计可奈,舍亲生将薛蛟调换下来。待老夫替你家抚养几载,将养\protect\hyperlink{fn377}{\textsuperscript{377}}起忠良后祭扫泉台。可怜我年半百绝了后代,绝了后代,

徐策 【反二黄散板】恨不得将张泰斧斫刀开。

徐策
【反二黄散板】这一旁搀扶起薛猛元帅,马夫人我不便搀你、你\ldots{}\ldots{}你自呃己起来。到九泉见先人呐把我话带,你把我舍子的情细说开怀。

徐策
【反二黄散板】悲切切哭出了法场以\textless{}\textbf{哭头}\textgreater{}外啊,

徐策 【反二黄散板】等候了大炮响啊,收儿的尸骸。

徐策 \textless{}\textbf{叫头}\textgreater{}薛猛!

徐策 \textless{}\textbf{叫头}\textgreater{}马氏!

徐策 我那金\ldots{}\ldots{}(惊介)

徐策 今生今世难得见的\ldots{}\ldots{}唉,亲儿啊\ldots{}\ldots{}(哭介)

徐策 罢!

(徐策一跺脚,下)

张泰 校尉等,时辰可到?\protect\hyperlink{fn378}{\textsuperscript{378}}

校尉 时辰已到呃。

张泰 拿去开刀!

薛猛、马氏 好贼------

校尉 斩首已毕,现有一婴孩。

张泰 呈上来。

张泰
呜哙呀,这一婴孩,生得是眉清目秀,不免带回府去,收为义子。唉------呃,``斩草不除根,萌芽依旧生;斩草除了根,萌芽永不生''。

张泰 校尉等,将这婴孩,腰铡三截。

校尉 斩首已毕。

张泰 打道,上殿交旨。

\textbf{按}:此戏中徐策法场哭祭的大段``反二黄''唱腔是余叔岩根据李吉甫的《法场换子》的本子(用余自己的《焚绵山》的本子换取)设计的唱腔。刘曾复先生为樊百乐君另外还示范了传统的谭派《法场换子》的``反二黄''唱法(可参考程君谋、蒋锡康的《法场换子》唱片录音),兹照录如下:

【反二黄慢板】见夫人哭出了席棚以外,可怜她年半百十月怀胎。催命鼓响嗵嗵魂飞天界,救生锣仓啷响魂又转来。

【反二黄中三眼】站席棚先埋怨薛猛元帅,大不该命薛刚私出府来。进什么京来把什么寿拜,二爹娘爱子心又把宴排。三杯酒下咽喉劣性还在,酒壮胆胆包天闯下祸来。御花园众神像打成土块,太子爷紫金冠也打落尘埃。探花郎张登荣也被打坏,最不该上金殿去打张泰。张泰贼奏一本将你来害,将你来害,我的儿啊!

【反二黄原板】因此上薛门中降下祸灾。这一边哭坏了薛猛元帅,转面来再埋怨马氏裙钗。在阳河你就该反出了边塞,为什么将娇儿带进京来。你夫妻双双死情理所在,【垛板】最可叹,小薛蛟,未满三月也被刀开,我的儿啊!

【反二黄原板】只为你薛门中绝了后代,舍金斗将薛蛟调换下来。待老夫替你家抚养几载,将养起忠良后祭扫泉台。可怜我年半百绝了后代,绝了后代,

(薛猛、马氏跪,徐策不看二人)

【反二黄散板】恨不得把张泰斧斫刀开。

哎呀!

【反二黄散板】这一旁搀扶起薛猛元帅,马夫人我不便搀你、你\ldots{}\ldots{}你自己起来。

【反二黄散板】悲切切哭出了法场以\textless{}\textbf{哭头}\textgreater{}外啊,

【反二黄散板】等候了大炮响收儿的尸骸。

\newpage
\hypertarget{ux53ccux72eeux56fe-ux4e4b-ux5f90ux7b56}{%
\subsection{\texorpdfstring{双狮图\protect\hyperlink{fn379}{\textsuperscript{379}}
之
徐策}{双狮图379 之 徐策}}\label{ux53ccux72eeux56fe-ux4e4b-ux5f90ux7b56}}

{[}第一场{]}

(内)开道!

【二黄原板】朝罢圣天子转回府门,见狮子并一处所为何情?

家院,今日(府门)何人值日?

唤书僮。

罢了。

今日可是你的值日?

我来问你:府门外玉石狮子缘何并在一处?

哦,是你并在一处的么?

好。当着老夫的面前,还要与我分开(或:再与我分开)。

哦?狮子会讲话?(它)讲些什么?

来,掌嘴!

哦?是你家少公爷并在一处的么?

(好,)快快唤他前来。

不像话。

(我儿)罢了,一旁坐下。

儿啊,为父今早下朝回来,观见府门外玉石狮子,缘何并在一处?

呃,呃,你这做什么?

就该打手。

儿啊(你)慢慢讲来。

呃,你这又做什么?

就该掌嘴。

(还不)下去。

儿啊,你只管地讲来。

怎么?是我儿并在一处的么?

为父的不信呐。

哦,你还能分开?

(好,你)看仔细。

啊,呵呵哈哈哈\ldots{}\ldots{}(笑介)

【二黄散板】他父是英雄儿好汉,强将手下无弱兵。张泰贼这就是尔对头到,薜家出了报仇人。

儿啊,自从你(或:自从儿)长大成人,还未曾祭过祖先(或:拜过祖先)。今日随为父祖先堂上一祭。

家院,祖先堂打扫。附耳上来。

儿啊,随为父的来呀!

呵呵哈哈哈哈\ldots{}\ldots{}(笑介)

{[}第二场{]}

(\textless{}\textbf{柳摇金}\textgreater{}上)

儿啊,随为父的来呀!

儿(啊,)要多拜几拜。

一旁坐下。

我儿有所不知。此乃我朝一家忠良,被奸臣陷害。全家问斩,后辈无人。为父的与他家世代交好,故而将他的真容悬挂在祖先堂上一祭。

这头一排么,此人姓薛名礼,字仁贵,乃山西绛州龙门县的人氏啊。此人英雄盖世,武艺超群。保定唐王,跨海征东。立下十大汗马功劳,唐王见喜,封为平辽王之位。

平辽王之位。

第二排,此乃仁贵之子,名唤丁山。那旁樊梨花樊氏夫人。夫妻二人,征西有功,唐王见喜,(到后来)封为两辽王之职(或:两辽王之位)。

正是。

此乃丁山长子,名唤薛猛,那旁双刀马氏夫人。夫妻二人镇守阳河,被奸臣陷害,调进京来,双双而死。呃\ldots{}\ldots{}(哭介)

那黑汉在哪里?那黑汉在\ldots{}\ldots{}

呀呀呸!好你大胆黑汉,闯下塌天之祸(或:满门被害,皆因你一人所起),还敢在此发笑,还不与我走、走、走\ldots{}\ldots{}

真真气、气\ldots{}\ldots{}气煞------我也\ldots{}\ldots{}(哭介)

非是为父的动怒啊,满门被害,皆因他一人所起,怎不令人发怒啊?(或:我儿有所不知,此乃丁山三子,满门被害,皆因他一人所起,故而为父的动怒啊!)

那小孩子在哪里?

(那)小孩子\ldots{}\ldots{}

\textless{}\textbf{三叫头}\textgreater{}金斗!儿啊!唉!儿啊\ldots{}\ldots{}(哭介)

呵呵哈哈哈\ldots{}\ldots{}(笑介)

非是为父悲中带喜,你看那小孩子虽然是腰铡三截\protect\hyperlink{fn380}{\textsuperscript{380}},他还不曾死啊!

何谓呆话?

我儿哪里知道,只因我朝有一家忠良,与他薛门世代交好,不忍他断绝香烟。将自己亲生的儿子,在法场之上调换下来,故而他还不曾死啊。

还在。

他今年多大年纪了?

儿啊,你站了起来。

(一旁)坐下。

与我儿般长般大。

论他的本领么? (或:问他的本领么?)

他,他,他\ldots{}\ldots{}他能力举千斤。

儿为何发笑?

唉,只因他单丝不线,孤树不林。故而也就耽误下了。

怎么(或:哦)?(我)儿要替他代报冤仇么?

呀呀呸!儿自己有血海冤仇,尚未报得,还要替人家报的什么冤仇?

儿啊!

(儿)说什么前堂有父,后堂有母\ldots{}\ldots{}

慢说没有冤仇,纵有冤仇,儿是即刻就报。可惜我不是儿的亲\ldots{}\ldots{}

儿啊,为父的今早起来,吃了几杯早酒,说话么有些个颠三倒四。儿不必细问,快快(快)攻书去罢。

【二黄导板】未开言不由人珠泪滚滚,

\textless{}\textbf{三叫头}\textgreater{}徐忠(或:薛蛟)!我儿!唉,儿啊\ldots{}\ldots{}(哭介)

【回龙】待为父细说那以往原因。我的儿啊!

【二黄原板】头一排(或:第一排),儿曾祖哇薛仁贵,跨海征东立下功勋。

【二黄原板】第二排,儿祖父丁山元帅,那一旁樊梨花樊氏夫人。

【二黄原板】双尸无头是儿的亲生父母,亲生父\textless{}\textbf{哭头}\textgreater{}母,我的儿啊!

【二黄原板】他夫妻双双问典刑。

【二黄原板】那黑汉是儿的三叔父,

【二黄垛板】都只为,进都城、逛花灯、吃醉酒、打伤人,连累了一家满门,绑赴法场,俱丧残生,是一个起祸根。

【二黄原板】腰铡三截是我的亲生子,掉换呐\textless{}\textbf{哭头}\textgreater{}你,喂呀我的儿啊!

【二黄散板】张泰贼是儿的对头人。

我儿哪里去?

我儿一人焉能报得?

无妨,儿有一三叔父,现在韩山招兵聚将。待为父修书一封,儿去往那里搬兵报仇就是。

(念)我儿换衣巾。

家院,溶墨。

【二黄碰板】说明了十数载冤仇恨,血海的冤仇要报清。老徐策在祖先堂上修书信,打发娇儿早早登程。

【二黄原板】未曾提笔泪难忍,骂一声小薛刚不肖的畜生。当初进京把寿进,吃醉酒、打伤人连累满门。老夫见情心不忍,法场之上舍亲生。到如今此子长成(或:到如今薛蛟长成)有本领,他两膀之上力千斤。见书即刻发人马,老夫内应共灭仇人。一封书信忙写定,

【二黄摇板】我儿此去要小心(或:打发娇儿早动身)。

(\textless{}\textbf{哭头}\textgreater{}啊,我的儿啊!)

转来。

我儿此番搬兵,不定是三年五载,才得回来。儿来看------

为父的年迈呀。倘若我二老下世,儿必须买上几陌纸钱,去至坟前烧化。也不枉为父的抚养儿一十几载,养育之恩呐!呃\ldots{}\ldots{}(哭介)

为何?

呃,话虽如此,为父的虽则年迈,身体倒还康健。儿只管地前去。

你当真不去(或:儿当真不去)?

果然不去?

不去为父就要打!

哦?打死儿也是不去的么?

呀呀呸!徐策呀徐策,你好没来由!倘若留得自己亲生儿子在世,焉能如此这般倔强(或:这等倔强)。唉!喂呀,我那亲生的儿啊\ldots{}\ldots{}(哭介)

好!上马去罢!

\textless{}\textbf{三叫头}\textgreater{}薛蛟!我儿!唉,儿啊\ldots{}\ldots{}(哭介)

\textless{}\textbf{叫头}\textgreater{}薛蛟!我儿!

\textless{}\textbf{哭头}\textgreater{}啊,我的儿啊!

【二黄散板】见娇儿上了马能行,好似开弓箭一根呐。悲悲切切后堂进(或:二堂进),见了夫人说分明。

唉!儿啊\ldots{}\ldots{}(哭介)

\textbf{打金枝 之 唐王}\protect\hyperlink{fn381}{\textsuperscript{381}}

{[}第一场{]}

(郭暧
【西皮摇板】吾主爷有道君长安驾坐,全凭着驾下臣保定山河。安禄山反河东文武胆破,我父子扫狼烟才定干戈。蒙圣恩将金枝招赘于我,父王位、子东床\protect\hyperlink{fn382}{\textsuperscript{382}}扶保朝阁。今日里八旬寿群臣齐贺,奉王命回府去呀敬致三多\protect\hyperlink{fn383}{\textsuperscript{383}}。)

{[}第二场{]}

(郭暧
【西皮摇板】唐君瑞\protect\hyperlink{fn384}{\textsuperscript{384}}失却了周公之礼,有天地有父母才有夫妻。似这等不贤妇要她何益,倒不如在府中独宿孤栖。)

{[}第三场{]}

摆驾!

【西皮慢板】金乌东升玉兔坠,景阳钟三下响王出宫闱\protect\hyperlink{fn385}{\textsuperscript{385}}。唐室连年遭颠沛,国乱只为杨贵妃。安禄山在河东【转西皮二六】曾起反意,兵破潼关夺社稷。陈元礼兵变在马嵬驿,可怜那贵妃丧沟渠。先皇驾幸西蜀地,多亏皇兄郭子仪。血战三载狼烟息,擒住了贼子剑下劈。到如今乐享这太平世,黄河清、北海晏有凤来仪。内侍臣摆御驾九龙【回龙】里,

【西皮摇板】君王有道福寿齐。

【西皮快板】一见皇儿泪悲啼,打碎珠冠扯破衣。你与驸马因何起\protect\hyperlink{fn386}{\textsuperscript{386}},一一从头奏孤知。

皇儿平身。

赐座。

慢慢奏来!

【西皮摇板】御妻休得本奏启,

【西皮摇板】皇儿且莫泪悲啼(或:皇儿也莫泪悲啼)。

【西皮摇板】你母女暂且回宫去。

【西皮摇板】内侍与孤传旨意,快宣皇兄郭子仪。

【西皮二六】九龙口内红光起,来了皇兄郭子仪。昨日里皇兄悬弧喜\protect\hyperlink{fn387}{\textsuperscript{387}},王未曾去拜寿也曾赐过你珍奇。王坐江山全亏你,从今后赐你剑、履上丹墀。内侍臣与孤搀扶起,

【西皮摇板】君臣对坐把话提。

【西皮摇板】殿角绑的何臣子,一一从头说孤知(或:一一从头奏孤知)。

(郭子仪 【西皮摇板】请王传旨将他斩。)

【西皮散板】老皇兄做事太心急。况且驸马轻年纪\protect\hyperlink{fn388}{\textsuperscript{388}},公主又是少年妻。自古道清官难断家务事,他夫妻吵闹常有之。孤皇传旨不降罪,快与驸马去换朝衣。

皇兄平身!

赐座。

皇兄,昨晚宫中,驸马缘何与公主争论?

内侍,宣驸马冠带上殿。

平身。

驸马,昨晚为何与公主争论?

皇兄,驸马所言甚是。

从今以后,红灯撤去,只行夫妻常礼。

呃,往下奏来。

啊,皇兄,听驸马所奏,孤是明白了。

昨日皇兄八旬双寿,众家哥弟,一个个成双结对,拜寿堂前。公主不在,驸马一人拜寿,自觉孝道有亏,是与不是?

唉!皇兄啊,自古道:不痴不聋,难做公翁。从今以后,他夫妻之事,你不必劳心呐。

听孤旨下!

【西皮二六】驸马奏本孤的龙心爽,颇知三纲并五常。但愿皇兄多欢畅(或:臣心若得君欢畅),福寿齐眉永安康。老皇兄暂且回府往,王与驸马有商量(或:共商量)。

【西皮散板】驸马近前听旨降:忠臣孝子永留芳。孤王赐你尚方剑,命公主赔罪到汾阳。

【西皮散板】内侍臣摆驾后宫进,见了御妻说分明。

\newpage
\hypertarget{ux73e0ux5e18ux5be8-ux4e4b-ux674eux514bux7528}{%
\subsection{珠帘寨 之
李克用}\label{ux73e0ux5e18ux5be8-ux4e4b-ux674eux514bux7528}}

{[}第一场{]}

\textless{}\textbf{点绛唇}\textgreater{}荆棘铜驼\protect\hyperlink{fn389}{\textsuperscript{389}},唐室残破。离朝阁,自立山河,沙陀全归我。

(念)太白斗酒诗百篇,长安市上酒家眠。摔死国舅段文楚,唐王一怒贬北番。

孤,李克用呃,祖居沙陀,先父朱(姓,讳)国昌\protect\hyperlink{fn390}{\textsuperscript{390}},归顺唐室。讨贼有功,因赐国姓(或:只因屡建奇功,唐王见喜,恩赐国姓)。唐王见孤左眼小比龙,右眼大比虎,生就龙虎之姿,认孤为螟蛉义子殿下(或:认孤为义儿殿下),赐名``鸦儿''\protect\hyperlink{fn391}{\textsuperscript{391}}啊。

只因那年,孤王得胜还朝,唐王见喜,在五凤楼前恩赐御宴,文武百官庆贺千秋,(或:只因那年,孤王得胜还朝,唐王在五凤楼前恩赐御宴,以贺千秋,)内有国舅段文楚,笑孤坐席不正,礼貌不周。怒恼孤家,隔席抓过,(我)就摔呃------摔在丹墀,那贼就口吐鲜血而亡了。唐王大怒,将孤推出午门斩首,多亏恩官程敬思连保数本,唐王赦了死罪,将孤削职,贬回故土。(或:唐王死罪已免,活罪难饶,将孤谪贬沙陀为民。)

来到沙陀(或:是孤来在沙陀),众家王子,顶盔贯甲,拦住孤的马头,俱要与孤(王)比试。那时孤哪有什么闲情逸致与他们玩耍,是孤稳坐雕鞍,心生一计,将孤的九九八十一斤定唐宝刀------哗喇喇------耍上数路,众家王子一个个拜服马前,尊孤为首。一路之上,收了(或:收下)二位皇娘,一十一家太保。来到沙陀(或:来在沙陀),风调雨顺,国泰民安。朝朝饮宴,夜夜笙歌。好不洒乐人也!正是:(念)红尘一点不到处,

太保,回来了?

打来(或:打了)多少飞禽走兽?

(哦,)什么新闻?

嚯------胆大黄巢,欺我唐室无人。

太保(听令,)传令(下去):二位皇娘挂帅,众家太保以为前站先行,带领(或:发动)沙陀国四十五万番汉兵将,前去兴唐灭巢!

慢,慢\ldots{}\ldots{}慢着!唐王无道,将孤谪贬(或:当年唐王将孤谪贬),哪有人马与他解围。

太保,原令追回!

太保,你是怎样知晓?

哦,程恩官来了。

他乃孤王(或:孤家)活命恩人,(必须迎接于他。)太保------

吩咐摆队相迎。

{[}第二场{]}

久违了。(或:啊,恩官。)

你也皓然\protect\hyperlink{fn392}{\textsuperscript{392}}了哇。

一样。

啊------呵呵呵哈哈哈\ldots{}\ldots{}(笑介)

恩官到此,乃是客位。

恩官请。

(如此)你我挽手而行。

{[}第三场{]}

且慢,你乃孤活命恩人,受孤一拜。

太保,见过儿的程叔父(或:拜见程叔父)。

唐王驾安?

满朝文武可好?

有劳他们。

请坐。

不知恩官驾到,未曾远迎,当面恕罪。

岂敢。

看酒来,待孤把盏。

太保代敬。

恩官请。

干!

正是:(念)忆昔五凤楼,相隔有数秋(或:相隔数十秋)。

好哇------好一个``叙叙旧根由''。

【西皮导板】太保传令把队收,

干!

【西皮原板】孤与贤弟叙一叙旧根由。

【西皮原板】忆昔当年五凤楼,文武百官庆贺千秋。内有个文楚段国舅,他笑孤王坐席不正、礼貌不周。怒恼了孤王气冲牛斗,隔席抓过摔死龙楼。摔死了国舅段文楚,唐主爷一怒要斩头。自从那年离朝后,今日里相逢在北州。

(程敬思\protect\hyperlink{fn393}{\textsuperscript{393}}
【西皮原板】自从千岁离朝后,满朝中文武泪双流。为千岁懒把朝房走,为千岁懒观五凤楼。山遥路远少来问候,望千岁恕学生礼貌不周。)

【西皮导板】太保推杯换大斗,

【西皮快板】李克用跪席前脸带惭羞。当初不该打死国舅,怒恼了唐王要斩人头。如不是恩官把本奏,孤王焉有活命留。天高地厚恩少有,这一斗水酒你要饮下喉。

(程敬思
【西皮快板】用手儿接过梨花盏,学生大胆把话言:甲子年,开科选,山东来了一生员。家住曹州并曹县,姓黄名巢字巨天\protect\hyperlink{fn394}{\textsuperscript{394}}。三篇文章作得好,试官点他为状元。跨马三日游宫苑,宫娥、彩嫔笑连天。唐王嫌他容貌丑,斩了试官革状元。斩了试官不要紧,革了状元起祸端。祥梅寺\protect\hyperlink{fn395}{\textsuperscript{395}},造了反,将我主驾逼在西祁\protect\hyperlink{fn396}{\textsuperscript{396}}美良川。学生到此无别干,一来搬兵二问安。)

【西皮快板】听说黄巢造了反,不由得孤王笑连天\protect\hyperlink{fn397}{\textsuperscript{397}}。贤弟饮宴且饮宴,提起了唐王孤不耐烦。

(程敬思
【西皮快板】我这里提起唐天子,这老儿一旁不耐烦。是是是,明白了,老儿是个爱宝男。叫人来将宝搭上殿,特请千岁把宝观。)

【西皮快板】一见珠宝帐前摆,不由得孤王笑颜开。上有蟒袍和玉带,凤冠头上插金钗。明明知道佯不解,假意儿上前问开怀。你做清官数十载,此宝打从何处来。

(程敬思
【西皮快板】此宝出在山海外,三年五载进宝来。唐王爱将恩似海,特命学生进宝来。)

【西皮快板】贤弟进宝因何故,

(程敬思 【西皮快板】特请千岁把兵排。)

【西皮快板】年纪迈,血气衰,难作国家的栋梁才。

(程敬思 【西皮快板】千岁爷虎老雄心在,黄巢闻名他不敢来。)

【西皮快板】贤弟休得把孤抬,有一辈古人说上来:昔日有个姜吕望,稳坐钓鱼台他不下来。

(程敬思
【西皮快板】钓鱼台,不下来,他保周朝八百载。千岁不发人和马,黄巢笑你老无才。)

【西皮快板】笑只笑唐天子,他笑孤王为何来。中军帐,挂了帅,众家太保两边排。一马儿踏入唐室界,万里的乾坤扭转来。

(程敬思 【西皮快板】说此话就该发人马,)

【西皮摇板】唐王晏驾你再来。

(程敬思 【西皮摇板】问千岁此宝爱不爱?)

【西皮摇板】孤念你千里迢迢路远来,却之不恭呃,受之有愧,来来来,一体全收哇往后抬。

(程敬思
【西皮快板】这老儿做事不公平,收了宝物不发兵。用手取出唐王旨,我奉圣旨来调兵。)

(程敬思 圣旨下。)

呃!

【西皮快板】程敬思做事太无情,不该圣旨欺寡人。用手拿过(或:接过)皇王旨(或:唐王旨),回手压下帝王(的)文。哪一个再提发兵事,定斩沙陀不徇情。

(程敬思
【西皮快板】一见千岁变了脸,回头埋怨李嗣源。我在松林寻短见,不该救我活命还。)

【西皮快板】奴才做事真胆大,胡言乱语少家法(或:把话答)。(或:我与恩官来讲话,大胆奴才把话答。)吩咐两旁武士手(或:刀斧手),推出帐去(或:推出午门)把头杀。

(程敬思 【西皮摇板】千岁要斩把学生斩,快快赦回太保还。)

【西皮快板】我与恩公(或:恩官)来讲话,奴才一旁(或:奴才竟敢)把话答。恩公若回长安转,耻笑孤王无家法(或:少家法)。

(程敬思 【西皮摇板】有家法来无家法,看学生薄面绕过他。)

【西皮摇板】贤弟(或:恩官)不必礼恭敬,帐外赦回太保身(或:午门赦回小畜生)。

【西皮摇板】一足将儿踏帐下(或:恨不得一足将儿踏),

【西皮摇板】程恩官讲情儿要谢过他。

【西皮导板】昔日有个三大贤,

【西皮原板】刘、关、张结义在桃园。弟兄们徐州曾失散,古城相逢又团圆。关二爷马上呼三弟,张翼德在城楼怒发冲冠。你既然降了奸曹操,看来是无义反桃园(或:负义反桃园)。耳边厢又听【转西皮快板】人呐喊,老蔡阳的人马来到了古城边。城楼上助你三通鼓,日月旌旗壮壮威严。哗喇喇打罢了头通鼓,关二爷提刀跨雕鞍。哗喇喇喇打罢了二通鼓,人有精神马又欢。哗喇喇打罢了三通鼓,蔡阳的人头落在马前。一来是老儿该丧命,二来弟兄得团圆。贤弟休回长安转,就在这沙陀过几年,落得个清闲。

{[}第四场{]}

(程敬思
【西皮快板】过了一天又一天,心中好似滚油煎。眼望长安难回转,不知唐王驾可安。)

【西皮摇板】贤弟不必(或:休得)想唐朝,长安哪有(或:焉有)此地高。沙陀国有你的乌纱帽,沙陀国有你紫罗袍。

【西皮导板】贤弟随孤哇来观瞧,

【西皮快板】队队旌旗空中飘。(在后营有的是粮和草。众家太保杀气高:)大太保亚赛温侯貌,二太保上阵似白袍。三太保上山擒虎豹,四太保下海斩龙蛟。五太保惯使开山斧,六太保手持丈八矛。七太保金枪(或:银枪)真奥妙,八太保手持青龙偃月刀。九太保双锏(或:金锏)耍得好,亚赛秦叔宝,十太保钢鞭逞英豪(或:鞭插马鞍桥)。还有个十一小太保,他的武艺好,双手能打火龙镖。哪怕黄巢兵来到,孤与他枪对枪来刀对刀。

(程敬思 【西皮摇板】众家太保武艺好,你不发兵我心焦。)

【西皮摇板】贤弟休得心内焦(或:心烦恼),当饮酒时且逍遥。来来来,吃几杯解烦恼,

(程敬思 【西皮摇板】程敬思一旁闷无聊。)

慌什么?

对她二人(或:对她们)言讲,现有长安贵客在此,少时(或:少刻)退帐再来传见。

又慌什么?

方才言过,退帐(再来)传见,为什么又来啰唣。

呃,忒以的啰嗦了。

哎呀,得罪了(,得罪了)。

罢了,你(们)二人进帐何事?

唐王无道(或:当初唐王将孤谪贬),哪有人马与他解围。

那是送与孤家的,提它则甚? (或:呃,呃------俱都入了库了。)

(这凤冠霞帔么,)呃------(也)一齐入了库了。

孤心已定,休得多言(或:不必多奏)。

嗯------孤就是不发兵。(或:孤心已定,就是不发兵呐。)

(这做什么?)

啊?慢说是三声,(就是)三十声、三百声,又有何妨(或:又待何妨)啊?

呃,(呃,呃,我)一个不发兵。

嗯,(呃,呃,这)两个不发兵。

呃,你不必前来插足啊。(或:我们之间的事体,与台驾无关呐。)

呃,(诶------)我就是不发兵。

(诶呀!)

哦,倘若来迟呢?

(啊)太保,我们商量商量。

呃,我们商量商量。

唉!

【西皮摇板】大太保本是惹祸精呐,到后宫搬来了两个夜叉妇人呐。顺水推舟我把人情送,我为你点动了(或:我为你发动了)番汉兵。

(程敬思
【西皮摇板】千岁休得人情送,学生心内明如灯,皇娘人马来点动,程敬思不领你这空头情。)

【西皮摇板】贤弟休得笑盈盈,休笑愚兄(或:孤王)我怕,(我)怕,(我\ldots{}\ldots{})怕妇人呐。沙陀国内访一访你再问一问,怕老婆的人儿我是第一名。

{[}第五场{]}

(念)白发白须似银条,胸中韬略智谋高。也是黄巢气数到呃,试试------孤的定唐刀。

唉!只因黄巢造反,勒逼唐王驾幸西祁美良。程恩官解押珠宝来到沙陀,借兵解围(或:搬兵解围)。本当不发人马(或:兵马),可笑我那两个无知的妇人,一个要发兵,一个要挂帅。发兵也罢,挂帅也罢,(或:是一个要什么挂帅,一个要什么发兵;唉,挂帅也罢,发兵也罢,)也不知怎么(或:怎样)糊里糊涂地,把一个(或:这个)前站先行,弄到孤家的头上来了。

本当不遵,怎奈她们的家法,实在地厉害呀(或:怎奈是她的家法十分地厉害)。为此紧急料理宫廷善后之事,全身披挂,校场听点。(或:为此急忙料理完毕宫廷善后之事,急忙辕门听点。)

来(来来),带马带马。

呃呃呃,你二人为何争论起来。

哦,你不是看守宫殿的老军么?

你(前)来则甚呐?

诶------两军阵前,刀枪无眼呐,倘有伤损(或:倘有差错),那还了得?(你呀,)还是养养你这老命吧!

哦,看你不出,倒有一片爱国的精神。(或:听你之言,倒有一片爱国的精神呐。)

嗯,就命你与孤带马。(或:好好好,我就教你带马。)

{[}第六场{]}

【西皮快板】又听辕门(或:耳听辕门;又听营门)放号炮,众家太保杀气高(或:众家儿郎逞英豪;或:儿郎个个杀气高)。来在辕门下鞍鞒,

呵嘿!

【西皮摇板】误卯牌悬挂(或:误卯牌高挂)要糟糕。

呃,来了!

我早就来了,怎么(说)误了呢?

传旨进去,就说孤王驾到,教她们快快迎接。(或:传话进去,就说孤王到了,教她们下位迎接于我。)

呃,我们是夫妇顺呐。

怎么讲?(或:啊?!)

呀呸!不来(下位)迎接,还则罢了,反教孤王(或:反教我)报门而进。

哼哼,呵呵,(这人马是孤家的,)我不干了,另请高明罢!

反了哇,反了哇!

【西皮摇板】大太保传令理不通,不由孤王怒气生。(或:太保传令山摇震\protect\hyperlink{fn398}{\textsuperscript{398}},不由孤王胆战惊。)

【西皮摇板】本当进帐(或:本当与她)来争论,怎奈是她的家法比国法还要狠十分。孤若不遵她的令,到晚来不教孤进孤的(或:她的)卧室门呐。

【西皮摇板】东宫不留把西宫进,

【西皮摇板】西宫也是照样行------关门熄了灯。

【西皮摇板】闹得孤(或:恼得孤)黑夜里无投奔,银安殿上把闷气生。

【西皮摇板】孤王一生好把酒来饮(或:也是孤王好心性),好酒贪杯惯坏了她们呐。

【西皮摇板】沙陀国内访一访,你再问一问,家家有本难念的经,个个观世音。叫老军与孤王【回龙】你就报门进,

【西皮摇板】上面坐定两个夜叉精。她二人狼狈为奸端了一个稳,她那里不语(或:不言)我也不作声。

(小嗓)不错,是我啊!(或:是啊,来啦!)

我这个先行,不是花钱买来的,也不是运动来的。乃是你们(或:你二人)亲自委派的。

孤王料理宫廷善后之事,一步来迟,何必(这样的)大惊小怪呀。

哎呀,糟了\ldots{}\ldots{}

(呃呃呃,还好还好啊。)

(哎呀呀,)这还了得?!哼!

这才是(或:这就是)父子亲呐。

哦,不用我了?这就好了。(或:怎么,你不用我了?)

劳您驾------

在。

得令啊!

古来无有的事,如今都有了!古来无有的事,如今都有了!

呃,(古来就有,)你且讲来。

看你不出,你还知道这么多的历史故事啊(或:倒晓得这么些个历史的知识啊)。

你讲得不错。

呃,孤(王)就是,呃,有这样(或:这么)一点点的短处。

呃,这就好了,这就好了。

呃,你不曾听见吗?

皇娘传令,赐孤(或:赐我)五千名虎卫军,压住后队。

嗯,这倒是一个美差。

哦,难道孤听错了?(或:怎么,难道孤听错了?)

呃------你这个人,怎么这样势利眼呐?

那些太保,你们看来一个个如狼似虎,他们只能在围场之上,行围射猎;在沙场之上,交锋对垒,还要看孤家。(或:呃,你不要看那些太保们,一个个如狼似虎,他们只能拿强捕盗,行围射猎;战场交锋,还要看孤家的。)

那个自然,带马!

{[}第七场{]}

【西皮快板】将令一出山摇动(或:山摇震),儿郎个个胆战惊。来在营门下金镫(或:来在辕门下能行;或:催马来在辕门近),

【西皮摇板】这样的紧急为何情呐。

哦,恩官。(或:哦哦哦,请坐请坐。)

哦,恩官。(呃呃呃,哎呀,得罪了得罪了。)

呃,(那)我的座位呢?

(哦,谢坐谢坐。)

呃,家无常礼呀。

调我前来则甚呐?(或:调孤前来有何军情议论?)

哦,谈谈心?

呃,(那)我们就谈谈心呐。

想是那周德威呀?

无名小辈,草莽贼寇,(或:草莽贼寇,无名小辈,)何足道哉?

会他一会么?(又有何妨啊?)

(哦,)今天不耐烦。

不伺候。

呃,什么叫作好处?你说将出来,我听上一听。(或:哦,我倒不晓得有什么好处,呃,有什么好处呢?)

(哦哦哦\ldots{}\ldots{})

噫------(哈哈哈\ldots{}\ldots{}(笑介))你不是骗了我一次了!我再也不上当了。(或:我再也不信了!你骗了我不是一次了!)

你说将出来,教大家听上一听(或:看上一看),看看使得使不得。

打仗也吃酒,不打仗也吃酒。这酒么------不稀罕(或:呃,吃酒不稀罕)。

呃,我们是家务事啊,不劳台驾呀。(或:诶,这是我们家务事,你不要前来插足哇。)

(你说)哪个老了?

(哦,你说孤王老了?)

(孤王)我老只老头上发,项下须,胸中韬略却还不老!

有道是:(念)虎老雄心在,这------年迈呀------力刚强。

你(呀,)拿过来吧!

【西皮二六】老只老孤的须发老,胸中的韬略比人高。非是孤王不服老,上阵全凭马和刀。草莽的贼寇何足道,教他来试一试孤的九九八十一斤定唐刀。

【西皮快板】你把酒宴安排好,得胜回来贺贺功劳。叫老军与爷带马到,

【西皮散板】会一会山寇小儿曹。

{[}第八场{]}

(堂鼓轻击,龙套在九龙口左右站斜旗门,\textless{}\textbf{撕边}\textgreater{}李克用、周德威左右旗门冲出,双出门,李从台中间左转身回到台中间漫周头,打周鼻子(周与李相反走,\textless{}\textbf{撕边}\textgreater{}李、周分别左右转身回到旗门,同时抱刀亮、龙套领起来分站左右,李、周出来架住)

(念)呔!马前来的敢是周德威?

周德威,看你相貌堂堂,为何失身落草?
(或:我看你相貌堂堂,文韬武略,不该落草为寇。)依孤相劝,归顺孤家,封你以为一家太保,你且三思。

呜哙呀,他还惦记孤(家)的珠宝哩!

唉,全都烧光了。

哼,你若胜得过孤(王的)这定唐宝刀,(孤)将珠宝与你留下。

你若不胜?

丈夫一言------

(你我各传一令!)

众将官,压住阵脚。

【西皮导板】叫三军与爷战鼓伐,

(堂鼓轻击,钻烟筒,\textless{}\textbf{冲头}\textgreater{}一合两合,搕开,\textless{}\textbf{紧锤}\textgreater{})

【西皮快板】马前闪出年少娃。量儿本领有多大,敢与老夫动杀法伐。

(一合两合,李接上下左右,刀头拉转身,李被漫头过去到小边,向里回头打周后蓬头,拉肚转身,李被勾走马腰封到大边,绞起来李压周刀,往里一盖两盖,往外一盖两盖,起大刀花蹦子转身剁周头,亮住(李平端刀指周)。\textless{}\textbf{香柳娘}\textgreater{}亮收对面互看,夸奖,双出门分别左右转身对面拉开,搕,里面拉开(\textbf{不要正冠}),搕,李回花转身到大边里面抱刀单腿立亮相(周小边外面矮相),李向外边大刀花转身到大边外面斜横刀弓箭步矮相(周小边里面亮相),李向上场门出刀转身砍过去到中间里边,面外抱刀亮相(周里面矮相),拉转身到小边,对面拉开,搕,拉开,搕,回花转身到小边外面斜横刀矮相,向里边大刀花转身到小边里面,倒手外边面里斜横刀矮相,拉,回到大边(\textless{}\textbf{牌子}\textgreater{}停),搭,拉到大边里角,一合到小边外角,从小边外角经小边内角向大边外角退,倒提柳到大边外角,横着一个大刀花过合,两个大刀花过合,又到大边,打周上下左右,勾周走马腰封到大边,李归小边,原地被勾刀转身向里接鼻子,在原地勾周刀左转身向外打周鼻子,刀鐏盖周刀再打一个鼻子,转身切刀亮住,接耍下场,普通大刀下场下,但在劈马、正花转身面外亮住、串腕转身之后,再来一个正花转身面外串腕,弓箭步拿刀杆中间向外亮,缓刀掠刀追下)
\protect\hyperlink{fn399}{\textsuperscript{399}}

【西皮散板】接过雕翎箭一条。

【西皮散板】这样的射法不算好,放箭哪有接箭高。

【西皮散板】接过雕翎箭二根。

【西皮散板】这样的射法不算准,孔夫子门前你卖的什么文。

【西皮散板】勒马停蹄战场等,停箭不射为何情?

周德威,战又不战,降又不降,又在那里弄的什么诡计?

你的箭法呀,哼,孤(王)方才领教过了。

(要)怎样比试?

但不知哪家先射?

孤若先射,就无有你的份了(或:孤王先射,就无有尔的份了!让你先射)!

站定了!

【西皮散板】量尔不是汉李广,养由基再世又何妨。

【西皮散板】满满搭上朱红扣,

(诶,)你把我闹糊涂了!

【西皮散板】观不见金钱在何方。

【西皮散板】低下头来暗思想,

啊?!

【西皮散板】忽然一计上胸膛。

(周德威 李克用,\ldots{}\ldots{}停箭不射?)

非是孤王停箭不射,想这金钱(或:想那金钱)乃是一面死物,纵然射中,也不足为奇。

抬头观看------

空中飞的何物?

好哇------孤今(或:孤王一箭上去)要射它(一)个双雕落地。

丈夫一言------

(太保,)站定了。

【西皮散板】克用暗地告上苍,祝告天地日月光、过往的神灵听端详:我若有福收此将,箭射双雕落平阳。

(哪里走。)

【西皮散板】德威可算英雄将(或:忠良将),封你太保在朝堂(或:在朝廊)。(或:你若真心把孤保,封你太保辅大唐。)

(罢了,)见过众家哥弟。

不必查点。

转至御营(或:同至御营),见过你二位皇娘。

据\textbf{陈超老师}介绍,《珠帘寨》有``\textbf{烧宫}''一场,刘曾复先生有特别传授,兹照录如下:

程敬思 (唱)李克用依记前仇恨,且喜皇娘发救兵。

程敬思
是我解宝到此搬兵,不想李克用记恨前仇不肯出兵,多蒙二位皇娘发动倾国人马,李克用以为先锋,兵马已在四十里外扎营。是我悄悄折回,将李克用的宫殿焚毁,教他此去难回也。

程敬思 (唱)非是我程敬思忒心狠,为保我主锦乾坤。

李克用 (内)【西皮导板】王宫火光冲天境,

(李克用、老军上)

老军 老千岁,您可小心这点儿。

李克用 (唱)急忙转回看分明,顾不得即出征军情紧(圆场、下马)

老军 好大火啊(李克用扑火)。都烧成这样了,您就别忙活啦!

李克用 嘿嘿,

李克用 (唱)宫殿全已被火焚。

老军 回来您再盖新的吧。

李克用 唉!(唱)孤多年以来积攒的奇珍异宝------

老军 我的被窝褥子------

李克用 (唱)顷刻之间就化为飞灰,叫孤怎不心疼啊。

老军
老千岁您不错啦!我连被窝褥子都没啦!(\textless{}\textbf{三鼓}\textgreater{})

老军 完了!聚将鼓响了,咱们再不走可来不及啦!

李克用 (唱)不住战鼓来催命,

李克用 马来

李克用 (唱)此一去定回归重建宫廷。

\newpage
\hypertarget{ux592aux5e73ux6865}{%
\subsection{太平桥}\label{ux592aux5e73ux6865}}

{[}第一场{]}

(四绿龙套大锣打上,站门,\textless{}\textbf{四击头}\textgreater{}朱温上)

朱温
\textless{}\textbf{点绛唇}\textgreater{}光照旌旗,青虚钓实,龙唾涕,惊走鳌鱼,要掌锦华夷。(小座正座)

朱温
(诗)忆昔当年雅观楼\protect\hyperlink{fn400}{\textsuperscript{400}},赌头夺带面惭羞。孤今驻军汴梁地,安排巧计报前仇。

孤,梁王朱温,今有李克用兵扎驼龙岗,也曾命王刚打探虚实,未见回报。

王刚 (内)走哇!(王刚上)

王刚 打听晋王事,报与驸马知。参见驸马。

朱温 罢了。

王刚 谢驸马。(站小边)

朱温 命你打探晋王虚实,有何消息?

王刚
启禀驸马,晋王兵扎驼龙岗,命十三太保李存孝巡查黄河,营中空虚。晋王每日饮酒,醉而复醒,醒而复罪,不理军情,特来禀报。

朱温
起过了。且住。老儿一到训地不理军情,存孝不在营中,我不免趁此机会,请他汴梁阅兵,会上杀他,以报前仇。左右溶墨伺候。(朱温进大座)

朱温
【西皮导板】上写朱温多拜上,【导板】拜上千岁李晋王。皇姑宫院身不爽,思念天子泪成行。千岁驾临驼龙岗,未获拜趋在道旁。汴梁设下阅兵会,酒席宴前饮琼浆。写罢书信唤王刚,驼龙下书走一场。

(王刚接书下,朱温出位站台中间)

\begin{quote}
【摇板】汴梁设下天罗网,管教亚儿丧无常。(朱温众下)
\end{quote}

{[}第二场{]}

(四龙套站门,周德威、史敬思、李克用上,李中,周大边,史小边站台口)

李克用 旌旗遮日月。

周德威、史敬思 (同念)龙虎保乾坤(或:龙虎扶乾坤)。

(李克用中间小座)

周德威、史敬思 (同念)参见父王。

李克用 二皇儿免礼,赐座。

周德威、史敬思 (同念) (谢父王。)儿臣谢座。

周德威 将军,请坐。

史敬思 先生,请坐。

(周大边,史小边八字座)

李克用 二位皇儿。

周德威、史敬思 (同念)父王。(将儿臣唤出,有何军情议论?)

李克用 我父子兵扎驼龙岗,个月有余,缘何不见朱温前来问安?

周德威、史敬思 (同念)他乃叛逆之臣(或:他乃叛逆之人),提他做甚!

李克用 来。

龙套 有。

李克用 伺候了。

(王刚上,台口小边)

王刚 领了一件事,千金不敢移。来此已是,营门哪位听事?

龙套 做什么?

王刚 烦劳通禀千岁,就说朱驸马差来下书人求见。

龙套 营外稍站,待我禀报。启禀千岁,朱驸马差来下书人求见。

李克用 教他自进。

龙套 千岁命你自进,要小心了。

王刚 有劳了。(王刚进帐,跪)

王刚 参见千岁。

李克用 罢了,立起讲话。

王刚 谢千岁。(王站大边)

李克用 你奉何人所差?

王刚 今奉朱驸马所差,这有书信呈上。

(王呈信、李接)

李克用 命你回禀驸马,孤照书行事。

王刚 谢千岁。

(王刚下)

李克用 二位皇儿。

周德威、史敬思 (同念)父王。

李克用 朱温有书信到来,哪位皇儿观看。

周德威、史敬思 (同念)父王御览。

李克用 待孤拆书,父子同观。(\textless{}三枪\textgreater{}李克用看信)

李克用 呜哙呀,原来朱温设下阅兵大会,请孤汴梁赴会,哪位皇儿保驾?

周德威 十三太保不在营中,无人保驾。

史敬思 先生,俺史敬思不才,愿(带)领四十名长枪手,保定父王汴梁赴会。

周德威 保得去?

史敬思 保得去。

周德威 保得归?

史敬思 保得归。

周德威 未必!

史敬思 为何?

周德威
自盘古以来,只有湘江、临潼大会,无有什么阅兵大会,我想那朱温多奸多诈,父王还是不去为妙。

李克用
皇儿。【西皮原板】皇儿说话差又差,为父言来听根芽。朱温皇宫招驸马,金枝玉叶招赘他。转面我把德威唤,【摇板】你在那八卦之中仔细查。

周德威 儿遵命。(周德威立,大边台口)

周德威
【西皮摇板】父王命我查八卦,【快板】背转身来仔细查。一请前朝文王卦,二请周公与桃花。三才四象安天下,五行六爻定邦家。七星袖内查八卦,啊,【摇板】白虎当头有凶煞。(白)将军,(接唱)我劝将军休保驾,此去难免动杀伐。

史敬思
先生。【二六】先生说话理太差(或:言太差),长他人的威风灭却咱。战场交锋(或:交锋对垒;两军阵前;战场之上)如戏耍,我把那朱温当作小娃。上阵全凭胯下马,虎头金枪掌中拿。倘若是席前有奸诈,学一个单刀赴会名扬天涯(或:学一个单刀赴会万古夸)。

周德威
【摇板】将军不听我的话,再把言语叮咛他。(白)将军,(接唱)你此去逢桥休下马,

史敬思 (为何?)又是为何?

周德威 (接唱)``太平''二字谨防它。

史敬思
【摇板】先生休说懦弱话,非是末将把口夸。(或:先生说话理太差,末将言来听根芽:)大丈夫生至在三光下,生死二字何惧他。

李克用 周德威听令。

周德威 在。

李克用 为父赐你大令一支,命你镇守驼龙岗。

周德威 得令。

李克用 看衣改换。

(牌子合龙,李、史换衣介,四上手上)

李克用
【西皮摇板】人来带过白龙马,(上手带马,李克用、史敬思上马,上手、史下,李收腿)

李克用 (接唱)汴梁阅兵免征杀。(周德威送,李克用下。周归中间)

周德威 【西皮摇板】敬思不听我的话,(龙套斜撤)

周德威 (接唱)太平桥前有凶煞。

(周德威、龙套下)

{[}第三场{]}

(四绿龙套站门引朱温上)

朱温
【西皮摇板】汴梁设下阅兵会,要把克用性命追。将身且坐宝帐内,(小座正座)等候王刚下书回。

(王刚上,进门大边站)

王刚 晋王驾到。

朱温 传卞意随进见。

王刚 卞意随进见。

卞意随 (内)来也。(卞意随上)

卞意随 驸马传唤,急到帐前。

(卞意随进门参见、小边站)

卞意随 参见驸马,有何将令?

朱温 命你埋伏太平桥下,刺杀李晋王不得有误。

卞意随 得令。(卞意随下)

朱温 吩咐众将,摆队相迎。

王刚 摆队相迎。

(起牌子,龙套摆队反下,王刚、朱温反下)

{[}第四场{]}

(牌子中史敬思上,马上起霸,勒马站中场,四上手引李克用上过场,史敬思小趟马亮相下)

{[}第五场{]}

(牌子中绿龙套反上斜胡同,朱温反上望上场门场,四上手上斜胡同,李克用上,李下马)

李克用 驸马。

朱温 千岁,来在长亭,千岁请往前行。

李克用 驸马请往前行。

朱温 这就不敢,千岁前行。

李克用 你我挽手而行。(同笑介)

(李克用下,四上手随下,史敬思上扎过去,朱温众归小边,史敬思亮相下,朱温背供指,做杀介,领众下)

{[}第六场{]}

(牌子中四龙套、四上手上挖门,各走各边,李克用、朱温上,进门李坐大边,朱坐小边,牌子停)

朱温 千岁驾到,本宫未曾远迎,千岁恕罪。

李克用 岂敢,某来得鲁莽,驸马海涵。

朱温 岂敢。

李克用 告便。

(李克用、朱温立)

朱温 千岁意欲何往?

李克用 探望皇姑疾病。

朱温 皇姑闻得千岁驾到,疾病已然痊愈。

李克用 此乃驸马洪福。

朱温 全仗千岁虎威。

李克用 啊?

朱温 啊,(同笑介)

朱温 千岁请坐。

李克用 请坐。(同坐)

李克用
请问驸马,自盘古以来,只有湘江、临潼大会,未闻有何阅兵大会,驸马指教。

朱温 阅兵会上不过是水酒薄肴,与千岁同饮。

李克用 如此说来,到此就要叨扰。

朱温 千岁后宫请。

李克用 请。

(起牌子,李克用、朱温,龙套、上手两边分下)

{[}第七场{]}

(牌子中史敬思上,王刚反上,史、王回身望门,二人见面,史用手漫王头过去,回身右脚蹬王弓箭步左腿上,史拔剑三笑,剑不出鞘,漫王头,退望王,双撩下甲,转身下,王望,比势史身高,杀,怕介下)

{[}第八场{]}

(牌子中李克用、朱温、史敬思、王刚上,李进门,史挡朱随李进,朱、王进,李、朱分坐八字桌大座,牌子停)

朱温 开宴。

(王刚归中间站)

王刚 上宴。

(史敬思轰开,拔剑两望,挑桌袱搜桌下,站台中间亮相)

朱温 千岁请。

李克用 驸马请。

(史敬思、王刚退桌外侧,牌子,朱温、李克用饮酒,王撞钟介)

史敬思
啊,\textless{}\textbf{撞金钟}\textgreater{}【西皮摇板】忽听(或:又听)金钟一声响,

(绿龙套上,站小边一字,史出门双望,回中间)

史敬思 (接唱)刀枪剑戟列两旁。转面我对(或:上前忙对)父王讲,

(史敬思拉李克用出位站中间,史站旁边,朱温出位小边)

史敬思 (接唱)儿臣言来听端详。今日饮酒休放量,酒席宴前要提防。

李克用
皇儿呀。(接唱)皇儿不必心慌忙,为父言来听端详。汴梁纵有千员将,我儿保驾料无妨。

朱温
千岁,(接唱)千岁说话有志量,本宫言来听端详。千岁好比刘先主,太保亚赛关二王。

李克用 (接唱)驸马说话孤心爽,不由克用喜洋洋。人来将酒满斟上,

(李克用进位饮酒,史敬思暗拉,李饮,史急)

李克用 (【转散板】)多吃几杯又何妨。吃酒要学刘伶样,(念)酒来,

(史敬思再暗拉,李克用饮,史急,李饮)

李克用 (接唱)太白斗酒诗成行。

(李克用醉,王刚又撞钟介,史敬思惊介)

史敬思
哎呀,【散板】又听金钟二次响,此地一定有埋藏(或:倒教豪杰着了忙)。二次再对父王讲,(念)父王,

(拉李克用出位,朱温随出位)

史敬思 (接唱)请出皇姑问端详(或:做主张)。

李克用 (接唱)转面我对驸马讲,请出皇姑问安康。

(史敬思 有请皇姑。)

朱温 有请皇姑。

王刚 有请皇姑。

(公主上)

公主 【西皮摇板】耳听前帐声喧嚷,见了驸马问端详。(进门,站小边里边)

李克用 儿呀,前来见过皇姑。

史敬思 (参见)皇姑。

公主 罢了,你父子到此做甚?

李克用 赴阅兵大会。

公主 哪里是阅兵大会,你父子快快回去。

(朱温打公主嘴巴,公主下,王刚抓李克用,史敬思拔剑杀王,回身抓朱温带,大推磨,李逃下,史放朱带,史亮相下)

朱温 带路进宫。(朱温众下)

{[}第九场{]}

(公主上)

公主
【西皮散板】心中只把驸马恨,要害兄王为何情?(念)且住,驸马屡次要害兄王,是我今日走漏消息,驸马回宫,岂肯与我甘休,也罢,我不免拜谢父王母后养育之恩,自尽了罢,(接唱)走近前来忙跪定,拜谢父母养育恩,手执钢刀来自尽,罢!(刎下)

(朱温众上,其中一龙套带马鞭,挖门进宫介)

朱温 啊,【西皮散板】一见贱人丧了命,怎不叫人咬牙根,手执宝剑来砍定。

(砍三剑介)

朱温 马来。

(朱温上马,龙套领下)

{[}第十场{]}

(\textless{}\textbf{水底鱼}\textgreater{}史敬思,李克用上,史卸靠,剑插大带中,李褶马褂)

李克用 儿呀,那贼四门紧闭,如何是好?

史敬思
父王,休得(或:不必)惊慌,西南角下(或:西北角下)有一水门,你我父子托闸而走(或:托闸而逃;托闸出城)。

(史敬思、李克用走圆场,见闸,二人下马,史躬身,马交李带,史中场,李小边,史拔剑砍左闸门大横栓环,砍三下,有尘土,小边里边挡脸,同样砍右边栓环,砍中间锁,剑插带中,卸大横栓,放大边,推门,拉开门,开左扇门,开右扇门,假岔,见闸,靠闸,拔剑豁闸底土左右左三下,插剑,与李耳语,一枕,两枕,紧带,岔,托闸,李拉二马钻闸出城,二人上马下)

{[}第十一场{]}

(龙套引朱温上,报,朱众追下,朱用双锏)

{[}第十二场{]}

(卞意随上)

卞意随
(念)领了驸马令,埋伏太平桥。我卞意随,领了驸马将令,在太平桥下行刺,就此埋伏者。(过下场门桥,桥后藏身)

史敬思 (内)马来。(上,勒马站)

史敬思
【西皮散板】人困马乏难交战,不知父王落哪边(或:不见父王在哪边;或:不见父王落哪边)。(念)且住,我与朱温勇战一日一夜,也不知父王逃往何方去了?(望桥)来此已是太平桥,太平桥,呜哙呀,是我保驾临行之时,先生对我言讲,逢桥休下马,太平要提防。待俺加鞭催马过桥。(或:来此已是太平桥,太平桥。且住,临行之时,先生也曾言过(或:先生也曾嘱咐),教我见桥休下马,俺不免打马过桥。)

(打马上桥,马见水中人影,倒退不行,史敬思惊介)

史敬思
且住,看此桥,桥身高大,龟背鱼脊,若不下马怎能(或:焉能)得过。哎呀,俺史敬思一生一世,就是不信那些阴阳八卦,有道是圣天子百灵相助,大将军八面威风,俺今日偏偏要下得马来,牵马过桥。(或:哎呀,俺史敬思一生一世,就是不信那些鬼阴阳八卦,俺今日偏偏要下马过桥。或:哎呀,俺史敬思一生一世,就是不信那些阴阳八卦,俺今日偏偏要下马过桥。)

(下马,拉马上桥,马退,再拉上桥,卞意随从桥后刺史敬思腹中,史左手抓枪,史由桌上退到椅上,卞上桌,史扔马鞭,拔剑砍卞,一二三漫头,向后扔剑,摘盔扔上场门边,由检场接,捋甩发,双手握枪头,在椅上下腰,松枪,蹬椅边,摔硬僵尸头朝小边台口)

(卞意随三笑,由桥后下场门下,李克用上,下马,扶史起坐地,李站大边)

李克用 皇儿醒来。

史敬思
【西皮导板】耳旁又听(或:耳旁听得)父王到,(念)(唉,)父王呀!(史跪)

史敬思
(接唱)【散板】抬头只见(或:开言禀告)老年高。臣子年幼妻年少,一家老小无下梢。

李克用 (接唱)皇儿但把心放了,一家大小永在朝。

史敬思 (接唱)多谢父王加封号。

(鼓架子,史敬思立)

李克用
(接唱)朱温人马似涌潮,(念)哎呀儿呀,朱温人马犹如潮水一般,如何是好?

史敬思
父王休得惊慌,将战袍割下半幅,与儿包裹伤痍,儿与那贼决一死战。(或:父王,休得惊慌,将战袍割下半幅与儿包裹伤痕,与那贼决一死\ldots{}\ldots{};或:父王不必惊慌,将战裙割下半幅,与儿包裹伤痕,与那贼决一死\ldots{}\ldots{})

(李克用扶史敬思裹伤,史里边面内,李过小边,朱温众追过场)

李克用
【西皮散板】眼前若有李存孝,哪怕朱温计千条。(或:\ldots{}\ldots{}闹吵吵。)

史敬思
(接唱)有劳父王带马到,哎呀,伤痍(或:伤痕)疼痛似火烧,咬定牙关跨虎豹(或:含悲忍泪战场到;或:含悲忍泪跨虎豹;或:咬定牙关战场到)。

(李克用带马,史敬思上马,李抄下,卞上)

卞意随 哪里走?

(卞意随漫头过大边,史敬思小边,架住)

(史敬思 来将通名。)

(卞意随 卞意随。)

史敬思 太平桥行刺可是尔?

卞意随 正是你老爷。

(史敬思 放马过来。)

(史敬思夺卞意随枪,杀卞倒地)

史敬思
【西皮散板】只说儿的武艺好(或:适才道尔武艺好),老爷看来不为高(或:依我看来也不高)。这也是儿眼前报(或:这也是尔现成报)。

(史敬思拿枪刺卞意随腹,剑柄打枪鐏三下,枪劐卞腹,扔枪下,朱温上,史杀,朱败下,史追下)

{[}第十三场{]}

(朱温众上)

朱温 弓箭伺候。

史敬思 (内念)哪里走?

(史敬思 罢!)\\
(史敬思上,中箭下,朱温众追下)

{[}第十四场{]}

(李克用上,史敬思上,下马,李下马扶史上桌,拔箭,史下桌,向大边外指,做朱温兵来状,李望,史拔李剑自刎下,李哭介,拾剑上马,朱温上,李败下,朱追下)

{[}第十五场{]}

(四红龙套飞虎旗引李存孝牌子上)

李存孝
俺,十三太保李存孝。奉了父王将令,巡查黄河一带等处,巡查完毕,回营交令。(鼓架子)耳旁听得人马呐喊,军士们,登高一望。

(上中间桌子,朱温追李克用过场,存孝下桌)

李存孝 且住,原来朱温追赶我父王,此时不救等待何时?军士们,迎上前去。

(存孝站大边台口椅上,存众大边一字。李克用上,存孝招手,李下。朱温追上,存众小边一字,朱见存孝,存孝用挝三漫朱头,朱领众从外边抄过去下,与此同时,存孝下椅,存众从里往外小边走,存孝最后押队,绕回来,从下场门追下)

{[}第十六场{]}

(朱温众上,下场门边拉城)

朱温 扯起吊桥。

(朱温众进城下,关城门,李存孝众同上,城里摇旗,免战锣)

李存孝 收兵。

(存孝众上场门下,存孝大边台口亮相下。)

(收城。\textless{}\textbf{尾声}\textgreater{})

\newpage
\hypertarget{ux4e09ux51fbux638c-ux4e4b-ux738bux5141}{%
\subsection{三击掌 之
王允}\label{ux4e09ux51fbux638c-ux4e4b-ux738bux5141}}

\textbf{{[}引子{]}一枝花抛出墙外,为三女,常挂心怀。}

\textbf{(念)食禄君恩数十秋,辅保吾主坐龙楼。皇恩浩荡须答报,赤胆忠心直到头(或:赤胆忠心不到头)。}

\textbf{老夫王允。唐帝驾前为臣,官居当朝首相。夫人陈氏,膝下无儿,所生三女:长女金钏,许配苏龙为妻;次女银钏,许配魏虎为室;惟有三女宝钏,生性高傲,是她为母之病,在后花园中许下心愿,拈香三载。后宫娘娘闻知见喜,恩赐我儿五色绒线,织成彩球一朵。也曾择于二月二日,
在十字街头,(高搭彩楼,)抛球招赘。实望}\protect\hyperlink{fn401}{\textsuperscript{401}}\textbf{打中哪家王孙公子,不想打中了乞丐花郎薛平贵。想我儿乃千金之体,焉能与那花郎匹配?}

\textbf{老夫今日早朝,(在金殿之上,)观见新科状元姓蔡名端,人才出众,我意欲将花郎亲事打退,另将我儿改配}\protect\hyperlink{fn402}{\textsuperscript{402}}\textbf{那新科状元。(也不知她的心意如何,我)不免将她唤出堂前。与她商议。}

\textbf{家院,后堂传话:三姑娘出堂。}

\textbf{我儿罢了!(或:我儿免礼。)}

\textbf{一旁坐下。}

\textbf{恭喜我儿,贺喜我儿。}

\textbf{我儿在十字街头,(高搭彩楼,)抛球招赘,岂非一喜?}

\textbf{但不知那一彩球打中了哪家王孙公子?}

\textbf{唉,哪里是什么王孙公子,就是那乞丐花郎薛平贵。}

\textbf{儿呀,不必如此。}

\textbf{(想我儿千金之体,焉能与那花郎匹配?)为父今日早朝,观见新科状元姓蔡名端,人才出众,为父意欲将花郎亲事打退,另将我儿改配那新科状元。}

\textbf{不知我儿意中如何?(或:也不知儿的心意如何?)}

\textbf{罢了,恕你无罪(或:一旁坐下)。}

\textbf{坐下。}

\textbf{抱什么?}

\textbf{难道说这一彩球就定了儿的终身不成么?}

\textbf{啊,为父的与儿(或:与你)讲话,难道儿与为父的致气}\protect\hyperlink{fn403}{\textsuperscript{403}}\textbf{不成(或:你敢是与为父生气不成)?}

\textbf{既然不与为父致气,就该打退花郎亲事才是。}

\textbf{儿才怎讲(或:儿待怎讲)?}

\textbf{儿就该------}

\textbf{掌嘴!}

\textbf{【西皮原板】小奴才说此话全然不想}\protect\hyperlink{fn404}{\textsuperscript{404}}\textbf{,不由得年迈人怒满胸膛。你大姐配苏龙户部执掌啊;你二姐配魏虎兵部侍郎。唯独你小冤家娇生惯养,千金体配花郎}\protect\hyperlink{fn405}{\textsuperscript{405}}\textbf{脸面无光。}

\textbf{【西皮原板】薛平贵生来命运低,每日里在长街叫化行乞。衣衫不周实实的褴褛,他好比失林鸟无枝可栖。}

\textbf{【西皮原板】我的儿既知【转西皮快板】古礼义,可知晓韩信、张良魏苏秦。}

\textbf{【西皮快板】登台拜帅是韩信,未央宫斩的什么人。}

\textbf{【西皮快板】董永卖身葬父母,仙姬女下凡配何人呐。}

\textbf{【西皮快板】奴才说话理不顺,叫骂为父你为何情。}

\textbf{【西皮快板】要退要退偏要退。}

\textbf{【西皮摇板】我儿不遵为父命,两件宝衣脱下身(或:脱离身)。}

\textbf{乃是圣上所赐。}

\textbf{以表君臣之义(或:君臣情谊)。}

\textbf{哎呀儿啊,只要我儿将花郎亲事打退(或:只要儿打退这门亲事),慢说是两件宝衣,就是这府内的金银,都任儿取用(或:也任儿取用)啊!}

\textbf{哪里去?}

前堂无有儿父,后堂焉有儿母。

\textbf{家院、丫鬟,哪个(或:有人)去至后堂,打折(尔等的)两腿!}

\textbf{为父的怎样(的)把心死了呢?}

\textbf{哎呀儿啊,方才为父的言过:只要我儿打退花郎的亲事(或:哎呀儿啊,只要儿将这门亲事打退,这)府内的金银是任儿取用}\protect\hyperlink{fn406}{\textsuperscript{406}}\textbf{啊!}

\textbf{裁女不裁父!}

\textbf{裁为父的何来?}

\textbf{为父嫌贫爱富(我)为的是哪(一)个啊?}

\textbf{就为的是你这个小冤家啊!(或:我为的就是你这个小冤家呐!)}

\textbf{【西皮快板】膝下无儿怨我的命,养不得老来,儿送不得终。}

\textbf{【西皮快板】若是为父身染病,自有煎汤下药人。}

\textbf{【西皮快板】倘若是为父遭不幸,自有披麻戴孝人。}

\textbf{【西皮快板】父死不见王宝钏,后来若是来相见------}

\textbf{为父的不信!}

\textbf{嚯------}

\textbf{【西皮摇板】活活地气坏了哇我年迈的人呐。}

\textbf{唉!}

\textbf{【西皮摇板】无奈何(或:莫奈何)与我儿三击掌。}

\textbf{罢!}

\textbf{儿啊,呃\ldots{}\ldots{}(哭介)}

\textbf{【西皮摇板】一见宝钏出府门,怎不教人珠泪淋(或:两泪淋)。}

\textbf{【西皮摇板】含悲忍泪后堂进,见了夫人说分明。(或:悲悲切切后堂进,见了夫人定计行。)}

\textbf{儿啊,呃\ldots{}\ldots{}(哭介)}

\newpage
\hypertarget{ux8d76ux4e09ux5173-ux4e4b-ux859bux5e73ux8d35}{%
\subsection{赶三关 之
薛平贵}\label{ux8d76ux4e09ux5173-ux4e4b-ux859bux5e73ux8d35}}

\textbf{{[}第一场{]}}

\textbf{{[}引子{]}驾坐西凉,蒙公主,辅保孤王。}

\textbf{(念)离长安一十八载,思宝钏常挂心怀。恨魏虎将孤谋害,这冤仇何日解开!}

\textbf{孤,薛平贵,大唐人氏。只因当年征战西凉,可恨魏虎将我谋害,用酒灌醉,绑在红鬃烈马之上,赶至两军阵前,被公主擒获,多蒙老王不斩,反将公主匹配。不幸老王晏驾,文武百官辅孤登基。孤继位以来,风调雨顺,国泰民安,算来一十八载。今当设立早朝。}

\textbf{内侍,闪放龙门。}

\textbf{有这等事?}

\textbf{待孤观看。}

\textbf{宾鸿大雁,口吐人言,不祥之兆。}

\textbf{内侍,弓弹伺候。}

\textbf{【西皮散板】自从盘古立地天,宾鸿哪有吐人言。内侍看过弓和弹,对准宾鸿撒了弦。}

\textbf{呈上来。}

\textbf{哎呀!}

\textbf{【西皮导板}】见血书不由人泪流满面,

\textless{}\textbf{三叫头}\textgreater{}宝钏!贤妻!唉,妻呀\ldots{}\ldots{}(哭介)

\textbf{【西皮慢板】点点珠泪洒落胸前。常随官与孤王(或:替孤王)把朝散,撩龙袍端玉带孤离银安。展开了血书从头看(或:展开了罗衫仔细看),字字行行看周全:上写着拜上啊多拜上,拜上了平贵无义(儿)男。自从分别【转西皮二六】汾河岸(或:西河岸),光阴不觉十八年。夫在西凉常征战,妻在寒窑伴月眠。早来三日【转西皮快板】还相见,迟来三日难团圆。看罢血书望长安(或:肝肠断),}

\textless{}\textbf{哭头}\textgreater{}\textbf{王三姐呀,妻宝钏,啊,受苦妻呀。(或:宝钏我的妻啊!)}

\textbf{【西皮快板】想起了魏虎怒冲冠。有朝一日长安转,仇报仇来冤报冤。低下头来心暗算,}

\textbf{【西皮快板】忽然一计上心间。二次撩袍上银安,代战公主把驾参。}

\textbf{平身。}

\textbf{赐座。}

\textbf{公主连日操演人马,甚是辛苦,备得酒宴,与公主同饮。}

\textbf{内侍,看酒。}

\textbf{公主请!}

\textbf{【西皮原板】夫妻们对坐饮琼浆,她哪知血书袖中藏(或:有衷肠)。本当实言对她讲,还须要谨开口慢作商量。}

\textbf{【西皮原板】西凉本是王执掌,怕只怕南朝动刀枪(或:怕南朝兴兵动刀枪)。虽然公主你的(或:公主你虽然)韬略广,你一人怎敌百}万儿郎。

\textbf{【西皮摇板】公主进酒王心爽,多吃几杯又何妨。}

\textbf{公主操演辛苦,孤王要敬酒三杯。}

\textbf{从先能饮多少?}

\textbf{如今呢?}

呵呵哈哈哈\ldots{}\ldots{}(笑介)

还是一样啊!

\textbf{来,看大杯伺候。}

\textbf{公主请!}

\textbf{干!}

\textbf{请------}

\textbf{干!}

\textbf{请------}

\textbf{你也醉了!}

\textbf{【西皮快板】公主醉倒银安殿,中了平贵巧机关。内侍与孤把衣换,}

\textbf{【西皮摇板】番邦令箭带身边。内侍带马休迟慢,}

\textbf{【西皮摇板】难舍公主十分贤。桌案现有笔和砚,}

\textbf{【西皮快板】手提羊毫写周全:你若念在夫妻义,带领人马到关前;你若不念夫妻义,西凉女王坐江山(或:西凉国改作女儿川)。书信放在龙书案,}

\textbf{带马!}

\textbf{【西皮摇板】公主醒来对她言,说王去阅边。}

\textbf{{[}第二场{]}}

\textbf{开关!}

\textbf{奉了公主将令,出关另有公干。}

\textbf{令箭在此。}

\textbf{{[}第三场{]}}

\textbf{开关!}

\textbf{且住,来此三关地带。乃是他国地界,看城上好像莫老将军,待我冒叫一声。}

\textbf{莫老将军请了!}

\textbf{先行平贵在此。}

\textbf{何出此言?}

\textbf{此乃仇人咒骂于我。}

\textbf{红鬃烈马为证呐。}

\textbf{老将军,后面追兵甚急,快快开城再来叙话。}

\textbf{有劳了。}

\textbf{来了!}

\textbf{老将军有何话讲?}

\textbf{有劳了!}

\textbf{【西皮快板】莫老将军对我言,公主领兵(或:带兵)到关前。}

\textbf{【西皮摇板】放心不下敌楼看,}

\textbf{【西皮快板】旌旗遮住半壁天。马达、江海一声唤,快请公主到关前,【转西皮摇板】王有话言:}

\textbf{【西皮快板】那一日驾坐银安殿,宾鸿大雁口吐人言。手持金弓银弹打,打下了半幅血罗衫。展开罗衫仔细看(或:从头看),才知长安(或:才知道寒窑受苦的)王宝钏。非是孤(或:我)私自离宫院,为的长安王宝钏。}

\textbf{是孤的前妻呀!}

\textbf{【西皮摇板】本当与你说真言,公主不放也枉然。}

\textbf{【西皮摇板】一见公主变了脸,不由平贵(或:不由孤王)心胆寒。眼望长安难回}\textless{}\textbf{哭头}\textgreater{}\textbf{转,(宝钏)我的妻呀!}

\textbf{【西皮摇板】夫妻们见面难上难。}

\textbf{【西皮摇板】多蒙公主开恩典,放我平贵转回还。马达、江海一声唤,(孤王言来听根源:人马休回西凉转,)就在此地(或:关前)扎营盘。}

\textbf{【西皮摇板】交还鸽儿金鈚箭,到长安会一会妻宝钏(或:王氏宝钏)。}

\textless{}\textbf{三叫头}\textgreater{}\textbf{公主!我妻!唉,妻啊(或:公主啊)\ldots{}\ldots{}(哭介)}

\textbf{罢!}

\newpage
\hypertarget{ux6b66ux5bb6ux5761-ux4e4b-ux859bux5e73ux8d35}{%
\subsection{武家坡 之
薛平贵}\label{ux6b66ux5bb6ux5761-ux4e4b-ux859bux5e73ux8d35}}

\textbf{{[}第一场{]}}

\textbf{(内)【西皮导板】一马离了西凉界}\protect\hyperlink{fn407}{\textsuperscript{407}}\textbf{,}

\textbf{【西皮原板】不由人一阵阵泪洒胸怀。青的山绿是水花花世界,薛平贵好一似孤雁归来。那王允在朝中官居太宰,他把我贫穷人哪放在心怀。恨魏虎是内亲将我谋害,苦害我薛平贵所为何来。柳林下拴战马武家坡外,}

\textbf{【西皮摇板】见了那众大嫂细问开怀。}

\textbf{列位大嫂请了。}

\textbf{并非失迷路途,我乃找名问姓的。}

\textbf{(提起此人,大大有名,)王丞相之女,薛平贵之妻,王氏宝钏。}

\textbf{为何(或:怎么)不凑巧?}

\textbf{如今呢?}

\textbf{烦劳大嫂转达一声,就说他丈夫(与她)带来万金家书,教她前来接取。}

\textbf{有劳了。}

\textbf{【西皮原板】这大嫂去送信【转西皮快板}】太也迟慢,武家坡站得我两腿酸。下得坡来用目看,见一位大嫂把菜剜。前影好似王三姐,后影儿又像妻宝钏。本当上前把妻唤,错认了民妻礼不端。

\textbf{大嫂请了。(或:大嫂请来见礼。)}

\textbf{并非失迷路途,我乃找名问姓的。}

\textbf{提起此人,大大有名,就是那王丞相之女,薛平贵之妻,王氏宝钏。}

\textbf{非亲。}

\textbf{非故。}

\textbf{(大嫂有所不知,)我与她丈夫同营吃粮,与她带来万金家书,故而动问。}

\textbf{(我那薛大哥言道,书信么,要面交本人。)}

\textbf{(原书带回。)}

\textbf{请便。}

\textbf{这哑迷么,略知一二。}

\textbf{远在天边,不能相见。}

\textbf{哦,莫非你就是薛大嫂么?}

\textbf{哎呀呀,问来问去,问到本人的头上来了。}

\textbf{来来来,重见一礼呀。}

\textbf{礼多人不怪呀。}

\textbf{大嫂请稍待。}

\textbf{哎呀且住,想我平贵离家一十八载,不知她光景(到底)如何?}

\textbf{嗯,嗯,嗯\ldots{}\ldots{}我自有道理!}

\textbf{【西皮快板】洞宾曾把牡丹戏,庄子也曾戏过妻。秋胡戏过了罗敷女,薛平贵调戏自己妻。弓韔袋}\protect\hyperlink{fn408}{\textsuperscript{408}}\textbf{中摸一把,}

\textbf{哎呀!}

\textbf{【西皮快板】我把大嫂的书信失。}

\textbf{失落了。}

\textbf{弓韔袋中。}

\textbf{正是紧要的所在啊。}

\textbf{呃,呃,呃\ldots{}\ldots{}想是我前村抽弓打雁------}

\textbf{打雁充饥呀。}

\textbf{诶------一封书信,能值几何,你怎么开口伤人(或:出口伤人)呐?}

\textbf{哎呀呀,到底是丞相之女,出口便是文呐(或:出口成文)。}

\textbf{啊大嫂,你不要着急呀,这书信上的言语,呃,我还记得几句。}

\textbf{明白何来?}

\textbf{诶,(不是哟,)私看人家的书信是有罪名的呀。}

\textbf{呃,我那薛大哥修书的时节,我在一旁打点行李,我偷看了几句,倒是有的。}

\textbf{我若有心呐,还不失落你的书信呢。}

\textbf{呵呵哈哈哈\ldots{}\ldots{}(笑介)}

\textbf{【西皮导板】八月十五月光明呐,}

\textbf{军营中苦得很呐,哪有许多灯火。}\protect\hyperlink{fn409}{\textsuperscript{409}}

\textbf{【西皮原板】薛大哥在月下修书文呐。}

\textbf{(王宝钏 【西皮原板】我问他好来,)}

\textbf{【接西皮原板】他倒好,}

\textbf{(王宝钏 【西皮原板】再问他安宁,)}

\textbf{【接西皮原板】倒也安宁。}

\textbf{(王宝钏 【西皮原板】三餐茶饭,)}

\textbf{【接西皮原板】小军造,}

\textbf{(王宝钏 【西皮原板】衣服破了)}

\textbf{【接西皮原板】自己补缝。}

\textbf{【西皮原板】薛大哥这几年运不通,在西凉军营中受了酷刑}\protect\hyperlink{fn410}{\textsuperscript{410}}\textbf{。}

\textbf{呃,(不错)正是挨了打呀。}

\textbf{一捆四十。}

\textbf{大嫂不要啼哭,这苦哇------}

\textbf{还在后头呢。}

\textbf{【西皮原板}\protect\hyperlink{fn411}{\textsuperscript{411}}\textbf{】在营中失落了一骑马,}

\textbf{自然是官马呀。}

\textbf{哼,哪怕他不赔。}

\textbf{自然有哇------}

\textbf{【西皮原板】为赔马借了我十两纹银。}

\textbf{一份。}

\textbf{也是一份。}

\textbf{大嫂你有所不知呀,我那薛大哥啊,原先么,本是个好人呐。}

\textbf{后来他学坏了。交了些无业的游民,吃喝嫖赌(或:浪荡逍遥),呃,无所不为,把一份钱粮俱都花费。不怕大嫂你笑话,为军的我乃是个贫寒出身呐,从来不晓得什么叫作花钱(呐),积攒下几两银子,都借与他赔马了。}

\textbf{怎么不对呢?}

\textbf{哦,我那薛大哥也是个贫寒出身?}

\textbf{哎呀呀,薛大哥呀薛大哥,我今日才晓得你也是个贫寒出身呐。}

\textbf{呵呵哈哈哈\ldots{}\ldots{}(笑介)}

\textbf{【西皮原板】本利算来二十两,并不曾还我半毫分。}

\textbf{无有也是枉然。}

\textbf{(岂不伤了朋友的和气?)}

\textbf{防身宝剑,你问它则甚?}

\textbf{诶,清平世界(或:青天白日),朗朗乾坤,杀人(岂不)是要偿命的呀。}

\textbf{唉,有道是:善财难舍呀。}

\textbf{【西皮原板】那一日过营去将账讨,他言说长安城有一个王氏宝钏。}

\textbf{不该。}

\textbf{不欠。}

\textbf{大嫂,(我来问你,)有道是:这父债------}

\textbf{夫债呢?}

\textbf{妻,妻\ldots{}\ldots{}妻怎么样?}

\textbf{呵呵,你倒推得个干净呐。}

\textbf{(呃,)有道是:这汗呐,要出在病人的身上哦。}

\textbf{【西皮原板】他无钱便把妻来卖,将大嫂卖与了当军的人呐。}

\textbf{喏喏喏,就是在下。(或:不才,在下。)}

\textbf{呃呃呃,呃,我有婚书为证呐!}

呃呃,你慢来慢来,我看大嫂变脸变色,婚书诓至手中,三把五把扯碎,为军的岂不落一个人财两空么?

呃,你我去至前村,大户人家,请上三老四少,同拆同观。

当真。

哪个骗你呀?

\textbf{呵呵,她倒骂起来了。}

\textbf{(王宝钏 【西皮二六】\ldots{}\ldots{}主婚的人呐。)}

\textbf{【西皮快板】苏龙魏虎为媒证,那王丞相是我的主婚人呐。}

\textbf{【西皮快板】他三人与我有仇恨,咬定牙关就不认承。}

\textbf{【西皮快板】西凉川四十单八站,为军的要人不要钱。}

\textbf{【西皮快板】大嫂休得巧言辩,为军哪怕到官前?衙里衙外我打点,管保大嫂断与咱。}

\textbf{【西皮快板】好一个贞节王宝钏,百般调戏也枉然。(自古道青酒红人面,动人心,财帛金银钱。)(在)腰中取出银一锭,将银放在地平川。这锭银(子)三两三,送与大嫂作养奁(或:做妆奁)。买绫罗、做衣衫,打首饰、置簪环,做一个少年的夫妻就过几年呐。}

\textbf{【西皮快板】是烈女就该在闺房,缘何来在大路旁。为军的起下不良意,}

\textbf{【西皮摇板】来来来一马双跨到西凉呃。}

\textbf{走走走,上马呀。}

\textbf{在哪里?}

\textbf{诶呀!}

\textbf{呵呵哈哈哈\ldots{}\ldots{}(笑介)}

\textbf{【西皮摇板】一见宝钏回窑转,果然为我受熬煎(或:一十八载受熬煎)。不上马来步下赶,回到窑中两团圆。}

\textbf{{[}第二场{]}}

\textbf{【西皮摇板】后面跟随平贵男。}

\textbf{【西皮摇板】将为丈夫关至在(这)窑外边。}

\textbf{妻呀!}

\textbf{【西皮导板】想起当年泪不干呐,}

\textbf{【西皮原板】夫妻们在寒窑受尽了熬煎。自从我降了红鬃马,唐主爷驾前去讨官。官封我后军都督府哇,你的父上殿把本参。自从盘古【转西皮快板】立地天,哪有岳父把婿参。西凉国,造了反,薛平贵倒做了先行官。两军阵前遇代战,将我擒过了马雕鞍。多蒙老王不肯斩,反把公主配良缘。西凉的老王把驾晏,(众)文武保我坐银安。那一日驾坐银安殿,宾鸿大雁口吐人言。手执金弓银弹打,打下了半幅血罗衫。展开罗衫从头看,才知道寒窑受苦的王宝钏。不分昼夜往前趱,为的是回家夫妻团圆。三姐不信从头算,连来带呃去十八年。}

\textbf{【西皮摇板】水流千遭归大海,原物交还本人观。}

\textbf{【西皮摇板】少年子弟江湖老,红粉佳人两鬓斑。三姐不信菱花看,容颜不似当年在彩楼前。}

\textbf{水盆里面。}

\textbf{话已说明,开门相见才是。(或:话已说明,快快开门,夫妻相见。)}

\textbf{哦,退一步。}

\textbf{哦,又退了一步。(或:哦,再退后一步。)}

\textbf{唉呀妻呀,后面无有路了。}

\textbf{唉!}

\textbf{【西皮摇板】三姐不必寻短见,为丈夫跪在地平川。}

\textbf{(王宝钏 【西皮摇板】\ldots{}\ldots{}什么官。)}

\textbf{进得窑来,不问饥寒,开口便是官。难道说还吃官、穿官不成么?}

\textbf{(呃,)我临行的时节(或:临行之时),也曾(与你)留下安家度用啊。}

\textbf{十担干柴,八斗老米。}

\textbf{就该去借。}

\textbf{相府去借呀。}

\textbf{哦,你不曾去过相府(或:你不曾进过相府)?}

\textbf{呵(或:好),有志气。告便。}

\textbf{去至相府,与你那爹爹算这一十八载的老米账啊。}

\textbf{哦,他病了?得何病症呐(或:他得的什么病症呐)?}

\textbf{呵呵(或:哦),他见不得我?}

\textbf{难道说,我还见不得他?}

\textbf{有朝一日,孤王(或:我)得了唐室天下,他与我牵马坠镫,(呃,)我还嫌他老呢。}

\textbf{不曾睡着。}

\textbf{(句句实言。)}

\textbf{有道是:龙行有宝。}

\textbf{无宝呢?}

\textbf{三姐观宝。}

\textbf{【西皮快板】在头上整整沿毡帽,避尘珠金光照满窑。用手取出番王宝,三姐拿去仔细呀瞧。}

\textbf{下跪何人?}

\textbf{跪在孤王(或:我)的面前则甚呐?}

\textbf{(王宝钏 讨封。)}

\textbf{方才你在武家坡前骂得我好苦。(呃,)我是不能封的了啊。}

\textbf{若是知道,必然是不骂的了啊。(或:哦,倘若知道是我呢?)}

\textbf{呃,越发的不封了。}

\textbf{(当真不封。)}

\textbf{(果然不封。)}

\textbf{呃,慢来慢来,焉有不封之理。}

\textbf{三姐听封------}

\textbf{【西皮快板】非是孤不把你来封,有一个缘故在其中(或:在内中)。西凉国有个代------}

\textbf{【西皮快板】西凉国有一个(或:西凉川有个)代战女,她保孤王立大功。}

\textbf{(王宝钏 【西皮快板】\ldots{}\ldots{},她为正来我为偏。)}

\textbf{【西皮快板】讲什么正来论什么偏,你我结发在她前(或:在她先)。有朝一日登宝殿(或:登龙殿),封你昭阳掌正权。}

\textbf{【西皮摇板】平贵离家十八年。}

\textbf{(王宝钏 【西皮摇板】\ldots{}\ldots{}王宝钏。)}

\textbf{【西皮摇板】今日夫妻重相见。}

\textbf{(王宝钏 【西皮摇板】\ldots{}\ldots{}在梦间。)}

三姐,你看红日当头,不是做梦啊。

(不是做梦。)

三姐。

来了。

呵呵哈哈哈\ldots{}\ldots{}(笑介)

\newpage
\hypertarget{ux5927ux767bux6bbf-ux4e4b-ux859bux5e73ux8d35}{%
\subsection{大登殿 之
薛平贵}\label{ux5927ux767bux6bbf-ux4e4b-ux859bux5e73ux8d35}}

\textbf{(内)【西皮导板】长安城内把兵点,}

\textbf{【西皮原板】孤王得报旧仇冤。马达、江海把旨传,晓谕孤王驾坐长安。龙行虎步上金殿,薛平贵也有今一天}\protect\hyperlink{fn412}{\textsuperscript{412}}\textbf{。}

\textbf{【西皮原板】马达、江海一声唤,朝房中文武臣来把驾参。}

\textbf{【西皮摇板】你二人忠心把孤保,文武当朝一品官。}

\textbf{【西皮摇板】人来王允押上殿,}

\textbf{【西皮摇板】孤王言来听根源:先前设计将孤害,事到头来后悔难。人来推出午门斩,斩他的首级挂高竿。}

\textbf{【西皮摇板】他与魏虎将孤害,今日斩他报仇冤。}

\textbf{定斩不赦!}

\textbf{且慢!}

\textbf{【西皮摇板】梓童不必寻短见,午门外快赦王允还。}

\textbf{【西皮摇板】殿角赐你金交椅,事平之后再封官。}

\textbf{【西皮摇板】快将魏虎押上殿,}

\textbf{【西皮摇板】一见贼子怒冲冠。}

\textbf{【西皮摇板】马达、江海推出斩,}

\textbf{但凭于你。}

\textbf{【西皮摇板】杀魏虎方称孤心愿,}

\textbf{【西皮摇板】代战公主把驾参。}

\textbf{【西皮摇板】孤王金殿用目看,二梓童打扮似天仙。宝钏封在昭阳院,代战公主掌兵权。赐你二人龙凤剑,三人同掌锦长安。}

\textbf{【西皮摇板】宝钏相府忙回转,快请岳母上金銮。}

\textbf{【西皮摇板】孤王金殿出赦诏,晓谕天下众群僚:一赦钱粮并钱钞,二赦囚犯出监牢。}

\textbf{【西皮导板】二梓童搀岳母待王拜见。}

\textbf{【西皮二六】尊一声老岳母细听儿言:不幸我的亲娘亡故早,你比我亲娘甚是贤。薛平贵本是花郎汉,到如今驾坐在长安。宝钏封在昭阳院,代战公主掌兵权。将岳母奉至在养老院,你在那寿星宫内乐安然。彩女、宫娥常陪伴,一日三次王去问安。请请请,老岳母请至养老院。}

\textbf{【西皮摇板】快宣王允上金殿,}

\textbf{【西皮摇板】孤封你当朝太师在朝班,有职无权。}

\textbf{【西皮摇板】朝事已毕把班散,养老宫去问岳母安。}


\item
  \leavevmode\hypertarget{fn307}{}%
  陈超老师说明:谭鑫培特地给秦琼设计了很多江湖气的身段。\protect\hyperlink{fnref307}{↩}
\item
  \leavevmode\hypertarget{fn308}{}%
  段公平君建议此处从俗作``仨人''。\protect\hyperlink{fnref308}{↩}
\item
  \leavevmode\hypertarget{fn309}{}%
  一般作``我发财啦''或``我发了财'',此处从《京剧新序》。\protect\hyperlink{fnref309}{↩}
\item
  \leavevmode\hypertarget{fn310}{}%
  《京剧新序》此处原未注板式,据文意补。\protect\hyperlink{fnref310}{↩}
\item
  \leavevmode\hypertarget{fn311}{}%
  《论语·公冶长》:``子路曰:`愿车马衣轻裘,与朋友共,敝之而无憾。'''又《雍也》:``子曰:`赤(公西华)之适齐也,乘肥马,衣轻裘。''\protect\hyperlink{fnref311}{↩}
\item
  \leavevmode\hypertarget{fn312}{}%
  《京剧新序》中作``王伯党''。据史料载,王伯当是隋末济阳县(今河南兰考)人,起义领袖,瓦岗军名将。\protect\hyperlink{fnref312}{↩}
\item
  \leavevmode\hypertarget{fn313}{}%
  𥋌念``撒(sā)''音,是冀鲁方言看的意思。《京剧新序》原作``用目洒'',《京剧新序(修订版)》改作``用目𥋌'',此处从后者。\protect\hyperlink{fnref313}{↩}
\item
  \leavevmode\hypertarget{fn314}{}%
  王荣山此戏的耍锏有研究,现介绍\textbf{锏架子}如下:

  (台中间)\textbf{唱完}``\ldots{}\ldots{}耍一耍''\textbf{之后},转身面向里躬揖,左手抱锏,右手伸指切掌停在胸前,向里一步两步三步,右脚贴在左脚(/腿)后边,左腿单腿右转身面向大边台口外角(亮住),退三步到上场门里边,右手拉开山膀亮住。云手跨右腿踢左腿分锏,双手举锏亮。缓锏,正花转身、再正花转身到大边台口矗双锏,虚右脚亮。云手从台口过到小边矗双锏(台口左锏在右肘(/臂)后),虚左脚亮。缓锏,正花转身大走到下场门边,面小边台口举双锏亮。正花转身,横走正花、回花转身,回花搓步三倒手横推锏,在小边面向外,右上左下横锏,虚左脚亮。缓锏,正花转身(下场门边举双锏,站),横走正花、回花转身,回花搓步三倒手横(单腿亮)推锏,在大边面向里横锏,虚左脚亮。缓锏,正花、回花转身,横走到小边,回花双锏点地,双锏向后反画圈举锏,右脚贴左脚(/腿)后边单腿立亮。正花、转身缓锏,正花、回花转身,横走到大边,回花双锏点地,双锏向后反画圈举锏,右脚贴左脚(/腿)后边单腿立亮。正花撤步,回花转身,打右靴底、小蹦子打左靴底,回花在脸前起反云手转身到上场门里边拉开,在脸前起正云手转身向下场门外角。边走边向下方三扎。面斜向下场门外角,起三个大刀花,双锏点地,起三个揉(/回)花。正花蹦子转身、右锏打地。在台口中间反云手转身,踢右腿正面分锏亮住。

  \textbf{《当锏卖马》的锏架子是秦琼威吓王伯当和谢映登的,要大耍劈抡双锏,王荣山《卖马》锏架子受到内外行赞许,特别是收时在台口侧身回花,像车轮绕身一样,非常好看。}

  \textbf{陈超老师按:}刘曾复先生记录这套的锏套子为王荣山所传,是谭鑫培的演法。\protect\hyperlink{fnref314}{↩}
\item
  \leavevmode\hypertarget{fn315}{}%
  《京剧新序》误作``先生''。\protect\hyperlink{fnref315}{↩}
\item
  \leavevmode\hypertarget{fn316}{}%
  \textbf{与韩擒虎开打是二人都使枪}。

  \begin{quote}
  头场开城会阵,伍云召在城内上马介冲出城,在大边跟龙套走、到大边里边见面,一扯两扯,一合两合,幺二三,兜,勾韩走马腰封到大边,往里一盖,上下、打韩下。龙套追过场。伍耍大下场下。
  \end{quote}

  头场伍云召打韩擒虎下,缓枪上步在台中间向外站、右手拿枪斜垂,左手掏过去挥手,叫龙套追过场。捋胡子左转身面向上场门,跨左步,用枪扫右腿趋步,转身到上场门向外举枪亮,走,小绕到小边台口,两手斜托枪,走,到下场门方向,枪扎出去,伸左手、右手收回来斜着平举枪跺泥一亮,手、枪不动回走,绕到台中间向外涮枪转向里矗枪站住,耍三个背花,回身向外耍三个迎面花,面略斜向左,用枪打左脚,回身略斜向右,枪经左手背绕过左手反手接枪,由下往右往上往左绕一圈,与右手一块儿拿枪打右脚,枪从上往左小蹦子打左脚,跨右腿,右手向上场门出枪,向大边台口踢左腿,上步右手枪在脸前从左下向上往右再往左画大圈,左转身中枪交左手、在大边台口斜身面向外,左手拿枪,右手跟着一块儿在脸前从上往右再往左画大圈,弓箭步,左手平出枪,右手胸前按胡子亮住,大绕从下场门下。\protect\hyperlink{fnref316}{↩}
\item
  \leavevmode\hypertarget{fn317}{}%
  二场伍追韩上,原地漫头,用枪头别,一拉右转身一过合两过合,回身往里一裹,上下、打韩下。伍耍小下场(收兵)下。

  \begin{quote}
  二场打韩下,伍面向外提枪花转身,面向外三个提枪花,出枪,左手向左上方伸出平托枪,右手往右掠枪杆往下绕过枪鐏反手扶鐏,跨左腿,踢右腿,两手不离枪,向右往里翻身,面向下场门,左手在前扶住枪杆,右手顺着枪鐏转过来、握住枪下端在右侧腰间,一绕枪头,平端枪、弓箭步亮住下。
  \end{quote}

  \protect\hyperlink{fnref317}{↩}
\item
  \leavevmode\hypertarget{fn318}{}%
  唱完\textless{}\textbf{扫头}\textgreater{}双剜萝卜,架住,钻烟筒,一扯两扯,一合两合,接上下左右,往里一盖两盖,接蓬头,绕枪杆枪头斜向下、弓箭步败下。\protect\hyperlink{fnref318}{↩}
\item
  \leavevmode\hypertarget{fn319}{}%
  陈超老师介绍:伍云召见朱灿,\textless{}\textbf{水底鱼}\textgreater{}身段,富连成不下马,余叔岩、王凤卿下马,蹉步。谭富英不下马。\protect\hyperlink{fnref319}{↩}
\item
  \leavevmode\hypertarget{fn320}{}%
  《京剧汇编》第九集
  李万春藏本作``一朝错''。此处从李元皓君建议。\protect\hyperlink{fnref320}{↩}
\item
  \leavevmode\hypertarget{fn321}{}%
  《京剧汇编》第九集
  李万春藏本作``英雄四海''。\protect\hyperlink{fnref321}{↩}
\item
  \leavevmode\hypertarget{fn322}{}%
  秦琼见杨林后括号内的对白是陈超老师跟刘曾复先生学的。\protect\hyperlink{fnref322}{↩}
\item
  \leavevmode\hypertarget{fn323}{}%
  《打登州》是秦琼为骗杨林给他马骑好逃走,所以《打登州》只耍一点步锏,让杨林看不上,罗周才好替秦琼说马上武艺好,杨林才给秦琼马骑。该戏与《当锏卖马》戏情不同,锏的耍法各异。

  陈超老师介绍的秦琼耍步锏记录如下:

  一请,至小边里角拉开,分锏,缓锏,举锏一亮;一扫腿(外),俩扫腿(里);一扎锏、俩扎锏、三扎锏至大边台口,缓锏,横场大刀花;云手至小边台口,立锏虚步亮相。再到大边、台中如法炮制两遍。\protect\hyperlink{fnref323}{↩}
\item
  \leavevmode\hypertarget{fn324}{}%
  在传统话本小说中王伯当名勇,字伯当,以字行。\protect\hyperlink{fnref324}{↩}
\item
  \leavevmode\hypertarget{fn325}{}%
  段公平君注:金堤关,堤字正音为``敌(dí)'',艺人讹念为``提(tí)''。隋代关名,在今河南荥阳市广武镇霸王城村北黄河道中。以关置于汉代``金堤''而得名。\protect\hyperlink{fnref325}{↩}
\item
  \leavevmode\hypertarget{fn326}{}%
  进身,指被录用或提升。\protect\hyperlink{fnref326}{↩}
\item
  \leavevmode\hypertarget{fn327}{}%
  《京剧汇编》第九集
  苏连汉藏本作``猖狂''。\protect\hyperlink{fnref327}{↩}
\item
  \leavevmode\hypertarget{fn328}{}%
  段公平君注:\textbf{史载,秦琼死后追改封``胡国公'',疑后讹为``护国公''。}\protect\hyperlink{fnref328}{↩}
\item
  \leavevmode\hypertarget{fn329}{}%
  《京剧汇编》第九集
  苏连汉藏本作``跨下''。\protect\hyperlink{fnref329}{↩}
\item
  \leavevmode\hypertarget{fn330}{}%
  段公平君建议此处李世民唱,此处从《京剧汇编》第九集
  苏连汉藏本。\protect\hyperlink{fnref330}{↩}
\item
  \leavevmode\hypertarget{fn331}{}%
  《京剧汇编》第九集
  苏连汉藏本作``矗矗''。\protect\hyperlink{fnref331}{↩}
\item
  \leavevmode\hypertarget{fn332}{}%
  《京剧汇编》第九集
  苏连汉藏本作``我把弟一枪给你挑歪了''。\protect\hyperlink{fnref332}{↩}
\item
  \leavevmode\hypertarget{fn333}{}%
  \textbf{汛地是明、清时代称军队驻防地段。``汛''通``讯'',``讯地''即为军事烽火之地,以传消息,地界不大,故而汛地为基本驻防之地。}\protect\hyperlink{fnref333}{↩}
\item
  \leavevmode\hypertarget{fn334}{}%
  此处``残母''可能是``戏母''之误。\protect\hyperlink{fnref334}{↩}
\item
  \leavevmode\hypertarget{fn335}{}%
  段公平君指出,``路卧''可能是``路剐''之误。\protect\hyperlink{fnref335}{↩}
\item
  \leavevmode\hypertarget{fn336}{}%
  ``干国良臣''是旧时戏曲中常见词汇,``干国''是治理国家之意。干的本意是盾牌,引申为捍卫、保卫之意。``干国良臣''即``保国良臣''、``治国良臣''。此处``捍国''与``干国''同意。\protect\hyperlink{fnref336}{↩}
\item
  \leavevmode\hypertarget{fn337}{}%
  本剧本中标注人物台上位置主要由段公平君标注。刘曾复先生在为樊百乐君说戏时曾说明:《取帅印》是二十四本《龙门阵》的头一本。《龙门阵》是大老板程长庚排的。《白良关》也是《龙门阵》里的,但《凤凰山》、《独木关》、《汾河湾》等都不是《龙门阵》里的。\protect\hyperlink{fnref337}{↩}
\item
  \leavevmode\hypertarget{fn338}{}%
  夏行涛君建议此四句应为一人一句接唱,此处从《京剧汇编》第九集
  赵荣鹏藏本。\protect\hyperlink{fnref338}{↩}
\item
  \leavevmode\hypertarget{fn339}{}%
  刘曾复先生特别说明:程咬金对唐王念韵白,对其他臣宰念京白。\protect\hyperlink{fnref339}{↩}
\item
  \leavevmode\hypertarget{fn340}{}%
  ``埋怨'',《京剧汇编》第九集
  赵荣鹏藏本皆作``瞒怨''。\protect\hyperlink{fnref340}{↩}
\item
  \leavevmode\hypertarget{fn341}{}%
  ``荏苒''是逡巡、一刹那的意思。\protect\hyperlink{fnref341}{↩}
\item
  \leavevmode\hypertarget{fn342}{}%
  段公平君建议作``不识良言'';《京剧汇编》第九集
  赵荣鹏藏本作``不良之言教子孙''。\protect\hyperlink{fnref342}{↩}
\item
  \leavevmode\hypertarget{fn343}{}%
  《京剧汇编》第九集
  赵荣鹏藏本作``令出山摇动,严法鬼神惊''。\protect\hyperlink{fnref343}{↩}
\item
  \leavevmode\hypertarget{fn344}{}%
  《京剧汇编》第九集
  赵荣鹏藏本作``定后跟''。\protect\hyperlink{fnref344}{↩}
\item
  \leavevmode\hypertarget{fn345}{}%
  《京剧汇编》第九集 赵荣鹏藏本作``替主''\protect\hyperlink{fnref345}{↩}
\item
  \leavevmode\hypertarget{fn346}{}%
  段公平君建议作``一阵火气''。\protect\hyperlink{fnref346}{↩}
\item
  \leavevmode\hypertarget{fn347}{}%
  录音中``把名提''疑作``压名敌'',此处从《京剧汇编》第九集
  赵荣鹏藏本。\protect\hyperlink{fnref347}{↩}
\item
  \leavevmode\hypertarget{fn348}{}%
  段公平君建议作``告个罪''。\protect\hyperlink{fnref348}{↩}
\item
  \leavevmode\hypertarget{fn349}{}%
  夏行涛君建议作``即送'',此处从《京剧汇编》第九集
  赵荣鹏藏本。\protect\hyperlink{fnref349}{↩}
\item
  \leavevmode\hypertarget{fn350}{}%
  刘曾复先生介绍说,《马上缘》一剧中开始薛仁贵唱与此段基本相同,只是后两句为``我的儿出兵无音信,且听探马报信音。''\protect\hyperlink{fnref350}{↩}
\item
  \leavevmode\hypertarget{fn351}{}%
  陈超老师注:打虎的身段,老谭有两种演法。\protect\hyperlink{fnref351}{↩}
\item
  \leavevmode\hypertarget{fn352}{}%
  此处``站窑门''原来是``坐窑门''。\protect\hyperlink{fnref352}{↩}
\item
  \leavevmode\hypertarget{fn353}{}%
  吴小如先生认为``门庭''应该是``门桯'',``桯''是门槛的意思。\protect\hyperlink{fnref353}{↩}
\item
  \leavevmode\hypertarget{fn354}{}%
  据《秋声集》\textsuperscript{{[}15{]}.}载,程砚秋曾指出,陕西方言称有的窑洞为``坡洼寒窑'',即黄土坡下挖的窑洞,``坡洼''与``破瓦''谐音,艺人以讹传讹,久之就念成了``破瓦寒窑''。此处从俗。\protect\hyperlink{fnref354}{↩}
\item
  \leavevmode\hypertarget{fn355}{}%
  陈超老师注:老谭晚年《汾河湾》炉火纯青,舞台状态极其放松。王凤卿叙述过很多老谭晚年演此戏的即兴身段,如``马头军''身段很特别,左手捋髯,右手勾起食指指出去,等旦角重复完``马头军'',晃晃勾起的手指,犹如马头吃草,配合微点几下头,再念``马头军''。舞台效果极佳。\protect\hyperlink{fnref355}{↩}
\item
  \leavevmode\hypertarget{fn356}{}%
  ``砷黄铜''和``砷白铜''又分别被称为``药金''和``药银'',是古代方士为实现以铜``炼制''金、银时,用砷化物(包括雄黄、雌黄、砒霜等)``点化''铜而生成的产物(砷铜合金)。砷黄铜和砷白铜的区别主要是砷含量不同,铜中含砷小于~10\%,呈金黄色;当砷含量超过~10\%,则呈银白色。砷元素易挥发,所谓``真金不怕火炼'',就是因为高温下砷黄铜中的砷会遇热流失,恢复铜的本质。\protect\hyperlink{fnref356}{↩}
\item
  \leavevmode\hypertarget{fn357}{}%
  段公平君建议作``他不回来了''。\protect\hyperlink{fnref357}{↩}
\item
  \leavevmode\hypertarget{fn358}{}%
  此处刘曾复先生念作``法弟''。\protect\hyperlink{fnref358}{↩}
\item
  \leavevmode\hypertarget{fn359}{}%
  ``提龙笔''一段原来全部唱【二黄原板】,唱法刘曾复先生同样做了示范,词句如下:

  ``提龙笔王亲书大唐国号,命御弟唐三藏奉旨出朝。各国的众王子休挡禁道,到西天取经回替朕代劳。赐御弟锦袈裟霞光万道,孤赐你紫金钵、禅杖一条。孤赐你装经箱、毗卢僧帽,孤赐你四徒儿来把经挑。侍内臣与孤王将宝抬到,金銮殿王与你改换佛袍。''\protect\hyperlink{fnref359}{↩}
\item
  \leavevmode\hypertarget{fn360}{}%
  夏行涛君建议作``即日归''。\protect\hyperlink{fnref360}{↩}
\item
  \leavevmode\hypertarget{fn361}{}%
  本剧本中有关人物台上地方由段公平君协助整理。\protect\hyperlink{fnref361}{↩}
\item
  \leavevmode\hypertarget{fn362}{}%
  古人以梦中见熊罴为生男的征兆。后以``梦熊''作生男的颂语。语本《诗·小雅·斯干》:``吉梦维何?维熊维罴。''又:``大人占之,维熊维罴,男子之祥。''
  郑玄笺注:``熊罴在山,阳之祥也,故为生男。''\protect\hyperlink{fnref362}{↩}
\item
  \leavevmode\hypertarget{fn363}{}%
  段公平君注:若不带``\textbf{观阵}''一场,此处可按如下处理:

  \begin{quote}
  (樊梨花 仙师有何吩咐?)

  薛丁山
  仙师赐你丹药二料,一要戴之胸膛,二要用清水服下。\ldots{}\ldots{},料无妨碍。

  薛丁山 夫人请至后帐。

  (薛丁山、樊梨花下)
  \end{quote}

  \protect\hyperlink{fnref363}{↩}
\item
  \leavevmode\hypertarget{fn364}{}%
  段公平君建议此处作``你是听'',即``你在此听之意''。\protect\hyperlink{fnref364}{↩}
\item
  \leavevmode\hypertarget{fn365}{}%
  刘曾复先生在说该戏总讲时,除徐策外的人都说得比较简略,整理时个别地方参考了程君谋、蒋锡康录音唱片进行了增补;刘曾复先生在《京剧书文指伪录》\textsuperscript{{[}16{]}.}一文中介绍,徐策和夫人穿蓝帔,徐策戴员外巾。该戏有关场次调度也参照此文。\protect\hyperlink{fnref365}{↩}
\item
  \leavevmode\hypertarget{fn366}{}%
  刘曾复先生在为樊百乐君说戏时说明,\textless{}\textbf{六幺令}\textgreater{}行路时用;\textless{}\textbf{大锣六幺令}后段\textgreater{}留着张泰上场时用。\protect\hyperlink{fnref366}{↩}
\item
  \leavevmode\hypertarget{fn367}{}%
  七煞是紫微斗数中十四颗主星之一,七煞主``肃杀''。据说``面带七煞''的人往往寿命不长。\protect\hyperlink{fnref367}{↩}
\item
  \leavevmode\hypertarget{fn368}{}%
  此句程君谋、蒋锡康录音作``欸------相爷说哪里话来,想你我二老,年过半百,只生金斗孩儿,要他替旁人去死,万万不能!''\protect\hyperlink{fnref368}{↩}
\item
  \leavevmode\hypertarget{fn369}{}%
  据程君谋、蒋锡康唱片录音增补。\protect\hyperlink{fnref369}{↩}
\item
  \leavevmode\hypertarget{fn370}{}%
  据程君谋、蒋锡康唱片录音增补。\protect\hyperlink{fnref370}{↩}
\item
  \leavevmode\hypertarget{fn371}{}%
  此处程君谋、蒋锡康唱片录音作:

  薛猛 【二黄散板】夫人不必泪双淋,忠良哪怕丧残生。回头便对奸贼论:

  薛猛 贼!

  薛猛 【二黄散板】阴曹地府勾尔的魂。\protect\hyperlink{fnref371}{↩}
\item
  \leavevmode\hypertarget{fn372}{}%
  据程君谋、蒋锡康唱片录音增补。\protect\hyperlink{fnref372}{↩}
\item
  \leavevmode\hypertarget{fn373}{}%
  据程君谋、蒋锡康唱片录音增补。\protect\hyperlink{fnref373}{↩}
\item
  \leavevmode\hypertarget{fn374}{}%
  据程君谋、蒋锡康唱片录音增补。\protect\hyperlink{fnref374}{↩}
\item
  \leavevmode\hypertarget{fn375}{}%
  据程君谋、蒋锡康唱片录音增加。\protect\hyperlink{fnref375}{↩}
\item
  \leavevmode\hypertarget{fn376}{}%
  据程君谋、蒋锡康唱片录音增加。\protect\hyperlink{fnref376}{↩}
\item
  \leavevmode\hypertarget{fn377}{}%
  ``将养''即``抚养''之意。\protect\hyperlink{fnref377}{↩}
\item
  \leavevmode\hypertarget{fn378}{}%
  以下至结尾全部据程君谋、蒋锡康录音增补。\protect\hyperlink{fnref378}{↩}
\item
  \leavevmode\hypertarget{fn379}{}%
  刘曾复先生在樊百乐君说戏时说明,该戏徐策穿白开氅,戴相巾。\protect\hyperlink{fnref379}{↩}
\item
  \leavevmode\hypertarget{fn380}{}%
  段公平君建议作``腰铡三截''均作``腰铡三节''。\protect\hyperlink{fnref380}{↩}
\item
  \leavevmode\hypertarget{fn381}{}%
  根据剧中人物推测,此剧中的唐王应为唐代宗李豫。\protect\hyperlink{fnref381}{↩}
\item
  \leavevmode\hypertarget{fn382}{}%
  刘曾复先生说戏录音中作``父公侯、子王位'',此处从《京剧汇编》第三十二集
  邢威明藏本。\protect\hyperlink{fnref382}{↩}
\item
  \leavevmode\hypertarget{fn383}{}%
  《京剧汇编》第三十二集
  邢威明藏本作``共贺三多''。\protect\hyperlink{fnref383}{↩}
\item
  \leavevmode\hypertarget{fn384}{}%
  有些地方戏作``唐君蕊'',此处从《京剧汇编》第三十二集
  邢威明藏本。\protect\hyperlink{fnref384}{↩}
\item
  \leavevmode\hypertarget{fn385}{}%
  陈超老师介绍,唐王唱完此句后整冠捋髯。\protect\hyperlink{fnref385}{↩}
\item
  \leavevmode\hypertarget{fn386}{}%
  夏行涛君建议作``因何气''。\protect\hyperlink{fnref386}{↩}
\item
  \leavevmode\hypertarget{fn387}{}%
  古代习俗,生了男孩子,就在门的左首悬挂一张弓。因此用``悬弧之喜''指男性的生日。\protect\hyperlink{fnref387}{↩}
\item
  \leavevmode\hypertarget{fn388}{}%
  ``轻年纪'',即年纪尚轻之意。\protect\hyperlink{fnref388}{↩}
\item
  \leavevmode\hypertarget{fn389}{}%
  ``铜驼''即铜铸的骆驼,古代置于宫门外。借指京城、宫廷。同时也比喻朝代兴亡。\protect\hyperlink{fnref389}{↩}
\item
  \leavevmode\hypertarget{fn390}{}%
  据史料载,李克用之父李国昌,本名朱邪赤心,唐末沙陀部落首领,唐懿宗时因镇压庞勋起义之功,被赐名``李国昌''。``朱邪''姓亦作``朱耶'',艺人不识,误作朱姓;``国''字系入声字,此处保留湖北方言念法。\protect\hyperlink{fnref390}{↩}
\item
  \leavevmode\hypertarget{fn391}{}%
  据史料载,李克用别名``李鸦儿'',一目失明,其主力部队因穿黑衣服而以``鸦儿军''闻名。\protect\hyperlink{fnref391}{↩}
\item
  \leavevmode\hypertarget{fn392}{}%
  夏行涛君建议作``皓髯''。\protect\hyperlink{fnref392}{↩}
\item
  \leavevmode\hypertarget{fn393}{}%
  据《新五代史·唐本纪第四》载``黄巢已陷京师,中和元年,代北起军使陈景思发沙陀先所降者,与吐浑、安庆等万人赴京师,行至绛州,沙陀军乱,大掠而还。景思念沙陀非克用不可将,乃以诏书召克用于鞑靼,承制以为代州刺史、雁门以北行营节度使。率蕃汉万人出石岭关\ldots{}\ldots{}二年十一月,景思、克用复以步骑万七千赴京师。''戏中程敬思的原型即为陈景思。\protect\hyperlink{fnref393}{↩}
\item
  \leavevmode\hypertarget{fn394}{}%
  段公平君注:黄巢的字,于史无载。《残唐五代史演义》作``巨天'',此处从之。

  吴小如先生早年曾撰文\textsuperscript{{[}17{]}.}指出,此两句原作``家住曹州定陶县,姓黄名巢字霸天''。``并曹县''是``定陶县''的讹传;旧时艺人文化程度低,将``霸天''记作``垻天''字,以致讹成``具天''。

  \begin{quote}
  姜骏按:据《新唐书》载,黄巢是``山东曹州冤句人'',据史料推测,冤句在曹县、定陶一带(具体方位有争议)。因此``曹州并曹县''中``并''理解为衬字亦可。
  \end{quote}

  \protect\hyperlink{fnref394}{↩}
\item
  \leavevmode\hypertarget{fn395}{}%
  《残唐五代史演义》作``藏梅寺''。\protect\hyperlink{fnref395}{↩}
\item
  \leavevmode\hypertarget{fn396}{}%
  《京剧汇编》第四十六集
  郝寿臣藏本作``西岐'',此处从《残唐五代史演义》,下同。\protect\hyperlink{fnref396}{↩}
\item
  \leavevmode\hypertarget{fn397}{}%
  ``笑连天''亦可作``笑颜添''。\protect\hyperlink{fnref397}{↩}
\item
  \leavevmode\hypertarget{fn398}{}%
  此处``山摇震''或``山摇动''从俗,方与``胆战惊''对,下同。\protect\hyperlink{fnref398}{↩}
\item
  \leavevmode\hypertarget{fn399}{}%
  这是谭鑫培、钱金福在《珠帘寨》李克用与周德威对刀的``刀架子'',余叔岩演《珠帘寨》也打这一套刀架子:\textbf{这套刀架子},\textbf{特别是头子},\textbf{非常有内容},\textbf{一上来周德威显得很冲},\textbf{使人有李克用要招架不住之感},\textbf{但是几下之后李就轻巧地控制了周}。\textbf{这套把子层次分明},\textbf{很有戏}。\protect\hyperlink{fnref399}{↩}
\item
  \leavevmode\hypertarget{fn400}{}%
  《残唐五代史演义》作``鸦馆楼'',此处从《京剧新序》。\protect\hyperlink{fnref400}{↩}
\item
  \leavevmode\hypertarget{fn401}{}%
  刘曾复先生提供的钞本亦作``指望''。\protect\hyperlink{fnref401}{↩}
\item
  \leavevmode\hypertarget{fn402}{}%
  刘曾复先生提供的钞本作``改匹''。\protect\hyperlink{fnref402}{↩}
\item
  \leavevmode\hypertarget{fn403}{}%
  ``致气''一般作``置气'',下同。\protect\hyperlink{fnref403}{↩}
\item
  \leavevmode\hypertarget{fn404}{}%
  刘曾复先生提供的钞本作``全然不讲''。\protect\hyperlink{fnref404}{↩}
\item
  \leavevmode\hypertarget{fn405}{}%
  刘曾复先生提供的钞本作``千金体匹花郎''。\protect\hyperlink{fnref405}{↩}
\item
  \leavevmode\hypertarget{fn406}{}%
  刘曾复先生提供的钞本作``任儿取去''。\protect\hyperlink{fnref406}{↩}
\item
  \leavevmode\hypertarget{fn407}{}%
  吴焕老师整理的剧本(经刘曾复先生审订)注``汪派此处唱`一马离了三关界(或:三关境)'''。\protect\hyperlink{fnref407}{↩}
\item
  \leavevmode\hypertarget{fn408}{}%
  通常作``弓靫袋''。段公平君注:\textbf{韔(音chàng):弓袋,如《秦风·小戎》``虎韔镂膺'',``交韔二弓''。亦谓将弓放入弓袋,如《小雅·采绿》``之子于狩,言韔其弓''。}\protect\hyperlink{fnref408}{↩}
\item
  \leavevmode\hypertarget{fn409}{}%
  吴焕老师整理的剧本注:``谭派没有`皓月当空'的词句。''\protect\hyperlink{fnref409}{↩}
\item
  \leavevmode\hypertarget{fn410}{}%
  吴焕老师整理的剧本记作``受了苦刑''。\protect\hyperlink{fnref410}{↩}
\item
  \leavevmode\hypertarget{fn411}{}%
  吴焕老师整理的剧本注:``此句汪派唱【西皮快板】''。\protect\hyperlink{fnref411}{↩}
\item
  \leavevmode\hypertarget{fn412}{}%
  夏行涛君建议作``今日天''。\protect\hyperlink{fnref412}{↩}
