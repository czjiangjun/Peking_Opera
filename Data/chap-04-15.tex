\newpage
\subsubsection{\large\hei {宝莲灯~{\small 之}~刘彦昌}}
\addcontentsline{toc}{subsection}{\hei 宝莲灯~{\small 之}~刘彦昌}

\hangafter=1                   %2. 设置从第1⾏之后开始悬挂缩进  %}
\setlength{\parindent}{0pt}{

{\vspace{3pt}{\centerline{{[}{\hei 第一场}{]}}}\vspace{5pt}}

{({\akai 念})身授罗州正印,判断黎民冤情。}

{啊,你二人为何这等模样?哦,}想是在南学不用心攻书,被先生责打,回得家来,为父的也要打。

怎么讲?

唉呀!

\setlength{\hangindent}{56pt}{【{\akai 二黄散板}】听说是二奴才伤人命,}

{\textless{}{\!\bfseries\akai 三叫头}\!\textgreater{}沉香!秋儿!唉,儿------呃$\cdots{}\cdots{}$({\hwfs 哭介})}

\setlength{\hangindent}{56pt}{【{\akai 二黄散板}】头浇凉水呀怀抱冰{\footnotesize 呃}。上前忙把沉香问,南学之事说分明({\akai 或}:~一一从头说原因)。}

{这秦府官保是你们哪个打死的?}

{哦,是你打死的?}

{你可晓得打死人是要偿命{\footnotesize 呐}?}

{可舍得儿一双爹娘?}

{儿自己的性命?}

{呀呸!}

\setlength{\hangindent}{56pt}{【{\akai 二黄散板}】听一言来({\akai 或}:~听罢言来)怒气生,大胆奴才乱胡云。}

\setlength{\hangindent}{56pt}{【{\akai 二黄散板}】问过了沉香把秋儿问{\footnotesize 呐},一一从头说原因({\akai 或}:~南学之事说分明)。}

{秦府官保是你们哪个打死的?}

{哦,是你打死的?}

{你可晓得打死人是要偿命的呀?}

{可舍得儿一双爹娘?}

{儿自己的性命?}

{呀呸!}

\setlength{\hangindent}{56pt}{【{\akai 二黄散板}】秋儿说话太欺情,怎不教人动无名。}

\setlength{\hangindent}{56pt}{【{\akai 二黄散板}】两个奴才一齐问,到底哪个打伤人?}

{是你打死的?}

{是你打死的?}

{唉!看你这两个奴才不出,倒有些兄友弟爱之意,为父的倒想起一辈古人来了。}

{你二人两旁坐下,听为父的道来!}

{听道:~({\akai 念})伯夷叔齐二贤人,推位不肯掌朝廷。首阳山前埋名姓,留得美名万古存。}

\setlength{\hangindent}{56pt}{【{\akai 二黄三眼}】昔日里孤竹君身染重病,传口诏命嗣子继位为君({\akai 或}:~命次子}\footnote{段公平{\scriptsize 君}告知:~``次子''为王凤卿唱法,词句为:~ 

\setlength{\hangindent}{48pt}{【{\akai 二黄三眼}】昔日里孤竹君身染重病,传口诏命次子继位为君。有伯夷和叔齐兄友弟敬,推位不肯掌龙庭。  弟兄们出午门无有踪影,留下了美名儿万古传闻。}}继位为君)。有伯夷和叔齐兄友弟敬,推位不肯掌朝廷({\akai 或}:~不肯掌龙庭)。弟兄们出午门无有踪影,留下了美名儿万古传闻\footnote{``留下了美名儿万古传闻''一句,陈超老师从刘曾复先生学的是``到后来同饿死在首阳山林''。}。为父的怎比得孤竹君,你二人难比那两大贤人。到如今伤了那官保性命,怕只怕小奴才性命难生。\footnote{``到如今伤了那官保性命,怕只怕小奴才性命难生。''两句,陈超老师从刘曾复先生学的是``漫说是打死了官宝性命,就是那庶民人父难担承。''}{我的儿啊!}

\setlength{\hangindent}{56pt}{【{\akai 二黄原板}】我这里({\akai 或}:~我本当)带沉香秦府抵命,秦府抵{\footnotesize 啊}命,我的儿啊!三圣母叮咛我有言在心。}

\setlength{\hangindent}{56pt}{【{\akai 二黄原板}】我这里带秋儿秦府抵命,秦府抵{\footnotesize 啊}命,我的儿啊!后堂内哭坏了王氏夫人。翻来覆去无有计定,}

\setlength{\hangindent}{56pt}{【{\akai 二黄原板}】后堂内快请出儿的娘亲。}

{唉!我看你这两个奴才是怎生得了哇,呃$\cdots{}\cdots{}$({\hwfs 哭介})}

{夫人,夫人你来了。}

{你,你$\cdots{}\cdots{}$你来的好哇,呃$\cdots{}\cdots{}$({\hwfs 哭介})}

{想这两个小冤家,是你我所生,你我所养。要打就打,要骂就骂。说什么不听教训。}

{唉,不是的。}

{想下官身授罗州正印,上与天子办事,下与黎民分忧。(}也就够了。\footnote{此句从陈超老师建议加。}{)难道说还要升上天去不成么?}

{越发地不对了。}

{夫人,}

{你看这两个奴才是怎生得了哇,呃$\cdots{}\cdots{}$({\hwfs 哭介})}

{哎呀,夫人{\footnotesize 呐}!这才是一场祸事未了,又出了一场祸事。}

{夫人!谁想这两个奴才在南学攻书,将秦府官保打死了!}

{将秦府官保打死了!}

{夫人醒来!}

{夫人,}

{下官问过沉香,沉香言道:~秦府官保乃是他打死的。}

{唉!下官也曾问过秋儿,秋儿言道:~秦府官保乃是他打死的。}

{打死他一个儿子。}

{是啊。下官正为此事,在此为难得紧呐!}

{着啊!我想夫人乃丞相之女,喏喏喏,我状元之妻,胸中必有高才。来来来呀------现有家法在此,望夫人打一个、问一个,问一个、打一个。下官这里------拜托了。}

{(我)看你这两个奴才是怎生得了,呃$\cdots{}\cdots{}$({\hwfs 哭介})}

{啊,夫人你打的是哪一个?}

{着啊!少娘无母的孩儿,你就打死了罢!唉,儿啊$\cdots{}\cdots{}$({\hwfs 哭介})}

{啊,夫人,这就是你的不是了,先前责打沉香,如今就该责打秋儿;如今不打秋儿,先前就不该打沉香。看将起来,你这为娘的呀,就有这两般心肠。}

{\textless{}{\!\bfseries\akai 三叫头}\!\textgreater{}沉香!我儿!唉,儿啊$\cdots{}\cdots{}$({\hwfs 哭介})}

{呃,你的儿子在那厢呢。}

{啊,打迟了。}

{夫人可曾问个明白。}

{慢来慢来,夫人可曾问过秋儿么?}

{怎么样,}

{好一个``也将人打死''。}

{夫人,下官把你好有一比。}

{好比一盆糨糊------糊涂得紧呐。}

{哎呀夫人[\footnotesize 呐},想下官身为罗州正堂,上与天子办事,下与黎民分忧。想这黎民百姓,犯在我手,轻者是打,重者是夹。}

{这两个奴才闹出事来,教我打------打在哪个的身上;}

{教我夹------夹在哪个的腿上?}

{有道是:~清官难断家务事啊!}

{夫人好一张利口。}

{我不要你审,}

{不要你问。}

{\textless{}{\!\bfseries\akai 三叫头}\!\textgreater{}沉香!我儿!唉,儿啊$\cdots{}\cdots{}$({\hwfs 哭介})}

{呃,为了这两个奴才,不要伤了我二老的和气。}

{啊夫人,这里来,下官有个拙见在此:~待下官前去问沉香,夫人前去问秋儿,两下一对便知明白。请------}

{啊,何事?}

{呃怎么,夫人去问沉香? 请------}

{儿啊,秦府官保何人打死的?}

{打死人可要偿命呀?}

{(可)舍得儿一双爹娘?}

{儿自己的性命?}

{着哇!({\akai 念})好汉做事好汉当,岂肯连累二爹娘!}

{我这一下才明白了,我这一下才明白了。}

{夫人明白何来?}

{下官方才问过秋儿,秋儿言道:~秦府官保乃是秋儿打死的,他的哥哥站在一旁,连手都未动啊。}

{怎么不对呢?}

{哎呀,还是不得明白{\footnotesize 呐}。}

{夫人何事?}

{夫人有何高见?}

{呃,这也使得。请。}

{儿啊,秦府官保是何人打死的?}

{啊,夫人你这做什么?}

{夫人你看这上------}

{这下------}

{你我为父母的------}

{着啊,夫人,你把心要放明白些{\footnotesize 呀}。}

{请------}

{秦府官保到底是何人打死的?}

{呃------量你这个奴才也不敢{\footnotesize 呐}。}

{这一下我才明白了,我才明白了。}

{夫人明白何来?}

{呃,这就不对了。}

{下官问过沉香,沉香言道:~秦府官保------呵,是秋儿打死的,他站在一旁,都吓傻了。}

{依下官看来,一定是秋儿,}

{一定是秋儿。}

{呃------我想那秦府官保,一不是沉香,二不是秋儿,乃是我刘彦昌私自出衙,将人打死。}

{家院,带马------}

{去到秦府,与你的儿子偿命{\footnotesize 呐}。}

{哪里去?}

{(啊夫人言来语去,下官心中明白了{。}\footnote{此句从陈超老师建议加。})}

{唉!我想秦府官保,若是沉香打死,呃,就让沉香前去偿命。}

{若是秋儿么,唉,也教沉香前去偿命呐。}

{夫人你想呐,那秋儿在南学攻书,惹下塌天大祸,回得衙来,叫道一声:~父,有下官与他作主;叫道一声:~娘,有夫人与她担待。}

{我想沉香这个奴才在南学惹下塌天大祸,回得衙来,叫道一声:~父,下官眼睁睁不能与他作主;叫道一声:~娘,夫人,他的老娘,你是晓得的呀。}

{看将起来,还是教这少娘无母的孩儿前去偿命{\footnotesize 呃}。}

{\textless{}{\!\bfseries\akai 三叫头}\!\textgreater{}沉香!我儿!唉,儿啊$\cdots{}\cdots{}$({\hwfs 哭介})}

{呃,你的儿子在那厢呢。}

\setlength{\hangindent}{56pt}{【{\akai 二黄散板}】看起来还是儿抵命,自己亲生自己疼。}\footnote{{``看起来还是儿抵命,自己亲生自己疼。''两句,}陈超老师从刘曾复先生学的是``{看起来还是儿偿命},{自己亲生自己疼。}''}{带定娇儿出府门,}

\setlength{\hangindent}{56pt}{【{\akai 二黄散板}】去至秦府抵罪名({\akai 或}:~秦府去做抵命人)。}

{不提三圣母之事,还则罢了,提起三圣母之事,教下官好恨!}

{唉!焉敢恨着夫人。}

{恨只恨当初进京时节,路过硭砀山,被妖魔吞吃腹内,可也就是了。要什么三圣母,送的什么红灯。生下这个奴才,如今才有这场大祸!}

{大祸。}

{既是洪福,夫人,你那心中要放明白些呀。}

{夫人你明白何来?}

{若是沉香呢?}

{夫人,你要醒来讲话。}

{句句梦话。}

{我却不信。}

{我就跪$\cdots{}\cdots{}$}

{儿啊,你母亲放了你,来,快来叩首哇。}

{夫人,下官跪久了------}

{怎么样?}

\setlength{\hangindent}{56pt}{【{\akai 二黄散板}】多谢夫人开了恩。}

{秦府人役。}

{后花园中。}

{随我来呃------}

{去远了。}

{有话何不早讲?}

{儿啊,回来,你母亲还有话讲啊。}

{\textless{}{\!\bfseries\akai 三叫头}\!\textgreater{}沉香!我儿!唉,儿啊$\cdots{}\cdots{}$({\hwfs 哭介})}

\setlength{\hangindent}{56pt}{【{\akai 二黄散板}】未开言不由人({\akai 或}:~父子们在堂前)珠泪滚滚,到如今才说出以往原因}\footnote{{``到如今才说出以往原因''一句,}陈超老师从刘曾复先生学的是``{到如今才说出以往真情}''。}{:~那王桂英她不是儿的亲$\cdots{}\cdots{}$\textless{}\!{\bfseries\akai 行弦}\!\textgreater{}}

{丫鬟,夫人到上房去了,你要打茶伺候哇。}

\setlength{\hangindent}{56pt}{【{\footnotesize 接}{\akai 二黄散板}】亲生母,}

\setlength{\hangindent}{56pt}{【{\akai 二黄散板}】三圣母是儿的生身的娘亲。}

\setlength{\hangindent}{56pt}{【{\akai 二黄散板}】我儿若是不肯信,现有血书作证凭。}

{唉呀!}

\setlength{\hangindent}{56pt}{【{\akai 二黄散板}】罗州生来罗州养}\footnote{{``罗州生来罗州养''一句,}陈超老师从刘曾复先生学的是``{罗州生来罗州长}''。}{,哪个不认识({\akai 或}:~谁人不知)小沉香。}

{唉呀!}

\setlength{\hangindent}{56pt}{【{\akai 二黄散板}】忙将灰尘涂脸上({\akai 或}:~忙取灰尘涂脸上;{\akai 或}:~忙把灰尘涂脸上;{\akai 或}:~忙将灰尘罩脸上)。\textless{}{\!\bfseries\akai 扫头}\!\textgreater{}}

\setlength{\hangindent}{56pt}{【{\akai 二黄散板}】凭空降下无情剑,}

{沉香------}

{(免。)}

{孩儿啊$\cdots{}\cdots{}$({\hwfs 哭介})}

\setlength{\hangindent}{56pt}{【{\akai 二黄散板}】斩断(了)人间骨肉情。}

\setlength{\hangindent}{56pt}{【{\akai 二黄散板}】他母子只哭得如酒醉({\akai 或}:~他母子只哭得珠泪滚),铁石人闻也泪淋。狠心肠将儿忙带定({\akai 或}:~狠心肠将儿来带定),去至秦府抵罪名。}

{你可记得堂前盟誓?}

{呃,哪有戏言的道理,你快快放手。}

{你不放手,我就$\cdots{}\cdots{}$}

{唉呀!}

{\vspace{3pt}{\centerline{{[}{\hei 第二场}{]}}}\vspace{5pt}}

{儿啊,不要害怕,有为父的在此啊。}

{呔,有人么,走出一个来呀!}

{前去通禀,就说刘彦昌带子抵命来了!}

{呀呸!}

{请了,你乃告老太师,我是现任的官员,我跪你何来?}

{(住了!打死你一子,有一子与你偿命,你又岂奈我何?)}

{现在府外。}

{打死你一子,有一子与你抵命。你问的什么沉香,你管的什么秋儿?}

{老太师,}

{想我儿将官保打死,太师并不曾亲眼得见呐。如今当着下官的面前,将我儿打死,教我这为父母的呀,好不痛心呐!啊$\cdots{}\cdots{}$({\hwfs 哭介})}

{罢!}

}
