\newpage
\phantomsection %实现目录的正确跳转
\section*{\large\hei {乌盆记~{\small 之}~刘世昌}}
\addcontentsline{toc}{section}{\hei {乌盆记~{\small 之}~刘世昌}}

\hangafter=1                   %2. 设置从第1⾏之后开始悬挂缩进  %}}
\setlength{\parindent}{0pt}{

{\vspace{3pt}{\centerline{{[}{\hei 第一场}{]}}}\vspace{5pt}}

{刘升,带路------}

\setlength{\hangindent}{56pt}{【{\akai 西皮摇板}】一路美景观不尽,人宿旅店鸟宿林。}

{卑人刘世昌,南阳人氏,贩卖绸缎为生。只因我离家日久,尤恐双亲在家悬念,为此算清账目,带领家人刘升,回转家园,以奉}甘旨{。刘升,前面什么所在?}

{何县所管?}

{你看天色不好哇,我们急急趱行。}

\setlength{\hangindent}{56pt}{【{\akai 西皮原板}】叹人生世间名利牵,抛父母撇妻子}\footnote{刘曾复先生录音作``抛父母别妻子'';吴焕老师整理的剧本(经刘曾复先生审订)记作``别父母抛妻子'';此处从吴小如先生的建议修改。}{离故园。道旁美景懒得看,披星戴月奔家园。行程之间把天变,}

\setlength{\hangindent}{56pt}{【{\akai 西皮散板}】狂风大雨遮满天。}

\setlength{\hangindent}{56pt}{【{\akai 西皮散板}】刘升带路往前趱。}

\vspace{3pt}{\centerline{{[}{\hei 第二场}{]}}}\vspace{5pt}

\setlength{\hangindent}{56pt}{【{\akai 西皮散板}】大雨顷至风雷紧}\footnote{李元皓君建议作``风雷劲''。}{,浑身上下水淋淋。}

{看前面有一人家,上前借宿。}

{好话多讲。}

{诶------下站!({\akai 或}:~放肆,下站!)}\footnote{此处及以下括号中的词句据吴焕老师整理的剧本添加。}

{这位大哥,卑人这厢有礼。({\akai 或}:~兄台请了。)}

{我们是远方来的,行至此处,天降大雨,堪堪黄昏。前不着村,后不着店。万般无奈,借宿一宵。望求大哥({\akai 或}:~望求兄台),多行方便,明日早行,自当重谢。}

{哦,有劳了。多谢大哥。}

{哦,是是是。}

{哦,有劳了,有劳了。}

{哦,是是是。}

{哦,有座。}

{在下姓刘名世昌,南阳人氏,贩卖绸缎为生。}

{呃,小买卖。}

{小本钱呐。}

{啊,呵呵哈哈哈$\cdots{}\cdots{}$({\hwfs 笑介})}

{请问大哥上姓({\akai 或}:~请问兄台上姓)。}

{哦,原来是赵大哥。}

{做何生意?}

{哦,乃是大生意呀。}

{大本钱。啊,呵呵哈哈哈$\cdots{}\cdots{}$({\hwfs 笑介})}

{呃,前途用过,不必费心呐。}

{打搅不当了!({\akai 或}:~如此打搅,不当了!)}

\setlength{\hangindent}{56pt}{【{\akai 西皮原板}】好一位赵大哥人慷慨,顷刻间酒饭有安排。行至在中途呃大雨盖,萍水相逢理不该。到明天自当多谢拜,昏昏沉沉倒卧土台。}

\setlength{\hangindent}{56pt}{【{\akai 西皮导板}】霎时一阵呐肝肠断,}

{(唉呀,唉呀!)}

\setlength{\hangindent}{56pt}{【{\akai 西皮散板}】刀绞柔肠为哪般?}

\setlength{\hangindent}{56pt}{【{\akai 西皮散板}】回头忙把刘升唤呐,}

{刘升!刘升!}

{唉呀!}

\setlength{\hangindent}{56pt}{【{\akai 西皮散板}】奴才早已丧黄泉。}

\setlength{\hangindent}{56pt}{【{\akai 西皮散板}】是是是来明白了,中了赵大巧机关。眼望着南阳高声\textless{}\!{\bfseries\akai 哭头}\!\textgreater{}喊,爹娘啊,}

\setlength{\hangindent}{56pt}{【{\akai 西皮散板}】阴曹地府啊走一番。}

\vspace{3pt}{\centerline{{[}{\hei 第三场}{]}}}\vspace{5pt}

{参见判爷。}

{本当前去,奈无见证之人。}

{多谢判爷。}

\vspace{3pt}{\centerline{{[}{\hei 第四场}{]}}}\vspace{5pt}

{张别古!}

{老丈------}

\setlength{\hangindent}{56pt}{【{\akai 二黄原板}】老丈不必胆怕惊,我有言来你试听:~休把我当作了妖魔论,我本屈死一鬼魂。我忙将树枝摆摇动,}

\setlength{\hangindent}{56pt}{【{\akai 二黄原板}】抓一呀把沙土扬灰尘。}

\setlength{\hangindent}{56pt}{【{\akai 二黄原板}】我和你远无冤,近无有仇恨,望求老丈把冤申。}

\vspace{3pt}{\centerline{{[}{\hei 第五场}{]}}}\vspace{5pt}

{有。}

{张别古。}

{老丈啊,呃$\cdots{}\cdots{}$({\hwfs 哭}{\hwfs 介})}

\setlength{\hangindent}{56pt}{【{\akai 反二黄慢板}】未曾开言泪满腮,尊呃一声老丈细听开怀:~家住在南阳城关外,离城十里太平街。}

\setlength{\hangindent}{67pt}{【{\akai 反二黄慢板}】刘世昌祖居有数代,务农为本颇有家财。奉(母)命上京做买卖,贩卖绸缎倒生财。前三年也曾把货卖,算清账目转回家来。行至在定远县地界,忽然间老天爷降下雨来。路过赵大的窑门以外,借宿一宵惹祸灾。赵大夫妻将我谋害,把我的尸骨何曾葬埋。烧作了乌盆窑中卖,幸遇老丈讨债来。可怜我冤仇有三载,有三载,老丈呐!}

\setlength{\hangindent}{67pt}{【{\akai 反二黄原板}】因此上随老丈转回家来。}

\setlength{\hangindent}{67pt}{【{\akai 反二黄原板}】劈头盖脸洒下来,奇臭难闻口难开。可怜我哇命丧他乡以外,可怜我魂在望乡台。父母盼儿,儿不能奉拜}\footnote{吴焕老师整理的剧本记作``奉待''。}{;妻子盼夫,夫不能回来。望求老丈将我带,你带我去见包县台。倘若是把我的冤仇来解,但愿你福寿康宁永无灾。}

{正是。}

{(你告我诉。)}

{老丈多行方便。}

{方便方便罢。}

{我拿你头疼。}

{你告我诉就是。}

{多谢老丈。}

{有!}

{有!}

\vspace{3pt}{\centerline{{[}{\hei 第六场}{]}}}\vspace{5pt}

{啊老丈,这就是太爷的衙门。}

{是。}

{有。}

{正待进入,门神老爷阻拦。求太爷赏下一陌纸钱焚化,也好入内。}

{有劳了。}

{有。}

{本当进去,因念当初遇害之日({\akai 或}:~是我被害之时),被赵大夫妻将衣帽剥去,赤身露体。太爷日后有三公之位,尤恐冲撞,望求太爷赏下青衣一件,遮盖乌盆,方好进入。}

{老丈多行方便。}

{方便方便罢。}

{我拿你头疼。}

{有。}

{有。}

{有。}

{有。}

{有哇------}

{太爷容禀:~}

\setlength{\hangindent}{56pt}{【{\akai 西皮快板}】未曾开言泪汪汪,尊一声太爷听端详:~家住南阳太平庄,姓刘名安字世昌。贩卖绸缎把京上,算清账目啊转还乡。赵大夫妻图财害命、主仆双双命丧({\akai 或}:~}赵大夫妻图财害命把身丧{),望求太爷与我做主张啊。}

钟馗爷爷!
