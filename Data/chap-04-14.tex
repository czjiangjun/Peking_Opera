\newpage\hspace{30pt}~

{%

\subsubsection{\large\hei {铁莲花 之

刘子忠}}

{\vspace{3pt}{\centerline{{[}{\hei 第一场}{]}}}\vspace{5pt}}

{走哇------}\hspace{10pt}~

\setlength{\hangindent}{56pt}{【{\akai 二黄散板}】这大雪不住纷纷飘荡,不知弟妹投奔何方。将身来在草堂以上,}

\setlength{\hangindent}{56pt}{【{\akai 二黄散板}】只见定生倒卧雪旁。}

{儿啊,醒来!}

{哎呀儿啊,这样的大雪寒天,儿不在学中攻书,为何躺在雪地呀?}

{哦?为伯的不信呐。}

{唉呀!}\hspace{20pt}~

\setlength{\hangindent}{56pt}{【{\akai 二黄散板}】儿前世念了断头经呐,今生遇着狠心的人。}

{儿啊,你的衣帽往哪里去了?}

{这个奴才今在何处?}

{好奴才!}\hspace{10pt}~

{啊?你为何穿他的衣帽啊?}

{还不与我脱将下来!}

{你姑母往哪里去了?}

{教她与我滚了出来。}

{回来。}

{不用。}

{滚回来!}\hspace{10pt}~

{也不用!}\hspace{10pt}~

{我把你这个贱人!}

{老夫不在家中,你将定生暴打一顿,打在前厅扫雪。雪内扫出干土便罢,如若不然,你要将他活活地打死。你,你$\cdots{}\cdots{}$你好狠毒心肠啊,呃$\cdots{}\cdots{}$({\hwfs 哭}{\hwfs 介})}

{你不曾打他?}

{他这身上的伤痕是哪里来的?}

{这$\cdots{}\cdots{}$}

{诶。}\hspace{30pt}~

{哎呀儿啊,这样的大雪寒天,儿还放的什么风筝啊?}

{你这是怎么样了?}

{他饿了哇。}

{有什么现成的东西与他拿来。}

{哦,快快取来。}

{儿啊,还有哇。}

{啊?!}\hspace{30pt}~

\setlength{\hangindent}{56pt}{【{\akai 二黄散板}】小冤家你不与我把气争呐,为何将碗摔埃尘。舍不得打来娇生子啊,马氏一旁发恨声。}

{罢!}\hspace{30pt}~

\setlength{\hangindent}{56pt}{【{\akai 二黄散板}】狠着心肠将儿打。}

\setlength{\hangindent}{56pt}{【{\akai 二黄散板}】我若不打小定生,你道老夫两般心。家中的事儿交与你,是好是歹去找寻。}

{\vspace{3pt}{\centerline{{[}{\hei 第二场}{]}}}\vspace{5pt}}

{儿啊,慢走------}

\setlength{\hangindent}{56pt}{【{\akai 二黄散板}】雪大地滑路难行,不知娇儿何方存。}

\setlength{\hangindent}{56pt}{【{\akai 二黄散板}】顺着足迹往前进。}

{\vspace{3pt}{\centerline{{[}{\hei 第三场}{]}}}\vspace{5pt}}

{儿啊,醒来!}

{哎呀儿啊,伯父不曾打儿,儿怎么倒跑了哇,呃$\cdots{}\cdots{}$({\hwfs 哭}{\hwfs 介})}

{伯父的越发的不信了。}

{好贱人呐!}

\setlength{\hangindent}{56pt}{【{\akai 二黄散板}】咬牙切齿贱人恨,苦苦害他为何情?}

{儿啊,随我回去,有我在家中,量他们也不敢呐。随我回去。}

{哦------儿两足疼痛,难以行走?}

{也罢!}\hspace{20pt}~

{待为伯的背儿一程就是。}

{这个$\cdots{}\cdots{}$}

{唉,伯父虽老,还背得儿动。}

{那旁有一石台,儿只管地站了上去。}

\setlength{\hangindent}{56pt}{【{\akai 二黄散板}】数九寒天风不稳,雪大地滑路不平。山风冽冽难扎挣,}

{哦,你爹爹------}

{在哪里?}\hspace{10pt}~

{唉!兄弟呀$\cdots{}\cdots{}$({\hwfs 哭}{\hwfs 介})}

\setlength{\hangindent}{56pt}{【{\akai 二黄散板}】只见贤弟面前迎。}\footnote{以下(至``{儿啊,你还是站了上去,为伯的背儿一程。''前})表演内容据陈超老师告知后添加。}

{(}刘子忠跪,定生随跪{)}

{(【{\akai 二黄散板}】}一见贤弟泪双淋,可叹你替我丧残生。望贤弟在阴曹慢慢相等。\textless{}\!{\bfseries\akai 扫头}\!\textgreater{}{)}

{(}魂子三挥袖,下,刘子忠坐地,定生扶起{)}

{(}定生\hspace{20pt}~

伯父,我爹爹去了。{)}

{(}刘子忠先望,再找介,叹介{)}

{儿啊,你还是站了上去,为伯的背儿一程。}

{哦,儿走得动了?}

{随为伯的走啊!}

{\vspace{3pt}{\centerline{{[}{\hei 第四场}{]}}}\vspace{5pt}}

{(}回来。{)}

{(}不用。{)}

{(转来。)}\hspace{10pt}~

{(不用。)}\footnote{以上念白据陈超老师告知后添加。}

{呀呸!我把你这个贱人!你一计不成,又生二计,将面碗烧得通红,你将他一双小手都烫烂了哇,呃$\cdots{}\cdots{}$({\hwfs 哭}{\hwfs 介})}

{你又不曾。}

{你来看。}\hspace{10pt}~

{这是怎么样了?}

{呀呸!}\hspace{20pt}~

{呀呸,老夫的名字也是你这个奴才叫得的吗?}

{官罢怎么讲,私罢怎么讲?}

{哦,原来是你这个奴才,要承受老夫的家财么?}

{哼!我不要了,我就打死你这个奴才!}

{你是怎样知道的?}

{你有何主意?}

{好!}\hspace{30pt}~

{我把你这个贱人,自到我刘氏门中,你不生不养,要你何用?}

{宝珠}\footnote{``宝珠''也可作``宝柱'',此处从《戏考》第七册。}{过来!}

{架起干柴与我烧呃!}

{呀------呸!}

{从今以后,我父子在楼上,你二人在楼下。早晚茶饭来早便罢,如若来迟,我就打死你这个贱人!}

{儿啊,随我来呀!}
