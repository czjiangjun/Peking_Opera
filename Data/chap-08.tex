\addcontentsline{toc}{section}{\hfill[\hei 神怪·降妖]\hfill}
\newpage
\chead{神怪·降妖} % 页眉中间位置内容
\hypertarget{ux9752ux77f3ux5c71-ux4e4b-ux5415ux7956ux5173ux5e1d}{%
\subsection{青石山 之
吕祖、关帝}\label{ux9752ux77f3ux5c71-ux4e4b-ux5415ux7956ux5173ux5e1d}}

{[}第一场{]}

吕祖 (内)【二黄导板】赴蟠桃辞王母离了仙境,

【回龙】迈步儿出洞府散淡精神。

【二黄原板】青是山(或:青的山)绿是水一派美景,有苍松和翠柏密密层层。莲池内鱼戏水甚是沉静(或:看来沉静),有糜鹿衔灵草倒也安宁。尘世里好一似蓬莱仙境(或:胜过了蓬莱仙境),看此处正好作养性修身。

吕祖 抬头观看。

吕祖 为何这等模样?

吕祖 你的法力呢?

吕祖 不消。 (或:罢了。)

吕祖 好大的妖气。

吕祖 此乃九尾玄狐。

吕祖 清香一枝,法鼓三通。

吕祖 善哉呀善哉。

吕祖
【二黄原板】稳坐在法坛上三光照定,提羊毫写牒文上达天庭。都只为青石山妖狐狂狞,害得那小周生不得安宁。望神圣发慈悲神兵遣定,灭却了这妖狐(或:这妖魔)黎民太平。

吕祖 一祭,天清;二祭,地靖;三祭,百宁。

吕祖 值日功曹何在?

吕祖 牒文一道,烦劳尊神(或:有劳尊神)南天门送达。

{[}第二场{]}

关帝 【唢呐二黄导板】想当年破黄巾威风缭绕,

关帝 【回龙】扶保我大兄王锦绣龙朝。

关帝
【唢呐二黄原板】都只为吕法师牒文来到,因此上统神将下了九霄。\protect\hyperlink{fn629}{\textsuperscript{629}}

关帝 吕法师牒文到来。

关帝 众神将!

关帝 驾起祥云者。

{[}第三场{]}

吕祖 尊神请了!

关帝 法师请了!相召有何见谕?

吕祖 今有青石山妖狐作乱,还望尊神收服(或:降伏者)。

关帝 法师稳坐坛台,看吾神降妖者。

吕祖 圣寿无疆。

关帝 众神将!

(众 有!)

关帝 撒下天罗地网。

(众 啊!)

关帝 关平、仓将,前去降妖者。

{[}第四场{]}

吕祖 且住。九尾玄狐十分狂狞(或:妖狐十分狂狞),天灵灵,地灵灵。

吕祖 周元丰\protect\hyperlink{fn630}{\textsuperscript{630}}何在?

\newpage
\hypertarget{ux4e94ux82b1ux6d1e-ux4e4b-ux5929ux5e08}{%
\subsection{五花洞 之
天师}\label{ux4e94ux82b1ux6d1e-ux4e4b-ux5929ux5e08}}

\textbf{{[}第一场{]}}

\textbf{开船!}

\textbf{【唢呐二黄导板】龙虎山学道法身登仙界,}

\textbf{【回龙】圣天子传旨意召我前来。}

\textbf{【唢呐二黄原板】远望着白茫茫一片大海,灭却了众妖魔方称胸怀。}

{[}第二场{]}

\textbf{(念)自幼学道龙虎山,江西一带乐安然。钦赐天师黄金印,天罡、地煞任我传。}

\textbf{吾乃------护国天师是也。}

\textbf{圣天子相召。}

\textbf{众法官:降落帆桅,缓缓而行。}

\textbf{催舟。}

\textbf{请了。}

\textbf{上差有和见谕?}

\textbf{上差先行,贫道随后。}

\textbf{众法官:将船湾在江边,随贫道拜见包大人去者。}

{[}第三场{]}

\textbf{贫道来得鲁莽,大人海函。}

\textbf{相召有何见谕?}

\textbf{带上贫道一观------}

\textbf{此乃二真二假。}

\textbf{且慢,逃脱妖魔那还了得?}

\textbf{法官何在?}

\textbf{撒下天罗地网!}

\textbf{请大人升堂。}

\textbf{(包拯 【二黄散板】这都是深山内千年精怪,)}

\textbf{妖魔大胆!}

\textbf{【二黄散板】包大人休吃惊贫道前来。施一礼坐至在大堂以外,若不现真原形五雷降灾。}

{[}第四场{]}

\textbf{唗!}

\textbf{胆大妖魔,该当何罪?}

\textbf{五雷殛之!}

\textbf{贫道前去见驾。}

\textbf{大人不必惊慌。}

\textbf{贫道一言,大人听了------}

\textbf{【}西皮快板\textbf{】大人休得心内惊,贫道言来听分明:哪怕那五花洞群魔作乱,全凭着神五雷、七星剑、混元盒、八宝如意针。辞别大人上龙庭,}

\textbf{【}西皮摇板\textbf{】五花洞内镇妖氛。}

\item
  \leavevmode\hypertarget{fn629}{}%
  据朱家溍先生的演出录像\textsuperscript{{[}26{]}.},此处不念的那四句为``蚕眉凤目美髯飘,手中青龙偃月刀。胯下赤兔千里马,一片丹心保汉朝。''

  《京剧汇编》第五十二集
  李万春藏本中此四句为``蚕眉凤目美髯飘,红光照耀偃月刀。胯下赤兔追风马,一片忠心保天曹。''其中``胯下''作``跨下''。\protect\hyperlink{fnref629}{↩}
\item
  \leavevmode\hypertarget{fn630}{}%
  按《三国演义》,周仓无字。《关羽戏集:李洪春演出本》\textsuperscript{{[}27{]}}中载,周仓字元福;此处从《京剧汇编》第五十二集
  李万春藏本。\protect\hyperlink{fnref630}{↩}
