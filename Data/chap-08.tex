\addcontentsline{toc}{section}{\hfill[\hei 唱腔·唱段]\hfill}
\newpage
\chead{唱腔·唱段} % 页眉中间位置内容
\addcontentsline{toc}{subsection}{\hei 硃痕记~{\small 之}~朱春登}
\subsubsection{{\hei\large 硃痕记~{\normalsize 之}~朱春登}$^\ast$}%*

{[}第一场{]}

【二黄散板】听说是老娘黄泉命染,好一似刀割肉箭把心穿。问婶娘她婆媳何处埋掩。叫李仁备祭品\protect\hyperlink{fn631}{\textsuperscript{631}}坟前祭奠,我只得身穿孝头戴麻冠。

{[}第二场{]}

【二黄导板】见坟台不由人泪流满面,

【回龙】尊一声去世娘细听儿言:

【反二黄慢板】都只为西域国黄龙造反,是孩儿替叔父去到军前。抖威风杀贼寇全凭神箭,灭黄龙平西域得胜回还。王封儿平西侯官高爵显,奉圣命回家来祭奠祖先。实指望母子们欢聚团圆,料不想儿的娘命染黄泉。哭老娘把儿的肝肠痛断,肝肠痛断,儿的娘啊,

【反二黄原板】食什么爵禄做的是什么官。哭罢了老娘亲把妻房呼唤,叫一声贤德妻你在哪边。我和你夫妻情恩爱不浅,撇下我独一人凄凉孤单。哭一声贤德妻难得相见,难得相见,

【反二黄散板】要相逢除非是梦里团圆。

【西皮三眼】听我妻赵锦棠言讲一遍,好一似刀割肉箭把心攒。婶娘道她婆媳黄泉命染,为什么她还在阳世之间。莫不是她死得苦冤魂不散,莫不是魍魉鬼来把我缠。我这里出席棚用目观看,观只见那红日未落西山。猛想起赵锦棠左手心硃砂红点,是不是真和假向前去细问根源。

\textless{}\textbf{哭头}\textgreater{}啊,我的妻呀!

【西皮散板】问贤妻老娘亲何方避难。

【西皮散板】有劳你前引路把母来见,

【西皮散板】儿就是朱春登做官回还。

\newpage
\textbf{蟠桃会}\protect\hyperlink{fn632}{\textsuperscript{632}}
\textbf{之 吕洞宾}*

\textbf{{[}第一场{]}}

\textbf{【西皮原板】忆昔当年赴科场,科场中提笔做文章。文章幸喜龙颜赏,赏赐我进士伴君王。陪王伴君心不想,一心只想上天堂。天堂就在瑶池上}\protect\hyperlink{fn633}{\textsuperscript{633}}\textbf{,瑶池以上福寿绵长。}

{[}第二场{]}

\textbf{【西皮散板】离了洞府到仙界,见了众仙说开怀。}

\textbf{【西皮散板】瑶池以上寿筵席开。}

\textbf{【西皮原板】今日里饮酒多爽快,好似仙子(或:好一似黄粱)赴瑶台。这仙女(或:这仙子)生得呀多娇态,眉清目秀送情来。趁此佳兴(或:趁此酒兴)破了戒,}

\textbf{【西皮原板】众仙道我理不该。将身且坐(或:将身来在)瑶池外,昏昏沉沉睡石台。}

{[}第三场{]}

\textbf{【西皮导板】沉醉东风}\protect\hyperlink{fn634}{\textsuperscript{634}}\textbf{月儿高,}

\textbf{【西皮原板】忆昔当初饮酕醄。两足徘徊任颠倒,湘子、仙姑发笑嘲。你道我当真吃醉了,任意随心乐逍遥。游戏三昧多奥妙,}

\textbf{【西皮快板】坎离}\protect\hyperlink{fn635}{\textsuperscript{635}}\textbf{二字}本相调。\textbf{不觉来到东海道,海水接天浪滔滔。}

\textbf{【西皮散板】宝剑扔在东海道,你看我醉仙家的道法高不高?}

\textbf{【西皮散板】柳仙带路东海道,万丈波涛走一遭。}

\textbf{【西皮散板】湘子说话不中听,丢了宝贝你问旁人。}

\textbf{【西皮散板】洞宾主意拿得稳,从今后不管闲事情。}

(李铁拐 【西皮散板】\ldots{}\ldots{}\textbf{入东海道,)}

\textbf{【西皮散板】从今后戒酒最为高。}

\newpage
\textbf{霸王别姬·山头 之 韩信}*

\textbf{【西皮散板】李左车引霸王入了阵道,众诸侯齐奋勇争立功劳。直杀得血成河尸如山倒,灭西楚擒霸王就在今朝。}

\textbf{【西皮散板】传一令犹如那泰山压倒,兵将涌如那海水临潮。楚项羽犹如那无翼之鸟,失彭城犹如那猛虎离巢。}

\textbf{【西皮散板】直杀得楚项羽人喊马叫,直杀得子弟兵四路奔逃。直杀得天昏暗日无光耀,直杀得夜更深月挂松梢。}

\textbf{(项羽 【西皮散板】越杀越勇心焦躁,)}

\textbf{【西皮散板】三军带马回营道,请出张良作计较。}

\newpage
\textbf{逍遥津 之 汉献帝}*

【二黄导板】苦汉帝在后宫伤心难忍,

【回龙】父子们悲切切好不伤情,贤御妻呀。

【二黄原板】叹伏后此时间必定丧命,我君妃生离散惨不忍闻。二皇儿年幼小孩童天性,哭啼啼与孤王要他的娘亲。想奸贼不由孤咬牙愤恨,上欺寡人下压群臣。欺寡人贼带剑上殿孤见他不敢责问,欺寡人贼独霸朝纲、目无君王、自专自尊。欺寡人孤只得百般谨慎,欺寡人孤只得时刻留心。欺寡人贼奏本是非曲直孤不敢争论,欺寡人孤有命贼大胆妄为抗旨不遵。欺寡人贼一意孤行孤不敢过问,欺寡人孤怒不敢言、忍耐在心。欺寡人孤见他气色不正吓得孤乱了方寸,欺寡人孤见他带怒发威吓得孤胆战心惊。欺寡人蹂躏百般、万分难忍,欺寡人贼败坏朝纲、逆了五伦。欺寡人好一似【转二黄慢板】奴仆受训,欺寡人好一似虐待家人。欺寡人好一似无辜良民被贼围困,欺寡人好一似冤屈罪犯无处冤申。欺寡人好一似蛇毒蝎狠,欺寡人好一似虎咽狼吞。欺寡人好一似前世冤孽今生报应,欺寡人好一似狭路相逢对头仇人。欺寡人好一似阎君索命,欺寡人好一似饿鬼孤魂。欺寡人好一似败阵残兵无投奔,反被贼困垓心难逃遁难存身,坐以待毙谁来救应,

【二黄散板】又听得一片喧哗声震乾坤。

附注:

《顺天时报》曾刊《逍遥津任辰辙之唱词》一文,所载词句与刘曾复教授所传词句非常相近,照录供参考(张斯琦君提供)

《逍遥津》``任辰''辙之唱词

旧本《逍遥津》,``欺寡人''一段,俱用``由求''辙,戏中汉献帝唱``欺寡人好一比鹰抓兔胁''句,过于俚俗,殊伤大雅。刘鸿升未故时,将``由求''改``任辰'',虽亦不免俗,但较旧本,似觉雅驯。李桂芬唱《逍遥津》,亦用斯词,爰将改词录于左端:

汉献帝在后宫伤心难忍,可叹我父子们悲切切冷清清、求生不得求死不能、好不惨情。叹伏后到此时难保活命,我君妃生离散惨不忍闻。二皇儿年幼小孩童之性,哭啼啼与孤王要他的娘亲。想奸贼不由孤嚼牙愤恨,上欺天子下压群臣。欺寡人(贼)带剑上殿孤见他不敢责问,欺寡人(贼)独霸朝纲、目无君、自耑自尊。欺寡人孤只得百般谨慎,欺寡人孤只得时刻留神。欺寡人(贼)奏本是非曲直孤不敢□□,欺寡人孤有命贼大胆妄为抗旨不遵。欺寡人贼自由行孤不敢过问,欺寡人孤怒不敢言、忍耐在心。欺寡人孤见他气色不正嚇得孤乱了方寸,欺寡人孤见他怒发威嚇得吊胆提心。欺寡人蹂躏百般、惨忍万分,欺寡人贼败坏纲常、逆了五伦。欺寡人好一似主仆受训,欺寡人好一似虐待家人。欺寡人好一似无辜良民被贼围困,欺寡人好一似冤屈罪犯□而受刑。欺寡人好一似蛇毒蝎狠,欺寡人好一似虎狼把孤吞。欺寡人好一似前世冤孽今生报应,欺寡人好一似夹路相逢对头仇人。欺寡人好一似阎王索命,欺寡人好一似饿鬼勾魂。欺寡人好一似败阵惨兵无投奔,反被贼困垓心、难逃命、难生存、认贼斩、恁贼擒,孤做一待毙谁来救应,又听得宫门外喧哗之声。

\item
  \leavevmode\hypertarget{fn631}{}%
  吴焕老师整理的剧本作``备祭礼''。\protect\hyperlink{fnref631}{↩}
\item
  \leavevmode\hypertarget{fn632}{}%
  此戏别名《海屋添筹》或《八仙庆寿》,是一出武旦戏。\protect\hyperlink{fnref632}{↩}
\item
  \leavevmode\hypertarget{fn633}{}%
  此句吴小如先生从刘曾复先生学的是``天堂远在瑶池上''。\protect\hyperlink{fnref633}{↩}
\item
  \leavevmode\hypertarget{fn634}{}%
  樊剑君建议作``沉醉洞府''。\protect\hyperlink{fnref634}{↩}
\item
  \leavevmode\hypertarget{fn635}{}%
  八卦中``坎''为水,``离''为火。此处寓为``水火既济''之意。\protect\hyperlink{fnref635}{↩}
