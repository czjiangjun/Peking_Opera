\newpage
\phantomsection %实现目录的正确跳转
\section*{\large\hei {文昭关~{\small 之}~伍员}}
\addcontentsline{toc}{section}{\hei 文昭关~{\small 之}~伍员}

\hangafter=1                   %2. 设置从第1⾏之后开始悬挂缩进  %
\setlength{\parindent}{0pt}{
	{\centerline{{[}\large 汪派{]}}}

\vspace{3pt}{\centerline{{[}{\hei 第一场}{]}}}\vspace{5pt}

{({\akai 内})马来!}

\setlength{\hangindent}{52pt} {【{\akai 西皮散板}】伍员马上怒气冲,逃出龙潭虎穴中。

{俺,伍员!幸喜逃出樊城,意欲往吴国借兵报仇。行至此处,四面俱是高山峻岭,不知哪条道路,可通吴国。}

{看那旁有一老丈,待我下马问来。}

{啊,老丈请了。}

{啊,俺乃行路之人,老丈休得错认。}

{愚下正是伍员。老丈何以知晓?}

{原来如此。}

{唉!俺有满腹含冤,意欲往吴国借兵报仇。行至此处,四面俱是高山峻岭,不知哪条道路可通吴国~?}

{可有别路?}

{哎呀,不、不、不,不好了!}

\setlength{\hangindent}{56pt}{【{\akai 西皮散板}】听说吴国路不通{\footnotesize 呃},好似狼牙箭穿胸。心猿意马终何{\textless{}\!{\bfseries\akai 哭头}\!\textgreater{}}用,爹娘啊~!

\setlength{\hangindent}{56pt}{【{\akai 西皮散板}】血海冤仇落场空。

{萍水相逢,怎好打搅。}

{这就不敢。}

{有座。}

{请问老丈尊姓大名。}

{哦,原来是前辈老先生,失敬了。}

{唉!一言难尽呐!}

\setlength{\hangindent}{56pt}{【{\akai 西皮原板}】恨平王无道乱楚宫,父纳子妻礼难容\footnote{段公平{\scriptsize 君}建议作``理难容'',亦通。}。我的父谏奏反把命送,满门家眷血染红。

{若得如此,感恩匪浅。}

{({\akai 念})愧煞男儿不丈夫。}

{惭愧!}

{\vspace{3pt}{\centerline{{[}{\hei 第二场}{]}}}\vspace{5pt}}

{哎!}

\setlength{\hangindent}{56pt}{【{\akai 西皮快板}】过了一天又一天,心中好似滚油煎。腰中枉挂三尺剑,不能报却父母冤。}

{俺伍员多蒙东皋公搭救,将我隐藏后花园中寻计出关。一连七日未见计出,思想起来好不焦虑人也~!}

{唉,爹娘啊!({\hwfs 哭介})}

\setlength{\hangindent}{56pt}{【{\akai 二黄慢板}】一轮明月照窗前,愁人心中似箭穿。实指望奔吴国借兵回转,又谁知昭关又有阻拦。幸遇东皋公行方便,他将我隐藏在后花园。一连七天我的眉不展,夜夜何曾又安眠。俺伍员好一似丧家犬,满腹含冤向谁言。我好比哀哀长空雁,我好比龙游在浅沙滩。我好比鱼儿吞了钩线,我好比波浪中失舵的舟船呐。思来想去我的肝肠断,今夜未过又盼明天。}

\setlength{\hangindent}{56pt}{【{\akai 二黄原板}】心中有事难合眼,翻来覆去睡不安。背地里只把东皋公怨,教人难解巧机关。你若是真心来救我,为何七日不周全。贪图着({\akai 或}: 贪图这)富贵将我害,你就该将我献与昭关。哭一声爹娘不能够见面,难得{\textless{}\!{\bfseries\akai 哭头}\!\textgreater{}}见,爹娘啊~!}

\setlength{\hangindent}{56pt}{【{\akai 二黄原板}】要相逢{\footnotesize 呃}除非是梦里团圆。}

\setlength{\hangindent}{66pt}{【{\akai 二黄快原板}】鸡鸣犬吠五更天,越思越想越伤惨。想起在朝为官宦,是朝臣待漏五更寒。到如今夜宿荒村馆,我冷冷清清向谁言呐?我本当拔宝剑自寻短见,父母的冤仇化灰烟。对天发下宏誓愿: 我不杀平王我的心怎甘~?}

\setlength{\hangindent}{56pt}{【{\akai 二黄散板}】适才朦胧将合眼, }

\setlength{\hangindent}{56pt}{【{\akai 二黄散板}】耳旁又听有人言。用手开门拔宝剑, }

\setlength{\hangindent}{52pt}{(东皋公\hspace{20pt}【{\akai 二黄散板}】$\cdots{}\cdots{}$因何白了髯~?) }

我却不信。

待我看来。

哎呀!不、不、不好了。

\setlength{\hangindent}{56pt}{【{\akai 二黄散板}】一见须白心好惨,点点珠泪洒胸前。冤仇未报容颜变,一事无成两鬓斑。但愿过得昭关险,满斗焚香谢上天。 }

{\vspace{3pt}{\centerline{{[}{\hei 第三场}{]}}}\vspace{5pt}}

({\akai 念})父母冤仇恨,常怀一片心。

老丈何事?

待我相见。

皇甫兄在哪里?皇甫兄在$\cdots{}\cdots{}$

请坐。

穷途末路,犹如丧家之犬。仁兄誉言({\akai 或}: 年兄誉言),惭愧~!

皇甫兄到此,有何妙计救我出关?

事不宜迟,就此改扮({\akai 或}: 装扮)起来。

\setlength{\hangindent}{66pt}{【{\akai 西皮快二六}】伍员在头上换儒巾,乔装改扮往东行。临潼会,曾举鼎,我在万马营中显异能。时来双挂盟府印,运退深山草不生({\akai 或}: 运退时衰夜宿在荒村)。多亏了东皋公行恻隐,请来了历阳山\footnote{ 历阳山一名历山,在历阳县(今安徽和县)西北四十里。}前皇甫官人。我三人同把巧计定,皇甫官人假扮俺伍员去闯关门。({\akai 或}: 思想起教人恨不恨,也是我的五行八字命生成。) }

(【{\akai 西皮快板}】回头来再对东皋公论: 你是我伍员活命的恩人。但愿过得昭关境,一重恩报九重恩。)\footnote{刘曾复先生为吴小如先生说戏时先唱的这几句。}

\setlength{\hangindent}{56pt}{【{\akai 西皮摇板}】皇甫兄请上受一礼, }

\setlength{\hangindent}{56pt}{【{\akai 西皮快板}】多谢你施下这全恩。焚香顶礼不为敬,来生犬马当报恩。 }

\setlength{\hangindent}{56pt}{【{\akai 西皮摇板}】东皋公请上礼恭敬, }

\setlength{\hangindent}{56pt}{【{\akai 西皮快板}】你是我伍员的活命恩人。但愿过得昭关境,一重恩报九重恩。伍员心中千般恨, }

\setlength{\hangindent}{56pt}{【{\akai 西皮摇板}】大胆且向虎山行。 }

{\centerline{{[}\large 谭派{]}\protect\footnote{谭派的唱法整理时参考了刘曾复先生为吴小如先生说戏及在其他场合说戏录音。}}}

{\vspace{3pt}{\centerline{{[}{\hei 第一场}{]}}}\vspace{5pt}}

{({\akai 内})马来!}

\setlength{\hangindent}{56pt}{【{\akai 西皮散板}】勒马停蹄威风勇\footnote{ 吴焕老师整理本记作``威风涌''。},只见道旁一老翁。

{俺,伍员!幸喜逃出樊城,来至此地,不知哪条道路可通吴国。}

{看那旁有一老丈,待我下马问来。}

{老丈请了。}

{啊,老丈------俺乃行路之人,老丈休得错认。}

{俺正是伍员。老丈何以知晓?}

{原来如此。}

{唉!我有满腹含冤,意欲到吴国借兵报仇。行至此地,四面俱是高山峻岭,不知哪条道路可通吴国~?}

{可有别路?}

{哎呀,不、不、不,不好了!}

\setlength{\hangindent}{56pt}{【{\akai 西皮散板}】听说吴国路不通,好似狼牙箭穿胸。心猿意马终何{\textless{}\!{\bfseries\akai 哭头}\!\textgreater{}}用,爹娘啊{!}

\setlength{\hangindent}{56pt}{【{\akai 西皮散板}】血海冤仇落场空。

{萍水相逢,怎好打搅。}

{这就不敢。}

{有座。}

{请问老丈尊姓大名。}

{呜哙呀,原来是前辈老先生,失敬了。}

{唉!一言难尽呐!}

\setlength{\hangindent}{56pt}{【{\akai 西皮原板}】恨平王无道乱楚宫,父纳子妻礼难容。我的父谏奏反把命送,满门家眷血染红。

{若得如此,感恩非浅。}

{({\akai 念})愧煞男儿不丈夫。}

{惭愧!}

{\vspace{3pt}{\centerline{{[}{\hei 第二场}{]}}}\vspace{5pt}}

{唉!}

\setlength{\hangindent}{56pt}{【{\akai 西皮快板}】过了一天又一天,心中好似滚油煎。腰中枉挂三尺剑,不能报却父母冤。

{俺伍员多蒙东皋公搭救,定计救我出关。一连七日未见计出,思想起来好不焦虑人也!}

{唉,爹娘啊!({\hwfs 哭介})}

\setlength{\hangindent}{56pt}{【{\akai 二黄慢板}】一轮明月照窗前,愁人心中似箭攒。想当年在朝为官宦,朝臣待漏五更寒。恨平王无道纲常乱,父纳了子的妻({\akai 或}: 子的媳)礼不端。我父谏奏反遭斩,一家满门被刀残。单人匹马({\akai 或}: 匹马单人)弃楚樊,行至在昭关又有阻拦。到如今独宿在荒村馆({\akai 或}: 踟蹰在荒村馆;宿至在荒村馆),冷冷清清向谁言。(我好比哀哀长空雁,我好比龙游在浅沙滩。我好比扑灯蛾身罹大难,我好比平阳虎离了深山。我好比鱼儿吞了钩线,我好比波浪中失舵的舟船。)思来想去我的肝肠断,今夜未过又盼明天。

\setlength{\hangindent}{56pt}{【{\akai 二黄原板}】心中有事难合眼,翻来覆去睡不安。背地里只把东皋公怨,教人难解巧机关。你若是真心来救我,为何七日不周全。贪图着({\akai 或}: 贪图这)富贵将我害,你就该将我献与昭关。哭一声爹娘不能够见面,难得{\textless{}\!{\bfseries\akai 哭头}\!\textgreater{}}见,爹娘啊!

\setlength{\hangindent}{56pt}{【{\akai 二黄原板}】要相逢除非是梦里团圆。

\setlength{\hangindent}{56pt}{【{\akai 二黄原板}】鸡不鸣犬不吠月淡星稀,孤雁飞惊动了杜鹃鸟啼。黑暗暗背地里祝告天地,二爹娘在天灵细听端倪: 保佑儿早到吴国地,借大兵杀平王灭却了费无极({\akai 或}: 灭却那费无极)。那时节方消儿心中恶气,大鹏展翅任空飞。叹爹娘叹得儿咽喉哽泣,咽喉哽泣,

\setlength{\hangindent}{56pt}{【{\akai 二黄散板}】今夜晚寂寞更长偏遇着不鸣金鸡。 }

\setlength{\hangindent}{56pt}{【{\akai 二黄导板}】适才朦胧将合眼, }

\setlength{\hangindent}{56pt}{【{\akai 二黄散板}】耳旁又听有人言。用手开门拔宝剑, }

我却不信。

待我看来。

哎呀!不、不、不好了。

\setlength{\hangindent}{56pt}{【{\akai 二黄散板}】一见须白心好惨,点点珠泪洒胸前。冤仇未报容颜变,一事无成两鬓斑。 }

喜从何来?

若得如此感恩匪浅。

老丈请上受我一拜。

\setlength{\hangindent}{56pt}{【{\akai 二黄散板}】但愿过得昭关险,借得吴兵报仇冤。 }

{\vspace{3pt}{\centerline{{[}{\hei 第三场}{]}}}\vspace{5pt}}

({\akai 念})父母冤仇恨,常怀一片心。

老丈何事?

待我相见。

皇甫兄在哪里?皇甫兄在$\cdots{}\cdots{}$

皇甫兄。

请坐。

穷途末路,犹如丧家之犬。仁兄({\akai 或}: 年兄)誉言,唉,惭愧!

啊,老丈,皇甫兄到此,有何妙计救我出关?

此计甚好,事不宜迟,就此改扮起来。

\setlength{\hangindent}{66pt}{【{\akai 西皮快二六}】伍员在头上换儒巾,乔装改扮往东行。临潼会曾举鼎,我在万马营中显异能。时来双挂盟府印,运退深山草不生。多亏了东皋公心恻隐,请来了历阳山前({\akai 或}: 历阳山下)皇甫官人。我三人同把巧计定,皇甫官人假扮俺伍员去闯关门。 }

\setlength{\hangindent}{56pt}{【{\akai 西皮摇板}】皇甫兄请上受一礼, }

\setlength{\hangindent}{56pt}{【{\akai 西皮快板}】多谢你施下这全恩。焚香顶礼不为敬,来生犬马当报恩。 }

\setlength{\hangindent}{56pt}{【{\akai 西皮摇板}】东皋公请上礼恭敬, }

\setlength{\hangindent}{56pt}{【{\akai 西皮快板}】你是我伍员的活命恩人。但愿过得昭关境,一重恩报九重恩。伍员心中千般恨, }

\setlength{\hangindent}{56pt}{【{\akai 西皮摇板}】大胆且向虎山行。 }

且喜逃出昭关,我不免吴国去者。}
