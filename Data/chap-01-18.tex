\newpage
\subsubsection{\large\hei 盗宗卷}
\addcontentsline{toc}{subsection}{\hei 盗宗卷}

\hangafter=1                   %2. 设置从第1⾏之后开始悬挂缩进  %
\setlength{\parindent}{0pt}{

{\centerline{{[}{\hei 第一场}{]}}}\vspace{5pt}

\setlength{\hangindent}{56pt}{({\bfseries\akai 西皮}\textless{}\!{\bfseries\akai 小开门}\!\textgreater{},{\hwfs 四}太监、大太监{\hwfs 引}吕后{\hwfs 上})}

\setlength{\hangindent}{52pt}{吕后\hspace{30pt}({\akai 念}){[}{\akai 引子}{]}身在深宫院,重整汉室锦江山。({\hwfs 大座},{\akai 念})未央宫中翡翠环,珍珠玛瑙堆成山。上殿独受君王宠,方显女皇将魁元。哀家吕后。昨日飞报禀告,有一游方道人,来到长安,盗取宗卷。不免将宗卷用火焚化,以消后患。内侍(侍{\hwfs 应}),宣张苍上殿。}

\setlength{\hangindent}{56pt}{内侍\hspace{30pt}国太有旨,张苍上殿呐!}

\setlength{\hangindent}{56pt}{张苍\hspace{30pt}({\akai 内白})领旨。}

\setlength{\hangindent}{56pt}{({\bfseries\akai 小锣}{\hwfs 上},{\hwfs 小边台口})}

\setlength{\hangindent}{56pt}{张苍\hspace{30pt}({\akai 念})忽听国太宣,迈步上金銮。({\hwfs 进门},{\hwfs 参拜})臣张苍见驾,国太千岁。}

\setlength{\hangindent}{56pt}{吕后\hspace{30pt}平身。}

\setlength{\hangindent}{56pt}{张苍\hspace{30pt}千千岁。({\hwfs 小边举笏站})}

\setlength{\hangindent}{56pt}{张苍\hspace{30pt}宣臣上殿有何(国)事议论?}

\setlength{\hangindent}{56pt}{吕后\hspace{30pt}卿家官居何职?}

\setlength{\hangindent}{56pt}{张苍\hspace{30pt}西台御史。}

\setlength{\hangindent}{56pt}{吕后\hspace{30pt}掌管何事{[}{\akai 或}:宗卷可是卿家掌管(看守){]}?}

\setlength{\hangindent}{56pt}{张苍\hspace{30pt}皇家宗卷{[}{\akai 或}:正是微臣掌管(看守){]}。({\akai 或}:~皇王宗卷。)}

\setlength{\hangindent}{56pt}{吕后\hspace{30pt}将宗卷呈上哀家一观。}

\setlength{\hangindent}{56pt}{张苍\hspace{30pt}领旨。(张苍{\hwfs 上场门下})}

\setlength{\hangindent}{56pt}{吕后\hspace{30pt}内侍,张苍取卷到来看我眼色行事。}

\setlength{\hangindent}{56pt}{张苍\hspace{30pt}({\hwfs 捧卷上})手捧皇家卷,国太凤目观。({\hwfs 进门}张苍{\hwfs 交卷给}内侍,侍{\hwfs 呈}吕后,张{\hwfs 站小边})}

\setlength{\hangindent}{56pt}{(张苍\hspace{30pt}国太请看。)}

\setlength{\hangindent}{56pt}{吕后\hspace{30pt}卿家看守宗卷有功,赐御酒一斗。}

\setlength{\hangindent}{56pt}{(内侍{\hwfs 递酒给}张苍,张{\hwfs 站小边台口})}

\setlength{\hangindent}{56pt}{(张苍\hspace{30pt}谢国太。)}

\setlength{\hangindent}{56pt}{张苍\hspace{30pt}且住,人言吕后之酒好饮难还,待我谢过神祗。({\akai 或}:~人言吕后之酒实实难饮,待我敬谢天地。)}

\setlength{\hangindent}{56pt}{吕后\hspace{30pt}想这太平年间要这宗卷何用?内侍,将宗卷用火焚化了。}

\setlength{\hangindent}{56pt}{(张苍{\hwfs 将酒谢神时}吕{\hwfs 念},张{\hwfs 还酒杯},太监{\hwfs 烧卷},张{\hwfs 站小边}{\akai 念}``哎呀''{\hwfs 时右手投袖},{\hwfs 摇头惊介})}

\setlength{\hangindent}{56pt}{张苍\hspace{30pt}哎呀!({\akai 或}:~使不得!)({\hwfs 右袖盖头},{\hwfs 左手撩袍},{\hwfs 上场门反下},{\hwfs 同时},陈平{\hwfs 上场门外侧上},\textless{}\!{\bfseries\akai 撞金钟}\!\textgreater{}{\hwfs 到小边台口})}

\setlength{\hangindent}{56pt}{陈平\hspace{30pt}【{\akai 西皮摇板}】金銮殿上红光现,陈平焉能袖手观。({\hwfs 进门},{\hwfs 参拜})}

\setlength{\hangindent}{56pt}{陈平\hspace{30pt}臣陈平见驾,国太千岁。}

\setlength{\hangindent}{56pt}{吕后\hspace{30pt}平身。}

\setlength{\hangindent}{56pt}{陈平\hspace{30pt}千千岁! ({\hwfs 大边站举笏})}

\setlength{\hangindent}{56pt}{吕后\hspace{30pt}卿家上殿有何本奏?}

\setlength{\hangindent}{56pt}{陈平\hspace{30pt}适臣观见金銮殿上红光一起,特地上殿保全江山。}

\setlength{\hangindent}{56pt}{吕后\hspace{30pt}卿家可知哀家火焚宗卷之故?}

\setlength{\hangindent}{56pt}{陈平\hspace{30pt}国太敢是想吞$\cdots{}\cdots{}$({\akai 或}:~国太莫非要吞$\cdots{}\cdots{}$)}

\setlength{\hangindent}{56pt}{吕后\hspace{30pt}卿家果有出将入相之才。}

\setlength{\hangindent}{56pt}{陈平\hspace{30pt}谢国太。}

\setlength{\hangindent}{56pt}{吕后\hspace{30pt}退班。}

\setlength{\hangindent}{56pt}{(陈平、吕后{\hwfs 搭话紧快},吕众{\hwfs 窝下},陈{\hwfs 揖归中间},\textless{}\!{\bfseries\akai 撞金钟}\!\textgreater{})}

\setlength{\hangindent}{56pt}{陈平\hspace{30pt}【{\akai 西皮摇板}】金钟三响王退殿,文武有怒不敢言。撩袍端带下金銮,({\hwfs 小圆场下殿归大边},张苍{\hwfs 上场门上},{\hwfs 站小边})只见张苍在眼前。宗卷本在孝廉殿,不该拿来献君前。}

\setlength{\hangindent}{56pt}{张苍\hspace{30pt}国太要看呐!}

\setlength{\hangindent}{56pt}{陈平\hspace{30pt}呀呸!({\akai 接唱})淮河十王人马到({\akai 或}:~淮河十王发人马),你一家大小难保全。(陈平{\hwfs 下},\textless{}\!{\bfseries\akai 撞金钟}\!\textgreater{})}

\setlength{\hangindent}{56pt}{张苍\hspace{30pt}【{\akai 西皮摇板}】陈平老儿礼不端,骂得我张苍不敢言。宗卷本在孝廉殿,谁知国太备火燃。淮河十王人马到,我一家大小难保全。(张苍{\hwfs 下})}

\vspace{3pt}{\centerline{{[}{\hei 第二场}{]}}}\vspace{5pt}

\setlength{\hangindent}{56pt}{(\textless{}\!{\bfseries\akai 小锣抽头}\!\textgreater{},田子春{\hwfs 上})}

\setlength{\hangindent}{56pt}{田子春\hspace{20pt}【{\akai 西皮摇板}】道家模样改装扮,不分昼夜往长安。({\akai 念})下官田子春,奉了幼主之命,去至长安盗取皇王宗卷,就此前往。}

\setlength{\hangindent}{56pt}{田子春\hspace{20pt}({\akai 接唱})昔日楚汉两争强,韩信弃楚投汉邦。未央宫中把命丧,为国忠良无下场。(田子春{\hwfs 下})}

\vspace{3pt}{\centerline{{[}{\hei 第三场}{]}}}\vspace{5pt}

\setlength{\hangindent}{56pt}{(\textless{}\!{\bfseries\akai 六幺令}\!\textgreater{}陈平、家院、{\hwfs 四}青袍{\hwfs 上}。陈{\hwfs 下轿},{\hwfs 牌子}\textless{}\!{\bfseries\akai 合龙}\!\textgreater{},陈{\hwfs 换衣},青袍{\hwfs 下},陈{\hwfs 正小座},院{\hwfs 大边站})}

\setlength{\hangindent}{52pt}{陈平\hspace{30pt}老夫陈平。今日早朝国太将宗卷焚化,莫非淮河有人前来盗卷,我不免八卦详查,来,香案伺候。({\akai 或}:~老夫陈平。适才国太在金銮殿上将宗卷焚化,不知是何缘故。不免在八卦之中查看吉凶便了。家院,香案摆下。)}

\setlength{\hangindent}{56pt}{({\hwfs 台前方设桌椅},\textless{}\!{\bfseries\akai 小开门}\!\textgreater{},陈{\hwfs 入座},{\hwfs 摇盒},{\hwfs 开看},{\hwfs 用笔记},{\hwfs 连三次},{\hwfs 看})}

\setlength{\hangindent}{52pt}{陈平\hspace{30pt}十字口中吞,了字加一横。三人共一日,凑成田子春。(田子春$\cdots{}\cdots{}$)嗯,昔日老王({\akai 或}:~昔日先王)驾下有一臣子名唤田子春,此人虽则年幼颇有胆识,(现在淮河。)莫非此人前来盗卷不成?(我自有道理,)香案撤去。}

\setlength{\hangindent}{56pt}{(\textless{}\!{\bfseries\akai 小开门}\!\textgreater{}{\hwfs 撤香案},陈{\hwfs 入小座},院{\hwfs 大边})}

\setlength{\hangindent}{56pt}{陈平\hspace{30pt}唤夜不收进见。({\akai 或}:~来,传夜不收\footnote{``夜不收'',本是明代辽东边防守军中的哨探或间谍的特有称谓,相当于侦察兵。}进见。)}

\setlength{\hangindent}{56pt}{家院\hspace{30pt}夜不收进见。}

\setlength{\hangindent}{56pt}{({\hwfs 二}夜不收{\hwfs 上},{\hwfs 一}人{\hwfs 提灯})}

\setlength{\hangindent}{56pt}{二夜不收\hspace{10pt}({\hwfs 上}{\akai 念})人平不语,水平不流。({\hwfs 进门跪})参见相爷,有何差遣?}

\setlength{\hangindent}{52pt}{陈平\hspace{30pt}命尔等去至十字街头,点起灯笼火把高声叫喊({\akai 或}:~高声喊叫):~会犯夜的前来犯夜,不会犯夜的不要错犯了夜,若是错犯了夜,先带去见都御史陈爷,然后送往({\akai 或}:~然后送到)有司衙门审问。}

\setlength{\hangindent}{56pt}{二夜不收\hspace{10pt}启相爷,世间之上只有误犯夜,哪有掌灯叫人犯夜之理,求相爷改差。}

\setlength{\hangindent}{56pt}{陈平\hspace{30pt}嗯,相爷一言既出,驷马难追,还不下去。}

\setlength{\hangindent}{56pt}{(陈平{\hwfs 下},家院{\hwfs 拿锁链交}夜不收)}

\setlength{\hangindent}{56pt}{家院\hspace{30pt}还不快去。}

\setlength{\hangindent}{56pt}{(家院{\hwfs 下},夜不收{\hwfs 出门})}

\setlength{\hangindent}{56pt}{夜不收甲\hspace{10pt}嘿,我说伙计,相爷是老糊涂啦,有这样拿犯夜的吗?}

\setlength{\hangindent}{56pt}{夜不收乙\hspace{10pt}没法子,上边差遣,概不由己,咱们就照办吧!}

\setlength{\hangindent}{56pt}{夜不收甲\hspace{10pt}走着。}

\setlength{\hangindent}{56pt}{({\hwfs 二}夜不收{\hwfs 圆场大边台口坐倒椅})}

\setlength{\hangindent}{56pt}{二夜不收\hspace{10pt}咱们就叫唤吧。我说来人听者:~会犯夜的前来犯夜,不会犯夜的别误犯,要是误犯,先带去见都御史陈爷,然后送到有司衙门审问。犯夜的来呀!}

\setlength{\hangindent}{56pt}{(田子春{\bfseries\akai 小锣}{\hwfs 上})}

\setlength{\hangindent}{56pt}{田子春\hspace{20pt}【{\akai 西皮摇板}】行来不觉夜色晚,清清冷冷到长安。}

\setlength{\hangindent}{56pt}{(田子春{\hwfs 站小边})}

\setlength{\hangindent}{56pt}{二夜不收\hspace{10pt}会犯夜的前来犯夜,不会犯夜的别误犯夜,误犯了夜,先带去见都御史陈爷,然后送到有司衙门审问。犯夜的来呀。}

\setlength{\hangindent}{56pt}{田子春\hspace{20pt}呜哙呀,世间之上只有误犯夜,哪有掌灯叫人犯夜之理?这,嗯,莫非是陈平老儿之计?}

\setlength{\hangindent}{56pt}{田子春\hspace{20pt}哼,我不免趁此机会,将计就计,也好会见那老儿。}

\setlength{\hangindent}{56pt}{二夜不收\hspace{10pt}犯夜的来呀!}

\setlength{\hangindent}{56pt}{田子春\hspace{20pt}二位。}

\setlength{\hangindent}{56pt}{二夜不收\hspace{10pt}干什么?}

\setlength{\hangindent}{56pt}{田子春\hspace{20pt}我是远方来的,一无亲友,二无宿处,来此犯夜。}

\setlength{\hangindent}{56pt}{二夜不收\hspace{10pt}好,跟我们去见都御史。}

\setlength{\hangindent}{56pt}{田子春\hspace{20pt}我来问你,都御史是哪个?}

\setlength{\hangindent}{56pt}{二夜不收\hspace{10pt}就是我们陈老相爷。}

\setlength{\hangindent}{56pt}{田子春\hspace{20pt}敢是陈平?}

\setlength{\hangindent}{56pt}{二夜不收\hspace{10pt}我打你的嘴。}

\setlength{\hangindent}{56pt}{田子春\hspace{20pt}慢来慢来,我是你家相爷外甥。}

\setlength{\hangindent}{56pt}{二夜不收\hspace{10pt}甭管他,带他交差。}

\setlength{\hangindent}{56pt}{({\hwfs 给}田子春{\hwfs 带链子},{\hwfs 三}人{\hwfs 走圆场到小边})}

\setlength{\hangindent}{56pt}{二夜不收\hspace{10pt}请爷。}

\setlength{\hangindent}{56pt}{(家院{\hwfs 下场门上},{\hwfs 出门})}

\setlength{\hangindent}{56pt}{家院\hspace{30pt}何事?}

\setlength{\hangindent}{56pt}{二夜不收\hspace{10pt}我们拿住犯夜的啦!}

\setlength{\hangindent}{56pt}{家院\hspace{30pt}候着。有请相爷。}

\setlength{\hangindent}{56pt}{(陈平{\hwfs 下场门上})}

\setlength{\hangindent}{56pt}{陈平\hspace{30pt}只为火焚宗卷事,老夫日夜费心机({\akai 或}:~昼夜费心机)。何事?}

\setlength{\hangindent}{56pt}{家院\hspace{30pt}拿住犯夜之人。}

\setlength{\hangindent}{56pt}{(陈平{\hwfs 正座小座})}

\setlength{\hangindent}{56pt}{陈平\hspace{30pt}传。}

\setlength{\hangindent}{56pt}{(家院{\hwfs 传唤}夜不收)}

\setlength{\hangindent}{56pt}{家院\hspace{30pt}丞相呼唤。}

\setlength{\hangindent}{56pt}{夜不收甲\hspace{10pt}我先去,你看着人。({\hwfs 进门跪})}

\setlength{\hangindent}{56pt}{夜不收甲\hspace{10pt}启禀相爷,小的我拿住犯夜的啦。}

\setlength{\hangindent}{56pt}{陈平\hspace{30pt}好,来看赏。({\akai 或}:~来,看赏。)}

\setlength{\hangindent}{56pt}{(家院{\hwfs 赏}夜不收甲)}

\setlength{\hangindent}{56pt}{夜不收甲\hspace{10pt}谢相爷。}

\setlength{\hangindent}{56pt}{陈平\hspace{30pt}带犯夜人。}

\setlength{\hangindent}{56pt}{夜不收甲\hspace{10pt}是。(夜不收甲{\hwfs 出门})}

\setlength{\hangindent}{56pt}{夜不收甲\hspace{10pt}嘿,捞了一份。}

\setlength{\hangindent}{56pt}{夜不收乙\hspace{10pt}我也来一份。(夜不收乙{\hwfs 进门跪})}

\setlength{\hangindent}{56pt}{夜不收乙\hspace{10pt}启禀相爷,小的我也拿住犯夜的啦。}

\setlength{\hangindent}{56pt}{陈平\hspace{30pt}好,也看赏。({\akai 或}:~哦,你也拿住犯夜的了?嗯,也有赏。)}

\setlength{\hangindent}{56pt}{(家院{\hwfs 赏}夜不收乙)}

\setlength{\hangindent}{56pt}{夜不收乙\hspace{10pt}谢相爷。}

\setlength{\hangindent}{56pt}{陈平\hspace{30pt}带犯夜人。}

\setlength{\hangindent}{56pt}{(夜不收乙{\hwfs 出门})}

\setlength{\hangindent}{56pt}{夜不收乙\hspace{10pt}你瞧,我也来了一份。}

\setlength{\hangindent}{56pt}{夜不收甲\hspace{10pt}我再来它一份。}

\setlength{\hangindent}{56pt}{夜不收乙\hspace{10pt}别蒙事去了,招人生气。}

\setlength{\hangindent}{56pt}{夜不收甲\hspace{10pt}相爷好脾气儿。(进门{\hwfs 跪})启禀相爷,小的我还拿了一个犯夜的,他蒙事,说是相爷的外甥,我打了他一个嘴巴,还踹了他一脚。}

\setlength{\hangindent}{56pt}{陈平\hspace{30pt}哼,相爷的外甥({\akai 或}:~老夫的外男)也是尔等打得的么?来,将他二人的赏银都与我追了回来,还不下去。({\akai 或}:~来,将银子追了回来。轰了下去!)}

\setlength{\hangindent}{56pt}{(家院{\hwfs 追回二}人{\hwfs 银子})}

\setlength{\hangindent}{56pt}{二夜不收\hspace{10pt}({\akai 对白})讨哇,要哇,打哇,闹哇,都搭进去了,还招事呐,走吧!({\hwfs 下})}

\setlength{\hangindent}{56pt}{陈平\hspace{30pt}带犯夜人。}

\setlength{\hangindent}{56pt}{家院\hspace{30pt}犯夜人进见。}

\setlength{\hangindent}{56pt}{(田子春{\hwfs 进门},{\hwfs 站小边},家院{\hwfs 取下链子},家院{\hwfs 下},陈平{\hwfs 站大边},{\hwfs 一望})}

\setlength{\hangindent}{56pt}{陈平\hspace{30pt}待我看看哪个大胆,竟敢冒称老夫的外甥。({\akai 或}:~待老夫观看,哪个大胆竟敢冒称老夫的外男。)}

\setlength{\hangindent}{56pt}{田子春\hspace{20pt}好醉呀好醉!}

\setlength{\hangindent}{56pt}{陈平\hspace{30pt}呵,我看此人,毫无酒意,为何声称好醉?这$\cdots{}\cdots{}$莫非他就是那田,待我冒叫他一声,那旁敢是田?({\akai 或}:~呜哙呀,我看他毫无酒意,为何自称好醉?莫非此人就是田子春不成?嗯,待我来冒叫一声。那旁敢是田$\cdots{}\cdots{}$?)}

\setlength{\hangindent}{56pt}{田子春\hspace{20pt}那旁敢是陈?}

\setlength{\hangindent}{56pt}{陈平\hspace{30pt}田大人。}

\setlength{\hangindent}{56pt}{田子春\hspace{20pt}陈相爷。}

\setlength{\hangindent}{56pt}{(陈平、田子春 啊,啊呵呵哈哈哈$\cdots{}\cdots{}$({\hwfs 笑介}))}

\setlength{\hangindent}{56pt}{陈平\hspace{30pt}请坐。}

\setlength{\hangindent}{56pt}{田子春\hspace{20pt}有座。}

\setlength{\hangindent}{56pt}{({\hwfs 二}人{\hwfs 挖门},田子春、陈平{\hwfs 坐},田{\hwfs 大边},陈{\hwfs 小边})}

\setlength{\hangindent}{56pt}{陈平\hspace{30pt}田大人,你为何自称老夫的外甥({\akai 或}:~田大人,为何自称老夫的外男)?}

\setlength{\hangindent}{56pt}{田子春\hspace{20pt}若不如此,焉能与相爷相见。}

\setlength{\hangindent}{56pt}{陈平\hspace{30pt}是是是,请问大人来到长安有何贵干?({\akai 或}:~哦,原来如此。大人不在淮河来到长安何干?)}

\setlength{\hangindent}{56pt}{田子春\hspace{20pt}奉了幼主之命前来盗取宗卷。}

\setlength{\hangindent}{56pt}{陈平\hspace{30pt}大人你来迟了。}

\setlength{\hangindent}{56pt}{田子春\hspace{20pt}何言来迟?}

\setlength{\hangindent}{56pt}{陈平\hspace{30pt}今日早朝(在)金殿之上(,被)国太用火焚化了。}

\setlength{\hangindent}{56pt}{(陈平\hspace{30pt}用火焚化了。)}

\setlength{\hangindent}{56pt}{田子春\hspace{20pt}不好了!\textless{}\!{\bfseries\akai 撞金钟}\!\textgreater{}【{\akai 西皮摇板}】好似霹雳当头震,倒教子春无计行。叫声吾儿来相等,父子做鬼一路行。}

\setlength{\hangindent}{56pt}{陈平\hspace{30pt}大人为何这等着急?}

\setlength{\hangindent}{56pt}{田子春\hspace{20pt}相爷有所不知,幼主言道,若无宗卷就将我父子一同斩首。}

\setlength{\hangindent}{56pt}{陈平\hspace{30pt}大人就该回转淮河({\akai 或}:~就该速回淮河)搭救令郎(的)才是呀。}

\setlength{\hangindent}{56pt}{田子春\hspace{20pt}哦哦,我是不回去的了。}

\setlength{\hangindent}{56pt}{陈平\hspace{30pt}你不回去你在何处安身({\akai 或}:~你在何处落足)呐?}

\setlength{\hangindent}{56pt}{田子春\hspace{20pt}哼哼,我就住在你的府中。}

\setlength{\hangindent}{56pt}{陈平\hspace{30pt}国太知道({\akai 或}:~呃,国太闻知)那还了得。}

\setlength{\hangindent}{56pt}{田子春\hspace{20pt}不妨,我就说你请我来的。}

\setlength{\hangindent}{56pt}{陈平\hspace{30pt}哦哦,你是犯了我的夜。({\akai 或}:~呃,呃$\cdots{}\cdots{}$是你犯了我的夜啊。)}

\setlength{\hangindent}{56pt}{田子春\hspace{20pt}呀呸!世间之上只有误犯夜,哪有点起灯笼火把叫人犯夜之理?陈平呐陈平,限你三日有了宗卷便罢,若无宗卷,淮河兴兵叫你全家诛戮。}

\setlength{\hangindent}{56pt}{陈平\hspace{30pt}大人不必如此,请至书房憩息。({\akai 或}:~呃,慢来慢来,请至书房。呃,请至书房歇息。)}

\setlength{\hangindent}{56pt}{田子春\hspace{20pt}你不请我我不来({\akai 或}:~天堂有路我不走)。}

\setlength{\hangindent}{56pt}{陈平\hspace{30pt}自投罗网怨谁来({\akai 或}:~地狱无门(你)闯进来)。}

\setlength{\hangindent}{56pt}{田子春\hspace{20pt}根深哪怕风摇摆。}

\setlength{\hangindent}{56pt}{陈平\hspace{30pt}准备棺木将你埋。}

\setlength{\hangindent}{56pt}{田子春\hspace{20pt}你埋哪个?}

\setlength{\hangindent}{56pt}{陈平\hspace{30pt}不是你还有哪个。}

\setlength{\hangindent}{56pt}{田子春\hspace{20pt}陈平呐陈平,三日之后若无宗卷你要打点了,你要仔细了!}

\setlength{\hangindent}{56pt}{陈平\hspace{30pt}哦哦,(大人不必动怒,)请至书房,请至书房。}

\setlength{\hangindent}{56pt}{田子春\hspace{20pt}哼。}

\setlength{\hangindent}{56pt}{(田子春{\hwfs 下},家院{\hwfs 上})}

\setlength{\hangindent}{56pt}{陈平\hspace{30pt}哎呀呀,哪里是他犯了我的夜,倒是我犯了他的夜({\akai 或}:~分明是我犯了他的夜了),哎呀这这这,嗨,我临死({\akai 或}:~我至死)也要拉上一个垫背的,来,拿我名帖去请张苍张大人夤夜过府饮宴,快去({\akai 或}:~不得有误)。}

\setlength{\hangindent}{56pt}{(家院\hspace{30pt}遵命。)}

\setlength{\hangindent}{56pt}{(家院{\hwfs 接帖下})}

\setlength{\hangindent}{56pt}{陈平\hspace{30pt}嗨,若能留得宗卷在,也免老夫挂心怀。嗨!({\akai 或}:~唉,但愿留得宗卷在,免得老夫挂心怀。唉!)}

\setlength{\hangindent}{56pt}{(陈平{\hwfs 下})}

\vspace{3pt}{\centerline{{[}{\hei 第四场}{]}}}\vspace{5pt}

\setlength{\hangindent}{56pt}{(\textless{}\!{\bfseries\akai 小锣抽头}\!\textgreater{}家院{\hwfs 掌灯笼引}张苍{\hwfs 上},{\hwfs 站台中间})}

\setlength{\hangindent}{56pt}{张苍\hspace{30pt}【{\akai 西皮摇板}】正在府中({\akai 或}:~正在衙中)愁闷坏,陈平有帖请我来。}

\setlength{\hangindent}{56pt}{张苍\hspace{30pt}({\akai 念})下官张苍,陈平有柬帖相邀,请我夤夜过府饮宴,却是为何?我二人虽是一殿为臣,却素无来往呵。哦是了,想是今日早朝,言语之间冲撞于我,请我过府与我赔礼也是有的,哎呀呀,哈哈哈老相爷呀,你我俱是炎汉忠良,哪个还怪你不成,你这是何苦哇?({\akai 或}:~下官张苍,陈平有柬帖相邀,不知为了何事。我二人素无来往啊。哦$\cdots{}\cdots{}$是了,想是今日早朝,金殿之上,冲撞于我,夤夜请我过府饮宴,与我赔上一个礼儿,呃,也是有之,哎呀呀老相爷呀,你我俱是炎汉忠良,一殿为臣,冲撞几句,又有何妨?你这是何苦哇。)}

\setlength{\hangindent}{56pt}{张苍\hspace{30pt}({\akai 接唱})【{\akai 摇板}】家院掌灯把路带,去至相府饮开怀({\akai 或}:~家院与爷把路带,见了相爷说开怀)。}

\setlength{\hangindent}{56pt}{(家院{\hwfs 引}张苍{\hwfs 下})}

\vspace{3pt}{\centerline{{[}{\hei 第五场}{]}}}\vspace{5pt}

\setlength{\hangindent}{56pt}{(陈平{\hwfs 上拿书})}

\setlength{\hangindent}{56pt}{陈平\hspace{30pt}\textless{}\!{\bfseries\akai 小锣抽头}\!\textgreater{}【{\akai 西皮快板}】淮河来了田子春,倒叫老夫心内惊({\akai 或}:~盗取皇王宗卷文)。将身且坐二堂等,等候张苍到来临。}

\setlength{\hangindent}{56pt}{(陈{\hwfs 坐桌大边座},{\hwfs 看书},家院{\hwfs 引}张苍{\hwfs 上},{\hwfs 小边},张苍{\hwfs 唱中小圆场})}

\setlength{\hangindent}{56pt}{张苍\hspace{30pt}【{\akai 西皮快板}】吕后做事太欺情,不该宗卷备火焚({\akai 或}:可恨吕后心太狠,火焚宗卷谋乾坤)。({\akai 或}:~宗卷不该用火焚。)家院掌灯({\akai 或}:~家院向前)把路{\akai 引},不觉来到相府门。}

\setlength{\hangindent}{56pt}{张苍\hspace{30pt}({\akai 念})你在府门稍待片刻,待我进府略饮几杯就要回去。({\akai 或}:~来此相府,你就在马待石\footnote{夏行涛{\scriptsize 君}建议``马待石''应作``马台石'';马台石是过去上、下马的垫脚石。此处从《京剧新序》原文。}前等候,你家老爷去至里面略饮几巡,就要出来。)}

\setlength{\hangindent}{56pt}{家院\hspace{30pt}您可悠着点儿。}

\setlength{\hangindent}{56pt}{张苍\hspace{30pt}(我)晓得(呀)。【{\akai 西皮摇板}】张苍撩衣({\akai 或}:~撩袍)进府门,}

\setlength{\hangindent}{56pt}{(家院{\hwfs 下},张苍{\hwfs 进门},{\hwfs 小圆场唱},陈{\hwfs 放下书在张唱中饮酒})}

\setlength{\hangindent}{56pt}{张苍\hspace{30pt}【{\akai 西皮快板}】静静悄悄无一人。来至在({\akai 或}:~站立在)二堂来观定,陈平一人饮杯巡。}

\setlength{\hangindent}{56pt}{张苍\hspace{30pt}({\akai 念})陈相爷请我过府饮宴,怎么他自斟自饮起来了,啊啊是了,想是他等我不及,酒兴发作,先行自饮几杯,等我到来再大排筵宴,也是有的。我不免痰嗽一声,他必然下位迎接于我,嗯喷,呵来了。({\akai 或}:~陈平老儿怎么一人自斟自饮起来了?哦,想是等我不及,先饮几杯,待我痰嗽一声,惊动于他,他必然迎接于我,嗯,嗯喷。呵呵,来了来了。)}

\setlength{\hangindent}{56pt}{(陈平{\hwfs 站出门},{\hwfs 唱中大边向外跪拜},张苍{\hwfs 随拜跪})}

\setlength{\hangindent}{56pt}{陈平\hspace{30pt}【{\akai 西皮快板}】陈平撩衣来下拜({\akai 或}:~对着苍天把礼拜),过往神祗({\akai 或}:~过往的神灵)听开怀。我若有意来降吕({\akai 或}:~我若是有意降了吕),天地({\akai 或}:~老天)与我降祸灾。叩罢头来深深拜,我看张苍怎起来。}

\setlength{\hangindent}{56pt}{(陈平{\hwfs 末句前站起来},{\hwfs 用手横指}张苍,张{\hwfs 跪坐介},陈{\hwfs 进门坐下饮酒})}

\setlength{\hangindent}{56pt}{(陈平\hspace{30pt}表罢忠心,再饮几杯。)}

\setlength{\hangindent}{56pt}{张苍\hspace{30pt}嘿嘿,(张苍{\hwfs 立})我道他下位迎接于我,原来他表起他的忠心来了。陈平呐陈平,你是炎汉忠良我张苍就不是炎汉忠良么?你表得我也表得,要表我们大家表上一表。({\akai 或}:~嘿嘿,(张苍{\hwfs 立})我道他下位迎接于我,他倒表起他的忠心来了。你是炎汉忠良,难道我张苍就不是炎汉忠良么?你表得我也表得,你表我也表,要表我们大家表上一表。)}

\setlength{\hangindent}{56pt}{张苍\hspace{30pt}【{\akai 西皮快板}】张苍撩衣跪尘埃({\akai 或}:~对着苍天忙下拜),过往的神灵听开怀。我若背汉({\akai 或}:~我若是有意)降了吕,老天爷与我降祸灾。叩罢头,深深拜({\akai 或}:~抽身拜),问声相爷可安泰。}

\setlength{\hangindent}{56pt}{张苍\hspace{30pt}参见相爷。}

\setlength{\hangindent}{56pt}{(张苍{\hwfs 唱中小边向外跪拜},{\hwfs 唱中站},{\hwfs 进门拜},陈平{\hwfs 站大边})}

\setlength{\hangindent}{56pt}{陈平\hspace{30pt}(啊?哦,哦,)原来是张大人,哦,夤夜来到鄙衙,敢是要查看老夫的弊病不成({\akai 或}:~敢是查看老夫的弊病来了么)?}

\setlength{\hangindent}{56pt}{张苍\hspace{30pt}慢来慢来,老相爷柬帖相邀,请我过府饮宴,何言弊病二字?({\akai 或}:~相爷说哪里话来,下官乃是奉相爷之命,夤夜前来饮宴的呀。)}

\setlength{\hangindent}{56pt}{陈平\hspace{30pt}(哦,呃,怎么还有此事么?)哎呀呀,(不是张大人提起,)我倒忘怀了。}

\setlength{\hangindent}{56pt}{张苍\hspace{30pt}你看你看,有这样请客的么?}

\setlength{\hangindent}{56pt}{陈平\hspace{30pt}备席不及,来来来这有残酒拿来去饮。({\akai 或}:~备酒不及,我这有残酒,拿去饮来。)}

\setlength{\hangindent}{56pt}{(张苍\hspace{30pt}谢相爷。)}

\setlength{\hangindent}{56pt}{(陈平\hspace{30pt}呀呸!)}

\setlength{\hangindent}{56pt}{(陈平{\hwfs 拿酒},{\hwfs 泼}张苍{\hwfs 脸}上,张{\hwfs 双袖擦脸}\textless{}\!{\bfseries\akai 叫头}\!\textgreater{})}

\setlength{\hangindent}{56pt}{张苍\hspace{30pt}老相爷,这酒吃与不吃,不关紧要,为何泼在下官脸上?({\akai 或}:~唉呀相爷呀,一杯水酒吃与不吃,不值紧要,为何将酒泼在下官的脸上?)}

\setlength{\hangindent}{56pt}{陈平\hspace{30pt}我这酒,你吃它不得。}

\setlength{\hangindent}{56pt}{张苍\hspace{30pt}哪个吃得?}

\setlength{\hangindent}{56pt}{陈平\hspace{30pt}炎汉忠良方能吃得。}

\setlength{\hangindent}{56pt}{张苍\hspace{30pt}难道说我张苍就不是炎汉忠良么?}

\setlength{\hangindent}{56pt}{陈平\hspace{30pt}(我来问你,)你官居何职?}

\setlength{\hangindent}{56pt}{张苍\hspace{30pt}西台御史。}

\setlength{\hangindent}{56pt}{陈平\hspace{30pt}掌管何事?({\akai 或}:~掌管何物?)}

\setlength{\hangindent}{56pt}{张苍\hspace{30pt}皇家宗卷。({\akai 或}:~皇王宗卷。)}

\setlength{\hangindent}{56pt}{陈平\hspace{30pt}拿来。}

\setlength{\hangindent}{56pt}{张苍\hspace{30pt}什么?}

\setlength{\hangindent}{56pt}{陈平\hspace{30pt}宗卷呐。}

\setlength{\hangindent}{56pt}{张苍\hspace{30pt}哎呀相爷呀,今日早朝,宗卷({\akai 或}:~宗卷今日早朝)被国太用火焚化,(呃,)还是老相爷你保的本呢。}

\setlength{\hangindent}{56pt}{陈平\hspace{30pt}住了,张苍呐张苍,限你三天有了宗卷便罢,不然就将你全家诛戮。({\akai 或}:~呀呸!~限你三天有了宗卷便罢,不然要将你全家诛戮。)}

\setlength{\hangindent}{56pt}{张苍\hspace{30pt}哎呀!【{\akai 西皮小导板}】听一言吓得我({\akai 或}:~听一言不由我)三魂不在。}

\setlength{\hangindent}{56pt}{陈平\hspace{30pt}唗,此地什么所在?}

\setlength{\hangindent}{56pt}{张苍\hspace{30pt}乃是堂堂相府。}

\setlength{\hangindent}{56pt}{陈平\hspace{30pt}既知堂堂相府,为何这等喧嚣({\akai 或}:~这等高声喊叫)?来人({\akai 或}:~家院,)与我轰,与我赶,轰赶轰,轰了出去。}

\setlength{\hangindent}{56pt}{(陈平{\hwfs 一}、{\hwfs 二转身单投袖},{\hwfs 一}、{\hwfs 二}、{\hwfs 三轰},{\hwfs 双投袖轰},{\hwfs 下}。{\hwfs 同时}张苍{\hwfs 一}、{\hwfs 二转身揖},{\hwfs 一}、{\hwfs 二}、{\hwfs 三退揖},{\hwfs 双投袖},{\akai 念}``哎呀'',{\hwfs 出门}。家院{\hwfs 暗上})}

\setlength{\hangindent}{56pt}{张苍\hspace{30pt}【{\akai 西皮散板}】黑洞洞摸出了(这)相府门(来)。}

\setlength{\hangindent}{56pt}{张苍\hspace{30pt}家院。(家院$\cdots{}\cdots{}$)}

\setlength{\hangindent}{56pt}{家院\hspace{30pt}散席啦?}

\setlength{\hangindent}{56pt}{张苍\hspace{30pt}(好奴才,)走,回去。}

\setlength{\hangindent}{56pt}{(\textless{}\!{\bfseries\akai 扫头}\!\textgreater{}张{\hwfs 推}家院{\hwfs 下},{\hwfs 圆场进门站中间},\textless{}\!{\bfseries\akai 叫头}\!\textgreater{})}

\setlength{\hangindent}{56pt}{张苍\hspace{30pt}哎呀且住!陈平老儿哪里是请我过府饮宴,限我三日有了宗卷便罢,若无宗卷就要将我全家诛戮。({\akai 或}:~且住!陈平老儿请我过府饮宴,哪里是饮宴,限我三天有了宗卷便罢,不然要将我的全家诛、诛$\cdots{}\cdots{}$戮。)也罢,我不免拜谢先王爵禄之恩,寻个自尽了罢。}

\setlength{\hangindent}{56pt}{张苍\hspace{30pt}【{\akai 西皮散板}】张苍撩衣跪埃尘,拜谢先王爵禄恩。一把钢刀拿在手,}

\setlength{\hangindent}{56pt}{({\hwfs 跪拜起来在桌上拿单刀},{\hwfs 台口看刀},{\hwfs 转身推刀出手落小边台口},{\hwfs 在大边里边左袖盖头}、{\hwfs 右手指刀},{\hwfs 抬右腿}、{\hwfs 左腿单腿立})}

\setlength{\hangindent}{56pt}{张苍\hspace{30pt}({\akai 接唱})(这)白亮亮钢刀吓煞人!}

\setlength{\hangindent}{56pt}{({\hwfs 出门向下场门})}

\setlength{\hangindent}{56pt}{张苍\hspace{30pt}夫人,为丈夫在此自刎({\akai 或}:~下官在此自尽),你要劝一劝呐,(你要)拉一拉呐,唉!(边过小边边接唱)我这里十叫九不应呐。}

\setlength{\hangindent}{56pt}{({\hwfs 到小边向上场门})}

\setlength{\hangindent}{56pt}{张苍\hspace{30pt}秀玉儿,为父的在此寻死,你要劝一劝呐,你要拉一拉呐。唉!({\hwfs 回身}{\akai 接唱})秀玉一边不作声。({\hwfs 归中}{\akai 接唱})千思万想无计定,祸到临头难逃生。}

\setlength{\hangindent}{56pt}{张苍\hspace{30pt}罢!({\akai 接唱})咬定牙关项上刎。}

\setlength{\hangindent}{56pt}{({\hwfs 拾刀},{\hwfs 台中间刎介},夫人{\hwfs 上场门上},{\hwfs 夺刀},张{\hwfs 大边},夫人{\hwfs 小边站})}

\setlength{\hangindent}{56pt}{夫人\hspace{30pt}【{\akai 西皮散板}】老爷自刎为何情?}

\setlength{\hangindent}{56pt}{张苍\hspace{30pt}哎呀夫人呐,那陈平老儿哪里是请我过府饮宴,他限我有了宗卷便罢,若无宗卷就将我全家诛戮哇$\cdots{}\cdots{}$({\akai 或}:~哎呀夫人呐,陈平老儿限我三天有了宗卷便罢,不然要将我全家,唉,诛,诛戮哇$\cdots{}\cdots{}$)({\hwfs 哭}{\hwfs 介})}

\setlength{\hangindent}{56pt}{夫人\hspace{30pt}不好了!}

\setlength{\hangindent}{56pt}{夫人\hspace{30pt}【{\akai 西皮散板}】心中只把陈平恨,因何谋害我满门?}

\setlength{\hangindent}{56pt}{(张苍{\hwfs 大边}、夫人{\hwfs 小边},{\hwfs 桌旁八字坐},张秀玉{\hwfs 下场门反上})}

\setlength{\hangindent}{56pt}{秀玉\hspace{30pt}【{\akai 西皮摇板}】正在书房看经纶,忽听前堂放悲声。(进门)}

\setlength{\hangindent}{56pt}{秀玉\hspace{30pt}参见爹爹。}

\setlength{\hangindent}{56pt}{(秀玉{\hwfs 拜},{\hwfs 站大边})}

\setlength{\hangindent}{56pt}{张苍\hspace{30pt}儿是({\akai 或}:~哦,你是)秀玉?}

\setlength{\hangindent}{56pt}{秀玉\hspace{30pt}正是。}

\setlength{\hangindent}{56pt}{张苍\hspace{30pt}儿来了?}

\setlength{\hangindent}{56pt}{秀玉\hspace{30pt}来了。}

\setlength{\hangindent}{56pt}{张苍\hspace{30pt}儿来得好哇({\akai 或}:~你来得好哇)$\cdots{}\cdots{}$}

\setlength{\hangindent}{56pt}{秀玉\hspace{30pt}爹爹为何如此?}

\setlength{\hangindent}{56pt}{张苍\hspace{30pt}小小年纪懂得什么({\akai 或}:~晓得什么),(快快)攻书去罢。}

\setlength{\hangindent}{56pt}{秀玉\hspace{30pt}有何大事?孩儿愿为爹爹分忧解愁。}

\setlength{\hangindent}{56pt}{夫人\hspace{30pt}是呀,老爷说将出来,孩儿与你分忧解愁。}

\setlength{\hangindent}{56pt}{张苍\hspace{30pt}呵,分忧解愁,哎呀儿呀,那陈平老儿哪里是请为父过府饮宴,他限我三日有皇家宗卷便罢,若无宗卷就将我全家诛戮哇$\cdots{}\cdots{}$({\akai 或}:~讲得的么?哦,讲得的,哎呀儿啊,陈平老儿限我三日有了宗卷便罢,不然要将我的全家,诛,诛戮哇$\cdots{}\cdots{}$({\hwfs 哭}{\hwfs 介}))}

\setlength{\hangindent}{56pt}{秀玉\hspace{30pt}啊,原来是一桩小事。}

\setlength{\hangindent}{56pt}{张苍\hspace{30pt}啊,小事,儿(啊,你)近前来,好奴才!}

\setlength{\hangindent}{56pt}{(秀玉{\hwfs 回身逃下},张苍{\hwfs 拿刀追},夫人{\hwfs 拉回坐下})}

\setlength{\hangindent}{56pt}{张苍\hspace{30pt}什么哦小事!({\akai 或}:~哼,小事?有这样的小事?!哼!)}

\setlength{\hangindent}{56pt}{(秀玉{\hwfs 下场门反上})}

\setlength{\hangindent}{56pt}{秀玉\hspace{30pt}【{\akai 西皮摇板}】汉室宗卷手捧定,上前交与老爹尊。}

\setlength{\hangindent}{56pt}{张苍\hspace{30pt}啊夫人,眼看大祸临门,世间之上还有这样的小事么?({\akai 或}:~夫人,你看,这样的大事,说什么是小事,有这样的小事吗?!)}

\setlength{\hangindent}{56pt}{(张苍{\hwfs 大摊手},秀玉{\hwfs 上前将卷放}张{\hwfs 左手上},张{\hwfs 看})}

\setlength{\hangindent}{56pt}{张苍\hspace{30pt}哈$\cdots{}\cdots{}$,夫人,这个奴才被我吓糊涂了,拿本古书前来搪塞老夫来了。({\akai 或}:~哈$\cdots{}\cdots{}$哎呀,这个奴才是被我吓糊涂了哇,拿本古书前来蒙哄老夫来了。)}

\setlength{\hangindent}{56pt}{秀玉\hspace{30pt}宗卷也罢,古书也罢,要看个明白。}

\setlength{\hangindent}{56pt}{夫人\hspace{30pt}是呀,要看个明白。}

\setlength{\hangindent}{56pt}{张苍\hspace{30pt}哦,(怎么,是与不是)要看个明白,(唉,)如此我就看、看(、看)呐!}

\setlength{\hangindent}{56pt}{(张苍{\hwfs 立},{\hwfs 站中间})}

\setlength{\hangindent}{56pt}{张苍\hspace{30pt}【{\akai 西皮快板}】这奴才({\akai 或}:~小奴才)被我吓懵懂,拿本古书当卷宗。是与不是从头看,有劳夫人({\akai 或}:~夫人与我)掌灯红。}

\setlength{\hangindent}{56pt}{({\hwfs 拿卷到台口},夫人{\hwfs 掌灯小边},秀玉{\hwfs 大边},{\hwfs 三}人{\hwfs 站},张{\hwfs 唱}、夫人{\hwfs 夹白})}

\setlength{\hangindent}{56pt}{张苍\hspace{30pt}\textless{}\!{\bfseries\akai 小锣抽头}\!\textgreater{}【{\akai 西皮导板}】初起义来在沛丰。}

\setlength{\hangindent}{56pt}{张苍\hspace{30pt}啊,这是宗卷呐,啊夫人,我在此作甚呐?}

\setlength{\hangindent}{56pt}{(夫人{\akai 夹白}:\hspace{10pt}在此看卷呐。)}

\setlength{\hangindent}{56pt}{张苍\hspace{30pt}只恐不是看卷罢。}

\setlength{\hangindent}{56pt}{(夫人{\akai 夹白}:\hspace{10pt}是做什么?)}

\setlength{\hangindent}{56pt}{张苍\hspace{30pt}是做梦罢?}

\setlength{\hangindent}{56pt}{(夫人{\akai 夹白}:\hspace{10pt}你看漫天星斗,怎说是做梦?)}

\setlength{\hangindent}{56pt}{张苍\hspace{30pt}(哦,是看卷?)不是做梦。夫人你将灯掌高些,啊,太高了要矮一些({\akai 或}:~忒高了),啊,又太矮了({\akai 或}:~又忒矮了),掌灯(是)要齐眉的呀,啊,烧了眉毛了。儿呀,将灯接了过来。啊夫人,你看你我的儿子掌灯比你强得多呀。}

\setlength{\hangindent}{56pt}{(夫人、秀玉{\hwfs 掌灯作势})}

\setlength{\hangindent}{56pt}{张苍\hspace{30pt}【{\akai 西皮快板}】剑斩白蛇路途中。第一排汉高祖,第二排吕正宫。三宫六院有牌供,关东十王一派宗。宗卷看到第七册,幼主本是赵娘生。({\akai 或}:~三宫六院承恩众,各个立者俱有封。关东十王有牌位,幼主本是赵妃生。)\textless{}\!{\bfseries\akai 三锣}\!\textgreater{}}

\setlength{\hangindent}{56pt}{(张苍、夫人{\hwfs 归座},\textless{}\!{\bfseries\akai 大锣原场}\!\textgreater{})}

\setlength{\hangindent}{56pt}{张苍\hspace{30pt}儿呀,宗卷已被国太焚化,这是哪里来的?({\akai 或}:~儿啊,这部宗卷是哪里来的?)}

\setlength{\hangindent}{56pt}{秀玉\hspace{30pt}爹爹在癸未年间染病在床,命孩儿代守宗卷,孩儿看到第七册第七篇,见襄宫赵娘娘死得可惨,犹恐日后有变,为此将宗卷抄下一部以防后患。}

\setlength{\hangindent}{56pt}{张苍\hspace{30pt}这有一事不合律。({\akai 或}:~只是有一事不应典呐。)}

\setlength{\hangindent}{56pt}{秀玉\hspace{30pt}哪一事?}

\setlength{\hangindent}{56pt}{张苍\hspace{30pt}这皇家玉玺是怎样来的?({\akai 或}:~这皇王玉玺是哪里来的?)}

\setlength{\hangindent}{56pt}{秀玉\hspace{30pt}那是孩儿用黄蜡雕成玉玺,真的上面原有一颗,假的上面也打上它一颗,与它真假难辨。}

\setlength{\hangindent}{56pt}{张苍\hspace{30pt}国太用火焚的呢?}

\setlength{\hangindent}{56pt}{秀玉\hspace{30pt}乃是假的。}

\setlength{\hangindent}{56pt}{张苍\hspace{30pt}这呢?}

\setlength{\hangindent}{56pt}{秀玉\hspace{30pt}这是历代历代的老宗卷。}

\setlength{\hangindent}{56pt}{张苍\hspace{30pt}(如此说来)这是历代历代的老宗卷?}

\setlength{\hangindent}{56pt}{秀玉\hspace{30pt}老谱头。}

\setlength{\hangindent}{56pt}{张苍\hspace{30pt}老谱头,哈哈哈,这才是我的好儿子呀。夫人,像这样的儿子你(为何不)与我多养上几个哇!}

\setlength{\hangindent}{56pt}{夫人\hspace{30pt}取笑了。}

\setlength{\hangindent}{56pt}{张苍\hspace{30pt}夫人,你们准备逃回原郡,如今有了宗卷,我怕他何来?我要与那陈平老儿大闹一场呐。({\akai 或}:~是啊,有了宗卷我怕他何来?我要去至相府,与那陈平老儿大闹一场啊。)}

\setlength{\hangindent}{56pt}{张苍\hspace{30pt}【{\akai 西皮散板}】辞别夫人出府门,({\akai 白})(夫人,呃,宗、宗卷呢?)宗卷呢?({\hwfs 笑介})}

\setlength{\hangindent}{56pt}{(张苍{\hwfs 出门},家院{\hwfs 暗上})}

\setlength{\hangindent}{56pt}{家院\hspace{30pt}出门找。({\hwfs 找介},秀玉{\hwfs 看}张{\hwfs 手中拿卷},张{\hwfs 笑},家院{\hwfs 领}张苍{\hwfs 左手托卷}、{\hwfs 右手袖盖头},\textless{}\!{\bfseries\akai 扫头}\!\textgreater{}{\hwfs 下})}

\setlength{\hangindent}{56pt}{夫人\hspace{30pt}({\akai 接唱})一见老爷出府门,稳坐内衙等信音。}

\setlength{\hangindent}{56pt}{(夫人、秀玉{\hwfs 下})}

\vspace{3pt}{\centerline{{[}{\hei 第六场}{]}}}\vspace{5pt}

\setlength{\hangindent}{56pt}{张苍\hspace{30pt}({\akai 内唱})【{\akai 西皮导板}】家院与爷({\akai 或}:~家院掌灯)把路{\akai 引},}

\setlength{\hangindent}{56pt}{(家院{\hwfs  引}张苍{\hwfs 上},院{\hwfs 倒},张{\hwfs 从}家院{\hwfs 身上绊倒},{\hwfs 宗卷扔到大边台口},张、院{\hwfs 起身},{\hwfs 找卷})}

\setlength{\hangindent}{56pt}{张苍\hspace{30pt}(呃呃呃,)宗卷呢?}

\setlength{\hangindent}{56pt}{家院\hspace{30pt}找。}

\setlength{\hangindent}{56pt}{(家院{\hwfs 提灯领}张苍,{\hwfs 走太极图},院{\hwfs 在大边照见宗卷},张{\hwfs 在小边里边转身蹉步拾卷},{\hwfs 掸宗卷上土三下},院、张{\hwfs 圆场},院{\hwfs 下},张{\hwfs 进门介})}

\setlength{\hangindent}{56pt}{张苍\hspace{30pt}【{\akai 西皮散板}】有了宗卷怕何人。小首不坐大首坐,({\hwfs 坐台口大边椅})}

\setlength{\hangindent}{56pt}{张苍\hspace{30pt}({\akai 接唱})问我一言(我)答一声。}

\setlength{\hangindent}{56pt}{(陈平{\hwfs 上})}

\setlength{\hangindent}{56pt}{陈平\hspace{30pt}那张苍(老儿)回得衙去,哪里去寻宗卷呐。他不是仰药伏刀,定是投井悬梁,他是不能来的了。啊,他怎么来了({\akai 或}:~他呀,他不能来了,他,他$\cdots{}\cdots{}$他不能$\cdots{}\cdots{}$呃,他倒先来了。),啊,张苍你来了?}

\setlength{\hangindent}{56pt}{张苍\hspace{30pt}我早就来了。({\akai 或}:~我为何不来呀?)}

\setlength{\hangindent}{56pt}{陈平\hspace{30pt}(诶,)张苍,你怎么连品级台位都不讲了({\akai 或}:~都不顾了)?}

\setlength{\hangindent}{56pt}{张苍\hspace{30pt}(嗯,)太平年间有个({\akai 或}:~太平年间讲的是)品级台位,这离乱年间,还讲什么品级台位,这个座位你张大人坐坐何妨啊?}

\setlength{\hangindent}{56pt}{(张苍{\hwfs 盘一腿坐})}

\setlength{\hangindent}{56pt}{陈平\hspace{30pt}老夫也不怪罪于你({\akai 或}:~呵呵呵,好好好,我也不计较于你),拿来。}

\setlength{\hangindent}{56pt}{张苍\hspace{30pt}什么?}

\setlength{\hangindent}{56pt}{陈平\hspace{30pt}宗卷呐。}

\setlength{\hangindent}{56pt}{张苍\hspace{30pt}你要几十部?}

\setlength{\hangindent}{56pt}{陈平\hspace{30pt}(诶,)一部可也就够了。}

\setlength{\hangindent}{56pt}{张苍\hspace{30pt}我当是要几十部,(嗯,)拿去(看来)!({\hwfs 递卷})}

\setlength{\hangindent}{56pt}{(陈平{\hwfs 拿卷})\hspace{10pt}}

\setlength{\hangindent}{56pt}{陈平\hspace{30pt}({\hwfs 笑}{\hwfs 介})这个老儿到底是(被我)吓糊涂了,拿本古书前来搪塞老夫来了。}

\setlength{\hangindent}{56pt}{张苍\hspace{30pt}宗卷也罢,古书也罢,你要看呐!({\akai 或}:~是与不是,你要看个明白。)}

\setlength{\hangindent}{56pt}{陈平\hspace{30pt}(嗯,)我倒要仔细地看上一看。}

\setlength{\hangindent}{56pt}{(陈平{\hwfs 坐桌旁小边座},{\hwfs 看卷})}

\setlength{\hangindent}{56pt}{陈平\hspace{30pt}``初起义来在沛丰'',哎呀张大人(呐),这是宗卷呀!(陈平立到台中间)}

\setlength{\hangindent}{56pt}{张苍\hspace{30pt}这不是宗卷呐。}

\setlength{\hangindent}{56pt}{陈平\hspace{30pt}是什么?}

\setlength{\hangindent}{56pt}{张苍\hspace{30pt}乃是卷宗啊。({\hwfs 右手画圈},{\hwfs 往上一指})}

\setlength{\hangindent}{56pt}{陈平\hspace{30pt}来来来({\akai 或}:~诶呵,取笑了。诶,有了宗卷),请来上坐。}

\setlength{\hangindent}{56pt}{张苍\hspace{30pt}慢来慢来,老相爷的座位是有品级台位的。({\akai 或}:~呃呃呃,这是有品级台位的呀。)}

\setlength{\hangindent}{56pt}{陈平\hspace{30pt}(诶,)有了宗卷就没有品级台位了,请坐请坐。}

\setlength{\hangindent}{56pt}{({\hwfs 二人桌旁八字坐},张苍{\hwfs 大边},陈平{\hwfs 小边})}

\setlength{\hangindent}{56pt}{陈平\hspace{30pt}(啊,张大人,)这部宗卷是哪里来的?}

\setlength{\hangindent}{56pt}{张苍\hspace{30pt}老相爷哪里知道,下官在癸未年间染病在床,({\akai 或}:~相爷有所不知,只因癸未年间下官染病在床,)命小儿({\akai 或}:~我儿)秀玉代守宗卷,是他看到(宗卷)第七部第七篇,见襄宫赵娘娘死得可惨,犹恐日后有变,为此抄写一部以防后患。}

\setlength{\hangindent}{56pt}{(陈平\hspace{30pt}金殿之上被国太用火焚化的?)}

\setlength{\hangindent}{56pt}{(张苍\hspace{30pt}乃是假的。)}

\setlength{\hangindent}{56pt}{(陈平\hspace{30pt}这一部呢?)}

\setlength{\hangindent}{56pt}{(张苍\hspace{30pt}这才是历代历代的老宗卷呐。)}

\setlength{\hangindent}{56pt}{陈平\hspace{30pt}(老宗卷$\cdots{}\cdots{}$呃,)只是有一桩不合律({\akai 或}:~不应典)。}

\setlength{\hangindent}{56pt}{张苍\hspace{30pt}哪一件不合律?({\akai 或}:~哪一事不应典?)}

\setlength{\hangindent}{56pt}{陈平\hspace{30pt}这皇家的玉玺({\akai 或}:~这皇王玉玺)是哪里来的?}

\setlength{\hangindent}{56pt}{张苍\hspace{30pt}(哎,)也是我儿一时聪明,将黄蜡({\akai 或}:~用黄蜡)雕成玉玺,真的上面原有一颗,假的上面也打上它一颗,与它(个)真假难辨。}

\setlength{\hangindent}{56pt}{陈平\hspace{30pt}金殿之上国太用火焚的?}

\setlength{\hangindent}{56pt}{张苍\hspace{30pt}乃是假的。}

\setlength{\hangindent}{56pt}{陈平\hspace{30pt}这呢?}

\setlength{\hangindent}{56pt}{张苍\hspace{30pt}乃是历代历代的老宗卷。}

\setlength{\hangindent}{56pt}{陈平\hspace{30pt}哦,老宗卷?({\akai 或}:~如此说来这是历代历代的老宗卷?)}

\setlength{\hangindent}{56pt}{张苍\hspace{30pt}老谱头。({\akai 或}:~老宗卷。)}

\setlength{\hangindent}{56pt}{陈平\hspace{30pt}老谱头。哦。({\akai 或}:~老谱头?)}

\setlength{\hangindent}{56pt}{张苍\hspace{30pt}哦。({\hwfs 二}人{\hwfs 同笑})哈哈哈!({\akai 或}:~老谱头,啊。({\hwfs 二}人{\hwfs 同笑})哈哈哈!)}

\setlength{\hangindent}{56pt}{陈平\hspace{30pt}(啊,张大人,)令郎今年多大年纪了?}

\setlength{\hangindent}{56pt}{张苍\hspace{30pt}我的儿子他今年一十六岁了。({\akai 或}:~他么,嗯,今年一十七岁了。)}

\setlength{\hangindent}{56pt}{陈平\hspace{30pt}哎呀呀,一十六岁就有这样见识,真乃是出将入相之材。({\akai 或}:~哦,一十几岁就是如此地聪明,日后定是出将入相之材。)}

\setlength{\hangindent}{56pt}{张苍\hspace{30pt}实不瞒老相爷说呀,我那儿子呀,日后定有你这位份。({\akai 或}:~嗯,我那个儿子啊,日后可以有相爷你这个份位。)}

\setlength{\hangindent}{56pt}{(陈平{\hwfs 立大边外场背供})}

\setlength{\hangindent}{56pt}{陈平\hspace{30pt}呜哙呀,你看这老儿,我不过是奉承他几句({\akai 或}:~两句),他倒冒起高来了,我倒要({\akai 或}:~呃,待我来)耍上他一耍,啊,张苍,你好哇?}

\setlength{\hangindent}{56pt}{张苍\hspace{30pt}我怎么不好哇?({\akai 或}:~我是怎的不好哇?)}

\setlength{\hangindent}{56pt}{(陈平\hspace{30pt}你好。)}

\setlength{\hangindent}{56pt}{(张苍\hspace{30pt}嗯,我好。)}

\setlength{\hangindent}{56pt}{陈平\hspace{30pt}你好大的胆量呐!}

\setlength{\hangindent}{56pt}{张苍\hspace{30pt}啊?}

\setlength{\hangindent}{56pt}{陈平\hspace{30pt}你过来。({\akai 或}:~你父子私抄皇王宗卷。)}

\setlength{\hangindent}{56pt}{张苍\hspace{30pt}做什么?({\akai 或}:~不曾。)}

\setlength{\hangindent}{56pt}{陈平\hspace{30pt}你父子誊写宗卷,私造玉玺,要谋篡国太的江山。({\akai 或}:~假造皇王玉玺。)}

\setlength{\hangindent}{56pt}{张苍\hspace{30pt}哪有此事?({\akai 或}:~哼,无有。)}

\setlength{\hangindent}{56pt}{陈平\hspace{30pt}谋篡江山。({\akai 或}:~分明有谋篡国太江山之意。)}

\setlength{\hangindent}{56pt}{张苍\hspace{30pt}无有此事。({\akai 或}:~呃,噤声。)}

\setlength{\hangindent}{56pt}{陈平\hspace{30pt}走走走,面见国太。({\akai 或}:~哼,走走走!)}

\setlength{\hangindent}{56pt}{(张苍\hspace{30pt}哪里去呃。)}

\setlength{\hangindent}{56pt}{(陈平\hspace{30pt}去见国太辩理。)}

\setlength{\hangindent}{56pt}{(陈平{\hwfs 右手拉}张苍{\hwfs 右手},陈{\hwfs 左袖搭在}张{\hwfs 右手里侧},张{\hwfs 左手摇阻介})}

\setlength{\hangindent}{56pt}{张苍\hspace{30pt}去不得。({\akai 或}:~使不得,使不得。)}

\setlength{\hangindent}{56pt}{陈平\hspace{30pt}走走走。}

\setlength{\hangindent}{56pt}{张苍\hspace{30pt}我有忏悔呀。({\akai 或}:~慢来慢来,我有忏悔啊。)}

\setlength{\hangindent}{56pt}{陈平\hspace{30pt}(哼,)我看你的忏悔,我看你(是)冒高不冒高。}

\setlength{\hangindent}{56pt}{(陈平{\hwfs 坐大边虎头椅},张苍{\hwfs 小边背供}{\akai 念}\footnote{《京剧新序》原文作``陈平{\hwfs 坐小边虎头椅},张苍{\hwfs 大边背供}{\akai 念}''可能有误。陈超老师介绍:~刘曾复先生教授的此处台上的``地方''是~{陈平{\hwfs 坐大边虎头椅},张苍{\hwfs 小边背供}{\akai 念},此处从陈超老师的建议。}})}

\setlength{\hangindent}{56pt}{张苍\hspace{30pt}嗨,有了宗卷,扬长而去,岂不是好,我冒的什么高,冒出祸来了。也罢,上前赔个笑脸也就是了。哈哈哈,啊老相爷,学生冒犯老相爷,这厢作揖赔礼了。({\akai 或}:~哎呀且住,交了宗卷,也就无有事了。我与他冒的什么高,咳,你看你看,冒出祸来了。这这这$\cdots{}\cdots{}$也罢,我上前赔个笑脸可也就拉倒了。哈哈哈,老相爷,方才是下官的不是,得罪了老相爷,我这厢赔礼了。)}

\setlength{\hangindent}{56pt}{(陈平{\hwfs 站},{\hwfs 如前拉}张苍,陈{\hwfs 坐})}

\setlength{\hangindent}{56pt}{陈平\hspace{30pt}唗,你父子私造玉玺,誊写宗卷,你我面见国太。({\akai 或}:~唗,胆大张苍,你父子私抄皇家宗卷。)}

\setlength{\hangindent}{56pt}{(张苍\hspace{30pt}不曾。)}

\setlength{\hangindent}{56pt}{(陈平\hspace{30pt}假造皇家玉玺。)}

\setlength{\hangindent}{56pt}{(张苍\hspace{30pt}哼,无有。)}

\setlength{\hangindent}{56pt}{(陈平\hspace{30pt}有谋篡国太江山之意。)}

\setlength{\hangindent}{56pt}{(张苍\hspace{30pt}噤声。)}

\setlength{\hangindent}{56pt}{(陈平\hspace{30pt}难道说作个揖就罢了不成么?)}

\setlength{\hangindent}{56pt}{张苍\hspace{30pt}去不得。}

\setlength{\hangindent}{56pt}{陈平\hspace{30pt}面见国太。}

\setlength{\hangindent}{56pt}{张苍\hspace{30pt}去不得,我还有大大的忏悔呀!({\akai 或}:~呃呃呃,我还有大大的忏悔呀!)}

\setlength{\hangindent}{56pt}{陈平\hspace{30pt}我看你的大大忏悔,作个揖就罢了不成?({\akai 或}:~哼,我看你的大大忏悔,嗯------只怕是忒轻了罢。)(陈平{\hwfs 坐})}

\setlength{\hangindent}{56pt}{张苍\hspace{30pt}哎呀呀,这个老儿分明是叫我与他下个全礼,我们俱是炎汉忠良,下一全礼又有何妨。啊,哈哈哈,老相爷,学生冒犯老相爷,我这里作揖跪下了。({\akai 或}:~唉,看他之意,是要我与他下一全礼呀。嗯,俱是炎汉忠良,下一全礼又有何妨。啊,老相爷,下官得罪了老相爷,我这厢跪下了。)}

\setlength{\hangindent}{56pt}{(张苍{\hwfs 向}陈平{\hwfs 跪})}

\setlength{\hangindent}{56pt}{陈平\hspace{30pt}下跪何人?}

\setlength{\hangindent}{56pt}{张苍\hspace{30pt}学生张苍。}

\setlength{\hangindent}{56pt}{陈平\hspace{30pt}跪在我的面前做甚呐?({\akai 或}:~跪在老夫的面前做甚?)}

\setlength{\hangindent}{56pt}{张苍\hspace{30pt}(得罪了老相爷,)与老相爷赔礼({\akai 或}:~磕头赔罪)来了。}

\setlength{\hangindent}{56pt}{陈平\hspace{30pt}你怕我不怕?({\akai 或}:~嗯,你怕了老夫不怕?)}

\setlength{\hangindent}{56pt}{张苍\hspace{30pt}(呃,)我怕了老相爷了。}

\setlength{\hangindent}{56pt}{陈平\hspace{30pt}你服(了)我不服?}

\setlength{\hangindent}{56pt}{张苍\hspace{30pt}(我)服了老相爷了。}

\setlength{\hangindent}{56pt}{陈平\hspace{30pt}起来。({\akai 或}:~既然如此,我恕你无罪。)}

\setlength{\hangindent}{56pt}{(张苍\hspace{30pt}多谢老相爷。)}

\setlength{\hangindent}{56pt}{(陈平\hspace{30pt}起来。)}

\setlength{\hangindent}{56pt}{(张苍\hspace{30pt}是。)}

\setlength{\hangindent}{56pt}{(张苍{\hwfs 起})}

\setlength{\hangindent}{56pt}{陈平\hspace{30pt}哼!(张苍{\hwfs 又跪},陈平{\hwfs 扶})}

\setlength{\hangindent}{56pt}{(陈平\hspace{30pt}大人请起。)}

\setlength{\hangindent}{56pt}{张苍\hspace{30pt}这是为何?({\akai 或}:~相爷这是何意呀?)}

\setlength{\hangindent}{56pt}{陈平\hspace{30pt}我与你作耍呢!({\akai 或}:~老夫与你作耍呢!)}

\setlength{\hangindent}{56pt}{张苍\hspace{30pt}(哎呀呀,)耍出汗来了。}

\setlength{\hangindent}{56pt}{陈平\hspace{30pt}有了宗卷,张大人(请)回衙理事。}

\setlength{\hangindent}{56pt}{张苍\hspace{30pt}下官告辞。({\akai 或}:~遵命,告辞了。)}

\setlength{\hangindent}{56pt}{陈平\hspace{30pt}奉送。}

\setlength{\hangindent}{56pt}{张苍\hspace{30pt}【{\akai 西皮摇板}】辞别相爷出府门。}

\setlength{\hangindent}{56pt}{(张苍{\hwfs 出门到外边},陈平{\hwfs 坐小边},张{\hwfs 想介})}

\setlength{\hangindent}{52pt}{张苍\hspace{30pt}不对呀,想这宗卷乃我张苍掌管,陈平老儿苦苦追求,却是为何?这,莫非成皋有人前来盗卷不成。只是何人敢来呢?啊啊是了,昔日老王驾前有一臣子名唤田子春,此人虽则年幼颇有胆识,莫非此人前来盗卷不成?陈平呐陈平,若无此事便罢,如有此事,管教你原礼退回。({\akai 或}:~诶,不对呀,想这宗卷乃是我西台御史掌管,他苦苦地要这宗卷,却是为何?嗯,其中必有缘故。呃,呃,呃$\cdots{}\cdots{}$莫非淮河前来盗卷不成?只是何人竟敢前来?嗯,昔日老王驾下有一臣子名唤田子春,此人虽则年幼,颇有胆识,莫非他前来盗卷不成?陈平呐陈平,若无此事便罢,如有此事么,管教你原礼退还。)(张苍{\hwfs 进门})}

\setlength{\hangindent}{56pt}{张苍\hspace{30pt}({\akai 接唱})再把相爷尊一声。}

\setlength{\hangindent}{56pt}{陈平\hspace{30pt}张大人为何去而复返?}

\setlength{\hangindent}{56pt}{张苍\hspace{30pt}非是下官去而复返,有一事不明要在相爷台前请教哇。}

\setlength{\hangindent}{56pt}{陈平\hspace{30pt}大人请讲。}

\setlength{\hangindent}{56pt}{张苍\hspace{30pt}想皇家宗卷乃是我张苍掌管,相爷苦苦追求,({\akai 或}:~想这宗卷乃是下官掌管,相爷苦苦索要。却是为何?)}

\setlength{\hangindent}{56pt}{(陈平\hspace{30pt}这$\cdots{}\cdots{}$呃$\cdots{}\cdots{}$)}

\setlength{\hangindent}{56pt}{张苍\hspace{30pt}莫非成皋有人前来盗卷?({\akai 或}:~分明是淮河前来盗卷。)}

\setlength{\hangindent}{56pt}{陈平\hspace{30pt}啊无有哇。}

\setlength{\hangindent}{56pt}{张苍\hspace{30pt}就是那田$\cdots{}\cdots{}$({\akai 或}:~就是那田子春。)}

\setlength{\hangindent}{56pt}{陈平\hspace{30pt}噤声!({\akai 或}:~不是$\cdots{}\cdots{}$)}

\setlength{\hangindent}{56pt}{张苍\hspace{30pt}我连人都晓得了。}

\setlength{\hangindent}{56pt}{陈平\hspace{30pt}无有此事。}

\setlength{\hangindent}{56pt}{(张苍{\hwfs 拉}陈平{\hwfs 如上但反向})}

\setlength{\hangindent}{56pt}{张苍\hspace{30pt}哈哈,陈平呐陈平,你犯在我的手内来了,你私通外藩。({\akai 或}:~唗,胆大陈平,私通淮河,隐藏奸细。哼,盗取皇王宗卷,有谋篡国太江山$\cdots{}\cdots{}$)}

\setlength{\hangindent}{56pt}{陈平\hspace{30pt}不曾。({\akai 或}:~呃呃呃$\cdots{}\cdots{}$)}

\setlength{\hangindent}{56pt}{张苍\hspace{30pt}盗取皇家宗卷。({\akai 或}:~走走走。)}

\setlength{\hangindent}{56pt}{陈平\hspace{30pt}无有。({\akai 或}:~哪里去?)}

\setlength{\hangindent}{56pt}{张苍\hspace{30pt}你我面见国太。({\akai 或}:~面见国太。)}

\setlength{\hangindent}{56pt}{陈平\hspace{30pt}使不得,我也有忏悔。({\akai 或}:~使不得,使不得,呃,老夫我也有忏悔。)}

\setlength{\hangindent}{56pt}{张苍\hspace{30pt}(嗯,)我也(要)看看你的忏悔呀。(张苍{\hwfs 坐大边外边})}

\setlength{\hangindent}{56pt}{陈平\hspace{30pt}糟糕哇糟糕,有了宗卷,平安无事,与他耍的什么,耍出祸来了。我也与他赔个礼儿可也就拉倒了。哈哈哈,张大人,俱是老朽的不是,我这厢与大人赔礼了。({\akai 或}:~唉,有了宗卷,平安无事啊,我与他耍的什么。唉,耍出祸来了。这这这$\cdots{}\cdots{}$唉,我也与他赔个笑脸可也就是了。哈哈哈,张大人,方才是老朽的不是,喏喏喏,我这厢赔礼了。)}

\setlength{\hangindent}{56pt}{(张苍{\hwfs 站},{\hwfs 拉}陈平)}

\setlength{\hangindent}{56pt}{张苍\hspace{30pt}唗,你私通外藩?({\akai 或}:~唗,你这老儿,私通淮河。)}

\setlength{\hangindent}{56pt}{陈平\hspace{30pt}不曾。({\akai 或}:~无有。)}

\setlength{\hangindent}{56pt}{张苍\hspace{30pt}盗取皇家宗卷。({\akai 或}:~隐藏奸细。)}

\setlength{\hangindent}{56pt}{陈平\hspace{30pt}无有。({\akai 或}:~不曾。)}

\setlength{\hangindent}{56pt}{张苍\hspace{30pt}谋篡国太江山,走走走,你我去见国太。({\akai 或}:~盗取皇王宗卷,谋篡国太的江山$\cdots{}\cdots{}$)}

\setlength{\hangindent}{56pt}{(陈平\hspace{30pt}噤声。)}

\setlength{\hangindent}{56pt}{(张苍\hspace{30pt}作个揖,哼,就算了不成?)}

\setlength{\hangindent}{56pt}{陈平\hspace{30pt}使不得,我也有大大的忏悔呀。({\akai 或}:~呃,我还有大大的忏悔。)}

\setlength{\hangindent}{56pt}{(张苍{\hwfs 放手})

\setlength{\hangindent}{56pt}{张苍\hspace{30pt}我也看看你的忏悔,作个揖就罢了不成,我还不够本呐,你也来罢。({\akai 或}:~我看你的大大的忏悔。呃,作个揖就算了么,只怕是不够本罢。)}

\setlength{\hangindent}{56pt}{(张苍{\hwfs 坐大边外边})}

\setlength{\hangindent}{56pt}{陈平\hspace{30pt}看他是叫我原礼退还呐。哎,俱是炎汉忠良,与他下一全礼又待何妨。哈哈哈,张大人呐,老朽的不是,我这里与大人磕头赔礼了。({\akai 或}:~呵呵呵,慢来慢来,看他之意是教我原礼退还呐。哎,俱是炎汉忠良,与他磕上一个头是又有何妨。哈哈哈,啊张大人,俱是老朽的不是,我这里与大人磕头赔礼了。)}

\setlength{\hangindent}{56pt}{张苍\hspace{30pt}下跪何人?}

\setlength{\hangindent}{56pt}{陈平\hspace{30pt}陈平,}

\setlength{\hangindent}{56pt}{张苍\hspace{30pt}跪在你张大人的面前({\akai 或}:~跟前)做甚呐?}

\setlength{\hangindent}{56pt}{陈平\hspace{30pt}与张大人磕头赔礼({\akai 或}:~与张大人赔罪)来了。}

\setlength{\hangindent}{56pt}{张苍\hspace{30pt}(嗯,)你怕我不怕?}

\setlength{\hangindent}{56pt}{陈平\hspace{30pt}(呃,)我怕了你了。}

\setlength{\hangindent}{56pt}{张苍\hspace{30pt}服我不服?}

\setlength{\hangindent}{56pt}{陈平\hspace{30pt}(呃,)服了你了。}

\setlength{\hangindent}{56pt}{张苍\hspace{30pt}看你偌大年纪,对着张大人的靴尖,磕上一个响头,也就是了。({\akai 或}:~哼,也罢,看在你偌大的年纪,对着我这靴尖尖,磕上一个响头,饶恕于你。)}

\setlength{\hangindent}{56pt}{陈平\hspace{30pt}(哦,)是是是。(陈平磕头)}

\setlength{\hangindent}{56pt}{张苍\hspace{30pt}起来。}

\setlength{\hangindent}{56pt}{陈平\hspace{30pt}谢大人。}

\setlength{\hangindent}{56pt}{(张苍{\hwfs ``嗯''介})}

\setlength{\hangindent}{56pt}{张苍\hspace{30pt}呵呵呵!}

\setlength{\hangindent}{56pt}{(陈平{\hwfs 又跪},张苍{\hwfs 搀})}

\setlength{\hangindent}{56pt}{张苍\hspace{30pt}相爷(快快)请起。}

\setlength{\hangindent}{56pt}{陈平\hspace{30pt}你这是做甚呐?({\akai 或}:~你这是为何啊?)}

\setlength{\hangindent}{56pt}{张苍\hspace{30pt}我也是作耍呢。}

\setlength{\hangindent}{56pt}{陈平\hspace{30pt}哎呀,你把我耍糊涂了({\akai 或}:~吓糊涂了)。}

\setlength{\hangindent}{56pt}{(张苍\hspace{30pt}相爷恕罪。)}

\setlength{\hangindent}{56pt}{(陈平\hspace{30pt}岂敢。)}

\setlength{\hangindent}{56pt}{张苍\hspace{30pt}可有此事?}

\setlength{\hangindent}{56pt}{陈平\hspace{30pt}大人怎样知晓?({\akai 或}:~正是田子春前来盗卷,你是怎样晓得的?)}

\setlength{\hangindent}{56pt}{张苍\hspace{30pt}被我糊里糊涂的蒙了出来。({\akai 或}:~哼哼,被我糊里糊涂的蒙出来了。)}

\setlength{\hangindent}{56pt}{陈平\hspace{30pt}倒被他蒙了去了。({\akai 或}:~哎呀呀,倒被他糊里糊涂的蒙了出来。)}

\setlength{\hangindent}{56pt}{张苍\hspace{30pt}田大人在此何不请来相见?}

\setlength{\hangindent}{56pt}{陈平\hspace{30pt}大人稍待。}

\setlength{\hangindent}{56pt}{({\hwfs 出门})\hspace{30pt}}

\setlength{\hangindent}{56pt}{陈平\hspace{30pt}田大人有请。({\akai 或}:~有请田大人。)}

\setlength{\hangindent}{56pt}{(田子春{\hwfs 上小边站})}

\setlength{\hangindent}{56pt}{田子春\hspace{20pt}相爷何事?}

\setlength{\hangindent}{56pt}{陈平\hspace{30pt}张苍大人请来相见。}

\setlength{\hangindent}{56pt}{田子春\hspace{20pt}待我相见,张大人哪里?}

\setlength{\hangindent}{56pt}{张苍\hspace{30pt}拿奸细,拿奸细!}

\setlength{\hangindent}{56pt}{(张苍{\hwfs 叫},陈平{\hwfs 拦}、田子春{\hwfs 回身到里面},张{\hwfs 同时回身到里面},{\hwfs 再叫},陈{\hwfs 拦},田、张{\hwfs 又回向外面},陈{\hwfs 中})}

\setlength{\hangindent}{56pt}{陈平\hspace{30pt}这是为何?({\akai 或}:~张大人这做什么?)}

\setlength{\hangindent}{56pt}{张苍\hspace{30pt}试试他的胆量如何?}

\setlength{\hangindent}{56pt}{陈平\hspace{30pt}他的胆量是好的。请坐。}

\setlength{\hangindent}{56pt}{(田子春{\hwfs 中间},陈平{\hwfs 大边},张苍{\hwfs 小边},陈{\hwfs 给}田{\hwfs 卷})}

\setlength{\hangindent}{56pt}{陈平\hspace{30pt}(现有)宗卷在此(,大人请看)。}

\setlength{\hangindent}{56pt}{田子春\hspace{20pt}宗卷已被吕后焚化,这是哪里来的?}

\setlength{\hangindent}{56pt}{(陈平\hspace{30pt}这金殿之上,国太用火焚化的乃是假的。)}

\setlength{\hangindent}{56pt}{(陈平\hspace{30pt}此乃是真的宗卷。)}

\setlength{\hangindent}{56pt}{(陈平\hspace{30pt}大人有所不知,只因癸未年间张大人染病在床,命张大人令郎代守宗卷,是他看到宗卷第七部第七篇,观见襄宫赵娘娘死得可惨,犹恐日后有变,为此抄写一部以防后患。金殿之上用火焚化的乃是假的。这才是真的宗卷。)\footnote{刘曾复先生说戏录音中这段词句由陈平念。}}

\setlength{\hangindent}{56pt}{陈平\hspace{30pt}张大人请讲。}

\setlength{\hangindent}{56pt}{张苍\hspace{30pt}田大人哪里知道,下官癸未年间染病在床,命我儿秀玉代守宗卷,是他看到第七部第七篇,见襄宫赵娘娘死得可惨,犹恐日后有变,为此誊写一部以防后患。}

\setlength{\hangindent}{56pt}{田子春\hspace{20pt}只是有一桩不合律。}

%陈平\\张苍\hspace{30pt}{\raisebox{5pt}{哪一桩不合律?}
\raisebox{0pt}[22pt][16pt]{\raisebox{8pt}{陈平}\raisebox{-8pt}{\hspace{-22pt}{张苍}}\raisebox{0pt}{\hspace{30pt}哪一桩不合律?}}

\setlength{\hangindent}{56pt}{田子春\hspace{20pt}皇家玉玺哪里来的?}

\setlength{\hangindent}{56pt}{(陈平\hspace{30pt}也是张大人令郎一时聪明,用黄蜡雕成玉玺,真的上面原有一颗,假的上面也打上它一颗,与它个真假难辨。)}

\setlength{\hangindent}{56pt}{张苍\hspace{30pt}也是我儿一时聪明,用黄蜡雕成玉玺,真的上面原有一颗,假的上面也打上一颗,与它真假难辨。}

\setlength{\hangindent}{56pt}{田子春\hspace{20pt}金殿之上用火焚化的?}

%陈平\\张苍\hspace{30pt}{\raisebox{5pt}{乃是假的。}
\raisebox{0pt}[22pt][16pt]{\raisebox{8pt}{陈平}\raisebox{-8pt}{\hspace{-22pt}{张苍}}\raisebox{0pt}{\hspace{30pt}乃是假的。}}

\setlength{\hangindent}{56pt}{田子春\hspace{20pt}这呢?}

%陈平\\张苍\hspace{30pt}{\raisebox{5pt}{这是({\akai 或}:~此乃是)历代历代的老宗卷。}
\raisebox{0pt}[22pt][16pt]{\raisebox{8pt}{陈平}\raisebox{-8pt}{\hspace{-22pt}{张苍}}\raisebox{0pt}{\hspace{30pt}这是({\akai 或}:~此乃是)历代历代的老宗卷。}}

\setlength{\hangindent}{56pt}{田子春\hspace{20pt}啊,老宗卷?}

%陈平\\张苍\hspace{30pt}{\raisebox{5pt}{老谱头。}
\raisebox{0pt}[22pt][16pt]{\raisebox{8pt}{陈平}\raisebox{-8pt}{\hspace{-22pt}{张苍}}\raisebox{0pt}{\hspace{30pt}老谱头。}}

\setlength{\hangindent}{56pt}{田子春\hspace{20pt}老谱头(三人笑)。张大人,令郎多大年纪?}

\setlength{\hangindent}{56pt}{张苍\hspace{30pt}我那儿子呀,今年一十六岁了。({\akai 或}:~今年么一十七岁了。)}

\setlength{\hangindent}{56pt}{田子春\hspace{20pt}日后定是出将入相之材。}

\setlength{\hangindent}{56pt}{张苍\hspace{30pt}我那个儿子呀,日后定有他$\cdots{}\cdots{}$({\akai 或}:~我的儿子啊,日后一定像他$\cdots{}\cdots{}$)}

\setlength{\hangindent}{56pt}{陈平\hspace{30pt}你又来了。}

\setlength{\hangindent}{56pt}{张苍\hspace{30pt}取笑了。}

\setlength{\hangindent}{56pt}{田子春\hspace{20pt}有了宗卷,事不宜迟,下官急转淮河,这有淮南王书信,日后发兵到来,还望二位大人做一内应。}

%张苍\\陈平\hspace{30pt}{\raisebox{5pt}{那个自然。}
\raisebox{0pt}[22pt][16pt]{\raisebox{8pt}{陈平}\raisebox{-8pt}{\hspace{-22pt}{张苍}}\raisebox{0pt}{\hspace{30pt}那个自然。}}

\setlength{\hangindent}{56pt}{田子春\hspace{20pt}告辞。}

\setlength{\hangindent}{56pt}{(田子春{\hwfs 走},张苍、陈平{\hwfs 出门送},陈{\hwfs 外}、张{\hwfs 里},\textless{}\!{\bfseries\akai 尾声}\!{\akai 前段}\!\textgreater{})}

%张苍\\陈平\hspace{30pt}{\raisebox{5pt}{奉送。}
\raisebox{0pt}[22pt][16pt]{\raisebox{8pt}{陈平}\raisebox{-8pt}{\hspace{-22pt}{张苍}}\raisebox{0pt}{\hspace{30pt}奉送。}}

\setlength{\hangindent}{56pt}{(田子春{\hwfs 下},张苍、陈平{\hwfs 进门},张{\hwfs 大边},陈{\hwfs 小边})}

\setlength{\hangindent}{56pt}{张苍\hspace{30pt}告辞。}

\setlength{\hangindent}{56pt}{陈平\hspace{30pt}慢来,后堂有酒,大家吃个太平饮宴。({\akai 或}:~后面备得有酒,大家饮上它一个太平宴。)}

\setlength{\hangindent}{56pt}{(张苍\hspace{30pt}到此就要讨扰。还要划拳。)}

\setlength{\hangindent}{56pt}{(陈平\hspace{30pt}如此七巧------)}

\setlength{\hangindent}{56pt}{(张苍\hspace{30pt}八马------)}

\setlength{\hangindent}{56pt}{(陈平\hspace{30pt}请呐------呵呵哈哈哈({\hwfs 笑介}))}

\setlength{\hangindent}{56pt}{张苍\hspace{30pt}慢来慢来,老相爷的酒,我是不能吃呀!}

\setlength{\hangindent}{56pt}{陈平\hspace{30pt}哪个吃得?}

\setlength{\hangindent}{56pt}{张苍\hspace{30pt}炎汉忠良方能吃得。}

\setlength{\hangindent}{56pt}{陈平\hspace{30pt}哎呀,取笑了,大人请。}

\setlength{\hangindent}{56pt}{张苍\hspace{30pt}相爷请。({\hwfs 同笑})}

\setlength{\hangindent}{56pt}{(张{\hwfs 先下},陈{\hwfs 后下}。\textless{}\!{\bfseries\akai 尾声}\!{\akai 合头}\!\textgreater{})}
}
