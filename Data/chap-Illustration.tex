\newpage
\phantomsection %实现目录的正确跳转
\setcounter{page}{0}
\pagenumbering{roman}
%\hypertarget{ux8bf4-ux660e}{
\addcontentsline{toc}{section}{\hei 说~~~~~~明}
	\section*{\hei \large 说\hspace{35pt}明}\label{ux8bf4-ux660e}%}
\pagestyle{fancy}    %与文献引用超链接style有冲突
\chead{说~明} % 页眉中间位置内容

\setlength{\parindent}{24pt}{     %
	此为个人整理的刘曾复教授说戏录音的文本稿,\textbf{主要根据刘曾复先生为中国戏曲学院提供的百余出说戏录音为底本,并结合刘曾老在其余场合的说戏录音}\upcite{Liu-Shuoxi-Record}%\textsuperscript{{[}1{]}}
\textbf{整理完成的}。其中《太平桥》、《盗宗卷》、《梅龙镇》、《辕门斩子》、《摘缨会》、《上天台》、《一捧雪》、《卖马》、《南阳关》的``总讲本''主要依据《京剧新序》\upcite{Liu_Xinxu-I,Liu_Xinxu-II}%\textsuperscript{{[}2{]}.}
中收录的刘曾复先生整理的剧本并结合说戏录音整理完成;《马鞍山》、《战长沙》的``总讲本''则参考了李舒先生遗作《涉艺所得》\upcite{Li-SheyiSuode}%\textsuperscript{{[}3{]}.}
收录的刘曾复先生手书稿和传本并结合说戏录音整理完成的。\textbf{有关剧目中的把子,主要摘录自}《京剧新序》和《京剧老生把子见闻录》\upcite{XQYS1-32_1983}%\textsuperscript{{[}4{]}.}
一文记录的开打和舞台调度。

除了上述《太平桥》等十一出剧目,其余剧目的场次安排主要参考了《京剧汇编~(1-109集)》\upcite{Jingju-Huibian-1}%\textsuperscript{{[}5{]}.}
、《传统剧目汇编》\upcite{Jingju-Huibian-2}%\textsuperscript{{[}6{]}.}
、《京剧丛刊~(1-50集)》\upcite{Jingju-Congkan}%\textsuperscript{{[}7{]}.}
和``中国京剧戏考''网站\upcite{PekingOpera-Scripts}%\textsuperscript{{[}8{]}.}
上的相应的剧目的安排,个别剧目的词句也参考了,``中国京剧老唱片''网站\upcite{PekingOpera-OldRecords}%\textsuperscript{{[}9{]}.}
上载的老唱片戏词。

剧目按照剧中人物年代排列,部分剧目的年代排序参考了《京剧大戏考》\upcite{Chai-DaXikao}%\textsuperscript{{[}10{]}.}
和《京剧知识词典(增订版)》\upcite{PekingOpera-Dictionary}%\textsuperscript{{[}11{]}.}
中的剧目顺序。

\vskip 5pt
基于全面、客观、忠实的记录原则,整理剧目文字的标记说明如下:
\begin{enumerate}
\def\labelenumi{\arabic{enumi}.}
\item
	{\CJKfamily{hei}因为本人学识浅陋、加之录音带存年较久,因此文字中有不少存疑处。凡是存疑处,尽量用\textcolor{red}{红色字体}标出,}表明此处可能文辞欠通顺,或只是根据字音听写臆测的词句;
\item
	{\CJKfamily{hei}刘曾复先生腹笥渊博,在不同的场合说戏时,即使是同一出戏,个别词句也略有出入,文本中尽量作了标注:~}

\begin{enumerate}
\def\labelenumi{\arabic{enumi}.}
\item
  每个剧目中凡有出入的唱、念词句标注为:

\begin{quote}
	\underline{\textrm{XX}词1}~({\akai 或}:~\textrm{XX}词2;~\textrm{XX}词3;$\cdots{}\cdots${})
\end{quote}
\begin{quote}
	\underline{\textrm{XX}句1}~({\akai 或}:~\textrm{XX}句2~{\akai 或}:~\textrm{XX}句3;$\cdots{}\cdots${})
\end{quote}

\def\labelenumi{\arabic{enumi}.}
\setcounter{enumi}{1}
\item
  每个剧目中可不念或某些衬字的唱、念标注为:

\begin{quote}
	(\textrm{XX}词句)
\end{quote}
\end{enumerate}

\def\labelenumi{\arabic{enumi}.}
\setcounter{enumi}{2}
\item
  \textbf{除``总讲本''外,``单词本''中,与表演配合的其他人物唱、念(盖口)标记为}:
\begin{quote}
	(人物\hspace{30pt} 唱、念词句\textrm{XXX}。)
\end{quote}
\item
  \textbf{在本人的知识范围内,对一些生僻的典故、词汇作了简要的注解。}
\item
  \textbf{刘曾复先生对唱、念中的虚词(衬字、垫字或语气词)非常重视,但文本中仅对极少部分虚词用小字号字体作了标注,挂一漏万。唱、念中虚词的使用,建议读者以先生的录音为准。}
\item
  \textbf{由于文字记录的功能有限,此书辑录的主要是说戏的文字内容,关于舞台表演过程中的唱、念的明细要求,本文都没有标注。}
\end{enumerate}
}
%----------------------------------------------------------------------------------------------------------------------------------------------------%
