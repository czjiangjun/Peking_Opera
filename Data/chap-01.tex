\addcontentsline{toc}{section}{\hfill[\hei 先秦·两汉]\hfill}
\chead{先秦·两汉} % 页眉中间位置内容
\setcounter{page}{1}

%\newpage
%\hypertarget{ux6e2dux6c34ux6cb3}{%
\subsubsection{\hei\large 渭水河}%\label{ux6e2dux6c34ux6cb3}}
\addcontentsline{toc}{subsection}{\hei 渭水河}

\hangafter=1                   %2. 设置从第1⾏之后开始悬挂缩进  %
\setlength{\parindent}{0pt}{
{\centerline{\textrm{{[}\hei 第一场{]}}}}
\vspace{5pt}
姬昌\hspace{30pt}~ {[}{\akai 引子}{]}为建帝基,一路平安,到西岐。

姬昌\hspace{30pt}~ ({\akai 念})纣王无道宠妲己,苦害忠良受凌逼。孤王回转西岐地,重整山河统华夷。

\setlength{\hangindent}{60pt}   %3. 设置悬挂缩进量                %
{姬昌\hspace{30pt}~ 孤,西伯侯姬昌。只因纣王无道,信宠妲己,苦害忠良,是孤回转西岐,自立基业。那日有一樵夫,名叫武吉,将孤门军\footnote{刘曾复先生说戏录音作``军门'',似非,此处从《京剧汇编》~第十三集陈少武、苏连汉~口述本。}%\protect\hyperlink{fn3}{\textsuperscript{3}}
打死,拿他问罪。他言道:~家有八旬老母,无人侍奉。孤王念他是一孝子,赐他斗米贯钱,限定七日前来抵罪,去了数十余天不见到来。我不免在八卦之中,查看吉凶。}

姬昌\hspace{30pt}~ 内侍,

内侍\hspace{30pt}~ 有。

姬昌\hspace{30pt}~ 香案伺候。

内侍\hspace{30pt}~ 香案伺候哇。

姬昌\hspace{30pt}~ 哎呀!

\setlength{\hangindent}{60pt}   %3. 设置悬挂缩进量                %
{姬昌\hspace{30pt}~ 【{\akai 二黄原板}】摇动了金钱告上苍,八卦之中显示明详\footnote{《京剧汇编》第十三集~陈少武、苏连汉~口述本~作``明祥''。}%\protect\hyperlink{fn4}{\textsuperscript{4}}
。单见单来仄见仄,查不出小武吉落于何方。}

姬昌\hspace{30pt}~ ({\akai 念})春有寅萌芽出土,夏有寅火炼金身。秋有寅黄叶落地,冬有寅滴水成冰。

\setlength{\hangindent}{60pt}   %3. 设置悬挂缩进量                %
{姬昌\hspace{30pt}~ 呜哙呀,我道此人还在,原来入土而亡。他今一死不大紧要,可叹他八旬老母,无人侍奉。唉,可叹呐可叹。}

姬昌\hspace{30pt}~ 回避了。

\setlength{\hangindent}{60pt}   %3. 设置悬挂缩进量                %
{姬昌\hspace{30pt}~ 【{\akai 二黄原板}】孤王建业在西方,只为江山昼夜忙。东路反了姜文焕,南路鄂广反陈塘。他两家俱有那书信来往,叫孤王领人马去反商王。臣反君来小犯上,倒不如稳坐西岐乐安康。移步儿\footnote{``移步儿''吴小如先生建议作``一步儿'',此处从《京剧汇编》第十三集~陈少武、苏连汉~口述本。}%\protect\hyperlink{fn5}{\textsuperscript{5}}
来至在灵台上,且做南柯梦一场。}

姬昌\hspace{30pt}~ 【{\akai 二黄导板}】孤王正在睡朦胧,

	姬昌\hspace{30pt}~ 【{\akai 二黄摇板}】只见飞熊扑帐中。手执宝剑将你斩,化阵清风无影踪。

内侍\hspace{30pt}~ 千岁醒来。

姬昌\hspace{30pt}~ ~【{\akai 二黄导板}】适才朦胧见一怪,

	姬昌\hspace{30pt}~ 【{\akai 二黄摇板}】醒来依然在灵台。

姬昌\hspace{30pt}~ 内侍。

内侍\hspace{30pt}~ 有。

姬昌\hspace{30pt}~ 宣散宜生上殿。

内侍\hspace{30pt}~ 领旨。散宜生灵台见驾。

散宜生\hspace{20pt}~ 领旨。

散宜生\hspace{20pt}~ ({\akai 念})袖里乾坤大,怀揣日月明。

散宜生 \hspace{20pt}~散宜生见驾,主公千岁。

姬昌\hspace{30pt}~ 平身。

散宜生\hspace{20pt}~ 千千岁。

姬昌\hspace{30pt}~ 赐坐。

散宜生\hspace{20pt}~ 谢坐。宣臣来见,有何圣谕?

姬昌\hspace{30pt}~ 孤王三更时分,梦一飞熊入帐,抓伤孤的左膀,不知主何吉凶?

散宜生\hspace{20pt}~ 这$\cdots{}\cdots{}$此乃大吉之兆。

姬昌\hspace{30pt}~ 怎见得?

散宜生\hspace{20pt}~ 主公传旨,郊外射猎,不得虎臣,必得良将。

姬昌\hspace{30pt}~ 先生替孤传旨。

散宜生\hspace{20pt}~ 领旨。

姬昌\hspace{30pt}~ 正是:~({\akai 念})夜梦飞熊入帐来,

	散宜生\hspace{20pt}~ ({\akai 念})郊外射猎访贤才。

\vspace{3pt}{\centerline{\textrm{{[}{\hei 第二场}{]}}}}\vspace{5pt}

武吉\hspace{30pt}~ 嗯哼!

武吉\hspace{30pt}~ ({\akai 念})胆小天下去得,刚强寸步难行。

\setlength{\hangindent}{60pt}   %3. 设置悬挂缩进量                %
{武吉\hspace{30pt}~ 小子武吉。自从那日上山砍樵,进城去卖,偶遇姬千岁门军,被我失手打死。那姬千岁拿我问罪。我曾言道:~家有八旬老母,无人侍奉。那姬千岁念我是一孝子,赏我斗米贯钱,回家见母一面,限定七日前来抵罪。不想行至渭水河边,见一老者,呃,在那里垂钓,他见我面带煞气,必有凶事。是我将打死门军之事,对他实言。他教我一个法术:~回到家中,老母床前,挖一土井,宽要七尺,深要丈二,口含灯芯、糯米,睡在井内。躲过七七四十九日,方保无事。今当四十八天,老母腹中饥饿,只得将我唤醒,命我上山砍樵,卖了钱文,买米度日。}

武吉\hspace{30pt}~ 正是:~({\akai 念})上山擒虎易,开口告人难。

\vspace{3pt}{\centerline{\textrm{{[}{\hei 第三场}{]}}}}\vspace{5pt}

	南宫适\hspace{20pt}~ \textless{}\!{\bfseries\akai 点绛唇}\!\textgreater{}扶保西岐,

北宫高\hspace{20pt}~  \textless{}\!{\bfseries\akai 点绛唇}\!\textgreater{}同心协力,

辛甲\hspace{30pt}~  \textless{}\!{\bfseries\akai 点绛唇}\!\textgreater{}立帝基,

辛免\hspace{30pt}~  \textless{}\!{\bfseries\akai 点绛唇}\!\textgreater{}四海归一,

众\hspace{41pt}~  \textless{}\!{\bfseries\akai 点绛唇}\!\textgreater{}方显英雄气。

南宫适\hspace{20pt}~ 俺,南宫适,

北宫高\hspace{20pt}~ 北宫高,

辛甲\hspace{30pt}~ 辛甲,

辛免\hspace{30pt}~ 辛免。

南宫适\hspace{20pt}~ 众位将军请了。

众\hspace{41pt}~ 请了。

南宫适\hspace{20pt}~ 主公郊外射猎,两厢伺候。

众\hspace{41pt}~ 请。

姬昌\hspace{30pt}~ ({\akai 念})旌旗遮日月,郊外访贤臣。

众\hspace{41pt}~ 参见主公。

姬昌\hspace{30pt}~ 人马可齐?

众\hspace{41pt}~ 俱已齐备。

姬昌\hspace{30pt}~ 郊外去者!

众\hspace{41pt}~ 得令。

众\hspace{41pt}~ 众将官,郊外去者。带马!

杂\hspace{41pt}~ 啊。

众\hspace{41pt}~ 前道为何不行?

杂\hspace{41pt}~ 樵夫挡道。

众\hspace{41pt}~ 人马列开!

杂\hspace{41pt}~ 樵夫当面。

武吉\hspace{30pt}~ 樵夫叩头。

姬昌\hspace{30pt}~ 下跪可是武吉?

武吉\hspace{30pt}~ 正是。

姬昌\hspace{30pt}~ 见了孤王为何不抬起头来?

武吉\hspace{30pt}~ 有罪不敢抬头。

姬昌\hspace{30pt}~ 恕你无罪。

武吉\hspace{30pt}~ 谢千岁。

姬昌\hspace{30pt}~ 呃嗯------限定七日前来抵罪。为何今日才来见孤,该当何罪?

\setlength{\hangindent}{60pt}   %3. 设置悬挂缩进量                %
{武吉\hspace{30pt}~ 千岁容禀:~那日多蒙千岁天恩,放小子回家,见母一面,不想行至渭水河边,见一老者,在那厢垂钓,他见我面带煞气,必有凶事。我将打死门军之事对他言明。他教我小小法术:~回到家去,老母床前,挖一土井,宽要七尺,深要丈二,口含灯芯、糯米,睡在其内。躲过七七四十九日,方保无事。今乃四十八日,我母不解其意,将我唤醒,命我上山砍柴,进城去卖,不想撞了\footnote{《京剧汇编》第十三集~陈少武、苏连汉~口述本~作``闯了''。}%\protect\hyperlink{fn6}{\textsuperscript{6}}
千岁御驾。唉,也是小人命该如此,情愿抵罪。}

姬昌\hspace{30pt}~ 可曾问过那渔人的名姓?

武吉\hspace{30pt}~ 这,不曾问得。

姬昌\hspace{30pt}~ 垂钓所在?

武吉\hspace{30pt}~ 渭水河边。

姬昌\hspace{30pt}~ 将柴担放下,引孤前往。

{\akai 二}太子\hspace{20pt}~ 且慢。启奏父王:~为国访贤,必须换了便服。

姬昌\hspace{30pt}~ 看衣改换。

众\hspace{41pt}~ 撒下围场。

姬昌\hspace{30pt}~ 武吉。

武吉\hspace{30pt}~ 有。

姬昌\hspace{30pt}~ 带路。

武吉\hspace{30pt}~ 领旨。

\setlength{\hangindent}{60pt}   %3. 设置悬挂缩进量                %
{姬昌\hspace{30pt}~ 【{\akai 二黄原板}】樵夫道那渔人颇有奥妙,因此上带皇儿亲走一遭。叫武吉与孤向前引道,青的山绿是水难画难描。}

\vspace{3pt}{\centerline{\textrm{{[}{\hei 第四场}{]}}}}\vspace{5pt}

\setlength{\hangindent}{60pt}   %3. 设置悬挂缩进量                %
{姜尚\hspace{30pt}~ 【{\akai 二黄原板}】纣王无道贪色酒,午门外修一座摘星楼。比干丞相遭毒手,贾氏夫人坠高楼。(我不愿在朝为官侯,隐居磻溪垂钓钩。\footnote{据段公平{\scriptsize 君}告知,原有此二句,可不唱。}%\protect\hyperlink{fn7}{\textsuperscript{7}}
) 昨夜晚在土台观看星斗,算就了西伯侯灭纣兴周。移步儿来至在渭水河口,那一边来的是西伯王侯。}

姬昌\hspace{30pt}~ ({\akai 内})武吉带路。

\setlength{\hangindent}{60pt}   %3. 设置悬挂缩进量                %
{姬昌\hspace{30pt}~ 【{\akai 二黄原板}】太平鸟不住当头叫,叫得孤王喜眉梢。土台上一道长稳坐垂钓,手执鱼竿自在逍遥。我观他倒有那仙家奥妙,想必是他腹中藏有略韬。叫武吉你与我暂且退了,二皇儿向前去细问根苗。}

{\akai 二}太子\hspace{20pt}~ 啊,那渔人请了。

姜尚\hspace{30pt}~ ({\akai 念})钓、钓、钓,大鱼不到小鱼到。用你不着,去吧。

{\akai 二}太子\hspace{20pt}~ 好生与你见礼,为何佯装不睬?

姜尚\hspace{30pt}~ ({\akai 念})从来未识金龙面,要见明君便开言。

{\akai 二}太子\hspace{20pt}~ 好大的口气。

{\akai 二}太子\hspace{20pt}~ 啊,父王,那渔翁言道:~从来未识金龙面,要见明君便开言。

姬昌\hspace{30pt}~ 呃嗯------敢是尔等轻慢于他?

{\akai 二}太子\hspace{20pt}~ 儿臣不敢。

姬昌\hspace{30pt}~ 待我向前。

姬昌\hspace{30pt}~ 啊,渔人请了。

姜尚\hspace{30pt}~ 贫道稽首。

姬昌\hspace{30pt}~ 口称``稽首'',在哪座名山修炼?

姜尚\hspace{30pt}~ 昆仑学道。稽首为尊。

姬昌\hspace{30pt}~ 在此作甚?

姜尚\hspace{30pt}~ 在此垂钓。

姬昌\hspace{30pt}~ 钓得何鱼?

姜尚\hspace{30pt}~ 西海鳌鱼。

姬昌\hspace{30pt}~ 小小竹竿,焉能钓得鳌鱼?

姜尚\hspace{30pt}~ 竹竿虽小,能坠千斤。

姬昌\hspace{30pt}~ 借来一观。

姜尚\hspace{30pt}~ 请看。

姬昌\hspace{30pt}~ 呜哙呀!为何用直钩垂钓?

姜尚\hspace{30pt}~ 贫道心直性直,故而用直钩垂钓,有道是``愿者上钩''。

姬昌\hspace{30pt}~ 但不知谁愿谁不愿?

姜尚\hspace{30pt}~ ({\akai 念})耐烦等到群鱼到,自有鱼儿来吞钩。

姬昌\hspace{30pt}~ 请问渔人尊姓大名?

姜尚\hspace{30pt}~ 贫道姓姜名尚字子牙。

姬昌\hspace{30pt}~ 道号?

姜尚\hspace{30pt}~ 飞熊。

姬昌\hspace{30pt}~ ``飞熊''!应孤梦兆也。

姜尚\hspace{30pt}~ 来者上姓?

姬昌\hspace{30pt}~ 在下西伯侯姬昌。

姜尚\hspace{30pt}~ 哦,原来是姬千岁,失敬了。

姬昌\hspace{30pt}~ 岂敢。

姜尚\hspace{30pt}~ 不在西岐,来在渭水则甚?

姬昌\hspace{30pt}~ 特请先生,扶保姬氏河山。

姜尚\hspace{30pt}~ 贫道才疏学浅,不敢当此重任。

姬昌\hspace{30pt}~ 先生不必推辞。

姬昌\hspace{30pt}~ 武吉。

武吉\hspace{30pt}~ 在。

姬昌\hspace{30pt}~ 封你开路先锋,准备车辇伺候。

武吉\hspace{30pt}~ 遵命。

武吉\hspace{30pt}~ ({\akai 念})不是姜太公,怎能作先锋。

姬昌\hspace{30pt}~ 请先生转至大营。

姜尚\hspace{30pt}~ 请。

姬昌\hspace{30pt}~ 【{\akai 二黄原板}】久闻先生今相见,

姜尚\hspace{30pt}~ 【{\akai 二黄原板}】姜子牙八十二才遇名贤。

姬昌\hspace{30pt}~ 【{\akai 二黄原板}】孤王江山全仗你,

姜尚\hspace{30pt}~ 【{\akai 二黄原板}】要保江山万万年。

\vspace{3pt}{\centerline{\textrm{{[}{\hei 第五场}{]}}}}\vspace{5pt}

\textrm{武吉\hspace{30pt}~ 启千岁:~车辇齐备。}

\textrm{姬昌\hspace{30pt}~ 请先生登辇。}

\textrm{姜尚\hspace{30pt}~ 贫道有僭了。}

\textrm{姜尚\hspace{30pt}~ 【{\akai 西皮导板}】听号炮三声响旌旗招展,}

\setlength{\hangindent}{60pt}   %3. 设置悬挂缩进量                %
{\textrm{姜尚\hspace{30pt}~ 【{\akai 西皮原板}】满营中众将官齐跨雕鞍。对主公施一礼忙登车辇,姜子牙坐车辇细把君观:~前三皇后五帝年深日远,有尧、舜并禹、汤四大名贤。西伯侯夜得兆飞熊扑面,散宜生奏一本【{\footnotesize 转}{\akai 西皮快板}】郊外访贤。打柴汉名武吉引君来见,姜子牙八十二才遇名贤。行八百单八步住了车辇,}}

\textrm{姜尚\hspace{30pt}~ 【{\akai 西皮摇板}】到后来保周室八百八年。}

\textrm{武吉\hspace{30pt}~ 启千岁:~西岐众百姓,头顶香盘,迎接千岁、先生进城。}

\textrm{姬昌\hspace{30pt}~ 香盘撤去,摆队进城。}

\textrm{武吉\hspace{30pt}~ 香盘撤去,摆队进城。}
}


%\newpage
\subsubsection{{\hei\large 焚烟墩·挡幽}~\protect\footnote{本剧的舞台调度与人物扮相参考了吴焕老师整理的剧本(经刘曾复先生审订)。}}%\protect\hyperlink{fn8}{\textsuperscript{8}}
\addcontentsline{toc}{subsection}{\hei 焚烟墩·挡幽}

\hangafter=1                   %2. 设置从第1⾏之后开始悬挂缩进  %
\setlength{\parindent}{0pt}{
	\textrm{(}【{\akai 大锣长锤}】,{\hwfs 四绿}龙套{\hwfs 手拿月华旗},{\hwfs 两个}旗牌,{\hwfs 上},{\hwfs 站门}\textrm{)}

	(申侯{\hwfs 上},\textrm{``}{\hwfs 九龙口}\textrm{''}{\hwfs 旁},{\hwfs 起唱}\textrm{)}

\textrm{申侯\hspace{30pt}~ 【{\akai 西皮三眼}】周天子行无道难逃我手,}

(申侯{\hwfs 至台中央},{\hwfs 接唱},\textrm{``}{\bfseries\hwfs 扯四门}\textrm{''})

\setlength{\hangindent}{60pt}{   %3. 设置悬挂缩进量                %
\textrm{申侯\hspace{30pt}~ 【{\akai 西皮三眼}】先帝呀爷他封我申地为侯。恨昏王废太子贬了申后,因此上领人马扎至山头({\akai 或}:~扎至山口)。将人马埋伏在三岔路口哇,}
}

\textrm{申侯\hspace{30pt}~ 【{\akai 西皮摇板}】等昏王到此来细问根由。}

\textrm{申侯\hspace{30pt}~ 大路埋伏,小路放起烟火,君妃到此({\akai 或}:~昏王到此),速报我知。}

\textrm{众\hspace{41pt}~ 啊!}

(申侯{\hwfs 下},{\hwfs 四个}龙套{\hwfs 拿月华旗在后场},{\hwfs 拉起成阵墙})

(幽王、褒姒{\hwfs 上})

\textrm{幽王\hspace{30pt}~ 【{\akai 西皮导板}】直杀得气冲天如同白昼,}

\textrm{幽王\hspace{30pt}~ 哎呀({\akai 或}:~罢)!}

\textrm{(幽王{\hwfs 和}褒姒{\hwfs 到小边},幽王{\hwfs 要摔倒},褒姒{\hwfs 搀扶})}

\textrm{褒姒\hspace{30pt}~ 万岁怎么样了?}

\textrm{幽王\hspace{30pt}~ 唉!}

\textrm{(褒姒{\hwfs 搀}幽王,幽王{\hwfs 接唱},{\hwfs 在月华旗前}``}{\bfseries\hwfs 扯四门}\textrm{'')}

\textrm{幽王\hspace{30pt}~ 【{\akai 西皮原板}】遭不幸尸堆山血水倒流。}

\textrm{(幽王{\hwfs 至小边})}

\setlength{\hangindent}{60pt}{   %3. 设置悬挂缩进量                %
\textrm{幽王\hspace{30pt}~ 【{\akai 西皮原板}】千层浪翻身转慈航搭救,王虽然得活命}龙落孤洲\textrm{。皆因是孤有错说不出口,都只为你进宫起祸根由。悔不该听奸臣【{\footnotesize 转}{\akai 西皮快板}】谗言启奏,悔不该把朝事一旦抛丢。悔不该废太子又贬申后,悔不该与国舅结下冤仇。悔不该在骊山【{\footnotesize 转}{\akai 西皮快板}】取乐饮酒,悔不该焚烟墩戏耍诸侯。看起来孤的命------}
}

\textrm{幽王\hspace{30pt}~ 嘿------呀!}

\textrm{幽王\hspace{30pt}~ 【{\akai 西皮摇板}】只怕是断送在你手,}

\textrm{褒姒\hspace{30pt}~ 万岁命丧我手({\akai 或}:~万岁命丧奴手),唉!待我死了罢!}

\textrm{幽王\hspace{30pt}~ 唉呀嚯,舍不得哟! ({\akai 或}:~慢来慢来,还舍不得哟!)}

\textrm{(幽王、褒姒{\hwfs 至小边})}

\setlength{\hangindent}{60pt}{   %3. 设置悬挂缩进量                %
\textrm{幽王\hspace{30pt}~ 【{\akai 西皮摇板}】劝梓童休伤悲且免忧愁。被西戎杀得孤无处逃呃走,三岔口迷了路不能相投。}
}

\textrm{幽王\hspace{30pt}~ 梓童,(你我)失迷路途,待孤探路。}

\textrm{褒姒\hspace{30pt}~ 小心了。}

\textrm{幽王\hspace{30pt}~ (呃,)晓得哟。}

\setlength{\hangindent}{60pt}{   %3. 设置悬挂缩进量                %
\textrm{幽王\hspace{30pt}~ 【{\akai 西皮摇板}】这一阵杀得孤魂飞胆丧,西戎兵一个个四路埋藏。又只见杏黄旗空呃中飘哇荡,倘若是遇仇敌孤命残伤。}
}

\textrm{(幽王{\hwfs 看``申''字大旗})}

\textrm{(幽王\hspace{30pt}~ 啊------)}

\setlength{\hangindent}{60pt}{   %3. 设置悬挂缩进量                %
\textrm{幽王\hspace{30pt}~ 【{\akai 西皮摇板}】上写着是申侯要除无道,拿住了小周王({\akai 或}:~小昏王)定斩不饶。}
}

\textrm{众\hspace{41pt}~ 君王到。}

\textrm{(\textless{}\!{\bfseries\akai 急急风}\!\textgreater{},{\hwfs 四个}龙套{\hwfs 拿月华旗撤到大边},{\hwfs 斜向},{\hwfs 后面摆桌子}、{\hwfs 椅子},申侯{\hwfs 坐桌上},{\hwfs 子午相}。{\hwfs 两个}旗牌{\hwfs 分别站在桌子两旁的椅子上})}

\textrm{申侯\hspace{30pt}~ 【{\akai 西皮导板}】听说是小昏王({\akai 或}:~听说是小周王)君妃驾到({\akai 或}:~君妃来到),}

\setlength{\hangindent}{60pt}{   %3. 设置悬挂缩进量                %
\textrm{申侯\hspace{30pt}~ 【{\akai 西皮原板}】正要他到此来细问根苗。权作了痴呆汉我佯装不哇晓,是何方反贼兵身穿龙袍。}
}

\textrm{幽王\hspace{30pt}~ 贤侯!}

\setlength{\hangindent}{60pt}{   %3. 设置悬挂缩进量                %
\textrm{幽王\hspace{30pt}~ 【{\akai 西皮原板}】尊贤侯你那里未必不晓,孤就是周天子逃难荒郊。皆因是孤无道命运不好,看起来孤倒运还是孤八字不高({\akai 或}:~看起来算是孤八字不高)。}
}

\setlength{\hangindent}{60pt}{   %3. 设置悬挂缩进量                %
\textrm{申侯\hspace{30pt}~ 【{\akai 西皮原板}】你既是周天子福分非小({\akai 或}:~八字非小;八字不小),你就该在宫中快乐逍遥。这是你人背时八字不好({\akai 或}:~命运不好),看起来天有眼报应在今朝。}
}

\setlength{\hangindent}{60pt}{   %3. 设置悬挂缩进量                %
\textrm{幽王\hspace{30pt}~ 【{\akai 西皮原板}】尊贤侯你心下不必计较,孤有言你那里细听根苗:~望贤侯保孤回重重相报,孤封你世代公侯与孤王同掌九朝。}
}

\setlength{\hangindent}{60pt}{   %3. 设置悬挂缩进量                %
\textrm{申侯\hspace{30pt}~ 【{\akai 西皮原板}】听他言不知羞令人呃可笑,我正要保你回同掌九朝。回宫去与奸妃把酒色贪好,你还把朝纲事一旦丢抛。}
}

\textrm{申侯\hspace{30pt}~ 那土台之上,坐一女子,身穿大红,怀抱婴儿,她是何人?}

\setlength{\hangindent}{60pt}{   %3. 设置悬挂缩进量                %
\textrm{幽王\hspace{30pt}~ 那就是褒姒娘娘,怀抱婴儿是孤的爱子名叫伯服。啊,梓童,上得前去端端正正见上一礼,他保你我君妃回朝也未可知。}
}

\textrm{(\textless{}\!{\bfseries\akai 行弦}\!\textgreater{}幽王{\hwfs 示意}褒姒{\hwfs 行礼})}

\textrm{(褒姒\hspace{25pt}~ 奴惧呀。)}

\textrm{(幽王\hspace{25pt}~ 你要去呀。)}\footnote{此句与上句褒姒念的``奴惧呀'',疑是同一句。}%\protect\hyperlink{fn9}{\textsuperscript{9}}

\textrm{(幽王{\hwfs 嗔介})}

\textrm{褒姒\hspace{30pt}~ 是。}

\textrm{褒姒\hspace{30pt}~ 申侯------万福!}

\textrm{申侯\hspace{30pt}~ 好个褒娘娘正宫主{\footnotesize 呃}母({\akai 或}:~正宫国母)!}

\setlength{\hangindent}{60pt}{   %3. 设置悬挂缩进量                %
\textrm{申侯\hspace{30pt}~ 【{\akai 西皮原板}】恨奸妃不行善你又不行好,无端地害吾妹天理自昭。每日呀里与昏王酒色欢好,你可知申正宫她与我一母同胞。}
}

\setlength{\hangindent}{60pt}{   %3. 设置悬挂缩进量                %
\textrm{褒姒\hspace{30pt}~ 【{\akai 西皮原板}】申国母待小奴恩高义好,我岂肯将她人一旦丢抛。望贤侯看薄面把君妃来保,早烧香晚点灯答报恩高。}
}

\setlength{\hangindent}{60pt}{   %3. 设置悬挂缩进量                %
\textrm{申侯\hspace{30pt}~ 【{\akai 西皮原板}】我自然保他回【{\footnotesize 转}{\akai 西皮快板}】看你的金面,为什么废太子上欺青天。叫人来准备下麻绳、铁链,不论君、不论妃锁拿军前。拿住了小昏王({\akai 或}:~拿住了小周王)大功来建,若有人违将令斩罪无宽。}
}

\setlength{\hangindent}{60pt}{   %3. 设置悬挂缩进量                %
\textrm{幽王\hspace{30pt}~ 【{\akai 西皮原板}】你那里要我命不敢强辩,哭干了黄河水难保命还。依然是申国母独掌宫院,我和你郎舅情结什么仇冤。}
}

\setlength{\hangindent}{60pt}{   %3. 设置悬挂缩进量                %
\textrm{申侯\hspace{30pt}~ 【{\akai 西皮摇板}】小昏王({\akai 或}:~教昏王)休得要巧言舌辩,待我将十条罪细表根源:~}
}

\setlength{\hangindent}{60pt}{   %3. 设置悬挂缩进量                %
	\textrm{申侯\hspace{30pt}~ 【{\akai 西皮二六}】一条罪初登基民女挑选,二条罪老王薨作乐喧天。三条罪废国母宫闱混乱,四条罪宠尹球灭忠害贤。五条罪命石父}\footnote{石父即虢石父。}
%\protect\hyperlink{fn10}{\textsuperscript{10}}
\textrm{【{\footnotesize 转}{\akai 西皮快板}】兴兵发难,六条罪把朝纲丢在一边。七条罪弃太子纲常败乱({\akai 或}:~纲常大变),八条罪宠奸妃诡计多端。九条罪在骊山君妃饮宴,十条罪焚烟墩戏耍群贤。你为君十条罪人心涣散,死九泉见先王有何话言。}
}

\setlength{\hangindent}{60pt}{   %3. 设置悬挂缩进量                %
\textrm{幽王\hspace{30pt}~ 【{\akai 西皮摇板}】大不该在骊山作乐饮宴,大不该焚烟墩戏耍群贤。你本是大丈夫英雄好汉,杀了我无用人你算什么能员。}
}

\setlength{\hangindent}{60pt}{   %3. 设置悬挂缩进量                %
\textrm{褒姒\hspace{30pt}~ 【{\akai 西皮摇板}】听他言不由我心中好惨,做一朝人王主这样凄然。眼见得君妃们不能回转,}
}

\textrm{褒姒\hspace{30pt}~ \textless{}\!{\bfseries\akai 哭头}\!\textgreater{}万岁呀!}

\textrm{褒姒\hspace{30pt}~ 【{\akai 西皮摇板}】我和你就死在燃眉之间。}

\textrm{褒姒\hspace{30pt}~ 喂呀,万岁呀\ldots{}\ldots{}({\hwfs 哭介})}

\setlength{\hangindent}{60pt}{   %3. 设置悬挂缩进量                %
\textrm{申侯\hspace{30pt}~ 【{\akai 西皮摇板}】又只见他君妃哭得好惨,杀君王犹如那子杀父般。叫三军({\akai 或}:~叫人来)放开路任他逃散({\akai 或}:~容他逃窜;容他逃散;任他逃窜),落一个忠义名万载流传。}
}

\textrm{幽王\hspace{30pt}~ 梓童,此乃一字长蛇阵,有放你我君妃之意,我们走了罢({\akai 或}:~我们溜了罢)。}

\textrm{褒姒\hspace{30pt}~ (唉,)逃了罢({\akai 或}:~走了罢)。}

\textrm{幽王\hspace{30pt}~ 溜了罢。}

\textrm{(幽王{\hwfs 在大边唱})}

\setlength{\hangindent}{60pt}{   %3. 设置悬挂缩进量                %
\textrm{幽王\hspace{30pt}~ 【{\akai 西皮摇板}】孤回朝({\akai 或}:~孤还朝)有一日登了大宝,捉住了这申侯({\akai 或}:~捉住了小申侯)万剐千刀。}
}

\textrm{幽王\hspace{30pt}~ 我们走了罢。}

\textrm{众\hspace{41pt}~ 幽王逃走。}

\textrm{申侯\hspace{30pt}~ 带马!}

\textrm{(旗牌{\hwfs 下去},{\hwfs 四绿}龙套{\hwfs 站门},{\hwfs 带马},申侯{\hwfs 上马})}

\setlength{\hangindent}{60pt}{   %3. 设置悬挂缩进量                %
	\textrm{申侯\hspace{30pt}~ 【{\akai 西皮快板}】非是我放开路任他逃窜({\akai 或}:~与他逃窜),到前面}\footnote{此处吴焕老师整理本作``到前边''。}%\protect\hyperlink{fn11}{\textsuperscript{11}}
\textrm{遇戎兵难逃命还。教三军------}
}

\textrm{众\hspace{41pt}~ 有!}

\textrm{申侯\hspace{30pt}~ 【{\akai 西皮摇板}{\footnotesize 叫散}】你与爷({\akai 或}:~你与我)忙往前趱{\footnotesize 呐},}

\textrm{({\hwfs 两个}旗牌{\hwfs 在小边},申侯{\hwfs 大边})}

\textrm{申侯\hspace{30pt}~ 【{\akai 西皮散板}】做一个{\footnotesize 哇}假人情顺{\footnotesize 呃}水推船。}

\textrm{(申侯{\hwfs 举马鞭},{\hwfs 台口亮相},{\hwfs 绕一圈},{\hwfs 打马下})}
}

\vspace{15pt}
{\bfseries\textrm{本戏人物扮相}}:~
\vspace{15pt}

申侯\hspace{30pt}~ 绿蟒,黑三,戴侯帽,不挎宝剑。

幽王\hspace{30pt}~ 大白粉脸,黑嘴窝,黑满,不戴帽,戴发鬏,穿黄团龙帔。\footnote{刘曾复先生注:~徐碧云排《褒姒》一剧时,萧长华以丑角饰演幽王。}%\protect\hyperlink{fn12}{\textsuperscript{12}}

褒姒\hspace{30pt}~ 不戴凤冠,穿黄帔。

龙套\hspace{30pt}~ 四名(一堂),俱穿绿。

旗牌\hspace{30pt}~ 两名,俱穿黄;一个戴髯口,一个不戴髯口。


%\newpage
%\hypertarget{ux5b5dux611fux5929-ux4e4b-ux5171ux53d4ux6bb5}{%
\subsubsection{%\texorpdfstring
{孝感天%\protect\hyperlink{fn13}{\textsuperscript{13}}
~{\small 之}~共叔段}~\protect\footnote{根据刘曾复先生和杨绍箕先生2009年9月25日在电话里说戏录音整理。录音由杨绍箕先生托梁剑峰老师提供,刘曾复先生在电话中向杨绍箕先生主要介绍了整出戏的唱词、调度,同时介绍了小生的唱法。}}%{孝感天13 之 共叔段}}\label{ux5b5dux611fux5929-ux4e4b-ux5171ux53d4ux6bb5}}
\addcontentsline{toc}{subsection}{\hei 孝感天~\small{之}~共叔段}

\hangafter=1                   %2. 设置从第1⾏之后开始悬挂缩进  %
\setlength{\parindent}{0pt}{
{\centerline{\textrm{{[}\hei 第一场{]}}}}
\vspace{5pt}

(\textless{}\!{\bfseries\akai 大锣打上}\!\textgreater{}{\hwfs 四}龙套{\hwfs 拿云牌上},{\hwfs 站门},共叔段\footnote{共叔段由旦角应工,且所唱小生腔不能与旦角相重。}%\protect\hyperlink{fn14}{\textsuperscript{14}}
、卫云环{\hwfs 同上},{\hwfs 到台中间})

共叔段\hspace{20pt}~ ({\akai 念})丹心天地惨,黄泉路途遥。

(卫云环\hspace{15pt}~ ({\akai 念})溺爱反遭害,恩情一旦抛\footnote{刘曾复先生存本此句作``恩情有限闲'',上注``恩情有显消''。}%\protect\hyperlink{fn15}{\textsuperscript{15}}
。)

共叔段\hspace{20pt}~ 吾乃共叔段鬼魂\footnote{刘曾复先生存本此处作``灵魂''。}%\protect\hyperlink{fn16}{\textsuperscript{16}}
是也。

(卫云环\hspace{15pt}~ 吾乃卫云环鬼魂是也\footnote{刘曾复先生存本此句作``吾乃卫氏灵魂是也''。}%\protect\hyperlink{fn17}{\textsuperscript{17}}
。)

\setlength{\hangindent}{60pt}{   %3. 设置悬挂缩进量                %
	共叔段\hspace{20pt}~ 生前吾母爱子爱媳,反误我夫妻性命。闻得吾母在颖地之中,思儿想媳,不免前去探望一番。\footnote{刘曾复先生存本此句作``(生白)~段在生为姜国母爱子爱媳,不想反遭兄王之害,我夫妻不免梦中解劝一番呵。夫人请。(旦接)~夫君请。''}%\protect\hyperlink{fn18}{\textsuperscript{18}}
}

共叔段、\\
卫云环\hspace{20pt}~ \raisebox{5pt}{请。}

共叔段\hspace{20pt}~ 正是:~({\akai 念})蜀魄啼残三月雨\footnote{此句据刘曾复先生存本,当系准词。``蜀魄''是典籍中常用的杜鹃的别称。刘曾复先生在介绍此剧时可能据别本,作``树破提惨三月雨'',不确。}%\protect\hyperlink{fn19}{\textsuperscript{19}
},

(卫云环\hspace{15pt}~ ({\akai 念})梦魂惊断五更风\footnote{刘曾复先生存本``惊断''上注``凄断''。}%\protect\hyperlink{fn20}{\textsuperscript{20}}
。)

(\textless{}\!{\bfseries\akai 大锣打下}\!\textgreater{}{\hwfs 四}龙套{\hwfs 下},共叔段、卫云环{\hwfs 下},\textless{}\!{\bfseries\akai 撤锣}\!\textgreater{})

\vspace{3pt}{\centerline{\textrm{{[}{\hei 第二场}{]}}}}\vspace{5pt}

(\textless{}\!{\bfseries\akai 小锣打上}\!\textgreater{}{\hwfs 二}宫女{\hwfs 上},{\hwfs 站门},姜氏{\hwfs 上})

(姜氏\hspace{25pt}~ {[}{\akai 引子}{]} 身居颖地,思娇儿,珠泪双悲。)

(姜氏坐{\hwfs 小座})

(姜氏\hspace{25pt}~ ({\akai 念})生离最是苦,死后何伤乎!世看姜国母,有子不如无。)

\setlength{\hangindent}{60pt}{   %3. 设置悬挂缩进量                %
(姜氏\hspace{25pt}~ 哀家姜皇后,只因溺爱次子共叔段,反累夫妇双亡。可恨寤生,将我迁入颖地居住。思想起来,好不伤惨人也。)
}

(\textless{}\!{\bfseries\akai 扎多乙}\!\textgreater{})

\setlength{\hangindent}{60pt}{   %3. 设置悬挂缩进量                %
(姜氏\hspace{25pt}~ 【{\akai 二黄原板}】姜国母在颖地长吁短叹,思皇儿昼夜里珠泪不干。恨寤生他与我母子情断,但不知何日里得转故园。)
}

({\bfseries\akai 起更},姜氏{\hwfs 站起},\textless{}\!{\bfseries\akai 小拉子}\!\textgreater{},姜氏{\hwfs 脱黄帔})

(姜氏\hspace{25pt}~ 回避了。)

({\hwfs 二}宫女{\hwfs 翻下})

\setlength{\hangindent}{60pt}{   %3. 设置悬挂缩进量                %
(姜氏\hspace{25pt}~ 【{\akai 二黄慢板}】居深宫冷清清无依无伴,怕只怕睡不安月照栏杆。对银灯眼昏花思自叹,)
}

(姜氏{\hwfs 进桌子})

(姜氏\hspace{25pt}~ 【{\akai 二黄慢板}】有谁人怜念我影孤形单。)

(\textless{}\!{\bfseries\akai 长锤}\!\textgreater{},共叔段{\hwfs 上},卫氏{\hwfs 同随上},{\hwfs 同站小边台口})

共叔段\hspace{20pt}~ 【{\akai 二黄慢板}】风飘飘冷飕飕黄昏惨淡, 曾记得在生时束带顶冠。

(共叔段{\hwfs 挖门进},{\hwfs 站大边},{\hwfs 朝里})
   
\setlength{\hangindent}{65pt}{   %3. 设置悬挂缩进量                %
	(卫云环\hspace{15pt}~ 【{\akai 二黄慢板}】此时间讲什么粉消香散,进寝宫({\akai 或}:~进宫去)愿母后免去心酸。\footnote{``粉消香散''四字从刘曾复先生存本,当系准词,说戏录音作``焚烧香泛'',李楠君以为作``焚烧香饭'',香饭是佛家饭食。存本上注``霎时间说不尽风流云散,进宫来见母后心内痛酸。''}%\protect\hyperlink{fn21}{\textsuperscript{21}}
)
}

(卫云环{\hwfs 挖门进},{\hwfs 站小边},{\hwfs 朝里})

共叔段\hspace{20pt}~ \textless{}\!{\bfseries\akai 双叫头}\!\textgreater{}母亲!老娘啊!$\cdots${}$\cdots{}$({\hwfs 哭介})

(卫云环\hspace{15pt}~ \textless{}\!{\bfseries\akai 双叫头}\!\textgreater{}母后!婆母!喂呀$\cdots${}$\cdots{}$({\hwfs 哭介}))\footnote{刘曾复先生存本此处作``(生旦同白)~国母醒来。''}%\protect\hyperlink{fn22}{\textsuperscript{22}}

(姜氏\hspace{25pt}~ 【{\akai 反二黄导板}】梦儿里又听得有人呼唤,)

(姜氏\hspace{25pt}~ 【{\akai 反二黄摇板}】又只见灯光下双影闪闪。看不明抚一抚昏花老眼,)

共叔段\hspace{20pt}~ 母亲啊$\cdots${}$\cdots{}$({\hwfs 哭介})

(卫云环\hspace{15pt}~ 母后,喂呀$\cdots${}$\cdots{}$({\hwfs 哭介}))

(\textless{}\!{\bfseries\akai 凤点头}\!\textgreater{})

\setlength{\hangindent}{65pt}{   %3. 设置悬挂缩进量                %
(姜氏\hspace{25pt}~ 【{\akai 反二黄摇板}】却原来子叔段、儿媳云环。曾记得寿宴前({\akai 或}:~曾记得寿筵前)金樽酒满,你夫妻双双拜母子同欢。自从儿到京都未能相见,)
}

({\hwfs 起}{\akai 反二黄}【{\akai 回龙}】)

共叔段、\\
卫云环\hspace{20pt}~  \raisebox{5pt}{【{\akai 回龙}】尊国母且不必双目泪涟\footnote{刘曾复先生存本此句作``虽是儿丧黄泉也却心甘''。}%\protect\hyperlink{fn23}{\textsuperscript{23}}
。}

\setlength{\hangindent}{60pt}{   %3. 设置悬挂缩进量                %
	共叔段\hspace{15pt}~ 【{\akai 反二黄慢板}】悲切切尊国母魂飞魄散,悔不该图兄位自惹身残。辜负了生身母无依无伴,儿死在黄泉路也难心甘。\footnote{刘曾复先生存本此处作``悲切切尊国母魂伤魄断,悔不该图兄位自惹身残。辜负了生身母无依无伴,儿死在黄泉路瞑目心甘。''末句上注``劝国母切莫要珠泪不干''。}%\protect\hyperlink{fn24}{\textsuperscript{24}}
}

\setlength{\hangindent}{60pt}{   %3. 设置悬挂缩进量                %
	(卫云环\hspace{10pt}~ 【{\akai 反二黄慢板}】这也是天命定数有修短,劝国母切莫言终日泪涟。为报恩来世里重会慈范,千古来有谁人百岁同欢。\footnote{刘曾复先生存本此处作``这也是天命定数有修短,劝国母终日里免却伤惨。为报恩来世里重亲慈范,千古来是何人百岁同欢。''末句上注``千古来有何人百岁同欢''。}%\protect\hyperlink{fn25}{\textsuperscript{25}})
}

\setlength{\hangindent}{60pt}{   %3. 设置悬挂缩进量                %
	(姜氏\hspace{20pt}~ 【{\akai 反二黄原板\footnote{姜氏也可以唱【{\akai 反二黄慢板}】,但一般都唱【{\akai 反二黄原板}】。}%\protect\hyperlink{fn26}{\textsuperscript{26}}
}】听他言不由我心中悲惨,原来是我的儿站在面前。莫非是母子们梦里相见,纵然是梦相会娘也心甘。)
}

({\hwfs 叫散},\textless{}\!{\bfseries\akai 扭丝}\!\textgreater{}\textless{}\!{\bfseries\akai 凤点头}\!\textgreater{})\footnote{刘曾复先生存本此处有``(生白)~母后哇$\cdots${}$\cdots{}$(接【{\akai 反调摇板}】)''}%\protect\hyperlink{fn27}{\textsuperscript{27}}

\setlength{\hangindent}{60pt}{   %3. 设置悬挂缩进量                %
	共叔段\hspace{15pt}~ 【{\akai 反二黄摇板}】娘在阳儿在阴两厢隔断,母子们要相逢({\akai 或}:~要相会;要重逢)梦里团圆。\footnote{刘曾复先生存本此处作``儿在阴娘在阳两厢隔断,母子们要相逢今世却难。''}%\protect\hyperlink{fn28}{\textsuperscript{28}}
}

(共叔段{\hwfs 出门},{\hwfs 下})

(卫云环\hspace{10pt}~ 【{\akai 反二黄摇板}】天将明儿要归不必怜念,指日里郑君侯迎请凤鸾。\footnote{刘曾复先生存本此处作``天将明儿要归不尽悲惨,郑君侯指日里迎请凤鸾。''}%\protect\hyperlink{fn29}{\textsuperscript{29}}
)

(卫云环{\hwfs 出门},{\hwfs 下}。{\bfseries\akai 五更},{\bfseries\akai 亮更}\footnote{刘曾复先生介绍唱词及场次时说明,在中间可以加更次。}%\protect\hyperlink{fn30}{\textsuperscript{30}}
,{\hwfs 二}宫女{\hwfs 上})

(二宫女\hspace{15pt}~ 国太醒来。)

(姜氏\hspace{20pt}~ 【{\akai 二黄导板}】适才间见娇儿肝肠痛断,)

(姜氏\hspace{20pt}~ 儿啊$\cdots${}$\cdots{}$({\hwfs 哭介}))

(姜氏{\hwfs 出位},{\hwfs 站在桌前})

(姜氏\hspace{20pt}~ 【{\akai 二黄摇板}】醒来时却原来南柯梦间。相劝我好言语甚是悲惨,)

({\hwfs 二}宫女{\hwfs 扯斜})

(姜氏\hspace{20pt}~ 【{\akai 二黄摇板}】等候了郑君侯接我回还({\akai 或}:~接我回銮)。)

(姜氏\hspace{20pt}~ 唉,儿啊$\cdots${}$\cdots{}$({\hwfs 哭介}))

(姜氏{\hwfs 下},{\hwfs 二}宫女{\hwfs 随下})

\vspace{25pt}
{\bfseries\textrm{本戏人物扮相}}:~
\vspace{15pt}

共叔段:~\hspace{20pt}~ 戴紫金冠,穿红蟒、玉带,拿云帚,不戴鬼发、魂帕;

卫云环:~\hspace{20pt}~  戴凤冠,穿女蟒、玉带,拿云帚,不戴鬼发、魂帕;

姜氏:~\hspace{30pt}~ 外穿黄帔、内穿深紫帔、罩黑坎肩、系黄绦子,绿裙,不戴凤冠。

刘曾老特别说明,如果作为开场戏,四龙套可不拿云牌,共叔段可穿开氅,卫云环可穿宫装。


%\setlength{\hangindent}{56pt}{\newpage}

\setlength{\hangindent}{56pt}

\setlength{\hangindent}{56pt}{%\subsection{马鞍山}\label{ux9a6cux978dux5c71}}}

\setlength{\hangindent}{56pt}{\subsubsection{{\hei\large 马鞍山}}%\protect\hyperlink{fn8}{\textsuperscript{8}}}

\setlength{\hangindent}{56pt}{\addcontentsline{toc}{subsection}{\hei 马鞍山}}

\setlength{\hangindent}{56pt}

\setlength{\hangindent}{56pt}{{\centerline{(李舒~遗作~~根据刘曾复先生手书原稿抄录)}}}

\setlength{\hangindent}{56pt}{\hangafter=1\hspace{20pt}~ %2. 设置从第1⾏之后开始悬挂缩进}

\setlength{\hangindent}{56pt}{\setlength{\parindent}{0pt}{

\setlength{\hangindent}{56pt}{{\centerline{{\bfseries\akai {[}\hei 第一场{]}}}}}

\setlength{\hangindent}{56pt}{\vspace{5pt}}

\setlength{\hangindent}{56pt}{(童儿、俞伯牙\textless{}\!{\bfseries\akai 小锣打上}\!\textgreater{})}

\setlength{\hangindent}{56pt}{\spacept{俞伯牙}{20pt} {[}{\akai 引子}{]}为访贤友,涉水登舟。}

\setlength{\hangindent}{56pt}{\spacept{俞伯牙}{20pt} {[}{\akai 诗}{]}青溪流过碧山头,空水澄鲜一色秋。隔断红尘三十里,白云红叶两悠悠。\footnote{该定场诗用的是北宋程颢的七绝《秋月》诗句,李舒先生钞录稿末句作``白云鸿雁两悠悠''。}}

\setlength{\hangindent}{56pt}{\spacept{俞伯牙}{20pt} 下官姓俞名瑞字伯牙,(乃)鲁国人氏,晋国为官。只因去岁往各国催贡,船行马鞍山前,偶遇钟子期,我二人共谈琴律,情意相投,结为金兰之好,临行(之时,)赠他黄金二笏,约定今岁中秋还在马鞍山前相会。来此不见贤弟到来,昨晚琴音缭乱\footnote{李元皓{\scriptsize 君}建议作``撩乱''。},不知是何缘故,我不免去往集贤村寻访于他便了。}

\setlength{\hangindent}{56pt}{\spacept{俞伯牙}{20pt} 童儿。}

\setlength{\hangindent}{56pt}{(童儿\hspace{30pt}~ 有。) }

\setlength{\hangindent}{56pt}{\spacept{俞伯牙}{20pt} 看衣改换。}

\setlength{\hangindent}{56pt}{(\textless{}\!{\bfseries\akai 小开门}\!\textgreater{},下,再上)}

\setlength{\hangindent}{56pt}{\spacept{俞伯牙}{20pt} 带了瑶琴,(随我往)集贤村去者!}

\setlength{\hangindent}{56pt}{(一翻两翻,半个\textless{}\!{\bfseries\akai 扯四门}\!\textgreater{},小童一直站大边)}

\setlength{\hangindent}{56pt}{\spacept{俞伯牙}{20pt} 【{\akai 二黄三眼}】我二人在山前金兰结好,今此来他不到所为哪条({\akai 或}:~今此来不见他所为哪条)。换冠裳我亲自义友寻找,此一去集贤村访见故交。}

\setlength{\hangindent}{56pt}{(\textless{}\!{\bfseries\akai 小锣打下}\!\textgreater{})}

\setlength{\hangindent}{56pt}{\vspace{3pt}{\centerline{\textrm{{[}{\hei 第二场}{]}}}}\vspace{5pt}}

\setlength{\hangindent}{56pt}{\spacept{钟元普\footnote{钟元普亦作钟元甫,此处从李舒先生钞录稿。}}{18pt} ({\akai 内白})走哇。}

\setlength{\hangindent}{56pt}{(提篮子,\textless{}\!{\bfseries\akai 小锣抽头}\!\textgreater{}{\hwfs 上})}

\setlength{\hangindent}{56pt}{\spacept{钟元普}{20pt} 唉!}

\setlength{\hangindent}{56pt}{\spacept{钟元普}{20pt} 【{\akai 二黄摇板}】屋漏偏遭连阴雨,破船又遇当头风。}

\setlength{\hangindent}{56pt}{\spacept{钟元普}{20pt} 老汉钟元普。吾儿({\akai 或}:~亡儿)名唤子期。只因去岁,中秋在马鞍山前砍柴,偶遇一位晋国大夫俞伯牙大人,他二人共谈琴律,情意相投,结为金兰之好。临别({\akai 或}:~临行)赠我儿黄金二笏,约定今岁中秋还在马鞍山前相会。谁想我儿回得家来,白日砍柴,夜晚攻书,朝暮积劳,染成疾病。他就此一命身亡了$\cdots${}$\cdots${}(钟元普{\hwfs 哭介})}

\setlength{\hangindent}{56pt}{\spacept{钟元普}{20pt} 咳,今当吾儿百日之期({\akai 或}:~今当亡儿百日之期),为此备了几陌纸钱,去往坟前烧化。天呐,天,({\akai 念})家有万贯终何用,老来无子一场空。}

\setlength{\hangindent}{56pt}{\spacept{钟元普}{20pt} 【{\akai 二黄原板}】老眼昏花路难行,又闻得({\akai 或}:~又听得)松林内百鸟喧声。乌鸦倒有反哺意,羊羔也有跪乳情。似乌云遮住了天边月,似狂风吹散了满天云。这才是黄梅已老青梅落,白发人反送了黑发儿的身。我的儿呀!}

\setlength{\hangindent}{56pt}{(\textless{}\!{\bfseries\akai 小锣抽头}\!\textgreater{}{\hwfs 下})}

\setlength{\hangindent}{56pt}{(俞伯牙{\hwfs 接}\textless{}\!{\bfseries\akai 小锣抽头}\!\textgreater{}{\hwfs 上})}

\setlength{\hangindent}{56pt}{\spacept{俞伯牙}{20pt} 【{\akai 二黄摇板}】昨夜晚抚瑶琴暗藏悲调,看起来这内中事有蹊跷。移步儿来至在双阳岔道,(\textless{}\!{\bfseries\akai 小锣抽头}\!\textgreater{}圆场)寻不着集贤村路走哪条?}

\setlength{\hangindent}{56pt}{\spacept{俞伯牙}{20pt} 哎呀且住,来此已是双阳岔道,但不知这集贤村往哪条道路而走。}

\setlength{\hangindent}{56pt}{\spacept{钟元普}{20pt} ({\akai 内嗽})嗯哼。}

\setlength{\hangindent}{56pt}{\spacept{俞伯牙}{20pt} 看那旁来一老丈。等他到来问明再走。}

\setlength{\hangindent}{56pt}{(钟元普{\hwfs 上})}

\setlength{\hangindent}{56pt}{\spacept{钟元普}{20pt} 【{\akai 二黄摇板}】曲弯弯行过了溪边小道,哪有个父与子把纸来烧({\akai 或}:~把纸化烧)。(过大边)}

\setlength{\hangindent}{56pt}{\spacept{俞伯牙}{20pt} 老丈请转。}

\setlength{\hangindent}{56pt}{\spacept{钟元普}{20pt} 呃,原来是一位先生,这位先生可是失迷路途?}

\setlength{\hangindent}{56pt}{\spacept{俞伯牙}{20pt} 正是。}

\setlength{\hangindent}{56pt}{\spacept{钟元普}{20pt} 但不知问的是何所在?}

\setlength{\hangindent}{56pt}{\spacept{俞伯牙}{20pt} 我问的是集贤村。}

\setlength{\hangindent}{56pt}{\spacept{钟元普}{20pt} 先生,你来看:这东去十里也是集贤村,西去十里也是集贤村。但不知是哪个集贤村呢?}

\setlength{\hangindent}{56pt}{\spacept{俞伯牙}{20pt} 这$\cdots${}$\cdots${}哎呀,贤弟呀,现有两个集贤村({\akai 或}:~既有两个集贤村),为何不对愚兄说明,如今叫我作难了。}

\setlength{\hangindent}{56pt}{\spacept{钟元普}{20pt} 啊先生,敢是指路不明?}

\setlength{\hangindent}{56pt}{\spacept{俞伯牙}{20pt} 呃呃,久住三五载,}

\setlength{\hangindent}{56pt}{\spacept{钟元普}{20pt} 无处不亲联。\footnote{李舒先生钞录稿作``无处不亲连。'',此处据樊百乐{\scriptsize 君}转述刘曾复先生确认文字。在表示通婚结成姻亲关系时,``联姻''通``连姻''。}

\setlength{\hangindent}{56pt}{\spacept{俞伯牙}{20pt} 正是。}

%俞伯牙、\\
%\spacept{钟元普}{20pt} \raisebox{5pt}{啊哈哈哈哈。}
\raisebox{0pt}[22pt][16pt]{\raisebox{8pt}{俞伯牙}\raisebox{-8pt}{\hspace{-32pt}{钟元普}}\raisebox{0pt}{\hspace{30pt}啊哈哈哈哈。}}

\setlength{\hangindent}{56pt}{\spacept{钟元普}{20pt} 但不知问的是哪一家?}

\setlength{\hangindent}{56pt}{\spacept{俞伯牙}{20pt} 我问的是钟子期。}

\setlength{\hangindent}{56pt}{\spacept{钟元普}{20pt} 哦,钟子期。}

\setlength{\hangindent}{56pt}{\spacept{俞伯牙}{20pt} 正是。}

\setlength{\hangindent}{56pt}{\spacept{钟元普}{20pt} 咳,儿呀。}

\setlength{\hangindent}{56pt}{\spacept{钟元普}{20pt} 【{\akai 二黄摇板}】相逢未说几句话,不由老汉泪如麻。({\akai 哭介})}

\setlength{\hangindent}{56pt}{\spacept{钟元普}{20pt} 先生你来迟了。}

\setlength{\hangindent}{56pt}{\spacept{俞伯牙}{20pt} (老丈)何言来迟?}

\setlength{\hangindent}{56pt}{\spacept{钟元普}{20pt} 老汉钟元普。吾儿子期({\akai 或}:~亡儿子期),只因去岁中秋与俞大人结拜({\akai 或}:~与俞大人结为金兰之好),分别之后回到家中,他白日砍柴,夜晚攻书,积劳成疾({\akai 或}:~积劳成病),百日前(他)一命身亡了$\cdots${}$\cdots${}}

\setlength{\hangindent}{56pt}{\spacept{俞伯牙}{20pt} 你待怎讲?}

\setlength{\hangindent}{56pt}{\spacept{钟元普}{20pt} 一命身亡了。}

\setlength{\hangindent}{56pt}{\spacept{俞伯牙}{20pt} 哎呀!~(\textless{}\!{\bfseries\akai 崩登仓冲头}\!\textgreater{})}

\setlength{\hangindent}{56pt}{(俞伯牙{\hwfs 昏介})}

\setlength{\hangindent}{56pt}{\spacept{钟元普}{20pt} 这是何人?}

\setlength{\hangindent}{56pt}{童儿\hspace{30pt}~ 这就是俞大人。 }

\setlength{\hangindent}{56pt}{\spacept{钟元普}{20pt} 哦哦({\akai 或}:~哎呀),大人醒来。}

\setlength{\hangindent}{56pt}{\spacept{俞伯牙}{20pt} 【{\akai 二黄导板}】听说是钟贤弟一命丧了,}

\setlength{\hangindent}{56pt}{\spacept{俞伯牙}{20pt} \textless{}\!{\bfseries\akai 三叫头}\!\textgreater{}贤弟! 子期!哎贤弟呀。}

\setlength{\hangindent}{56pt}{\spacept{俞伯牙}{20pt} 【{\akai 二黄散板}】此一番好一似马行断桥。他的父是尊长急忙拜倒,}

\setlength{\hangindent}{56pt}{\spacept{钟元普}{20pt} 【{\akai 二黄散板}】请大人莫折煞年迈山樵。}

\setlength{\hangindent}{56pt}{(\spacept{俞伯牙}{20pt} 老伯。)}

\setlength{\hangindent}{56pt}{\spacept{俞伯牙}{20pt} 【{\akai 二黄散板}】我就是俞伯牙伯父知晓,贤弟死留何言细说根苗。}

\setlength{\hangindent}{56pt}{\spacept{钟元普}{20pt} 【{\akai 二黄散板}】我的儿临危时也曾言道:葬埋在马鞍山候驾来瞧。}

\setlength{\hangindent}{56pt}{\spacept{俞伯牙}{20pt} 【{\akai 二黄散板}】烦伯父你与我坟台引道,}

\setlength{\hangindent}{56pt}{(\textless{}\!{\bfseries\akai 扭丝}\!\textgreater{},钟元普、俞伯牙{\hwfs 同走圆场})}

\setlength{\hangindent}{56pt}{\spacept{钟元普}{20pt} 【{\akai 二黄散板}】这就是新坟土尚挂纸标。}

\setlength{\hangindent}{56pt}{(\spacept{俞伯牙}{20pt} 哎呀!)}

\setlength{\hangindent}{56pt}{\spacept{俞伯牙}{20pt} 【{\akai 二黄散板}】见坟台不由我双膝跪倒,呼不应、唤不醒生死故交。}

\setlength{\hangindent}{56pt}{\spacept{俞伯牙}{20pt} 贤弟呀$\cdots${}$\cdots${}({\hwfs 哭介})}

\setlength{\hangindent}{56pt}{\spacept{俞伯牙}{20pt} (啊,)老伯,那旁有一石台,老伯稍坐一时,待侄儿一祭。}

\setlength{\hangindent}{56pt}{(俞伯牙{\hwfs 哭介})}

\setlength{\hangindent}{56pt}{\spacept{钟元普}{20pt} 有劳大人。}

\setlength{\hangindent}{56pt}{\spacept{俞伯牙}{20pt} 童儿。}

\setlength{\hangindent}{56pt}{(童儿\hspace{30pt}~ 有。) }

\setlength{\hangindent}{56pt}{\spacept{俞伯牙}{20pt} 将我瑶琴摆在坟前。}

\setlength{\hangindent}{56pt}{({\hwfs 此处上渔}、{\hwfs 樵})}

\setlength{\hangindent}{56pt}{\spacept{俞伯牙}{20pt} 唉!~({\akai 念})此来空枉费,人琴付东流\footnote{李舒先生钞录稿作``人情付东流'',似非。}。灵魂渺茫去呀,可叹一土丘。}

\setlength{\hangindent}{56pt}{\spacept{俞伯牙}{20pt} \textless{}\!{\bfseries\akai 帽子头}\!\textgreater{}【{\akai 二黄慢板}】想去岁中秋节论琴交好,今日里见坟台不见故交。来时喜去时悲愁云渺渺,又只见秋风起黄叶飘飘。为贤弟我不爱黄金荣耀,为贤弟我不爱玉带紫袍。为贤弟二双亲少行孝道,为贤弟辞王驾亲走这遭。为贤弟终日里梦魂颠倒,为贤弟千里迢迢,涉水登山,枉费徒劳。实指望与贤弟同饮香醪,实指望与贤弟共论琴操。实指望与贤弟朝夕欢笑,实指望与贤弟春游芳草,夏赏荷香,秋饮菊酒,冬藏梅阁,散淡逍遥。在坟台抚瑶琴以为祭吊,}

\setlength{\hangindent}{56pt}{(俞伯牙{\hwfs 抚琴介})}

\setlength{\hangindent}{56pt}{(\spacept{俞伯牙}{20pt} 唉!)}

\setlength{\hangindent}{56pt}{\spacept{俞伯牙}{20pt} 【{\akai 二黄散板}】子期死少知音琴对谁调。我这里将瑶琴摔碎不要,}

\setlength{\hangindent}{56pt}{(俞伯牙{\hwfs 摔琴介}\textless{}\!{\bfseries\akai 乱锤}\!\textgreater{})}

\setlength{\hangindent}{56pt}{\spacept{钟元普}{20pt} 【{\akai 二黄散板}】问大人摔瑶琴所为哪条?}

\setlength{\hangindent}{56pt}{\spacept{俞伯牙}{20pt} 老伯,}

\setlength{\hangindent}{56pt}{\spacept{俞伯牙}{20pt} ({\akai 念})摔碎瑶琴凤尾寒,子期不在向谁弹?春风满面皆朋友,要会知音难上难。(俞伯牙{\akai 哭介})}

\setlength{\hangindent}{56pt}{\spacept{俞伯牙}{20pt} 【{\akai 二黄散板}】问伯父贤弟死家有何靠,}

\setlength{\hangindent}{56pt}{\spacept{钟元普}{20pt} 【{\akai 二黄散板}】隐居在集贤村倒也逍遥({\akai 或}:~倒还逍遥)。}

\setlength{\hangindent}{56pt}{\spacept{俞伯牙}{20pt} 【{\akai 二黄散板}】这黄金与伯父甘旨\footnote{甘旨,原意是美味的食品。引申为对双亲的奉养。}养老,且待我迎接你替他代劳。}

\setlength{\hangindent}{56pt}{\spacept{俞伯牙}{20pt} 老伯,侄儿去后伯父不要思他。}

\setlength{\hangindent}{56pt}{\spacept{钟元普}{20pt} 我不思他。}

\setlength{\hangindent}{56pt}{\spacept{俞伯牙}{20pt} 不要想他。}

\setlength{\hangindent}{56pt}{\spacept{钟元普}{20pt} 我也不想他。({\akai 或}:~呃,我不想他。)}

\setlength{\hangindent}{56pt}{\spacept{俞伯牙}{20pt} 子期是我。}

\setlength{\hangindent}{56pt}{\spacept{钟元普}{20pt} (呃,)不敢。}

\setlength{\hangindent}{56pt}{\spacept{俞伯牙}{20pt} 我是子期。}

\setlength{\hangindent}{56pt}{\spacept{钟元普}{20pt} 实实不敢({\akai 或}:~唉,越发地不敢呐)。}

\setlength{\hangindent}{56pt}{\spacept{俞伯牙}{20pt} 小侄告辞了。}

\setlength{\hangindent}{56pt}{\spacept{俞伯牙}{20pt} 【{\akai 二黄散板}】辞伯父别坟墓扬长就道,}

\setlength{\hangindent}{56pt}{\spacept{钟元普}{20pt} 【{\akai 二黄散板}】虽异姓似手足犹如同胞。}

\setlength{\hangindent}{56pt}{\spacept{俞伯牙}{20pt} 【{\akai 二黄散板}】伯牙在$\cdots${}$\cdots${}}

\setlength{\hangindent}{56pt}{\spacept{钟元普}{20pt} 【{\akai 二黄散板}】子期死(啊)$\cdots${}$\cdots${}}

\setlength{\hangindent}{56pt}{\spacept{俞伯牙}{20pt} ({\akai 接唱})【{\akai 二黄散板}】知音缺少,摔瑶琴谢知音不负故交。}

\setlength{\hangindent}{56pt}{\spacept{俞伯牙}{20pt} \textless{}\!{\bfseries\akai 三叫头}\!\textgreater{}老伯,子期,唉贤弟呀。}

\setlength{\hangindent}{56pt}{(\spacept{钟元普}{20pt} \textless{}\!{\bfseries\akai 三叫头}\!\textgreater{}大人,我儿,唉儿呀。)}

\setlength{\hangindent}{56pt}{\spacept{俞伯牙}{20pt} 罢!}

\setlength{\hangindent}{56pt}{(俞伯牙{\hwfs 下},小童{\hwfs 同下})}

\setlength{\hangindent}{56pt}{(\spacept{钟元普}{20pt} 唉!)}

\setlength{\hangindent}{56pt}{\spacept{钟元普}{20pt} 【{\akai 二黄散板}】似这等金兰友如同管鲍,转身来见坟台不见儿曹。猛抬头见红日西山落了,回家去与老妻细说根苗。}

\setlength{\hangindent}{56pt}{\spacept{钟元普}{20pt} 儿呀!}

\setlength{\hangindent}{56pt}{(\textless{}\!{\bfseries\akai 小锣打下}\!\textgreater{}\textless{}\!{\bfseries\akai 尾声}\!\textgreater{})}

\vspace{20pt}
\setlength{\hangindent}{56pt}{{\bfseries 本戏人物扮相}:~ }

\setlength{\hangindent}{56pt}{\spacept{俞伯牙}{20pt}~ 纱帽,蓝帔,黑三,高方巾,宝蓝褶子,绦子。}

\setlength{\hangindent}{56pt}{\spacept{钟元普}{20pt}~ 白氈帽,白满,白老斗衣,腰包,鞋。}

\setlength{\hangindent}{56pt}{童儿\hspace{30pt}~ 抓髻,白花褶子,鞋。 }

\vspace{25pt}
\setlength{\hangindent}{56pt}{{\hei 道具}:~ }

\setlength{\hangindent}{56pt}{小篮一只,内装纸钱。}

\setlength{\hangindent}{56pt}{石台,即倒椅两把。}

\vspace{25pt}
\setlength{\hangindent}{56pt}{{\bfseries\hei 附:~  渔、樵词}~ (念法很多,此为其中的一种) 

\setlength{\hangindent}{56pt}{两个小花脸,一老一少扮相如渔、樵。}

\setlength{\hangindent}{56pt}{(俞伯牙{\akai 念}``将摇琴摆在坟前$\cdots${}$\cdots${}'',渔、樵{\akai 内}``啊哈''\textless{}\!{\bfseries\akai 小锣五击}\!\textgreater{}{\hwfs 上})}

\setlength{\hangindent}{56pt}{渔\hspace{40pt}~ ({\akai 念})渔翁夜傍西岩宿,}

\setlength{\hangindent}{56pt}{樵\hspace{40pt}~ ({\akai 念})更殚余力付樵苏。}

\setlength{\hangindent}{56pt}{渔\hspace{40pt}~ 伙计,你说什么呐? }

\setlength{\hangindent}{56pt}{樵\hspace{40pt}~ 我这念诗呐。 }

\setlength{\hangindent}{56pt}{渔\hspace{40pt}~ 你这长相还会念诗。 }

\setlength{\hangindent}{56pt}{樵\hspace{40pt}~ 就算我不会,那你干什么呐? }

\setlength{\hangindent}{56pt}{渔\hspace{40pt}~ 我这可是念诗呐。 }

\setlength{\hangindent}{56pt}{樵\hspace{40pt}~ 许你念就不许我念。 }

\setlength{\hangindent}{56pt}{渔\hspace{40pt}~ 咱俩别吵,我是道听途说。 }

\setlength{\hangindent}{56pt}{樵\hspace{40pt}~ 我也是胡说八道。 }

\setlength{\hangindent}{56pt}{渔\hspace{40pt}~ 哎,你看这些人在这干什么呐? }

\setlength{\hangindent}{56pt}{樵\hspace{40pt}~ 我看是上坟的。 }

\setlength{\hangindent}{56pt}{渔\hspace{40pt}~ 走累了,咱俩一边一个靠着树坐会儿。 }

\setlength{\hangindent}{56pt}{樵\hspace{40pt}~ 坐着坐着。 }

\setlength{\hangindent}{56pt}{(俞伯牙{\hwfs 弹完琴}$\cdots${}$\cdots${})}

\setlength{\hangindent}{56pt}{渔\hspace{40pt}~ 伙计,你听他们干什么呐? }

\setlength{\hangindent}{56pt}{樵\hspace{40pt}~ 八成是弹棉花的。 }

\setlength{\hangindent}{56pt}{渔\hspace{40pt}~ 没事憩会儿好不好,跑这坟圈子里弹棉花干什么。 }

\setlength{\hangindent}{56pt}{樵\hspace{40pt}~ 吃饱了在这凉快凉快。 }

\setlength{\hangindent}{56pt}{渔\hspace{40pt}~ 别挨骂了。正是:~({\akai 念})兰浦秋来烟雨深,}

\setlength{\hangindent}{56pt}{樵\hspace{40pt}~ ({\akai 念})几多情思在琴心。}

\setlength{\hangindent}{56pt}{渔\hspace{40pt}~ 又拽上了。 别听弹棉花的了。}

\setlength{\hangindent}{56pt}{樵\hspace{40pt}~ 回家睡大觉去喽,哈$\cdots${}$\cdots${} }

\setlength{\hangindent}{56pt}{(渔、樵{\hwfs 同下})}

\vspace{25pt}
\bfseries\akai\hspace{10pt}~ 注:~}
\begin{enumerate}
	\item 上~渔、樵,俞、钟、童三人面向里。上渔、樵无非是不懂琴音,俞伯牙无知音,实际无此必要,故后来删掉。
	\item 《马鞍山》是乔玉林传下来的路子,传统、规矩,是学徒的唱法,中华戏曲专科学校的唱法略同与此,时慧宝的唱法与此有出入。此戏在过去是前三出的大路戏。
\end{enumerate}
}

%\newpage
\subsubsection{\large\hei 焚绵山~\protect\footnote{《京剧汇编》第七十四集作``焚棉山''。剧中介子推的词句部分参考了吴焕老师记录的刘曾复先生说戏录音文稿。}{\small 之}介子推~\protect\footnote{介子推亦作介之推。}、介母、解张}
\addcontentsline{toc}{subsection}{\hei 渭水河}

\hangafter=1                   %2. 设置从第1⾏之后开始悬挂缩进  %
\setlength{\parindent}{0pt}{
\vspace{3pt}{\centerline{{[}{\hei 第一场}{]}}}\vspace{5pt}

介子推\hspace{20pt}{[}{\akai 引子}{]}弃官离朝({\akai 或}: 弃职离朝),名利抛,侍奉年高。

介子推\hspace{20pt}({\akai 念})见机逃出是非地,落得清闲自在身。冷眼识破君王意,功成身退奉{\footnotesize 呃}娘亲。
 
\setlength{\hangindent}{56pt}{介子推\hspace{20pt} 卑人,介(子)推。晋献公驾前为臣,官居谏议大夫(之职)。只因吾主({\akai 或}: 只因君王无道,)听信谗言,毒害大臣,害死申生太子。是我与狐毛、狐偃、颠颉、魏犨等九人,保定(公子)重耳逃出罗网,周游列国一十九载。归来共渡黄河,是俺看破其意({\akai 或}: 是我识破其意),弃职归林。正是:~({\akai 念})奉君何足乐,还是孝{\footnotesize 呃}当先。}

\setlength{\hangindent}{56pt}{
介子推\hspace{20pt}【{\akai 西皮慢板}】介子推坐草堂前思后想,想起了晋国事好不凄凉。晋献公听信那谗言毁谤,宠骊姬害申生命赴黄粱。我十人保重耳呐 }【{\footnotesize 转}{\akai 西皮原板}】逃出罗网,朝同行夜同寝({\akai 或}: 夜共寝)伴随君旁。我也曾在荒郊觅食取浆,我也曾割股肉奉献君王。渡黄河一个个争呃功邀赏啊,又谁知出恶言暗把人伤。既这等做什么忠呃臣良将,因此上怀戒心弃职还乡。回家来织呐履舄\footnote{ 古代单底鞋称履,复底鞋称舄,故以``履舄''泛称鞋。}心宽【{\akai 回龙}】意爽,

介子推\hspace{20pt}【{\akai 西皮摇板}】斩断了名利锁侍奉老娘。

解张\hspace{30pt}【{\akai 西皮摇板}】晋公子登龙位各官封赠,相劝那介子推前去面君。

解张\hspace{30pt}来此已是,介兄在家么?

介子推\hspace{20pt}是哪位?

介子推\hspace{20pt}哦,原来是解兄,请进!

解张\hspace{30pt}请,这厢有礼。

介子推\hspace{20pt}还礼,请坐!

解张\hspace{30pt}有座。

介子推\hspace{20pt}解兄到此({\akai 或}: 来此)何事?

解张\hspace{30pt}介兄有所不知,今有重耳公子回朝犒赏功臣。你乃有功之臣,何不前去请功受赏?

介子推\hspace{20pt}解兄这些言语,依我看来,尽都是荒谬之言({\akai 或}: 依弟看来,尽是些荒谬之言)。

解张\hspace{30pt}何谓荒谬之言?

\setlength{\hangindent}{56pt}{
	介子推\hspace{20pt}既是公子重耳回朝({\akai 或}: 还朝复位),犒赏功臣,就该有旨前来,召我入朝({\akai 或}: 还朝)。今无旨意到此({\akai 或}: 今无圣命),岂不是({\akai 或}: 岂非)荒谬之言?}

解张\hspace{30pt}介兄啊!

\setlength{\hangindent}{56pt}{ 
解张\hspace{30pt}【{\akai 西皮原板}】莫道榜文是虚谎,老汉言来听端详: 昔日有个姜吕望,八十三岁遇文王。他也曾保主江山创,他也曾领兵去伐商。既是重耳加封赏,就该前去见君王。 }

(介子推\hspace{20pt}解兄。)

\setlength{\hangindent}{56pt}{
介子推\hspace{20pt}【{\akai 西皮原板}】解兄不必说比方,弟今言来听心旁: 讲什么兴周姜吕望,讲什么伐纣周武王。我不啊为官身闲荡,散淡逍遥侍奉高堂({\akai 或}: 侍奉萱堂)。 }

\setlength{\hangindent}{56pt}{ 
解张\hspace{30pt}【{\akai 西皮原板}】介兄不必【{\footnotesize 转}{\akai 西皮二六}】性刚强,你本盖世一忠良。周游列国随驾往,割股之功天下扬。}

\setlength{\hangindent}{56pt}{介子推\hspace{20pt}【{\akai 西皮二六}】你道为官把名扬,哪知为官无下场。一十九载【{\footnotesize 转}{\akai 西皮快板}】远飘荡,鞍前马后伴君王。重耳为君无度量,弃旧迎新理不当。既是有功该受赏,三冬的梅花自然香。冷眼识破君行状,不在山林伴虎狼。蟒袍玉带我不想,侍奉萱堂在故乡。}

\setlength{\hangindent}{56pt}{解张\hspace{30pt}【{\akai 西皮摇板}】介兄生来秉性刚,必定不肯入朝堂。施罢一礼出草堂, }

解张\hspace{30pt}告辞了。

介子推\hspace{20pt}少送。

解张\hspace{30pt}【{\akai 西皮摇板}】休怪老汉语癫狂。 

解张\hspace{30pt}哈哈哈$\cdots{}\cdots{}$({\hwfs 笑}{\hwfs 介})

\setlength{\hangindent}{56pt}{介子推\hspace{20pt}【{\akai 西皮摇板}】好一个邻舍老解张({\akai 或}: 好一个仁义老解张),絮絮滔滔语言长({\akai 或}: 话绵长)。他劝我入朝请封赏,怎知({\akai 或}: 哪知)我不愿奉君王({\akai 或}: 伴君王)。闷恹恹且坐草堂上, }

介母\hspace{30pt} 【{\akai 西皮摇板}】母子寂寞苦度时光。 

介子推\hspace{20pt}孩儿拜揖!  

介母\hspace{30pt}罢了,一旁坐下!

介子推\hspace{20pt}谢母亲!({\akai 或}: 谢座。)

介母\hspace{30pt}儿啊,方才何人到此?

介子推\hspace{20pt}邻舍解张到此。

介母\hspace{30pt}到此何事?

介子推\hspace{20pt}(是)他言道,今有公子重耳回朝({\akai 或}: 还朝复位),犒赏功臣,(他)教孩儿前去请封受赏。

介母\hspace{30pt}哦,既然如此,我儿就该前去才是。

\setlength{\hangindent}{56pt}{
	介子推\hspace{20pt}母亲,重耳既然犒赏功臣,就该有旨召我入朝({\akai 或}: 就该有旨前来,将孩儿召回朝去)。今无圣旨到来({\akai 或}: 今无圣命),岂不是把孩儿看成朽木一般了么({\akai 或}: 岂非将儿看作朽木一般)?}

\setlength{\hangindent}{56pt}{介子推\hspace{20pt}【{\akai 西皮原板}】晋君复位坐朝堂,犒赏功臣举栋梁。我昔年有功他全忘,做一个({\akai 或}: 落一个)清闲自在行孝儿郎。 }

\setlength{\hangindent}{56pt}{介母\hspace{30pt}【{\akai 西皮原板}】我的儿说话欠思量,且听为娘说端详:~既是重耳论功赏,我的儿前去又有何妨。 }

(介子推\hspace{20pt} 母亲!)

\setlength{\hangindent}{56pt}{介子推\hspace{20pt}【{\akai 西皮二六}】四四方方一垛墙,许多的迷人内中藏。有人跳出是非网,才能得蓬莱不老方。 }

\setlength{\hangindent}{56pt}{介母\hspace{30pt}【{\akai 西皮摇板}】子推生来性情刚,执意不肯回朝廊。晋君回朝行封赏,还是回朝讨风光。 }

\setlength{\hangindent}{56pt}{介子推\hspace{20pt}【{\akai 西皮摇板}】古来({\akai 或}: 自古)多少忠良将,哪个忠良有下场: 比干谏奏({\akai 或}: 比干丞相)把命丧,微子见机先逃亡哇。越思越想心火上,儿誓死({\akai 或}: 儿至死)不愿回朝廊({\akai 或}: 回朝堂)。 }

\setlength{\hangindent}{56pt}{介母\hspace{30pt}【{\akai 西皮摇板}】我的儿不愿去请功受赏,母子们坐草堂细作商量。 }

介母\hspace{30pt}我儿不去请功受赏,唉,也罢,我母子就该隐居起来才是。

介子推\hspace{20pt}啊母亲,此处有一绵山,高山峻岭,倒可安身。

介母\hspace{30pt}待为娘收拾包裹衣服。就此前往。

介子推\hspace{20pt}(唉!)好个贤德老母!

\setlength{\hangindent}{56pt}{介子推\hspace{20pt}【{\akai 西皮摇板}】老母贤德世无双({\akai 或}: 老母年迈六旬上),助儿埋名实贤良\footnote{ 此句吴焕老师整理的剧本记作``故而埋名是贤良''。}。脱衣去巾\footnote{ 吴焕老师整理的剧本记作``脱衣去襟''。}呐草堂上,老母到此({\akai 或}: 老母到来)离村庄。 }

\setlength{\hangindent}{56pt}{介母\hspace{30pt}【{\akai 西皮摇板}】收拾包裹与行囊,母子一同逃外乡。母子们双双跪草堂,祖先爷呀,拜别祖先泪汪汪。 }

\setlength{\hangindent}{56pt}{介母\hspace{30pt}【{\akai 西皮摇板}】用手拨开名利网, }

\setlength{\hangindent}{56pt}{介子推\hspace{20pt}【{\akai 西皮摇板}】翻身跳出是非墙。 }

介母\hspace{30pt}儿啊,咱们的家园$\cdots{}\cdots{}$

介子推\hspace{20pt}唉!不要了。

介母\hspace{30pt}唉,喂呀$\cdots{}\cdots{}$({\hwfs 哭}{\hwfs 介})

\vspace{3pt}{\centerline{{[}{\hei 第二场}{]}}}\vspace{5pt}

\setlength{\hangindent}{56pt}{介母\hspace{30pt}【{\akai 西皮摇板}】这几年奔走在天涯, }

\setlength{\hangindent}{56pt}{介子推\hspace{20pt}【{\akai 西皮摇板}】撇却({\akai 或}: 不贪)富贵与荣华。 }

介子推\hspace{20pt}母亲,来此已是绵山。

介母\hspace{30pt}高山峻岭,叫为娘怎样上去?

介子推\hspace{20pt}老娘!

\setlength{\hangindent}{56pt}{介子推\hspace{20pt}【{\akai 西皮二六}】绵山峻险多峰岬\footnote{ 吴焕老师整理的剧本记作``峰峡''。},四壁巍峨景物呃佳。云环峻岭({\akai 或}: 云环翠岭)雁难下,那涧下(的)清泉照眼花。猿猴、麋鹿衔枝耍,喜鹊依枝({\akai 或}: 喜鹊争鸣)叫喧哗。手攀藤条上山崖, }

\setlength{\hangindent}{56pt}{介子推\hspace{20pt}【{\akai 西皮摇板}】古树森森({\akai 或}: 古树幽森)可为家。 }

介子推\hspace{20pt}此处可好安身?

介母\hspace{30pt}唉,正好安身,只是难以度日呀!

介子推\hspace{20pt}老娘!

\setlength{\hangindent}{56pt}{介子推\hspace{20pt}【{\akai 西皮二六}】老娘亲休得要来嗟呀,草衣木食度年华。无是无非多潇洒, }

\setlength{\hangindent}{56pt}{介子推\hspace{20pt}【{\akai 西皮摇板}】胜似蓬莱第一家。 }

\setlength{\hangindent}{56pt}{介母\hspace{30pt}【{\akai 西皮摇板}】我的儿说的是哪里话,瓜果怎能度日华。母子且把山岗下, }

\setlength{\hangindent}{56pt}{介子推\hspace{20pt}【{\akai 西皮摇板}】我把那名利({\akai 或}: 我把那功劳)二字哇付与尘沙({\akai 或}: 付与流沙)。 }

\vspace{3pt}{\centerline{{[}{\hei 第三场}{]}}}\vspace{5pt}

介子推\hspace{20pt}({\akai 内})【{\akai 西皮导板}】春草青青隐翠微呀,

\setlength{\hangindent}{56pt}{介子推\hspace{20pt}【{\akai 西皮原板}】老母叮咛结草衣\footnote{ 《京剧汇编》第七十四集作``结草依''。}。山高也有长流水呀,杜鹃不住花前啼。晋重耳归国登龙位,割股功劳({\akai 或}: 割股之功)全不提。劝世人莫贪名和利,朝东暮西却为谁({\akai 或}: 为了谁)。纵然是争得呀三公位,难免荒郊坟土堆。我好比鱼儿惊钩起,我好比杨花信风吹({\akai 或}: 随风吹)。我好比孤凤丹山立\footnote{ 吴焕老师整理的剧本记作``孤凤单山立''。},我好比鸿雁隐山栖({\akai 或}: 飞雁隐山栖)。 }

\setlength{\hangindent}{56pt}{介子推\hspace{20pt}【{\akai 西皮散板}】霎时遍地({\akai 或}: 霎时一阵)风沙起,雀鸟不住往空飞\footnote{ 作``望空飞''似亦通。}。 }

\setlength{\hangindent}{56pt}{介子推\hspace{20pt}【{\akai 西皮散板}】金鼓呐喊听耳底,教人心中费猜疑({\akai 或}: 起猜疑)。 }

\setlength{\hangindent}{56pt}{介子推\hspace{20pt}【{\akai 西皮散板}】站立山头用目觑呀:  }

\setlength{\hangindent}{56pt}{介子推\hspace{20pt}【{\akai 西皮快板}】满山人马似云飞\footnote{ 吴焕老师整理的剧本记作``似影飞''。}。五色旌旗空中立,刀枪剑戟摆得齐。见几个头戴双凤翅,见几个身穿衮龙衣。见几个怀抱双环镋\footnote{ 吴焕老师整理的剧本记作``双环档''。},见几个怀抱打将锤。莫不是哪国烟尘起,莫不是重耳把兵提。莫不是来把绵山洗\footnote{ 吴焕老师整理的剧本记作``来把绵山袭''。},莫不是来访介子推。越思越想心火起呀, }
\setlength{\hangindent}{56pt}{介子推\hspace{20pt}【{\akai 西皮快板}】一腔怒气往上提。我也曾对天发宏誓,永不还朝挂紫衣。任你搜来任你洗,稳坐绵山永不离。 }

\vspace{3pt}{\centerline{{[}{\hei 第四场}{]}}}\vspace{5pt}

\setlength{\hangindent}{56pt}{介子推\hspace{20pt}【{\akai 西皮散板}】四下人马齐围困,重耳带兵搜山林。回头便把母亲请({\akai 或}: 忙把老娘请), }

\setlength{\hangindent}{56pt}{介母\hspace{30pt}【{\akai 西皮散板}】我儿为何着了惊({\akai 或}: $\cdots{}\cdots{}$为何情)。 }

\setlength{\hangindent}{56pt}{介子推\hspace{20pt}【{\akai 西皮散板}】老母({\akai 或}: 老娘)有所不知情,重耳入山将儿寻。 }

\setlength{\hangindent}{56pt}{介母\hspace{30pt}【{\akai 西皮散板}】既是重耳把你请,我儿就该去见君。 }

\setlength{\hangindent}{56pt}{介子推\hspace{20pt}【{\akai 西皮散板}】母亲说话欠思忖,孩儿立誓不回程。({\akai 或}: 曾对苍天发誓盟,至死也不转回程。)哪怕人马重重紧,教儿下山万不能({\akai 或}: 想儿下山万不能)。 }

介母\hspace{30pt}儿往哪里安身?

介子推\hspace{20pt}随儿来啊!

\vspace{3pt}{\centerline{{[}{\hei 第五场}{]}}}\vspace{5pt}


\setlength{\hangindent}{56pt}{介子推\hspace{20pt}【{\akai 西皮散板}】搀定老娘东山进({\akai 或}: 东山隐),隐姓埋名谁知情。 }

\setlength{\hangindent}{56pt}{介子推\hspace{20pt}【{\akai 西皮散板}】东山人马乱纷纭,母子无处把身存呐。 }

\vspace{3pt}{\centerline{{[}{\hei 第六场}{]}}}\vspace{5pt}

\setlength{\hangindent}{56pt}{介子推\hspace{20pt}【{\akai 西皮散板}】搀定老娘西山进({\akai 或}: 西山隐),西山里面({\akai 或}: 西山以内)躲朝廷。 }

\setlength{\hangindent}{56pt}{介子推\hspace{20pt}【{\akai 西皮散板}】西山人马似麻林,倒教子推无计行。 }

介母\hspace{30pt}为娘不耐烦了。

\setlength{\hangindent}{56pt}{介子推\hspace{20pt}【{\akai 西皮导板}】劝老娘要耐烦随儿投奔({\akai 或}: 随儿逃奔)。 }

\vspace{3pt}{\centerline{{[}{\hei 第七场}{]}}}\vspace{5pt}

介子推\hspace{20pt}哎呀!

\setlength{\hangindent}{56pt}{介子推\hspace{20pt}【{\akai 西皮散板}】只见四下烈火升呐,重耳放火烧山林。回头再呀把({\akai 或}: 回头忙把;急忙再把)老娘请, }

\setlength{\hangindent}{56pt}{介母\hspace{30pt}【{\akai 西皮散板}】我儿着急为何情。 }

\setlength{\hangindent}{56pt}{介子推\hspace{20pt}【{\akai 西皮散板}】重耳做事心太狠,不该举火绵山焚。 }

\setlength{\hangindent}{56pt}{介母\hspace{30pt}【{\akai 西皮散板}】重耳放火烧山林,快背为娘去见君。 }

\setlength{\hangindent}{56pt}{介子推\hspace{20pt}【{\akai 西皮散板}】任把绵山火焚尽({\akai 或}: 俱焚尽),情愿一死不回程({\akai 或}: 至死也不去见君)。 }

介母\hspace{30pt}哎呀!

\setlength{\hangindent}{56pt}{介母\hspace{30pt}【{\akai 西皮散板}】绵山好比酆都城,要想活命万不能。 }

\setlength{\hangindent}{56pt}{介子推\hspace{20pt}【{\akai 西皮散板}】搀定老娘东山隐({\akai 或}: 东山临\footnote{ 吴焕老师整理的剧本记作``东山岭'';《京剧汇编》第七十四集作``东山进''。}), }

\setlength{\hangindent}{56pt}{介子推\hspace{20pt}【{\akai 西皮散板}】火光四起吓煞人。 }

\setlength{\hangindent}{56pt}{介子推\hspace{20pt}【{\akai 西皮散板}】搀定老娘西山岭, }

\setlength{\hangindent}{56pt}{介子推\hspace{20pt}【{\akai 西皮散板}】四下烈火难存身。 }

\setlength{\hangindent}{56pt}{介子推\hspace{20pt}【{\akai 西皮散板}】搀定老娘({\akai 或}: 背定老娘)上山呐岭,\footnote{ 陈超老师注: 此时台上摆放横场桌,两把椅子。老生从小边上椅子,老旦滑下来头冲里躺地下,老生上桌发现老旦落山,在桌子上甩发``屁股坐子''甩发盖脸,挡脸,蹬椅子吊毛下桌。} }

\setlength{\hangindent}{56pt}{介子推\hspace{20pt}【{\akai 西皮散板}】儿的老娘啊! }
}

\newpage
\hypertarget{ux711aux7ef5ux5c71-ux4e4b-ux4ecbux5b50ux63a8ux4ecbux6bcdux89e3ux5f20}{%
\subsection{\texorpdfstring{焚绵山\protect\hyperlink{fn36}{\textsuperscript{36}}
之
介子推\protect\hyperlink{fn37}{\textsuperscript{37}}、介母、解张}{焚绵山36 之 介子推37、介母、解张}}\label{ux711aux7ef5ux5c71-ux4e4b-ux4ecbux5b50ux63a8ux4ecbux6bcdux89e3ux5f20}}

{[}第一场{]}

介子推 {[}引子{]}弃官离朝(或:弃职离朝),名利抛,侍奉年高。

介子推
(念)见机逃出是非地,落得清闲自在身。冷眼识破君王意,功成身退奉呃娘亲。

介子推
卑人,介(子)推。晋献公驾前为臣,官居谏议大夫(之职)。只因吾主(或:只因君王无道,)听信谗言,毒害大臣,害死申生太子。是我与狐毛、狐偃、颠颉、魏犨等九人,保定(公子)重耳逃出罗网,周游列国一十九载。归来共渡黄河,是俺看破其意(或:是我识破其意),弃职归林。正是:

\begin{quote}
(念)奉君何足乐,还是孝呃当先。
\end{quote}

介子推
【西皮慢板】介子推坐草堂前思后想,想起了晋国事好不凄凉。晋献公听信那谗言毁谤,宠骊姬害申生命赴黄粱。我十人保重耳呐【转西皮原板】逃出罗网,朝同行夜同寝(或:夜共寝)伴随君旁。我也曾在荒郊觅食取浆,我也曾割股肉奉献君王。渡黄河一个个争呃功邀赏啊,又谁知出恶言暗把人伤。既这等做什么忠呃臣良将,因此上怀戒心弃职还乡。回家来织呐履舄\protect\hyperlink{fn38}{\textsuperscript{38}}心宽【回龙】意爽,

介子推 【西皮摇板】斩断了名利锁侍奉老娘。

解张 【西皮摇板】晋公子登龙位各官封赠,相劝那介子推前去面君。

解张 来此已是,介兄在家么?

介子推 是哪位?

介子推 哦,原来是解兄,请进!

解张 请,这厢有礼。

介子推 还礼,请坐!

解张 有座。

介子推 解兄到此(或:来此)何事?

解张
介兄有所不知,今有重耳公子回朝犒赏功臣。你乃有功之臣,何不前去请功受赏?

介子推
解兄这些言语,依我看来,尽都是荒谬之言(或:依弟看来,尽是些荒谬之言)。

解张 何谓荒谬之言?

介子推
既是公子重耳回朝(或:还朝复位),犒赏功臣,就该有旨前来,召我入朝(或:还朝)。今无旨意到此(或:今无圣命),岂不是(或:岂非)荒谬之言?

解张 介兄啊!

解张
【西皮原板】莫道榜文是虚谎,老汉言来听端详:昔日有个姜吕望,八十三岁遇文王。他也曾保主江山创,他也曾领兵去伐商。既是重耳加封赏,就该前去见君王。

(介子推 解兄。)

介子推
【西皮原板】解兄不必说比方,弟今言来听心旁:讲什么兴周姜吕望,讲什么伐纣周武王。我不啊为官身闲荡,散淡逍遥侍奉高堂(或:侍奉萱堂)。

解张
【西皮原板】介兄不必【转西皮二六】性刚强,你本盖世一忠良。周游列国随驾往,割股之功天下扬。

介子推
【西皮二六】你道为官把名扬,哪知为官无下场。一十九载【转西皮快板】远飘荡,鞍前马后伴君王。重耳为君无度量,弃旧迎新理不当。既是有功该受赏,三冬的梅花自然香。冷眼识破君行状,不在山林伴虎狼。蟒袍玉带我不想,侍奉萱堂在故乡。

解张 【西皮摇板】介兄生来秉性刚,必定不肯入朝堂。施罢一礼出草堂,

解张 告辞了。

介子推 少送。

解张 【西皮摇板】休怪老汉语癫狂。

解张 哈哈哈\ldots{}\ldots{}(笑介)

介子推
【西皮摇板】好一个邻舍老解张(或:好一个仁义老解张),絮絮滔滔语言长(或:话绵长)。他劝我入朝请封赏,怎知(或:哪知)我不愿奉君王(或:伴君王)。闷恹恹且坐草堂上,

介母 【西皮摇板】母子寂寞苦度时光。

介子推 孩儿拜揖!

介母 罢了,一旁坐下!

介子推 谢母亲!(或:谢座。)

介母 儿啊,方才何人到此?

介子推 邻舍解张到此。

介母 到此何事?

介子推
(是)他言道,今有公子重耳回朝(或:还朝复位),犒赏功臣,(他)教孩儿前去请封受赏。

介母 哦,既然如此,我儿就该前去才是。

介子推
母亲,重耳既然犒赏功臣,就该有旨召我入朝(或:就该有旨前来,将孩儿召回朝去)。今无圣旨到来(或:今无圣命),岂不是把孩儿看成朽木一般了么(或:岂非将儿看作朽木一般)?

介子推
【西皮原板】晋君复位坐朝堂,犒赏功臣举栋梁。我昔年有功他全忘,做一个(或:落一个)清闲自在行孝儿郎。

介母
【西皮原板】我的儿说话欠思量,且听为娘说端详。既是重耳论功赏,我的儿前去又有何妨。

(介子推 母亲!)

介子推
【西皮二六】四四方方一垛墙,许多的迷人内中藏。有人跳出是非网,才能得蓬莱不老方。

介母
【西皮摇板】子推生来性情刚,执意不肯回朝廊。晋君回朝行封赏,还是回朝讨风光。

介子推
【西皮摇板】古来(或:自古)多少忠良将,哪个忠良有下场:比干谏奏(或:比干丞相)把命丧,微子见机先逃亡哇。越思越想心火上,儿誓死(或:儿至死)不愿回朝廊(或:回朝堂)。

介母 【西皮摇板】我的儿不愿去请功受赏,母子们坐草堂细作商量。

介母 我儿不去请功受赏,唉,也罢,我母子就该隐居起来才是。

介子推 啊母亲,此处有一绵山,高山峻岭,倒可安身。

介母 待为娘收拾包裹衣服。就此前往。

介子推 (唉!)好个贤德老母!

介子推
【西皮摇板】老母贤德世无双(或:老母年迈六旬上),助儿埋名实贤良\protect\hyperlink{fn39}{\textsuperscript{39}}。脱衣去巾\protect\hyperlink{fn40}{\textsuperscript{40}}呐草堂上,老母到此(或:老母到来)离村庄。

介母
【西皮摇板】收拾包裹与行囊,母子一同逃外乡。母子们双双跪草堂,祖先爷呀,拜别祖先泪汪汪。

介母 【西皮摇板】用手拨开名利网,

介子推 【西皮摇板】翻身跳出是非墙。

介母 儿啊,咱们的家园\ldots{}\ldots{}

介子推 唉!不要了。

介母 唉,喂呀\ldots{}\ldots{}(哭介)

{[}第二场{]}

介母 【西皮摇板】这几年奔走在天涯,

介子推 【西皮摇板】撇却(或:不贪)富贵与荣华。

介子推 母亲,来此已是绵山。

介母 高山峻岭,叫为娘怎样上去?

介子推 老娘!

介子推
【西皮二六】绵山峻险多峰岬\protect\hyperlink{fn41}{\textsuperscript{41}},四壁巍峨景物呃佳。云环峻岭(或:云环翠岭)雁难下,那涧下(的)清泉照眼花。猿猴、麋鹿衔枝耍,喜鹊依枝(或:喜鹊争鸣)叫喧哗。手攀藤条上山崖,

介子推 【西皮摇板】古树森森(或:古树幽森)可为家。

介子推 此处可好安身?

介母 唉,正好安身,只是难以度日呀!

介子推 老娘!

介子推 【西皮二六】老娘亲休得要来嗟呀,草衣木食度年华。无是无非多潇洒,

介子推 【西皮摇板】胜似蓬莱第一家。

介母 【西皮摇板】我的儿说的是哪里话,瓜果怎能度日华。母子且把山岗下,

介子推
【西皮摇板】我把那名利(或:我把那功劳)二字哇付与尘沙(或:付与流沙)。

{[}第三场{]}

介子推 (内)【西皮导板】春草青青隐翠微呀,

介子推
【西皮原板】老母叮咛结草衣\protect\hyperlink{fn42}{\textsuperscript{42}}。山高也有长流水呀,杜鹃不住花前啼。晋重耳归国登龙位,割股功劳(或:割股之功)全不提。劝世人莫贪名和利,朝东暮西却为谁(或:为了谁)。纵然是争得呀三公位,难免荒郊坟土堆。我好比鱼儿惊钩起,我好比杨花信风吹(或:随风吹)。我好比孤凤丹山立\protect\hyperlink{fn43}{\textsuperscript{43}},我好比鸿雁隐山栖(或:飞雁隐山栖)。

介子推
【西皮散板】霎时遍地(或:霎时一阵)风沙起,雀鸟不住往空飞\protect\hyperlink{fn44}{\textsuperscript{44}}。

介子推 【西皮散板】金鼓呐喊听耳底,教人心中费猜疑(或:起猜疑)。

介子推 【西皮散板】站立山头用目觑呀:

介子推
【西皮快板】满山人马似云飞\protect\hyperlink{fn45}{\textsuperscript{45}}。五色旌旗空中立,刀枪剑戟摆得齐。见几个头戴双凤翅,见几个身穿衮龙衣。见几个怀抱双环镋\protect\hyperlink{fn46}{\textsuperscript{46}},见几个怀抱打将锤。莫不是哪国烟尘起,莫不是重耳把兵提。莫不是来把绵山洗\protect\hyperlink{fn47}{\textsuperscript{47}},莫不是来访介子推。越思越想心火起呀,

介子推
【西皮快板】一腔怒气往上提。我也曾对天发宏誓,永不还朝挂紫衣。任你搜来任你洗,稳坐绵山永不离。

{[}第四场{]}

介子推
【西皮散板】四下人马齐围困,重耳带兵搜山林。回头便把母亲请(或:忙把老娘请),

介母 【西皮散板】我儿为何着了惊(或:\ldots{}\ldots{}为何情)。

介子推 【西皮散板】老母(或:老娘)有所不知情,重耳入山将儿寻。

介母 【西皮散板】既是重耳把你请,我儿就该去见君。

介子推
【西皮散板】母亲说话欠思忖,孩儿立誓不回程。(或:曾对苍天发誓盟,至死也不转回程。)哪怕人马重重紧,教儿下山万不能(或:想儿下山万不能)。

介母 儿往哪里安身?

介子推 随儿来啊!

{[}第五场{]}

介子推 【西皮散板】搀定老娘东山进(或:东山隐),隐姓埋名谁知情。

介子推 【西皮散板】东山人马乱纷纭,母子无处把身存呐。

{[}第六场{]}

介子推
【西皮散板】搀定老娘西山进(或:西山隐),西山里面(或:西山以内)躲朝廷。

介子推 【西皮散板】西山人马似麻林,倒教子推无计行。

介母 为娘不耐烦了。

介子推 【西皮导板】劝老娘要耐烦随儿投奔(或:随儿逃奔)。

{[}第七场{]}

介子推 哎呀!

介子推
【西皮散板】只见四下烈火升呐,重耳放火烧山林。回头再呀把(或:回头忙把;急忙再把)老娘请,

介母 【西皮散板】我儿着急为何情。

介子推 【西皮散板】重耳做事心太狠,不该举火绵山焚。

介母 【西皮散板】重耳放火烧山林,快背为娘去见君。

介子推
【西皮散板】任把绵山火焚尽(或:俱焚尽),情愿一死不回程(或:至死也不去见君)。

介母 哎呀!

介母 【西皮散板】绵山好比酆都城,要想活命万不能。

介子推
【西皮散板】搀定老娘东山隐(或:东山临\protect\hyperlink{fn48}{\textsuperscript{48}}),

介子推 【西皮散板】火光四起吓煞人。

介子推 【西皮散板】搀定老娘西山岭,

介子推 【西皮散板】四下烈火难存身。

介子推
【西皮散板】搀定老娘(或:背定老娘)上山呐岭,\protect\hyperlink{fn49}{\textsuperscript{49}}

介子推 【西皮散板】儿的老娘啊!

\newpage
\hypertarget{ux6458ux7f28ux4f1a}{%
\subsection{摘缨会}\label{ux6458ux7f28ux4f1a}}

{[}第一场{]}

(打朝众官上\textless{}\textbf{点绛唇}\textgreater{}报名)

众报名
大司马申无畏(台顶,红三块瓦,黪满)。上大夫苏从(夫子盔,黪三)。下大夫虞邱(荷叶盔,黑三)。上将军熊负羁(帅盔,黑三)。车左将军潘尪(踏镫,紫三块瓦,黑满)。车右将军乐伯(倒缨盔,武生)。左殿将军公子婴齐(虎头盔,武生)。右殿将军公子侧(虎头盔、元宝脸、黑一字)。左队先锋养由基(硬扎巾、大额子、武生)。右队先锋襄老(狮子盔、老丑)。(左右分站)

申无畏 列公请了。

众 请了。

申无畏 大王设朝,两厢伺候。

众 请。

(左右分下,\textless{}\textbf{大锣打下}\textgreater{}接\textless{}\textbf{打上}\textgreater{})

{[}第二场{]}

(四太监站门、大太监、庄王\textless{}\textbf{大锣打上}\textgreater{},庄王九龙冠,黄开氅)

庄王 (引)征战戎夷,立功勋,息卷旌旗(另本: 盛世兴隆,图霸业,一统江洪)。

(正面小座念)

庄王
为征陆浑(或:为征陆戎)动兵机,越椒无理(或:越椒无故)把孤欺。幸有众卿逞雄力,才得凯歌马停蹄。(另本:
楚国逞强有数秋,龙争虎斗统貔貅。图霸中原孤为首,山河一统乐无忧。)孤,熊旅。坐镇荆州,承先王基业,复立楚国,三年以来号令未申。大夫申无畏、苏从直谏,重整先王原例,是孤欲图中原。适逢川贼陆戎兄妹造反,孤御驾亲征,在朝摘去斗越椒兵权印信(或:兵权信印),不想奸贼妒贤嫉能(或:嫉贤妒能),司马蒍贾满门尽丧,斗贼兴兵劫驾,多亏养由基灭贼除患,保孤回朝。寡人今日设宴渐台,犒赏群臣(或:犒赏三军),以表孤(王)爱戴贤臣之意。内侍,与孤传旨,不论裨、副牙将大小官员(或:不论偏、副牙将,大小将官),满朝文武齐上渐台领宴共赏。

太监 大王有旨,不论裨副牙将大小官员,满朝文武齐上渐台领宴共赏。

内众 领旨。

(两边分上,四牙将左右跟上,牙将软靠,唐狡红靠、扎巾盔,站大边外角)

众 臣等参见大王。

庄王 众卿平身。

众 千千岁。

(分站左右)

申无畏 宣臣等上殿有何旨意?

庄王 孤今得胜还朝(或:孤王今日得胜回朝),在渐台设宴,与众卿同饮。

申无畏 臣等领旨。

庄王 内侍。

太监 有。

庄王 酒宴可曾齐备?

太监 俱已齐备。

庄王 带路渐台。

太监 带路渐台呀。

{[}连场第二场{]}

(``\textbf{吹打}''众转场,庄王最后下,众又领上上渐台,庄上正面小帐子高台大座,大太监大边椅子上,众分坐两边大座,唐狡坐大边最外椅,牌子停)

庄王 看宴。(众卿请。)

众 千岁请。

(\textless{}\textbf{玉芙蓉}\textgreater{}前段,饮酒,饮完)

庄王 内侍。

太监 有。

庄王 宣许娘娘上渐台。

(太监下椅子)

太监 大王有旨,宣许娘娘上渐台。

内众 领旨。

(西皮\textless{}\textbf{小开门}\textgreater{},四宫女引许姬上念)

许姬 深感隆恩君锡宠,轻移莲步上渐台。

(宫女挖门站高台后侧两边,许姬参拜)

许姬 妾妃见驾,大王千岁。

庄王 梓童平身。

许姬 千千岁。(许姬上小边椅)宣妾妃上渐台不知有何旨意?

庄王
孤王今日设宴犒赏功臣,梓童在众臣(或:在众将)席前行酒一巡,以表孤王爱戴贤臣之意。

许姬 妾闻男女不渎,何况君臣?

申无畏、苏从 臣等叩沐恩宠,何敢再劳娘娘赐酒。

庄王
孤王今日设宴,原为君臣共欢,众卿不必谦逊(或:不必过谦),孤王言重如山,岂能反悔,梓童巡酒。

许姬 领旨。

(许下椅唱【西皮二六】,太监随下端壶)

许姬
【西皮二六】征夷戎立功勋天时顺应,诛越椒回朝转河晏海清。捧香醪献君王虔诚恭敬(给庄王斟酒),愿吾主乐陶陶福寿康宁。我这里奉王命把酒侍饮(给大边臣斟),在宴前依次序来敬杯巡(给小边臣斟)。似这等有道君万民之幸(给小边牙将斟),辅国家秉忠义万世留名(给唐狡连斟三杯)。耳听得风声响阶檐震动(壶递太监),

许姬 【西皮散板】霎时间阖台上黑暗不明。

(\textless{}\textbf{乱锤}\textgreater{},唐狡,许姬推磨,许摘唐狡缨,上小边椅与庄王语、递缨即面牌,狡坐下)

申无畏、苏从 快快点烛。

庄王 慢、慢\ldots{}\ldots{}掌灯。

(众应)

庄王 众卿。

众 大王。

庄王
今日此宴原为君臣共饮(或:君臣共欢),诸事无忌,孤王出一酒令,众卿可将盔缨摘下,丢过席前,作一绝缨大会,但有不遵者,孤王不以法度治之,只罚酒三巨觥。

众 臣等遵旨。

(\textless{}\textbf{冲头}\textgreater{}中太监拿盘两边收缨,锣中唐狡摸地找缨不见仍坐下,太监收完端盘上椅放盘在桌上)

庄王 可曾摘齐?

众 俱已摘齐。

庄王 吩咐掌灯。

(太监、众应)

(庄王左右两望)

庄王 哈\ldots{}\ldots{}许姬回宫。

(\textless{}\textbf{小开门}\textgreater{},许姬下椅,拜辞,宫女领下)

许姬 【西皮散板】暗中牵袂醉中情,玉手如风已绝缨。

(许姬下)

庄王 众卿再饮一回(或:再饮几回)。

众 臣等酒足,请驾回宫。

庄王 退班。

(\textless{}\textbf{玉芙蓉}合头\textgreater{},庄王众窝下。留唐狡,狡比势、摸颈,怕介,下)

{[}第三场{]}

(\textless{}\textbf{水底鱼}\textgreater{}公子宋上,着武生巾、黑三、白箭衣、黑马褂,持马鞭)

公子宋
晋国去求兵,要把楚邦平。俺郑国大夫公子宋是也,只因楚国屡与郑国不和,横行天下,奉主钧旨去往晋国借兵,共敌楚邦,马不宜迟火速躜行。

(\textless{}\textbf{水底鱼}\textgreater{}下)

{[}第四场{]}

(四宫女站门,许姬上)

许姬
【西皮慢板】吾主爷在渐台庆功犒赏,料不想无知辈酒后癫狂。摘盔缨奏大王未把罪降,

(许姬小座)

许姬 (接唱)岂可容乱礼法败坏纲常。

太监 (内念)大王回宫啊。

(四小太监上小边一字,大太监引庄王上)

庄王 【西皮摇板】饮罢了功臣宴神清气爽,

(\textless{}\textbf{小开门}\textgreater{}许姬出门接驾,庄王进门正面小座)

许姬 (\textless{}\textbf{小开门}\textgreater{}中念)大王千岁。

庄王 平身。

许姬 千千岁。

庄王 赐座。

许姬 谢座。(坐大边)

许姬 喂呀\ldots{}\ldots{}(\textless{}\textbf{小开门}\textgreater{}停)

庄王 (接唱)【西皮摇板】问梓童因何故面带惆怅(或:心意彷徨;面带彷徨)。

庄王 梓童为何面带愁容?

许姬
启奏大王,大王使妾妃献觞于众臣以示敬意,牵妾之袂,王不加察,何以肃上下之礼,正男女之别也?

庄王
哈哈哈\ldots{}\ldots{}梓童非所知也。孤王犒赏功臣原为君臣共饮(或:君臣共乐),不该白昼连夜,酒后癫狂乃人情之常,孤若查而罪之,一来众臣必然心神沮丧;二者道孤君妃有陷害贤臣之意,三则外邦闻之不雅,故以酒令掩盖(或:以酒令遮掩)岂不三全齐美,毛皮小事(或:此乃小事)梓童何必挂怀。哈哈哈\ldots{}\ldots{}

许姬 大王啊,

许姬
【西皮原板】吾主爷有道君皇恩浩荡,沧海量宽宏度福寿绵长。似尧舜统大业千秋以上,畜鳞鱼忌流水太过清香。

庄王
【西皮慢板】劝梓童休得要把本奏上,听孤王把前情细说端详。都只为斗越椒欺君罔上,他父子掌兵权搅乱家邦。摘去了司马印蒍贾执掌,又谁知那老儿心怀不良。孤兴兵灭陆戎狼烟扫荡,中途路竟叛逆与孤争强。杀司马搜宫院带兵对仗,楚山河险些儿被贼称王。天生来养由基英雄良将,【西皮二六】只杀得他父子鼠窜獐狂。(立)斗越椒生得来性情倔强,清河桥比箭法老贼身亡。才能得阖朝中清平欢畅,江水静郢都宁重整朝纲。因此上在渐台论功行赏,命梓童斟御酒面带彷徨(或:命梓童代孤王赐过了琼浆)。又谁知霎时节狂风天降,吹熄了华堂上银烛无光。文武臣坐端然四无声响,竟有那无知徒酒后癫狂。孤若是查明了把罪来降(或:孤本当查明了把罪来降;或:孤本当查明了把罪下降),怕只怕文武官意沮神伤。论国法本不该行令发放(或:行令放荡),也是孤做此事自有主张(或:也是孤一时里失了主张)。劝梓童把此事休挂心上,劝梓童把此事付与(了)汪洋。劝梓童与孤王同欢同畅,劝梓童与孤王同酌同觞。宫娥女掌银灯引归罗帐,

(宫女斜门,庄王收腿)

庄王 【西皮摇板】孤与你(或:孤和你)同偕老地久天长。

(庄王、许姬下,宫女随下。\textless{}\textbf{小锣打下}\textgreater{})

{[}第五场{]}

(\textless{}\textbf{小锣打上}\textgreater{}四太监引晋成王上,勾蓝三块瓦,戴黑满、草王盔,着绿蟒)

晋成王 {[}引{]}周室东迁,恨楚庄独霸横行。

(正面大座)

晋成王
(念)周室衰微中原丧,举都东迁移洛阳。群雄并起刀兵攘,楚庄横行霸一方。

晋成王
孤晋侯是也。坐镇绛州,边邦平静,可恨楚庄欲霸中原,为此孤王每日操兵演将,与楚相斗雌雄,今当接报之期,设朝御览。内待展放龙棚。

太监 展放龙棚。

先蔑 (内)呵吓。

(\textless{}\textbf{四击头}\textgreater{}先蔑上,勾黑花三块瓦,着黑满,紫金盔、翎,着黑硬靠、黑蟒,持牙笏,枪,或红三块瓦、红靠蟒)

先蔑 (念)郑国请兵将,把本奏丹墀。(进门参拜)臣先蔑见驾,大王千岁。

晋成王 平身。

先蔑 千千岁!

晋成王 赐座。

先蔑 谢座。(坐大边)

晋成王 上殿有何本奏?

先蔑 今有郑大夫公子宋前来请兵征伐楚邦,朝门候旨。

晋成王 呵呀妙哇!孤正欲伐楚,郑国使臣到来合孤意也。宣来见孤。

先蔑 领旨。(先蔑立)宣郑国大夫上殿。

公子宋
(内)领旨。(公子宋上)为救倒悬危,求请上国兵。(宋进门参拜)臣公子宋见驾,大王千岁。

晋成王 大夫平身。

公子宋 千千岁。

晋成王 看座。

公子宋 告坐。

(公子宋坐大边,先蔑过去坐小边)

晋成王 来到我邦有何见谕?

公子宋
只为楚王图霸要灭陈、郑二邦,臣奉主命恳请大王起兵伐楚,小国愿为后队,未知大王意下如何?

晋成王
孤久有伐楚之心,晋郑二国同体相关,大夫回去上复你主,孤王提兵伐楚,倘有不胜再来接应。

公子宋 如此告退。感谢君金诺,同心伐楚邦。

(公子宋下,先蔑送,回来坐大边)

晋成王
先卿,孤命你为上将军元帅,统领公子凯、公子有,全军人马,兵伐楚邦,即日兴师。下殿。

先蔑 领旨。(念)统领虎豹士,扫荡楚强兵。(先蔑下)

晋成王
内侍,传孤旨意,命荀林父解押粮草,军前使用。正是:(念)两国同心争社稷,何愁海鳌不吞钩!

(晋成王众下)

{[}第六场{]}

(公子凯、公子有着硬靠、扎巾盔,一武生,一花脸,双起霸)

公子凯 杀气腾腾挂铁衣,单枪匹马谁敢欺!

公子有 钢刀一举无人敌,保定大晋锦华夷。

公子凯、公子有 (报名)某,左军先锋公子凯。右军先锋公子有。

公子凯 大司马升帐发兵,你我两厢伺候!

公子有 请。

(四军士打上、站门、先蔑上,\textless{}点绛唇\textgreater{}上高台,二将参)

公子凯 末将打躬。

先蔑 免,站立两厢。

先蔑
(念)凛凛雄师统貔貅,将令一出鬼神愁。号炮一声惊天地,两军对垒凭机谋。

先蔑
某,晋国大司马先蔑。统领全军对敌楚王。啊众将官,此番出兵非比寻常,听本帅令下(\textless{}\textbf{三枪}\textgreater{}牌子)

公子凯、公子有 元帅令出如山,末将等自然奋勇当先。

先蔑 公子凯、公子有听令。

公子凯、公子有 在。

先蔑 命你二人打探楚兵虚实动静,不得有误。

公子凯、公子有 得令。马来!( 公子凯、公子有上马下)

先蔑 众将官,起兵前往。

(先蔑下高台,脱蟒,拿枪。\textless{}小朱奴\textgreater{}牌子,众领先蔑下)

{[}第七场{]}

(\textless{}\textbf{大锣打上}\textgreater{}四龙套、潘尪、伯乐、公子婴齐、公子侧四将着硬靠,站门,庄王上)

庄王 (引子)统领雄师,要把那晋国扫平。(庄正面小座)

庄王
(念)可恨晋邦礼不端,勾结陈、郑起狼烟。孤王领兵(或:孤王带兵)来征战,但愿齐奏凯歌还。

庄王
孤,楚王熊旅,只为图霸王室,扫荡中原。可恨晋邦反复无常,勾结陈、郑,兴兵犯境,为此命苏从、养由基护理国政,孤王亲统大兵(或:孤王御驾亲征),先伐晋国,后灭陈、郑。今命襄老以为前站先行,众位将军,人马可齐?

众 俱已齐备。

庄王 吩咐文武免送,众将随营调遣,起兵前往。

潘尪 起兵前往。(\textless{}\textbf{泣颜回}\textgreater{}上马,众领下)

{[}第八场{]}

(\textless{}\textbf{长锤}\textgreater{}武小生唐狡上,着大叶巾,黑箭衣、红号坎)

唐狡
【西皮摇板】感受君恩未曾报,不该渐台醉酕醄。楚王宽宏量非小,摘缨罪名一笔消。

唐狡
俺,唐狡,棠邑人也,父母早逝,家业凋零。投在楚王驾下当裨将。前者渐台大宴公卿,俺唐狡并无寸箭之功,蒙恩犒赏有名。不想酒后失仪,掠抱君妃暗摘盔缨,自忖性命不保;岂知君王度量宽宏,传旨众臣俱将盔缨摘去,名曰绝缨大会。想我知恩不报非丈夫也。如今晋国前来犯界,楚王御驾亲征,命襄老以为前部先锋。俺不免奔往前部,讨一差使与晋兵对敌,以报君恩也。

唐狡
【西皮摇板】楚王恩德真非小,不把国法斩儿曹。如此宽宏古来少,不辞劳碌报当朝。(唐狡下)

{[}第九场{]}

(四龙套拿枪引襄老上,{[}引子{]},襄着狮子盔、白箭衣、黑马褂、白花开氅)

襄老 {[}引子{]}先行是我,我是先行。

襄老 (念)老将勇猛不可当,全凭精气逞豪强。忠心耿耿扶楚室,何日凯歌转还乡?

襄老
某,襄老是也。大王征战晋国,命我以为前站先行,今日黄道正好发兵。众将官,起兵前往。

众 啊。

唐狡 (内白)住着!

众 有人阻令。

襄老 嗯,何人竟敢阻令,传他进帐。

众 阻令者进帐。

唐狡  (内白)俺来也。(唐狡上)

唐狡 欲为世上奇男子,须建人间未有功。卑将唐狡参见。

襄老 噢,原来是你。大兵正欲起行,你为何阻令?

唐狡 小将自投麾下并未建功;今主将领兵伐晋,小将愿为前站立功报国。

襄老
啊,楚营多少大将,尚且全扣束身\protect\hyperlink{fn50}{\textsuperscript{50}};你一随使将校,胆敢大言阻令,本欲取斩,犹恐出兵不利。还不下去。

唐狡 主将差矣。

唐狡 【西皮散板】唐狡虽然裨将校,胸怀韬略胆气豪。食君粮饷恩当报,

唐狡 主将!

唐狡 (接唱)要与君王扫贼巢。

襄老 嘟,

襄老
【西皮散板】我国大将有多少,遵令钳口不逞豪。小小裨将胡乱道,抗吾军令绑市曹\protect\hyperlink{fn51}{\textsuperscript{51}}。

唐狡
【西皮散板】主将何以气量小,欺压英雄为哪条。年老出令语颠倒,焉能对垒动枪刀。交锋岂论年纪小,

唐狡 主将,

唐狡 (接唱)定把晋国化海潮。

襄老 一派胡言,无知小卒,杀之无益,将唐狡重打四十扯下去。

(二卒、唐狡下,内打,搀上,狡跪念)

唐狡 谢主将责。

襄老 念你帐下多年,留一线之情发往后队,收拾锣锅帐房。下去。

唐狡 哎呀。(唐狡下)

襄老 众将官,起兵前往。

(襄老脱氅,拿枪上马,众领下)

{[}第十场{]}

(\textless{}\textbf{风入松}\textgreater{}头段,龙套引先蔑上,下场门骨牌对)

先蔑 为何不行?

众 来此楚地不远。

先蔑 列开旗门。

(\textless{}\textbf{风入松}\textgreater{}二段,众站门,先蔑站中间)

先蔑 众将官,
楚王出兵多有奸诈,闻得前锋乃是襄老,虽不足惧,但必须人人努力,将他君臣一鼓而擒。

众 啊!

(公子凯、公子有上)

公子凯、公子有 启司马,楚兵扎颖川地方,先行襄老离此不远。

先蔑 啊,楚王亲自出兵,真是天助人愿。众将官,杀上前去。

(\textless{}\textbf{风入松}\textgreater{}三段,先蔑众领起,襄老众抄上,襄龙套下,留襄大边与先架住)

襄老 呔,来将通名。

先蔑 听者,某乃晋国大司马先蔑是也,你这老将通名受死。

襄老 听者,俺乃楚王驾下前站先锋襄老是也。

先蔑 哈哈\ldots{}\ldots{}老弱残兵,非某对手,快教楚王自受其绑。

襄老 孺子,你嫌我老,且试演试演家伙。

(先蔑打襄老败下,先众追下,先耍下场下)

{[}第十一场{]}

(\textless{}\textbf{长锤}\textgreater{}众引庄王上,众站门)

庄王
【西皮摇板】旌旗招展空中飘(或:空飘绕;空中绕),满营将官(或:将士个个)逞英豪。孤王兴兵(或:孤王领兵)把贼扫,

(庄王正面小座)

庄王 (接唱)旗开得胜转还朝。

(襄老\textless{}\textbf{长锤}\textgreater{}上)

襄老
【西皮快板】先蔑武艺果然好,一战未交我就逃。年纪衰迈精神老,奔回大营奏根苗。

襄老 老臣交令。

庄王 可曾会过阵来?晋国将官哪个?

襄老 晋国元帅名叫先蔑。

庄王 呵,先蔑。(胜负如何?)

襄老 老臣出马就被他一枪,哎呀\ldots{}\ldots{}

庄王 (呃,)敢是带了伤了?

襄老 枪回来了。

庄王 敢是败了?(或:哦,败了。)

襄老 败了。

庄王 后营憩息。(或:老将军后营歇息。)

襄老 谢大王。(谢襄老)

庄王 (且住,)先蔑老儿十分骁勇,必须孤王亲自会他,众将官,奋勇当先。

(众领起,先蔑众上,二龙出水会阵)

先蔑 呔,来者敢是楚王?

庄王 正是。来者可是先蔑?

先蔑 然。

庄王 先蔑,楚邦(或:孤王)有何亏负你国,无故兴兵是何理也?

先蔑 昏庄!你横行天下,某奉晋君旨意,领兵扫荡。还不束手受绑?

庄王 (贼子)住口!众将官排开阵势者。

庄王 【西皮导板】叫三军与孤战鼓操,

(龙套钻烟筒,一合两合拉开唱)

庄王
【西皮快板】先蔑老儿听根苗:列国早已(或:各国俱已)结盟好,同心协力保周朝。你主不该把孤藐,平地生波为哪条。陆戎小国被孤扫,陈、郑不敢犯边辽。(或:陆戎小国被孤扫,陈、郑不敢犯边辽。你主若是行无道,定把晋国永勾销。或:陈、郑二邦写降表,陆戎不敢犯边辽。你主不该行无道,无故兴兵为哪条。或:你主不该行无道,无故兴兵为哪条。陆戎小国何足道,陈、郑不敢犯边辽。)劝你马前写降表(或:归顺好),免得尸首马后抛。

先蔑
【西皮摇板】大晋明君存仁道,【转西皮快板】各守疆土见识高。你图中原行霸道,称孤道寡犯天条。屡次兴兵各国扫,横行天下夺城壕。两军对垒战场道,各显奇能逞英豪。

庄王 【西皮摇板】好言说尔说不倒(或:好话说尔说不倒)。

先蔑 【西皮摇板】管教昏王丧荒郊。

庄王 【西皮摇板】三军摆开(或:三军排开)长蛇道。

(先蔑扫一句,开打,钻烟筒,打枪剑,庄王败下,上楚将一二败下,追过场,先耍下场下)

{[}第十二场{]}

庄王
(上唱)【西皮散板】一霎时玉石焚金山颓倒,闯东西、奔南北生路哪条(或:闯东西、
奔南北生路何条)。

庄王
(念)且住!先蔑老儿十分骁勇,连败孤王数员大将,呜哙呀,事到如今孤王身边连一个保驾的臣子都没有了,看将起来真是成了孤家了。(或:连挑孤家数员上将,哎呀,孤王如今身边连一个保驾的臣子都没有了,哎呀,看将起来,真是孤家了。)

先蔑 (内白)哪里走!

庄王 哎呀来了。

(先蔑追上打庄王下,楚将三四上,败下,先耍下场追下)

{[}第十三场{]}

(唐狡甩发、黑箭衣、背单刀、手拿梢子帽,上唱)

唐狡 【西皮摇板】只望立功把恩报,主将不用枉心劳。

唐狡
(念)俺,唐狡。我主兵伐晋邦,只望先锋面前讨一前站,不想反被罚为小卒,收拾锣锅帐房与老卒同行。好不丧气人也。(鼓架子)且住!耳听喊杀之声,待俺登高一望。

(上桌子望。庄王领楚众上,晋众压队追上,庄众下,先蔑众追下,唐狡跳下桌)

唐狡
且住!前面败的我主,后面追的先蔑。此时不救,待等何时,呔,先蔑休要逞强,唐老爷来也。(扔帽,拔刀,耍下)

{[}第十四场{]}

(庄王上,先蔑追上,打庄抢背,唐狡上挑开,襄下场门上,搀庄上桌子,狡打先下,狡单刀耍下场,庄桌上云手踢腿,左右一、二外望,比势摸颈,唱)

庄王
【西皮散板】适才被贼挑下马,忽然间闪出了年少(的)娃。满营将官俱个在孤的功劳簿上跨,

襄老 老臣我在其内。

庄王 【西皮散板】这一员小将孤就不认识他。

襄老 您猜我呐(唱)【西皮散板】我也不认识他。

庄王 【西皮散板】看起来是孤王(拍腰)洪福大,天赐良将把贼拿。

(先蔑上)

先蔑
【西皮摇板】昏王被某挑下马,猛然来了年少娃。手使钢刀迎面扎,某家不曾提防他。落地梅花耍一耍,

(先蔑提枪花大边台口落地梅花势,唐狡换枪上勒马背枪单腿站)

先蔑 娃娃为何不敢前进?

唐狡 你用落地梅花暗施诡计非英雄也。

先蔑
【西皮摇板】倒教娃娃耻笑咱,楚国兵将全不怕,偏遇无名小冤家。扳鞍踏镫把马跨,

唐狡 【西皮摇板】老爷擒你献皇家。抖擞精神催战马,

先蔑 【西皮摇板】这枪刺得某两眼花。多少将官丧马下,何惧小小井底蛙。

(唐狡打先蔑下,狡耍枪下场,下)

庄王
【西皮散板】气宇轩昂武艺佳(或:小将生来实可夸),能征惯战果不差。但愿先蔑早拿下,千刀万剐不饶他(\textless{}\textbf{三锣}\textgreater{})。

(晋楚四将开打,先、狡两边上,漫对方将头,唐狡擒先蔑下,庄王、襄老下桌椅,趴地,襄扶庄起,不要有逗笑动作,望)

庄王 那先蔑呢?

襄老 被小将军擒住了。

庄王 你可曾看得清楚?(或:呃,老将军你可曾看见?)

襄老 没错,我戴着花镜呐!

庄王 这就好了,与孤带马。(或:哦,拿住了。呃,带马带马。)

襄老 被小将军骑了去了。

庄王 骑你的马。(或:呃,带你的马。)

襄老 还没有安尾巴呢!

庄王 孤王怎样回营呢?

襄老 只好开步走了。

庄王 如此摆驾。

襄老 咦。

(襄老领庄王下)

{[}第十五场{]}

(牌子,庄王众上站门,庄正面大座,襄老上报)

襄老 先蔑擒到。

庄王 押上帐来。(或:带先蔑。)

(襄老拉先蔑手杻上,先在襄后踹襄,襄趴下,再起来)

先蔑 【西皮摇板】龙入铁网难撑架,

先蔑
【西皮快板】虎落平阳被擒拿。列国英雄也有咱,遇这无名小冤家。某既被擒凭刀剐,落得忠名扬天涯。将身站立大帐下,(进帐)

先蔑 【西皮摇板】看他把某怎开发。

庄王 【西皮摇板】孤王帐中用目洒,

庄王
【西皮快板】先蔑老儿带锁枷(或:披锁枷)。阵前何等威风大,运败时衰被孤拿。

庄王 (白)先蔑。(孤王有何亏负你国,何故兴兵犯界,是何理也?)

先蔑 昏王。

(踢桌,庄王站躲,再坐下)

庄王
还是如此厉害,先蔑你在两军阵前何等威风,如今被擒帐下,有何话讲?(或:呜哙呀,你在阵前何等威风,何等煞气,今日被擒,有何话讲?)

先蔑 昏王何必多言。

庄王
孤王何曾亏负你国,无故兴兵犯界,先斩你这老头,再擒晋侯与他辩理,来,将先蔑推出斩了。(或:呜哙呀,还是这等的厉害,哼,先斩你这个老头,再擒晋侯与他辩理。来,将先蔑推出斩了。)

(襄老拉先蔑下,\textless{}\textbf{五锣三鼓}\textgreater{},襄上报)

襄老 先蔑斩首,小将回营。

庄王 有请。

(庄王出位。唐狡\textless{}\textbf{紧锤}\textgreater{}上,下马。庄拉狡换边,庄小边、狡大边台口,襄老托狡腿或可扳狡朝天镫)

庄王
【西皮快板】一见小将到帐下,功劳(或:战伐)魁首第一家。孤将龙衣来脱下,

(吹打合龙,唐狡穿庄王黄马褂、戴武生巾,庄穿开氅,庄收腿)

庄王
【西皮快板】得胜御酒(或:得胜琼浆;功劳簿上)把功加。(或:得胜御酒付卿拿。)

(递酒,唐狡接酒谢天地,庄王正面小座,狡参拜)

唐狡 参见大王,救驾来迟大王恕罪。

庄王 平身。(或:罢了。)

唐狡 谢大王。

庄王 赐座。(或:一旁坐下。)

唐狡 谢座。

(唐狡坐大边,襄老站小边)

庄王 小将军哪里人氏,姓甚名谁,(孤王有何恩惠于你,)竟敢一人前来救驾。

唐狡 小臣唐狡,棠邑人氏。大王待小臣有天高地厚之恩,特来救驾。

庄王 啊,孤王有何恩惠于你(或:哦,孤有何恩典于你)?

唐狡 大王可记得绝缨会之故否?

庄王
哦,不必深言(不要背供)。你今(日)救驾有功,封为上军副帅。(同孤扫晋。)

唐狡 谢大王。

(襄老``哎呀''蹲下)

庄王 老将军为何如此? (哎呀,老将军你这是怎么样了?)

襄老
大王有所不知,唐将军乃老臣帐下兵卒,老臣曾将他重责,不料他勤王救驾封官,上军副帅,正管我这个前站先行,老臣我这回可真玩不开了。

庄王
(哦,)原来如此,这样吧,从今以后将老将军拨在唐将军帐中(或:将军帐下),倘有差迟,(呃,)按军令施行如何?

襄老 哎哟。

唐狡
啊老将军,为将者当以军法为重。唐狡自应以德报德,以直报怨。焉有记恨之理,老将军何必挂怀?

襄老 将军乃奇男子也。

庄王
二卿为孤不惜\protect\hyperlink{fn52}{\textsuperscript{52}}身躯,岂能怨恨,后帐摆宴与二卿解和贺功。

(庄王下,唐狡、襄老互让下,众下)

\newpage
\hypertarget{ux641cux5b64ux6551ux5b64-ux4e4b-ux7a0bux5a74ux516cux5b59ux6775ux81fc}{%
\subsection{搜孤救孤 之
程婴、公孙杵臼}\label{ux641cux5b64ux6551ux5b64-ux4e4b-ux7a0bux5a74ux516cux5b59ux6775ux81fc}}

{[}第一场{]}

公孙杵臼 {[}引子{]}赵、屠结冤仇,恨奸贼,何日罢休。

公孙杵臼
(念)屠贼专权乱朝纲,欺君藐法似虎狼。可叹忠良满门丧,铁石人儿也悲伤。

公孙杵臼
老汉公孙杵臼,昔年曾在赵家以为门客。可恨屠贼戮杀\protect\hyperlink{fn53}{\textsuperscript{53}}赵家三百余口,只剩庄姬一人逃进宫去,生下孤儿。屠贼闻知,带剑进宫,搜孤不出。如今出了赏格在外,十日之内,有人献出孤儿便罢,倘若无人献出,要将晋国中与孤儿同庚者,俱要斩尽杀绝。天呐,天!眼见孤儿无救了。

公孙杵臼
【二黄原板】恼恨屠贼心太狠,戮杀赵家一满门。眼见得忠良无有救应,大事还要问程婴。

程婴 【二黄散板】屠贼做事心太狠,三百余口赴幽冥。

程婴 公孙兄在家么?

公孙杵臼 是哪一位?

程婴 小弟来了。

公孙杵臼 哦,贤弟来了。请到里面。

程婴 请。

公孙杵臼 请坐。

程婴 有座。

程婴 唉!

公孙杵臼 贤弟为何长叹?

程婴 这晋国之中又出了一桩奇事,你还不晓么?

公孙杵臼 哦,什么奇事,愚兄不知呀。

程婴 可恨屠贼带剑进宫,搜孤不出,如今又起了狠毒之心呐。

公孙杵臼 什么狠毒之心?

程婴
那贼出了赏格在外,十日之内,有人献出孤儿,赏赐千金。不然,要将晋国中与孤儿同庚者,俱要斩尽杀绝。

公孙杵臼 呃,此事愚兄早已知晓。贤弟,你的来意如何?

程婴 弟今此来,与兄商议这救孤之策。

公孙杵臼 愚兄我是忙中无计。

程婴 你也无计------唉!弟倒有一两全之计。或能救得孤儿。

公孙杵臼 何谓两全之计?

程婴 若有一人舍得一命,一人舍得一子,或能救得孤儿。

公孙杵臼 贤弟,你来看,愚兄偌大年纪,情愿舍命。但不知何人舍子?

程婴
呃,弟新生一子,与孤儿诞期相近。就将我儿藏至你处,待弟前去出首。就说你隐藏孤儿不报,那屠贼闻知,必定带领人役前来搜寻。哎呀,那时只怕你的性命难保哇。

公孙杵臼 呃,愚兄方才言过,偌大年纪,死何足惜。呃,你只管地抱来就是。

程婴 话虽如此,你那弟妇她还不曾知道呢。

公孙杵臼 哎呀,弟妹不允,可也是枉然呐。

程婴 不妨不妨,你那弟妇虽是女流,颇知大义,不能不允呐。

公孙杵臼 话虽如此,你且先行,愚兄随后就到。

程婴 告辞了。

程婴 【二黄散板】你我二人把计定,立孤(或:抚孤)的事儿我担承。

公孙杵臼 【二黄散板】但愿救得孤儿命,不绝赵家后代根。

{[}第二场{]}

(程妻 (念)仗义救孤身,妻随夫志行。)

程婴 (念)大事安排定,劝妻舍亲生。

程婴 娘子。

程婴 唉!事到如今还讲什么天理报应(或:上苍报应)。

程婴 屠贼进宫,搜孤不出,又起了狠毒之心呐。

程婴
那贼如今出了赏格在外,十日之内,有人献出孤儿,赏赐千金。不然,要将晋国中与孤儿同庚者,俱要斩尽杀绝。

程婴 我与公孙老爷定下两全之计,可以救得孤儿。

程婴 若有一人舍得一命,一人舍得一子,就可以救得孤儿。

程婴 就是那公孙老爷他情愿舍命呐。

程婴
舍子么\ldots{}\ldots{}唉,娘子,想你我夫妻曾受赵相厚恩,焉能坐观成败。我意欲将你我的儿子与孤儿调换下来,抚养成人。一来接得赵家宗嗣\protect\hyperlink{fn54}{\textsuperscript{54}},二来日后也好报仇雪恨。

程婴 啊,娘子,你看这一条计策(或:你看此计)可好啊?

程婴 唉,娘子啊!

程婴
【二黄原板】娘子不必太烈性,卑人言来你试听:赵、屠二家有仇恨,三百余口命赴幽冥。我与那公孙杵臼把计定,他舍命来你我舍亲生。舍子搭救忠良后,老天爷岂绝我的后代根。你今舍了亲生子,来年必定降麒麟。

程婴 【二黄原板】千言万语她不肯,不舍娇儿难救孤身。无奈何我只得双膝跪,

程婴 【二黄摇板】哀求娘子舍亲生。

(程妻 【二黄摇板】你要跪来只管跪,要我舍子万不能。)

程婴 【二黄散板】人道妇人心肠狠呐,狠毒毒不过妇人的心。

(程妻 【二黄摇板】\ldots{}\ldots{}不食子,你比虎狼狠十分。)

程婴 【二黄散板】不如程婴死了罢,

(程妻 【二黄摇板】或生或死一同行。)

程婴 【二黄散板】手执钢刀要你命,

(程妻 【二黄摇板】用手关上绣房门。)

公孙杵臼 【二黄摇板】程婴与我把计定,未知他心似我心。

公孙杵臼 贤弟,愚兄来了。

程婴 请坐。

程婴 呃,呃,呃,这\ldots{}\ldots{}这,这边坐。

公孙杵臼 呃,俱是一样啊。

公孙杵臼 啊,贤弟,弟妹可曾应允呐?

程婴 那贱人她执意地不允呐。

公孙杵臼 呃,先前言过,弟妹十分贤德,颇知大义,呃,怎么如今她不允起来了?

程婴 呃呃呃,是,是她不允呐!

公孙杵臼 呃,不必如此,请将出来,愚兄良言相劝。

程婴 遵命。

程婴 贱人走出来!

程婴 公孙老爷有话讲啊。

公孙杵臼 弟妹少礼,请坐。

公孙杵臼 劝弟妹听了丈夫之言,舍了亲生之子。

公孙杵臼 (念)弟妹搭救孤儿命,留得美名万古存。

公孙杵臼
【二黄原板】人有善念天有应,莫把阴骘当浮云。弟妹搭救忠良后,赵家代代不忘恩。

公孙杵臼 【二黄摇板】老朽薄面情要准,

程婴 【二黄散板】看起来你是个不贤妇哇,

程婴 【二黄散板】手持钢刀项上刎,

公孙杵臼 【二黄散板】贤弟息怒且消停。

公孙杵臼 【二黄散板】走向前来良言劝,死了丈夫靠何人。

公孙杵臼 【二黄散板】无奈何我只得屈膝跪,

程婴 不要跪。

公孙杵臼 你也跪下。

公孙杵臼 【二黄散板】哀求弟妹救孤生。

公孙杵臼 【二黄散板】弟妹舍了亲生子,列国之中标美名。

程婴 【二黄散板】多谢娘子开了恩,母子快快两离分。

{[}第三场{]}

(\textless{}\textbf{水底鱼}\textgreater{},程婴上)

程婴 唉!

程婴 击鼓人告进。

程婴 叩见大人。

程婴 大人前番搜孤可曾搜出?

程婴 孤儿在------

程婴 现在首阳山公孙杵臼的家中。

程婴
小人与他昔年皆为赵相门客,又有八拜之交。只因他隐藏孤儿不报,是小人劝他献出,不想他是执意地不肯,反将小人辱骂。小人本不愿出首,因见大人有言在先,知情不举是罪加一等,为此小人不敢隐瞒,特地前来禀明大人。

程婴 小人名叫程婴。

程婴 有。

程婴 谢大人。

公孙杵臼 小人隐藏孤儿,何人得见?

公孙杵臼 哎呀大人呐,此人名叫程婴,与小人旧有仇恨,乃是诬告小人!

公孙杵臼 诬告小人!

公孙杵臼
【二黄散板】白虎大堂一声禀,大人息怒听详情。程婴与我有仇恨,把什么孤儿予大人。

公孙杵臼 【二黄散板】纵然打死我难招承。

(屠岸贾 程婴。)

(屠岸贾
【二黄散板】\ldots{}\ldots{}赐你鞭一根。一边打来一边问,看他招承不招承。)

程婴 【二黄导板】白虎大堂奉了命,

(屠岸贾 程婴!)

程婴 有。

(屠岸贾 与我着实地打!)

程婴 【回龙】都只为救孤儿舍亲生,连累了年迈苍苍受苦刑,眼见得两离分。

程婴
【二黄原板】我与他人定巧计,到如今连累他受苦刑。开言便把公孙兄问,小弟言来你试听:你若是再三地不肯招认\protect\hyperlink{fn55}{\textsuperscript{55}},大人的王法不徇情。手执皮鞭将你打,

程婴
【二黄散板】你\ldots{}\ldots{}你,你切莫要胡言攀扯(或:连累)我好人。

公孙杵臼 贼。

公孙杵臼
【二黄散板】指着程婴骂高声,苦苦害我为何情。我今一死何足论,你留得骂名列国闻。

程婴 【二黄散板】老儿执意不招认,急往首阳去搜寻(或:大人首阳去搜寻)。

公孙杵臼 屠贼。

公孙杵臼 【二黄散板】奸贼做事心太狠,苦害忠良为何情。我今与你拼性命,

程婴 【二黄散板】(这是你)飞蛾投火自烧身。

程婴 小人讨祭。

(屠岸贾 你为何祭他?)

程婴 小人与他有八拜之交,若不祭奠于他,旁人道小人不义了。

程婴 谢大人。

程婴 【二黄散板】虽然杯酒寻常饮,略表当年结拜情。

{[}第四场{]}

公孙杵臼 (内)【二黄导板】一片好心反成恨,

公孙杵臼 【回龙】年迈苍苍血染身。

公孙杵臼
【二黄原板】我与他人把计定,一人舍命一人舍亲生。含悲忍泪法场进,

公孙杵臼 【二黄散板】咬定牙关等时辰。

程婴
\textless{}\textbf{撞金钟}\textgreater{}【二黄摇板】迈步儿来在法场中,只见孤儿与公孙。

程婴
\textless{}\textbf{叫头}\textgreater{}公孙兄,赵公子,你二人死在九泉(之下),休怨我程婴。(哭介)

程婴
【二黄碰板原板】躬身下拜礼恭敬,眼望孤儿泪淋淋。法场上看的人(或:法场上人人)都来叫骂,一个个骂的是我程婴,是一个无义的人。贪享荣华受富贵,断送了忠良人的后代根。这是我好意反成恶意,满怀心腹事向谁云。

公孙杵臼
【二黄原板】法场上绑得我昏迷不醒,抬头只见小程婴。去掉好言换恶语,高声叫骂小程婴。我今一死不要紧,留得美名万古闻。

程婴
【二黄原板】公孙兄说话需谨慎,泄漏了机关大事难成。先前抚孤是你我,到如今知心(或:同心)还有谁人。你为忠良舍了性命,可叹我程婴绝了后根。无奈何烧钱把酒奠,我那亲------\textless{}\textbf{哭头}\textgreater{}我\ldots{}\ldots{}我,我的儿啊!

程婴 公孙兄啊。

程婴 【二黄散板】但愿你灵魂早超生。

程婴 祭奠已毕。

程婴 啊,大\ldots{}\ldots{}大人。

程婴
小人先前言过,与那公孙杵臼有八拜之交,如今见他身首异处,思想前情,故而落泪呀。

程婴 且慢,小人不愿领赏。

程婴
小人新生一子,与孤儿诞期相近。今将孤儿出首,又恐旁人加害我父子,还望大人另外相照。

程婴 谢大人。

程婴
【二黄散板】两全之计全孤命,再把立孤的巧计生。(或:背转身来笑吟吟,奸贼中了我的巧计生。)

(程婴大边下,小边上)

程婴
【二黄散板】心中大事安排定,孤儿长大杀仇人。(或:怀抱孤儿法场进,但愿你长大杀仇人。)

程婴 有。

程婴 多谢大人!

\newpage
\hypertarget{ux6218ux6a0aux57ce}{%
\subsection{战樊城}\label{ux6218ux6a0aux57ce}}

{[}第一场{]}

伍尚 (念)边外狼烟净,

伍员 (念)共享太平春。

伍尚 贤弟请坐。

伍员 有座(或:请坐)。

伍尚 唉!

伍员 兄长自到樊城,为何终日忧闷?

伍尚 你我弟兄镇守樊城,不知双亲在京安否,令人悬念。

伍员 吉人自有天相,兄长何必多虑?

伍尚 但愿如此。

鄢将师 (念)离了京城地,来此是棠邑。

鄢将师 门上哪位在?

(家院 什么人?)

鄢将师 京城下书人求见。

伍尚、伍员 两厢伺候(或:外厢伺候)。

伍尚、伍员 吩咐书先进,人落后。

鄢将师 是。

(家院 书信呈上。)

伍尚 呈上来。

伍尚 贤弟,爹娘有书信到来,贤弟请看。

伍员 兄长请看。

伍尚 一同观看。

伍尚、伍员 爹娘在上,恕儿等不孝罪也。

伍尚
【西皮原板】未曾拆书泪先淋,纸上相逢父子情。平王思念临潼会,伍尚、伍
员快回京。``外加走之''书后遁,骏马``十疋''莫留停。看罢书信喜不胜,

伍员 哦!

伍员 【西皮散板】伍员呐心中自沉吟。

伍尚 贤弟你再仔细观看。

伍员 不必观看,书信上言语,兄长可解?

伍尚 愚兄不解。

伍员 既是调我弟兄进京,加官授爵,书信之上为何有``逃走''二字?``外''加``走
之''是``迯'',骏马``十疋''是``走''。分明是``迯走''二字啊,令人难解。(或:
``外''加``走之''是``迯'',骏马``十疋''是``走''。分明是``迯走''二字。
既是调我弟兄进京,加官授爵,为何有``逃走''二字?)

伍尚 传下书人一问。

伍员 传下书人(或:唤下书人)。

(家院 下书人。)

鄢将师 在。

鄢将师 参见二位少老爷。

伍尚、伍员 罢了。

伍尚、伍员 圣上驾安?

鄢将师 我主驾安。

伍尚、伍员 太老爷?

鄢将师 安泰。

伍尚、伍员 太夫人安泰?(或:太夫人?)

鄢将师 福寿康宁。

伍尚、伍员 你叫什么名字?

鄢将师 小人名叫鄢将师。

伍尚、伍员 你是新进相府还是久在相府?

鄢将师 乃是新进相府。

伍尚、伍员 相府书信怎样(或:相府书信何人)交付与你?

鄢将师 里封外传。

伍尚、伍员 什么时候?

鄢将师 黄昏时候。

伍尚、伍员 圣上调我弟兄进京何事?

鄢将师 这\ldots{}\ldots{}

伍尚、伍员 讲!

鄢将师 不过是------呃,加官进爵而已。

伍员 呵!呵!呵呵呵\ldots{}\ldots{}(冷笑介)

伍尚、伍员 下去!

鄢将师 是,是,是\ldots{}\ldots{}

鄢将师 好一个仔细的二老爷!险呐!

伍尚 贤弟你看如何?

伍员 下书人言语吱唔,凶多吉少,去之无益。

伍尚 既是爹娘亲笔书信,焉有不去之理?

伍员 有道是:将在外,君命有所不受哇。

伍尚 唉!贤弟呀------

伍尚
【西皮原板】昔日里有个商纣君,囚禁文王整七春。伯邑考许父丧了命,留得美名
万古存。

伍员 兄长!

伍员
【西皮原板】兄长说话欠思论,休把今人比古人。那文王被囚天注定,伯邑考粉身命里生成。既是平王【转西皮二六】加官赠,就该有圣旨到樊城。若是爹娘修书信,为什么有``迯走''二字在书后存。怕的是失足罹陷阱\protect\hyperlink{fn56}{\textsuperscript{56}},那时节插翅亦难腾。我一心坐定樊城镇,愿作个不忠不孝人。

伍尚
【西皮快板】听他不肯进京城,背转身来自思忖:长子就该遵父命,是好是歹走一程。

伍尚 贤弟,你是不进京的了?

伍员 凶多吉少,去之无益。

伍尚 待愚兄一人前去。

伍员
兄长一人前去(或:兄长要去),弟放心不下。可命家将跟随,一路之上也好侍奉鞍马。

伍尚 言得极是,贤弟安排。

伍员 唤家将。

伍尚、伍员 罢了!

伍尚 二老爷有差。

伍员 命你跟随大老爷进京,一路之上侍奉鞍马。倘有事故,速报我知。

伍员 外厢备马。

伍尚 看衣改换。

伍尚
【西皮原板】头上乌纱来摘定,紫绶官衣脱离身。教家将备马府门等,你老爷即刻就要登程。

伍员 看酒!

伍员
【西皮原板】一封书信到樊城,拆散我弟兄两离分。叫家院看过酒一樽,弟与兄长【转西皮二六】来饯行:登山涉水多安稳,披星戴月奔都呃城。若是阖家同欢庆,在爹娘台前问安宁。倘若是家门遭不幸,报仇之事有弟伍员。非是小弟不从命,为的是``迯走''二字呃解不明。兄长饮干杯中酒,一路平安早到京。

伍尚
【西皮快板】用手接过酒一樽,背转身来谢神灵。回头再对贤弟论,愚兄言来听分明:倘若此去遭不幸,你是伍家报仇人。

伍尚 【西皮散板】含悲忍泪跨金镫,不分昼夜奔都城。

(伍员 呃!)

伍员
【西皮快板】兄长上马珠泪淋,教人难舍又难分。流泪眼观流泪眼,断肠人送断肠人。倘若家门遭不幸,杀上天子午朝门。吉凶二字未分定,稳坐樊城等信音。

{[}第二场{]}

伍尚
【西皮快板】伍尚跨马奔帝京,眼观绿水青山行。无心观看路旁景,但愿早见二双亲。

{[}第三场{]}

费无极 (念)量小非君子,无毒不丈夫。

费无极
老夫------费无极。也曾命鄢将师诓那伍氏兄弟,未见到来。左右------伺候了。

鄢将师 (念)诓来棠公,回复将令。

鄢将师 参见相爷。

费无极 罢了。伍氏弟兄可曾诓到?

鄢将师 伍员不肯进京,伍尚现已诓到。

鄢将师 监中探父。

费无极 下面歇息。

鄢将师 谢相爷。

费无极 且住。\ldots{}\ldots{}伍尚诓到,不免上朝启奏,来------打轿上朝。

费无极 臣,费无极见驾,吾主千岁。

费无极 伍奢拿到。

楚平王 将他父子绑至金阶斩首,不得有误,领旨下殿。

费无极 领旨。

费无极 校尉等------打道法场!

{[}第四场{]}

费无极 来,将伍奢、伍尚父子绑了上来!

费无极 拿去开刀!

伍尚 唉!爹爹呀!

伍奢 蠢材!

伍奢
【西皮快板】一见我儿绑金阶,骂声伍尚太无才:枉读诗书为相宰,为父言语解不开。

伍尚
【西皮快板】老爹爹休得把儿怪,书信一到怎不来?父子们犯了何条戒,因何捆绑在金阶?

伍奢
【西皮快板】平王无道纲常坏,父纳子媳礼不该。金顶轿换了银顶轿,为父谏奏惹祸灾。

伍尚 【西皮散板】听一言来恼心怀,骂声奸佞如狼豺。恨不得将尔来踏坏------

伍尚 【西皮散板】阴曹地府等尔来。

伍奢 【西皮散板】父子们辜负了幽冥界!

费无极 起过。

费无极
且住!伍奢、伍尚虽已斩首,今有伍员坐镇樊城,若不铲除,终是后患。不免二
次上朝启奏。

费无极 来!\ldots{}\ldots{}上朝。

费无极 臣费无极二次上殿,吾主万岁。

楚平王 二次上殿,有何启奏?

费无极
伍奢、伍尚已然斩首,今有伍家满门老小尚在,伍员抗旨不遵,坐镇樊城,请
我主裁处。

楚平王
就请卿家带领人役,抄杀伍府满门大小。命武城黑带领三千人马,去至樊城,
捉拿伍员进京问罪,领旨下殿。

费无极 校尉等------抄杀伍府去者!

{[}第五场{]}

太夫人
【二黄摇板】伍家世代忠良臣,平王无道掌乾坤。父纳子妻行不正,尽忠谏责显忠诚。

丫鬟 参见太夫人,大事不好了!

太夫人 何事惊慌?

丫鬟 太老爷与大老爷不知身犯何罪,斩首金阶。

太夫人 哎呀!

丫鬟 太夫人醒来!

太夫人
【二黄摇板】听一言来冷汗淋,吓得七魂掉三魂。哭一声老相夫丧了命,\textbf{\textless{}哭头\textgreater{}}老爷呀,

太夫人 【二黄摇板】昏王无道斩忠臣。回头我对家将论:你快到樊城报信音。

家将 遵命!

(费无极挎宝剑上)

家院 太夫人,大事不好了!

太夫人 何事惊慌?

家院 那费无极带领校尉抄杀满门来了!

太夫人 哎呀!

太夫人
【二黄摇板】未觉乌鸦叫声震,今日大祸降临门。耳旁听得人声震,定是奸贼到来临。

太夫人 好贼!

太夫人
【二黄摇板】手指奸贼骂高声:祸国欺君害忠臣,拚着一死抛性命,或生或死一路行。

校尉 抄杀已毕。

费无极 上殿交旨!

(【急急风】下)

{[}第六场{]}

伍员 (念)独坐(或:闷坐)樊城心忧闷, 吉凶二字未分明。

伍员 什么大事?

伍员 你才怎讲(或:你待怎讲)?!

伍员 \textless{}\textbf{叫头}\textgreater{}爹娘!兄长!

伍员 哎呀!

伍员 【西皮导板】听说爹娘丧了命呐,

伍员 \textless{}\textbf{三叫头}\textgreater{}爹爹!母亲!唉!兄长呃!

伍员 【西皮散板】珠泪点点洒前胸。忍泪含悲家将问,

伍员 家将!

伍员 【西皮散板】犯罪的情由(或:被害的情由)说分明。

(伍员 好贼!)

伍员
【西皮散板】骂声无极贼奸佞,无道昏君灭人伦。我若人马来点动,不忠的名儿万古闻(或:天下闻)。

伍员 再探!

伍员 \textless{}\textbf{叫头}\textgreater{}家将!

伍员 今有武城黑带兵围困樊城,如何是好?

伍员 反得的?

伍员 如此------反、反、反呐!

伍员 【西皮散板】兵来将挡(或:兵临将挡)自古论,水来自有土来屯。

伍员 【西皮散板】家将备马传将令。

伍员
【西皮散板】玲珑铠甲放光明(或:闪光明),三军与爷开城禁(或:人来与爷城开定)。

{[}第七场{]}

伍员 马前来的敢是武城黑?

伍员 武城黑!

伍员 带兵意欲何往?

伍员 一派胡言!放马过来!

(伍子胥大边(\textbf{念完}``\textbf{放马过来}''\textbf{一合},\textbf{两合},\textbf{架住}),钻完烟筒,一扯两扯,半个剜萝卜归小边,幺二三,把兜一别,撤枪刺耳被勾走马腰封到大边,往里一盖转身,打前蓬头、后蓬头到小边,转身接前蓬头、后蓬头又归大边,转身,掣肘,拉转身,幺二三,把兜左转身走里边,再一个把兜左转身过到小边,外边打一个腰封,两个腰封,被盖右转身到里边、接一个腰封、两个腰封,回身掣肘,拉转身,幺二三,里外各一二三绕,外里外一二三压,(\textbf{往})里外各一盖两盖,打腰封,武左转身,伍往里漫武头、扎脖,武右手推出去,亮住。\protect\hyperlink{fn57}{\textsuperscript{57}})

(武下伍接耍下场:出枪提枪花转身,三个提枪花,出枪右手转枪鐏,反手扶枪把,左手在上正手扶枪杆,双手抱枪,右腿抬,左腿单腿站,枪鐏放在右腿膝盖后马面上,枪头向上矗住,右手打鐏将枪打出去,右手抄枪杆下端接住再扔枪右转身面向下场门左手接枪杆下端,在左腰间平端,右手举拳过顶\protect\hyperlink{fn58}{\textsuperscript{58}},左腿抬,右腿单腿站亮住,下场门下。)

伍员
且住,武城黑来得厉害,不免伤他一箭!\protect\hyperlink{fn59}{\textsuperscript{59}}

伍员
【西皮散板】张弓布矢威风凛(或:本帅开弓放雕翎),武城黑鼠窜逃了生(或:带箭逃了生)。本帅逃出呃棠邑郡\protect\hyperlink{fn60}{\textsuperscript{60}}(或:天罗境)呐,可叹我的家将丧残生。

伍员
\textless{}\textbf{三叫头}\textgreater{}爹娘!兄长!哎!家将啊,呃\ldots{}\ldots{}(哭介)

\newpage
\hypertarget{ux957fux4eadux4f1a-ux4e4b-ux4f0dux5458}{%
\subsection{长亭会 之
伍员}\label{ux957fux4eadux4f1a-ux4e4b-ux4f0dux5458}}

且住!

看那旁来了一哨人马(或:看前面一哨人马,),旌旗招展,斗大``申''字门旗,(或:门旗招展,上写斗大``申''字;门旗之上斗大``申''字;旌旗之上斗大``申''字,)想是申包胥(或:申贤弟)催贡还朝。俺不免勒住马头(或:我不免在此等候),将我的血海冤仇对他细说一遍(或:细表一遍)!

贤弟若问,你且听道------

【西皮二六】\textbf{未曾开言我的泪先流,尊一声贤弟听从头:兄保平王功劳有,可叹我的忠心付水流。临潼会,兄为首,力举千斤压定了诸呃侯。双挂盟府}\protect\hyperlink{fn61}{\textsuperscript{61}}\textbf{印二口,各国不敢统貔貅。恨平王无道贪色酒,父纳子媳礼不周。金顶轿换为银顶扣,}无祥女\textbf{改换马氏女流。我的父谏奏反遭斩首,可怜我的老娘也被刀割头}\protect\hyperlink{fn62}{\textsuperscript{62}}\textbf{。多亏了家将越墙逃命走,来到了樊城细报根由。愚兄闻言怒冲牛斗,武城黑带兵围困城头。万般无奈我才下了毒手啊,也是我}认\textbf{扣搭弓}\protect\hyperlink{fn63}{\textsuperscript{63}}\textbf{箭射他咽喉。观只见门旗``申''字大如斗,就知晓贤弟催贡转回头。诉一诉我的含冤}:\textbf{杀我的全家三百余口,就是那鸡犬也不留。似这等的冤仇怎忍受,不杀平王气怎休​?!}

\textbf{【}西皮快板\textbf{】}贤弟把话错出口,愚兄言来听从头:君无道,臣逃走,父不正来子外游。吾与贤弟共乡土\protect\hyperlink{fn64}{\textsuperscript{64}},相交胜似亲骨肉。你今回朝休泄漏,不念今朝念从头。我今借兵来伐楚,不杀平王誓不休(或:气怎休)。

\textbf{【}西皮快板\textbf{】}申包胥与我把智斗(或:把誓斗)\protect\hyperlink{fn65}{\textsuperscript{65}},背转身来又加愁。如是领兵(或:若是领兵)来伐楚,不忠的名儿万古留。我不领兵(或:若不领兵)来伐楚,血海冤仇一旦休。

\textbf{【}西皮快板\textbf{】}贤弟请上兄叩首,

\textbf{【}西皮快板\textbf{】}多谢你放我吴国投(或:往东流)。你兴楚来我灭楚,你为公来我为仇。辞别贤弟跨马走,

马来!

【西皮摇板】扬鞭打马吴国投。

\newpage
\hypertarget{ux6587ux662dux5173-ux4e4b-ux4f0dux5458}{%
\subsection{文昭关 之
伍员}\label{ux6587ux662dux5173-ux4e4b-ux4f0dux5458}}

\textbf{{[}汪派{]}}

{[}第一场{]}

\textbf{(内)马来!}

\textbf{【}西皮散板】伍员马上怒气冲,逃出龙潭虎穴中。

\textbf{俺,伍员!幸喜逃出樊城,意欲往吴国借兵报仇。行至此处,四面俱是高山峻岭,不知哪条道路,可通吴国。}

\textbf{看那旁有一老丈,待我下马问来。}

\textbf{啊,老丈请了。}

\textbf{啊,俺乃行路之人,老丈休得错认。}

\textbf{愚下正是伍员。老丈何以知晓?}

\textbf{原来如此。}

\textbf{唉!俺有满腹含冤,意欲往吴国借兵报仇。行至此处,四面俱是高山峻岭,不知哪条道路可通吴国?}

\textbf{可有别路?}

\textbf{哎呀,不、不、不,不好了!}

\textbf{【}西皮散板】听说吴国路不通呃,好似狼牙箭穿胸。心猿意马终何\textbf{\textless{}哭头\textgreater{}}用,爹娘啊!

\textbf{【}西皮散板】血海冤仇落场空。

\textbf{萍水相逢,怎好打搅。}

\textbf{这就不敢。}

\textbf{有座。}

\textbf{请问老丈尊姓大名。}

\textbf{哦,原来是前辈老先生,失敬了。}

\textbf{唉!一言难尽呐!}

\textbf{【}西皮原板】恨平王无道乱楚宫,父纳子妻礼难容\protect\hyperlink{fn66}{\textsuperscript{66}}。我的父谏奏反把命送,满门家眷血染红。

\textbf{若得如此,感恩匪浅。}

\textbf{(念)愧煞男儿不丈夫。}

\textbf{惭愧!}

\textbf{{[}第二场{]}}

\textbf{哎!}

\textbf{【}西皮快板】过了一天又一天,心中好似滚油煎。腰中枉挂三尺剑,不能报却父母冤。

\textbf{俺伍员多蒙东皋公搭救,将我隐藏后花园中寻计出关。一连七日未见计出,思想起来好不焦虑人也!}

\textbf{唉,爹娘啊!(哭介)}

\textbf{【}二黄慢板】一轮明月照窗前,愁人心中似箭穿。实指望奔吴国借兵回转,又谁知昭关又有阻拦。幸遇东皋公行方便,他将我隐藏在后花园。一连七天我的眉不展,夜夜何曾又安眠。俺伍员好一似丧家犬,满腹含冤向谁言。我好比哀哀长空雁,我好比龙游在浅沙滩。我好比鱼儿吞了钩线,我好比波浪中失舵的舟船呐。思来想去我的肝肠断,今夜未过又盼明天。

\textbf{【}二黄原板】心中有事难合眼,翻来覆去睡不安。背地里只把东皋公怨,教人难解巧机关。你若是真心来救我,为何七日不周全。贪图着(或:贪图这)富贵将我害,你就该将我献与昭关。哭一声爹娘不能够见面,难得\textbf{\textless{}哭头\textgreater{}}见,爹娘啊!

\textbf{【}二黄原板】要相逢呃除非是梦里团圆。

\textbf{【}二黄快原板】鸡鸣犬吠五更天,越思越想越伤惨。想起在朝为官宦,是朝臣待漏五更寒。到如今夜宿荒村馆,我冷冷清清向谁言呐?我本当拔宝剑自寻短见,父母的冤仇化灰烟。对天发下宏誓愿:我不杀平王我的心怎甘?

【二黄散板】适才朦胧将合眼,

【二黄散板】耳旁又听有人言。用手开门拔宝剑,

(东皋公 【二黄散板】\ldots{}\ldots{}因何白了髯?)

我却不信。

待我看来。

哎呀!不、不、不好了。

【二黄散板】一见须白心好惨,点点珠泪洒胸前。冤仇未报容颜变,一事无成两鬓斑。但愿过得昭关险,满斗焚香谢上天。

\textbf{{[}第三场{]}}

(念)父母冤仇恨,常怀一片心。

老丈何事?

待我相见。

皇甫兄在哪里?皇甫兄在\ldots{}\ldots{}

请坐。

穷途末路,犹如丧家之犬。仁兄誉言(或:年兄誉言),惭愧!

皇甫兄到此,有何妙计救我出关?

事不宜迟,就此改扮(或:装扮)起来。

【西皮快二六】伍员在头上换儒巾,乔装改扮往东行。临潼会,曾举鼎,我在万马营中显异能。时来双挂盟府印,运退深山草不生(或:运退时衰夜宿在荒村)。多亏了东皋公行恻隐,请来了历阳山\protect\hyperlink{fn67}{\textsuperscript{67}}前皇甫官人。我三人同把巧计定,皇甫官人假扮俺伍员去闯关门。(或:思想起教人恨不恨,也是我的五行八字命生成。)

(【西皮快板】回头来再对东皋公论:你是我伍员活命的恩人。但愿过得昭关境,一重恩报九重恩。)\protect\hyperlink{fn68}{\textsuperscript{68}}

【西皮摇板】皇甫兄请上受一礼,

【西皮快板】多谢你施下这全恩。焚香顶礼不为敬,来生犬马当报恩。

【西皮摇板】东皋公请上礼恭敬,

【西皮快板】你是我伍员的活命恩人。但愿过得昭关境,一重恩报九重恩。伍员心中千般恨,

【西皮摇板】大胆且向虎山行。

\textbf{{[}谭派{]}}\protect\hyperlink{fn69}{\textsuperscript{69}}

\textbf{{[}第一场{]}}

\textbf{(内)马来!}

\textbf{【}西皮散板】勒马停蹄威风勇\protect\hyperlink{fn70}{\textsuperscript{70}},只见道旁一老翁。

\textbf{俺,伍员!幸喜逃出樊城,来至此地,不知哪条道路可通吴国。}

\textbf{看那旁有一老丈,待我下马问来。}

\textbf{老丈请了。}

\textbf{啊,老丈------俺乃行路之人,老丈休得错认。}

\textbf{俺正是伍员。老丈何以知晓?}

\textbf{原来如此。}

\textbf{唉!我有满腹含冤,意欲到吴国借兵报仇。行至此地,四面俱是高山峻岭,不知哪条道路可通吴国?}

\textbf{可有别路?}

\textbf{哎呀,不、不、不,不好了!}

\textbf{【}西皮散板】听说吴国路不通,好似狼牙箭穿胸。心猿意马终何\textbf{\textless{}哭头\textgreater{}}用,爹娘啊\textbf{!}

\textbf{【}西皮散板】血海冤仇落场空。

\textbf{萍水相逢,怎好打搅。}

\textbf{这就不敢。}

\textbf{有座。}

\textbf{请问老丈尊姓大名。}

\textbf{呜哙呀,原来是前辈老先生,失敬了。}

\textbf{唉!一言难尽呐!}

\textbf{【}西皮原板】恨平王无道乱楚宫,父纳子妻礼难容。我的父谏奏反把命送,满门家眷血染红。

\textbf{若得如此,感恩非浅。}

\textbf{(念)愧煞男儿不丈夫。}

\textbf{惭愧!}

\textbf{{[}第二场{]}}

\textbf{唉!}

\textbf{【}西皮快板】过了一天又一天,心中好似滚油煎。腰中枉挂三尺剑,不能报却父母冤。

\textbf{俺伍员多蒙东皋公搭救,定计救我出关。一连七日未见计出,思想起来好不焦虑人也!}

\textbf{唉,爹娘啊!(哭介)}

\textbf{【}二黄慢板】一轮明月照窗前,愁人心中似箭攒。想当年在朝为官宦,朝臣待漏五更寒。恨平王无道纲常乱,父纳了子的妻(或:子的媳)礼不端。我父谏奏反遭斩,一家满门被刀残。单人匹马(或:匹马单人)弃楚樊,行至在昭关又有阻拦。到如今独宿在荒村馆(或:踟蹰在荒村馆;宿至在荒村馆),冷冷清清向谁言。(我好比哀哀长空雁,我好比龙游在浅沙滩。我好比扑灯蛾身罹大难,我好比平阳虎离了深山。我好比鱼儿吞了钩线,我好比波浪中失舵的舟船。)思来想去我的肝肠断,今夜未过又盼明天。

\textbf{【}二黄原板】心中有事难合眼,翻来覆去睡不安。背地里只把东皋公怨,教人难解巧机关。你若是真心来救我,为何七日不周全。贪图着(或:贪图这)富贵将我害,你就该将我献与昭关。哭一声爹娘不能够见面,难得\textbf{\textless{}哭头\textgreater{}}见,爹娘啊!

\textbf{【}二黄原板】要相逢除非是梦里团圆。

\textbf{【}二黄原板】鸡不鸣犬不吠月淡星稀,孤雁飞惊动了杜鹃鸟啼。黑暗暗背地里祝告天地,二爹娘在天灵细听端倪:保佑儿早到吴国地,借大兵杀平王灭却了费无极(或:灭却那费无极)。那时节方消儿心中恶气,大鹏展翅任空飞。叹爹娘叹得儿咽喉哽泣,咽喉哽泣,

【二黄散板】今夜晚寂寞更长偏遇着不鸣金鸡。

【二黄导板】适才朦胧将合眼,

【二黄散板】耳旁又听有人言。用手开门拔宝剑,

我却不信。

待我看来。

哎呀!不、不、不好了。

【二黄散板】一见须白心好惨,点点珠泪洒胸前。冤仇未报容颜变,一事无成两鬓斑。

喜从何来?

若得如此感恩匪浅。

老丈请上受我一拜。

【二黄散板】但愿过得昭关险,借得吴兵报仇冤。

\textbf{{[}第三场{]}}

(念)父母冤仇恨,常怀一片心。

老丈何事?

待我相见。

皇甫兄在哪里?皇甫兄在\ldots{}\ldots{}

皇甫兄。

请坐。

穷途末路,犹如丧家之犬。仁兄(或:年兄)誉言,唉,惭愧!

啊,老丈,皇甫兄到此,有何妙计救我出关?

此计甚好,事不宜迟,就此改扮起来。

【西皮快二六】伍员在头上换儒巾,乔装改扮往东行。临潼会曾举鼎,我在万马营中显异能。时来双挂盟府印,运退深山草不生。多亏了东皋公心恻隐,请来了历阳山前(或:历阳山下)皇甫官人。我三人同把巧计定,皇甫官人假扮俺伍员去闯关门。

【西皮摇板】皇甫兄请上受一礼,

【西皮快板】多谢你施下这全恩。焚香顶礼不为敬,来生犬马当报恩。

【西皮摇板】东皋公请上礼恭敬,

【西皮快板】你是我伍员的活命恩人。但愿过得昭关境,一重恩报九重恩。伍员心中千般恨,

【西皮摇板】大胆且向虎山行。

且喜逃出昭关,我不免吴国去者。

\newpage
\hypertarget{ux6d63ux7eb1ux8bb0-ux4e4b-ux4f0dux5458}{%
\subsection{浣纱记 之
伍员}\label{ux6d63ux7eb1ux8bb0-ux4e4b-ux4f0dux5458}}

{[}第一场{]}

且住!前有大江,后有追兵,如何是好?

远望扁舟一叶,待我呼唤:小舟过来!

渡我过江,自当酬谢。

哦,原来是老丈。

我乃落难之人,望求方便。

有劳了!

老丈作歌何意?

原来如此。

这\ldots{}\ldots{}

我若说出名姓,又恐连累老丈。

愚下姓伍名员字子胥,楚国人也。

岂敢。

多谢老丈。

啊,老丈,这有祖传宝剑,上有七星,价值连城。赠与老丈,以表谢意。

多承美意。

老丈尊姓大名,日后也好图报。

如此老丈------

渔丈人。

啊------呵呵哈哈哈\ldots{}\ldots{}(笑介)

告辞了。

【西皮摇板】老丈渡我过江河,千金谢仪不为多。辞别老丈忙走却,

【西皮摇板】还有一事再相托。

啊,老丈。我虽过江,后面追兵甚急。倘若到来,老丈千万莫提你我之事。

在哪里?

哎呀!(念)\textless{}\textbf{扑灯蛾}\textgreater{}老丈投江河,投江河,不由人,珠泪落。世上多少英雄汉,好教我,难话说。

且住!老丈投江,我急急走去呀!

{[}第二场{]}

【西皮导板】豪杰打马奔吴国,

【西皮快板】龙离沧海虎离窝。樊城一呼人百诺,令出山岳不敢挪\protect\hyperlink{fn71}{\textsuperscript{71}}。力举千斤伍盟府,各国不敢动干戈。天下的诸侯皆服我,秦邦惧怕求讲和。也是我当初做事错,大不该秦楚临潼做媒妁。可叹我一家无有结果,负仇含冤奔吴国。秋半蓉花溪边落,见一娘行浣纱罗。(貌似三月桃花朵,柳眉杏眼似秋波。\protect\hyperlink{fn72}{\textsuperscript{72}})一路行来腹中饿,她篮中有饭又有馍。上前求她赒济\protect\hyperlink{fn73}{\textsuperscript{73}}我,

【西皮摇板】自觉惭愧难定夺。

娘行有礼。我乃落难之人,穷途无食,还望娘行赒济。

唉!娘行听了!

【西皮二六】未曾开言心难过,两眼不住泪如梭。家住在楚国监利玉皇阁,我父伍相扶保山河。伍子胥,就是我,临潼会斗宝压倒万国。恨平王无道纲常错,父纳子媳礼不合。我的父谏奏反遭大祸,可怜我的满门一家大小见阎罗。只剩下伍员人一个,弃走楚樊逃奔吴国。昭关赴险得逃脱,幸遇着渔父【转西皮快板】渡江河。一路行来腹中饿,只见篮中饭与馍。望求娘行赒济我,此生不忘这恩德。

【西皮摇板】这也是苍天怜惜我,凭空降下女娇娥。忙将饭食来取过,千金谢礼不为多。有朝伍员时转过,不忘娘行这恩德。

【西皮摇板】娘行一言提醒我,男女交谈礼不合。伍员溪边忙走过,

【西皮摇板】再把娘行来嘱托。

我今此去,倘有楚国人马至此,千万莫说我打此经过。

在哪里?

哎呀!

【西皮散板】一见------浣纱女子投江河。可叹你为我自尽\textless{}\textbf{哭头}\textgreater{}死,娘行啊,

【西皮散板】留得清白万古播。

(念)尔浣纱,我行乞。我腹饱,尔身溺\protect\hyperlink{fn74}{\textsuperscript{74}}。十年之后,千金报德,
千金报德。

且住!事已至此,我当急行投吴去也。

\newpage
\hypertarget{ux9c7cux80a0ux5251-ux4e4b-ux4f0dux5458}{%
\subsection{鱼肠剑 之
伍员}\label{ux9c7cux80a0ux5251-ux4e4b-ux4f0dux5458}}

{[}第一场{]}

(内)马来!

【西皮摇板】单人匹马(或:单枪匹马)弃楚樊,

【西皮快板】龙奔沧海虎奔山。历阳伏匿七夜晚,须似银条(或:发似银条)过昭关。

俺,伍员。幸喜逃出楚地,来此离吴都不远,马上加鞭。(或:且喜逃出楚地,看前面离吴都不远,就此马上加鞭。)

【西皮原板】一事无成两鬓斑,叹光阴一去不回还呐。日月轮流催晓箭,青山绿水呀常在眼前。恨平王无道纲常乱,信宠无极狗奸谗。他害我满门呐真悲惨,我与奸贼不共戴天。实指望到吴国【转西皮快板】借兵转,行至在昭关有阻拦。单人匹马常遮掩,在历阳山下(或:在历阳山前)遇高贤。设计救出了昭关险,马到长江无渡船。多蒙渔父行方便,他为我投江实可怜。浣纱女,心好善,一饭之恩前世缘。眼望吴城路不呃远,

【西皮散板】报仇心急马加鞭。

{[}第二场{]}

\textbf{老丈请转。}

\textbf{方才那一大汉与人争斗,见一妇人手执拐杖,一唤即回,是何故也(或:是何缘故)?}

\textbf{原来如此。}

\textbf{本不是(或:原非)此地人氏。}

\textbf{楚国监利人也。}

\textbf{这\ldots{}\ldots{}}

\textbf{寄居在此。}

\textbf{改日领教。}

\textbf{请------}

\textbf{(哎呀且住,)适才听老丈之言,专诸孝义双全,胁力}\protect\hyperlink{fn75}{\textsuperscript{75}}\textbf{无比(或:力大无比)。我不免与他结交,以为膀臂(或:日后定是膀臂)。}

\textbf{正是:(念)交友当交真君子,求人须求大丈夫。(或:交友须交奇男子,求人当求大丈夫。)}

\textbf{来此已是。}

\textbf{专兄开门来。}

\textbf{愚下来访。}

\textbf{请。}

\textbf{有座。}

\textbf{愚下姓伍名员字子胥,(乃)楚国监利人也。}

\textbf{岂敢。}

\textbf{唉,专兄啊!}

\textbf{【}西皮快板\textbf{】平王无道乱楚邦,父纳子媳(或:父纳子妻)乱纲常。特到贵邦借兵将,奈无机会见吴王。}

\textbf{(奈无机缘,不敢造次。)}

\textbf{若得如此,乃伍员之幸也。}

\textbf{闻得专兄孝义双全,愚下愿与专兄结为金兰(或:愚意欲与专兄结为金兰之好)。幸勿见却。}

\textbf{愚下真心实意,专兄休得谦逊(或:休得过谦)。}

\textbf{理当拜见(或:原要拜见)。}

\textbf{(请!)}

\textbf{请呐。}

\textbf{【}西皮摇板\textbf{】孝义双全人钦仰,}

\textbf{【}西皮摇板\textbf{】报仇之事全仰仗。}

\textbf{(专诸 【}西皮摇板\textbf{】\ldots{}\ldots{}见吴王。)}

{[}第三场{]}

\textbf{唉!}

\textbf{【西皮快板】行过东来又转西,举目无亲独自依。落魄的英雄谁怜惜(或:衣衫褴褛谁赒济),吹箫怎能(或:焉能)充得饥。}

\textbf{唉!俺伍员自到吴地求见姬光,奈无机会。(或:自与专诸结拜,吴王难见,)行囊典尽,衣履全无,只落得吹箫讨乞!}

【西皮原板】姜太公不仕啊隐磻溪呀,运败时衰鬼神欺。周姬昌治西岐(或:梦飞熊)呀在灵台殿里,渭水河访贤保社稷。东迁雒邑\protect\hyperlink{fn76}{\textsuperscript{76}}【转西皮二六】王纲呃坠,各国诸侯把心离。背盟毁约失信呃义,图霸争强各自为。吴子寿梦立王位,力压诸侯服四夷。某单人独骑弃楚呃地,要见姬光恨无机。困苦的英雄似蝼蚁,眼见得含冤化灰泥。落魄天涯谁赒济,只落得吹箫暂充饥。

【西皮快板】正在街前闲站立,见一位官长(或:见一位君官)相貌奇。头戴珠冠双凤翅,身穿一件衮龙衣。莫非他是姬太子,特地前来(或:有意前来)访子胥。本当向前去见礼,帽破衣残不整齐。眉头一皱心生计,将我的冤仇提一提。

\textless{}\textbf{三叫头}\textgreater{}爹爹!母亲!唉!兄长啊!

【反西皮散板】子胥阀阅门楣第,到如今(或:俺好似)凤褪翎毛怎能飞。我本是英雄不得第,落魄天涯有谁知。可叹我父母的冤仇沉海底呀,空负我堂堂七尺躯啊。伍子胥呀,伍盟府哇,父母冤不能报,爹娘啊!

来了!

【西皮快板】听说一声唤子胥,愁人脸上笑微微。走向前,施一礼,愿王福寿与天齐。

千岁!

【西皮二六】富贵穷通不由己,也是我的时衰命运低。我本是楚国的功臣家住在监利,姓伍名员字子胥。恨平王无道宠无极,败坏纲常父纳子媳。我的父谏奏剑下死,一家满门血染衣。闻千岁招贤纳士多仁义,我是特地前来借兵报冤屈。孤身到了吴国地,还望纳员\protect\hyperlink{fn77}{\textsuperscript{77}}【转西皮快板】把难人提。伍子胥一朝得了第,知恩报德不敢移。

{[}第四场{]}

(谢座。)

(二位)先生请坐。

【西皮快板】含悲忍泪叫贤弟,有苦之人把话提。特地借兵到此地,不杀平王气怎息。

(谢座。)

(启禀千岁:)臣有一结义之弟(或:结拜义弟),名唤专诸。此人胁力无比(或:此人英勇双全),堪当此任。

姑苏城外(或:姑苏城内),一唤即至。

领旨。(或:遵命。)

【西皮摇板】辞别千岁奉聘礼,

\textbf{【}西皮摇板\textbf{】}风吹呀云散现虹霓。

\newpage
\hypertarget{ux6b66ux662dux5173-ux4e4b-ux4f0dux5458}{%
\subsection{武昭关 之
伍员}\label{ux6b66ux662dux5173-ux4e4b-ux4f0dux5458}}

\textbf{{[}第一场{]}}

\textbf{(念)拜相封侯印,盖世掌乾坤。扶保楚王主,四路扫烟尘。}

\textbf{俺,伍员,楚国监利人也。可恨楚王无道,纲常败坏,父纳子媳。是我心头不忿,保定皇家四口}\protect\hyperlink{fn78}{\textsuperscript{78}}\textbf{,我就反------出了朝堂。}

\textbf{俺伍员好比淤泥圬住车}\protect\hyperlink{fn79}{\textsuperscript{79}}\textbf{,沙滩------困蛟龙。}

\textbf{天呐,天!俺伍员好比一轮明月,却被乌云}\protect\hyperlink{fn80}{\textsuperscript{80}}\textbf{遮盖也。}

\textbf{且住!耳旁听得金声呐喊,想是卞庄追兵到来,不免请出国太凤驾启程。}

\textbf{国太有请!}

\textbf{臣,伍员见驾。}

\textbf{国太、幼主,千岁,千千岁!}

\textbf{是臣听得金鼓呐喊,想是卞庄追兵到来,特请国太凤驾启程。}

\textbf{臣。}

\textbf{领呐------旨!}

\textbf{【二黄散板】耳旁听得金鼓震,必是卞庄发来兵。远望松林一寺院,请国太下龙驹臣挡贼兵。}

\textbf{(马昭仪 【二黄导板】兵困禅宇)}

\textbf{【接二黄导板】马后悲呀,}

\textbf{【回龙】那卞庄追兵似虎威。}

\textbf{【二黄原板】在葵花井(或:葵花柱)边拴战马,}

\textbf{虎头,}

\textbf{【二黄原板】虎头金枪插在丹墀。将身儿来至在西廊下,且听国太降玉音。}

\textbf{【二黄原板】未曾开言先落泪,臣有一本奏得知:那卞庄贼好一比得胜狸猫,他欢似虎,为臣我虎落平阳反被犬欺。幼主爷似蛟龙,龙离了沧海反遭虾戏,龙国太似凤凰难腾飞。臣要保保定小千岁,难保国太杀出重围。}

\textbf{【二黄原板】子胥闻言屈膝跪,臣有二本奏端的:国太年轻花容粉,为臣年纪正青春。知者说是臣保主,}

\textbf{(马昭仪 不知者?)}

\textbf{【二黄散板】是一对少年人。}

\textbf{【二黄散板】一句话儿错出唇,国太一旁怒气生。屈膝跪在尘埃地,过往神灵听分明:我若保主有假意,}

\textbf{罢!}

\textbf{【二黄散板】死在千军万马营。}

\textbf{【二黄散板】西廊下又来了保驾臣。}

\textbf{【二黄散板】撩铠甲且把佛殿进呐,见了国太愧煞人。}

\textbf{【二黄散板】用手接过小储君。}

\textbf{【二黄散板】打开甲胄藏幼主呃,拜别国太忙登程。}

\textbf{【二黄散板】葵花井边马牵定,国太有话快些云。}

\textbf{【二黄散板】有朝幼主登龙位,君是君来臣是臣。}

\textbf{【二黄散板】国太快把心拿稳,放我君臣早逃生。}

\textbf{【二黄散板】一见国太寻自尽,子胥倒做不忠臣。}

\textbf{【二黄散板】推倒土墙忙掩井呐。}

\textbf{{[}第二场{]}}

\textbf{杀败了啊,杀败了。}\protect\hyperlink{fn81}{\textsuperscript{81}}

\textbf{(念)走国离家日如年,不杀楚王心不甘。}

\textbf{(念)打开甲胄观幼主,}

\textbf{哈哈、哈哈、啊,呵哈哈哈\ldots{}\ldots{}(笑介)}

\textbf{(念)一夜须白过昭关。}

\textbf{呔!卞庄贼子休赶,伍将军去也。}

\newpage
\hypertarget{ux897fux65bdux6e38ux6e56-ux4e4b-ux8303ux8821}{%
\subsection{西施·游湖 之
范蠡}\label{ux897fux65bdux6e38ux6e56-ux4e4b-ux8303ux8821}}

(刘曾复 饰 范蠡、梁小鸾 饰 西施;李斌植 司鼓、屠楚材 操琴)

范蠡 【西皮导板】整顿山河心事了啊,

西施 【西皮原板】五湖烟水任逍遥。浮哇云------

范蠡 【西皮原板】百年富贵谁能料,

范蠡 功成------

西施 【西皮原板】功成身隐是英豪。远呢望------

范蠡 【西皮原板】远望西山颜色好,

范蠡 桃花------

西施 【西皮原板】桃花千树逞新娇。云呢水------

范蠡 【西皮原板】云水光中来放棹,

西施 【西皮摇板】一行白鹭上春潮。

范蠡
你我功成身退,泛游五湖,好不洒乐\protect\hyperlink{fn82}{\textsuperscript{82}}人也!

西施 相公言得极是!看前面桃花深处,不知何等所在。

范蠡 乃是西塞山。

西施 何不前去游玩一番?

范蠡 言得极是!船家------

船娘 有。

范蠡 将船开往西塞山去者!

船娘 是。

范蠡 将酒安排好了!

船娘 是。

范蠡 娘子请!

西施 相公请!

范蠡 干!

范蠡
你我二人在此饮酒,无拘无碍。若非娘子去往吴国,建立大功,焉有今日。今日卑人得享清福,俱是娘子所赐的了!

西施
相公说哪里话来,妾身忍辱事仇,无非为国,是功是过,都听后人的评论,提它怎的?

范蠡 娘子请!

西施 相公请!

范蠡 干!

范蠡
啊,娘子,何不乘此酒兴,将当年吴宫之事,细表\protect\hyperlink{fn83}{\textsuperscript{83}}一回。

西施 提起来惭愧得很!相公容禀:

西施
【西皮二六】提起了吴宫心惆怅,犹如一梦熟黄粱。朝朝暮暮在姑苏台上,馆娃宫西畔又筑响屧廊。三千粉黛人人怅惘,一身宠爱迷惑吴王。佯欢假媚多勉强,柔肠百转度流光。功成喜见贤君相,

西施 相公啊!

西施 【西皮摇板】这才是天从人愿【回龙】配才郎。

范蠡 哦------

范蠡 【西皮摇板】听罢言来喜心上,巾帼英雄世无双。

范蠡 娘子这平吴霸越之功,足令千古女流吐气了!

西施 相公夸奖!

范蠡 看天色不早哇,我们回去罢!

西施 回去罢。

范蠡 收拾下去!

船娘 是。

范蠡 开船呐!

范蠡 【西皮散板】猛回头又只见江山似锦呐,

\textbf{西施 【西皮散板】博得个浣纱女万古留名。}

\textbf{桑园会 之 秋胡}\protect\hyperlink{fn84}{\textsuperscript{84}}

\textbf{{[}第一场{]}}

(四青袍引秋胡上)

{[}引子{]}归心似箭,辞王驾,转回家园。

(念)抛妻别母二十春,千里迢迢无信音。人生若得全忠孝,臣报君恩子奉亲。

下官秋胡,鲁国人也。在楚为官,官居上大夫之职。只因我离家日久,不知老母妻室怎生度日。为此辞王别驾,回家探望。多蒙楚王赐我黄金人役,来此离家不远,我不免改换行装,以免惊动乡邻。

左右,看衣改换。

尔等远远跟随。

与爷带马。

【西皮原板】想当年悲切切楚国投奔,今日里笑吟吟荣转家门。带黄金喜孜孜萱堂侍奉,安慰了娇滴滴我那少年夫人。

\textbf{{[}第二场{]}}

(内)马来!

【西皮快板】秋胡打马奔家乡,路上行人马蹄忙。勒住丝缰【转西皮摇板】来观望,

【西皮快板】见一个妇人手攀桑。前影好似罗敷女,后影儿又像我的妻房。本当上前把妻认,错认了民妻罪非常。\protect\hyperlink{fn85}{\textsuperscript{85}}

哎呀且住!看那旁采桑妇人,好像我妻罗敷模样,怎奈我离家日久,不敢冒认。

待我下马问来。

啊,大嫂请了!

并非失迷路途,我乃找名问姓的。

请问大嫂,此处有一秋胡,大嫂可晓得?

那秋胡与我同朝为官,又有八拜之交,托我带来万金家书,故而动问。

我那秋兄言道,这书信么,要面交本人。

原书带回。

大嫂听了!

【西皮快板】站立在桑田把话讲,尊一声大嫂听端详:家住鲁国古田桑,姓秋名胡字高翔。他父名叫秋楚望,二十年前早已亡。老母柯氏六旬上,白发苍苍年迈的娘。娶妻本是罗敷女,苦持贞节守空房。临行送在了阳关上,叮咛的言辞记在心旁。但愿得早登龙虎榜,即刻里修书转还乡。此乃是秋兄对我讲,并无有虚言哄娘行。

大嫂要看书信,但不知你是他家何人?

哦,原来是秋大嫂,失敬了。

哦。

【西皮摇板】听罢言来喜心上,果然是罗敷来采桑。

哎呀且住。想我秋胡离家二十余载,也不知她的光景如何?呃\ldots{}\ldots{}我自有道理呀。

啊大嫂,卑人有几句言语,要对大嫂言讲啊。

大嫂听了!

【西皮导板】秋胡他把良心丧,

唉!大嫂哇!

【西皮原板】他在那楚国配了鸾凰。我劝他回家他不往,丢下了大嫂守空房。你好比鲜花【转西皮二六】空开放,又好比明珠土内藏。你好比皓月当空明朗朗,

【西皮快板】晴天无日少阴阳。趁此\protect\hyperlink{fn86}{\textsuperscript{86}}桑田无人往,学一个巫山神女会襄王。

【西皮快板】大嫂不是这样讲,细听卑人说比方:枯木逢春花又放,人生几度好时光。千里的姻缘从天降,陌上相逢非寻常。阳关大道无人往,学一个织女配牛郎。

哦!

【西皮摇板】我二人俱是空言讲,平地怎能托空梁。取出黄金有数两,假意求作凤鸾凰。

啊,大嫂。常言说得好(或:道得好):``力田辜负青年少,采桑何如嫁富郎''。我这里有马蹄金一锭,赠与大嫂,来来来,看金呐。

哦,在哪里?哦\ldots{}\ldots{}

呵哈哈哈\ldots{}\ldots{}(笑介)

【西皮摇板】黄金不顾回家往,贞节烈女世无双。拾起黄金忙赶上(或:把马上),回到家去奉高堂。

\textbf{{[}第三场{]}}

【西皮快板】村庄以外下了马,杨柳深处是我家\protect\hyperlink{fn87}{\textsuperscript{87}}。来在门前用目洒,

(秋母 哦\ldots{}\ldots{})

【西皮快板】老娘亲缘何怒声哗\protect\hyperlink{fn88}{\textsuperscript{88}}。

【西皮快板】进得门,忙跪下,\protect\hyperlink{fn89}{\textsuperscript{89}}儿是秋胡转还家。

【西皮快板】打罢春,又转夏,日月轮回催岁华。少年子弟江湖老,娘的青丝也转白发。

母亲请上,待孩儿大礼参拜。

久离膝下,少奉甘旨。母亲恕罪。

谢母亲。

谢座。

乃上大夫之职。

当谢天地。

谢座。

啊母亲,孩儿回家半日,为何不见娘的儿媳,她往哪里去了?

呃,母亲,不要唤她,呃,不要唤她\ldots{}\ldots{}

哎呀,糟了。

在。

在这里。

哎呀,母亲呐!孩儿一时难以分辩,恐她后面自、自、自\ldots{}\ldots{}自尽去了!

【西皮散板】大不该在桑田调戏她呀。

\textbf{{[}第四场{]}}

啊娘子,千不是,万不是,俱是卑人的不是,我这厢赔礼了。

呵呵哈哈哈\ldots{}\ldots{}(笑介)

老娘!

【西皮快板】老娘亲息怒容儿禀,水有源流树有根:孩儿打马桑田进,夫妻们见面我认、是认也认不清(或:认不真)。千不是来万不正,情愿上前我赔个小心。

【西皮摇板】走向前来把礼敬,

啊娘子,方才在桑田是卑人的不是,喏喏喏,我这厢赔礼了。

啊娘子,卑人这厢就又赔礼了。

诶。

【西皮快板】\textbf{佯偢不睬}\protect\hyperlink{fn90}{\textsuperscript{90}}人上人。是是是,明白了,想是道我礼貌轻。我这里上前忙跪定,

诶。

【西皮快板】背转身来自思忖。常言道男儿膝下有黄金,岂肯低头跪妇人。

是。

【西皮快板】母亲教训儿遵命,秋胡岂是不孝的人。二次向前屈膝跪,

我这一条腿么,一路之上,受尽风霜,不跪也罢。

哎,不错不错,是得了风寒之症了。

哦,母亲会治。

哎哟!

【西皮快板】尊一声娘子听分明:老母多亏你孝敬,一来赔罪二谢恩。

母亲不要跪。

呵呵哈哈哈\ldots{}\ldots{}(笑介)

【西皮摇板】多谢娘子开了恩,再谢老娘讲人情。

【西皮摇板】一家骨肉团圆庆,只享荣华不受贫。

啊母亲,孩儿挣来官诰,母亲请来穿戴。

是。

正是:(念)秋胡离家二十春,

(秋母
(念)盼得为娘两眼穿\protect\hyperlink{fn91}{\textsuperscript{91}}。)

(罗敷 (念)今日夫妻重相见,)

(念)只享荣华不受贫。

母亲。

来了。

回来。

我来问你,方才在桑田,呃,是我的不是呀。你不该,在母亲面前搬动是非,教我罚跪在堂前。幸喜无人前来呀,倘若有人看见,呃,成何体统啊?

呃,若是不看在母亲的份上呢\protect\hyperlink{fn92}{\textsuperscript{92}}?

哼哼,还是这样的生气呀。

我若是不看在母亲的份上啊,我早就------跪下了。

呵呵哈哈哈\ldots{}\ldots{}(笑介)

哈哈哈\ldots{}\ldots{}(笑介)

哎呀,列位呀,不要见笑哇,这是我秋胡的家规哟。

\textbf{附注}:

文献{[}12{]}.和吴焕老师整理的剧本都曾提到,刘曾复先生本剧的开场的唱词来源于石君宝的元杂剧《鲁大夫秋胡戏妻》第三折:

``(秋胡冠带上,云)小官秋胡是也。自当军去,见了元帅,道我通文达武,甚是见喜,在他麾下,累立奇功,官加中大夫之职。小官诉说,离家十年,有老母在堂,欠缺侍养,乞赐给假还家。谢得鲁昭公可怜,赐小官黄金一饼,以充膳母之资。如今衣锦荣归,见母亲走一遭去。

(诗云)想当日哭啼啼远去从军,今日个笑吟吟荣转家门。捧着这赤资资黄金奉母,安慰了我那娇滴滴年少夫人。(下)''

\newpage
\hypertarget{ux9ec4ux91d1ux53f0-ux4e4b-ux7530ux5355}{%
\subsection{黄金台 之
田单}\label{ux9ec4ux91d1ux53f0-ux4e4b-ux7530ux5355}}

\textbf{{[}第一场{]}}

(内)掌灯。

(内)【二黄导板】听呐谯楼打四更玉兔明亮,

【回龙】为国家秉忠心昼夜奔忙。

【二黄快三眼】西凉国前三载来把贡上\protect\hyperlink{fn93}{\textsuperscript{93}},进邹妃和伊立来献大王。吾主爷见邹妃龙心欢畅,每日里贪酒色不理朝纲。满朝中文武臣狐群狗党,眼见得齐疆土付与汪洋。

(念)身在朦胧睡,怀揣社稷忧。

下官田单,齐王驾前为臣,官居巡城御史。只因君王无道,听信谗言,不理朝政。适才闻报,东宫世子连夜逃出皇城,不知为何何事(或:不知是何缘故)。为此亲查御街。

左右,掌灯!

【二黄散板】教人来呀掌红灯把路引,倘有那面生人细问分明。

哦,你拿住犯夜的了!

回衙有赏。

哦,你也拿住犯夜的了!

好,也有赏。

呃!夜静更深,哪有许多犯夜之人。

掌灯!

待我观看!

噤声!

\textbf{{[}第二场{]}}

千岁醒来。

【二黄散板】问千岁因何故逃出宫禁,一桩桩、一件件细对臣云。

【二黄散板】咬牙切齿把贼恨,苦害幼主为呀何情。

再去打探。

哎呀千岁呀!今有伊立带领校尉,前来搜寻,这便如何是好?

\textless{}\textbf{叫头}\textgreater{}千岁!

事到如今,只好扮作臣妹模样,混过一时,再作道理。

乳娘,搀扶千岁,改扮起来。

【二黄导板】水不清呐皆因是呃鱼儿搅混啊,

【二黄散板】我朝中又出了哇卖国谗臣(或:谋国奸臣)。

【二黄散板】走向前施一礼把千岁爷驾请呐。

(田法章 【二黄散板】\ldots{}\ldots{}扮妇人。)

(田法章 扮得可像?)

扮得倒像。千岁可晓得妇人家行走?

呃,乳娘教导千岁。

好,着着着。

公公。

与公公掸座。

当得的。

(啊。)呵呵哈哈哈\ldots{}\ldots{}(笑介)

公公这是则甚?

呜哙呀,下官怎敢呐?

哎,不敢呐。

请坐。

不知公公驾到,未曾远迎,当面恕罪!

岂敢呐岂敢!

哦!

啊公公,这夜静更深,带领许多校尉,来至敝衙(或:来到敝衙),为了何事?

这\ldots{}\ldots{}(或:呃,)下官倒还不知呀。

公公请讲。

啊公公,想朝中出了这样(的)大事,你我为大臣者,就该(或:皆应)上殿保奏的才是啊。

呃,哪个?

哦,公公保过本了。

也罢,待下官差人四下寻访。有了世子的下落报与公公,可也就是了!

哦,要怎样的讲法(呢)?

多承公公的美意。只是世子他\ldots{}\ldots{}

(呃,)不在敝衙。

当真不在。

果然不在。

公公要怎样?

公公(当真)要搜?

果然要搜?

(好,)请搜!

(啊,)呵呵呵哈哈哈\ldots{}\ldots{}(笑介)

我道是谁(或:我当是哪个;我道是哪个) ,原来是乳娘同小妹
,小妹同乳娘啊。哈哈哈哈\ldots{}\ldots{}(笑介)

呃,不敢,乃是小妹。

小妹礼貌不周,冒犯公公,那还了得。呃,不见也罢。

要见?

呃,就见!

乳娘搀扶小姐,见过伊公公。

天高无路,怎能上去?

地厚无门,如何下得去呀。

公公。

岂敢。

呃,忒谦了。

(说)哪里话来。

好贼!

(伊立 \ldots{}\ldots{}再搜查。)

请!

奸贼!

【二黄散板】一见贼子(或:一见奸贼)出府门,不由田单咬牙根呐。

【二黄散板】再与千岁把计定。

(田法章 【二黄散板】\ldots{}\ldots{}出府门。)

哎呀千岁呀,虽然混过一时。只恐那贼再来搜寻,如何是好?

哎呀千岁呀,趁此天还未明,你我君臣扮作进香之人模样,混出城去,再作道理。

(如此你我君臣)改扮起来。

乳娘,这有纹银,收拾收拾,逃回原郡去罢!

来也!

噤声!

\textbf{{[}第三场{]}}

(田法章 哎呀,卿\ldots{}\ldots{})

噤声!

【二黄碰板】千岁爷休得要放悲声,

【二黄原板】泄漏了机关事难成。那一旁松林内将臣等,寻一个妙计好过此城。我这里取尘埃(或:取灰尘)把脸呐罩定,

【二黄散板】我假装疯魔(要)混出城。

(皂隶 \ldots{}\ldots{}拿过来罢。)

``田''什么?

哦你还认得我么?

呃,你可晓得我姓什么?

我不赵\protect\hyperlink{fn94}{\textsuperscript{94}}。

我没有钱。

哎呀你好眼力呀。

我好你可好啊?

呃,老太太么?呃,病了。

吃了哇。

吃了你的药么,病就好了。

只因家母病体痊愈,我们要出城烧香还愿呐。

怎么不凑巧?

诶,无官不私,无水不鱼呀。

呃,行个方便吧。

你行个方便吧。

(呃,我是怯官呐。)

哎呀,我怯官呐。

正是。

东岳庙。

有一个东岳庙哇。

有一个,

(呃,呃,)有一个东岳庙哇。

呃,不错。

是的是的。

哎呀,这样的啰嗦啊。(或:哎呀,太啰嗦了。)

乃是兄妹二人。

方才我言过(或:方才言道),是兄妹二人呐。

是兄妹二人。

呃,是兄妹二人。

呃。

不错,她也来了。

小妹快来。

``东''什么?

哦,你还记得她?(或:哼,你还记得么?)

(皂隶 嘿,看她是长大喽。)

(皂隶 \ldots{}\ldots{}教您过去呐。)

哦,是是是。

哎呀,我辛苦你了。

难为你了。

来来来,这有(或:我这里有)一茶之敬,吃饭,不饱,饮酒(或:吃酒),不醉,

哎呀,我的盘费要紧呐。

呃,请了,请了。

\newpage
\hypertarget{ux53d6ux8365ux9633}{%
\subsection{取荥阳}\label{ux53d6ux8365ux9633}}

\textbf{{[}第一场{]}}

钟离眛\protect\hyperlink{fn95}{\textsuperscript{95}}
(念)并吞六国秦始皇,

季布 (念)修造\protect\hyperlink{fn96}{\textsuperscript{96}}长城建阿房。

钟离眛 (念)义帝有道江山掌,

季布 (念)楚汉分兵定咸阳。

钟离眛 俺,钟离眛,

季布 季布。

钟离眛 请了。

季布 请了。

钟离眛 大王发兵,夺取荥阳,两厢伺候!

季布 请!

项羽
\textless{}\textbf{点绛唇}\textgreater{}浩荡旌旗,战鼓擂齐,军雄威;要战如雷,催马龙虎退。

项羽
(念)二将壮军威,曾交战如雷\protect\hyperlink{fn97}{\textsuperscript{97}}。似雪倒缨盔,照耀映光辉。

项羽
孤,西楚项羽,自江淮起义以来,百战百胜。昨日探马报道:今有韩信领兵燕赵去了,荥阳必定空虚,为此带领人马夺取荥阳,以令天下。

项羽 来,传钟、季二将进帐。

众 二将进帐。

钟离眛、季布 来也!

钟离眛 (念)万马军中气概雄,

季布 (念)人似飞虎马似龙。

钟离眛、季布 参见大王。

项羽 免。

项羽 二位将军,人马可齐?

钟离眛、季布 俱已齐备。

项羽 兵发荥阳。

钟离眛、季布 得令!众将官,兵发荥阳!

众 啊!

\textbf{{[}第二场{]}}

张良 (念)口似悬河语似流,

陈平 (念)舌似钢锋运机谋。

张良 (念)楚汉交兵何日息,

陈平 (念)灭却重瞳方罢休。

张良 山人张良,

陈平 下官陈平。

张良 请了,主公升帐,两厢伺候。

陈平 请。

刘邦 {[}引子{]}保义安天命,但愿得,早定乾坤。

张良、陈平 臣等见驾,主公千岁。

刘邦 平身。

张良、陈平 千千岁。

刘邦 赐坐。

张良、陈平 谢座。

刘邦
(念)提剑斩蛇聚英雄,干戈一起定关中。重瞳不遵怀王约,强霸虎踞冒吾功。

刘邦
孤刘邦,字季子。自沛丰起义以来,扫灭嬴秦,先定关中。可恨项羽不遵怀王之约,强霸为王,将孤贬封蜀中。幸得萧何、子房等,保荐韩信,登台拜帅,暗渡陈仓,复夺三秦。大兵已扎荥阳,暂为犄角之势。前者霸王被韩信车战,败回彭城。如今韩信兵伐燕赵去了,彭越去往东京,九江王染病在床。又恐霸王乘虚来夺荥阳,也曾命人前去打探,未见回报。

大太监 报!启大王:今有霸王前来夺取荥阳,请大王敌楼答话。

刘邦 再探。

大太监 啊!

刘邦 二位先生:霸王兵困荥阳,教孤敌楼答话,孤去也不去?

张良
此乃范增之谋,知韩信兵发燕赵去了,趁此虚空,来取荥阳。臣等保定主公,敌楼观看动静,回来再作道理。

刘邦 言之有理。带马------

刘邦
【西皮摇板】荥阳城外摆战场,将士纷纷马蹄忙。韩信领兵燕赵往,无有能将敌霸王。君臣且把敌楼上,

刘邦 【西皮摇板】旌旗不住空中扬。重重叠叠兵和将,

刘邦 哎呀!

刘邦
【西皮摇板】刀枪剑戟似秋霜。孤王一见心胆丧(或:魂胆丧),只恐难以保荥阳(或:难保这荥阳)。

项羽 (内)【西皮导板】忆昔当年渡淮江,

项羽
【西皮原板】百战百胜威名扬。多少能将枪尖丧,能征惯战鞭下亡。坐至在雕鞍用目望,城楼站的小刘邦。左有陈平狗奸党,右边站立张子房。勒住马头【转西皮快板】把话讲,开言大骂小刘邦:你本是沛县一亭长,你敢与孤夺家邦。快快开城来打仗,看看谁弱哪家强。

刘邦
【西皮二六】站立在敌楼把话讲:开言尊声楚霸王。休提起沛县一亭长,提起了当年孤就怒满胸膛。同扶怀王把业创(或:基业创),楚汉分兵进咸阳(或:定咸阳)。先到咸阳为皇上,后到咸阳辅保朝堂。你不遵王约太狂妄,反将(或:反把)孤刘邦赶出了咸阳。若不是韩信他的韬略广,孤的人马也不能暗渡陈仓。复夺三秦军威壮,岂不知强中自有强中强?

项羽
【西皮快板】闻言怒发三千丈,开口大骂小刘邦。说什么同把江山创,楚汉合兵定咸阳。孤王出兵谁敢挡,纵横天下楚霸王。鸿门宴将你放,放虎归山反逞强。你道韩信韬略广,雪霜焉能见太阳。

刘邦
【西皮快板】霸王不必(或:休得)夸口讲,肉眼不识紫金梁。自从起义在沛上,分茅裂土古之常。你不该强把诸侯抗(或:挡),你不该举火焚阿房。你不该斩杀楚降将(或:归降将),你不该逼死楚怀王。明知韩信燕赵往,乘虚兴兵取荥阳。任你兵多将又广,休教刘邦(或:休劝孤王)来归降,(你)枉费心肠。

项羽
【西皮快板】你父太公在罗网,吕氏夫人笼中藏。你若不肯来归降,太公、吕后刀下亡。父子不能同欢畅,结发夫妻两分张。手摸胸膛想一想,看你归降不归降。

刘邦 【西皮导板】刘邦闻言心欢畅,

刘邦 呵呵呵哈哈哈\ldots{}\ldots{}(笑介)

刘邦
【西皮快板】开言尊声楚霸王。说什么(或:讲什么)吾父太公入罗网,(讲什么)吕氏妻子笼中藏。当初结义情谊长(或:义气长),犹如同父共同娘。老父、妻室蒙君养,生死存亡你主张。你父我父俱一样,忍心杀害也无妨。劝你早退兵和将,韩信到来你又着忙。

项羽
【西皮摇板】匹夫出言太无状,你拿韩信压孤王。回头再叫二员将,孤王言来听端详:快将荥阳齐围上,休要放走小刘邦。

刘邦 不好了!

刘邦 【西皮摇板】一言激怒楚霸王,犹如倒海似翻江。君臣且归黄罗帐,

刘邦 【西皮摇板】再与二卿作商量。

刘邦 二位先生,霸王攻打荥阳甚急,有何良策?

张良 大王修书一封,去往燕赵,调韩信回来,以挡霸王之勇。

刘邦 待孤修书。

陈平
且慢。想荥阳与燕赵相隔甚远,一时焉能得到?荥阳乃弹丸之地,倘有人献计,将荥河之水从上而下冲灌前来,荥阳化为齑粉矣!

刘邦 不好了!

刘邦
【西皮摇板】层层围困小荥阳,倒教孤王无主张。插翅不能出罗网,有何良策保孤王?

陈平 【西皮摇板】君忧臣愁心惆怅,倒教陈平少主张。回头便对先生讲,

陈平 先生,

陈平 【西皮摇板】有何妙计救君王。

张良
【西皮快板】一片丹心扶汉王,全凭韬略定封疆。汉王闷坐\protect\hyperlink{fn98}{\textsuperscript{98}}黄罗帐,这一旁难坏张子房。我也曾背剑把韩信访,我也曾赚楚定咸阳。三分天下有二项,一时无计救汉王。低下头来暗思想,

张良 有了!

张良 【西皮快板】猛然一计在心旁。回头我对大王讲,君臣宽怀慢商量。

张良 启主公:臣有一计可保我主出得荥阳。

刘邦 有何妙计?

张良
臣思得东周列国,一段忠义故事。臣回至馆驿。约请文武,倘有忠义之臣,替主赴难,也未可知。

刘邦 但凭先生。

张良 领旨。

张良
【西皮摇板】主公但把宽心放,圣主驾前有栋梁。霸王纵有天罗网,管保困龙上天堂。

陈平 【西皮摇板】躬身施礼出宝帐,但愿我主离荥阳。

刘邦
【西皮摇板】张良、陈平出宝帐,再叫帐前众儿郎:免战牌高挂敌楼上,滚木擂石要谨防(或:要提防)。

\textbf{{[}第三场{]}}

张良 【西皮摇板】适才离了黄罗帐,回到馆驿想良方。

张良
山人张良,今有霸王攻打荥阳甚急,山人思得一计,以解此危\protect\hyperlink{fn99}{\textsuperscript{99}}。来。

童儿 有。

张良 拿我名帖邀集文武,明日齐至馆驿饮宴,共议军机。

童儿 是。

张良 正是:(念)设计救君难,开筵画图悬。

\textbf{{[}第四场{]}}

曹参 下官曹参。

周勃 下官周勃。

纪信 下官纪信。

随何 下官随何。

曹参 列位大人请了。

众 请了。

曹参 军师有帖相邀,不知为了何事?

众 齐至馆驿,便知明白。

众 来,

众军士 有!

众 打道馆驿!

众军士 啊!

随从甲 (念)奉了军师命,堂上挂图形。

随从乙 奉了军师之命:打扫中堂,悬挂图形。

随从甲 就此悬挂起来!

随从乙 悬挂已毕!

张良 门外伺候!

众军士 (来至)馆驿!\protect\hyperlink{fn100}{\textsuperscript{100}}

众 回避了!

曹参 门上哪位在?

童儿 什么人?

曹参 我等到齐。

童儿 众位大人稍候,有请军师。

张良 何事?

童儿 列位大人到。

张良 有请。

童儿 有请。

众 请------

众 军师请上,受我等参拜。

张良 山人也有一拜。

张良 请坐------

众 告坐。

张良 不知列位驾到,有失远迎,望乞恕罪。

众 我等来得鲁莽,军师海涵。

张良 岂敢。

众 军师相邀,有何见谕?

张良
君忧臣愁,古之常理。列公乃忠义之士,我主被困荥阳,列公有何高见,解君之忧?

众 我等才疏学浅,全仗军师。

童儿 宴齐。

张良 看宴。

张良 待我把盏。

众 不敢,摆下就是。

张良 众位大人请------

张良 大人请------

众 军师请!

众 请问军师:上面挂的画图,是哪朝故事?

张良
此乃东周列国齐晋交兵,一段忠义故事,大有可谓\protect\hyperlink{fn101}{\textsuperscript{101}}。

众 军师请道其详。

张良 少时饮罢再叙。

众 我等酒已够了。

张良 将宴撤去。

众 军师请道其详。

张良
此乃齐晋交兵,战于曲阳,晋兵强盛\protect\hyperlink{fn102}{\textsuperscript{102}},齐军大败。

众 车内敢莫就是齐顷公?

张良
非也。此乃参军逄丑父\protect\hyperlink{fn103}{\textsuperscript{103}}。

众 哦,逄丑父。

张良
见齐军大败,齐顷公吓得面如土色。丑父奏道:事已危急,大王可将衣帽脱下,与臣穿戴,坐于车中,大王林中藏躲。顷公言道:我虽然逃难\protect\hyperlink{fn104}{\textsuperscript{104}},卿家必定遭擒,存亡难定,吾不忍也。丑父奏道:食王之禄,当报君恩。臣一命好比大树林中落下一叶耳;若存大王,称为万姓之主,使天下受福不小也。

张良
【西皮摇板】人说光阴重黄金,我把光阴比浮云。舍身救主忠义尽,留得美名万古存。

张良 顷公见丑父忠心耿耿,遂将衣帽脱下与丑父穿戴,自己向林中藏躲呵。

众 那林中藏躲的就是齐顷公么?

张良 正是。

众 那逄丑父后来呢?

张良
晋军赶上,将丑父擒住,只道他是齐顷公,献与了晋侯,查其来历,知是丑父,彼时推出斩首,丑父大笑。晋公曰:不避死而代君难,君得生,全其忠也;杀之不祥,当赦其罪。遂将丑父释放回国。顷公见其忠义,封以爵位。今汉王被困荥阳,我等空食君禄,竟无一人学那逄丑父之故事耳。

众 哦!

张良
【西皮快板】昔日丑父身代君,舍身救主不畏刑。若得一人效丑父,假扮汉王诓楚君。忠义凛凛鬼神敬,封妻荫子标美名。

曹参 【西皮摇板】军师悬图(或:挂图)表忠论,

周勃 【西皮摇板】方显我等不忠臣。

随何 【西皮摇板】堂堂君王遭围困,

纪信 呵!

纪信 【西皮摇板】愿学前辈古贤人。

众 师爷,自古道:父有难子当代,君有难臣当替。我等愿效丑父舍身救主。

张良 公等愿舍生救主,真乃难得。但只一件\ldots{}\ldots{}

众 哪一件?

张良 必须与我主容颜相似,年貌相当,假扮汉王,方可救得吾主。

众 我等站齐,师爷请看。

张良 待山人看来。

张良
我观列位大人,各有不同;惟有纪将军与主龙颜相似,替主赴难,非纪将军不可。

纪信 哦!

纪信
【西皮二六】纪信闻言心不定(或:心不稳),背转身来自思忖。汉王荥阳遭围困,好似(或:犹如)孔子困至在蔡、陈。韩信领兵燕、赵境,无有能将退楚兵。师爷悬图(或:师爷挂图)【转西皮快板】表忠论,逄丑父救主标美名。与主同貌是纪信,要学先贤贯古今。走向前来把话论,纪信替主无二心。

纪信 纪信情愿替主赴难。

张良
将军替主赴难,可解荥阳之危。但霸王性如烈火,此去存亡难定,犹恐将军难舍一死。

纪信 师爷说哪里话来?慢说替主之难,就是赴汤蹈火,万死不辞。

张良 将军可是实言?

纪信 焉有二意?!

张良 如此将军请上,受我一拜。

纪信 岂敢!

张良
【西皮摇板】忠心贯日纪将军,古往今来少见闻。但愿解得荥阳困,青史名标万古存。

纪信 师爷------

纪信
【西皮快板】说什么青史标名姓,臣报君恩子奉亲。孝当竭力忠尽命,贪生怕死岂忠臣。

众
【西皮摇板】好个仁义纪将军,为主一片忠义心。舍身救主不惜命,愧煞我等不忠臣。

纪信
【西皮快板】多蒙列位美言赠,大事全仗众将军。纪信替主解围困,功成保主(或:公等保主)要小心。我与霸王逞舌论,要学子牙骂纣君。纵然将我碎尸粉,留得美名万古存。

张良
【西皮摇板】家贫\protect\hyperlink{fn105}{\textsuperscript{105}}孝子令人敬,

张良
【西皮快板】国难方显忠良臣。大事全仗纪将军,明日早朝依计行。众位将军且归寝,

张良 【西皮摇板】机密大事莫漏真。

曹参 【西皮摇板】深谢将军金石论,

周勃 【西皮摇板】盖世英雄纪将军。

随何 【西皮摇板】臣替君难心情忍,

纪信 【西皮摇板】心怀忠义别先生。含悲忍泪跨金镫,

张良 将军请转------

纪信 【西皮摇板】师爷有话快些云。

张良
【西皮摇板】将军替主心拿稳,莫作三心二意人。倘若临时呼不应,功不就来名不成。

纪信
【西皮摇板】先生不必细叮咛,纪信岂是等闲人。忠心一片岂失信,要想回心万不能,你但放宽心。

张良
【西皮摇板】大忠大义是纪信,片言感动忠良臣。非是张良无德行\protect\hyperlink{fn106}{\textsuperscript{106}},都只为创业兴邦的汉刘君,我不得不行。

\textbf{{[}第五场{]}}

刘邦 【西皮摇板】霸王日夜把城攻,

刘邦 【西皮快板】四面围困不透风。孤王心中担惊恐,

刘邦 【西皮摇板】无有良策破重瞳。

张良 【西皮摇板】一片丹心是忠信,

陈平 【西皮摇板】点破英雄众群臣。

张良 【西皮摇板】纪信可称真梁栋,

陈平 【西皮摇板】急忙进帐奏主公。

张良、陈平 臣等见驾,大王千岁。

刘邦 平身。

张良、陈平 千千岁!

刘邦 赐坐。

张良、陈平 谢坐。

刘邦 可有忠义之臣,替孤赴难?

张良 今有纪信与大王容貌相似,情愿替主赴难。

刘邦 哦!如此宣纪信将军进帐。

张良 纪信将军进帐!

纪信 (内)来也!

纪信
【西皮快板】十年窗前习孔孟,几载又学箭和弓。不见封侯成何用,功劳出在画图中。

纪信 臣,纪信见驾,大王千岁!

刘邦 平身。

纪信 千千岁!

刘邦 赐坐。

纪信 谢坐。

刘邦 二位先生,纪将军果然与孤王面貌相似么?

张良、陈平 与大王,相似无二。

刘邦 纪将军,方才二位先生言道,将军愿替孤王赴难,可是真情?

纪信 愿解荥阳之危。

刘邦
多蒙将军美意,不避刀锋之苦,此乃寡人福浅,累及卿等;只是将军如此忠义,教寡人怎能割舍?此事断然不可!

纪信 臣启大王:臣替君难,理所当然,何言不可?

刘邦
寡人江山未定,众卿徒劳,未享一日之荣,反而连累将军,这样损人利己之事,孤心不忍也。

纪信
哎呀大王啊!今日事在危急,城池难保;倘若荥阳一破,玉石俱焚。那时臣死轻如鸿毛,今日臣死何惜,他日留得美名重如泰山。不要臣替君难,臣便拔剑自刎君前,以报主上之恩也。

纪信
【西皮原板】大王起义在沛丰,扫灭嬴秦定关中。招贤纳士恩义重,多少英雄反重瞳。唯愿我主成一统,岂知今日遭困中。舍身救主理当奉,打开\protect\hyperlink{fn107}{\textsuperscript{107}}金锁走蛟龙。纪信一死成何用,保全大王数载功。

刘邦 将军!

刘邦
【西皮二六】孤王闻言心酸痛,怎舍得将军去尽忠。众卿创业各保重(或:功劳重),多亏文武众英雄。披霜戴雪真骁勇,血战疆场马蹄红。江山还未归一统,将士何曾受荣封。宁可城破孤命送,怎舍得将军遭剑锋。

纪信
【西皮摇板】大王再三不依从,回避(或:退避)何能表忠诚。投王麾下蒙恩宠(或:蒙恩重),

纪信 罢!

纪信 【西皮摇板】不如拔剑自尽忠。

刘邦 且慢。

刘邦
【西皮摇板】将军不必拔剑锋,心如铁石(或:心如石铁)一般同。\protect\hyperlink{fn108}{\textsuperscript{108}}

刘邦 先生,虽生死事急,孤实不忍;先生还是另想别策才是。

张良 纪将军执意效忠,大王,唉,依允了罢。

刘邦 先生有何妙计,你我君臣哪里逃走?

张良
待臣修书一封,命随何下到楚营,约定今晚大开东门纳降。主公将衣帽脱下,与纪将军穿换;再选美女数百名,随行车后。那霸王一见降书,必定深信;众楚兵观见美女,必定争攘。我君臣趁此喧哗之中,暗暗开了西门逃走,岂不是好?

刘邦 如此依计而行,就命先生修书。

张良 领旨。

张良 (念)兵马重围困,唤醒忠良臣。

刘邦 先生与孤传旨,吩咐文武,准备马摘鸾铃,美女梳妆伺候。

陈平 领旨。

陈平 (念)巧设良谋计,诈出荥阳城。

纪信 主公快将衣帽脱下,与臣穿戴,免得临时忙迫。

刘邦 唉,这是孤连累你、你\ldots{}\ldots{}你了!呃\ldots{}\ldots{}(哭介)

刘邦 卿家!

纪信 大王!

刘邦 将军!

纪信 我主!

刘邦 哎呀,卿家呀!

纪信 大王啊!

刘邦 【西皮小导板】楚汉年年大交锋(或:大交兵),

刘邦
\textless{}\textbf{三叫头}\textgreater{}将军,卿家,唉,将军!呃\ldots{}\ldots{}(哭介)

纪信 大王啊!呃\ldots{}\ldots{}(哭介)

刘邦
【西皮摇板】不想今日【转西皮二六】遭困中。丹心一片真梁栋,臣替君死第一功。江山若得归一统,在忠臣阁内画真容。

纪信
【西皮快二六】嬴秦无道社稷崩,楚汉分兵定关中。大王本是真命主,天降真龙下九重。又生我纪信容貌共,五行八字各不同。忍悲含泪(或:眼含珠泪)谢恩宠,

纪信 【西皮摇板】恕为臣假冒王号扮真龙。

刘邦
【西皮快板】见卿家哭得心酸痛,有辈古人听从容:昔日重耳走西东,文武十人患难从。介子推\protect\hyperlink{fn109}{\textsuperscript{109}}割股把殷勤奉,黄河渡口保真龙。重耳回国归一统(或:成一统),却忘了(或:独忘了)子推他的割股的功。母子隐居不受封,绵山一旦被火焚。卿比子推功劳重,寡人非比(那)晋文公。问卿家你可有

刘邦 \textless{}\textbf{哭头}\textgreater{}高堂母,

纪信 \textless{}\textbf{哭头}\textgreater{}老娘亲呐,

刘邦 \textless{}\textbf{哭头}\textgreater{}孤的老伯母啊,

刘邦、纪信 \textless{}\textbf{哭头}\textgreater{}啊,

纪信 \textless{}\textbf{哭头}\textgreater{}老娘亲呐,

纪信 【西皮原板】家有年迈老慈容。臣受君恩难敬奉,未尽孝来先尽忠。

刘邦
【西皮二六】将军替孤把忠尽,可谓\protect\hyperlink{fn110}{\textsuperscript{110}}人间第一功。卿母即是刘邦母,将伯母送至在(或:请至在)养老宫。生养死葬孤侍奉,金井玉葬送至在山中。问卿家你可有\textless{}\textbf{哭头}\textgreater{}妻和子,

纪信 \textless{}\textbf{哭头}\textgreater{}韩氏妻,

刘邦 \textless{}\textbf{哭头}\textgreater{}孤的皇嫂,

纪信 \textless{}\textbf{哭头}\textgreater{}小娇儿啊,

刘邦 \textless{}\textbf{哭头}\textgreater{}皇侄啊(或:孤皇儿啊),

刘邦、纪信 \textless{}\textbf{哭头}\textgreater{}啊,

纪信 \textless{}\textbf{哭头}\textgreater{}我的妻、儿啊,

纪信
【西皮原板】家有寒妻(或:家有贤妻)受苦穷。老母年高她侍奉,臣子还在襁褓中。

刘邦
【西皮快板】卿家替孤把忠尽,封妻荫子代代荣。卿妻就是(或:卿妻即是)刘邦嫂,你子我子一般同。太平年间归一统,举家满门受荣封。孤王若把良心昧,国破家亡不善终(或:无善终),天理不容。

纪信
【西皮二六】纪信闻言心酸痛,叩谢君王加荣封。寒门若得君王宠,死在九泉也欢荣。远望家乡珠泪滚(或:珠泪恸),

纪信 \textless{}\textbf{哭头}\textgreater{}我的娘啊,

纪信
【西皮快板】难舍高堂老慈容。为儿不孝少侍奉,只为君王遭困中。儿替君难(或:儿替君死)留名重,要学丑父立奇功。妻子悬望我空\textless{}\textbf{哭头}\textgreater{}梦,我的妻呀,

纪信
【西皮快板】难舍娇儿小孩童。父不能教儿习孔孟,父不能教儿马和弓。但愿儿成人把君奉,不枉后代受荣封。回头来奏一本,早作准备出牢笼。

陈平 参见主公!

刘邦 随何去献降书一事,怎么样了?

随何 霸王见了降书,十分欢喜。特来交旨。

刘邦 既然大事已成,准备女子可曾停当?

陈平
俱已停当。命周勃、滕公黄昏时候将东门大开,先将美女送出城去,纪将军随后出城;我君臣从西门逃走便了。

刘邦 如此改扮起来!

刘邦
【西皮摇板】君臣定下计牢笼,哭得天愁地也崩。打开玉笼飞彩凤,斩断金锁走蛟龙。

刘邦 \textless{}\textbf{三叫头}\textgreater{}将军,卿家,唉,将军呐!

刘邦 罢!

纪信 摆驾楚营去者!

\textbf{{[}第六场{]}}

\textbf{刘邦 且住。幸喜你我君臣逃出虎口,同往燕赵去者。}

{[}第七场{]}

\textbf{项羽 【西皮摇板】孤王兴兵威风大,}

\textbf{项羽 【西皮快板】杀得鬼哭并神哀。刘邦小儿无可奈,}

\textbf{项羽 【西皮摇板】快写降表称孤怀。}

\textbf{项羽 困住小荥阳,捉拿汉刘邦。}

\textbf{钟离眛、季布
启主公:今有刘邦亲来投降,又带来无数美女。臣等俱已拿到。请主公发落。}

\textbf{项羽 将美女带上。}

\textbf{钟离眛、季布 下面听者,主公有旨:美女押上。}

\textbf{众女 叩见大王。}

\textbf{项羽 抬起头来。}

\textbf{众女 有罪不敢抬头。}

\textbf{项羽 恕你无罪。}

\textbf{众女 谢大王!}

\textbf{项羽 三军听令!}

\textbf{众 啊!}

\textbf{项羽 将美女分派各营。}

\textbf{众 得令!}

\textbf{项羽 沛君何在?}

\textbf{众 现在帐外。}

\textbf{项羽 吩咐有请。}

\textbf{众 有请千岁。}

\textbf{纪信
【西皮快板】荥阳替主代了命,假扮汉王诓楚君。含悲忍泪下车轮,}

\textbf{纪信
【西皮快板】豪杰心下暗思忖:此番进帐把话论,霸王必定问真情。我若表出真名姓,霸王一定问斩刑。拚着一死留名姓(或:幽冥进),臣替君来(或:臣替君难)也甘心。纵死黄泉无怨恨,堂上哭坏老娘亲。妻儿悬望无音信,怀中舍了小娇生。指望阖家同欢庆,谁知(或:哪知)今日丧楚营。大胆且把宝帐进,}

\textbf{纪信
【西皮快板】上面坐的是楚君。两旁儿郎威风凛,刀枪剑戟杀气腾。大摇大摆且站定,各人怀揣一片心。}

\textbf{项羽
【西皮快板】灯光之下来观定,面前站定刘沛君。从先劝你来归顺,倚仗韩信会用兵。城楼你把巧言论,左有张良右陈平。好汉与孤来站定}\protect\hyperlink{fn111}{\textsuperscript{111}}\textbf{,孤家与你定输赢。草苗烧灰不必论,贪生怕死岂为人。韩信就该来救应,张良为何不用兵。劝主归降自惜命,方食爵禄保朝廷}\protect\hyperlink{fn112}{\textsuperscript{112}}\textbf{。看来孤王有福分,}

\textbf{项羽 啊,呵呵哈哈哈\ldots{}\ldots{}(笑介)}

\textbf{项羽
【西皮快板】孤王本是有道君。宽洪量大人人敬,些许小仇哪在心。从前之事不究问}\protect\hyperlink{fn113}{\textsuperscript{113}}\textbf{,还念当年结拜情。今日既然来归顺,前后之事我问清。见孤为何不跪定,佯瞅不睬为何情。开言便把刘邦问,难道还有两般心。}

\textbf{项羽 唗!胆大刘邦,见了孤王为何不跪?}

\textbf{纪信
住了,你乃一君,我乃一王,岂肯跪你?既知孤王到此,就该下位迎接才是。}

\textbf{项羽 啊?!听他声音不像刘邦,左右,掌灯待孤看来。}

\textbf{项羽 嘿嘿,中了他人之计也!}

\textbf{项羽
【西皮快板】看罢相貌怒气生,又中他人巧计行。开言便把儿郎问:假扮刘邦哄寡人。好汉说出真名姓,你是他驾下什么人。}

\textbf{纪信
【西皮快板】霸王解开其中情,浑身上下冷汗淋。本当上前通名姓,必定斩首在辕门。本当不说真名姓,岂肯放我转回程。拚着一死把忠尽,叫声霸王听分明:我的名字叫纪信,臣替君死标芳名(或:替主一死标美名)。}

\textbf{项羽
【西皮摇板】听说来了小纪信,不由孤王动无名。左右将他推出斩,}

\textbf{纪信 呵呵哈哈哈\ldots{}\ldots{}(笑介)}

\textbf{项羽 招回来!}

\textbf{项羽 【西皮摇板】纪信发笑为何情。}

\textbf{项羽 纪信为何发笑?}

\textbf{纪信 霸王,我来问你:这君有难?}

\textbf{项羽 臣当替。}

\textbf{纪信
着哇!君有难臣当替。只因你兵困荥阳,攻打甚急,因此替主赴难,以解君忧也。}

\textbf{项羽 难道你不怕死?}

\textbf{纪信
忠臣不怕死,怕死不忠臣。昔日齐晋交兵,齐顷公领兵伐晋,谁想敌兵强盛,齐兵大败。杀得尸横遍野,血流成河,只剩顷公一人坐在车中。晋兵紧紧追赶,他驾前有一臣子名叫逄丑父,替主坐于车内,顷公才得脱难。彼时将他推出斩首,那丑父言道:今日将我斩首,可惜只恐后来无人再替主难。晋公闻言将丑父放回,后来反加封赏。今吾主有难,臣当替主,你将我斩首,有何难哉?大王乃仁义之君,细心思忖!}

\textbf{项羽 呀!}

\textbf{项羽
【西皮快板】听他言语一番讲,可算得大义一忠良。本当将他来斩首,孤家岂是铁心肠。前朝君王遭魔障,留与后世标名芳。开言便叫二员将,快劝纪信把孤降。}

\textbf{钟离眛、季布 【西皮摇板】黄罗宝帐领将令,开言叫声纪将军。}

\textbf{钟离眛、季布 纪将军,归顺吾主,何愁封王爵位。}

\textbf{纪信 俺纪信生为汉臣,死为汉鬼。你劝我归顺,痴心妄想!}

\textbf{纪信 【西皮摇板】纪信忠心把主替,岂肯背主把降归。}

\textbf{项羽 纪信,孤家有意放你回去。}

\textbf{纪信 恐大王三心二意。}

\textbf{项羽 孤家焉有二意!}

\textbf{纪信 如此谢大王!}

\textbf{项羽 纪信!}

\textbf{项羽
【西皮快板】孤家今日将你放,后来且莫把恩忘。回营与我多拜上,拜上沛君小刘邦。趁早降书来呈上,免得孤家动刀枪。若有半字不停当,管教你君臣马前亡。}

\textbf{项羽 回去罢!}

\textbf{纪信 多谢大王!}

\textbf{纪信 【西皮摇板】谢过大王抽身往,}

\textbf{纪信
【西皮快板】隆恩似海福寿康。上钩金鳌脱了网,犹如会过五阎王。替主死难未斩丧,}

\textbf{纪信 【西皮摇板】也落得美名后世扬。}

\textbf{范增 【西皮摇板】听说吾主将他放,忙上宝帐问端详。}

\textbf{范增 范增见驾,愿主公千岁!}

\textbf{项羽 亚父平身。}

\textbf{范增 千千岁!}

\textbf{项羽 亚父进帐何事议论?}

\textbf{范增
老臣闻知,吾主已擒刘邦,问悉情由,乃是纪信替死。这样匹夫之辈,将他放回,犹恐}\protect\hyperlink{fn114}{\textsuperscript{114}}\textbf{生出别事。主公快快将他赶回斩首,免却后患。}

\textbf{项羽 嗯,亚父之言甚是。}

\textbf{项羽 钟、季二将!}

\textbf{钟离眛、季布 在!}

\textbf{项羽 快将纪信赶回!}

\textbf{钟离眛、季布 得令!}

\textbf{项羽 【西皮摇板】亚父暂且宽心放,要把纪信碎尸亡。}

\textbf{范增 【西皮摇板】辞别千岁下宝帐,管叫纪信一命亡。}

\textbf{纪信 【西皮摇板】适才放我得活命,赶我回来必有因(或:定有因)。}

\textbf{项羽
【西皮摇板】非是孤不肯饶你命,放虎归山反伤人。倒不如将你碎尸粉,作一个斩草除了根。}

\textbf{纪信 住口!}

\textbf{纪信
【西皮快板】纪信闻言笑盈盈,大骂匹夫听详情:你二人昔年曾盟定,先到咸阳定为君。吾主宽宏行仁政,各路英雄(或:诸侯)俱归心。吾主咸阳江山定,谁像}\protect\hyperlink{fn115}{\textsuperscript{115}}\textbf{你是杀戮星。到一郡杀一郡,到一州来一州平。普天之下都怨恨,食你之肉(或:食尔之肉)方称心。将我主赶出咸阳郡,今日又困荥阳城。韩信带兵燕、赵境,无有能将领雄兵。纪信替主解围困,臣替君难古常情。蒙你释放将情领,赶我回来问典刑。你的言语不定准(或:无定准),易翻易覆无信人。纪信一死无怨恨,只恐你呀千年万载落骂名。}

\textbf{项羽 住口!}

\textbf{项羽 【西皮摇板】匹夫出言令人恨,要想活命万不能。}

\textbf{项羽 纪信,孤家本当将你斩首,有言在先,也罢!如今与你全尸。}

\textbf{项羽 钟、季二将,将纪信用火焚化!}

\textbf{钟离眛、季布 啊!}

\textbf{钟离眛、季布 \ldots{}\ldots{}焚化!}

\textbf{项羽 搭了下去。}

\textbf{项羽 众将官!}

\textbf{众 有!}

\textbf{项羽 随孤追赶刘邦去者!}

\textbf{众 啊!}

\newpage
\hypertarget{ux76d7ux5b97ux5377}{%
\subsection{盗宗卷}\label{ux76d7ux5b97ux5377}}

{[}第一场{]}

(西皮\textless{}\textbf{小开门}\textgreater{},四太监、大太监引吕后上)

吕后
(念{[}引子{]})身在深宫院,重整汉室锦江山。(大座,念)未央宫中翡翠环,珍珠玛瑙堆成山。上殿独受君王宠,方显女皇将魁元。哀家吕后。昨日飞报禀告,有一游方道人,来到长安,盗取宗卷。不免将宗卷用火焚化,以消后患。内侍(侍应),宣张苍上殿。

内侍 国太有旨,张苍上殿呐!

张苍 (内白)领旨。

(小锣上,小边台口)

张苍 (念)忽听国太宣,迈步上金銮。(进门,参拜)臣张苍见驾,国太千岁。

吕后 平身。

张苍 千千岁。(小边举笏站)

张苍 宣臣上殿有何(国)事议论?

吕后 卿家官居何职?

张苍 西台御史。

吕后 掌管何事{[}或:宗卷可是卿家掌管(看守){]}?

张苍 皇家宗卷{[}或:正是微臣掌管(看守){]}。(或:皇王宗卷。)

吕后 将宗卷呈上哀家一观。

张苍 领旨。(张苍上场门下)

吕后 内侍,张苍取卷到来看我眼色行事。

张苍
(捧卷上)手捧皇家卷,国太凤目观。(进门张苍交卷给内侍,侍呈吕后,张站小边)

(张苍 国太请看。)

吕后 卿家看守宗卷有功,赐御酒一斗。

(内侍递酒给张苍,张站小边台口)

(张苍 谢国太。)

张苍
且住,人言吕后之酒好饮难还,待我谢过神祗。(或:人言吕后之酒实实难饮,待我敬谢天地。)

吕后 想这太平年间要这宗卷何用?内侍,将宗卷用火焚化了。

(张苍将酒谢神时吕念,张还酒杯,太监烧卷,张站小边念``哎呀''时右手投袖,摇头惊介)

张苍
哎呀!(或:使不得!)(右袖盖头,左手撩袍,上场门反下,同时,陈平上场门外侧上,\textless{}\textbf{撞金钟}\textgreater{}到小边台口)

陈平 【西皮摇板】金銮殿上红光现,陈平焉能袖手观。(进门,参拜)

陈平 臣陈平见驾,国太千岁。

吕后 平身。

陈平 千千岁! (大边站举笏)

吕后 卿家上殿有何本奏?

陈平 适臣观见金銮殿上红光一起,特地上殿保全江山。

吕后 卿家可知哀家火焚宗卷之故?

陈平 国太敢是想吞\ldots{}\ldots{}(或:国太莫非要吞\ldots{}\ldots{})

吕后 卿家果有出将入相之才。

陈平 谢国太。

吕后 退班。

(陈平、吕后搭话紧快,吕众窝下,陈揖归中间,\textless{}\textbf{撞金钟}\textgreater{})

陈平
【西皮摇板】金钟三响王退殿,文武有怒不敢言。撩袍端带下金銮,(小圆场下殿归大边,张苍上场门上,站小边)只见张苍在眼前。宗卷本在孝廉殿,不该拿来献君前。

张苍 国太要看呐!

陈平
呀呸!(接唱)淮河十王人马到(或:淮河十王发人马),你一家大小难保全。(陈平下,\textless{}\textbf{撞金钟}\textgreater{})

张苍
【西皮摇板】陈平老儿礼不端,骂得我张苍不敢言。宗卷本在孝廉殿,谁知国太备火燃。淮河十王人马到,我一家大小难保全。(张苍下)

{[}第二场{]}

(\textless{}\textbf{小锣抽头}\textgreater{},田子春上)

田子春
【西皮摇板】道家模样改装扮,不分昼夜往长安。(念)下官田子春,奉了幼主之命,去至长安盗取皇王宗卷,就此前往。

田子春
(接唱)昔日楚汉两争强,韩信弃楚投汉邦。未央宫中把命丧,为国忠良无下场。(田子春下)

{[}第三场{]}

(\textless{}\textbf{六幺令}\textgreater{}陈平、家院、四青袍上。陈下轿,牌子合龙,陈换衣,青袍下,陈正小座,院大边站)

陈平
老夫陈平。今日早朝国太将宗卷焚化,莫非淮河有人前来盗卷,我不免八卦详查,来,香案伺候。(或:老夫陈平。适才国太在金銮殿上将宗卷焚化,不知是何缘故。不免在八卦之中查看吉凶便了。家院,香案摆下。)

(台前方设桌椅,\textless{}\textbf{小开门}\textgreater{},陈入座,摇盒,开看,用笔记,连三次,看)

陈平
十字口中吞,了字加一横。三人共一日,凑成田子春。(田子春\ldots{}\ldots{})嗯,昔日老王(或:昔日先王)驾下有一臣子名唤田子春,此人虽则年幼颇有胆识,(现在淮河。)莫非此人前来盗卷不成?(我自有道理,)香案撤去。

(\textless{}\textbf{小开门}\textgreater{}撤香案,陈入小座,院大边)

陈平
唤夜不收进见。(或:来,传夜不收\protect\hyperlink{fn116}{\textsuperscript{116}}进见。)

家院 夜不收进见。

(二夜不收上,一人提灯)

二夜不收 (上念)人平不语,水平不流。(进门跪)参见相爷,有何差遣?

陈平
命尔等去至十字街头,点起灯笼火把高声叫喊(或:高声喊叫):会犯夜的前来犯夜,不会犯夜的不要错犯了夜,若是错犯了夜,先带去见都御史陈爷,然后送往(或:然后送到)有司衙门审问。

二夜不收 启相爷,世间之上只有误犯夜,哪有掌灯叫人犯夜之理,求相爷改差。

陈平 嗯,相爷一言既出,驷马难追,还不下去。

(陈平下,家院拿锁链交夜不收)

家院 还不快去。

(家院下,夜不收出门)

夜不收甲 嘿,我说伙计,相爷是老糊涂啦,有这样拿犯夜的吗?

夜不收乙 没法子,上边差遣,概不由己,咱们就照办吧!

夜不收甲 走着。

(二夜不收圆场大边台口坐倒椅)

二夜不收
咱们就叫唤吧。我说来人听者:会犯夜的前来犯夜,不会犯夜的别误犯,要是误犯,先带去见都御史陈爷,然后送到有司衙门审问。犯夜的来呀!

(田子春小锣上)

田子春 【西皮摇板】行来不觉夜色晚,清清冷冷到长安。

(田子春站小边)

二夜不收
会犯夜的前来犯夜,不会犯夜的别误犯夜,误犯了夜,先带去见都御史陈爷,然后送到有司衙门审问。犯夜的来呀。

田子春
呜哙呀,世间之上只有误犯夜,哪有掌灯叫人犯夜之理?这,嗯,莫非是陈平老儿之计?

田子春 哼,我不免趁此机会,将计就计,也好会见那老儿。

二夜不收 犯夜的来呀!

田子春 二位。

二夜不收 干什么?

田子春 我是远方来的,一无亲友,二无宿处,来此犯夜。

二夜不收 好,跟我们去见都御史。

田子春 我来问你,都御史是哪个?

二夜不收 就是我们陈老相爷。

田子春 敢是陈平?

二夜不收 我打你的嘴。

田子春 慢来慢来,我是你家相爷外甥。

二夜不收 甭管他,带他交差。

(给田子春带链子,三人走圆场到小边)

二夜不收 请爷。

(家院下场门上,出门)

家院 何事?

二夜不收 我们拿住犯夜的啦!

家院 候着。有请相爷。

(陈平下场门上)

陈平 只为火焚宗卷事,老夫日夜费心机(或:昼夜费心机)。何事?

家院 拿住犯夜之人。

(陈平正座小座)

陈平 传。

(家院传唤夜不收)

家院 丞相呼唤。

夜不收甲 我先去,你看着人。(进门跪)

夜不收甲 启禀相爷,小的我拿住犯夜的啦。

陈平 好,来看赏。(或:来,看赏。)

(家院赏夜不收甲)

夜不收甲 谢相爷。

陈平 带犯夜人。

夜不收甲 是。(夜不收甲出门)

夜不收甲 嘿,捞了一份。

夜不收乙 我也来一份。(夜不收乙进门跪)

夜不收乙 启禀相爷,小的我也拿住犯夜的啦。

陈平 好,也看赏。(或:哦,你也拿住犯夜的了?嗯,也有赏。)

(家院赏夜不收乙)

夜不收乙 谢相爷。

陈平 带犯夜人。

(夜不收乙出门)

夜不收乙 你瞧,我也来了一份。

夜不收甲 我再来它一份。

夜不收乙 别蒙事去了,招人生气。

夜不收甲
相爷好脾气儿。(进门跪)启禀相爷,小的我还拿了一个犯夜的,他蒙事,说是相爷的外甥,我打了他一个嘴巴,还踹了他一脚。

陈平
哼,相爷的外甥(或:老夫的外男)也是尔等打得的么?来,将他二人的赏银都与我追了回来,还不下去。(或:来,将银子追了回来。轰了下去!)

(家院追回二人银子)

二夜不收 (对白)讨哇,要哇,打哇,闹哇,都搭进去了,还招事呐,走吧!(下)

陈平 带犯夜人。

家院 犯夜人进见。

(田子春进门,站小边,家院取下链子,家院下,陈平站大边,一望)

陈平
待我看看哪个大胆,竟敢冒称老夫的外甥。(或:待老夫观看,哪个大胆竟敢冒称老夫的外男。)

田子春 好醉呀好醉!

陈平
呵,我看此人,毫无酒意,为何声称好醉?这\ldots{}\ldots{}莫非他就是那田,待我冒叫他一声,那旁敢是田?(或:呜哙呀,我看他毫无酒意,为何自称好醉?莫非此人就是田子春不成?嗯,待我来冒叫一声。那旁敢是田\ldots{}\ldots{}?)

田子春 那旁敢是陈?

陈平 田大人。

田子春 陈相爷。

(陈平、田子春 啊,啊呵呵哈哈哈\ldots{}\ldots{}(笑介))

陈平 请坐。

田子春 有座。

(二人挖门,田子春、陈平坐,田大边,陈小边)

陈平 田大人,你为何自称老夫的外甥(或:田大人,为何自称老夫的外男)?

田子春 若不如此,焉能与相爷相见。

陈平
是是是,请问大人来到长安有何贵干?(或:哦,原来如此。大人不在淮河来到长安何干?)

田子春 奉了幼主之命前来盗取宗卷。

陈平 大人你来迟了。

田子春 何言来迟?

陈平 今日早朝(在)金殿之上(,被)国太用火焚化了。

(陈平 用火焚化了。)

田子春
不好了!\textless{}\textbf{撞金钟}\textgreater{}【西皮摇板】好似霹雳当头震,倒教子春无计行。叫声吾儿来相等,父子做鬼一路行。

陈平 大人为何这等着急?

田子春 相爷有所不知,幼主言道,若无宗卷就将我父子一同斩首。

陈平 大人就该回转淮河(或:就该速回淮河)搭救令郎(的)才是呀。

田子春 哦哦,我是不回去的了。

陈平 你不回去你在何处安身(或:你在何处落足)呐?

田子春 哼哼,我就住在你的府中。

陈平 国太知道(或:呃,国太闻知)那还了得。

田子春 不妨,我就说你请我来的。

陈平
哦哦,你是犯了我的夜。(或:呃,呃\ldots{}\ldots{}是你犯了我的夜啊。)

田子春
呀呸!世间之上只有误犯夜,哪有点起灯笼火把叫人犯夜之理?陈平呐陈平,限你三日有了宗卷便罢,若无宗卷,淮河兴兵叫你全家诛戮。

陈平
大人不必如此,请至书房憩息。(或:呃,慢来慢来,请至书房。呃,请至书房歇息。)

田子春 你不请我我不来(或: 天堂有路我不走)。

陈平 自投罗网怨谁来(或: 地狱无门(你)闯进来)。

田子春 根深哪怕风摇摆。

陈平 准备棺木将你埋。

田子春 你埋哪个?

陈平 不是你还有哪个。

田子春 陈平呐陈平,三日之后若无宗卷你要打点了,你要仔细了!

陈平 哦哦,(大人不必动怒,)请至书房,请至书房。

田子春 哼。

(田子春下,家院上)

陈平
哎呀呀,哪里是他犯了我的夜,倒是我犯了他的夜(或:分明是我犯了他的夜了),哎呀这这这,嗨,我临死(或:我至死)也要拉上一个垫背的,来,拿我名帖去请张苍张大人夤夜过府饮宴,快去(或:不得有误)。

(家院 遵命。)

(家院接帖下)

陈平
嗨,若能留得宗卷在,也免老夫挂心怀。嗨!(或:唉,但愿留得宗卷在,免得老夫挂心怀。唉!)

(陈平下)

{[}第四场{]}

(\textless{}\textbf{小锣抽头}\textgreater{}家院掌灯笼引张苍上,站台中间)

张苍 【西皮摇板】正在府中(或:正在衙中)愁闷坏,陈平有帖请我来。

张苍
(念)下官张苍,陈平有柬帖相邀,请我夤夜过府饮宴,却是为何?我二人虽是一殿为臣,却素无来往呵。哦是了,想是今日早朝,言语之间冲撞于我,请我过府与我赔礼也是有的,哎呀呀,哈哈哈老相爷呀,你我俱是炎汉忠良,哪个还怪你不成,你这是何苦哇?(或:下官张苍,陈平有柬帖相邀,不知为了何事。我二人素无来往啊。哦\ldots{}\ldots{}是了,想是今日早朝,金殿之上,冲撞于我,夤夜请我过府饮宴,与我赔上一个礼儿,呃,也是有之,哎呀呀老相爷呀,你我俱是炎汉忠良,一殿为臣,冲撞几句,又有何妨?你这是何苦哇。)

张苍 (接唱) 【摇板】家院掌灯把路带,去至相府饮开怀(或:
家院与爷把路带,见了相爷说开怀)。

(家院引张苍下)

{[}第五场{]}

(陈平上拿书)

陈平
\textless{}\textbf{小锣抽头}\textgreater{}【西皮快板】淮河来了田子春,倒叫老夫心内惊(或:盗取皇王宗卷文)。将身且坐二堂等,等候张苍到来临。

(陈坐桌大边座,看书,家院引张苍上,小边,张苍唱中小圆场)

张苍 【西皮快板】吕后做事太欺情,不该宗卷备火焚(或:
可恨吕后心太狠,火焚宗卷谋乾坤)。(或:宗卷不该用火焚。)家院掌灯(或:家院向前)把路引,不觉来到相府门。

张苍
(念)你在府门稍待片刻,待我进府略饮几杯就要回去。(或:来此相府,你就在马待石\protect\hyperlink{fn117}{\textsuperscript{117}}前等候,你家老爷去至里面略饮几巡,就要出来。)

家院 您可悠着点儿。

张苍 (我)晓得(呀)。【西皮摇板】张苍撩衣(或:撩袍)进府门,

(家院下,张苍进门,小圆场唱,陈放下书在张唱中饮酒)

张苍
【西皮快板】静静悄悄无一人。来至在(或:站立在)二堂来观定,陈平一人饮杯巡。

张苍
(念)陈相爷请我过府饮宴,怎么他自斟自饮起来了,啊啊是了,想是他等我不及,酒兴发作,先行自饮几杯,等我到来再大排筵宴,也是有的。我不免痰嗽一声,他必然下位迎接于我,嗯喷,呵来了。(或:陈平老儿怎么一人自斟自饮起来了?哦,想是等我不及,先饮几杯,待我痰嗽一声,惊动于他,他必然迎接于我,嗯,嗯喷。呵呵,来了来了。)

(陈平站出门,唱中大边向外跪拜,张苍随拜跪)

陈平
【西皮快板】陈平撩衣来下拜(或:对着苍天把礼拜),过往神祗(或:过往的神灵)听开怀。我若有意来降吕(或:我若是有意降了吕),天地(或:老天)与我降祸灾。叩罢头来深深拜,我看张苍怎起来。

(陈平末句前站起来,用手横指张苍,张跪坐介,陈进门坐下饮酒)

(陈平 表罢忠心,再饮几杯。)

张苍
嘿嘿,(张苍立)我道他下位迎接于我,原来他表起他的忠心来了。陈平呐陈平,你是炎汉忠良我张苍就不是炎汉忠良么?你表得我也表得,要表我们大家表上一表。(或:嘿嘿,(张苍立)我道他下位迎接于我,他倒表起他的忠心来了。你是炎汉忠良,难道我张苍就不是炎汉忠良么?你表得我也表得,你表我也表,要表我们大家表上一表。)

张苍 【西皮快板】
张苍撩衣跪尘埃(或:对着苍天忙下拜),过往的神灵听开怀。我若背汉(或:我若是有意)降了吕,老天爷与我降祸灾。叩罢头,深深拜(或:抽身拜),问声相爷可安泰。

张苍 参见相爷。

(张苍唱中小边向外跪拜,唱中站,进门拜,陈平站大边)

陈平
(啊?哦,哦,)原来是张大人,哦,夤夜来到鄙衙,敢是要查看老夫的弊病不成(或:敢是查看老夫的弊病来了么)?

张苍
慢来慢来,老相爷柬帖相邀,请我过府饮宴,何言弊病二字?(或:相爷说哪里话来,下官乃是奉相爷之命,夤夜前来饮宴的呀。)

陈平 (哦,呃,怎么还有此事么?)哎呀呀,(不是张大人提起,)我倒忘怀了。

张苍 你看你看,有这样请客的么?

陈平
备席不及,来来来这有残酒拿来去饮。(或:备酒不及,我这有残酒,拿去饮来。)

(张苍 谢相爷。)

(陈平 呀呸!)

(陈平拿酒,泼张苍脸上,张双袖擦脸\textless{}\textbf{叫头}\textgreater{})

张苍
老相爷,这酒吃与不吃,不关紧要,为何泼在下官脸上?(或:唉呀相爷呀,一杯水酒吃与不吃,不值紧要,为何将酒泼在下官的脸上?)

陈平 我这酒,你吃它不得。

张苍 哪个吃得?

陈平 炎汉忠良方能吃得。

张苍 难道说我张苍就不是炎汉忠良么?

陈平 (我来问你,)你官居何职?

张苍 西台御史。

陈平 掌管何事?(或:掌管何物?)

张苍 皇家宗卷。(或:皇王宗卷。)

陈平 拿来。

张苍 什么?

陈平 宗卷呐。

张苍
哎呀相爷呀,今日早朝,宗卷(或:宗卷今日早朝)被国太用火焚化,(呃,)还是老相爷你保的本呢。

陈平
住了,张苍呐张苍,限你三天有了宗卷便罢,不然就将你全家诛戮。(或:呀呸!
限你三天有了宗卷便罢,不然要将你全家诛戮。)

张苍 哎呀!【西皮小导板】听一言吓得我(或:听一言不由我)三魂不在。

陈平 唗,此地什么所在?

张苍 乃是堂堂相府。

陈平
既知堂堂相府,为何这等喧嚣(或:这等高声喊叫)?来人(或:家院,)与我轰,与我赶,轰赶轰,轰了出去。

(陈平一、二转身单投袖,一、二、三轰,双投袖轰,下。同时张苍一、二转身揖,一、二、三退揖,双投袖,念``哎呀'',出门。家院暗上)

张苍 【西皮散板】黑洞洞摸出了(这)相府门(来)。

张苍 家院。(家院\ldots{}\ldots{})

家院 散席啦?

张苍 (好奴才,)走,回去。

(\textless{}\textbf{扫头}\textgreater{}张推家院下,圆场进门站中间,\textless{}\textbf{叫头}\textgreater{})

张苍
哎呀且住!陈平老儿哪里是请我过府饮宴,限我三日有了宗卷便罢,若无宗卷就要将我全家诛戮。(或:且住!陈平老儿请我过府饮宴,哪里是饮宴,限我三天有了宗卷便罢,不然要将我的全家诛、诛\ldots{}\ldots{}戮。)也罢,我不免拜谢先王爵禄之恩,寻个自尽了罢。

张苍 【西皮散板】张苍撩衣跪埃尘,拜谢先王爵禄恩。一把钢刀拿在手,

(跪拜起来在桌上拿单刀,台口看刀,转身推刀出手落小边台口,在大边里边左袖盖头、右手指刀,抬右腿、左腿单腿立)

张苍 (接唱)(这)白亮亮钢刀吓煞人!

(出门向下场门)

张苍
夫人,为丈夫在此自刎(或:下官在此自尽),你要劝一劝呐,(你要)拉一拉呐,唉!(边过小边边接唱)我这里十叫九不应呐。

(到小边向上场门)

张苍
秀玉儿,为父的在此寻死,你要劝一劝呐,你要拉一拉呐。唉!(回身接唱)秀玉一边不作声。(归中接唱)千思万想无计定,祸到临头难逃生。

张苍 罢!(接唱)咬定牙关项上刎。

(拾刀,台中间刎介,夫人上场门上,夺刀,张大边,夫人小边站)

夫人 【西皮散板】老爷自刎为何情?

张苍
哎呀夫人呐,那陈平老儿哪里是请我过府饮宴,他限我有了宗卷便罢,若无宗卷就将我全家诛戮哇\ldots{}\ldots{}(或:哎呀夫人呐,陈平老儿限我三天有了宗卷便罢,不然要将我全家,唉,诛,诛戮哇\ldots{}\ldots{}(哭介))

夫人 不好了!

夫人 【西皮散板】心中只把陈平恨,因何谋害我满门?

(张苍大边、夫人小边,桌旁八字坐,张秀玉下场门反上)

秀玉 【西皮摇板】正在书房看经纶,忽听前堂放悲声。(进门)

秀玉 参见爹爹。

(秀玉拜,站大边)

张苍 儿是(或:哦,你是)秀玉?

秀玉 正是。

张苍 儿来了?

秀玉 来了。

张苍 儿来得好哇(或:你来得好哇)\ldots{}\ldots{}

秀玉 爹爹为何如此?

张苍 小小年纪懂得什么(或:晓得什么),(快快)攻书去罢。

秀玉 有何大事?孩儿愿为爹爹分忧解愁。

夫人 是呀,老爷说将出来,孩儿与你分忧解愁。

张苍
呵,分忧解愁,哎呀儿呀,那陈平老儿哪里是请为父过府饮宴,他限我三日有皇家宗卷便罢,若无宗卷就将我全家诛戮哇\ldots{}\ldots{}(或:讲得的么?哦,讲得的,哎呀儿啊,陈平老儿限我三日有了宗卷便罢,不然要将我的全家,诛,诛戮哇\ldots{}\ldots{}(哭介))

秀玉 啊,原来是一桩小事。

张苍 啊,小事,儿(啊,你)近前来,好奴才!

(秀玉回身逃下,张苍拿刀追,夫人拉回坐下)

张苍 什么哦小事!(或:哼,小事?有这样的小事?!哼!)

(秀玉下场门反上)

秀玉 【西皮摇板】汉室宗卷手捧定,上前交与老爹尊。

张苍
啊夫人,眼看大祸临门,世间之上还有这样的小事么?(或:夫人,你看,这样的大事,说什么是小事,有这样的小事吗?!)

(张苍大摊手,秀玉上前将卷放张左手上,张看)

张苍
哈\ldots{}\ldots{},夫人,这个奴才被我吓糊涂了,拿本古书前来搪塞老夫来了。(或:哈\ldots{}\ldots{}哎呀,这个奴才是被我吓糊涂了哇,拿本古书前来蒙哄老夫来了。)

秀玉 宗卷也罢,古书也罢,要看个明白。

夫人 是呀,要看个明白。

张苍 哦,(怎么,是与不是)要看个明白,(唉,)如此我就看、看(、看)呐!

(张苍立,站中间)

张苍
【西皮快板】这奴才(或:小奴才)被我吓懵懂,拿本古书当卷宗。是与不是从头看,有劳夫人(或:夫人与我)掌灯红。

(拿卷到台口,夫人掌灯小边,秀玉大边,三人站,张唱、夫人夹白)

张苍
\textless{}\textbf{小锣抽头}\textgreater{}【西皮导板】初起义来在沛丰。

张苍 啊,这是宗卷呐,啊夫人,我在此作甚呐?

(夫人夹白: 在此看卷呐。)

张苍 只恐不是看卷罢。

(夫人夹白: 是做什么?)

张苍 是做梦罢?

(夫人夹白: 你看漫天星斗,怎说是做梦?)

张苍
(哦,是看卷?)不是做梦。夫人你将灯掌高些,啊,太高了要矮一些(或:忒高了),啊,又太矮了(或:又忒矮了),掌灯(是)要齐眉的呀,啊,烧了眉毛了。儿呀,将灯接了过来。啊夫人,你看你我的儿子掌灯比你强得多呀。

(夫人、秀玉掌灯作势)

张苍
【西皮快板】剑斩白蛇路途中。第一排汉高祖,第二排吕正宫。三宫六院有牌供,关东十王一派宗。宗卷看到第七册,幼主本是赵娘生。(或:三宫六院承恩众,各个立者俱有封。关东十王有牌位,幼主本是赵妃生。)\textless{}\textbf{三锣}\textgreater{}

(张苍、夫人归座,\textless{}\textbf{大锣原场}\textgreater{})

张苍
儿呀,宗卷已被国太焚化,这是哪里来的?(或:儿啊,这部宗卷是哪里来的?)

秀玉
爹爹在癸未年间染病在床,命孩儿代守宗卷,孩儿看到第七册第七篇,见襄宫赵娘娘死得可惨,犹恐日后有变,为此将宗卷抄下一部以防后患。

张苍 这有一事不合律。(或:只是有一事不应典呐。)

秀玉 哪一事?

张苍 这皇家玉玺是怎样来的?(或:这皇王玉玺是哪里来的?)

秀玉
那是孩儿用黄蜡雕成玉玺,真的上面原有一颗,假的上面也打上它一颗,与它真假难辨。

张苍 国太用火焚的呢?

秀玉 乃是假的。

张苍 这呢?

秀玉 这是历代历代的老宗卷。

张苍 (如此说来)这是历代历代的老宗卷?

秀玉 老谱头。

张苍
老谱头,哈哈哈,这才是我的好儿子呀。夫人,像这样的儿子你(为何不)与我多养上几个哇!

夫人 取笑了。

张苍
夫人,你们准备逃回原郡,如今有了宗卷,我怕他何来?我要与那陈平老儿大闹一场呐。(或:是啊,有了宗卷我怕他何来?我要去至相府,与那陈平老儿大闹一场啊。)

张苍
【西皮散板】辞别夫人出府门,(白)(夫人,呃,宗、宗卷呢?)宗卷呢?(笑介)

(张苍出门,家院暗上)

家院
出门找。(找介,秀玉看张手中拿卷,张笑,家院领张苍左手托卷、右手袖盖头,\textless{}\textbf{扫头}\textgreater{}下)

夫人 (接唱)一见老爷出府门,稳坐内衙等信音。

(夫人、秀玉下)

{[}第六场{]}

张苍 (内唱)【西皮导板】家院与爷(或:家院掌灯)把路引,

(家院引张苍上,院倒,张从家院身上绊倒,宗卷扔到大边台口,张、院起身,找卷)

张苍 (呃呃呃,)宗卷呢?

家院 找。

(家院提灯领张苍,走太极图,院在大边照见宗卷,张在小边里边转身蹉步拾卷,掸宗卷上土三下,院、张圆场,院下,张进门介)

张苍 【西皮散板】有了宗卷怕何人。小首不坐大首坐,(坐台口大边椅)

张苍 (接唱)问我一言(我)答一声。

(陈平上)

陈平
那张苍(老儿)回得衙去,哪里去寻宗卷呐。他不是仰药伏刀,定是投井悬梁,他是不能来的了。啊,他怎么来了(或:他呀,他不能来了,他,他\ldots{}\ldots{}他不能\ldots{}\ldots{}呃,他倒先来了。),啊,张苍你来了?

张苍 我早就来了。(或:我为何不来呀?)

陈平 (诶,)张苍,你怎么连品级台位都不讲了(或:都不顾了)?

张苍
(嗯,)太平年间有个(或:太平年间讲的是)品级台位,这离乱年间,还讲什么品级台位,这个座位你张大人坐坐何妨啊?

(张苍盘一腿坐)

陈平 老夫也不怪罪于你(或:呵呵呵,好好好,我也不计较于你),拿来。

张苍 什么?

陈平 宗卷呐。

张苍 你要几十部?

陈平 (诶,)一部可也就够了。

张苍 我当是要几十部,(嗯,)拿去(看来)!(递卷)

(陈平拿卷)

陈平 (笑介)这个老儿到底是(被我)吓糊涂了,拿本古书前来搪塞老夫来了。

张苍 宗卷也罢,古书也罢,你要看呐!(或:是与不是,你要看个明白。)

陈平 (嗯,)我倒要仔细地看上一看。

(陈平坐桌旁小边座,看卷)

陈平 ``初起义来在沛丰'',哎呀张大人(呐),这是宗卷呀!(陈平立到台中间)

张苍 这不是宗卷呐。

陈平 是什么?

张苍 乃是卷宗啊。(右手画圈,往上一指)

陈平 来来来(或:诶呵,取笑了。诶,有了宗卷),请来上坐。

张苍
慢来慢来,老相爷的座位是有品级台位的。(或:呃呃呃,这是有品级台位的呀。)

陈平 (诶,)有了宗卷就没有品级台位了,请坐请坐。

(二人桌旁八字坐,张苍大边,陈平小边)

陈平 (啊,张大人,)这部宗卷是哪里来的?

张苍
老相爷哪里知道,下官在癸未年间染病在床,(或:相爷有所不知,只因癸未年间下官染病在床,)命小儿(或:我儿)秀玉代守宗卷,是他看到(宗卷)第七部第七篇,见襄宫赵娘娘死得可惨,犹恐日后有变,为此抄写一部以防后患。

(陈平 金殿之上被国太用火焚化的?)

(张苍 乃是假的。)

(陈平 这一部呢?)

(张苍 这才是历代历代的老宗卷呐。)

陈平 (老宗卷\ldots{}\ldots{}呃,)只是有一桩不合律(或:不应典)。

张苍 哪一件不合律?(或:哪一事不应典?)

陈平 这皇家的玉玺(或:这皇王玉玺)是哪里来的?

张苍
(哎,)也是我儿一时聪明,将黄蜡(或:用黄蜡)雕成玉玺,真的上面原有一颗,假的上面也打上它一颗,与它(个)真假难辨。

陈平 金殿之上国太用火焚的?

张苍 乃是假的。

陈平 这呢?

张苍 乃是历代历代的老宗卷。

陈平 哦,老宗卷?(或:如此说来这是历代历代的老宗卷?)

张苍 老谱头。(或:老宗卷。)

陈平 老谱头。哦。(或:老谱头?)

张苍 哦。(二人同笑)哈哈哈!(或:老谱头,啊。(二人同笑)哈哈哈!)

陈平 (啊,张大人,)令郎今年多大年纪了?

张苍 我的儿子他今年一十六岁了。(或:他么,嗯,今年一十七岁了。)

陈平
哎呀呀,一十六岁就有这样见识,真乃是出将入相之材。(或:哦,一十几岁就是如此地聪明,日后定是出将入相之材。)

张苍
实不瞒老相爷说呀,我那儿子呀,日后定有你这位份。(或:嗯,我那个儿子啊,日后可以有相爷你这个份位。)

(陈平立大边外场背供)

陈平
呜哙呀,你看这老儿,我不过是奉承他几句(或:两句),他倒冒起高来了,我倒要(或:呃,待我来)耍上他一耍,啊,张苍,你好哇?

张苍 我怎么不好哇?(或:我是怎的不好哇?)

(陈平 你好。)

(张苍 嗯,我好。)

陈平 你好大的胆量呐!

张苍 啊?

陈平 你过来。(或:你父子私抄皇王宗卷。)

张苍 做什么?(或:不曾。)

陈平 你父子誊写宗卷,私造玉玺,要谋篡国太的江山。(或:假造皇王玉玺。)

张苍 哪有此事?(或:哼,无有。)

陈平 谋篡江山。(或:分明有谋篡国太江山之意。)

张苍 无有此事。(或:呃,噤声。)

陈平 走走走,面见国太。(或:哼,走走走!)

(张苍 哪里去呃。)

(陈平 去见国太辩理。)

(陈平右手拉张苍右手,陈左袖搭在张右手里侧,张左手摇阻介)

张苍 去不得。(或:使不得,使不得。)

陈平 走走走。

张苍 我有忏悔呀。(或:慢来慢来,我有忏悔啊。)

陈平 (哼,)我看你的忏悔,我看你(是)冒高不冒高。

(陈平坐大边虎头椅,张苍小边背供念\protect\hyperlink{fn118}{\textsuperscript{118}})

张苍
嗨,有了宗卷,扬长而去,岂不是好,我冒的什么高,冒出祸来了。也罢,上前赔个笑脸也就是了。哈哈哈,啊老相爷,学生冒犯老相爷,这厢作揖赔礼了。(或:哎呀且住,交了宗卷,也就无有事了。我与他冒的什么高,咳,你看你看,冒出祸来了。这这这\ldots{}\ldots{}也罢,我上前赔个笑脸可也就拉倒了。哈哈哈,老相爷,方才是下官的不是,得罪了老相爷,我这厢赔礼了。)

(陈平站,如前拉张苍,陈坐)

陈平
唗,你父子私造玉玺,誊写宗卷,你我面见国太。(或:唗,胆大张苍,你父子私抄皇家宗卷。)

(张苍 不曾。)

(陈平 假造皇家玉玺。)

(张苍 哼,无有。)

(陈平 有谋篡国太江山之意。)

(张苍 噤声。)

(陈平 难道说作个揖就罢了不成么?)

张苍 去不得。

陈平 面见国太。

张苍 去不得,我还有大大的忏悔呀!(或:呃呃呃,我还有大大的忏悔呀!)

陈平
我看你的大大忏悔,作个揖就罢了不成?(或:哼,我看你的大大忏悔,嗯------只怕是忒轻了罢。)(陈平坐)

张苍
哎呀呀,这个老儿分明是叫我与他下个全礼,我们俱是炎汉忠良,下一全礼又有何妨。啊,哈哈哈,老相爷,学生冒犯老相爷,我这里作揖跪下了。(或:唉,看他之意,是要我与他下一全礼呀。嗯,俱是炎汉忠良,下一全礼又有何妨。啊,老相爷,下官得罪了老相爷,我这厢跪下了。)

(张苍向陈平跪)

陈平 下跪何人?

张苍 学生张苍。

陈平 跪在我的面前做甚呐?(或:跪在老夫的面前做甚?)

张苍 (得罪了老相爷,)与老相爷赔礼(或:磕头赔罪)来了。

陈平 你怕我不怕?(或:嗯,你怕了老夫不怕?)

张苍 (呃,)我怕了老相爷了。

陈平 你服(了)我不服?

张苍 (我)服了老相爷了。

陈平 起来。(或:既然如此,我恕你无罪。)

(张苍 多谢老相爷。)

(陈平 起来。)

(张苍 是。)

(张苍起)

陈平 哼!(张苍又跪,陈平扶)

(陈平 大人请起。)

张苍 这是为何?(或:相爷这是何意呀?)

陈平 我与你作耍呢!(或:老夫与你作耍呢!)

张苍 (哎呀呀,)耍出汗来了。

陈平 有了宗卷,张大人(请)回衙理事。

张苍 下官告辞。(或:遵命,告辞了。)

陈平 奉送。

张苍 【西皮摇板】辞别相爷出府门。

(张苍出门到外边,陈平坐小边,张想介)

张苍
不对呀,想这宗卷乃我张苍掌管,陈平老儿苦苦追求,却是为何?这,莫非成皋有人前来盗卷不成。只是何人敢来呢?啊啊是了,昔日老王驾前有一臣子名唤田子春,此人虽则年幼颇有胆识,莫非此人前来盗卷不成?陈平呐陈平,若无此事便罢,如有此事,管教你原礼退回。(或:诶,不对呀,想这宗卷乃是我西台御史掌管,他苦苦地要这宗卷,却是为何?嗯,其中必有缘故。呃,呃,呃\ldots{}\ldots{}莫非淮河前来盗卷不成?只是何人竟敢前来?嗯,昔日老王驾下有一臣子名唤田子春,此人虽则年幼,颇有胆识,莫非他前来盗卷不成?陈平呐陈平,若无此事便罢,如有此事么,管教你原礼退还。)(张苍进门)

张苍 (接唱)再把相爷尊一声。

陈平 张大人为何去而复返?

张苍 非是下官去而复返,有一事不明要在相爷台前请教哇。

陈平 大人请讲。

张苍
想皇家宗卷乃是我张苍掌管,相爷苦苦追求,(或:想这宗卷乃是下官掌管,相爷苦苦索要。却是为何?)

(陈平 这\ldots{}\ldots{}呃\ldots{}\ldots{})

张苍 莫非成皋有人前来盗卷?(或:分明是淮河前来盗卷。)

陈平 啊无有哇。

张苍 就是那田\ldots{}\ldots{}(或:就是那田子春。)

陈平 噤声!(或:不是\ldots{}\ldots{})

张苍 我连人都晓得了。

陈平 无有此事。

(张苍拉陈平如上但反向)

张苍
哈哈,陈平呐陈平,你犯在我的手内来了,你私通外藩。(或:唗,胆大陈平,私通淮河,隐藏奸细。哼,盗取皇王宗卷,有谋篡国太江山\ldots{}\ldots{})

陈平 不曾。(或:呃呃呃\ldots{}\ldots{})

张苍 盗取皇家宗卷。(或:走走走。)

陈平 无有。(或:哪里去?)

张苍 你我面见国太。(或:面见国太。)

陈平 使不得,我也有忏悔。(或:使不得,使不得,呃,老夫我也有忏悔。)

张苍 (嗯,)我也(要)看看你的忏悔呀。(张苍坐大边外边)

陈平
糟糕哇糟糕,有了宗卷,平安无事,与他耍的什么,耍出祸来了。我也与他赔个礼儿可也就拉倒了。哈哈哈,张大人,俱是老朽的不是,我这厢与大人赔礼了。(或:唉,有了宗卷,平安无事啊,我与他耍的什么。唉,耍出祸来了。这这这\ldots{}\ldots{}唉,我也与他赔个笑脸可也就是了。哈哈哈,张大人,方才是老朽的不是,喏喏喏,我这厢赔礼了。)

(张苍站,拉陈平)

张苍 唗,你私通外藩?(或:唗,你这老儿,私通淮河。)

陈平 不曾。(或:无有。)

张苍 盗取皇家宗卷。(或:隐藏奸细。)

陈平 无有。(或:不曾。)

张苍
谋篡国太江山,走走走,你我去见国太。(或:盗取皇王宗卷,谋篡国太的江山\ldots{}\ldots{})

(陈平 噤声。)

(张苍 作个揖,哼,就算了不成?)

陈平 使不得,我也有大大的忏悔呀。(或:呃,我还有大大的忏悔。)

(张苍放手)

张苍
我也看看你的忏悔,作个揖就罢了不成,我还不够本呐,你也来罢。(或:我看你的大大的忏悔。呃,作个揖就算了么,只怕是不够本罢。)

(张苍坐大边外边)

陈平
看他是叫我原礼退还呐。哎,俱是炎汉忠良,与他下一全礼又待何妨。哈哈哈,张大人呐,老朽的不是,我这里与大人磕头赔礼了。(或:呵呵呵,慢来慢来,看他之意是教我原礼退还呐。哎,俱是炎汉忠良,与他磕上一个头是又有何妨。哈哈哈,啊张大人,俱是老朽的不是,我这里与大人磕头赔礼了。)

张苍 下跪何人?

陈平 陈平,

张苍 跪在你张大人的面前(或:跟前)做甚呐?

陈平 与张大人磕头赔礼(或:与张大人赔罪)来了。

张苍 (嗯,)你怕我不怕?

陈平 (呃,)我怕了你了。

张苍 服我不服?

陈平 (呃,)服了你了。

张苍
看你偌大年纪,对着张大人的靴尖,磕上一个响头,也就是了。(或:哼,也罢,看在你偌大的年纪,对着我这靴尖尖,磕上一个响头,饶恕于你。)

陈平 (哦,)是是是。(陈平磕头)

张苍 起来。

陈平 谢大人。

(张苍``嗯''介)

张苍 呵呵呵!

(陈平又跪,张苍搀)

张苍 相爷(快快)请起。

陈平 你这是做甚呐?(或:你这是为何啊?)

张苍 我也是作耍呢。

陈平 哎呀,你把我耍糊涂了(或:吓糊涂了)。

(张苍 相爷恕罪。)

(陈平 岂敢。)

张苍 可有此事?

陈平 大人怎样知晓?(或:正是田子春前来盗卷,你是怎样晓得的?)

张苍 被我糊里糊涂的蒙了出来。(或:哼哼,被我糊里糊涂的蒙出来了。)

陈平 倒被他蒙了去了。(或:哎呀呀,倒被他糊里糊涂的蒙了出来。)

张苍 田大人在此何不请来相见?

陈平 大人稍待。

(出门)

陈平 田大人有请。(或:有请田大人。)

(田子春上小边站)

田子春 相爷何事?

陈平 张苍大人请来相见。

田子春 待我相见,张大人哪里?

张苍 拿奸细,拿奸细!

(张苍叫,陈平拦、田子春回身到里面,张同时回身到里面,再叫,陈拦,田、张又回向外面,陈中)

陈平 这是为何?(或:张大人这做什么?)

张苍 试试他的胆量如何?

陈平 他的胆量是好的。请坐。

(田子春中间,陈平大边,张苍小边,陈给田卷)

陈平 (现有)宗卷在此(,大人请看)。

田子春 宗卷已被吕后焚化,这是哪里来的?

(陈平 这金殿之上,国太用火焚化的乃是假的。)

(陈平 此乃是真的宗卷。)

(陈平
大人有所不知,只因癸未年间张大人染病在床,命张大人令郎代守宗卷,是他看到宗卷第七部第七篇,观见襄宫赵娘娘死得可惨,犹恐日后有变,为此抄写一部以防后患。金殿之上用火焚化的乃是假的。这才是真的宗卷。)\protect\hyperlink{fn119}{\textsuperscript{119}}

陈平 张大人请讲。

张苍
田大人哪里知道,下官癸未年间染病在床,命我儿秀玉代守宗卷,是他看到第七部第七篇,见襄宫赵娘娘死得可惨,犹恐日后有变,为此誊写一部以防后患。

田子春 只是有一桩不合律。

陈平、张苍 哪一桩不合律?

田子春 皇家玉玺哪里来的?

(陈平
也是张大人令郎一时聪明,用黄蜡雕成玉玺,真的上面原有一颗,假的上面也打上它一颗,与它个真假难辨。)

张苍
也是我儿一时聪明,用黄蜡雕成玉玺,真的上面原有一颗,假的上面也打上一颗,与它真假难辨。

田子春 金殿之上用火焚化的?

陈平、张苍 乃是假的。

田子春 这呢?

陈平、张苍 这是(或:此乃是)历代历代的老宗卷。

田子春 啊,老宗卷?

陈平、张苍 老谱头。

田子春 老谱头(三人笑)。张大人,令郎多大年纪?

张苍 我那儿子呀,今年一十六岁了。(或:今年么一十七岁了。)

田子春 日后定是出将入相之材。

张苍
我那个儿子呀,日后定有他\ldots{}\ldots{}(或:我的儿子啊,日后一定像他\ldots{}\ldots{})

陈平 你又来了。

张苍 取笑了。

田子春
有了宗卷,事不宜迟,下官急转淮河,这有淮南王书信,日后发兵到来,还望二位大人做一内应。

张苍、陈平 那个自然。

田子春 告辞。

(田子春走,张苍、陈平出门送,陈外、张里,\textless{}\textbf{尾声}前段\textgreater{})

张苍、陈平 奉送。

(田子春下,张苍、陈平进门,张大边,陈小边)

张苍 告辞。

陈平
慢来,后堂有酒,大家吃个太平饮宴。(或:后面备得有酒,大家饮上它一个太平宴。)

(张苍 到此就要讨扰。还要划拳。)

(陈平 如此七巧------)

(张苍 八马------)

(陈平、张苍 请呐------呵呵哈哈哈(笑介))

张苍 慢来慢来,老相爷的酒,我是不能吃呀!

陈平 哪个吃得?

张苍 炎汉忠良方能吃得。

陈平 哎呀,取笑了,大人请。

张苍 相爷请。(同笑)

(张先下,陈后下。\textless{}\textbf{尾声}合头\textgreater{})

\newpage
\hypertarget{ux6218ux84b2ux5173-ux4e4b-ux738bux9738ux5218ux5fe0}{%
\subsection{战蒲关 之
王霸、刘忠}\label{ux6218ux84b2ux5173-ux4e4b-ux738bux9738ux5218ux5fe0}}

\textbf{{[}第一场{]}}

\textbf{王霸
【二黄原板】恨贼臣暗地里长安城献,害得那军民好不惨然。二幼主虽来在蒲关避险,那胡兵围困得铁桶一般。可怜我月余来未曾合眼,我只得头戴盔、身披甲、腰悬昆吾、昼夜不眠。}\protect\hyperlink{fn120}{\textsuperscript{120}}

\textbf{(搭架子 (内)苦哇!)}

\textbf{王霸 【二黄原板】这一旁人叫苦哀声悲惨,}

\textbf{(搭架子 (内)天呐!)}

\textbf{王霸
【二黄原板】那一边一个个跌足怨天}\protect\hyperlink{fn121}{\textsuperscript{121}}\textbf{。似这等凄苦情有谁怜念,}

\textbf{王霸
【二黄散板】蒲关城好一似首阳荒山}\protect\hyperlink{fn122}{\textsuperscript{122}}\textbf{。}

\textbf{刘忠 【二黄摇板】呼庚唤癸天地惨,易子而食实可怜。}

\textbf{刘忠 参见老爷。}

\textbf{王霸 罢了。}

\textbf{王霸 刘忠。}

\textbf{刘忠 有。}

\textbf{王霸 夜牌可曾发放?}

\textbf{刘忠 俱已发放过了。}

\textbf{王霸 这几日胡英贼子为何全无动静?}

\textbf{刘忠
那胡英贼子,这几日不来攻打,依小人看来,要作一``远困待降''之计。}

\textbf{王霸 何谓``远困待降''之计?}

\textbf{刘忠 老爷你不曾见么?}

\textbf{王霸 见什么?}

\textbf{刘忠 喏:}

\textbf{刘忠
(念)战马耕牛已断影,何须再问犬猫形。树皮剥来也当顿,土根铲出}\protect\hyperlink{fn123}{\textsuperscript{123}}\textbf{也救生。六畜已完草木尽,还有一事不忍闻。}

\textbf{王霸 何事不忍闻呢?}

\textbf{刘忠
城内有孝道之家,双亲在堂,别无侍奉,欲将亲生儿女,杀而烹之。又不忍得下手。无奈:(念)暗地商量不作声,易子而食奉双亲。}

\textbf{王霸 可曾杀了。}

\textbf{刘忠
杀是杀了,怎奈无有柴草奈何?只得将:(念)门窗户壁俱烧尽,拾来白骨当柴薪。到如今:(念)男男女女哭声震,}

\textbf{刘忠 哎呀,老爷呀!}

\textbf{刘忠 (念)不久军民要变更。}

\textbf{王霸 快快安慰他们去罢。}

\textbf{刘忠 遵命。}

\textbf{王霸
且住!听刘忠之言,军心已变,倘若贼兵打破城池,难道说教我束手被擒?!唉呀!这这这\ldots{}\ldots{}}

\textbf{王霸
也罢!我不免去至西院,将我那爱妾徐艳贞杀死,犒赏三军。我就是这个主意呀,诶------我就是这个------主呃意。}

\textbf{{[}第二场{]}}

\textbf{王霸
【二黄散板】听她言来祈心愿}\protect\hyperlink{fn124}{\textsuperscript{124}}\textbf{,口口声声求我安。成败只好凭天断,}\protect\hyperlink{fn125}{\textsuperscript{125}}

\textbf{刘忠 【二黄散板】眼见军民要变迁。}

\textbf{刘忠 参见老爷。}

\textbf{王霸 刘忠,安慰他们怎么样了。}

\textbf{刘忠
小人正在安慰他们,是他们言道:明日后日,救兵再若不到,他们就要开城逃命去了!}

\textbf{王霸 哦!这是他们讲的么?}

\textbf{刘忠 正是。}

\textbf{王霸 唉!这也难怪他们。}

\textbf{王霸
哎呀且住!军心已变,要安抚众心,非那条道路不可。只是教我怎样下手?唉呀!这这这\ldots{}\ldots{}}

\textbf{王霸 有了!}

\textbf{王霸 刘忠。}

\textbf{刘忠 有。}

\textbf{王霸 老爷平日待你如何?}

\textbf{刘忠 老爷待小人,情同骨肉,恩重如山。}

\textbf{王霸
好,既然情同骨肉,恩重如山。老爷如今有件为难之事,教你代办,你可愿去?}

\textbf{刘忠 慢说有件为难之事,就是粉身碎骨,万死不辞。}

\textbf{王霸 好哇。这有宝剑一口,命你去至西院,将你那二主母杀来见我!}

\textbf{刘忠 哎呀,老爷呀,你敢是饿糊涂了?}

\textbf{王霸 老爷饿虽饿,心中却还明白。}

\textbf{刘忠 既然明白,为何好端端要杀我那二\ldots{}\ldots{}}

\textbf{王霸 噤声!}

\textbf{王霸 刘忠啊!}

\textbf{王霸
【二黄散板】只因蒲关军心变,无有良策把众安。因此起了杀妻念,暂济燃眉顾眼前。}

\textbf{刘忠
老爷忠心一点与日月争光,慢说教小人前去杀她,就是令人闻了这番言语,肝肠寸断,心似箭穿。也罢!老爷倒不如将小人杀死,犒赏三军,你看如何?}

\textbf{王霸 你么?}

\textbf{刘忠 正是!}

\textbf{王霸 唉!难得你一片忠心,可惜你的名份又差别了。}

\textbf{刘忠 怎么我的名份又差别了?}

\textbf{王霸 刘忠,你是去与不去?}

\textbf{刘忠 小人我不、不,不敢前去。}

\textbf{王霸 你当真不去?}

\textbf{刘忠 当真不去。}

\textbf{王霸 果然不去?}

\textbf{刘忠 果然不去。}

\textbf{王霸 诶------为国亡家何劳妻、奴,待我自刎了罢!}

\textbf{刘忠 老爷不必如此,小人我愿、愿,愿去就是。}

\textbf{王霸 好哇!。}

\textbf{王霸
(念)含悲付剑珠泪淋,西厢去见二夫人。就说老爷三不忍,我今立等在前厅。}\protect\hyperlink{fn126}{\textsuperscript{126}}

\textbf{{[}第三场{]}}

\textbf{(刘忠
走哇!}\protect\hyperlink{fn127}{\textsuperscript{127}}\textbf{)}

\textbf{刘忠
【二黄散板】在前堂领了老爷命,西院去杀二夫人。悲悲切切西院进。}

\textbf{刘忠 不是小人我要来的,乃是我家老爷要小人来的呀。}

\textbf{刘忠 我家老爷教小人前来与夫人借,借、借\ldots{}\ldots{}}

\textbf{刘忠 借粮啊,呃\ldots{}\ldots{}(哭介)}

\textbf{刘忠 饿虽饿,心中却还明白。}

\textbf{刘忠 不是与夫人借,还是与夫人要,要、要\ldots{}\ldots{}}

\textbf{刘忠 要粮啊,呃\ldots{}\ldots{}(哭介)}

\textbf{刘忠 夫人明鉴,呃\ldots{}\ldots{}(哭介)}

\textbf{刘忠
哎呀夫人呐,天交四鼓,堪堪五鼓天明。我家老爷现在前厅立等。倘若迟延,他、他,他就碰\ldots{}\ldots{}}

\textbf{(徐艳贞 这宝剑是哪里来的?)}

\textbf{刘忠
这宝剑么\ldots{}\ldots{}乃是老爷交与小人的。呃\ldots{}\ldots{}(哭介)}

\textbf{刘忠 正是。}

\textbf{刘忠
我家老爷言道:刘忠啊刘忠,这有宝剑一口,命你去至西院,与二主母借粮。她若有粮,还则罢了;她若无粮,你就说我家老爷不忍、不忍、三不忍。她、她,她就明白了哇,呃\ldots{}\ldots{}(哭介)}

\textbf{刘忠 夫人明鉴呐\ldots{}\ldots{}(哭介)}

\textbf{刘忠 夫人你哪里去?}

\textbf{刘忠 我那二位幼主现在前堂,夫人前去,岂不惊动于他?}

\textbf{刘忠 夫人舍了罢,啊\ldots{}\ldots{}(哭介)}

\textbf{刘忠 夫人你哪里去?}

\textbf{刘忠 夫人你若前去,老爷、夫人焉能割舍?}

\textbf{刘忠 夫人舍了罢,啊\ldots{}\ldots{}(哭介)}

\textbf{{[}第四场{]}}

\textbf{旗牌 (念)贼兵围困城,人马乱纷纷。}

\textbf{旗牌 啊,刘忠、夫人被何人杀死?}

\textbf{王霸 呃------}

\textbf{旗牌 老爷在此。}

\textbf{王霸
你二夫母与刘忠尽忠一死,将他二人尸首搭了下去,烹煮好了,犒赏三军。}

\textbf{旗牌 遵命!(哭介)}

\textbf{王霸 哎,贞娘,刘忠啊!}

\textbf{王霸
【二黄散板】叹贞娘可算得贞节烈性,那刘忠真乃是义仆忠臣。愿邳彤和万修}\protect\hyperlink{fn128}{\textsuperscript{128}}\textbf{早来救应,奏明了幼主爷超度阴魂。}\protect\hyperlink{fn129}{\textsuperscript{129}}

\newpage
\hypertarget{ux95f9ux6606ux9633-ux4e4b-ux9a6cux63f4}{%
\subsection{闹昆阳 之
马援}\label{ux95f9ux6606ux9633-ux4e4b-ux9a6cux63f4}}

\textbf{{[}第一场{]}}

\textbf{马援 {[}引子{]}镇守昆阳,禀忠心,扶保家邦。}

\textbf{马援
(念)年迈苍苍两鬓霜,忠心一片保家邦。父子镇守昆阳地,哪怕贼人犯边疆。}

\textbf{马援
老夫,马援,莽主驾前为臣,奉命镇守昆阳一带等处。可恨妖人刘秀,已在白水村造反。也曾命我儿马洪}\protect\hyperlink{fn130}{\textsuperscript{130}}\textbf{探听贼人动静,去了日久,不见回来。老夫放心不下,不免请出夫人一同商议。}

\textbf{马援 来,请夫人出堂。}

\textbf{(中军 请夫人出堂。)}

\textbf{(马夫人 (念)夫受皇家爵,妻沾雨露恩。)}

\textbf{(马夫人 啊,王爷。)}

\textbf{马援 夫人。}

\textbf{马援 请坐。}

\textbf{马援 唉!}

\textbf{(马夫人 啊,王爷今日为何这等烦闷?)}

\textbf{马援
夫人有所不知,今有妖人刘秀,在白水村造反。也曾命马洪探听贼人动静,去了日久,不见回来,老夫放心不下。}

\textbf{(马夫人
啊,王爷,马洪探听军情,必有好信回来,王爷何必忧虑。今当王爷寿诞之期,妾身备得有酒,与王爷上寿。)}

\textbf{马援 有劳夫人,摆下就是。}

\textbf{马援
【西皮原板】二堂以上饮琼浆,思想娇儿挂心肠。我命马洪探贼党,至今未见转还乡。}

\textbf{(马夫人 王爷呀!)}

\textbf{(马夫人 【西皮散板】马洪探听敌贼党,候儿回来问端详。)}

\textbf{(马洪 (内)马来!)}

\textbf{(马洪 【西皮散板】离了汉营到昆阳,父母台前问安康。)}

\textbf{(马洪 爹娘在上,孩儿拜见!)}

\textbf{马援 罢了。一旁坐下。}

\textbf{(马洪 告坐。)}

\textbf{马援 儿啊,命你探听妖人刘秀消息,为何去了许久,今日才得回来。}

\textbf{(马洪
啊------爹爹,儿奉命探听妖人刘秀消息,中途遇见一个好友,盘桓几日,故而来迟。)}

\textbf{马援 原来如此。}

\textbf{(马夫人 儿啊,今当你父寿诞之期,还不与你爹爹拜寿?)}

\textbf{(马洪 哦,爹爹请上,孩儿拜寿。)}

\textbf{马援 不拜也罢!}

\textbf{马援 呵呵哈哈哈\ldots{}\ldots{}(笑介)}

\textbf{马援
【西皮散板】一见我儿转还乡,倒教(或:不由)老夫喜心肠。回头我对马洪讲,把贼虚实说端详(或:将贼虚实说端详)。}

\textbf{(马洪 父帅呀------)}

\textbf{(马洪 【西皮散板】邓禹那里人马广,特请父帅降汉王。)}

\textbf{马援 啊,我儿言讲降什么?}

\textbf{(马洪 你我父子前去降汉。)}

\textbf{马援 好,近前来!}

\textbf{马援 好奴才!}

\textbf{马援
【西皮散板】听一言来怒满膛,胆大奴才把妖降。怒气不息出二堂,}

\textbf{马援 【西皮散板】管教奴才一命亡。}

\textbf{(马洪 哎呀母亲呐,我爹爹怒出二堂,如何是好?)}

\textbf{(马夫人 我儿前去,为娘随后就到。)}

\textbf{(马洪 唉!)}

\textbf{{[}第二场{]}}

\textbf{马援 (念)奴才犯将令,定斩不徇情。}

\textbf{马援 来,将马洪绑了上来!}

\textbf{(马洪 参见父帅。)}

\textbf{马援 唗!儿在酒席筵前言道,降什么?}

\textbf{(马洪 孩儿只是一句戏言。)}

\textbf{马援 唗!}

\textbf{马援
【西皮散板】骂声奴才好胆量,竟敢背父把贼降。人来与爷忙上绑,斩他首级挂营房。}

\textbf{(马夫人 刀下留人!)}

\textbf{(马夫人
【西皮散板】一见马洪上了绑,倒教老身心内慌。迈步且把大堂上,要斩马洪为哪桩。)}

\textbf{马援 【西皮小导板】耳旁听得人声嚷,}

\textbf{马援 啊?}

\textbf{马援
【西皮快板】原来(是)夫人上大堂(或:到大堂)。你今不在二堂上,来到宝帐为哪桩?}

\textbf{(马夫人 老爷呀------)}

\textbf{(马夫人 【西皮散板】马洪年小应原谅,老爷恕他这一桩。)}

\textbf{马援 住口! (或:住了!)}

\textbf{马援
【西皮散板】夫人把话错来讲,老夫言来听端详:你若再次胡言讲(或:你若再次把情讲),定与奴才一路亡。}\protect\hyperlink{fn131}{\textsuperscript{131}}

\textbf{(马夫人
【西皮散板】老爷不把马洪放,倒教老身无主张。迈步且把二堂上,再想良谋救儿郎。)}

\textbf{马援 中军听令:命你午时三刻监斩马洪,不得有误!}

\textbf{马援 掩门。}

\textbf{{[}第三场{]}}

\textbf{马援
(念)风吹旌旗动,气吹刁斗寒。}\protect\hyperlink{fn132}{\textsuperscript{132}}

\textbf{(中军 汉将劫法场!)}

\textbf{马援 抬枪带马------}

\textbf{{[}第四场{]}}

\textbf{马援 鞭挝伺候!}

\textbf{{[}第五场{]}}

\textbf{(邓禹 城下敢是马老将军?)}

\textbf{马援 嗯------你是何人?}

\textbf{(邓禹 山人邓禹在此。)}

\textbf{马援
呜哙呀,人言邓禹用兵如神,今日一见果不虚传。纵然战死沙场,也是无用的了!}

\textbf{(邓禹
老将军,莽主当灭,我主当兴。老将军若是归顺,不失封侯之位。)}

\textbf{马援 教某归降却也不难,将我儿放出城来,老夫方可归顺。}

\textbf{(邓禹 这有何难,众将官,将马公子放出城去。)}

\textbf{(马洪 请父帅归降了罢!)}

\textbf{马援
儿啊!你既降汉,就该与为父说明,也免得你我父子一场争斗。你来看------为父年迈苍苍,为了你这奴才,如今只落得降汉了!}

\textbf{(马洪 父帅归降了罢!)}

\textbf{马援 为父归降就是。}

\textbf{(马洪 我父归降了。)}

\textbf{(邓禹 请老将军进城。)}

\textbf{(马洪 请父帅进城。)}

\textbf{(众 请啊------)}

\textbf{{[}第六场{]}}

\textbf{(马洪 启禀先生:我父归降,现在辕门候令。)}

\textbf{(邓禹 有请老将军。)}

\textbf{(众 有请马老将军。)}

\textbf{马援 来也!}

\textbf{马援 (念)弃莽归汉王,还是旧昆阳。}

\textbf{马援 臣,马援降顺来迟,主公恕罪。}

\textbf{(刘秀 老将军平身。)}

\textbf{马援 谢座。}

\textbf{(报子 报!牛邈统带人马夺取昆阳。)}

\textbf{(邓禹 再探。)}

\textbf{(报子 啊。)}

\textbf{马援 \textless{}叫头\textgreater{}先生,}

\textbf{马援 既然牛邈讨战,待我父子先立头功。}

\textbf{(邓禹
且慢!牛邈既然到此,必然困倦,必须定计而行:马老将军以为正帅,耿老将军以为副帅。满营悬灯结彩,诱敌安歇,然后袭之!)}

\textbf{马援、耿弇 得令!}

\textbf{(刘秀 后面备酒,与众位将军贺功!)}

\textbf{(众 请!)}

\textbf{{[}第七场{]}}

\textbf{马援 号炮攻山!}

\textbf{马援 穷寇莫追,回复幼主。}

\textbf{马援 请呐!}

\newpage
\hypertarget{ux4e0aux5929ux53f0}{%
\subsection{上天台}\label{ux4e0aux5929ux53f0}}

汉光武 (内白)摆驾!

(四太监,大太监,汉光武上,\textless{}\textbf{帽子头}\textgreater{}起)

汉光武
【二黄慢板】金钟响玉磬鸣王出龙廷(或:宫廷)\protect\hyperlink{fn133}{\textsuperscript{133}},有寡人(或:汉光武)喜的是五谷丰登。君有道民安乐风调雨顺,文安邦武定国四海升平。文凭着邓先生阴阳有准,武仗着姚皇兄\protect\hyperlink{fn134}{\textsuperscript{134}}扶保乾坤。内侍臣摆御驾九龙口进,

郭妃 (内)喂呀\ldots{}\ldots{}

(汉光武进大座)

汉光武 (接唱)殿角下是何人来放悲声?(或:又听得殿角下大放悲声。)

(\textless{}\textbf{小锣抽头}\textgreater{}四宫女引郭妃上)

郭妃 【二黄摇板】轻移莲步上龙廷,万岁驾前说详情。

郭妃 (白)喂呀万岁呀!

(跪哭)

汉光武 梓童为何这等模样?

郭妃 启禀万岁,今有姚刚将郭老太师剑劈府门,万岁做主哇\ldots{}\ldots{}

汉光武 呜哙呀,有这等事,梓童暂且回宫,寡人自有裁处。

郭妃 谢万岁!

郭妃 【二黄摇板】谢罢万岁后宫进,

(郭妃立)

郭妃 (接唱)管叫姚刚一命倾。

郭妃
(念)喂呀\ldots{}\ldots{}(\textless{}\textbf{小锣打下}\textgreater{})

汉光武 内侍,宣伴驾王带子上殿。

内侍 万岁有旨,伴驾王带子上殿呐!

姚期 (内) 领旨,

姚期 【二黄导板】安定府绑姚刚怒气爆发,

(姚刚上到小边台口,姚期上到台口)

姚期
(接唱)【二黄散板】只气得年迈人二目昏花。郭太师在朝中势力甚大,满朝中文武臣谁不让他。似这等王法儿全不惧怕,少时节见万岁定把儿杀。

姚刚
【二黄散板】老爹爹休得要担惊害怕,容孩儿把此事细说根芽,在府门那郭荣将儿辱骂,杀却了老奸贼不犯王法。

姚期 (接唱)小奴才说此话胆比天大,杀皇亲还说是不犯王法。

姚期 儿是好汉?

姚刚 儿是好汉。

姚期 (接唱)是好汉随为父参王见驾,

(姚刚到大边台口,姚期小边立指刚)

姚期 跪下!

(姚刚面外跪)

姚期 (接唱)一桩桩一件件启奏皇家。

(姚期参拜)

姚期 臣姚期见驾吾皇万岁。

汉光武 皇兄平身。

(姚期立,大边立)

汉光武 姚皇兄你可知罪?

姚期 臣知罪,但不知罪犯何条?

汉光武 只因你三子姚刚将郭老太师剑劈府门,还说无罪?

姚期 太师也有一项大罪。

汉光武 太师何罪之有?(或:哪一项大罪?)

姚期 太师府外,立有禁地,文官落轿,武官离鞍,万岁可曾降旨?

汉光武 (这\ldots{}\ldots{})寡人有意,尚未传旨。

姚期 如此斩者无亏。

汉光武 好个斩者无亏。

姚刚 绑坏了。

汉光武 殿角绑的何臣?

姚期 臣子姚刚。

汉光武 哎呀呀,不要绑坏孤的小爱卿,(内侍,)快快松绑。

(内侍松绑,姚刚立进殿里跪)

姚刚 谢万岁不斩之恩。

汉光武
非是寡人不斩于你,孤登基之日有言在先(或:寡人有言在先),姚不反汉,汉不斩姚,赐你三千人马发放湖北(或:发往湖北;发往湖广)宛子城,扶保(或:保定)殿下刘庄,你父子就在午门一别,领旨下殿。

姚期、姚刚 (同)谢万岁。

(姚期、姚刚下殿,大边台口)

姚期 【二黄散板】万岁爷赦了姚霸林,好似枯木又逢春。手拉娇儿下龙廷,

(姚期拉姚刚到大边外角)

姚期 (接唱)为父言来听分明。此去湖北休逞性,切莫再来惹祸根。

姚期 儿来看,

姚期 (接唱)为父的两鬓如霜年耳顺,好一似风前灯、瓦上霜能活几春?

姚刚
(接唱)爹爹不必细叮咛,孩儿岂是不孝人。在朝为官多惊恐,不如弃职回故林。午门别父心酸痛,回府拜别老娘亲。

姚刚 爹爹,我父,爹爹呀!(同期)

姚期 姚刚,霸林,儿呀!(同刚)

姚刚 罢!

(姚刚下,姚期望)

姚期 姚刚,霸林,(哭)啊\ldots{}\ldots{}我的儿呀!

姚期
【二黄散板】午门外去了姚霸林,娇儿言语记在心。二次再把龙廷进,告职归林乐安宁。

(姚再次上殿跪)

姚期 臣姚期二次见驾吾皇万岁。

汉光武 姚皇兄为何去而复返?

姚期 臣启万岁,臣告职归林。

汉光武 姚皇兄告职归林,(教)寡人怎能舍得(或:怎生舍得)?

姚期 微臣也舍不得万岁。

汉光武
既然如此(或:既然舍不得寡人;既然舍不得孤王)就该在朝奉君(的)才是。

姚期 万岁教臣在朝奉君,愿君依臣一项大事。

汉光武 皇兄奏来。(或:哪一项大事?)

姚期 愿吾主戒酒百日。

汉光武
只要皇兄在朝,漫说(是)戒酒百日,就是周年半载又有何妨,内侍,快快搀起孤的姚皇兄。

(内侍扶起姚期,姚站小边)

汉光武
【二黄慢板】姚皇兄休得要告职归林,你本是擎天柱一根。汉江山多亏了皇兄所挣,叫寡人怎舍得开国元勋,你我是布衣的君臣(也可唱【顶板满江红】)。

姚期
【二黄原板】非是臣在金殿告职归林,为的是汉室锦乾坤。愿吾皇休听宫闱本,普天下黎民享太平。

汉光武 【二黄原板】孤离了龙书案,(汉光武出位站大边,内侍分下)
【转二黄慢板】把皇兄带定,有寡人传口诏细说分明:都只为那牛邈(或:牛邈贼)屡犯边境,老皇兄去征讨统领雄兵。又谁知叛逆贼甚是狂狞,用诡计将皇兄围困边庭。多亏了小爱卿少年英俊(或:少年英勇),一杆枪救皇兄得胜回京。孤封他平南王(或:靖南王)金殿畅饮,那郭太师在一旁他心怀不平。他二人在金殿结下仇恨,次日里(或:因此上)将太师剑劈府门。(一来是小爱卿刚强逞性,二来是郭太师误国欺君。)今早朝郭娘娘启奏一本,求寡人斩姚刚把父冤伸。孤岂肯行无道曲直不论,(孤岂肯宠嫔妃是非不明(或:宠爱妃虚实不明)。因此上孤传旨亲自细问,)宣皇兄上金殿细问详情(或:细说详情)。想当年孤(或:孤当年;孤登极)也曾把免死牌赠,姚不反汉,汉不斩姚万古留名。孤避难走南阳东逃西奔,白水村遇皇兄扶保寡人(或:老皇兄接驾在白水西村)。孤念你老伯母悬梁自尽,孤念你三年孝改三月,三月孝改三日,三日孝改三时,三时孝改三刻,三刻孝改三分,三年三月三日三时三刻三分永不戴孝未报娘恩。(孤念你为国家心血用尽,孤念你为国家费尽辛勤。)孤念你诛苏献乾坤重整,孤念你灭王莽社稷重兴。孤念你草桥关亲临大阵,孤念你镇边廷受尽辛勤(或:领人马威镇边廷;统雄师威镇边廷;镇边廷费尽辛勤)。孤念你昔年间东挡西除,南征北剿,昼夜杀砍,马不停蹄,到如今鬓发苍苍,你还是忠心耿耿。孤念你是一个开国的老臣。孤念你生三子(或:三个子)二子丧命,孤念你剩姚刚一脉后根。(此二句台上可不唱:将姚刚发湖北国法明正,事平后出赦旨再赦他回京。)劝皇兄你且把愁眉展定,劝皇兄你那里宽放忧心(或:但放忧心)。劝皇兄西宫去在娘娘台前把好言奉进,劝皇兄受屈膝(或:受屈情)为的是霸林。适才间卿奏本寡人已准(或:寡人皆允;孤王皆允),(有)寡人戒酒(百日我)不听谗言岂斩你这开国的元勋,孤是一个有道明君。叫一声姚皇兄、子匡、伴驾王,孤的爱卿,休流泪,免悲声,放大胆,一步一步步步随定寡人。(\textless{}\textbf{长锤}\textgreater{}下)

姚期
【二黄原板】光\protect\hyperlink{fn135}{\textsuperscript{135}}武爷走南阳闯荡四海,闹昆阳众文武聚首起来,宛子城收岑彭邓禹为帅,取洛阳搜云台马武奇才。自盘古哪有臣把君酒戒,这也是老姚期昔年间东荡西除,南征北剿感动了王的心怀,我还怕谁来。

(\textless{}\textbf{大锣下}\textgreater{})


\item
  \leavevmode\hypertarget{fn3}{}%
  \textbf{刘曾复先生说戏录音作``军门'',似非,此处从《京剧汇编》第十三集
  陈少武、苏连汉口述本。}\protect\hyperlink{fnref3}{↩}
\item
  \leavevmode\hypertarget{fn4}{}%
  《京剧汇编》第十三集
  陈少武、苏连汉口述本作``明祥''。\protect\hyperlink{fnref4}{↩}
\item
  \leavevmode\hypertarget{fn5}{}%
  ``移步儿''吴小如先生建议作``一步儿'',此处从\textbf{《京剧汇编》第十三集
  陈少武、苏连汉口述本}。\protect\hyperlink{fnref5}{↩}
\item
  \leavevmode\hypertarget{fn6}{}%
  \textbf{《京剧汇编》第十三集
  陈少武、苏连汉口述本作``闯了''。}\protect\hyperlink{fnref6}{↩}
\item
  \leavevmode\hypertarget{fn7}{}%
  据段公平君告知,原有此二句,可不唱。\protect\hyperlink{fnref7}{↩}
\item
  \leavevmode\hypertarget{fn8}{}%
  本剧的舞台调度与人物扮相参考了吴焕老师整理的剧本(经刘曾复先生审订)。\protect\hyperlink{fnref8}{↩}
\item
  \leavevmode\hypertarget{fn9}{}%
  此句与上句褒姒念的``奴惧呀'',疑是同一句\protect\hyperlink{fnref9}{↩}
\item
  \leavevmode\hypertarget{fn10}{}%
  石父即虢石父。\protect\hyperlink{fnref10}{↩}
\item
  \leavevmode\hypertarget{fn11}{}%
  此处吴焕老师整理本作``到前边''。\protect\hyperlink{fnref11}{↩}
\item
  \leavevmode\hypertarget{fn12}{}%
  刘曾复先生注:徐碧云排《褒姒》一剧时,萧长华以丑角饰演幽王。\protect\hyperlink{fnref12}{↩}
\item
  \leavevmode\hypertarget{fn13}{}%
  根据刘曾复先生和杨绍箕先生2009年9月25日在电话里说戏录音整理。录音由杨绍箕先生托梁剑峰老师提供,刘曾复先生在电话中向杨绍箕先生主要介绍了整出戏的唱词、调度,同时介绍了小生的唱法。\protect\hyperlink{fnref13}{↩}
\item
  \leavevmode\hypertarget{fn14}{}%
  共叔段由旦角应工,且所唱小生腔不能与旦角相重。\protect\hyperlink{fnref14}{↩}
\item
  \leavevmode\hypertarget{fn15}{}%
  刘曾复先生存本此句作``恩情有限闲'',上注``恩情有显消''。\protect\hyperlink{fnref15}{↩}
\item
  \leavevmode\hypertarget{fn16}{}%
  刘曾复先生存本此处作``灵魂''。\protect\hyperlink{fnref16}{↩}
\item
  \leavevmode\hypertarget{fn17}{}%
  刘曾复先生存本此句作``吾乃卫氏灵魂是也''。\protect\hyperlink{fnref17}{↩}
\item
  \leavevmode\hypertarget{fn18}{}%
  刘曾复先生存本此句作``(生白)
  段在生为姜国母爱子爱媳,不想反遭兄王之害,我夫妻不免梦中解劝一番呵。夫人请。(旦接)
  夫君请。''\protect\hyperlink{fnref18}{↩}
\item
  \leavevmode\hypertarget{fn19}{}%
  此句据刘曾复先生存本,当系准词。``蜀魄''是典籍中常用的杜鹃的别称。刘曾复先生在介绍此剧时可能据别本,作``树破提惨三月雨'',不确。\protect\hyperlink{fnref19}{↩}
\item
  \leavevmode\hypertarget{fn20}{}%
  刘曾复先生存本``惊断''上注``凄断''。\protect\hyperlink{fnref20}{↩}
\item
  \leavevmode\hypertarget{fn21}{}%
  ``粉消香散''四字从刘曾复先生存本,当系准词,说戏录音作``焚烧香泛'',李楠君以为作``焚烧香饭'',香饭是佛家饭食。存本上注``霎时间说不尽风流云散,进宫来见母后心内痛酸。''\protect\hyperlink{fnref21}{↩}
\item
  \leavevmode\hypertarget{fn22}{}%
  刘曾复先生存本此处作``(生旦同白)
  国母醒来。''\protect\hyperlink{fnref22}{↩}
\item
  \leavevmode\hypertarget{fn23}{}%
  刘曾复先生存本此句作``虽是儿丧黄泉也却心甘''。\protect\hyperlink{fnref23}{↩}
\item
  \leavevmode\hypertarget{fn24}{}%
  刘曾复先生存本此处作``悲切切尊国母魂伤魄断,悔不该图兄位自惹身残。辜负了生身母无依无伴,儿死在黄泉路瞑目心甘。''末句上注``劝国母切莫要珠泪不干''。\protect\hyperlink{fnref24}{↩}
\item
  \leavevmode\hypertarget{fn25}{}%
  刘曾复先生存本此处作``这也是天命定数有修短,劝国母终日里免却伤惨。为报恩来世里重亲慈范,千古来是何人百岁同欢。''末句上注``千古来有何人百岁同欢''。\protect\hyperlink{fnref25}{↩}
\item
  \leavevmode\hypertarget{fn26}{}%
  姜氏也可以唱【反二黄慢板】,但一般都唱【反二黄原板】。\protect\hyperlink{fnref26}{↩}
\item
  \leavevmode\hypertarget{fn27}{}%
  刘曾复先生存本此处有``(生白)
  母后哇\ldots{}\ldots{}(接【反调摇板】)''\protect\hyperlink{fnref27}{↩}
\item
  \leavevmode\hypertarget{fn28}{}%
  刘曾复先生存本此处作``儿在阴娘在阳两厢隔断,母子们要相逢今世却难。''\protect\hyperlink{fnref28}{↩}
\item
  \leavevmode\hypertarget{fn29}{}%
  刘曾复先生存本此处作``天将明儿要归不尽悲惨,郑君侯指日里迎请凤鸾。''\protect\hyperlink{fnref29}{↩}
\item
  \leavevmode\hypertarget{fn30}{}%
  刘曾复先生介绍唱词时说明,在中间可以加更次。\protect\hyperlink{fnref30}{↩}
\item
  \leavevmode\hypertarget{fn31}{}%
  该定场诗用的是北宋程颢的七绝《秋月》诗句,李舒先生钞录稿末句作``白云鸿雁两悠悠''。\protect\hyperlink{fnref31}{↩}
\item
  \leavevmode\hypertarget{fn32}{}%
  李元皓君建议作``撩乱''。\protect\hyperlink{fnref32}{↩}
\item
  \leavevmode\hypertarget{fn33}{}%
  钟元普亦作钟元甫,此处从李舒先生钞录稿。\protect\hyperlink{fnref33}{↩}
\item
  \leavevmode\hypertarget{fn34}{}%
  李舒先生钞录稿作``人情付东流'',似非。\protect\hyperlink{fnref34}{↩}
\item
  \leavevmode\hypertarget{fn35}{}%
  甘旨,原意是美味的食品。引申为对双亲的奉养。\protect\hyperlink{fnref35}{↩}
\item
  \leavevmode\hypertarget{fn36}{}%
  《京剧汇编》第七十四集作``焚棉山''。剧中介子推的词句部分参考了吴焕老师记录的刘曾复先生说戏录音文稿。\protect\hyperlink{fnref36}{↩}
\item
  \leavevmode\hypertarget{fn37}{}%
  介子推亦作介之推。\protect\hyperlink{fnref37}{↩}
\item
  \leavevmode\hypertarget{fn38}{}%
  古代单底鞋称履,复底鞋称舄,故以``履舄''泛称鞋。\protect\hyperlink{fnref38}{↩}
\item
  \leavevmode\hypertarget{fn39}{}%
  此句吴焕老师整理的剧本记作``故而埋名是贤良''。\protect\hyperlink{fnref39}{↩}
\item
  \leavevmode\hypertarget{fn40}{}%
  吴焕老师整理的剧本记作``脱衣去襟''。\protect\hyperlink{fnref40}{↩}
\item
  \leavevmode\hypertarget{fn41}{}%
  吴焕老师整理的剧本记作``峰峡''。\protect\hyperlink{fnref41}{↩}
\item
  \leavevmode\hypertarget{fn42}{}%
  《京剧汇编》第七十四集作``结草依''。\protect\hyperlink{fnref42}{↩}
\item
  \leavevmode\hypertarget{fn43}{}%
  吴焕老师整理的剧本记作``孤凤单山立''。\protect\hyperlink{fnref43}{↩}
\item
  \leavevmode\hypertarget{fn44}{}%
  作``望空飞''似亦通。\protect\hyperlink{fnref44}{↩}
\item
  \leavevmode\hypertarget{fn45}{}%
  吴焕老师整理的剧本记作``似影飞''。\protect\hyperlink{fnref45}{↩}
\item
  \leavevmode\hypertarget{fn46}{}%
  吴焕老师整理的剧本记作``双环档''。\protect\hyperlink{fnref46}{↩}
\item
  \leavevmode\hypertarget{fn47}{}%
  吴焕老师整理的剧本记作``来把绵山袭''。\protect\hyperlink{fnref47}{↩}
\item
  \leavevmode\hypertarget{fn48}{}%
  吴焕老师整理的剧本记作``东山岭'';《京剧汇编》第七十四集作``东山进''。\protect\hyperlink{fnref48}{↩}
\item
  \leavevmode\hypertarget{fn49}{}%
  陈超老师注:此时台上摆放横场桌,两把椅子。老生从小边上椅子,老旦滑下来头冲里躺地下,老生上桌发现老旦落山,在桌子上甩发``屁股坐子''甩发盖脸,挡脸,蹬椅子吊毛下桌。\protect\hyperlink{fnref49}{↩}
\item
  \leavevmode\hypertarget{fn50}{}%
  李元皓君认为此处当作``钳口、束身'',即取``钳口不言、束身自好''之意。李楠君以为此处``全扣''当作``拳扣'',并注``拳扣,又名指虎,俗称`手撑子',古时士兵所用掌上兵器''。\protect\hyperlink{fnref50}{↩}
\item
  \leavevmode\hypertarget{fn51}{}%
  市曹,指城市中商业集中之处。古代常在这样的地方处决人犯,因此``市曹''也代指行刑场所。\protect\hyperlink{fnref51}{↩}
\item
  \leavevmode\hypertarget{fn52}{}%
  段公平君建议也可作``不恤''。\protect\hyperlink{fnref52}{↩}
\item
  \leavevmode\hypertarget{fn53}{}%
  夏行涛君建议``戮杀''均作``诬杀''。\protect\hyperlink{fnref53}{↩}
\item
  \leavevmode\hypertarget{fn54}{}%
  ``宗嗣''为宗族继承人、子孙后代之意。有人建议用``宗祀''。``宗祀''是对祖宗的祭祀之意。本文稿中其他剧目中亦同。\protect\hyperlink{fnref54}{↩}
\item
  \leavevmode\hypertarget{fn55}{}%
  据柴俊为老师介绍,余叔岩的词句是``你若是在丹墀不肯招认''。\protect\hyperlink{fnref55}{↩}
\item
  \leavevmode\hypertarget{fn56}{}%
  据李楠君告知,余叔岩唱片套封此句记作``失足遗陷阱''。\protect\hyperlink{fnref56}{↩}
\item
  \leavevmode\hypertarget{fn57}{}%
  \textbf{《战樊城》是楚平王派武城黑去捉拿伍子胥},\textbf{与伍会战是殊死之战},\textbf{开打不能太少}。\textbf{余叔岩打的这套小快枪由钱金福、余叔岩一起设计。}\protect\hyperlink{fnref57}{↩}
\item
  \leavevmode\hypertarget{fn58}{}%
  ``顶''字在《京剧老生把子见闻录》一文中误作``项'',据《京剧新序》一书更正。\protect\hyperlink{fnref58}{↩}
\item
  \leavevmode\hypertarget{fn59}{}%
  李元皓君认为作``赏他一箭''。

  据陈超老师告知,之后的开打与目前舞台上的很不一样。\textbf{夹鞭}的身段也不一样:是先出鞭,再抬腿同时转腰,不抬弓。武也不被射死,搂扑虎后,起身(这个时间伍刚好完成夹鞭身段),挡脸下。\protect\hyperlink{fnref59}{↩}
\item
  \leavevmode\hypertarget{fn60}{}%
  《战樊城》别名《出棠邑》。\protect\hyperlink{fnref60}{↩}
\item
  \leavevmode\hypertarget{fn61}{}%
  ``盟府''俗作``明辅''。承郝以鑫君告,据元明间无名氏杂剧《临潼斗宝》:``穆公与百里奚、秦姬辇设计,由百里奚与诸侯试文辞,秦姬辇比试武力,以争盟府地位。\ldots{}\ldots{}伍子胥文胜百里奚,武胜秦姬辇,夺得盟府地位,保定众诸侯。''
  盟府是掌管盟约文书档案的机构,这里代指司盟之官。故``双挂盟府印二口''、``时来双挂盟府印'',均谓伍员文武双全之意。\protect\hyperlink{fnref61}{↩}
\item
  \leavevmode\hypertarget{fn62}{}%
  ``刀割头''亦可作``刀过头''。\protect\hyperlink{fnref62}{↩}
\item
  \leavevmode\hypertarget{fn63}{}%
  张斯琦君认为此处当作``纫扣搭弓''。\protect\hyperlink{fnref63}{↩}
\item
  \leavevmode\hypertarget{fn64}{}%
  夏行涛君建议此处``共乡土''似作``共相楚''亦通。\protect\hyperlink{fnref64}{↩}
\item
  \leavevmode\hypertarget{fn65}{}%
  钱盛君指出,``把智斗''可能是京剧流变过程中,湖北方言``把志赌''的讹误,存此以备一说。\protect\hyperlink{fnref65}{↩}
\item
  \leavevmode\hypertarget{fn66}{}%
  段公平君建议作``理难容'',亦通。\protect\hyperlink{fnref66}{↩}
\item
  \leavevmode\hypertarget{fn67}{}%
  历阳山一名历山,在历阳县(今安徽和县)西北四十里。\protect\hyperlink{fnref67}{↩}
\item
  \leavevmode\hypertarget{fn68}{}%
  刘曾复先生为吴小如先生说戏时先唱的这几句。\protect\hyperlink{fnref68}{↩}
\item
  \leavevmode\hypertarget{fn69}{}%
  谭派的唱法整理时参考了刘曾复先生为吴小如先生说戏及在其他场合说戏录音。\protect\hyperlink{fnref69}{↩}
\item
  \leavevmode\hypertarget{fn70}{}%
  吴焕老师整理本记作``威风涌''。\protect\hyperlink{fnref70}{↩}
\item
  \leavevmode\hypertarget{fn71}{}%
  樊百乐君介绍,刘曾复先生曾说明:``令出山摇''是``令出山岳动,言发鬼神惊''的讹误。因昆腔北曲``岳''念``要(yào)''音,后误作``山摇'',因袭至今,此处从俗。其余剧目中亦同。\protect\hyperlink{fnref71}{↩}
\item
  \leavevmode\hypertarget{fn72}{}%
  陈超老师介绍,这两句是贾丽川的词句,台上一般不唱,刘曾复先生为保留贾丽川的词句而唱了这两句。\protect\hyperlink{fnref72}{↩}
\item
  \leavevmode\hypertarget{fn73}{}%
  赒济:接济、救助之意。\protect\hyperlink{fnref73}{↩}
\item
  \leavevmode\hypertarget{fn74}{}%
  ``我腹饱,尔身溺''这一句,陈超老师跟刘曾复先生学的是``\textbf{我腹果},\textbf{尔身溺}''。\protect\hyperlink{fnref74}{↩}
\item
  \leavevmode\hypertarget{fn75}{}%
  ``胁''本意为两臂。夏行涛君认为,因``胁''繁体作``脅'',``胁力''可能是``膂力''之误。\protect\hyperlink{fnref75}{↩}
\item
  \leavevmode\hypertarget{fn76}{}%
  ``雒邑''亦作``洛邑'',据明张岱所著《夜航船》载:周公筑雒邑二城,后即为洛阳。汉光武定都洛邑。汉以火德王,忌水,故去水而加佳,改洛为雒;后魏以土德王,以水得土,而流土得水而柔,故又除佳加水。\protect\hyperlink{fnref76}{↩}
\item
  \leavevmode\hypertarget{fn77}{}%
  李楠君认为此处当作``纳允'',夏行涛君则认为此处当作``拿云''。\protect\hyperlink{fnref77}{↩}
\item
  \leavevmode\hypertarget{fn78}{}%
  ``皇家四口''一般指蔡国母、皇姑(一说为公子建)、马昭仪和王孙(芈胜);国母和皇姑(或公子建)在逃亡出京城时丧命,所以后面马昭仪有唱词为``保皇家四口丧两口,不怨将军怨着谁。''\protect\hyperlink{fnref78}{↩}
\item
  \leavevmode\hypertarget{fn79}{}%
  李元皓君建议此处作``误驻车''。\protect\hyperlink{fnref79}{↩}
\item
  \leavevmode\hypertarget{fn80}{}%
  《京剧汇编》第六集
  赵桐珊藏本作``浮云''。\protect\hyperlink{fnref80}{↩}
\item
  \leavevmode\hypertarget{fn81}{}%
  段公平君注:刘曾复先生曾介绍,此处面向里念白。\protect\hyperlink{fnref81}{↩}
\item
  \leavevmode\hypertarget{fn82}{}%
  洒乐同``洒落'',``洒脱飘逸,不拘束''之意。\protect\hyperlink{fnref82}{↩}
\item
  \leavevmode\hypertarget{fn83}{}%
  夏行涛君建议作``提表''。\protect\hyperlink{fnref83}{↩}
\item
  \leavevmode\hypertarget{fn84}{}%
  整理过程中参考了徐芃君的硕士研究生学位论文\textsuperscript{{[}12{]}.}载的刘曾复先生《桑园会》本词句和吴焕老师整理的剧本(经刘曾复先生审订)。\protect\hyperlink{fnref84}{↩}
\item
  \leavevmode\hypertarget{fn85}{}%
  刘曾复先生曾介绍梆子剧中``秋胡打马''一段的唱法词句,兹照录如下:

  ``秋胡打马离山岗,那山上有个刘大王。那大王不把乡里念,他将我连人带马赶下山。回朝难见楚王面,因此上抛官弃印回家转。正行走,用目观,见一位大嫂在桑园。前影看也看不见,后影儿好似妻银环。有心上前把妻认,错认民妻礼不端。''\protect\hyperlink{fnref85}{↩}
\item
  \leavevmode\hypertarget{fn86}{}%
  徐芃君硕士研究生学位论文载刘曾复先生《桑园会》本作``乘此''。\protect\hyperlink{fnref86}{↩}
\item
  \leavevmode\hypertarget{fn87}{}%
  徐芃君硕士研究生学位论文载刘曾复先生《桑园会》本作``吾家''。\protect\hyperlink{fnref87}{↩}
\item
  \leavevmode\hypertarget{fn88}{}%
  徐芃君硕士研究生学位论文载刘曾复先生《桑园会》本作``怒气哗''。\protect\hyperlink{fnref88}{↩}
\item
  \leavevmode\hypertarget{fn89}{}%
  徐芃君硕士研究生学位论文载刘曾复先生《桑园会》本作``进得门来忙跪下''。\protect\hyperlink{fnref89}{↩}
\item
  \leavevmode\hypertarget{fn90}{}%
  ``偢''同``瞅'',``睬''同``倸'';《京剧汇编》第八十七集
  臧岚光藏本作``佯愀不睬''。\protect\hyperlink{fnref90}{↩}
\item
  \leavevmode\hypertarget{fn91}{}%
  夏行涛君建议作``两眼昏''。\protect\hyperlink{fnref91}{↩}
\item
  \leavevmode\hypertarget{fn92}{}%
  刘曾复先生所有的``呢''念``嗫(nie)''音。\protect\hyperlink{fnref92}{↩}
\item
  \leavevmode\hypertarget{fn93}{}%
  夏行涛君建议作``欠三载才把贡上'';《京剧汇编》第九十二集
  虞仲衡藏本作``前三载未把贡上''。\protect\hyperlink{fnref93}{↩}
\item
  \leavevmode\hypertarget{fn94}{}%
  据钱盛君告,有某程长庚纪念文集中文章指出,此处当作``不照''。``照''与``赵''谐音,安徽方言``照''有``发迹''的意思,``不照''与``没有钱''对应。\protect\hyperlink{fnref94}{↩}
\item
  \leavevmode\hypertarget{fn95}{}%
  《京剧汇编》第七集
  马连良藏本作``钟离昧'',钟离眛多误为``钟离昧''或``钟离眜''。\protect\hyperlink{fnref95}{↩}
\item
  \leavevmode\hypertarget{fn96}{}%
  《京剧汇编》第七集
  马连良藏本作``修筑''。\protect\hyperlink{fnref96}{↩}
\item
  \leavevmode\hypertarget{fn97}{}%
  李元皓君指此句可能是``阵脚站如垒''即古代作战讲究的阵脚不能动,此处定场诗是项羽阅兵,头句言将军善於练兵,二句言军威之壮,三句言军容之整,,末句概写。\protect\hyperlink{fnref97}{↩}
\item
  \leavevmode\hypertarget{fn98}{}%
  《京剧汇编》第七集
  马连良藏本作``稳坐''。\protect\hyperlink{fnref98}{↩}
\item
  \leavevmode\hypertarget{fn99}{}%
  夏行涛君建议作``以解此围'',此处从《京剧汇编》第七集
  马连良藏本。\protect\hyperlink{fnref99}{↩}
\item
  \leavevmode\hypertarget{fn100}{}%
  根据《京剧汇编》第七集 马连良藏本补齐。\protect\hyperlink{fnref100}{↩}
\item
  \leavevmode\hypertarget{fn101}{}%
  《京剧汇编》第七集
  马连良藏本``可谓''作``可为''。\protect\hyperlink{fnref101}{↩}
\item
  \leavevmode\hypertarget{fn102}{}%
  《京剧汇编》第七集
  马连良藏本``强盛''均作``强胜''。\protect\hyperlink{fnref102}{↩}
\item
  \leavevmode\hypertarget{fn103}{}%
  ``逄''字音``庞(páng)'',因古字``逄''与``逢''通假,故``逄丑父''亦作``逢丑父''。旧时艺人误念``冯(féng)'',辗转因袭,几成定例。此处刘曾复先生遵从传统习惯,下同。\protect\hyperlink{fnref103}{↩}
\item
  \leavevmode\hypertarget{fn104}{}%
  《京剧汇编》第七集
  马连良藏本``逃难''作``脱难''。\protect\hyperlink{fnref104}{↩}
\item
  \leavevmode\hypertarget{fn105}{}%
  《京剧汇编》第七集
  马连良藏本作``嘉臣''。\protect\hyperlink{fnref105}{↩}
\item
  \leavevmode\hypertarget{fn106}{}%
  段公平君建议作``德性''。\protect\hyperlink{fnref106}{↩}
\item
  \leavevmode\hypertarget{fn107}{}%
  《京剧汇编》第七集
  马连良藏本作``放开''。\protect\hyperlink{fnref107}{↩}
\item
  \leavevmode\hypertarget{fn108}{}%
  夏行涛君认为``心如铁石(或:心如石铁)一般同''一句当由纪信接唱,此处从《京剧汇编》第七集
  马连良藏本。\protect\hyperlink{fnref108}{↩}
\item
  \leavevmode\hypertarget{fn109}{}%
  《京剧汇编》第七集
  马连良藏本作``介之推''。\protect\hyperlink{fnref109}{↩}
\item
  \leavevmode\hypertarget{fn110}{}%
  《京剧汇编》第七集
  马连良藏本作``可为''。\protect\hyperlink{fnref110}{↩}
\item
  \leavevmode\hypertarget{fn111}{}%
  夏行涛君建议作``战定'',此处从《京剧汇编》第七集
  马连良藏本。\protect\hyperlink{fnref111}{↩}
\item
  \leavevmode\hypertarget{fn112}{}%
  夏行涛君建议作``报朝廷'',此处从《京剧汇编》第七集
  马连良藏本。\protect\hyperlink{fnref112}{↩}
\item
  \leavevmode\hypertarget{fn113}{}%
  段公平君建议作``咎问'',此处从《京剧汇编》第七集
  马连良藏本。\protect\hyperlink{fnref113}{↩}
\item
  \leavevmode\hypertarget{fn114}{}%
  段公平君建议作``又恐''。\protect\hyperlink{fnref114}{↩}
\item
  \leavevmode\hypertarget{fn115}{}%
  夏行涛君建议作``谁想''。\protect\hyperlink{fnref115}{↩}
\item
  \leavevmode\hypertarget{fn116}{}%
  ``夜不收'',本是明代辽东边防守军中的哨探或间谍的特有称谓,相当于侦察兵。\protect\hyperlink{fnref116}{↩}
\item
  \leavevmode\hypertarget{fn117}{}%
  夏行涛君建议``马待石''应作``马台石'';马台石是过去上、下马的垫脚石。此处从《京剧新序》原文。\protect\hyperlink{fnref117}{↩}
\item
  \leavevmode\hypertarget{fn118}{}%
  《京剧新序》原文作``陈平坐小边虎头椅,张苍大边背供念''可能有误。陈超老师介绍:刘曾复先生教授的此处台上的``地方''是(\textbf{陈平坐大边虎头椅,张苍小边背供念}),此处从陈超老师的建议。\protect\hyperlink{fnref118}{↩}
\item
  \leavevmode\hypertarget{fn119}{}%
  刘曾复先生说戏录音中这段词句由陈平念。\protect\hyperlink{fnref119}{↩}
\item
  \leavevmode\hypertarget{fn120}{}%
  ``\textbf{二幼主虽来到蒲关避险,那胡兵围困得铁桶一般。可怜我月余来未曾合眼,我只得头戴盔身披甲腰悬昆吾昼夜不眠。}''四句,陈超老师从刘曾复先生学的是``\textbf{幼主爷虽来在蒲关避难,那贼兵围困得铁桶一般。可叹我月余来未曾合眼,我只得头戴盔、身披甲、腰挂昆吾昼夜不眠。}''\protect\hyperlink{fnref120}{↩}
\item
  \leavevmode\hypertarget{fn121}{}%
  ``\textbf{那一边一个个跌足怨天}''一句,陈超老师从刘曾复先生学的是``\textbf{那一旁只落得口怨苍天}''。\protect\hyperlink{fnref121}{↩}
\item
  \leavevmode\hypertarget{fn122}{}%
  ``\textbf{蒲关城好一似首阳荒山}''一句,陈超老师从刘曾复先生学的是``\textbf{蒲关城尽作了首阳荒山}''。\protect\hyperlink{fnref122}{↩}
\item
  \leavevmode\hypertarget{fn123}{}%
  陈超老师从刘曾复先生学的是``土根餐吃''。\protect\hyperlink{fnref123}{↩}
\item
  \leavevmode\hypertarget{fn124}{}%
  \textbf{陈超老师介绍,}二场旦角念白``三炷香''时,王霸有三个重要身段,尤其最后一炷香有``掉剑''身段,是这出戏的特色,应略注。\protect\hyperlink{fnref124}{↩}
\item
  \leavevmode\hypertarget{fn125}{}%
  ``口口声声求我安。成败只好凭天断''两句,陈超老师从刘曾复先生学的是``\textbf{口口声声求苍天。心碎成败凭天断}''。\protect\hyperlink{fnref125}{↩}
\item
  \leavevmode\hypertarget{fn126}{}%
  陈超老师介绍:此处从刘曾复先生学的是``\textbf{含悲付剑嘱家人},\textbf{西院去见二夫人}。\textbf{待等五鼓天明后},\textbf{我就碰}------(刘忠念``老爷'',两人一翻两翻)\textbf{碰死在庭前。快去、快走。}(身段,分下)''刘曾复先生这点表演很精彩。念白、身段配合好。\protect\hyperlink{fnref126}{↩}
\item
  \leavevmode\hypertarget{fn127}{}%
  此处从陈超老师建议加。\protect\hyperlink{fnref127}{↩}
\item
  \leavevmode\hypertarget{fn128}{}%
  右将军槐里侯万修。\protect\hyperlink{fnref128}{↩}
\item
  \leavevmode\hypertarget{fn129}{}%
  \textbf{陈超老师告知}:刘曾复先生教授时不唱这四句,哭完就尾声,紧凑。录音时为留资料保留了这四句。\protect\hyperlink{fnref129}{↩}
\item
  \leavevmode\hypertarget{fn130}{}%
  《京剧汇编》第六十八集
  苏连汉藏本作``马虹''。《京剧谈往录续编》\textsuperscript{{[}13{]}.}载刘曾复先生著《回忆杨小楼的演出》中作``马洪'',此处从之。\protect\hyperlink{fnref130}{↩}
\item
  \leavevmode\hypertarget{fn131}{}%
  刘曾复先生为吴小如先生说戏时,此处板式略有调整,唱完\textbf{【西皮小导板】,``原来是夫人到大堂''一段唱【西皮散板】,}此段唱\textbf{【西皮快板】。}\protect\hyperlink{fnref131}{↩}
\item
  \leavevmode\hypertarget{fn132}{}%
  《京剧汇编》第六十八集
  苏连汉藏本作``风摆旌旗动,气吹刁斗寒。'',夏行涛君建议作``风吹旌旗动,旗垂刁斗寒。''\protect\hyperlink{fnref132}{↩}
\item
  \leavevmode\hypertarget{fn133}{}%
  ``龙廷''一般作``龙庭'',此处从《京剧新序》。

  陈超老师注:此句唱完后,刘秀在过门里整冠、捋髯。与一般演法刘秀在``金钟响''前面的过门里整冠不同。\protect\hyperlink{fnref133}{↩}
\item
  \leavevmode\hypertarget{fn134}{}%
  据《后汉书》,``姚期''当作``铫期'',故``姚皇兄''当作``铫皇兄'',``姚刚''当作``铫刚''。此处从《京剧新序》,下同。\protect\hyperlink{fnref134}{↩}
\item
  \leavevmode\hypertarget{fn135}{}%
  《京剧新序》中``光''字误作``建''。\protect\hyperlink{fnref135}{↩}
