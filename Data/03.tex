\addcontentsline{toc}{section}{\hfill[\hei 三国·两晋]\hfill}
\newpage
\hypertarget{ux6349ux653eux66f9-ux4e4b-ux9648ux5bab}{%
\subsection{捉放曹 之
陈宫}\label{ux6349ux653eux66f9-ux4e4b-ux9648ux5bab}}

{[}第一场{]}

{[}引子{]}官居县令,与黎民,判断冤情。

(念)头戴乌纱奉行先\protect\hyperlink{fn136}{\textsuperscript{136}},四乡开可\protect\hyperlink{fn137}{\textsuperscript{137}}万民欢。家邑有语呼循吏\protect\hyperlink{fn138}{\textsuperscript{138}},德配汪洋水地天\protect\hyperlink{fn139}{\textsuperscript{139}}。

下官姓陈名宫,字公台。蒙圣恩,身授中牟县令(或:职授中牟县正堂)。(自到任以来,地方宁靖。)前者董太师有钧谕到来,命各府州县,画影图形,捉拿刺客曹操。也曾命班头王申\protect\hyperlink{fn140}{\textsuperscript{140}}(、李顺)等四门严拿(或:严查),未见交差回报(或:未见交签)。今当三、六、九日放告之期(或:今当三、六、九日升堂理事)。

左右,伺候了。\\
罢了。

(捉拿刺客之事如何?)

喜从何来?

有何为证?

呈上来。

左右,看赏。(或:来,看赏。)

愿者何来?

呵呵哈哈哈\ldots{}\ldots{}(笑介)

官升吏赏,理所当然。国家法度自无不行。

吩咐下去,将刺客带上堂来。

【西皮摇板】曹孟德进衙来齐声威吓,胥吏们列两旁虎立山坡。观此人面貌上带定凶恶,见本县不下跪却是为何。

下站可是曹操?

见了本县大胆不跪(或:为何不跪)。

你可知(或:岂不知)王子犯法,与民同罪。

刺杀太师,还说无罪。

虽非亲眼得见,现有董太师钧谕到来,(你)还敢强辩?

帘外之官(或:我在帘外为官),不问朝阁之事。

住口。

【西皮二六】曹孟德休得要谤毁朝阁,董太师他自有治国韬略。扶灵帝虽无功却也无过(或:扶灵帝为都尉并无过错),十常侍乱宫闱扫荡妖魔。收下了吕奉先威震海角,传一令好一似地动山挪。我将你解进京是我份责(或:是我所责),千金赏万户侯加官进爵(或:并非为千金赏加官晋爵)。你好比扑灯蛾自来投火,又好比抢食鱼自投在网罗。你好比平阳虎把路走错,擒虎易放虎难自己揣摩。

【西皮快板】听他言不由我双眉皱锁,这件事好教我无计奈何。既擒住若放他罪归于我,若不放又恐怕惹出风波。左思量右辗转(或:前思量后辗转)无计定夺\protect\hyperlink{fn141}{\textsuperscript{141}},

有了!

【西皮快板】学苏秦仿张仪计上心窝。既擒住放不放全凭于我,就是放也说个情理相合。

【西皮摇板】曹孟德说此话如梦方觉,七品官焉能得辅相朝阁。(倒不如弃县令随他入伙,走天涯奔海角重整山河。)下位去与明公忙松绳索,

【西皮摇板】胥吏们且回避爷有发落。

{[}第二场{]}

【西皮摇板】手挽手与明公二堂里坐,驾光临少奉迎望乞恕过。

适才关前、堂上多有冒犯,明公海涵(或:望乞恕罪。)。

明公今欲(或:明公意欲)何往?

我意欲随同明公,奔走(或:海走)天涯,(会合诸侯,)共图大事(,幸勿见却)。

不妨,老母妻室,现在原郡,料然无事。

明公请至书房(待茶),容我安排(或:待我吩咐他们)。

(曹操 暂时别。)

少刻奉(陪)。来,

吩咐下去,老爷下乡查旱,多则半月,少则十天(或:多则十日,少则七天)。将信印付与佐堂执掌(或:掌管),不可(或:休得)罗唣。

附耳上来。

记下了。

小心把守。

{[}第三场{]}

【西皮散板】路上行人马蹄忙。

【西皮散板】见一老丈坐道旁。

明公,(你我)还是趱路要紧呐。

明公,去得的?

这就不敢。

嗯哼。

冒造宝庄,老丈海涵。

有座。

老丈,

【西皮快板】多蒙老丈美言奖,释放皇家一栋梁。七品的郎官何足讲,同奔原为汉家邦。

前途用过,不必费心(或:不用费心呐)。

(啊)明公,适才(或:方才)老丈提起令尊大人,明公双目落泪,真乃忠孝双全。

明公啊。

【西皮快板】休流泪来免悲伤,忠孝二字天下扬。同心协力除奸党,凌烟阁上把名扬。

这般时候,哪道而去?

家常随便,万勿劳心。(或:前途用过,万勿费心呐。)

呵呵哈哈哈\ldots{}\ldots{}(笑介)

【西皮摇板】老丈亲自沽佳酿,待人礼仪似孟尝。

听见什么?

诶,老丈与令尊大人(有)八拜之交,焉有此事,你何必多疑呀?(或:焉有此意,你不要多疑呀。)

呃,这倒使得。

【西皮散板】言语恍惚实难详。

又听见什么?

老丈临行言道,家中菜蔬俱有,只是缺少美酒(或:缺少好酒)。亲自去往前村沽酒回来,还要把敬你我。你不要见差了。

明白何来?

我看老丈面带厚道,断非贪赏之辈。

(依你之见?)

\textless{}\textbf{叫头}\textgreater{}明公!

待等老丈沽酒回来,问上他三言两语(或:三言五语),他若无言,那时节再动手也还,不,不,不迟呀。

依你之见?

使不得。

唉!

【西皮散板】未必他有此心肠。

【西皮散板】求赏焉有此风光。

(使不得!)

(唉!)

【西皮散板】他一家大小要遭祸殃。

【西皮散板】吓得我三魂七魄茫啊。

【西皮散板】陈宫上前扯衣裳。

明公(,将老丈一家杀死,)你意欲何往?

杀人放火不是你我所为。

【西皮散板】杀人还要火焚房。

哎呀!

【西皮散板】见一捆猪在厨房。

明公,你到底将他一家错杀了。

老丈一片好心,杀猪宰羊,款待你我,岂不是错杀了?

你去看呐。

嘿嘿。

\textless{}\textbf{叫头}\textgreater{}明公!

你将老丈一家杀死,待等老丈沽酒回来,问起情由,你有(或:你是)何言答对?

事到如今,也只好是一走哇。

走哇,

走哇,

走走走!

{[}第四场{]}

啊?!

【西皮快板】背转身来自参详。指望他是定国安邦将,却原来贼是个人面兽心肠。

啊,老丈,不必强留,回家自然明白。

你我后会有期,就此谢谢了。

【反西皮散板】陈宫心中似刀扎。

老丈!

【反西皮散板】多蒙老丈你的美意大,好意反成恶冤家。急忙里难说你我的知心话,

老丈!

【反西皮散板】休怨我陈宫你怨他。

【西皮摇板】他人不走事有差。

有什么言语,(你)饶他一条老命吧。

不要回来。

哎呀!

【西皮散板】陈宫一见咽喉哑,白发老丈染黄沙。一家大小丧剑\textless{}\textbf{哭头}\textgreater{}下,老丈啊,

呀呸!

【西皮散板】再与孟德把话答。

\textless{}\textbf{叫头}\textgreater{}明公!

你将老丈一家杀死,尚且追悔不及,出庄又将老丈剑劈道旁,是何理也?

似你这等(或:这般)行事,岂不怕天下人咒骂于你?

哦!

【西皮慢板】听他言吓得我心惊胆怕,背转身自埋怨我自己做差。我先前指望他宽宏量大,却原来贼是个无义的冤家。马行在夹道内我难以回马,这才是花随水水不能恋花。这时候我只得暂且忍耐在心下,既同行共大事必须要劝解于他。

【西皮二六】休道我言语多必有奸诈,你本是大义人把事作差。吕伯奢与你父相交不假,为什么起疑心杀他的全家。一家人被你杀也就该罢,出庄来你为何把老丈来杀,是何根芽。

【西皮摇板】好言语劝不醒蠢牛木马,把此贼好一比井底之蛙。

但凭于你。

{[}第五场\protect\hyperlink{fn142}{\textsuperscript{142}}{]}

马不要下鞍。

明灯一盏。

鞍马劳顿,吞吃不下呀。

既已同行,何言(或:说什么)不服?

你的疑心呐,也忒大了。(或:你那疑心,也忒大了。)

明公。

睡着了。

我陈宫好悔也!

【二黄慢板】一轮明月照窗下,陈宫心中乱如麻。悔不该心猿并意马,悔不该随他人到吕家。吕伯奢可算得义气大,杀猪沽酒款待于他。又谁知此贼的疑心太大,拔出剑将他的满门杀。一家人俱丧在宝剑之下,年迈老丈命染黄沙。屈死的冤鬼魂休来怨咱,自有那神灵天理鉴察。

【二黄原板】听谯楼打罢了二更鼓下\protect\hyperlink{fn143}{\textsuperscript{143}},越思越想把事来做差。悔不该把家属一旦撇下,悔不该弃县令抛却了乌纱。我只说贼是个宽宏量大,汉室后来贼是惹祸的根芽。

明公。

睡着了。

【二黄原板】观此贼睡卧真潇洒,安眠好似井底之蛙。贼好比蛟龙未生鳞甲,贼好比猛虎未曾长牙。虎在笼中我不打,我岂肯放虎归山又把人抓。

【二黄散板】执宝剑将贼的头割下,

【二黄散板】险些儿把事又做差。

我若将他(一剑)杀死,旁人岂不道我与董卓同党?

(这这这\ldots{}\ldots{})

看桌案之上现有笔砚,我不免题诗一首,打动于他。

但不知以何为题?

就以四更为题。

(念)鼓打四更月正浓,心猿意马归旧踪。杀死吕家人数口,方知曹------

(啊,明公。

明公,睡着了。)

(念)方知曹操是奸雄!

陈宫题。

趁此天色朦胧,我不免寻找马匹逃走了罢。

唉!

【二黄散板】也是我陈宫做事差,不该随贼走天涯(或:悔不该随贼奔天涯)。落花有意随流水,流水无情恋落花。

我好悔也!

\newpage
\hypertarget{ux501fux8d75ux4e91-ux4e4b-ux5218ux5907}{%
\subsection{借赵云 之
刘备}\label{ux501fux8d75ux4e91-ux4e4b-ux5218ux5907}}

\textbf{{[}第一场{]}}

\textbf{(念)千军容易得,一将最难求。}

\textbf{俺刘备,只因曹操带领十万雄兵攻打徐州,要与他父曹嵩报仇。陶恭祖向吾弟兄求救。怎奈我兵微将寡,恐难取胜。为此去至北}邳\textbf{公孙瓒那里借兵解围。}

\textbf{军士们,人马缓缓而行!}

\textbf{{[}第二场{]}}

\textbf{【西皮原板】我心中恨曹操奸雄太甚,欺天子霸中原想谋朝廷。屡次里兴人马抢夺沛郡,因此上到北}邳\textbf{亲走一程。忙吩咐众将校前把路引,}

\textbf{【西皮摇板】但愿得见公孙借将回程。}

\textbf{公孙兄!哈哈哈\ldots{}\ldots{}(笑介)}

\textbf{备来得鲁莽,仁兄海涵。}

\textbf{今有曹操,因张闿杀死他父曹嵩,他就赖在陶恭祖的身上。为此带领十万雄兵攻打徐州,要与他父报仇。陶恭祖向弟借兵相助。怎奈备兵微将少,不是他人对手。是以到此与仁兄借兵解围。}

\textbf{望求赵子龙将军前往。}

\textbf{不妨,破曹之后,弟亲自送将回营。}

\textbf{岂肯失信?}

\textbf{如此当面谢过!}

\textbf{告辞。}

\textbf{来此就要讨扰。}

\textbf{请------}

\textbf{干。}

\textbf{徐州危在旦夕,备不敢久停。}

\textbf{告辞了!}

\textbf{【西皮摇板}\protect\hyperlink{fn144}{\textsuperscript{144}}\textbf{】陶恭祖望救兵营门立等}\protect\hyperlink{fn145}{\textsuperscript{145}}\textbf{,备哪有闲心肠来饮杯巡。施一礼辞仁兄足踏金镫,破曹后弟亲自送将回营。}

\textbf{{[}第三场{]}}

\textbf{【西皮摇板】在帐中辞别了公孙仁兄,一心要会一会大将子龙。}

\textbf{【西皮摇板】来至在校场地用目观定,我见了赵将军施礼打躬。}

\textbf{那厢敢是赵将军?}

\textbf{赵将军!呵呵哈哈哈\ldots{}\ldots{}(笑介)}

\textbf{请。}

\textbf{诶,这是则甚?}

\textbf{哎呀,不敢不敢!}

\textbf{与将军掸座。}

\textbf{有坐。}

\textbf{前者磐河之役,使备久已倾倒,今日得见,三生有幸。}

\textbf{岂敢。}

\textbf{只因曹兵甚强,陶恭祖邀我弟兄破曹,不能取胜,为此特来公孙兄帐下借兵解围。今有赵将军前去,大功必成}\protect\hyperlink{fn146}{\textsuperscript{146}}\textbf{也!}

\textbf{怎么,赵将军是顺情而去么?}

\textbf{这个\ldots{}\ldots{}失言了!}

\textbf{但不知将军几时起兵?}

\textbf{赵将军,你来看,天色尚早哇。你我吩咐众将缓缓而行,你我马上叙谈叙谈,以慰平生渴慕,不知赵将军意下如何?}

\textbf{将军请来传令!}

\textbf{你我一同传令:}

\textbf{众将官,一路之上不准扰害百姓,马踏青苗,违令者斩!}

\textbf{天色尚早,我与赵将军马上叙谈叙谈,尔等缓缓而行。}

\textbf{带马!}

\textbf{待我来与将军牵马坠镫。}

\textbf{呃,岂敢岂敢呐。}

\textbf{请------}

\textbf{这------}

\textbf{啊------呵呵哈哈哈\ldots{}\ldots{}(笑介)}

\textbf{赵将军,你看当今之世,天下大乱,汉室衰微。众诸侯各霸一方,争雄赌胜,看来日后成王立帝
,不知又是哪一家了。}

\textbf{哦,赵将军不知?}

\textbf{请------}

\textbf{赵将军,备倒想起一家来了哇:}

\textbf{想河北袁绍,四世三公,文有田丰、沮授、逢纪、郭图之谋,武有颜良、文丑万夫不当之勇,日后汉室基业,定属此人了。}

\textbf{比作何来?}

\textbf{哦,凤毛鸡胆。}

\textbf{呃,守户之犬。}

\textbf{怎见得?}

\textbf{哦,袁绍,他不能。}

\textbf{这该是哪一家呢?}

\textbf{赵将军,备又想起一家来了:}

\textbf{袁绍之弟淮南袁术,占据寿春,兵精粮足,又有纪灵、桥蕤,勇冠三军。况且又得了天子玉玺,不久必要称帝。日后只怕是此人了。}

\textbf{比作何来?}

\textbf{诶------怎么将当世英雄比作了冢中枯骨?}

\textbf{哦,呃,他不能。}

\textbf{哦,是了。我想他每每兴兵,霸占民女,掳抢民财,鞭挞士卒,焉能成王霸业呀。}

\textbf{呃,呃,呃\ldots{}\ldots{}备又想起一家来了:}

\textbf{想那荆襄王刘表,坐镇荆州,占有长江之险,统戍一十三郡,兵多将广,又有蔡瑁、张允善习水战,嗯,想天下大势,一定是我那宗兄刘表的了!}

\textbf{呃,正是刘景升。}

\textbf{不能?}

\textbf{哦,请------}

\textbf{唉!如今曹操在青州招贤纳士,将勇兵强,占得天时}\protect\hyperlink{fn147}{\textsuperscript{147}}\textbf{,挟天子以令诸侯。难道说,炎汉基业就归于曹操不成么?}

\textbf{哦,那曹操不足论哉。}

\textbf{呃,请------}

\textbf{啊,赵将军,备想起我那公孙兄,宽宏量大,汉室功臣,足智多谋,又有赵将军辅助。日后成王霸业,呃,一定是我那公孙兄了!}

\textbf{呃,他不能?}

\textbf{如此说来,天下竟无一人了!}

\textbf{备倒明白了:想赵将军天生威武,英雄盖世,智勇双全。日后天下定是赵将军的了!}

\textbf{原要直言!}

\textbf{哪个?}

\textbf{我刘备?!}

\textbf{唉呀呀,哈哈哈\ldots{}\ldots{}(笑介)}

\textbf{小小平原县令,兵不过三千,将不过关、张。}

\textbf{有道是:(念)天上无云难降雨,掌中无剑怎杀人?}

\textbf{【西皮摇板】刘备生来命运薄,小县令焉能掌山河。}

\textbf{啊,赵将军,你看光阴似箭。人生在世------}

\textbf{着啊!备乃困水蛟龙,陷阱猛虎。}

\textbf{古人云:有美玉于斯,韫匮而藏之,求善价而沽之。沽之哉,沽之哉,我待贾者也!}\protect\hyperlink{fn148}{\textsuperscript{148}}

\textbf{【西皮摇板】刘备身旁少英雄。}

\textbf{【西皮摇板】你好比皓月在当空。}

\textbf{赵将军,你问我的志啊------}

\textbf{将军!}

\textbf{【西皮摇板】有朝一日春雷动,得会风云上九重。}

\textbf{哎呀,失言呐失言!}

\textbf{分明``失言''怎说``实言''?}

\textbf{哦,赵将军在磐河牧马的时节,就有心扶助刘备么?}

\textbf{哎呀呀真乃桃园之幸也!}

\textbf{赵将军,你看天色不早,你我马上加鞭。}

\textbf{请!}

\textbf{【西皮摇板】明明知道故意问,侥幸打动子龙心。}

\textbf{我好恨!}

\textbf{【西皮摇板】恨只恨足下不生云,}

\textbf{呵呵哈哈哈\ldots{}\ldots{}(笑介)}

\textbf{【西皮摇板】}聪明不过赵将军。二人催马朝前进。

\textbf{{[}第四场{]}}

将军\textbf{请,将军请!}

\textbf{好将!}

\textbf{{[}第五场{]}}

\textbf{啊,赵将军来此已是徐州境界,备要先行一步。}

\textbf{请。}

\textbf{三弟,代我下马。}

\textbf{请。}

\textbf{{[}第六场{]}}

\textbf{赵将军请坐。}

\textbf{为国勤劳,何言辛苦。}

\textbf{此番借得兵马三千,大将一员。}

\textbf{就是赵将军。}

\textbf{这是我三弟。}

\textbf{三弟,见过赵将军。}

\textbf{赵将军请坐请坐,想我那三弟张翼德,乃是鲁莽之夫,言语傲慢。}

\textbf{备这厢赔罪了!}

\textbf{呃,备这厢有礼了!}

\textbf{{[}第七场{]}}

\textbf{唉!糟糕!}

\textbf{赵将军,我三弟万分粗鲁,言语冒犯,得罪将军,休得见怪。}

\textbf{我这厢下马------}

\textbf{哎呀,将军呐,典韦人马犹如潮水一般,我只得马上赔礼,马上赔礼!}

\textbf{{[}第八场{]}}

\textbf{且住!赵云到此,一战未交,为何将人马撤回北}邳\textbf{?}

\textbf{呃,莫非此人外实内虚,不免去至两军阵前,假意败在他的马前,看他救我不救。}

\textbf{正是呀:(念)大事安排定,打动子龙心。}

\textbf{{[}第九场{]}}

\textbf{啊三弟!}

\textbf{哼,我看那赵云,不如三弟------你呀!}

\textbf{{[}第十场{]}}

\textbf{呵呵哈哈哈\ldots{}\ldots{}(笑介)}

\textbf{后帐摆宴与赵将军贺功!}

\newpage
\hypertarget{ux6253ux9f13ux9a82ux66f9-ux4e4b-ux7962ux8861}{%
\subsection{打鼓骂曹 之
祢衡}\label{ux6253ux9f13ux9a82ux66f9-ux4e4b-ux7962ux8861}}

{[}第一场{]}

{[}引子{]}天宽地阔,运机谋,智广才多。

(念)口似悬河语似流,全凭舌尖运机谋。男儿若得擎天手,自然谈笑觅封侯。

卑某(或:卑末)姓祢名衡,字正平,乃平原孝义村人氏。幼读诗书,深通战策。(虽怀王佐之才,)少游北海,偶遇孔融。他将我荐与曹府门下作幕\protect\hyperlink{fn149}{\textsuperscript{149}}。我想那曹操,名为汉相,实为汉贼,焉能敬贤礼士?此番进得相府(或:去至相府),必须见机而行。

正是:(念)未遇圣命主,又愧栋梁才。(或:未逢圣明主,又愧栋梁才。)

【西皮快三眼】平生志气运未通,似蛟龙困在浅水中。有朝一日春雷动,得会风云上九重。

{[}第二场{]}

(来也。)

【西皮快板】相府门前杀气高,密密层层排枪刀。画阁雕梁双凤绕,亚赛天子九龙朝。

丞相在上,卑某(或:卑末)礼到。

卑某(或:卑末)姓祢名衡,字正平,乃平原孝义村人也。

呜哙呀!人言曹操轻贤慢士,今日一见果然名不虚传(或:果然话不虚传)。孔大夫,你把我错荐了。

【西皮快板】人言曹贼多奸巧,果然亚似秦赵高。欺君误国非正道,全凭势力压当朝。站在丹墀微微笑,哪怕虎穴与笼牢。

呵呵哈哈哈\ldots{}\ldots{}(冷笑介)

吾笑天地宽阔,并无一人。

你(或:丞相)道你帐下,文能安国武能定邦(或:文能安邦,武能定国)。请问丞相,帐下文有谁能,武有谁高?

卑某(或:卑末)愿闻一二。

呵,呵,呵呵呵呵呵\ldots{}\ldots{}(冷笑介)

你道你帐下,尽是英雄豪杰。依卑某(或:卑末)看来,尽是些无用之辈呀。

听道: 荀彧、荀攸,可使吊丧问疾;

郭嘉、程昱,可使看墓守坟;

乐进、李典,可使牧羊放马;

许褚、张辽,

哎,也只可使击鼓鸣金呐。

曹子孝,呼为要钱太守;

夏侯惇,人称完肤将军。

余下者,尽是些衣架、饭囊,酒桶、肉袋,碌碌之辈,何足道哉?

区区不才,幼读诗书,深通战策。天文地理之书,无所不读;三教九流之事,无所不晓。上,可以致君为尧、舜;下,可以配德与孔、颜。吾乃天下名士,岂与你这奸贼同党。孔大夫,你把我错荐了。

【西皮快板】平生志气与天高,不愿金钱结富豪。我本是堂堂青史表,岂与犬马共同槽。

(量你也不敢呐。)

这\ldots{}\ldots{}

愿为鼓吏。

呵,呵,呵呵呵\ldots{}\ldots{}(冷笑介)

【西皮二六】丞相委用恩非小,用为鼓吏怎敢辞劳。背转身来微微笑,孔融做事也不高。明知曹贼多奸巧,全凭势力【转西皮快板】压当朝。我越思越想心头恼,安排巧计骂奸曹。罢罢罢暂且忍下了,明天自有我的巧妙高。

{[}第三场{]}

【西皮导板】适才与贼一席话,

【西皮散板】气得我正平乱如麻。

(念)酒逢知己千杯少,语不投机半句多。

适才进得相府,与那贼深施一礼,他坐在上面,昂然不动,倒还罢了哇(或:还则罢了),反道我的礼貌不周。明日大宴群臣,将我用为鼓吏。分明是羞辱于我哇。我不免明日当着满朝文武,将贼(或:将他)辱骂一回。纵然将我斩首,也落得个青史名标!正是:

(念)明知山有虎,偏向虎山行。

【西皮快板】昔日里韩信受胯下,英雄落魄(或:落魄英雄)走天涯。到后来登台把帅挂,扶保汉室锦邦家。到明天进帐把贼骂,拚着一命染黄沙。纵然将我的头割下,落一个骂贼的名儿扬天涯。

{[}第四场{]}

来也!

(内)【西皮导板】谗臣当道谋汉朝,

【西皮原板】楚汉相争动枪刀。汉王爷咸阳登大宝,一统山河乐唐尧。到如今出了个奸曹操,上欺天子下压群僚。我有心替主爷把贼讨\protect\hyperlink{fn150}{\textsuperscript{150}},掌中缺少杀人的刀。陪席坐定【转西皮快板】奸曹操,左右文武众群僚。元旦节与贼个不祥兆,假装疯魔骂奸曹。我把蓝衫\protect\hyperlink{fn151}{\textsuperscript{151}}来脱掉,

【西皮原板】破衣褴衫摆摆摇。怒气不息登甬道,帐下的儿郎闹吵吵。

【西皮快板】你二人休得呵呵笑,有辈古人听根苗:昔日太公曾垂钓,张良进履在圯桥。为人受得苦中苦,脱去蓝衫换紫袍。

呸!

【西皮快板】你二人把话讲差了,休把虎子当狸猫。有朝一日时运到,拔剑要斩海底鳌。

【西皮快板】休道我白日梦颠倒,顷刻就要上青霄。我把破衣也脱掉,

【西皮快板】赤身露体逞英豪。耀武扬威往上跑,

【西皮快板】你丞相降罪有我承招。

【西皮快板】将身来在西廊道,\protect\hyperlink{fn152}{\textsuperscript{152}}

【西皮散板】看奸贼他把我怎开销。

曹操。

你叫得我祢衡,我就叫得你曹操!

我露父母之遗躰\protect\hyperlink{fn153}{\textsuperscript{153}},方显我是清洁的君子。

你就是混浊的小人!

听道:你不识贤愚,眼浊也;不纳忠言,耳浊也;不读诗书,口浊也;(或:不读诗书,口浊也;不纳忠言,耳浊也;)常怀篡逆,乃是心浊也!

我乃天下名士,将我用为鼓吏,犹如臧仓毁孟子,阳货轻仲尼。曹操啊,奸贼!(你)真乃匹夫之辈也!

【西皮快板】开言怒发三千丈,大骂曹操听比方:昔日文王访姜尚,亲临渭水请栋梁。臣坐君辇联辔往,为国求贤理所当。我本是堂堂奇男子,把我当作小儿郎。枉在朝中为首相,狗奸贼不识臭和香(或:不知臭和香)。

【西皮散板】曹操把话错来讲,无水怎把蛟龙藏。

【西皮散板】鼓打一通天地响,

【西皮散板】鼓打二通震朝纲。

【西皮散板】鼓打三通灭奸党,

【西皮散板】鼓打四通国安康。

【西皮散板】鼓伐一阵连声响,

【西皮散板】管教你狗奸贼死无下场。

列公啊。

【西皮二六】未曾开言我的心头恨,尊一声列公大人听详情:家住在平原孝义村,姓祢名衡字表正平。我胸中颇有安邦论,曾与孔融当过了幕宾。他将我荐与曹奸佞,贼有眼不识宝和珍。我宁做那忠良门下客,不愿做奸贼帐下的人。

【西皮快板】贼道我正平舌辩徒,舌辩之徒有张、苏。苏秦六国为相首,全凭舌辩压诸侯。有朝大展昆仑手\protect\hyperlink{fn154}{\textsuperscript{154}},要把奸贼一笔勾。

【西皮快板】贼那里道我井底蛙,井底下蛙也不差。有朝一日风云驾,要把奸贼一把拿(或:一把抓)。

【西皮散板】狗奸贼他那里故意问道,尊一声列公卿细听根苗:自幼儿举孝廉官职卑小,他本是夏侯子过继姓曹。到如今做高官忘了祖考\protect\hyperlink{fn155}{\textsuperscript{155}}(或:忘了宗祧\protect\hyperlink{fn156}{\textsuperscript{156}}),全不怕骂名儿万载笑嘲。

量尔也不敢呐。(或:哼,我量你也不敢呐。)

住了! (或:呀呸!)

【西皮散板】要往荆州不能够,岂与奸贼作马牛。

(哦。)

【西皮二六】列公大人齐来劝我,犹如方醒(或:犹如推醒)梦南柯。自古道未曾责人先要责己过,手摸胸膛自揣摩。罢罢罢暂息我的心头火,

【西皮快板】学一个陆贾与随何。丞相有事交与我,顺说刘表做定夺。

【西皮摇板】丞相宽心安闲坐,披星戴月奔江河(或:渡江河)。顺说事儿若不妥,

【西皮散板】愿死他乡做鬼魔。

\newpage
\hypertarget{ux6c49ux6d25ux53e3-ux4e4b-ux5173ux516c}{%
\subsection{汉津口 之
关公}\label{ux6c49ux6d25ux53e3-ux4e4b-ux5173ux516c}}

{[}第一场{]}

{[}引子{]}威震乾坤,扶汉室,一点丹忱。

(念)忠义一腔贯古今,补天扶日志平生。英雄几见称夫子,豪杰如斯乃圣人。\protect\hyperlink{fn157}{\textsuperscript{157}}

某,汉寿亭侯关------。可恨曹操,诓哄孺子,刘琮献了荆襄,反遭其害。刘皇叔弃了新野,欲取荆州。曹兵百万,追赶甚急。因此诸葛军师,命某前来江夏,向大公子刘琦,搬兵取救。怎奈他连日染病未痊,不能发兵。某今在此,心悬两地,好不焦虑人也!

【西皮原板】想国家气运衰令人悲悼,叹不尽创业难英雄功高。刘皇叔帝室后欲将国保,时不至空使人忧虑焦劳。

大公子。

有座。

公子贵恙既已痊愈,克即发兵,与某前去接应,恐刘皇叔悬念之至。

即刻点将发兵。

哦,军师为何来此?快快有请。

哦,军师到了。

得令!

【西皮二六】某正在心悬急军师驾到,好一似风云会波浪腾蛟。府堂上领雄兵谕令军校,

【西皮摇板】斩曹操准备某偃月钢刀。

{[}第二场{]}

(刘备
【西皮散板】败当阳过长桥夏口而奔,猛回头又只见襄阳古城。实可怜数万的无辜百姓,懦弱子失荆州苦及黎民。)

(刘备
【西皮原板】自桃园结义起同扶汉鼎,我三人投公孙屡建奇勋。在安喜鞭督邮弃了信印,仗大义救孔融陶谦让城。收吕布却反被吕布兼并,饮曹操青梅酒饱受【转西皮二六】虚惊。失徐州投河北袁绍不信,兄弟们遭失散相会古城。好容易得新野暂时安稳,)

(刘备 【西皮散板】到如今依然是颠沛飘零。)

(刘备
【西皮摇板】说什么年半百儿是根本,说什么汉宗室儿是皇孙。为蠢子叹糜氏自投枯井,为蠢子险伤我股肱之臣。今日里势已败要儿做甚?)

(刘备 【西皮散板】我岂肯学袁氏溺爱不明?)

{[}第三场{]}

【西皮导板】青龙偃月\protect\hyperlink{fn158}{\textsuperscript{158}}威风凛,

【西皮快板】赤兔胭脂起风云。桃园弟兄忠心耿,誓挽汉室天日倾\protect\hyperlink{fn159}{\textsuperscript{159}}。英雄此时当效命,

【西皮散板】除奸扶汉镇乾坤。

{[}第四场{]}

曹操休得逞强,关老爷来也!

\newpage
\hypertarget{ux4e34ux6c5fux4f1a-ux4e4b-ux5218ux5907}{%
\subsection{临江会 之
刘备}\label{ux4e34ux6c5fux4f1a-ux4e4b-ux5218ux5907}}

{[}第一场{]}

{[}引子{]}奸雄并立,起戈矛。怎能够,中原尽扫。

(念)临难仁心存百姓,登舟挥泪动三军。至今凭吊襄江口,父老犹然忆使君。

孤,刘备,自败当阳,兵屯夏口,只因孔明先生去往东吴,共议破曹之策。只是未见音信回来,教孤放心不下。

来,唤糜竺进帐。

罢了。

只因孔明先生去往东吴,一去渺无音信。意欲命你备下礼物,去往东吴。名为犒军,暗探先生。听我吩咐。

【西皮摇板】过江去探孔明虚实动静,必须要一同回免孤忧心。

【西皮摇板】叹只叹汉刘备未得天运,似困龙何日里平步登云。

【西皮摇板】恨曹瞒勒逼我困于江夏,每日里操兵将习练战法。

{[}第二场{]}

【西皮摇板】糜子仲往江东探其虚诈,待等那先生回细问根芽。

罢了,孔明先生何在?

哦,想是周瑜已定破曹之策,邀我面议。如此准备船只,即刻便行。

二弟少礼,请坐。

兄欲往江东赴会,二弟为何阻拦?

这\ldots{}\ldots{}

无有哇。

啊二弟,如今孙、刘结盟,共破曹操。周瑜相请,必有大事商议。今若不去,则两下猜疑,事不谐矣。

二弟同去,兄无忧矣。

就命三弟、四弟守寨,你我弟兄即刻过江。

{[}第三场{]}

看,江水波涛,水天一色。好一派江景也。

【西皮原板】汉阳江上把船开,波涛滚滚风云来。两旁排列旌旗摆,临江会上逞英才。

二弟。孔明自往江东,渺无音信。今周郎邀我临江赴会,难免有诈,你我弟兄须要留心一二。

{[}第四场{]}

都督,备过江来了。

不敢,都督请。

啊,呵呵哈\ldots{}\ldots{}(陪笑介)

(周瑜 \ldots{}\ldots{}参拜。)

不敢,都督名闻天下,备不才无学,怎当将军重礼。

来,就分宾主而坐。

请坐。

岂敢,都督身挂金印,备少来恭贺,望祈海涵。

今日孙刘结盟,共破曹贼,乃天下之幸也。

不敢,摆下就是。

都督这是何意?

不敢,请起。

都督请!

【西皮原板】多蒙美意礼相邀,临江会上似琼瑶。孙、刘两家结盟好,同心协力破奸曹。

【西皮原板】都督英名天下晓,

【西皮摇板】定有妙计展雄韬。

多谢都督。

干。

乃备二弟云长。

呃,乃他昔年之事,何足都督挂齿。

岂敢呐岂敢。

请便。

请问都督,帐下多少人马?何计破曹?

如此待等都督大功成就,备专当叩贺。

告别了。

【西皮摇板】临江会上备讨扰。

{[}第五场{]}

【西皮散板】此来不见诸葛亮,倒教刘备挂心肠。弟兄同回江夏往。

哎呀,先生呐,你想煞我也。

呃,不知呀。

哎呀,险呐。

先生,同回夏口去罢!

【西皮散板】诸葛亮果奇才世间少有,准备着迎他回整顿貔貅。

\newpage
\hypertarget{ux7fa4ux82f1ux4f1a-ux4e4b-ux9c81ux8083ux8bf8ux845bux4eae}{%
\subsection{\texorpdfstring{群英会\protect\hyperlink{fn160}{\textsuperscript{160}}
之
鲁肃、诸葛亮}{群英会160 之 鲁肃、诸葛亮}}\label{ux7fa4ux82f1ux4f1a-ux4e4b-ux9c81ux8083ux8bf8ux845bux4eae}}

{[}第一场{]}

鲁肃 来也!

鲁肃 (念)运筹除汉逆,参赞保东吴。

鲁肃 参见都督,孔明先生到。

鲁肃 现在帐外。

鲁肃 有请诸葛先生。

诸葛亮 嗯哼!

诸葛亮 (念)不惜一身探虎穴,智高哪怕入龙潭。

诸葛亮 (啊)都督。

诸葛亮
有座。(亮来得卤莽,都督海涵。\protect\hyperlink{fn161}{\textsuperscript{161}})

诸葛亮 岂敢。(都督)相召,有何见谕?

诸葛亮 人马未动,粮草先行。

诸葛亮 承都督委派,亮自当效劳。

诸葛亮 就请都督传令。

诸葛亮 得令。

诸葛亮 正是:(念)明知周郎借刀计,佯装假作不知情。

诸葛亮 哈哈哈\ldots{}\ldots{}(笑介)

鲁肃 啊,都督,命孔明劫粮却是何意?

鲁肃 哦,是是是。

(鲁肃出帐在大边台口略一沉思介下)

鲁肃
哈哈哈\ldots{}\ldots{}(笑介)\protect\hyperlink{fn162}{\textsuperscript{162}}

鲁肃 【西皮摇板】诸葛亮背地里将人嘲笑,他道那周都督用计不高。

鲁肃
那孔明他回到馆驿,哈哈大笑,道他水战、陆战,车战、步战,件件精通。\protect\hyperlink{fn163}{\textsuperscript{163}}非比都督只习水战一能耳!

鲁肃 然也。

鲁肃 是是是。

鲁肃 咳,原令追回。

(鲁肃出帐在大边台口轻摊手轻叹下)

{[}第二场{]}

(周瑜 但不知\ldots{}\ldots{})

鲁肃 乃是荆襄降将蔡瑁、张允。

鲁肃 啊,都督闻得蒋干过江,为何发笑?

鲁肃 哦哦哦\ldots{}\ldots{}

鲁肃 啊,都督,想那蒋干乃都督同窗契友,恐识笔迹,肃来代笔。

鲁肃 是是是。

(鲁肃出帐在大边台口左手轻捋胡子、轻转腰看右手信)

鲁肃 噗。(笑介)

(鲁肃用手中信一挡头,边转身边把信送入左袖口内下)

(鲁肃藏书,有细致做工\protect\hyperlink{fn164}{\textsuperscript{164}})

(鲁肃藏完书出帐先往上场门边走一小步,听见蒋干来了,立即撤左脚往下场门边轻退,边退边把灯从右手交左手,右手投袖遮灯,站稳偷眼一望、一点头,沉着转身轻轻稳步下)

鲁肃 啊,蒋先生,请了请了。

鲁肃 都督醒来,都督醒来。

(周瑜出帐子)

鲁肃 哈哈哈\ldots{}\ldots{}(笑介)

鲁肃 那蒋干果然他盗书逃走了。

鲁肃 都督请看。

鲁肃 呵,哈哈哈\ldots{}\ldots{}(笑介)

鲁肃 量他们不知。

鲁肃 那孔明么,哼,他也未必料到。

鲁肃 呵,哈哈哈\ldots{}\ldots{}(笑介)

鲁肃 【西皮摇板】周都督运机谋神鬼不觉。

(周瑜唱完第三句【西皮摇板】下,鲁肃跟过去到下场门边向里站住,一想、轻摇头,回身右手投袖、颠袖露手)

鲁肃 【西皮摇板】怕只怕瞒不过南阳诸葛。

(用右手指,抓袖转身,\textless{}\textbf{大锣抽头}\textgreater{}稳步甩下摆,微摇头下)

{[}第三场{]}

鲁肃 哈哈哈\ldots{}\ldots{}(内笑介出)

鲁肃 【西皮摇板】曹孟德果杀了蔡瑁、张允,周都督可算得第一能人。

鲁肃 呵哈哈哈\ldots{}\ldots{}(笑介)

(周瑜 \ldots{}\ldots{}为何发笑?)

鲁肃
那曹操果然中了都督借刀之计,杀了蔡瑁、张允。水军头目换了毛玠、于禁掌管了。

鲁肃 量他们不知。

鲁肃 量他也不晓。

鲁肃 遵命。有请诸葛先生。

诸葛亮 嗯哼。

诸葛亮
【西皮摇板】昨夜晚听消息早已料定(或:昨夜晚观天象早已算就;或:昨夜晚观天象早已料定),曹孟德中巧计误杀水军。

诸葛亮 (啊)都督。

诸葛亮 有座。恭喜都督,贺喜都督。

诸葛亮
那曹操中了都督借刀之计,杀了蔡瑁、张允,水军头目换了毛玠、于禁。此二人不习水战,岂非一喜?

鲁肃 怎么他\ldots{}\ldots{}(惊异介,背躬)(知道了。)

诸葛亮 你我不必言明,各写一字在手,看看心意如何。

诸葛亮 请------

诸葛亮 大夫请看。

鲁肃 嗳呀!他二人俱是``火''字。

(鲁肃退至小边)

鲁肃 你二人俱是``火''字。

诸葛亮 未必。

诸葛亮 啊,哈哈哈\ldots{}\ldots{}(笑介)

鲁肃 哦,哈哈哈\ldots{}\ldots{}(笑介)

诸葛亮 军国大事焉能泄漏。

诸葛亮 水面交锋弓箭当先。

诸葛亮 愿当此任。

诸葛亮 但不知宽限多少日期?

(周瑜 三月如何\ldots{}\ldots{})

诸葛亮 忒多了哇。

(周瑜 十日\ldots{}\ldots{})

诸葛亮 倘若曹操进军,岂不误了大事?还多啊哇。

(周瑜 七日如何\ldots{}\ldots{})

诸葛亮 军务紧急,还多哇。

(周瑜 \ldots{}\ldots{})

诸葛亮 容山人计算,三日交箭。

鲁肃 啊?三日焉能造得齐十万支雕翎(狼牙)箭呐?先生!

(周瑜 三日无箭?)

诸葛亮 三日无箭,甘当军令。

(周瑜 军无戏言。)

诸葛亮 愿立军状。

诸葛亮 亮乎?

鲁肃 使不得!

鲁肃 完了。

诸葛亮 大夫,这是山人的军令状,请大夫收藏。

诸葛亮 三日后江边搬箭。

鲁肃 搬你的尸灵吧。

诸葛亮 取笑了。

诸葛亮 告辞了。

诸葛亮 【西皮摇板】在帐中辞公瑾再别子敬,三日后到江边搬取雕翎。

鲁肃 啊,都督,那孔明莫非有逃走之意?

鲁肃 是。

鲁肃 啊,都督,此二人恐是诈降。

(周瑜叫鲁肃``出帐去罢'')

鲁肃 哦,是是是。

(鲁肃到大边台口)

鲁肃
(小\textless{}\textbf{叫头}\textgreater{})哎呀且住!分明是诈,怎说是实?哎呀,这这这\ldots{}\ldots{},呵,有了,我不免去至馆驿,问过孔明先生,呃,问过孔明先生。(\textbf{不念对儿}\protect\hyperlink{fn165}{\textsuperscript{165}})

(鲁肃转身扬右手、轻摇)

鲁肃 哈哈哈\ldots{}\ldots{}(笑介)

(鲁肃边笑边下)

{[}第四场{]}

诸葛亮
【西皮原板】周公瑾命鲁肃行监坐守,叫山人背地里暗笑不休。他那里要杀我怎能得够,一桩桩一件件记在心头。

鲁肃 【西皮原板】限三天造雕翎不多时候,

(鲁肃 嘿嘿。)

鲁肃 【西皮原板】为什么坐一旁不睬不愁。

(鲁肃 嗨!)

鲁肃 【西皮快板】昨日里在帐中夸下海口,这时候倒叫我替你担忧。

诸葛亮 我(又)没有什么大事,要大夫替我担得什么忧哇?

鲁肃
啊?!昨日你(或:你昨日)在帐中夸下海口,立了军(令)状:三日造齐十万支狼牙箭。你算算,今天第几天了?

诸葛亮 怎么还有此事么?

鲁肃 呵?

诸葛亮 不错不错,不是大夫提出,我倒忘怀了。

鲁肃 哎呀,哎呀,你怎么忘怀了。

诸葛亮 来来来,(我们)算算日期吧。昨日,

鲁肃 一天。

诸葛亮 今日,

鲁肃 两天,

诸葛亮 明日。

鲁肃 三天。拿来。

诸葛亮 什么?

鲁肃 拿箭来呀。

诸葛亮 哎呀,我是一支也无有哇!

鲁肃 哎呀,这这\ldots{}\ldots{}

诸葛亮 (哎呦)大夫,你要救我一救(,你要救我一救)哇。

鲁肃 咳,事到如今,倒不如你驾一小舟逃回江夏去罢!

诸葛亮
呵,我奉主之命前来同心破曹。如今寸功未立,回去怎样回覆吾主?我如何走得?(呃,)走不得呀!

鲁肃 (哎呀)走不得\ldots{}\ldots{}咳,我只有这一条主意了。

诸葛亮 大夫有何高见。

鲁肃 你呀,投江死了罢。

诸葛亮 呵。

鲁肃 还落一全尸呀。

诸葛亮
嗳,蝼蚁尚且贪生\protect\hyperlink{fn166}{\textsuperscript{166}},为人岂不惜命?你救不了我,还则罢了,你不该劝我一死。这叫作什么朋友哇。

鲁肃 唉,叫你走你说走不得,叫你死你又舍不得。真真的教我鲁肃为难呐!

诸葛亮 大夫哇!

鲁肃 大夫哇,不能下药了。

诸葛亮 【西皮摇板】鲁大夫平日里待人宽厚,

鲁肃 本来的不错哇。

诸葛亮 【西皮摇板】你保我过江来无祸无忧。

鲁肃 是我的保荐呐。

诸葛亮 【西皮摇板】周都督要杀我你不来搭救,

鲁肃 我是怎样救你呀?

诸葛亮 【西皮摇板】看起来算不得好朋友哇。

鲁肃 呵。

鲁肃 【西皮快板】这件事本是你自作自受,为什么反把我埋怨不休。

鲁肃 你怎么倒埋怨起我来了?

诸葛亮 大夫,(大夫)你是救不了我了。

鲁肃 咳,我怎样救你呀?

诸葛亮 嗯,你救不了我,我与你借上几样物件如何哇?

鲁肃 什么物件啊?呵呵,我早已预备下了。

诸葛亮 什么?

鲁肃 寿衣、寿帽,大大(的)一口棺木。

诸葛亮 要这些物件何用呐?

鲁肃 事后将你盛殓起来,送回江夏。你就不必挂念了。

诸葛亮 哎呀呀,不是这些宝贝呐。

鲁肃 什么宝贝?

诸葛亮 乃是军中需用之物。

鲁肃 什么军中需用之物?

诸葛亮
战船二十只,军士五百名,茅草千担,青布帐幔,金鼓全份,还要备酒一席。

鲁肃 啊,(这)备酒一席何用呐?

诸葛亮 少时我与大夫同往舟中饮酒取乐哇。

鲁肃 明日无箭,我看你是饮酒哇,还是取乐哇。

诸葛亮 大夫,千万莫对人言,你去办呐。

鲁肃 咳,办呐。

鲁肃
【西皮摇板】十万箭焉能够(或:焉能得)一夜造就,为朋友我只得顺水推舟(右手指)。

(鲁肃左手撩官衣左转身右手盖头,上场门反下(示背着周瑜办事之意))

诸葛亮
【西皮摇板】这件事料鲁肃难以猜透,哪知我袖儿中(或:他哪知我袖中)暗藏机谋。要借箭待等到四更时候,趁大雾到曹营去把箭收。

{[}第五场{]}

鲁肃 【西皮快板】一桩桩一件件俱已办就,请先生到江边即刻登舟。

诸葛亮 大夫,备齐了?

鲁肃 备齐了。

诸葛亮 请。

鲁肃 请到何处(去)哇?

诸葛亮 同往舟中饮酒取乐哇。

鲁肃
要去你去,我不去。(右手手心朝上往外一挥,托胡子扔出去、摇左手右转身要走同时递左手,孔明左手拉住鲁肃左手,右手从鲁肃左胳膊上方过去,用扇柄挑鲁肃胡子)

诸葛亮 走走走。(孔明拉鲁肃下)

诸葛亮 将船往江北而发。

鲁肃 呵,慢来慢来,曹营现在江北,那如何去得的?来来来,待我下船。

诸葛亮 (呃,)船已离岸,(你)下不去了。

鲁肃 便怎么样呀?

诸葛亮 你我一同吃酒(或:一同饮酒)哇。

鲁肃 诶,什么吃酒哇。

诸葛亮
【西皮原板】一霎时白茫茫漫江雾厚,顷刻间观不出在岸在舟。似这等巧机关世间少有,学轩辕造指南以制蚩尤。

(鲁肃 哎!)

鲁肃
【西皮原板】鲁子敬在舟中浑身战抖,把性命当儿戏全不担忧。这时候他还有心肠饮酒\protect\hyperlink{fn167}{\textsuperscript{167}},

(鲁肃 唉!)

鲁肃 【西皮原板】顷刻间到曹营难保人头。

诸葛亮 将船直往曹营进发。

鲁肃 呵,我要下船。

诸葛亮 船行半江,你(是)下不去了。

鲁肃 (呃,)便怎么样呀?

诸葛亮 请来吃酒哇。

鲁肃 呃,吃酒,吃酒,好,吃酒(、吃酒)哇。

诸葛亮 大夫哇,

诸葛亮 【西皮摇板】劝大夫放宽怀且自饮酒,些许事又何必这等担忧。

诸葛亮 擂鼓呐喊。

鲁肃 (哎呀,)不要擂鼓。

(曹操 \ldots{}\ldots{}吩咐放箭。)

(\textless{}\textbf{风入松}\textgreater{}头段)

(军士 \ldots{}\ldots{}经受不住。)

诸葛亮 拨转船头,军士大喊三声:南阳诸葛先生谢曹丞相赠箭。

诸葛亮 大夫,请来观看呐。

(鲁肃先是右手翻袖盖头,往台中间一望,再用左袖盖头、右手撩官衣、窝下)

{[}第六场{]}

(\textless{}\textbf{风入松}\textgreater{}二段,鲁肃、诸葛亮上)

鲁肃 先生,你怎样知道(或:知晓)今晚有此一场大雾哇?

诸葛亮 为谋士者,不知天文,不晓地理,乃庸才也。

鲁肃 先生真乃神人也。

诸葛亮 岂敢。大夫看看,这令可以交得的了么?

鲁肃 交令呐,有我。

诸葛亮 大夫请。

鲁肃 呃,先生请转。

诸葛亮 何事?

鲁肃 我实实服了你了。

诸葛亮 大夫服我何来(呢)?

鲁肃 我服你好阴阳,好八卦。好大的胆量呐。

诸葛亮 我也服了你了。

鲁肃 你服我何来呢?

诸葛亮 我服你在舟中这样呵------(抖介)

鲁肃 𠳶,𠳶,𠳶。(同时``欺''孔明、三指孔明,孔明小撤步三挡下,鲁肃追下)

{[}第七场{]}

鲁肃 参见都督。

(周瑜 \ldots{}\ldots{})

鲁肃 他造齐了。

(周瑜 \ldots{}\ldots{}怎样造\ldots{}\ldots{})

鲁肃
都督容禀:那孔明他回到馆驿,一天也不慌,两天也不忙。到了三日也不用我国工匠人等,只用战船二十只,军士五百名,茅草千担,青布帐幔,金鼓全份。四更时分,去至曹营,擂鼓呐喊。那时满江(的)大雾,(那)曹贼闻知,吩咐水陆两寨一齐放箭。顷刻之间,借来十万支雕翎。特来交令呐。

鲁肃 可算得是活神仙。

鲁肃 有请呵,呵,活神仙。

诸葛亮 (念)狼牙已造就,只在险中求。

诸葛亮 有座。

诸葛亮 都督关照。

诸葛亮 请。

(周瑜 \ldots{}\ldots{}一百脊杖。)

(鲁肃 哎呀!)

(``打盖''时黄盖是跪左腿面里躬\textbf{身受脊杖};诸葛亮正襟危坐,\textbf{不是在喝酒}\protect\hyperlink{fn168}{\textsuperscript{168}};鲁肃跪黄右侧,双袖覆黄背、两轰牢子手,\textbf{是阻周瑜},\textbf{欲向孔明作色},\textbf{不是一劲作揖})

(周瑜下)

鲁肃 这一下我可不服你了(或:我可又不服你了)。

诸葛亮 怎么(又)不服我了?

鲁肃
方才周都督怒责黄公覆,我等俱是他麾下之人,不好讲情呐。你是客位,坐在席前,一言不发,是何道理(或:是何理也)?呵,难倒(说)这酒就是这样(的)好吃的么?

诸葛亮 他二人一个愿打,一个愿挨,与我何干呐?

鲁肃 世间之上只有愿打,哪个愿挨?你愿挨我就来打。

诸葛亮 他二人又是一计呀。

鲁肃 呵,又是一计?呃,倒要领教。

诸葛亮 大夫哇,

诸葛亮 【西皮摇板】他二人定下了苦肉之计,

鲁肃 收蔡中与蔡和呢?

诸葛亮 【西皮摇板】收蔡中与蔡和暗通消息。

鲁肃 今日之事?

诸葛亮 【西皮摇板】黄公覆受苦刑俱是假意,

鲁肃 (哎呀,)我是哪里晓得呀!

诸葛亮 【西皮摇板】进帐去切莫说我诸葛先知呀。

(孔明溜下,胡琴\textless{}\textbf{行弦}\textgreater{}鲁肃转身朝外、左手托右肘、右手摸脖子(手不动颈动))

鲁肃 我是哪里晓得呀!

(鲁肃回身拱手)

鲁肃 呵,先生。(看孔明不见了)

鲁肃 先生,先生,先生!(同时招手边追孔明下)

\newpage
\hypertarget{ux534eux5bb9ux9053-ux4e4b-ux5173ux516c}{%
\subsection{华容道 之
关公}\label{ux534eux5bb9ux9053-ux4e4b-ux5173ux516c}}

(刘曾复 饰 关公、王家祺 饰 曹操;李斌植 司鼓、屠楚材 操琴)

关公 {[}引子{]}一片丹心,辅汉室,锦绣疆宏。

关公
(念)军师将令守华容,恼得某家怒气冲。怎肯因私废公义,擒曹方显肝胆忠。

关公 某,汉室关------某与军师赌头争印,擒拿曹贼。

关公 众军校,

(众 哦!)

关公 备马伺候。

(众 啊!)

关公 【西皮导板】背地里笑诸葛用兵不到,

【西皮原板】在大营他那里藐视英豪。自幼儿读春秋韬略颇晓,为不平斩雄虎\protect\hyperlink{fn169}{\textsuperscript{169}}怒诛土豪。蒙圣母赐清泉呐改换容貌,与大哥和三弟【转西皮快板】结下了生死的故交。初起手\protect\hyperlink{fn170}{\textsuperscript{170}}破黄巾立功不小,酒未寒斩华雄吓坏群僚。过五关斩六将力保皇嫂,古城边斩蔡阳匹马单刀。今奉命埋伏在华容小道,

关公 【西皮散板】今日里一心要活拿奸曹。

关公 小路埋伏,大路点起烟火。曹贼到此,速报我知。

(众 啊!)

曹操 【西皮导板】曹孟德在马上长吁短叹,

曹操 唉!

曹操
【西皮原板】手捶胸眼落泪恨怨苍天。在许昌领人马八十三万,实指望讨东吴夺取江南。庞士元他把那连环计献,诸葛亮借东风火烧战船。在赤壁烧得我头焦肉烂,只剩下十八骑好不惨然。曹孟德勒丝缰喜笑满面,

(曹将 【西皮散板】丞相发笑为哪般?)

曹操 将军!

曹操
【西皮快板】笑只笑周郎见识浅,孔明心中无计谈。此处若有人和马,大家的性命难保全。

曹操 啊!

曹操 【西皮散板】正然说话人呐喊,莫非此地有狼烟。

曹操 看看前面什么旗号哇?

(曹将 乃是``关''字旗号。)

曹操 怎么讲?

(曹将 ``关''字旗号。)

曹操 啊哈,啊哈,啊呵呵哈哈\ldots{}\ldots{}(笑介)

(曹将 丞相为何发笑?)

曹操 你们哪里知道,那关云长曾许下老夫我三不死,难道今日一次不饶?

曹操 如此说来,不用你们杀了。

(曹将 我们杀不得了。)

曹操 不用你们战了。

(曹将 我们也战不得了。)

曹操 席地而坐。老夫亲自向前------嗯哼------搭话。

(曹将 啊。)

曹操
【西皮快板】听说来了关美髯,不由得孟德喜心间。走上前来把礼见,许昌一别有数年。

(众 曹操到,曹操到啊!)

关公 【西皮导板】耳边厢又听得人喊马闹,

(众 曹操到啊!)

关公 【西皮原板】睁开了单凤眼仔细观瞧。

曹操 二君侯,别来无恙。

关公 【西皮原板】狭路上莫不是冤家来到,

曹操 诶,你我朋友相交,何出此言?

关公
【西皮原板】今日里用武时\protect\hyperlink{fn171}{\textsuperscript{171}}怎念故交。

曹操 呃,此言忒重了。

关公
【西皮原板】三国中论奸雄算你曹操,一派的假殷勤笑里藏刀。观天时巳已完午刻来到,拿住了奸曹贼岂肯相饶。

曹操 君侯!

曹操
【西皮原板】曹孟德近前来满面赔笑,尊一声二君侯细听根苗:误中了小周郎牢笼圈套,诸葛亮借东风把我的战船烧。只剩下十八骑残兵来到,望君侯你不信仔细观瞧。

关公 周仓,

周仓 在呃!

关公 查点曹操多少人马。

周仓 得令。

周仓 嘿!站起来嘿!

曹操 (呃,)站起来!站起来!站起来!

周仓
一五,一十,十五,一、二、三。啊哈,啊哈,哇呀呀呀\ldots{}\ldots{}呵呵哈哈\ldots{}\ldots{}(笑介)

周仓 启父王,只有一十八骑残兵败将呃。

关公 怎么讲?

周仓 一十八骑残兵败将。

关公 起过了。

周仓 呃!

关公 \textless{}\textbf{叫头}\textgreater{}军师!

关公 慢说是一十八骑残兵败将,就是一十八只猛虎,关某何惧?

关公 【西皮二六】你好比钓金鲤怎能游遨,大鹏鸟------

曹操 惊弓之鸟。

关公
【西皮二六】褪翎羽难腾青霄。似蛟龙脱离了蓬莱海岛,伤弓鸟纵有翅也难飞逃。

曹操 君侯!

曹操
【西皮快板】想当初我待你恩德不小,上马金、下马银美酒红袍。官封到寿亭侯可算不小,大英雄怎忘却当年故交。

关公
【西皮快板】你虽然待我的恩高义好,我亦曾答报过你的功劳。斩颜良、诛文丑立功报效,挂信印封黄金留柬辞曹。

曹操 【西皮快板】我也曾派张辽文凭送到,酴醿酒大红袍送至灞桥。

关公
【西皮快板】休提起送文凭令人可恼,诛孔、卞,刺孟、韩,王植被枭。过黄河斩秦琪文凭才到,似丞相假殷勤呃哪放心梢。

曹操 唉!

曹操 【西皮摇板】你曾经许下我云阳三报,难道说今日里一次不饶。

关公
【西皮快板】非是我忘却了云阳答报,皆因你这奸曹其罪难逃。在许田射鹿时把君欺藐,挟天子命诸侯势压群僚。逼死了董贵妃其罪非小,董承毙、马腾斩欲谋汉朝。恨不得把奸贼剥皮楦草,

关公 刀来!

关公 【西皮摇板】来来来,试一试偃月钢刀。

曹操 哎呀!

曹操 【西皮摇板】一见君侯变了脸,倒教孟德无话言。往日的恩情无半呃点,

曹操 \textless{}\textbf{哭头}\textgreater{}君侯啊!

曹操 【西皮摇板】杀了我曹孟德你、你、你\ldots{}\ldots{}算不得能员。

曹操 \textless{}\textbf{叫头}\textgreater{}二君侯,二将军!

曹操
想当年在我营中,三日一小宴,五日一大宴,上马黄金,下马白银。临别之时,曾许我三不死。呃,大丈夫需要言而有信呐。今日若放我等逃生,漫说是曹操,就是众将,也感君侯的大恩大------德啊\ldots{}\ldots{}(哭介)

关公
【西皮摇板】往日里杀人不眨眼,铁打心肠软似绵。关某岂是无义汉,宁斩我头挂高竿。叫人来------

(众 哦!)

关公 (接)【西皮摇板】一字长蛇忙开展(或:忙开道),

(众 啊!)

关公 【西皮摇板】认识此阵你快加鞭。

曹操 来。

(曹将 在。)

曹操 向前看看什么阵式呃。

(曹将 乃是一字长蛇大阵。)

曹操 怎么讲?

(曹将 一字长蛇大阵。)

曹操 哦,想是那关云长有放你我逃走之意。趁此机会,咱们溜了罢。

(曹将 逃了罢。(或:呃,逃走了罢。)

曹操 溜了吧。

曹操 【西皮快板】多谢云长开恩典,放俺曹操回中原。三军与爷朝前趱,

曹操 【西皮散板】扭转回头谢美髯。此番回到许昌转,

曹操 周郎啊!

曹操 【西皮散板】重整人马我二下江南。

曹操 走了呃。

(众 曹瞒勿走!)

关公 回营交令呐!

关公
【西皮快板】悔当初错许他云阳答报,今日里徇人情又犯律条。叫小校辕门来通报,就说你爷放奸曹。七星剑将头斫,一腔热血洒战袍。盖世英雄辜负了,

关公 【西皮散板】汗马功劳一旦抛。

\newpage
\hypertarget{ux6218ux957fux6c99}{%
\subsection{战长沙}\label{ux6218ux957fux6c99}}

(李舒遗作 录 刘曾复先生传本,按余叔岩、王凤卿演法)

{[}第一场{]}

\textbf{(大帐、堂桌、印盒、文房。\textless{}发点\textgreater{},四绿龙套执月华旗站门,关羽上。夫子盔、黑三、千金、绿靠褶绿蟒、红彩裤、黑厚底靴。以袖挡脸上,至台口中间)}

\textbf{关羽
{[}引子{]}正气冲霄汉(放袖\textless{}一锣\textgreater{}),秉忠心,保汉疆宏。}

\textbf{(\textless{}发点合头\textgreater{},归内座)}

\textbf{关羽
(念)蚕眉凤目丹赤心,青龙偃月建奇勋。苍天若助三分力,扭转汉室锦乾坤。}

\textbf{关羽
(白)某,汉室关------三弟、四弟取了武陵、桂阳。某奉军师将令夺取长沙。众军校,}

\textbf{众 有!}

\textbf{关羽 听某一令!}

\textbf{(关羽坐)}

\textbf{关羽 【西皮导板】某奉军师将令差,}

\textbf{关羽
【西皮原板】威风凛凛坐将台。旌旗不住空中摆,大小将官逞雄才(或:抖雄才)。正气冲开凌霄汉,文光射入斗、牛开。某家出世英名在,哪把长沙挂心怀。吩咐三军把马带,}

\textbf{(出位、上马,龙套插门下,关收腿)}

\textbf{关羽 (接唱)【西皮原板】一战成功列三台。}

\textbf{(打一下战马,下)}

{[}第二场{]}

\textbf{(黄忠、魏延双起霸}\protect\hyperlink{fn172}{\textsuperscript{172}}\textbf{。黄硬扎巾、白三、黄靠;魏硬扎巾、黑满、紫靠}\protect\hyperlink{fn173}{\textsuperscript{173}}\textbf{。黄大边、魏小边,念)}

\textbf{黄忠 (念)老将年高大,}

\textbf{魏延 (念)镇守在长沙(或:威镇在长沙)。}

\textbf{黄忠 (念)丹心贯日月,}

\textbf{魏延
(念)保主锦中华。}\protect\hyperlink{fn174}{\textsuperscript{174}}

\textbf{黄忠 某,姓黄名忠字汉升(或:某,黄忠)。}

\textbf{魏延 某,姓魏名延字文长(或:某,魏延)。}

\textbf{黄忠 将军请了,}

\textbf{魏延 请了。}

\textbf{黄忠 都督升帐,你我两厢伺候。}

\textbf{魏延 请。}

\textbf{(黄、魏分坐双门椅,黄大边,魏小边。四白龙套站门,堂桌、文房、大座,打上。韩玄纱帽、黪三、缃色蟒或白蟒)}

\textbf{韩玄 {[}引子{]}镇守长沙,秉忠心,扶保乾坤。}

\textbf{(归内大座)}

\textbf{韩玄 (念)执掌兵权印,决策扫烟尘。忠心扶社稷,赤胆保龙庭。}

\textbf{韩玄
本督,韩玄。奉我主之命镇守长沙。今有(或:闻)桃园弟兄前来夺取长沙,不免传(或:与)黄、魏二将进帐,议论迎敌之策。来,传黄、魏二将进帐。}

\textbf{众 都督有令,黄、魏二位将军进帐。}

\textbf{黄忠、魏延 来也!黄忠、魏延告进!(挖进去)参见都督。}

\textbf{韩玄 二位将军少礼,请坐。}

\textbf{黄忠、魏延
谢座。(双跨椅,黄大边,魏小边)传末将(等)进帐,有何军情议论?}

\textbf{韩玄
今有桃园弟兄前来夺取长沙。请二位将军议论迎敌之计(或:请二位将军进帐,商议退敌之策)。}

\textbf{黄忠
启禀都督(或:元帅}\protect\hyperlink{fn175}{\textsuperscript{175}}\textbf{),想长沙、桂阳、零陵、武陵四郡各有军兵把守,何惧桃园(弟兄)!}

\textbf{魏延 但不知何人领兵前来?}

\textbf{韩玄 且听探马一报(,自见分晓)。}

\textbf{(报子上)}

\textbf{报子 报,关羽讨战。}

\textbf{韩玄 再探!}

\textbf{探子 啊!}

\textbf{(探子下。\textless{}冲头\textgreater{}黄、魏站立,撤椅)}

\textbf{韩玄 二位将军,关羽讨战,哪位将军出马?}

\textbf{黄忠
\textless{}叫头\textgreater{}都督,既是关羽讨战,黄忠情愿带领一哨人马(或:俺请出马),生擒那关羽入帐。}

\textbf{魏延
\textless{}叫头\textgreater{}且慢呐!老将军,我想关羽出世以来过五关斩六将,何等威风,人人皆知。诚恐老将军前去,不是他人对手哇!}

\textbf{黄忠 魏将军此言差矣。}

\textbf{魏延 何差?}

\textbf{(黄忠、魏延站)}

\textbf{黄忠
俺老只老头上发,项下须(,胸中韬略却还不老)。有道是:(念)虎老雄心在,这年迈力刚强!}

\textbf{(起\textless{}夺头\textgreater{})}

\textbf{黄忠
【西皮二六】魏将军(或:魏文长)把话错来讲,长他人的威风灭自强。
周室(或:昔日)有个姜吕望,八十三岁遇文王。战国(或:秦国)的姬颜韬略广,他也曾赴会到过(了)湘江。黄忠今年六旬上,杀人妙计腹中(或:腹内)藏。}

\textbf{魏延 老将军!}

\textbf{魏延 【西皮摇板】老将军说话欠思量,}

\textbf{魏延
【西皮快板】某家言来听端详:关羽生来韬略广,千里迢迢保皇娘。过五关曾斩六员将,擂鼓三通斩蔡阳。你今出征阵头上,只恐难胜关云长。}

\textbf{(黄、魏比粗)}

\textbf{韩玄 呃------将军!}

\textbf{(黄、魏分开,面里,不必跪。四上手持枪两边暗上)}

\textbf{韩玄
【西皮摇板】魏将军说话欠思量(或:不必多言讲),黄老将军(或:黄忠近前)听端详:一支将令往下降,命你大战关云长。}

\textbf{黄忠 得令!}

\textbf{黄忠 【西皮摇板】黄忠接令出宝帐,}

\textbf{黄忠 马来!}

\textbf{(黄忠上马,上手插门下)}

\textbf{黄忠 (接唱)【西皮摇板】会一会蒲州关云长。}

\textbf{(黄忠下)}

\textbf{韩玄 将军!}

\textbf{(四下手两边上)}

\textbf{韩玄
【西皮摇板】黄忠接令(或:得令)出宝帐,开言叫声魏文长。四路催粮需谨慎,鞍前马后要提防!}

\textbf{魏延 得令!}

\textbf{魏延 【西皮摇板】元帅(或:都督)将令往下降,}

\textbf{(魏延上马,下手插门下)}

\textbf{魏延 (接唱)【西皮摇板】不分昼夜去催粮。}

\textbf{(魏延下,韩玄出位)}

\textbf{韩玄 (接唱)【西皮摇板】黄忠、魏延出宝帐,}

\textbf{(四白龙套分下)}

\textbf{韩玄 (接唱)【西皮摇板】且听探马报端详。}

\textbf{(韩玄下)}

{[}第三场{]}

\textbf{(\textless{}风入松\textgreater{}头段,四绿龙套、关羽脱蟒穿靠执刀上)}

\textbf{关羽 夺取长沙去者!}

\textbf{(\textless{}风入松\textgreater{}二段,龙套二龙出水会阵,黄忠上,关、黄架住)}

\textbf{黄忠 \textless{}叫头\textgreater{}呔!马前来的敢是关羽?}

\textbf{关羽 既知某姓,何不下马归顺?}

\textbf{黄忠
\textless{}叫头\textgreater{}住口(或:关羽)。你桃园弟兄有多大本领,竟敢前来夺取长沙!}

\textbf{关羽 你且听道哇:}

\textbf{关羽 【西皮导板】勒马停蹄站疆场,}

\textbf{关羽
(脸冲里,拄刀,边转身边用刀指黄忠,接唱)【西皮二六】黄忠老将听端详:某大哥堂堂帝王相,当今的皇叔天下扬。某三弟翼德英雄将,}\protect\hyperlink{fn176}{\textsuperscript{176}}

\textbf{关羽
【西皮快板】大吼一声桥断梁。温酒未寒某华雄斩(或:某破黄巾兵百万),颜良、文丑丧疆场。过五关曾斩六员将,擂鼓三通斩蔡阳。劝你早把长沙让,稍有迟延刀下亡。}

\textbf{黄忠 【西皮小导板】稳坐雕鞍用目望,}

\textbf{黄忠
(接唱)【西皮快板】关公打扮非寻常:丹凤眼、眉蚕样,五绺长髯洒胸膛。(头戴金盔明又亮,身穿铠甲闪秋霜。)胯下赤兔胭脂马,青龙偃月放毫光(或:闪秋霜)。勒住了丝缰把话讲,开言叫声关云长:你夺长沙休妄想,除非你死或我亡(或:你死并我亡)。}

\textbf{(剜萝卜,钻烟筒,一扯两扯,一合两合,一拉转身,关大边,黄小边,架住,往外一绕两绕三绕,往里一绕两绕三绕,关推黄刀,打腰封,\textless{}四击头\textgreater{}亮相。关先下,黄后下}\protect\hyperlink{fn177}{\textsuperscript{177}}\textbf{)}

\textbf{(关羽\textless{}水底鱼\textgreater{}上)}

\textbf{关羽
(\textless{}叫头\textgreater{})且住!黄忠甚是骁勇(或:刀法厉害;或:十分骁勇),(再若来时)拖刀计伤他。}

\textbf{黄忠 哪里走!}

\textbf{(黄上,关用拖刀计,黄抢背落马,跪大边,关小边举刀亮相}\protect\hyperlink{fn178}{\textsuperscript{178}}\textbf{)}

\textbf{黄忠 (呔,)关羽,俺今落马,为何不杀(或:你为何不斩)?}

\textbf{关羽
(黄忠,)关某(或:某家)出世以来,不斩落马之将,回营换马再战。}\protect\hyperlink{fn179}{\textsuperscript{179}}

\textbf{(黄上马,回身一望,下。关一望两望)}

\textbf{关羽 好将!}

\textbf{(关羽挥手,四绿龙套插门下。关打下)}

{[}第四场{]}

\textbf{(四白龙套站门,韩玄上)}

\textbf{韩玄
(念)眼观旌旗起}\protect\hyperlink{fn180}{\textsuperscript{180}}\textbf{,耳听好消息。}

\textbf{(\textless{}水底鱼\textgreater{},黄忠上,下马,挖进门,站大边)}

\textbf{黄忠 末将交令。}

\textbf{韩玄 一旁坐下。}

\textbf{黄忠 谢坐。}

\textbf{(黄忠坐大边,跨椅坐)}

\textbf{韩玄 老将军胜负如何?}

\textbf{黄忠 (末将出兵,)两军阵前不分胜负。明日定要生擒关羽入帐。}

\textbf{韩玄 好,听本督一令:}

\textbf{韩玄
【西皮散板】本帅帐中(或:本督帐前)传令号,黄忠老将听根苗:你若擒得关羽到,凌烟阁上美名标。}

\textbf{(四白龙套下,韩玄下)}

\textbf{黄忠 得令。}

\textbf{(黄忠接令,转身站台中间)}

\textbf{黄忠 【西皮散板】黄忠接令出宝帐,}

\textbf{(黄忠出门)}

\textbf{黄忠 【西皮散板】背转身来自参详:}

\textbf{黄忠
(\textless{}叫头\textgreater{})且住!想某(或:适才)在两军阵前被关羽打下(或:挑下)马来,不忍杀害(或:伤害)于我。想俺这百步穿杨百发百中,我若暗地陷害于他(或:我若暗箭伤害于他),岂不被天下英雄耻笑?也罢!明日两军(或:明日去至)阵前,(我不免)只射盔缨不射咽喉,以报阵前不杀之义(或:恩)也。}

\textbf{黄忠 【西皮散板】明日战场来会阵,}

\textbf{黄忠 【西皮散板】百步穿杨射盔缨。}

\textbf{(\textless{}抽头\textgreater{},黄忠下)}

\textbf{(\textless{}长锤\textgreater{}四绿龙套站门上,关羽上)}

\textbf{关羽 【西皮散板】黄忠老将(或:老儿)失了计,}

\textbf{关羽
【西皮快板】他与某家比高低。我把黄忠好一比,绵羊遇虎把头低。(或:中了某家拖刀计,不忍杀之放他回。)将身且坐虎榻椅,(或:某家不杀放他回,将身且坐宝帐里,)}

\textbf{(关羽外场椅小座)}

\textbf{关羽 【西皮散板】且听探马报端的。}

\textbf{探子 报!黄忠讨战。}

\textbf{关羽 再探。}

\textbf{(关羽站起)}

\textbf{关羽 【西皮散板】三军(或:人来)带过赤兔骑。}

\textbf{(\textless{}扫头\textgreater{},关拿刀上马,四绿龙套下,关站大边,黄上站小边)}

\textbf{关羽 黄忠,昨日饶尔不死,你又来则甚?}

\textbf{黄忠 今日一定要与你决一死战。}

\textbf{关羽 口出大言,终何用耳(或:中何用了)。放马过来。}

\textbf{(关黄开打,一合,关小边,黄大边,一拉转身,架住往里一绕两绕三绕,,往外一绕两绕三绕,黄压关刀,关撤刀,用刀鐏盖黄刀头,鼻子,削头。黄败下,关追下)}

\textbf{(\textless{}扭丝\textgreater{},四白龙套站门,韩玄骑马上)}

\textbf{韩玄
【西皮散板】催马加鞭到战场,观看两军比刚强。下得马来敌楼上,}

\textbf{(韩玄下马,四白龙套下,韩玄上城)}

\textbf{韩玄 (接唱)【西皮散板】旌旗招展尘飞扬(或:土飞扬)。}

\textbf{(黄忠执弓箭,\textless{}扭丝\textgreater{}上)}

\textbf{黄忠 【西皮散板】催马加鞭战场到,关公追赶不肯饶。
(或:关公刀法真奥妙,念他不斩恩义高。)箭射盔缨把恩报,}

\textbf{(关上,黄放箭,关接箭,黄下)}

\textbf{关羽
(\textless{}扭丝\textgreater{}接唱)【西皮散板】接过雕翎箭一条。}

\textbf{(关拿箭到台口)}

\textbf{关羽
(\textless{}凤点头\textgreater{}接唱)【西皮散板】明知深山藏虎豹,大胆单身去採樵。}

\textbf{(关用刀头拨箭到小边台口,回身看箭,下。黄忠上)}

\textbf{黄忠
(\textless{}扭丝\textgreater{})【西皮散板】某家箭法人知名,只为报答(或:答报)不斩恩。二次盔缨射得准,}

\textbf{(黄放箭,关上接箭,黄下,关把箭扔到大边台口)}

\textbf{关羽
(\textless{}扭丝\textgreater{}接唱)【西皮散板】接过雕翎箭二根。恼恨黄忠欺人甚,放箭哪有接箭能。(或:黄忠不伤某性命,箭射盔缨为何情?)}

\textbf{(关下。黄\textless{}水底鱼\textgreater{}上)}

\textbf{黄忠
(\textless{}叫头\textgreater{})且住!关羽(或:关公)不解其意,后面紧紧跟随(或:不知进退;或:紧紧追赶),如何是好?也罢,待某磕掉(或:去)箭头伤他一箭,惊吓于他。呔,看箭!}

\textbf{(黄下。关上接箭。四绿龙套两边上)}

\textbf{关羽
(\textless{}叫头\textgreater{})且住!黄忠百步穿杨百发百中,连射三箭,只射盔缨不射咽喉是何意也?嚯嚯是了(或:唔),想是黄忠有降顺桃园之意,军士们(或:众军校),将长沙团团围住者。}

\textbf{(\textless{}风入松\textgreater{}三段,堂鼓\textless{}急急风\textgreater{},四绿龙套抄过合,分下,关往里扔箭,关下)}

\textbf{韩玄 哎呀!}

\textbf{韩玄
(\textless{}扭丝\textgreater{})【西皮散板】敌楼之上看分明,黄忠老儿起反心。三军带路宝帐进,}

\textbf{(韩下城,拉幕换堂桌。四白龙套上挖门,韩上,一番两番,下马,进门入大座)}

\textbf{韩玄 (接唱)【西皮散板】快传老将黄汉升。}

\textbf{(黄忠上)}

\textbf{黄忠
(\textless{}扭丝\textgreater{})【西皮散板】辕门下马心不定,}

\textbf{(黄下马,挖进门站大边,韩怒视,拍桌子)}

\textbf{黄忠 (接唱)【西皮散板】都督发怒为何情?}

\textbf{黄忠 都督,为何发怒?}

\textbf{韩玄
我来问你(或:我且问你),你的箭法百步穿杨百发百中,今日连射三箭,为何只射盔缨,不射咽喉,是何意也?}

\textbf{(黄忠 这\ldots{}\ldots{})}

\textbf{(韩玄 讲!)}

\textbf{黄忠
(\textless{}叫头\textgreater{})都督!末将昨日在两军阵前,被关羽挑下马来,是他不忍伤害于我(或:末将昨日在两军阵前,马失前蹄,关羽不忍伤害于我),故而末将今日只射盔缨,不射咽喉,以报(昨日阵前)不杀之意也!}

\textbf{韩玄 你待怎讲(或:怎么讲)?}

\textbf{黄忠 以报(昨日阵前)不杀之意也。}

\textbf{韩玄 住口。}

\textbf{(黄忠面朝里跪中间)}

\textbf{韩玄
(\textless{}扭丝\textgreater{}接唱)【西皮散板】听一言来怒气生,老贼不该起反心。吩咐两旁刀斧手,绑出(或:推出)辕门问斩刑。}

\textbf{(黄忠转身向外屁股坐子,两刀斧手上,给黄上绑)}

\textbf{黄忠 【西皮散板】情愿一死仁义尽,岂肯做那无义人。}

\textbf{(\textless{}叭嗒仓\textgreater{}、\textless{}冲头\textgreater{},黄、刀斧手出门下)}

\textbf{韩玄 (接唱)【西皮散板】滚木擂石安排定,}

\textbf{(韩玄出位,四白龙套下)}

\textbf{韩玄 (接唱)【西皮散板】等候魏延定计行。}

\textbf{(韩玄\textless{}抽头\textgreater{}下)}

{[}第五场{]}

\textbf{(\textless{}长锤\textgreater{},四下手执车旗站门,魏延换箭衣、马褂上)}

\textbf{魏延 【西皮摇板】某家奉了都督命,解押(或:押运)粮草转回营。}

\textbf{魏延
某(或:俺),魏延。奉了都督{之命}(或:将令),押运粮草军前{需用}(或:听用)。粮草催齐回营交令。军士们,}

\textbf{众 有!}

\textbf{魏延 催军!}

\textbf{魏延 (接唱)【西皮摇板】三军押粮(或:与爷)往前进。}

\textbf{(下手插门下)}

\textbf{魏延 (接唱)【西皮摇板】见了都督说分明(或:问军情)。}

\textbf{(打下)}

{[}第六场{]}

\textbf{(黄忠换红箭衣,戴条子)}

\textbf{黄忠 (内)【西皮导板】将令(或:号令)一出绑帐口,}

\textbf{(二刀斧手上,黄忠\textless{}四击头\textgreater{}上至九龙口)}

\textbf{黄忠 【西皮原板】汗马功劳一笔勾。}

\textbf{(扯正,扯四门)}

\textbf{黄忠
(接唱)【西皮原板}\protect\hyperlink{fn181}{\textsuperscript{181}}\textbf{】桃园弟兄来争斗,一来一往统貔貅。误中了(或:某中了)关公拖刀计,蒙他不斩把我留(或:将我留;或:把情留)。都只为他人情谊(或:恩义)厚,百步穿杨把恩酬,都督一见冲牛、斗,绑出了辕门(或:绑出了营门)要斩头。移步儿来在营门首,}

\textbf{(黄忠坐大边门椅,二刀斧手站身后)}

\textbf{黄忠 (接唱)【西皮散板】到此时我只得气忍咽喉。}

\textbf{(四下手``一条鞭''上,魏延\textless{}长锤\textgreater{}上)}

\textbf{魏延 【西皮散板】来在营门下走兽,}

\textbf{(魏延下马,四下手下,魏延看)}

\textbf{魏延 啊?!}

\textbf{魏延 (接唱)【西皮散板】老将军醒来问根由。}

\textbf{魏延 老将军醒来!}

\textbf{黄忠 【西皮小导板】法场上绑得我如醉酒,}

\textbf{魏延 老将军!}

\textbf{黄忠
(\textless{}凤点头\textgreater{})【西皮散板】抬头只见魏参谋。}

\textbf{魏延
(接唱)【西皮快板】老将军身犯何罪由?快对某家(或:快与某家)说从头。}

\textbf{黄忠
(\textless{}三锣\textgreater{})【西皮快板】都只为龙争并虎斗,两军阵前运机谋。误中关公拖刀计,蒙他不杀将我留。都只为他人情谊(或:恩义)厚,箭射盔缨把恩酬。进帐去不容我开口,绑出(了)辕门要斩头。}

\textbf{魏延 (接唱)【西皮快板】老将军不必心担忧,末将进帐把情求。}

\textbf{黄忠
(接唱)【西皮快板】你与韩玄长沙守,不可(或:休要)为我结冤仇。}

\textbf{魏延
(接唱)【西皮快板】他若是(或:倘若是)人情来准下,万般事儿一旦丢。韩玄若是不罢手,定教老贼一命休。吩咐两旁(或:开言叫声)刀斧手,你把老将留一留。}

\textbf{(\textless{}叭嗒仓\textgreater{}\textless{}扭丝\textgreater{},魏延下,黄忠站)}

\textbf{黄忠 (接唱)【西皮散板】一见魏延(或:魏延讲情)进帐口,}

\textbf{(二刀斧手扯斜)}

\textbf{黄忠 (接唱)【西皮散板】法场之上我担忧(或:心忧)。}

\textbf{(二刀斧手押黄忠下)}

{[}第七场{]}

\textbf{(\textless{}扭丝\textgreater{}四白龙套站门,韩玄上)}

\textbf{韩玄
【西皮散板】闷坐帐中(或:斩了黄忠)心不定(或:神不定),眼跳心惊为何情?闷恹恹坐在大堂(或:坐在宝帐;或:且坐宝帐)等,}

\textbf{(韩玄归大座)}

\textbf{韩玄 (接唱)【西皮散板】魏延到来定计行。}

\textbf{(魏延上)}

\textbf{魏延 (接唱)【西皮散板】将身且把(或:怒气不息)宝帐进,}

\textbf{(魏延挖进,到大边)}

\textbf{魏延
(接唱)【西皮散板】见了都督来求情(或:讲人情)。(\textless{}住头\textgreater{})}

\textbf{魏延 末将交令。}

\textbf{韩玄 将军请坐。}

\textbf{魏延 谢座。}

\textbf{(\textless{}五击头\textgreater{},魏延坐大边跨椅)}

\textbf{韩玄 粮草可曾催齐?}

\textbf{魏延 俱已催齐,都督查点。}

\textbf{韩玄 将军之功也!}

\textbf{(魏延一望两望)}

\textbf{韩玄 将军你看些什么?}

\textbf{魏延 请问都督,黄老将军哪里去了?}

\textbf{韩玄  老贼起了降刘之心,绑赴法场问斩去了。}

\textbf{魏延
啊,都督斩了黄忠,不值紧要(或:要紧),桃园弟兄兴兵前来,如何是好(或:何人出马)?}

\textbf{韩玄 自然是将军出马呀。}

\textbf{魏延
(哼哼,)赦了黄忠,我便出马;不赦黄忠,哼,我就不管你的闲事了。}

\textbf{(韩玄 你待怎讲?)}

\textbf{(魏延 不管你的闲事了。)}

\textbf{韩玄 (我)定斩不赦。}

\textbf{魏延 你待怎讲?(或:怎么讲?)}

\textbf{韩玄 定斩不赦。}

\textbf{魏延 啊?!(或:呀呸!)}

\textbf{(\textless{}扭丝\textgreater{},魏延站起)}

\textbf{魏延
【西皮散板】放了(或:赦了)黄忠我便罢,不放(或:不赦)黄忠我的怒气发!}

\textbf{韩玄 唗!(或:大胆!;或:住口!)}

\textbf{韩玄
【西皮散板】骂声魏延真胆大,敢在帐前(或:帐中)乱军法。快将魏延来拿下,推出(或:绑至)辕门把他杀(或:将他杀)!}

\textbf{(\textless{}崩登仓\textgreater{},魏延站台口打``哇呀呀'')}

\textbf{魏延
(\textless{}扭丝\textgreater{}接唱)【西皮散板】听一言来怒气发,不由某家咬钢牙。两旁儿郎一起杀,}

\textbf{(四白龙套杀死下。韩玄抱印下,魏延追下,韩玄抱印上,一举两举,魏延一漫头,两漫头,回身刺韩玄倒,魏延拿印)}

\textbf{魏延
(\textless{}扭丝\textgreater{}接唱)【西皮散板】看你饶他(\textless{}叭嗒、仓、仓、仓\textgreater{},魏用剑打``彩头''三下)不饶他。}

\textbf{(\textless{}叭嗒仓\textgreater{}亮相,\textless{}一锤锣\textgreater{}下)}

{[}第八场{]}

\textbf{(二刀斧手、黄忠下场门上,\textless{}扭丝\textgreater{},黄坐大边台口)}

\textbf{黄忠
【西皮散板】魏延进帐时已久,为何一去不回头。眼观日落西山后(或:眼观红日疾行走),法场一刻似千秋。}

\textbf{(魏延上)}

\textbf{魏延
(\textless{}扭丝\textgreater{})【西皮散板】开刀先杀刽子手(或:先杀两旁刀斧手),}

\textbf{(魏延杀刀斧手)}

\textbf{魏延
(\textless{}凤点头\textgreater{}接唱)【西皮散板】再与老将说从头(或:老将军醒来听从头)。(\textless{}住头\textgreater{})}

\textbf{魏延 老将军醒来,老将军醒来。}

\textbf{黄忠 【西皮散板】霎时(或:一时)昏迷如梦走,}

\textbf{魏延 老将军醒来。}

\textbf{黄忠 【西皮散板】再与将军说从头。}

\textbf{(黄忠站大边,魏延小边)}

\textbf{黄忠
将军进帐讲情,都督可准(或:都督可曾应允)?(或:魏将军,你这是何意呀?)}

\textbf{魏延 老贼不准,是我将他杀死了!}

\textbf{黄忠 我却不信。}

\textbf{魏延 首级在此,拿去看来(或:人头在此。你且看来)。(黄接``彩头'')}

\textbf{黄忠 哎呀!(拿``彩头''在台口一对)}

\textbf{黄忠
(\textless{}扭丝\textgreater{})【西皮散板】一见人头珠泪滚,{怎不叫人痛伤情}}(\textbf{或:}点点珠泪痛伤情)\textbf{。(哭、哭一声韩太守,我叫、叫一声{忠良臣}(或:韩大人)。)可叹你为国家丧了(\textless{}哭头\textgreater{})命。}

\textbf{魏延 你拿过来罢!}

\textbf{(魏延抢彩头)}

\textbf{黄忠 (接唱)【西皮散板】回头再叫魏将军,你我同把后堂进!}

\textbf{(黄拉魏欲往里进)}

\textbf{魏延 哪里去?}

\textbf{黄忠 (接唱)【西皮散板】后堂去见韩夫人。}

\textbf{魏延 嘿嘿!都被我杀光了(或:也被我杀了)哇!}

\textbf{黄忠 哎呀!}

\textbf{黄忠 (接唱)【西皮散板】你我同把许昌进(或:许昌奔)。}

\textbf{(黄拉魏欲往外(台口)去)}

\textbf{魏延 哪里去?(或:做什么?)}

\textbf{(放手,回来。黄、魏八字站,黄大边,魏小边)}

\textbf{黄忠
(接唱)【西皮散板】魏王台前{领罪名}(或:请罪名)。}\protect\hyperlink{fn182}{\textsuperscript{182}}

\textbf{魏延 哎呀老将军,某家(或:我)把你好有一比。}

\textbf{黄忠 比作何来?}

\textbf{魏延 咸鱼放生。}

\textbf{黄忠 此话怎讲?}

\textbf{魏延 你连死活都不知道了。(你我不去逃生,反来送死不成?)}

\textbf{黄忠 依将军之见?}

\textbf{魏延 (以我之见,)你我{归顺}(或:投降)桃园弟兄岂不是好。}

\textbf{黄忠 怎么讲?}

\textbf{魏延 归顺桃园。}

\textbf{黄忠 嗯------(要去)你去,我不去。}

\textbf{魏延 你不去?(或:当真不去?)}

\textbf{黄忠 我不去。}

\textbf{魏延 你若不去,嘿嘿,我就是一刀杀了你呀!}

\textbf{(魏漫黄头,欺黄)}

\textbf{黄忠 哎!}

\textbf{黄忠 【西皮散板】长沙的儿郎散了队,}

\textbf{魏延 (接唱)【西皮散板】东逃西奔各自归。}

\textbf{黄忠 (接唱)【西皮散板】你做此事悔不悔?}

\textbf{魏延 (接唱)【西皮散板】事到临头(或:事到头来)埋怨谁。}

\textbf{黄忠 (接唱)【西皮散板】我哭,哭一声韩太守,}

\textbf{魏延 呔!我不准(或:我不教)你哭!}

\textbf{(黄忠 (接唱)【西皮散板】叫,叫一声韩元戎啊\ldots{}\ldots{})}

\textbf{魏延 我不许你嚎!}

\textbf{黄忠 (接唱)\textless{}哭头\textgreater{}啊\ldots{}\ldots{}}

\textbf{(魏延 我看你们哪一个敢嚎!)}

\textbf{黄忠 魏延!}

\textbf{黄忠 (\textless{}哆啰\textgreater{}接唱)【西皮散板】你这冒失鬼!}

\textbf{魏延 走,走。}

\textbf{(\textless{}冲头\textgreater{},二人往里转身,往外转身,魏轰黄幺二三,(换双楗子),黄托胡子,扔胡子,下)}

{[}第九场{]}

\textbf{(四绿龙套上,关羽上)}

\textbf{关羽 【西皮摇板】奉令夺取长沙地,}

\textbf{关羽 【西皮快板】黄忠箭法果然奇,一统汉室三分鼎,}

\textbf{(关羽归坐外场椅)}

\textbf{关羽 (接唱)【西皮快板】扶保兄王锦华夷。}

\textbf{(报子上)}

\textbf{报子 报!黄忠、魏延辕门投降!}

\textbf{关羽 知道了。升帐!}

\textbf{(关羽归坐内场椅)}

\textbf{报子 升帐。}

\textbf{(报子下)}

\textbf{关羽 架起刀门,传黄忠、魏延进帐!}

\textbf{众 黄忠、魏延进帐!}

\textbf{(\textless{}长锤\textgreater{}黄忠、魏延上,亮弦,\textless{}闪锤\textgreater{},小边台口)}

\textbf{黄忠 【西皮摇板】来在辕门(或:营门)用目觑,}

\textbf{魏延 【西皮摇板】刀枪剑戟摆得齐。}

\textbf{黄忠 【西皮摇板】我不归降转回去,}

\textbf{(魏延拦)}

\textbf{魏延 【西皮摇板】你不归降我不依。}

\textbf{黄忠 【西皮摇板】进得帐来屈膝跪,}

\textbf{(黄忠、魏延双挖门,跪,黄在大边,魏在小边)}

\textbf{黄忠、魏延 【西皮摇板】黄忠、魏延归降迟。}

\textbf{关羽 【西皮摇板】丹凤眼来观仔细,}

\textbf{关羽
【西皮快板】只见二将跪丹墀,你今归降因何意(或:尔等归降从何起;或:你今归降因何起)?一一从头说端的。}

\textbf{黄忠 【西皮摇板】只为韩玄不仁义,}

\textbf{魏延 【西皮摇板】要斩老将命归西。}

\textbf{黄忠 【西皮摇板】今日归降桃园地,}

\textbf{魏延 【西皮摇板】赤胆忠心永不移。}

\textbf{(魏延献``彩头'',关看,手一摆)}

\textbf{关羽
【西皮摇板】一见人头心惨凄,只因为国血染衣(或:命归西)。人头悬挂辕门地(或:辕门里),}

\textbf{(龙套回身}\protect\hyperlink{fn183}{\textsuperscript{183}}\textbf{)}

\textbf{关羽 【西皮摇板】二位可算将中奇。}

\textbf{(关羽出位)}

\textbf{关羽 (接唱)【西皮摇板】下得位来搀扶起,}

\textbf{(关羽搀二将,黄坐大边,魏坐小边,关坐中间)}

\textbf{关羽 (接唱)【西皮摇板】兄长到此把功提。}

\textbf{关羽 二位将军请坐。}

\textbf{黄忠、魏延 谢坐。}

\textbf{(黄忠大边,魏延小边,两边跨椅,关羽中间)}

\textbf{关羽 昨日阵前交战(或:交锋),老将军果然好刀法也。}

\textbf{黄忠
二君侯夸奖了。(或:二君侯刀法神妙。或:二君侯的刀法黄忠不及!)}

\textbf{关羽 岂敢,(或:休得过谦。啊,)昨日阵前为何不见魏将军?}

\textbf{魏延
末将奉命押解粮草,为此不在军前(或:军中),(现有)长沙印信(或:信印)呈上。}

\textbf{(魏交印,关拿印放袖内)}

\textbf{关羽 兄长到此必有封赠。}

\textbf{(内(白,搭架子) 主公(或:主上)到。)}

\textbf{关羽 二位将军暂退,}

\textbf{黄忠、魏延 是。}

\textbf{(黄忠、魏延下。黄先魏后)}

\textbf{关羽 有请!}

\textbf{(黄、魏应是下场门下。\textless{}吹打\textgreater{}四红龙套上``一条鞭'',诸葛亮、刘备上,挖进门。刘坐中,孔明大边,关羽小边,外场椅)}

\textbf{刘备 恭喜二弟(,贺喜二弟),一战成功,可喜可贺!}

\textbf{关羽 此乃兄长(或:兄王)洪福,先生妙算。小弟何功之有?}

\textbf{诸葛亮 二千岁虎威。(\textless{}撕边一锣\textgreater{})}

\textbf{关羽 小弟收得降将黄忠、魏延,现在帐外(候令)。}

\textbf{诸葛亮 二将进帐!}

\textbf{众 二将进帐!}

\textbf{(\textless{}冲头\textgreater{},黄忠、魏延上,挖门进)}

\textbf{黄忠、魏延 参见主公。}

\textbf{刘备 少礼,见过先生。}

\textbf{黄忠、魏延 参见先生。}

\textbf{诸葛亮 老将军(请起,)后帐歇息。}

\textbf{黄忠 谢先生。}

\textbf{(黄忠\textless{}小锣五锤\textgreater{}下)}

\textbf{诸葛亮 来,将魏延推出斩了(或:将魏延绑了)!}

\textbf{关羽 且慢!啊,先生,为何将魏延斩首(或:绑了)?}

\textbf{诸葛亮 魏延脑后有一反骨,故而将他斩首。}

\textbf{关羽
啊,先生,若是斩了魏延,只恐天下英雄道俺桃园(弟兄)就不义了!}

\textbf{刘备 着哇!}

\textbf{诸葛亮
日后魏延若有反意,休怪山人。来,将魏延解下桩来(或:将魏延松绑)。}

\textbf{关羽 魏延松绑。}

\textbf{魏延 多谢军师(或:先生)不斩之恩!}

\textbf{诸葛亮
非是山人不斩于你,此乃二君侯讲情,从今以后莫要离开山人左右,违令者斩。(站起)你要小心了!}

\textbf{魏延 是。}

\textbf{诸葛亮 你要打点了!}

\textbf{魏延 是。}

\textbf{诸葛亮 下去!}

\textbf{魏延 喳,喳,喳。(魏出门)嘿!}

\textbf{(\textless{}冲头\textgreater{}魏延下)}

\textbf{关羽 (现有)长沙印信(或:信印)献上。}

\textbf{(关羽左手拿印右手扶,站台口,刘、诸葛站,刘接印交孔明)}

\textbf{刘备 后帐摆宴,与贤弟贺功!}

\textbf{关羽 请驾。(或:请------)}

\textbf{(\textless{}尾声\textgreater{} 同下)}

\textbf{注:《战长沙》传统经典剧目,程长庚、余三胜、张二奎、汪桂芬、王鸿寿均擅演。}

《战长沙》对刀和接箭(王凤卿演法)

王凤卿演关公戏除用胭脂揉个淡红脸之外,其它与一般老生没有什么特殊分别,这与钱金福演周仓戏与一般武二花一样,不使``判儿''身段的情况相似。王的关公戏身段稳重,把子简捷。

他的《战长沙》对刀比较简单:

\textbf{第一场}开打是黄忠唱毕剜萝卜,架住,钻烟筒,一扯两扯,一合两扯(\textbf{不是大刀花过合}),回大边,刀头一拉转身架住,往外一二三绕,往里一二三绕,架住用刀头把黄刀推出去,面向里斜身回头望黄,扁刀先下。

\textbf{第二场}开打是关念``放马过来''一合过小边,刀头一拉转身架住,往里一二三绕,往外一二三绕,在外边架住,用鐏盖黄刀头,打鼻子,捋胡子不转身削黄头,亮,掠刀追下。

\textbf{三场}拖刀计,是原地漫头,转身从里边褪到小边抱刀亮(黄忠从外边翻)。

三次接箭使人有细致寓于平凡之中的感觉:

\textbf{第一箭}是黄放箭时,关上场门上,听见弓弦小锣声立即左手抄箭,右手用刀挡,先不向前大走,边走边在锣鼓中看箭,在九龙口站住,接唱``接过雕翎箭一条'',在锣鼓中走到台中间看一下箭站住,唱,唱完撤步到下场门一边再看箭,用刀头一绕箭把箭拨到上场门外边台上,再面向里胸前斜着横刀,回头望一下箭,扁刀下。

\textbf{第二箭}接箭法同上,接往后一看,在锣鼓中边走边把箭扔到下场门一边上,一直走到台中间,唱,唱得尺寸较快,唱毕刀画圈掠刀追下。

\textbf{第一次表示一惊后沉着追赶},\textbf{第二次是激怒拍马紧追}。

\textbf{第三箭}射中盔缨,又一惊,用手扶住箭走到台中间再拔下箭来看箭起叫头念,龙套两边上

念完龙套两边抄下,关下。

\newpage
\hypertarget{ux9ec4ux9e64ux697c-ux4e4b-ux5218ux5907}{%
\subsection{黄鹤楼 之
刘备}\label{ux9ec4ux9e64ux697c-ux4e4b-ux5218ux5907}}

{[}第一场{]}

{[}引子{]}义得人和,灭孙曹,孤心安乐。

(念)日月重明照英雄,全凭卧龙建奇功。虽得土地归王化,未能遂意高祖风。

孤,刘备,大树楼桑人氏。自与关、张结义桃园,三顾茅庐,请来卧龙先生,屡建奇功。客荆虽得安顿,只为孙、曹未得安定,叫孤常忧心也。正是:(念)苍天遂孤意,重整汉帝基。

罢了,进帐何事?

呈上来。东吴有书信到来,待孤拆开一观。

有请先生。

先生少礼,请坐。

东吴有书信到来,先生请来观看。

此番过江,那东吴是好意,还是歹意?

既然如此,孤就不去了。

先生计将安在?

四弟少礼,见过先生。

坐下,先生有差。

呃,慢来,慢来,前番去至东吴,就是我君臣二人,险些命丧周郎之手;此番又是我君臣二人。要去你去,孤是不去的了!

【西皮原板】先生把话错来讲,休提起当年赴会河梁。孙、刘仇结山海样,孤岂肯把性命送与周郎。

【西皮摇板】他二人把话一样讲,倒教孤王少主张。回头便对先生讲,孤王言来听端详。倘若孤王东吴丧,引孤的灵魂入庙廊。

去,孤便去。

还是多带人马才是啊。

四弟,打开一观。

哪里是不灵,分明是孤的引魂幡喏!

呃,迎孤的灵魂吧!

【西皮摇板】好个大胆诸葛亮,勒逼孤王过长江。虎穴龙潭孤去闯,

【西皮散板】你分明是送孤王去见阎王。

{[}第二场{]}

啊,都督,备过江来了。

都督请。

四弟子龙。

{[}第三场{]}

有坐。

远隔大江,少来问安,都督海涵。

为何不见吴侯?

告便。

进宫问安。

谨遵台命。

(周瑜 (念)相逢花中锦,)

(念)知己叙衷肠。

{[}第四场{]}

呃,大夫,备过江来了。

呵呵哈哈哈\ldots{}\ldots{}

有劳大夫。

大夫请便。

啊,啊,啊,都督请。

都督有何金言,当面请讲。

啊\ldots{}\ldots{}

呃,这\ldots{}\ldots{}

唉,都督哇,呃\ldots{}\ldots{}(哭介)

(周瑜 又来了。)

【西皮原板】周都督他那里提前情,倒教我汉刘备有话难云。借荆州取西川以为根本,望都督禀吴侯再等几春。

放肆。

下站。

【西皮摇板】四弟做事太莽撞,恶言恶语把人伤。周都督他倒有容人量,

都督,四弟莽撞,备这厢赔礼了。

备这厢赔礼了。

哎,都督哇!

【西皮摇板】还望都督好商量。

诸葛亮啊,害死孤王也。

【西皮摇板】勒逼孤王把宴饮,黄鹤楼上遇杀星。周郎苦苦要孤命,

四弟。

【西皮摇板】想一良谋好逃生。

四弟,有何妙计?

又是他那长坂坡!

四弟,在长坂坡前,你胯下有马,掌中有枪;今日在这黄鹤楼上,难道说你拳打------足踢------不成?

那是妖道的谣言呐。

``水军都督周''。

嗯哼,真乃是孤的好先生!

四弟,搀孤下楼。

告辞了。

\textbf{甘露寺 之 乔玄}

\textbf{{[}第一场{]}}

(刘备 看------江水波涛,水天一色,好一派江景也!)

(刘备
【西皮原板】看长江白茫茫银蛇滚滚,水与天共一色白浪纷纷。回头来再对四弟论,此一番到东吴见机而行。)

\ldots{}\ldots{}

(刘备 四弟,准备厚礼。明日你我君臣前去拜访。)

(刘备 正是:(念)来到东吴地,)

(赵云 (念)先去见乔玄。)

\textbf{{[}第二场{]}}

\textbf{呃------哼!(}内嗽介\textbf{)}

{[}引子{]}丹心镇国,辅君王,社稷安康。

(念)天子渊源重老臣,为子孝亲臣奉君。皇图永固民安乐,但愿东吴万万春。

老夫------乔玄,字嵩山,乃江东人氏。吴侯驾前为臣,官居首相,执掌江东十二内阁。夫人姜氏,膝下无儿,所生二女,长女大乔,许配孙策;次女小乔,配许周郎。适才朝罢归来,见街市之上,悬灯结彩;府下人等,一个个交头接耳,也不知他们说些甚么。

啊------家院。

老夫问你话呀。

老夫问你话呀。

适才老夫朝罢而归,见街市之上,悬灯结彩;府下人等,一个个交头接耳,不知他们说些么?

孙、刘两家结亲?呃,怎么老夫一些儿也不晓得呀!

哦,有这等事?老夫当朝首相,怎么一些儿也不知呢。

(思忖介)既是刘皇叔过江,也该前来拜拜老夫啊。

好好好,你且门上伺候!

(刘备 (念)身在东吴地,)

(赵云 (念)昼夜费心机。)

哦,果然来了。动乐有请。

啊------皇叔!

过江来了。

啊------呵呵呵哈哈哈\ldots{}\ldots{}(笑介)

请------

请坐。

皇叔驾到,蓬荜生辉。老朽有失远迎,望祈恕罪。

岂敢。

哦,罢了。

皇叔,此位是------

哦?!这就是在长坂坡前救幼主的子龙将军么?

真乃是虎将也。

哎呀呀,老朽怎敢受礼,万难从命。

呃,不、不,不敢收啊。

哎,老夫还未曾吩咐,你怎么就收下了?

好不中用!\\
呃呃呃,皇叔,如此我愧领了!

为何去心太急?

是啊,他们那里也该走走啊,只是老朽未得领教。

另日奉迎。

好,送客。

呃------嗯,皇叔到此,乃是贵客,我不肯收他的礼物,怎么你就大胆地收下了啊?

有道是:无功不受禄哇。

呃,我功在哪里?

呵呵呵,你这老狗才的话,倒也中听。

呵!

哎呀且住!刘备既已过江,孙、刘两家若能结亲,一同出兵,共敌曹操,与我东吴大大有利。我不免进宫,与太后贺喜。

来,吩咐外厢打道进宫!

\textbf{{[}第三场{]}}

\textbf{呃------哼!(}内嗽介\textbf{)}

(念)天上生瑞彩,人间配鸾凰。

来此已是,待我叩环!

乔玄求见。

领旨。

臣------乔玄见驾,国太千岁!

千千岁。

谢座。

呃,恭喜太后,贺喜太后!

太后将郡主招赘刘备,岂不是一喜?

这样的大事,太后不知,谁敢作主?

(思介)呃------莫非二千岁\ldots{}\ldots{}主意。

领旨!

太后有旨,二千岁进宫!

老臣参驾。

谢座。

太后醒来!

啊,千岁,若用此计,岂不被旁人耻笑么?

怎么?!又是周郎?

唉!他明明是害你呀。

呵呵\ldots{}\ldots{}我多口,多口哇\ldots{}\ldots{}

(孙权   【西皮原板】\ldots{}\ldots{}誓不休!)

千岁!

【西皮原板】劝千岁杀字休出口,细听老臣说根由:那刘备他本是------【转西皮二六】靖王后,【西皮快板】汉帝玄孙一脉流。他有个二弟关羽汉寿亭侯,青龙偃月神鬼皆愁。他斩颜良、诛文丑,古城又斩蔡阳的头。他三弟翼德性情有\protect\hyperlink{fn184}{\textsuperscript{184}},丈八蛇矛惯取咽喉。虎牢关前来争斗,枪挑金冠战温侯。当阳桥前一声吼,喝断桥梁水倒流。他四弟赵云常山将,盖世英名贯九州。长坂坡,救阿斗,杀得曹兵个个愁。这班武将哪国有?还有诸葛运计谋。杀了刘备不要紧,荆州岂肯来罢休?若是兴兵来争斗,曹操坐把渔利收。扭转回身启太后,老臣言来听从头:龙凤呃呈祥天造就,将计就计结鸾俦。

太后,孙、刘若能结亲,\protect\hyperlink{fn185}{\textsuperscript{185}}一同出兵,共灭曹操,与我东吴大大有利,不可失此机会也。

可以配得。

国太若相得上?

啊,太后,那刘备乃英雄之相,不相也罢。

领旨!

\textbf{{[}第四场{]}}

唉!明日太后在甘露寺面相刘备,我想刘备须发苍白,太后若相他不上,必被周郎所害,唉呀,这这这这\ldots{}\ldots{}

唉!他人闲事,不管也罢呀。

呃------都是你这个老狗才,我不肯收他礼物,你就大胆地收下了,如今岂不叫老夫作难么?

是啊,总要想个计策才是啊------

哦,有了!

乔福过来。这有乌须药一匣,命你送到馆驿,面交刘皇叔,教他连夜将须发染黑,明日在甘露寺中一相就相上了,快去快去!

啊\ldots{}\ldots{}转来!

对刘皇叔去说:明日席前,恐其有诈,命保驾将军内穿铠甲,外罩袍服,作一个``防而不备,备而不防''!

记下了。

快去快去。

这个老狗才。

唉\ldots{}\ldots{}从今以后,老夫再也不贪人家的小利了。

唉,这才是``不经一事,不长一智''哦!

\textbf{{[}第五场{]}}

刘皇叔到。

是。

啊------皇叔。

上面就是太后,见了就拜呀。

啊,太后,新姑老爷,总是要拜的。

皇叔,你要多拜几拜。

领旨。

太后有旨,二千岁上佛殿呐------

啊太后,可知皇叔的根基呀?

皇叔乃中山靖王之后,汉景帝陛下之玄孙,荆襄王刘表之堂弟,当今天子之皇叔。喏喏喏,太后请看------生得是龙眉凤目,两耳垂肩,双手过膝,真不愧是帝王的根本呐!

呃,帝王的根本!

说说也无妨啊!

哦,是,是,是。

啊太后,关美髯太后可晓得?

此人姓关名羽字云长,乃蒲州解良人也。弟兄桃园结义以来,在徐州失散,万般无奈,暂归曹营。那曹操待他十分恩厚,三日一小宴,五日一大宴,上马金、下马银,美女十名,俱不肯受哇。闻得皇叔有了下落,彼时挂印封金,在灞桥挑袍,过五关、斩六将,弟兄在古城相会。这位将军的义气------哼,不小哇!

好义气!

呃,虽不是我亲眼得见,谁人不知,呃呃,哪个不晓哇!

哦哦,好、好、好。

啊太后,张翼德太后可知?

此人姓张名飞字翼德,乃涿郡范阳人也。这位将军,在当阳桥前大吼一声,吓得曹操收去青龙伞,惊死夏侯杰。这位将军好威风,好煞气呀!

呃好威风,好煞气!

啊太后,赵子龙太后可晓得?

这位将军姓赵名云字子龙,乃真定常山人也。在长坂坡前与曹兵交战,杀入曹营,是七进七出!

呃不不不,七进七出!

七出七进,是七进七出啊!

呃本来是七进七出啊!

诸葛亮太后可知啊?这位先生,复姓诸葛名亮字孔明,道号卧龙,乃阳都人也。皇叔三顾茅庐,他是才得出山。这位先生在我东吴南屏山,高设一台,名曰七星祭风坛,借来三日三夜东风,烧退曹兵八十三万,好烧哇好烧!

领旨。

哪个的主意?

莫非二千岁?

太后有旨,二千岁上佛殿。

啊太后,我东吴有员大将,名叫贾化。

呃太后,新姑老爷讲情,总是要准的呀!

下去!

是。

太后,老臣眼力如何?

遵旨。

太后回宫。

带马------

\textbf{回荆州 之 鲁肃}

\textbf{(众将 得令!)}

\textbf{慢------慢,慢(,慢)\ldots{}\ldots{}慢着!}

\textbf{【西皮散板】美人计成画饼早已料就,到此时切莫要另结冤仇哇。鲁子敬怎能够旁观袖手啊,劝都督三思行再定良谋(或:劝都督再思行另定良谋)}\protect\hyperlink{fn186}{\textsuperscript{186}}\textbf{。}

\textbf{(哎呀)都督哇!(想)那郡主(随)同刘备回转荆州,乃是正理。你为何要将她赶回(或:你为何拦阻),是何意也?}

\textbf{哎呀,使不得,使不得呀。}

\textbf{那刘备乃是东吴的娇客呀。}

\textbf{(唉,难为你献那美人之计,诓哄刘备过江招亲,谁想以假成真。)}

\textbf{太后在甘露寺中面相刘备(或:太后做主),将郡主招赘刘备,(呃,那)岂不是东吴的娇客吗?}

\textbf{呃,不难,不难呐,}

\textbf{都督,再备美人,连那张飞也诓了前来!(或:只要都督,再备美人,漫说是那刘备,就是那张飞,嘿,他也是要来的呀。)}

\textbf{太后未必依你。}

\textbf{荆州兴兵?}

\textbf{哎呀,我怕呀------}

\textbf{(唉,)那诸葛亮的诡计,实在地厉害呀!(或:是厉害得很呐!)}

\textbf{(啊)都督,难道你就忘怀了?}

\textbf{(周瑜 忘怀了什么?)}

\textbf{(想)当年赤壁鏖兵,他在(那)南屏山上祭借东风,都督派了丁奉、徐盛,刺杀于他,尚且被他逃走(或:他尚且逃走)。}

\textbf{(都督又派他二人驾舟追赶,又被赵云箭射篷索而回。)}

\textbf{(哼,那时)几乎将你气死啊。}

\textbf{哼,难道这不是孔明的诡计么?(或:难道这不是孔明的诡计吗?)}

\textbf{啊都督,(你)不要生气呀。}

\textbf{这生气的日子还在后头呢。(或:那生气的日子还在后头呢。)}

\textbf{呃都督\ldots{}\ldots{}}

\textbf{(周瑜 不要管我的闲事。)}

\textbf{啊呀,这可不是闲事啊。}

\textbf{国家大事,唉,我不能不管呐。(或:军国大事,我是不能不问呐。)}

\textbf{呃呃呃,少弟,少弟。(或:呃呃,不敢不敢。)}

\textbf{不敢,不敢。(或:少弟。)}

\textbf{我本是个老实人,老实人才说这老实话呀!(或:唉,老实人才讲这老实话呀!)}

\textbf{哦哦,我,我,我吃醉了?(或:呃呃,呃,呃,呃呃,我,我\ldots{}\ldots{}我醉了。呃呃,我醉了?)}

\textbf{呃,我不曾吃酒,怎么我醉了?}

\textbf{诶------}

\textbf{(念)此时不听我言语,损兵折将后悔迟!}

\textbf{回荆州 之 刘备}\protect\hyperlink{fn187}{\textsuperscript{187}}

\textbf{{[}第一场{]}}

\textbf{【西皮原板】深宫无处不飞花,年老得配女娇娃。朝欢暮乐无牵挂,愿把东吴当故家。}

\textbf{【西皮散板】听说曹操发人马,攻破荆州把孤拿。四弟之言并非假,想一良谋\ldots{}\ldots{}}\protect\hyperlink{fn188}{\textsuperscript{188}}

\textbf{郡主,备要逃走了。}

\textbf{【西皮散板】本当在此多潇洒,失却荆州无有家。见郡主难说分别\textless{}哭头\textgreater{}话,}

\textbf{【西皮散板】花言巧语瞒哄她。}

\textbf{【西皮散板】根深哪怕狂风大,树正何惧日影斜。}

\textbf{{[}第二场{]}}

\textbf{(刘备穿箭衣上)}

\textbf{【西皮散板】郡主进宫辞太后,为何一去不回头。四弟且站宫门口,准备鳌鱼脱金钩。}

\textbf{认得也!}

\textbf{【西皮散板】多蒙太后恩德厚,此去只怕孙仲谋。}

\textbf{【西皮散板】四弟与孤带走兽,}

\textbf{(赵云接唱收腿,刘备下)}

\textbf{{[}第三场{]}}

\textbf{(刘备上)}

\textbf{【西皮导板】身躬步臃路途远,}

\textbf{【西皮散板】\ldots{}\ldots{}往前赶,怕的吴兵追赶还。}

\textbf{【西皮散板】你姑老爷要走呃你们谁敢拦。}

\textbf{(刘备下)}

\textbf{{[}第四场{]}}

\textbf{(刘备上)}

\textbf{【西皮散板】急急赶来如风涌,插翅难飞到九重(或:上九重)。}\protect\hyperlink{fn189}{\textsuperscript{189}}

\textbf{四弟,前有大江,后有追兵,如何是好?}

\textbf{(赵云 待我望来。)}

\textbf{\ldots{}\ldots{}}

\textbf{刘备 搭了扶手。}

\textbf{(诸葛亮 诸葛亮接驾。)}

\textbf{\ldots{}\ldots{}}

\newpage
\hypertarget{ux8ba9ux6210ux90fd-ux4e4b-ux5218ux748b}{%
\subsection{让成都 之
刘璋}\label{ux8ba9ux6210ux90fd-ux4e4b-ux5218ux748b}}

{[}第一场{]}

{[}引子{]}坐镇西川,恨张松,降顺桃园。

(念)君为民忧,又为国愁。忧国忧民,何日罢休?

孤,刘璋字季玉。祖镇西川。前者误听张松之言,致招刘备入川,指望同振汉室基业,不想他暗起图谋之意。是孤在张鲁王驾前,聘请一将,名唤马超,也曾命他在葭萌拒敌,今与桃园交战,未知胜负,且听探马一报。(或:孤,刘璋字季玉。祖镇西川。前者误听张松之言,致招刘备入川,指望同兴汉业,谁知他暗起图谋之心。是孤在张鲁王驾下,聘请一将,名曰马超,孤命他镇守葭萌关,今与桃园弟兄交战,不知胜负如何,且听探马一报。)

再探。

不,不,不好了!

【西皮散板】闻报不由心内惊(或:忽听探马报一声),不料(或:大胆)马超降他人。王到敌楼把贼问,

【西皮散板】皇儿上殿问分明。

皇儿平身,一旁坐下。(或:平身,赐座。)

(皇儿上殿,有何本奏?)

孤想刘备,兵多将广,意欲将成都让与他人就是。

【西皮原板】皇儿奏本欠思论,哪有能将敌雄兵。心中只把张松恨,竟将那地理图献与他人。老严颜巴州【转西皮二六】早降顺,张任不降命归阴。聘来的马超威风凛,反顺刘备取都城(或:反顺刘备降他人)。王有心开城把贼问,文武个个起异心。左思右想心不定,王倒做进退两难人。

【西皮快板】那刘备仁义从天命,诸葛先生赛苏秦。孤把好话对他论,难道不念同宗人。

平身。(赐座。)

卿家上殿,有何本奏?

我父子正为此事筹议,何言(或:何谓)坐视不理?

何人(或:何臣)保驾?

卿家保驾,孤无忧也。

听孤旨下。

【西皮摇板】皇儿敌楼把贼问,大事全仗王爱卿。四门人马安排定,莫教那马超贼杀进都城。

(窝下)

{[}第二场{]}

摆驾。

【西皮散板】适才王累进宫报,王儿(或:皇儿)敌楼赴阴曹。

【西皮散板】耳旁又听放火炮,马超贼放火把孤的民房烧。侍内臣摆驾上城道,

唉,皇儿\ldots{}\ldots{}(哭介)

【西皮散板】那旁来了贼马超。

(马超 蜀主请了!)

【西皮散板】见马超不由我(或:不由孤)心如刀绞,尊一声马孟起细听根苗:为王的待你是哪些儿不好,你那里降刘备所为哪条。

唉!

【西皮散板】王聘你原本为(或:王聘你为的是)西川有靠,看起来你是个无义儿曹。

【西皮散板】一言怒恼贼马超,放火把孤的民房烧。

【西皮散板】只烧得众黎民苦哀\textless{}\textbf{哭头}\textgreater{}告,

【西皮摇板】刘季玉失疆土就在今朝。

马将军,将人马暂退一箭之地,孤将成都让与刘备就是。

(马超 休得失信于我。)

唉!岂肯失信于你?

唉!

【西皮摇板】这也是成都地兵微将少,眼见得锦绣春付与水漂。

众将,

开城。

(王累 且慢!此城开不得!)

怎样开不得?

卿家,你来看!

为孤一人,岂肯连累百姓?

【西皮摇板】宁愿失却成都郡,岂肯连累好子民。

(王累 哎呀!\ldots{}\ldots{}我只为\ldots{}\ldots{}不如碰头死在都城。)

哎呀!

【西皮摇板】一见卿家丧了命,斩断擎天柱一根。

【西皮摇板】但愿你(或:但愿得)灵魂呐归仙\textless{}\textbf{哭头}\textgreater{}境,

【西皮摇板】凌烟阁上第一名(或:凌烟阁上标美名)。

众将,

开城。(哭介)

{[}第三场{]}

啊,宗兄!

宗兄请。

宗兄到此,乃是客位。

还是宗兄请

如此你我挽手而行。

呵,呵,呵,呃\ldots{}\ldots{}(哭介)

{[}第四场{]}

(宗兄,)此位是?

(刘备 这就是诸葛先生。)

哦,这就是卧龙先生。

请坐。

不知宗兄驾到,未曾远迎,当面恕罪。

啊,宗兄,前番敦请宗兄入川,共掌汉室之基业。(或:前番将成都让与宗兄执掌,)宗兄言道:不夺同宗之基业,致招天下人耻笑(或:惹天下人笑骂)。如今宗兄又兴此无义之师(或:今日兴此无义之师),失信于天下,是何意也?

(诸葛亮 这个\ldots{}\ldots{}我主乃是不得以而为之。)

哦,呵,呵,呵\ldots{}\ldots{}(冷笑介)

好一个不得已而为之!

(刘备 【西皮原板】\ldots{}\ldots{}进成都城。)

宗兄!(或:宗兄啊!)

【西皮原板】你我本是【转西皮二六】同宗姓,你今到来王心惊。有什么大事早议论,又何必带兵夺取我都城。先前让你掌蜀郡,一心要做仁义人。实指望两下【转西皮快板】结秦晋,又谁知反学吴越动刀兵。勒逼孤让成都郡,难道要我命不成。

【西皮快板】这几句言语实难听,俱是诸葛定计行。大胆难免把头刎,胆小也要见阎君。走向前来将他问(或:把话论),问他几语(或:问他几句)待怎生。

【西皮快板】此处好比鸿门宴,缺少樊哙保驾臣。孤若不念同宗姓,岂肯容你进都城。

哎呀!

【西皮摇板】两旁武将杀气生,

哎呀!

【西皮摇板】只见严颜老将军。孤命你(或:孤让你)镇守那巴州郡,为什么背孤王降顺他人。

呀呸!

【西皮摇板】孤道你(或:指望你)年迈苍苍忠心耿,却原来背主求荣狗肺心。

(众 让印!)

哎呀!

【西皮摇板】蝼蚁尚且贪性命,不让成都命难生。无奈何取出了先王印,

\textless{}\textbf{哭头}\textgreater{}先王啊,

【西皮摇板】从今后让你掌龙庭(或:掌乾坤)。

呵,还要拜过。

(过去\textbf{,}坐大边外场)

事到如今,但凭你君臣所为。

哦!

【西皮慢板】听说是一声要饯行,好一似狼牙箭攒心。舍不得成都花花美景,实难舍西川老少子民。含悲忍泪换衣巾(或:换衣衿\protect\hyperlink{fn190}{\textsuperscript{190}}),

【西皮原板】辞别了宗兄就要启行。但愿你把曹【转西皮二六】早扫定,但愿你在此享太平。但愿你各国把贡进,但愿你天降福禄亚似个尧君。西川的文武刀刀斩尽,尽都是那贪生怕死臣。王失却西川无怨恨,望宗兄开恩照看孤的这些好子民。

【西皮摇板】到此时他还是假殷勤,花言巧语宽王心。咽喉紧哽跨金蹬,

(刘备 唉,宗兄啊\ldots{}\ldots{})

【西皮摇板】刘备一旁假悲淋。(或:刘备送我假殷勤。)

【西皮摇板】我刘璋不把你别事愿,

【西皮摇板】但愿你后辈的儿孙也照孤样行。

\newpage
\hypertarget{ux767eux5bffux56fe}{%
\subsection{百寿图}\label{ux767eux5bffux56fe}}

\textbf{{[}第一场{]}}

\textbf{管辂 {[}引子{]}乾坤兴衰,日月相连,一卷生平。}

\textbf{管辂
(念)天上星辰日月,人间山水物华。争长论短空嗟呀,还是天伦为大。}

\textbf{管辂
贫道,姓管名辂字公明,乃平原人氏。自幼生就一双慧眼,能知过去、未来之事。在这十字街前,摆了一座卦棚,无非是指引世人。看今日天气晴和,我不免卦棚走走。}

\textbf{管辂
【二黄慢板】叹光阴似箭穿过目烟云,猛抬头见草木又已发青。观前面山岗上松竹茂盛,又看见山坡下花开缤纷。花开时比人生越开越盛,花败落}\protect\hyperlink{fn191}{\textsuperscript{191}}\textbf{比人老迟暮光阴。有等人贪酒色昏迷不醒,有等人为妻妾家业凋零。有等人为财产伤了性命,有等人为小事大祸临身。平安日必须要安守本分,切不可倚势力}\protect\hyperlink{fn192}{\textsuperscript{192}}\textbf{欺压旁人。轻移步来至在十字路径,等候了繁华世痴迷之人。}

\textbf{赵颜 【西皮摇板】在家中遵奉了双亲严命,手牵着青牛儿去把田耕。}

\textbf{赵颜 小生,赵颜。奉了双亲之命,下田耕种。就此走走。}

\textbf{赵颜 【西皮导板】世间人必须要耕种为本,}

\textbf{赵颜
【西皮原板】官出民民出土土内生金。来至在十字路用目观定,卦棚内坐定了算命的先生。放下犁拴上牛卦棚来进,看一看他手托哪部古文。}

\textbf{管辂
【西皮原板】稳坐在卦棚内心中烦闷,观前朝和后世累代帝君:前三皇后五帝尧王传舜,舜传禹、禹传商、商王为君。殷纣王坐江山天心不顺,宠爱妃妲己女残害忠臣。摘星楼摆筵宴比干丧命,黄飞虎反五关去投明君。且不论前朝事用目观定,}

\textbf{管辂 【西皮摇板】猛抬头见小哥令人吃惊。}

\textbf{管辂 可叹呐,可叹!}

\textbf{赵颜 啊,先生叹者何来?}

\textbf{管辂 请问小哥,家住哪里,姓氏名谁?身背犁杖,要往何方?}

\textbf{赵颜 小子赵颜,奉了双亲之命,下田耕种。}

\textbf{管辂
贫道管辂,字公明,能知未来。我劝你休要耕种呃;急速回家,好酒好饭,饱餐三天才是啊。}

\textbf{赵颜 先生你何出此言?}

\textbf{管辂 我看你气色不正,三日后你定夭寿而亡。}

\textbf{赵颜 诶,先生,你看我行路有影,痰嗽有声,怎见得我三日后必死呢?}

\textbf{管辂 唉,小哥啊!}

\textbf{管辂 【西皮摇板】休道你行有影痰嗽有声,岂不知天有那不测风云。}

\textbf{赵颜 【西皮摇板】管先生说此话我却不信,哪有个平白地死了好人。}

管辂 小哥!

\textbf{管辂
【西皮摇板】劝小哥听此话休得不信,待贫道下位去观看五行:耳属金金不能生水半寸,眉属木木生火枝叶凋零。口属水水已干犹如枯井,眼属火火无光是不能生金。鼻属土土入陷死气已真,三日后你必定命赴幽冥。}

\textbf{赵颜
【西皮摇板】听他言不由我心神不定,背转身我这里自己思忖。急忙忙向前去先生来问,三日后我不死有何为凭。}

\textbf{管辂
【西皮摇板】三日后你不死只管议论,如不然我和你去到公厅}\protect\hyperlink{fn193}{\textsuperscript{193}}\textbf{。再不然将我的卦棚拆损,任你羞任你辱任你施行。}

\textbf{赵颜
【西皮摇板】听他言吓得我心意不定,想必是三日后要见阎君。转过身牵犁牛急往家奔,见双亲把此话细说分明。}

\textbf{管辂 【西皮摇板】可叹他少年人大数已尽,这也是五阎君造定死生。}

\textbf{{[}第二场{]}}

\textbf{赵范 【西皮摇板】一家人全凭着耕种为本,小娇儿下田去未见回程。}

\textbf{赵颜 走啊!}

\textbf{赵颜 【西皮摇板】放下犁拴上牛家门来进,见双亲泪汪汪跪在埃尘。}

\textbf{赵颜 喂呀,爷娘啊\ldots{}\ldots{}(哭介)}

\textbf{赵范、赵母 儿啊,为何啼哭?}

\textbf{赵颜
爷娘有所不知,孩儿下田耕种,行至十字街前,有一算命先生,与儿看了一相。他道孩儿三日后夭寿,唉,而亡啊\ldots{}\ldots{}(哭介)}

\textbf{赵范、赵母 不好了!}

\textbf{赵范 【西皮摇板】听说是三日后我儿丧命,}

\textbf{赵母 【西皮摇板】只恐怕绝了我赵氏后根。}

\textbf{赵母 呃,那算命先生姓氏名谁,现在何处?}

\textbf{赵颜 此人姓管名辂,现在十字街前。}

\textbf{赵范、赵母 待我二老前去哀求先生,唉,倘有活命,亦未可知。}

\textbf{赵范、赵母 儿啊,带路!}

\textbf{赵颜 是。}

\textbf{赵范 【西皮摇板】叫妈妈你那里将门栓定,卦棚内去哀告算命先生。}

\textbf{{[}第三场{]}}

\textbf{管辂 【西皮摇板】将身儿来至在卦棚坐定,算人间吉凶事不差毫分。}

\textbf{赵范、赵母 (内)走啊!}

\textbf{赵范 【西皮摇板】教娇儿你与我把路来领,见先生泪汪汪跪在埃尘。}

\textbf{赵范、赵母 哎呀,先生呐!}

\textbf{管辂 【西皮摇板】耳边厢又听得悲声一阵,见二老泪汪汪跪在埃尘。}

\textbf{管辂 赵颜。}

\textbf{赵颜 有。}

\textbf{管辂 这二老是你何人?}

\textbf{赵颜 二老双亲。}

\textbf{管辂 哎呀,年迈之人,快快请起呀。}

\textbf{赵范、赵母 多谢先生!}

\textbf{管辂 你二老到此何事?}

\textbf{赵范、赵母 唉,先生呐!}

\textbf{赵范 【西皮摇板】小老儿名赵范六十三岁,}

\textbf{赵母 【西皮摇板】我二老年半百有此娇生。}

\textbf{赵范 【西皮摇板】先生道我娇儿三日丧命,}

\textbf{赵母 【西皮摇板】望先生发慈悲搭救娇生。}

\textbf{管辂
【西皮摇板】我不是五阎君秦广宫殿,我不是阴曹府掌簿判官。我不是观世音救苦救难,我不是西天佛法力无边。}

\textbf{赵范、赵母 唉呀,先生呐!}

\textbf{赵范 【西皮摇板】我哭、哭一声管先生,}

\textbf{赵母 【西皮摇板】叫、叫一声管辂仙。}

\textbf{赵范 【西皮摇板】为娇儿我二老朝山拜顶,}

\textbf{赵母 【西皮摇板】为娇儿我二老把香来焚。}

\textbf{赵范 \textless{}哭头\textgreater{}管先生,}

\textbf{赵母 \textless{}哭头\textgreater{}仙长爷,}

\textbf{赵范、赵母 \textless{}哭头\textgreater{}啊,先生呐。}

\textbf{管辂
【西皮摇板】见二老只哭得我心好惨,不由人一阵阵心内痛酸。这时候怎救得残生命转,}

\textbf{管辂 哦,有了!}

\textbf{管辂
【西皮摇板】又只见南北斗已奔}\protect\hyperlink{fn194}{\textsuperscript{194}}\textbf{高山。}

\textbf{管辂 二老请起。}

\textbf{赵范、赵母 多谢先生!}

\textbf{管辂 赵颜有了救了。}

\textbf{赵范、赵母 救在哪里?}

\textbf{管辂 回到家去,准备鹿脯美酒,去至终南山,有二\ldots{}\ldots{}}

\textbf{赵颜 先生,二什么?}

\textbf{管辂
有二位仙长在那里着棋,你将这鹿脯美酒,暗暗献上。他饮了你的酒,必要与你添寿。}

\textbf{赵范、赵母 请问先生,这二位仙长怎样打扮?}

\textbf{管辂 你们听了!}

\textbf{管辂
【西皮摇板】有一个穿白袍斯文体相,有一个穿红服气宇轩昂。你将这鹿脯酒暗暗献上,饮了酒必与你添寿绵长。}

\textbf{赵范、赵母 先生!}

\textbf{赵范 【西皮摇板】辞别了管先生忙往家奔,}

\textbf{赵母 【西皮摇板】准备下鹿脯酒送到山林。}

\textbf{赵颜 【西皮摇板】辞别了管先生忙回家门,}

\textbf{管辂 转来!}

\textbf{赵颜 【西皮摇板】问先生唤回我所为何情。}

\textbf{管辂 【西皮摇板】你把这鹿脯酒暗暗献上,必须要隐身形跌跪一旁。}

\textbf{赵颜
【西皮摇板】仙长爷你不必仔细叮咛,我赵颜纵一死不忘大恩。此一番到南山把酒来敬,见仙长求增寿小心殷勤。}

\textbf{管辂
【西皮摇板】这也是小赵颜不该命丧,他二老前世里积下善良。哀告那南北斗将寿添上,到后来子孙多瓜瓞绵长。}

\textbf{{[}第四场{]}}

\textbf{南斗、北斗 (内)请啊!}

\textbf{南斗 【西皮摇板】观天地和日月乾坤浩荡,}

\textbf{北斗 【西皮摇板】水连天天连水渺渺茫茫。}

\textbf{南斗、北斗 吾乃南、北斗星君是也。}

\textbf{南斗 星君请了。}

\textbf{北斗 请了。}

\textbf{南斗
你我奉了玉帝敕旨,巡查人间善恶。来此已是南瞻部洲,今日闲暇无事,将历代君王之事,细表一番。}

\textbf{南斗 请------}

\textbf{南斗 【西皮原板】自盘古分天地乾坤始创,}

\textbf{北斗 【西皮原板】先太极分两仪八卦阴阳。}

\textbf{南斗 【西皮原板】按金木水火土五行方向,}

\textbf{北斗 【西皮原板】先君臣后父子三纲五常。}

\textbf{南斗 【西皮原板】尧传舜舜传禹天下揖让,}

\textbf{北斗 【西皮原板】夏桀暴商纣淫自取灭亡。}

\textbf{南斗 【西皮原板】秦始皇归一统山河执掌,}

\textbf{北斗 【西皮原板】他不该焚诗书兴建阿房。}

\textbf{南斗 【西皮原板】楚霸王他倒有帝王之相,}

\textbf{北斗 【西皮原板】他不该杀义帝强霸为王。}

\textbf{南斗
【西皮原板】把前朝君王事【转西皮快板】暂且慢讲,有一辈忠良臣细说端详:淮阴侯小韩信功高智广,为什么未央宫一命身亡。}

\textbf{北斗
【西皮快板】休道那小韩信功高智广,他不该活埋母九里山旁。他不该问道路把樵哥斩丧,他不该逼高祖拜他为王。他不该逼霸王乌江命丧,因此上未央宫一命身亡。}

\textbf{南斗 【西皮快板】成萧何败萧何萧何该丧,为什么那老儿寿命延长。}

\textbf{北斗 【西皮快板】休道那汉萧何该当命丧,他本是忠良臣寿命延长。}

\textbf{南斗 【西皮快板】叹不尽前朝的忠臣良将,}

\textbf{北斗 【西皮摇板】松林内摆棋盘散闷一场。}

\textbf{赵颜 走啊!}

\textbf{赵颜
【西皮摇板】手捧着鹿脯酒终南山上,又只见二仙长分坐两旁。我这里将鹿脯暗暗献上,吞着气躲着身跌跪一旁。}

\textbf{南斗、北斗 请呐!}

\textbf{南斗 【西皮原板】在石台摆棋盘一帅一将,}

\textbf{北斗 【西皮原板】红棋先黑棋后各霸一方。}

\textbf{南斗 【西皮原板】走一步当头炮千军难挡,}

\textbf{北斗 【西皮原板】还一个连环马士相奔忙。}

\textbf{南斗 【西皮原板】又只见鹿脯酒从空而降,}

\textbf{北斗 【西皮原板】想必是天赐我美味清香。}

\textbf{南斗 【西皮原板】我和你棋不胜共饮佳酿,}

\textbf{北斗 【西皮摇板】飞来物饮几杯又有何妨。}

\textbf{南斗 你我再下一盘。}

\textbf{北斗 请啊!}

\textbf{南斗 【西皮摇板】战胜了好一似汉高皇上,}

\textbf{北斗 【西皮摇板】战败了好一似西楚霸王。}

\textbf{赵颜 求寿啊!}

\textbf{南斗 【西皮摇板】耳边厢又听得有人喧嚷,}

\textbf{北斗 【西皮摇板】猛抬头见小子跌跪道旁。}

\textbf{北斗 那一小子,家住哪里,姓氏名谁,到此何事?慢慢讲来。}

\textbf{赵颜 二位仙长容禀!}

\textbf{赵颜
【西皮摇板】家住在城厢外绿柳村上,我的名叫赵颜耕种田庄。都只为在卦棚先生看相,他道我三日后一命身亡。}

\textbf{南斗 哦!}

\textbf{南斗 【西皮摇板】见小子说此话倒也响亮,}

\textbf{北斗 【西皮摇板】仙家事是何人泄漏阴阳。}

\textbf{南斗 啊,星君,你我在此着棋,凡人怎能知晓?}

\textbf{北斗
星君有所不知,只因凡间有一个管辂,生就一双慧眼,能知人间过去未来之事,想是他指引前来,也未可知。}

\textbf{南斗 星君何不将他阳寿查上一查。}

\textbf{北斗 待我查来。}

\textbf{北斗
查得山西平原郡绿柳村赵范之子,名唤赵颜,前生作恶多端,今投赵门为子。注定大汉建安一十二年,寿活一十九岁,夭寿而亡。}

\textbf{北斗 赵颜,你今年多大了?}

\textbf{赵颜 一十九岁。}

\textbf{南斗 嘿嘿,完了!}

\textbf{北斗
【西皮摇板】叫小子抬头看生死簿上,这上面造定了字字行行。十九岁你就该把命夭丧,这时候并无有解救良方。}

\textbf{赵颜 不好了!}

\textbf{赵颜
【西皮摇板】听他言吓得我魂魄飘荡,不由我小赵颜无有主张。望仙长发慈悲将寿添\textless{}哭头\textgreater{}上,}

\textbf{赵颜 【西皮摇板】可怜我家还有二老爷娘。}

\textbf{南斗、北斗 哦!}

\textbf{南斗 【西皮摇板】小赵颜只哭得泪如雨降,}

\textbf{北斗 【西皮摇板】可怜他家还有二老爷娘。}

\textbf{南斗 啊,星君!看这赵颜哭得可怜,何不将他阳寿与他添上。}

\textbf{北斗
星君说哪里话来。你我奉了玉帝敕旨,巡查人间善恶。私添阳寿,玉帝闻知,吃罪不起。}

\textbf{南斗 上苍也有好生之德。何况你我\ldots{}\ldots{}}

\textbf{北斗 这是你吃了好酒!}

\textbf{南斗 你也好贪杯!}

\textbf{北斗 彼此?}

\textbf{南斗 一样!}

\textbf{南斗、北斗 啊,呵呵哈哈哈\ldots{}\ldots{}(笑介)}

\textbf{北斗星 如此说来,这阳寿添得的?}

\textbf{南斗星 添得的!}

\textbf{北斗星
赵颜,你命活一十九岁,将这``一''字改为``九''字。寿活九十九岁,也就够了。}

\textbf{赵颜 啊,仙长,将那一岁添上,岂不是百岁老人?}

\textbf{南斗、北斗 诶------贪心不足。听我等道来:}

\textbf{南斗 【西皮原板】我本是南斗星从空而降,}

\textbf{北斗 【西皮原板】我本是北斗星降下天堂。}

\textbf{南斗 【西皮原板】我掌生他掌死分毫不爽,}

\textbf{北斗 【西皮原板】查人间生和死善恶昭彰。}

\textbf{南斗 【西皮原板】我赐你子孙多富贵永享,}

\textbf{北斗 【西皮原板】我赐你财源盛金玉满堂。}

\textbf{南斗 【西皮原板】我赐你椿萱茂代代兴旺,}

\textbf{北斗 【西皮原板】我赐你一家人无有灾殃。}

\textbf{南斗 【西皮原板】我赐你百寿图悬挂堂上,}

\textbf{北斗 【西皮原板】我赐你九十九大寿延长。}

\textbf{南斗 【西皮原板】在人间休得要胡言乱讲,}

\textbf{北斗 【西皮摇板】泄露了仙家事五雷身亡。}

\textbf{南斗、北斗 去罢!}

\textbf{赵颜 【西皮摇板】谢罢了二仙长忙下山岗,}

\textbf{南斗、北斗 转来!}

\textbf{赵颜 【西皮原板】星君爷唤回我所为哪桩。}

\textbf{南斗
回去见了管辂,叫他从今以后,不要胡言乱语;再若胡言乱语,难免五雷殛顶。那旁有人来了。}

\textbf{赵颜 在哪里?}

\textbf{赵颜 哎!二位仙长不见,待我望空一拜。}

\textbf{赵颜 【西皮原板】手捧着百寿图忙下山林,回家去见爷娘细说分明。}

\newpage
\hypertarget{ux5b9aux519bux5c71-ux4e4b-ux9ec4ux5fe0}{%
\subsection{\texorpdfstring{定军山\protect\hyperlink{fn195}{\textsuperscript{195}}
之
黄忠}{定军山195 之 黄忠}}\label{ux5b9aux519bux5c71-ux4e4b-ux9ec4ux5fe0}}

{[}第一场{]}

慢着。

黄忠来也。

参见军师。

谢军师呃。

军师呃。攻取葭萌关,何劳三千岁。赐末将一哨人马,生擒那张郃入帐呃。

\textless{}\textbf{叫头}\textgreater{}军师呃。

(右手弹髯口,跨左腿,反云手,骗右腿,拱手从左向右转身(腰))

(念)末将年迈勇,血气贯长虹。杀人如削土,跨马走西东。两膀千斤力,能开铁胎弓。若论交锋事,还算老黄忠。

得令!

【西皮二六】师爷说话言太差,不由得黄忠怒气发。一十三岁习弓马,威名镇守在长沙。自从归顺皇叔爷的驾,匹马单刀取过了巫峡。抢关夺寨功劳大,师爷不信你在功劳簿上查一查。不是我黄忠夸大话,

弓来!

【西皮快板】铁胎宝弓手中拿。满满搭上朱红扣,帐下的儿郎把咱夸。

【西皮快板】二次运动这千斤的力,

【西皮散板】人有精神气又加。

【西皮散板】三次开弓秋月样,

【西皮散板】再与师爷把话答。

得令。

在。

得令。

【西皮摇板】黄忠接令把帐下,

(严颜 【西皮摇板】不由严颜笑哈哈。)

【西皮摇板】一不用战鼓这嗵嗵地打,

(严颜 【西皮摇板】二不用副将把队押。)

【西皮摇板】事不宜迟把马跨,

马来呃!

(严颜 【西皮摇板】\ldots{}\ldots{}把张郃一马踏。)

{[}第二场{]}

(罢了,)一旁坐下。

可曾与那贼会过阵来(或:见过阵来)。

(严颜 且慢,军家胜败,古之常理啊。)

(严颜 老将军开恩。)

还不谢过严老将军。

陈式听令啊,城头之上高扯红旗二面,上写黄忠、严颜。那贼闻名丧胆。

再探。

老将军,张郃小儿他来了啊!

(严颜 你我会他一会。)

(一派胡言。)

带马。

{[}第三场{]}

老夫黄忠

(严颜 严颜。)

尔为何发笑?

一派胡言,放马过来。

{[}第四场{]}

老将军追赶何人?

那贼去远了。

韩浩、夏侯尚,便宜了他们。

老将军抬头观看。

老将军,看前面已是天荡山,乃曹操屯粮之所。此山不破,你我大功难成。

老将军你有何妙计?

老将军妙计!(或:此计甚好,你我一同传令。)

众将官。

照计而行。

{[}第五场{]}

【西皮快板】背地里暗笑诸葛亮,他道老夫少刚强。虽然年迈精神爽,杀人犹如宰鸡羊。催马来在阵头上,

【西皮摇板】那旁来了送死郎。

来将通名。

(韩浩 韩浩。)

你来则甚?

(韩浩 替兄报仇。)

放马过来。

【西皮快板】走投鱼儿入罗网,败阵绵羊敢逞强。老夫倒有容人量,怎奈宝刀世无双。眼前若有诸葛亮,

【西皮摇板】管教他含羞带愧他的脸无光。

{[}第六场{]}

(念)大将军八面威风。

有请。

在。

得令。

后帐留宴。

带马。

老将军,你我一笑而别了哇。呵呵哈哈\ldots{}\ldots{}(笑介)

带马。

{[}第七场{]}

请。

一赖\protect\hyperlink{fn196}{\textsuperscript{196}}主上洪福,二赖先生妙算。末将(或:老臣)何功之有。

不敢呐不敢。

慢着。(或:且慢呐)

军师啊,攻取定军山,何劳二千岁远路而来。赐末将一哨人马,生擒那夏侯渊入帐呃。

军师呃,想那张郃乃中原有名上将,被末将杀得是望风而逃,何况那夏侯渊,乃一勇之夫(或:是一勇之夫)。

也罢。

俺若(或:倘某;我若)胜不过那夏侯渊,愿输项上的人头。

打赌啊,

得罪了!

在。

得令。

哦!

【西皮二六】在黄罗宝帐领将令,气坏了老将黄汉升。某昔年镇守长沙郡,偶遇云长(或:圣贤)二将军。某中了他人的拖刀计,我的百步穿杨射他的盔缨。弃暗投明【转西皮快板】来归顺,食王的爵禄当报王的恩。孝当竭力忠尽命,再与师爷把话论。一不用战鼓嗵嗵打,二不用副将随后跟。只要我黄忠一骑马,匹马单刀取定军。十日之内攻得胜\protect\hyperlink{fn197}{\textsuperscript{197}},军师的大印付与某的身。十日之内不得胜,愿将人头挂营门。来来来带过爷的马能行,

【西皮摇板】我要把定军山一扫平。

{[}第八场{]}

【西皮快板】吾主爷攻打葭萌关,将士纷纷取东川。可笑军师见识浅,道我难胜那夏侯渊。张郃被某杀破胆,卸甲丢盔奔荒山。坐至在雕鞍将令传,大小儿郎听爷言:埋鹿角、掘沟堑,金晶铠甲扣连环。上前个个把功建,退后的人头挂高竿。大吼一声催前站,

【西皮散板】十日之内取东川。

(唱``取东川'',右手马鞭,反云手,左手拉开平亮,正亮在``川''字上。再反云手,抬右腿,推左掌(向左斜场),抱鞭,跨右腿,转身,甩髯口,左脚找地方,左腿弓步,觑地,勒马,亮相,拿神。\textless{}\textbf{四击头}\textgreater{}接\textless{}\textbf{急急风}\textgreater{}。倒脚活法儿,起身,抬右腿,向右圆着走三步,反云手转身,跨左腿,打马出右腿(面外),向左甩髯口,左手勒马,抬头,\textless{}\textbf{八嗒仓}\textgreater{}亮相(面对下场门)。\textless{}\textbf{急急风}\textgreater{},下)

{[}第九场{]}

【西皮快板】(两军对垒动干戈,一来一往战几合。)夏侯渊打扮真不错,黑面长须似阎罗。劝你马前归顺我,宝刀下去尔的命难活。

{[}第十场{]}

【西皮快板】两家交锋来会过,一来一往动干戈。魏营打罢得胜的鼓,

【西皮摇板】我营缘何不鸣锣。

再探。

【西皮散板】听一言来心冒火,不由老夫咬牙车\protect\hyperlink{fn198}{\textsuperscript{198}}。人来与爷马带过。

(\textless{}\textbf{叫头}\textgreater{},右手扣腕横刀,左手弹髯口举手,看夏侯尚)

哈哈,

(顺手于左侧,双手握刀,刀头冲外,看刀)

哈哈,

(横刀,面向外亮)

啊哈哈哈\ldots{}\ldots{}(笑介)

(退步,大刀花)

{[}第十一场{]}

【西皮快板】夏侯渊武艺果然好,可算得中原将英豪。将身且坐(或:将身来在)宝帐到,

【西皮摇板】营外缘何闹吵吵。

传。

(罢了,)奉何人所差?

书信呈上,下去。(或:下面伺候。)

夏侯渊来的书信,待我拆开一观。

唤下书人。

回去言讲(或:回覆夏侯将军),就说老夫修书不及,照书行事。

且住。老夫正在营中无计可施,夏侯渊这封书信来得是将将凑巧啊,明日午时三刻与老夫(或:约定老夫明日午时三刻)走马换将。那时先教他放过我国先行陈式,然后再放他侄男夏侯尚。习就百步穿杨,将他侄男一箭射死。那夏侯渊必定带领人马与他侄男报仇。那时老夫杀一阵,败一阵,败至在荒郊。习学关公拖刀之计,将他斩于马下。

夏侯渊呐,我的儿啊。你若来时,定中老夫拖刀之计也。

【西皮快板】这一封书信来得巧,天助黄忠成功劳。站立在辕门传令号,大小儿郎听根苗:一通鼓战饭造,二通鼓紧战袍。三通鼓刀出鞘,四通鼓把兵交。向前个个俱有赏,退后项上吃一刀。就此与爷归营号,

【西皮散板】到明天午时成功劳。

{[}第十二场{]}

请了。

正为书信而来,但不知哪家先放?

呃,老夫到此,乃是客位,自然是你家先放。

老夫若有二意,日后死在药箭之\ldots{}\ldots{}

焉有不放之理。

来,将夏侯尚放了过去。

弓箭伺候。

夏侯尚,看箭。

{[}第十三场{]}

且住,夏侯渊来得厉害。再若来时,拖刀计伤他。

(向左转半圈,横握刀,\textless{}\textbf{冲头}\textgreater{},\textless{}\textbf{叫头}\textgreater{},向右转身上步,右手横握刀,左手举起,抬头)

哈哈,

(右手背刀,刀头在下,撤右步,左手托髯口,左弓步,冲左斜场,面对外)

哈哈,(此处左臂撑开,手对左腿马面,右臂伸直,手对右胯,两腕着力,身上不能使劲)

(收左步面右前,丁字步,横握刀,亮)

啊呵呵哈哈哈\ldots{}\ldots{}(笑介)

(退步,大刀花)

\newpage
\hypertarget{ux9633ux5e73ux5173-ux4e4b-ux9ec4ux5fe0}{%
\subsection{阳平关 之
黄忠}\label{ux9633ux5e73ux5173-ux4e4b-ux9ec4ux5fe0}}

{[}第一场{]}

哈哈,哈哈,啊呵呵哈哈哈\ldots{}\ldots{}(笑介)

老臣奉命,斩得夏侯渊首级,特来献上。

号令辕门。

臣,

谢主隆恩。

老臣怎敢?

折煞老臣了。

主公请。

黄忠愿往。

食王爵禄,当报王恩,何言``劳倦''二字。

俺今出马,立斩张郃头来,四将军以为如何?

四将军。

【西皮二六】讲什么军家无有常胜,仔细看一看我黄汉升。黄忠今年七十整,还要在阵前抖一抖老精神。一马直将曹营进,恰好似猛虎入羊群。眼前若有军师令,看看我老而------我是能不能。

【西皮摇板】长吾黄忠整几春。

【西皮摇板】躬身施礼请将令,

【西皮摇板】差池甘当军令行。

【西皮摇板】黄忠越老越好胜,

马来呃。

【西皮摇板】他那里越欺越侮我偏要行。

{[}第二场{]}

【西皮导板】亦非人前夸老硬,

【西皮快板】胸中韬略亘古今。我在宝帐领将令,赵云道我老无能。任他兵来如潮涌,任他人马似秋云。杀得他血流人头滚,杀得他尸横遍野马难行。洋洋得意朝前进,

啊?

【西皮摇板】张著赶来必有因。

将军赶来则甚?

哦,想是军师以我不能成功,要你赶我回去不成么?

啊,哈哈哈哈\ldots{}\ldots{}(笑介)

军师可谓知我者也。

将军既来相助,你我今晚三更饱餐,四更时分,去至北山脚下,烧贼的粮草。那张郃必然前来营救,就而擒之,岂不美哉?

你我一同传令。

众将官,北山去者。

{[}第三场{]}

【西皮导板】越杀越勇精神好,

【西皮快板】层层密密似涌潮。我若今日遭圈套,一世英名任笑嘲。抖擞精神往前蹈\protect\hyperlink{fn199}{\textsuperscript{199}},

(赵云 【西皮摇板】好似天神下九霄。)

哦,四将军你来了。

杀啊!

\newpage
\hypertarget{ux4f10ux4e1cux5434}{%
\subsection{伐东吴}\label{ux4f10ux4e1cux5434}}

{[}第一场{]}

黄忠
(念)英雄回首忆长沙,百战威名逞虎牙\protect\hyperlink{fn200}{\textsuperscript{200}}。

黄忠 圣上宣召。

黄忠 一同进帐。

黄忠 臣等见驾。

黄忠 愿吾皇万岁,(万岁,)万万岁。

黄忠 宣臣等进帐有何旨意?

黄忠 领旨。

{[}第二场{]}

刘备
【西皮原板】风吹旌旗山岳动,关兴、张苞出御营。未知此去可得胜。举首翘望心不宁。

黄忠
【西皮原板】忆昔当年长沙镇,算来不觉有数春(或:转眼不觉数十春)。荆襄、阆中遭不幸,一心要把东吴平(或:吾主爷要把东吴平)。黄汉升撩袍御营进,

刘备 【西皮原板】老将军免礼且平身。暂陪朕坐消愁闷,

黄忠 【西皮原板】行兵不必泪伤心(或:兴兵不必泪常涔)。

张苞 【西皮摇板】斩将擒贼破敌阵,

关兴 【西皮摇板】弟兄御前显奇能。

张苞
启禀皇伯,儿臣出阵,不料谭雄暗放雕翎,射死战马;幸得关兴赶到,不然性命难保。

关兴 儿臣见张苞兄长落马,赶到阵前,刀劈谢旌,活捉谭雄,特来交令。

刘备 快将谭雄绑了上来!

刘备 好吴狗!

刘备
【西皮散板】四百年来争汉鼎,东吴不君也不臣。鼠窃犬偷真堪恨,快斩逆贼立即行。

黄忠 号令辕门。

刘备 将这厮首级祭奠二千岁灵前;洒下热血,以祭死马。搭了下去。

众 啊------

刘备 朕今兵发东吴,与二位贤弟报仇,幸得二虎侄头阵取胜,惊破吴人之胆。

刘备 左右,看酒。与二位皇侄贺功。

内侍 是。

黄忠 嗯哼。(黄忠痰嗽介)

刘备 哦------老将军你也来呀。

黄忠 臣领旨。

刘备
【西皮原板】庆贺功劳把酒饮,想起了当年破黄巾。战贼挽手威风凛,虎牢关前显威名。

刘备
想当年与尔父等桃园结义之后,破黄巾、得徐州、收襄阳、入西川,皆尔父等之力也,今他们一旦去世啊,所有当年之将,尽是些老迈无用。

刘备 幸有二皇侄斩将破敌,如此英勇,何愁东吴不平。

刘备 看酒来,朕亲为二皇侄贺功。

刘备 【西皮摇板】幸喜皇侄多英俊,此酒酬劳庆功勋。

黄忠 老了哇,老了哇!

黄忠
【西皮散板】主公说话不思忖,他道老将便无能。(或:主公说话欠思忖,怎知老将便无能。)

黄忠
且住,关兴、张苞子侄之辈,阵前擒来谭雄,无非是些许功劳。主公帐中(或:主公隆宠,)夸了又夸,讲了又讲。反讲当年五虎上将尽是些老迈无用。这这这\ldots{}\ldots{}

黄忠
也罢!我不免(或:俺不免)去至两军阵前,斩那东吴八员上将,看看俺黄忠老是不老。

黄忠 【西皮快板】太公八十方交运,廉颇七旬挡秦军。黄忠年迈有本领,

黄忠 【西皮摇板】再学走马取定军。

报子 黄老将军私自出营,人向东而去。

刘备 快去打探。

探子 得令。

刘备
哎呀且住。黄汉升绝非叛逆之人,想是适才朕言老将无能,故而一怒出营,意在斩将显能尔。既然如此,诚恐有失。

刘备 关兴、张苞,

关兴、张苞 在

刘备 命你二人急速前去保护。倘若老将军得胜,劝他回营,不得有误。

关兴、张苞 得令。

刘备 将宴撤去。

刘备
【西皮摇板】得意忘形错是朕,激怒老将黄汉升。但愿他马到成功呃早得胜,平安无事转回程。

{[}第三场{]}

黄忠 【西皮导板】黄忠马上呵呵笑,

黄忠 哈哈,哈哈,啊呵呵哈哈\ldots{}\ldots{}(笑介)

黄忠
【西皮快板】主公帐中论英豪。溺爱不明夸年少,反道老将无略韬。只要杀人胆量好,哪怕胡须似银条。催马来在阳关道,

关兴、张苞 老将军慢走。

黄忠 【西皮摇板】二小将赶来为哪条。

关兴、张苞 老将军且慢。

黄忠 二位小将赶来则甚?

关兴、张苞
(我等奉了)皇伯之命,请老将军回营,(诚恐)年迈有失。\protect\hyperlink{fn201}{\textsuperscript{201}}

黄忠 呀呸!

黄忠
【西皮快板】二小将把话讲差了,讲什么阵前把命抛。我也不图凌烟标,恢复汉室锦皇朝。见了主公好言告,你就说年迈的黄忠要立功劳。

张苞 【西皮摇板】黄忠年迈性情傲,

关兴
【西皮摇板】(相随)保护莫辞劳。\protect\hyperlink{fn202}{\textsuperscript{202}}

{[}第四场{]}

吴班
【西皮摇板】大将出川把贼剿,挂印先行不辞劳。连营下寨恐非妙,见机而行稳重高。

报子 报!

报子 黄老将军到。

吴班 有请。

吴班 啊,老将军。

黄忠 哼!

吴班 黄老将军,怒气不息,为着何来?

黄忠 呃!

黄忠 \textless{}\textbf{叫头}\textgreater{}吴将军,

黄忠
想那关兴、张苞乃子侄之辈,阵前擒来谭雄,无非是些许功劳。主公帐中(或:主公隆宠,)夸了又夸,讲了又讲。反讲当年五虎上将尽是些老迈无用。你道恼是不恼?

吴班 哎,本来的老了哇。

黄忠 啊?(或:呀呸!)

黄忠 【西皮摇板】为什么人人道我老哇,

吴班 唉,本来是老了。

黄忠 呀呸!

黄忠
【西皮快板】不由怒气上眉梢。吾十岁(或:某十岁)弓马颇知晓,十三、十四使宝刀。交锋对垒有多少,数十年未离马鞍鞒。战长沙已然须发皓,取东川谁不道我是英豪。我也曾天荡、定军一齐扫,夏侯渊一命赴阴曹。到如今八十三岁何曾老,我是哪些儿老,

吴班 老将军本来是老了啊。

黄忠 呃!

黄忠 【西皮快板】年迈也要逞英豪。来来来与爷带马到,斩几个人头你瞧一瞧。

吴班 【西皮摇板】老将人老心不老,

吴班 带马。

吴班 【西皮摇板】暗地保护走一遭。

{[}第五场{]}

崔禹、史蹟 俺,东吴大将------

崔禹 崔禹。

史蹟 史蹟。

崔禹
我等奉了吴侯旨意,镇守猇亭。探子报道,黄忠前来讨战,你我二人前去会他一会。

史蹟 请。

黄忠 来将通名!

崔禹 哼,连你家老爷东吴大将崔禹全不认识?

黄忠 通上名来。

崔禹 某乃东吴大将崔禹。

史蹟 俺乃史蹟。哇呀呀呀\ldots{}\ldots{}你还不跑?

黄忠 诶呀!我道是东吴八员上将,原来是两个无名的狗头。

崔禹、史蹟 什么狗头,这是人头。

黄忠 饶尔等不死,去罢!

崔禹、史蹟 什么人头、狗头的,你这老头儿叫什么名字?

黄忠 老夫黄忠。

崔禹 咦咦咦,

史蹟 哇啊啊------

崔禹、史蹟 呵呵哈哈哈\ldots{}\ldots{}(笑介)

黄忠 尔为何发笑?

崔禹、史蹟 呵呵黄忠啊,我道是天神下界,原来是一个老倭瓜。

黄忠 休得胡言,快(快)教那潘璋出马,饶尔等不死。去罢!

崔禹 我这里待我耍个``提枪花'',摘就一个老倭瓜。

崔禹、史蹟 呵呵将军呐。

(崔禹
人道黄忠乃是好将,未战两个回合,他为何败下阵去。)\protect\hyperlink{fn203}{\textsuperscript{203}}

史蹟 想是不忍杀害于你。

崔禹 哼,休得胡言,你我赶上前去。定然死在他手。

{[}第六场{]}

黄忠 啊?!

吴班 老将军刀劈崔禹、史蹟,就是莫大之功,可以回营交令了。

黄忠
俺要去至吴营,斩那东吴八员上将,看看我黄忠老是不老(或:俺要去至吴营,斩那东吴八员上将,方显我黄忠不老)。

吴班 唉,老将军呐,

吴班
【西皮摇板】老将军威风谁不晓,破敌须防战马劳。\protect\hyperlink{fn204}{\textsuperscript{204}}

黄忠 吴将军,

黄忠
【西皮摇板】这几句话儿讲得好,黄忠的怒气一半消。回营报功呃休取笑,暂且饶他这一宵。

吴班
老将军不如请暂回师。\protect\hyperlink{fn205}{\textsuperscript{205}}

黄忠
(啊)吴将军,你我今日暂回大营(或:你我暂且回营),教那些吴狗们多活上一夜。

吴班 是啊,教他们多活上一夜。

黄忠 便宜了他们。

吴班 呃,便宜了他们。

黄忠 啊吴将军,你看我黄忠老是不老?

吴班 呃,将军么------嗯,不老。

黄忠 嗯,不老?

吴班 呃,不老。

黄忠、吴班 啊,呵呵哈哈哈\ldots{}\ldots{}(笑介)

{[}第七场{]}

潘璋
【西皮摇板】探马不住急来报,黄忠斩我两英豪。\protect\hyperlink{fn206}{\textsuperscript{206}}

潘璋
俺,潘璋。前者同吕蒙定计袭取荆州\protect\hyperlink{fn207}{\textsuperscript{207}},我主大喜,将关羽刀、马赐俺,赤兔马不食草料而死;青龙刀虽在我手,却未斩一将。适才探子报道,黄忠踏营,岂肯容他张狂,待俺擒他便了。

黄忠 来将通名。

潘璋 东吴大将潘------

黄忠 潘什么?

潘璋 潘璋。

黄忠 啊!

黄忠
【西皮快板】一见潘璋把牙咬,手持青龙偃月刀。怎不教人珠泪掉,斩尔的狗头马后捎。

{[}第八场{]}

马忠
(念)旌旗飞龙影,干戈耀日月。\protect\hyperlink{fn208}{\textsuperscript{208}}

马忠 俺,马忠。只因潘璋出营,大战黄忠,不知胜负如何,俺且出营一望。

马忠 将军胜负如何?

潘璋 黄忠十分骁勇,难以取胜。

马忠 将军且退后阵,待俺前去会他。

潘璋 多加小心。

{[}第九场{]}

潘璋 将军。

马忠 将军。

潘璋 你我被黄忠杀败,主公降罪如何是好?

马忠 黄忠虽然骁勇,潘将军你且与他交战,待俺暗放一箭。

潘璋 黄忠善射,百步穿杨,若是射他不中,只恐你``画虎不成反类其犬''。

马忠 岂不知``会家不防''?

潘璋 既然如此,待俺再会他一阵。

马忠 须要小心。

{[}第十场{]}

黄忠 【西皮导板】黄忠今日遭圈套,

黄忠 【西皮快板】中了奸贼计笼牢(或:谅我插翅也难逃)。

黄忠 【西皮快板】大将临危有神保,

关兴、张苞 【西皮快板】来了关兴和张苞。

(潘璋
黄忠带箭,被二小将救出重围,你我速速追赶。)\protect\hyperlink{fn209}{\textsuperscript{209}}

马忠 赶上前去。

{[}第十一场{]}

刘备
【西皮摇板】黄忠性傲见识浅,不该匹马去争先。张苞、关兴料难劝,但愿平安得胜还。

黄忠
(念)\textless{}\textbf{金钱花}\textgreater{}渭城朝雨清尘、清尘;轮台古月黄云、黄云。催花羯鼓去从军。枕头上,别情人;刀头上,做功臣。

刘备 【西皮散板】一见老将身带箭,霎时胆落百丈渊。早知出兵遭凶呃险,

刘备 \textless{}\textbf{哭头}\textgreater{}将军呐------

刘备 【西皮摇板】朕悔一时错出言。

黄忠 【西皮散板】精神恍惚四肢软,耳旁又听有人言。大骂潘璋休弄呃险,

刘备 老将军!

黄忠 【西皮散板】只见主公在眼前。急忙叩谢龙恩典,黄忠的性命难保全。

刘备
唉呀老将军呐,朕一言之错,使你怒出大营,如今带箭而归,教朕痛断肝肠了哇啊\ldots{}\ldots{}

黄忠 哎呀主公啊,老臣出马,刀劈史蹟、崔禹------

刘备 就该回营。

黄忠
因见吴狗潘璋手持二君侯青龙宝刀,老臣一见,肝胆俱裂。正要擒贼下马,不想贼营暗放冷箭,中臣肩窝。

刘备 啊------

刘备 啊,老将军乃是善射的能手,为何不防?

黄忠 哎呀陛下呀。

黄忠
【西皮散板】老臣智不如王翦,临阵怎敢不当先。况且仇人两相见,心急哪顾听弓弦(或:哪有闲心听弓弦)。

黄忠 此乃老臣自不小心。

刘备
【西皮散板】真是风云不测变,空将血泪洒胸前。回头便把小将怨:年轻无知小儿男。

关兴、张苞 儿臣等知罪。

黄忠 陛下,此乃臣自不小心,休要埋怨二位小将军。

刘备 既然如此,待朕与老将军起箭。

黄忠 哎呀万岁呀,这箭上有药,箭在------臣在,这箭去------臣亡。

刘备
老将军带箭不起,那敢是怕痛?\protect\hyperlink{fn210}{\textsuperscript{210}}

黄忠 老臣死且不惧,焉能畏痛?一言永别,伏乞圣听:

黄忠
【反西皮二六】平生今洒泪几点,回首功名八十年。主上待臣恩非浅,粉身碎骨理当然。幸得全尸已无怨(或:幸得全身已无怨),叩谢圣恩归九泉。万岁须当谋虑远\protect\hyperlink{fn211}{\textsuperscript{211}}(或:主上须当韬略远;主上须当\textless{}\textbf{哭头}\textgreater{}谋略远),

黄忠 【西皮散板】平吴不及定中原。

刘备
【西皮散板】老将军休得心惊战,起箭医疗早愈痊。康复之后功臣宴,愿你康宁寿百年。

黄忠
【西皮散板】见主公说话(或:见主公只哭得)泪满面,关兴、张苞哭两边。大丈夫一死终难免,强打精神假流连。

刘备 【西皮散板】事到临头难挽转,张苞、关兴听朕言:

刘备 关兴、张苞,搀扶老将军,待朕与老将军起箭。

黄忠 且慢呐,大将取箭,不用人搀,待老臣自取。

黄忠 闪开了!

黄忠 唉呀------

刘备
【西皮散板】一见老将归九天,冷水浇头落空潭。从今何处再相见,\protect\hyperlink{fn212}{\textsuperscript{212}}

刘备 \textless{}\textbf{哭头}\textgreater{}老将军呐------

刘备 【西皮散板】热泪行行洒征衫。

张苞 【西皮散板】大将尸全世少见,

关兴 【西皮散板】皇伯不必损龙颜。

张苞 【西皮散板】尸首后帐好收殓,

关兴 【西皮散板】准备灭吴报仇冤。

刘备 【西皮散板】五虎大将三不见呐,

(刘备 \textless{}\textbf{三叫头}\textgreater{}汉升!二弟,三弟呀!)

刘备 【西皮散板】休想古城再团圆。黄忠有灵当应显,踏平东吴在眼前。

刘备 【西皮散板】张苞、关兴传令箭,

刘备 拿潘璋------

刘备 【西皮散板】刀出鞘来弓上弦。

\textbf{*王荣山教《定军山》、《阳平关》和《伐东吴》大刀把子}

王荣山说《定军山》、《阳平关》和《伐东吴》三出黄忠戏戏情不同,大刀把子不同,很明显的是《阳平关》有大战,打挡棒攒,其它两出没有,实际上许多处都不一样。

《定军山》有个小``三股档'',用在黄忠见韩浩、夏侯尚那一场。这场是韩、夏侯正式奉令出马,黄也是正式迎战交锋。黄忠抖擞精神把韩和夏侯杀得大败而逃。这场开打不能大又不能小,太大显不出韩、夏侯弱,太小显不出黄忠勇。用这个``三股档''黄把韩和夏侯拨拉过来,拨拉过去,两合就连削带抓
他们打下去,紧接着黄来一个大刀花下场,表示奋勇追击。

《阳平关》黄忠见张郃、杜袭一场,也是三个人,但情节与《定军山》黄见韩、夏侯不同。张、杜是曹营名将,非韩、夏侯可及,黄忠夜半劫营,他们仓猝迎战,溃散之下,投奔曹兵主力开始大战,这场开打如果用《定军山》的三股档就不太合适,可以打``硬三枪''头子或其它套子。

《伐东吴》是黄忠一肚子气,拼了老命要斩东吴八员上将,潘璋虽勇但招架不住,因此潘、黄开打不要多,但黄要耍三个下场以示其奋勇冲杀深入重围。

《定军山》三股档(谭派打法)

刘砚芳介绍:

韩浩、夏侯尚在大边台口,韩里,夏侯外,黄上场门上,一指,向大边台口漫夏侯头,夏侯过小边外边,黄用刀鐏勾韩腰到大边外边(即一肘),再从下场门向小边外边漫夏侯头,夏侯又过大边,黄到小边,回身与韩穿肚左转身回来打夏侯鼻子,右转身削韩头,打夏侯靠旗,亮,接大刀花下场。

《阳平关》见张郃、杜袭硬三枪头子(王凤卿打法):

黄众人搭轿上到台口站齐,黄上时左手抱刀右手扶刀,到台口刀交右手平出刀,举左手数更,(白)``放起火来'',往里一砍,张著上手下,下手张、杜两边抄过合,黄从中间到小边,张郃留在大边,余者分下,一扯,两扯,剜萝卜黄到大边,拉转身,一枪、两枪、三枪,打张鼻子,转到里面打腰封,接背躬。向外漫张头,向里两穿,向外两盖,打张鼻子,用鐏勾杜袭上,从中间向外起大刀花在台中间被压住,搅起来,用鐏勾杜腰过小边(即一肘),黄归大边,在二人中间一合到小边,拉肚转身,鼻子,削头,抓靠旗,看左拳,看马后,左手指,耍下场。\textbf{看拳和马后是看擒着人没有},\textbf{指是指张}、\textbf{杜跑了},\textbf{耍下场是猛追}。

\textbf{王荣山说这一场也可以少打些},打法是:

剜萝卜黄归大边后,张刺黄一压,打张腰封,勾杜上,一压,搅起来,用鐏勾杜腰过小边(即一肘),黄归大边,中间过合,穿肚往里转,鼻子,削头,抓靠旗,亮,耍下场下。

《阳平关》挡棒攒头子(王凤卿打法):

第一场曹操上,唱,上桌唱毕,黄上场门上出刀被徐晃漫头过小边,一滑,打上下左右,鐏一盖,打后蓬头。拉转身,搕,回花转身,大刀花转身搕,下叉,从里面漫头过去到大边,打鼻子,勾王平上,下接打棒攒,亮下场门下。

第二场曹又唱毕,黄下场门上,出刀,被徐晃漫头,搕反抄勾王平上,下接反攒,亮上场门下。

第三场,曹唱毕,黄内导(/倒)板,边唱上场门上,一亮,一趱子,台口刀头朝后洒,拨拉众上,到小边,一、二大扯,推到中间架住,唱。

\textbf{王荣山说前边不打硬三枪},\textbf{则接攒第一场可用硬三枪},打法是:

黄上场门上,晃漫头上黄归小边,一兜往里转身,一枪、两枪、三枪,绕刀鐏一盖前蓬头,一盖后蓬头,拉肚转身,漫头过去到大边,晃刺,黄压打腰封,左转身勾王平上,下略。

二场是黄下场门上,晃漫头,一勾打晃腰封,反勾王平上,下略。

三场,唱上一亮,一趱子,三甩胡,出刀拨拉众上,领起来由大边到小边,一指,左转身向里一合,向外两合,向里架住,唱。

《伐东吴》三个大刀花下场(王荣山演此戏用):

\textbf{第一个}:即常用的大刀花下场,出刀,三个正花转身过大边,串腕回花转身,正花转身,大刀花,劈马,正花转身,面外,亮住,串腕转身,出刀,转身背刀,弓箭步,向里亮下。

\textbf{第二个}:开始同第一个,到亮住,串腕转身,出刀,用右手背垫刀反手接刀,脸前绕大圈,平着往左齐腰横砍,撤右脚,向外正面正花转向大边,亮住,串腕右转身,向外正面出刀交左手反蹦子转身弓箭步左手握刀杆中间向外亮,下。

\textbf{第三个}:开始同第一个,到劈马,再耍三个正花转身到小边,大刀花劈马,正花转身到大边,面向小边左右两涮刀,右转身右手持刀撕开,刀上膀子左转身,刀交左手串腕,在大边弓箭步左手握刀杆中间向外亮,大绕下场门下。

\textbf{陈超老师说明:}

刘曾复先生传授的贾洪林配演刘备的词很讲究,而且很重要,兹举两例:

刘备念``昔年随朕开基创业之将,死的死了,亡的亡了'',而黄忠会错意,认为``开基创业之将''就是``五虎上将''。刘备没有针对,因此不念``昔年五虎上将,死的死了,亡的亡了''也不至于失言至此。

刘备见谭雄不唱``孤与孙权冤仇深\ldots{}\ldots{}''而是``四百年来争汉鼎,东吴不君也不臣。''

一句道出了刘备伐吴的真正原因,以报仇为借口,灭吴统一。

潘璋是东吴八员上将,黄忠再勇也不至于一个``扫头''就落荒而逃。因此有些演法是黄忠、潘璋对刀。

谭鑫培、余叔岩认为不能对刀的原因是:孙权将青龙刀赏赐潘璋,而潘璋并不会使青龙偃月刀。特别值得一提的是,钱金福为潘璋设计的开打,总是用刀鐏杵,不用刀头砍。加之黄忠勇武,因此一个\textless{}\textbf{扫头}\textgreater{},潘璋落花流水。

\newpage
\hypertarget{ux8fdeux8425ux5be8-ux4e4b-ux5218ux5907}{%
\subsection{\texorpdfstring{连营寨\protect\hyperlink{fn213}{\textsuperscript{213}}
之
刘备}{连营寨213 之 刘备}}\label{ux8fdeux8425ux5be8-ux4e4b-ux5218ux5907}}

{[}第一场{]}

(诸葛瑾
【西皮摇板】奉王命行路程不耽时候,此一番到蜀营讲和罢休。但愿得此一去旗偃歌奏\protect\hyperlink{fn214}{\textsuperscript{214}},免生灵遭涂炭民死蜉蝣。)

{[}第二场{]}

【西皮摇板】孙仲谋与孤王结成仇寇,只杀得他兵和将尸堆山丘。望空中二贤弟神灵保佑,灭却了东吴贼方肯罢休。

有请。

平身。

请坐。

到此乃依\protect\hyperlink{fn215}{\textsuperscript{215}}(或:乃是)客位,有话叙谈,哪有(或:焉有)不坐之理?(请坐。)

子瑜远来,有何事故?

哼,汝东吴现在危急,故命汝以巧言来说和。

住口!

汝东吴不仁,杀弟之仇,不共戴天。欲朕罢兵,哼哼,(或:汝东吴诡谋,损孤二弟,此仇不共戴天,欲孤罢兵,)除死方休!

不看我家丞相之面,先斩汝首(或:定斩汝首)。今且放汝回去,说与孙权,洗颈就戮。(或:今且放你回去,说与孙权,教他洗颈待戮。)

去罢!

(诸葛瑾 【西皮摇板】适才间在蜀营申述利害,见主公定良谋好把兵排。)

{[}第三场{]}

关兴、张苞,传令吩咐:满营大小将官,俱穿孝服。将你父等灵牌请在灵堂,一概仇人绑好。(或:关兴、张苞,打扫灵堂,安放灵位。满营将官,俱穿孝服。将一干人犯绑至灵堂。)为伯亲自祭奠\ldots{}\ldots{}(哭介)

摆驾!

【西皮摇板】想当年结桃园同天发咒,愿同年同月日(或:同日月)同刻罢休。到如今一旦间死别分手,孤岂肯独一人乐享无忧。

【西皮导板】白盔白甲白旗号,

\textless{}\textbf{哭头}\textgreater{}二弟呀,三弟呀!啊\ldots{}\ldots{}

【回龙】孤的好兄弟!

【西皮原板】满营将官哭嚎啕。孤王兴兵把仇报,扫灭了东吴恨方消。请过了神牌怀中抱,

【反西皮二六】点点珠泪往下抛。当年桃园结义好哇,胜似一母共同胞。不幸徐州失散了,万般无奈暂且降曹。那曹操待你的情义好,上马金银也曾赠过你锦袍(或:赐过你锦袍)。美女十名你不要,挂印封金辞奸曹。匹马单刀保皇嫂,过五关斩六将擂鼓三通把蔡阳的首级枭,可算得盖世的英豪。华容道上放曹操,大仁大义志量高(或:亘古流表\protect\hyperlink{fn216}{\textsuperscript{216}})。单刀赴会天下晓,英雄美名亘古标(或:志量高)。可恨(那)孙权行计巧,害孤二弟归天曹。愚兄兴兵把仇报,扫平了东吴气才消。还望二弟神灵保,

【西皮散板】神灵呐保,

\textless{}\textbf{哭头}\textgreater{}孤的好兄弟呀,(或:二弟呀,)

【西皮摇板】不灭孙权不回朝(或:不还朝)。

【西皮摇板】非是为伯伤心泪掉,孤与你父(或:我与你父)生死交。哭罢了二弟把三弟叫,

\textless{}\textbf{哭头}\textgreater{}翼德(弟)呀,桓侯哇,啊,孤的好兄弟呀,

【反西皮二六】叫声三弟听根苗:大破黄巾天下晓,敌人见你望风逃。虎牢关曾把吕布的发冠挑,长坂坡前喝断当阳桥(或:喝断灞桥)。夜战马超胆气好,义释严颜颇有略韬。可恨那范疆、张达两个贼强盗,谋害英雄二贼脱逃。愚兄兴兵与你把仇报,只杀得孙权魄散魂消。情愿罢兵写降表,同心合意共灭奸曹。锦绣山河孤不要,一心与你把仇消(或:一心只想把仇消)。哭哑了咽喉把三弟叫,把三弟\textless{}\textbf{哭头}\textgreater{}叫,豹头环眼的三弟呀,

【西皮摇板】拿住孙权两开销。

看酒来,待孤亲自祭奠。

(两弟受孤一拜。)

儿要多拜几拜。(或:儿等多拜几拜。呃\ldots{}\ldots{}(哭介))

一概仇人(或:将一干仇人),拿去开刀。

啊?

此贼为何不斩?

哦------剑来!

【西皮散板】到此时(或:到如今)还讲什么郎舅之分,献荆州贼有何亲戚之情。三尺剑正国法又报仇恨,死眼前看贼子你埋怨何人。

关兴、张苞,吩咐文武官员(或:传孤旨意,满营将官),歇兵三日,兵发东吴。

正是:(念)满腔怒气冲昊天,誓把(或:要把)东吴踏平川。

{[}第四场{]}

(念)起居梦寐恨吴寇,不报冤仇誓不休。

孤自兴兵以来,(势如破竹,)吾国(或:我国)人马屡屡得胜,他邦兵将(或:他国兵将)节节败溃。那些吴狗们望风而逃,此乃诸将之功也。

(众 主公妙计,臣等何功之有?)

诸将俱各有功。

(马良 启奏主公,闻得孙权又拜陆逊为都督,兵扎猇亭。)

坐下。

(马良 谢座。)

陆逊何路人也?

(马良
乃九江太守陆骏\protect\hyperlink{fn217}{\textsuperscript{217}}之后。)

哼,懦弱书生,统领人马(或:担此重任),岂不贻笑大方?

(马良
那陆逊虽是一介书生年幼,前番吕蒙白衣渡江,暗取荆州,乃此人之计也。)

哦!竖子诡谋(或:孺子诡谋),损孤二弟,今当擒之!

关兴、张苞,传令进兵。

(马良 且慢!主公请息龙怒。)

为何拦阻?

(马良 陆逊智胜周郎,不可轻敌。)

唉!孤用兵老矣!岂反不如一黄口孺子么?

(马良 如今陆逊不战不退,莫非有何诡计?)

哼!黄口孺子,有多大能为?既敢当此重任,就该领兵前来,与孤对敌。战又不战,退又不退,其情可恼!

(马良 倘若陆逊以逸待劳,如之奈何?)

孤兵精粮足,与他对守何惧(或:对垒何惧)?

(马良
堪堪天气炎热,暑气难当,兵扎离火之中,汲水不便。又恐将士多生疾病。)

不妨,孤将营寨,移于茂林深处(或:孤将人马,移至茂林深处),待(等)过夏到秋,并力进兵,东吴自然休矣(或:吴国可图也)。

(马良 我若兵动,倘陆逊踏营,如何是好?)

孤命关兴、张苞各带人马(或:带领精兵),埋伏山谷之中。倘陆逊来击,引兵突出,孺子可擒也。(或:倘陆逊劫营,我军伏兵杀出,一鼓擒之。)

(众 主公妙计,臣等不及也。)

(马良 主公要移营寨,可画成地图,问过丞相?)

孤亦颇知兵法,此事何必又问(或:再问)丞相?

(马良 古语云:兼听则明,偏听则暗。望陛下详之。)

也罢。卿可自去各营,画成四至八道\protect\hyperlink{fn218}{\textsuperscript{218}}图本,亲到东川,去问丞相。若有不便,即来回奏。孤再作裁处。(或:如此就命你将山势、营盘画成图本,去至东川,送与丞相观看。倘有不到,急速回来,孤再作裁处。)

(马良 领旨。)

关兴、张苞,就此移营者(或:择吉移营者)。

正是:(念)龙麟启祚\protect\hyperlink{fn219}{\textsuperscript{219}}如反掌,干戈霸业定太平。

{[}第五场{]}

(念)月当空乌鸦嘶叫,帅字旗无风自摇(或:无风自飘)。

(报子 报!东吴人与马缓缓移动。)

再探。

(此乃疑兵。)

啊,沙摩柯听令。

(沙摩柯 在。)

带领蛮兵女将前去探看虚实,相机击之。

(沙摩柯 得令。)

此乃疑兵,何足道哉?

(报子 报!江北营中火起。)

再探。

关兴前去救火。(或:关兴营救。)

(关兴 得令。)

江北营中火起,(此乃)我军自不小心。

(报子 报!两岸火起。)

再探。

张苞去救。(或:张苞急救。)

两岸火起,我军大不利也。

(报子 报!满营火起。)

(再探。)

不、不、不\ldots{}\ldots{}不好了!

(念)\textless{}\textbf{蛮牌令}\textgreater{}看、看、看,看呐,风助火威狂,火乘猛风飏。满天飞烈焰,遍地闪金光(或:撒金光)。祸从天降,祸从天降呃。寻不出路当央\protect\hyperlink{fn220}{\textsuperscript{220}},寻不出路当央。快带丝缰,快带丝缰。

{[}第六场{]}

(赵云 赵云接驾。)

哎呀四弟呀!你看孤被他们烧得乌焦巴弓了。(或:四弟你来了,杀出重围。)

(赵云 主公保重。)

(四弟,)孤命休矣!

杀呀。

(赵云 杀呀。)

(赵云 赵云救驾来迟,死罪呀死罪呀!)

唉,孤虽得脱,诸将奈何!

(赵云 由臣断后。)

前面什么所在?

(赵云 乃是白帝城。)

兵撤白帝城。

带马。

\textless{}\textbf{三叫头}\textgreater{}二弟,三弟,唉!兄弟啊(或:贤弟呀)\ldots{}\ldots{}(哭介)

(罢!)

(赵云 马僮,抬枪带马。)

\textbf{(安居)平五路}\protect\hyperlink{fn221}{\textsuperscript{221}}

{[}第一场{]}

(打朝,末扮贾诩、外扮辛毗、副扮曹真、净扮司马懿,上)

贾诩 (念)自古良禽择木栖,

辛毗 (念)而今喜得拜丹墀。

曹真 (念)男儿须当封侯印,

司马懿 (念)正是英雄得志时。

贾诩 请了。

辛毗、曹真、司马懿 请了。

贾诩 今日万岁升殿,必有军情议论。

辛毗、曹真、司马懿 大家分班伺候。

贾诩、辛毗、曹真、司马懿 请。

(四太监、一大太监,\textless{}\textbf{小开门}\textgreater{},小生扮曹丕\protect\hyperlink{fn222}{\textsuperscript{222}}上)

曹丕 {[}引子{]}驾坐朝阁受三分,重整山河。

曹丕
(念)献帝无福民不安,人心归朕乐尧天。上苍若肯遂孤愿,扫平东吴灭西川。

曹丕
寡人曹丕,国号黄初在位。蒙众卿忠勇,扶孤禅位,更改国号。深感上天之福佑也。朕闻刘备兵伐东吴,中了陆逊火攻之计,败入白帝城,气忿身亡。朕闻此信心无忧矣。孤有心攻取西川,我想必获全胜。众贤卿。

贾诩、辛毗、曹真、司马懿 万岁。

曹丕
朕想刘备新亡,乘他国内无主,人心未定,攻取西川,蜀可得矣。卿等意下如何?

贾诩 臣贾诩奏闻陛下。

曹丕 当面奏来。

贾诩
臣想刘备虽亡,必托孤与诸葛亮,那孔明感刘备知遇之恩,必要倾心竭力扶持嗣主,陛下不可轻伐。

司马懿 臣司马懿有本启奏。

曹丕 卿有何良谋,奏与朕知。

司马懿 西蜀新败,休容他养成锐气,若不乘此发兵,待等何时?

曹丕 卿言正合孤意,当用何计?

司马懿
若用中原之兵,恐难取胜。须用五路大兵,四面攻打。那诸葛亮首尾不能相顾,西川之地,必然唾手可得。

曹丕 哪五路呢?

司马懿
可修国书遣使臣去到鲜卑国见那国王轲比能,贿以金帛,令他起羌兵十万\protect\hyperlink{fn223}{\textsuperscript{223}}攻打西平关,此一路也;

曹丕 二路呢?

司马懿
差人直入蛮洞买通蛮王孟获,领起蛮兵十万攻打益州、永昌四郡,此二路也;

曹丕 三路呢?

司马懿
再遣能言使臣入吴和好\protect\hyperlink{fn224}{\textsuperscript{224}},许以割地为约,领起吴兵十万入峡口取涪城,此三路也;

曹丕 四路呢?

司马懿 急调孟达起上庸兵十万攻打汉中,此四路也;

曹丕 那五路呢?

司马懿
就命大将军曹真起中原大兵十万攻打阳平关,此五路也。五路大兵共五十万,併力攻取西川,那孔明纵有吕望之才,难逃五路雄兵也。

曹丕
此本奏之有理,孤王依计而行。曹真进位\protect\hyperlink{fn225}{\textsuperscript{225}}。

曹真 万岁。

曹丕 卿领大兵十万攻取阳平关,得胜回朝,另加升赏。

曹真 领旨。

曹真 (念)金殿领君命,校场排雄兵。

(曹真下)

曹丕 退班\protect\hyperlink{fn226}{\textsuperscript{226}}。

(众分下)

{[}第二场{]}

(二小军抬杠箱上,丑扮差官跳上)

差官
吾乃北魏国王驾下差官是也,今奉我主之命押解礼物去到鲜卑国,聘请国王轲比能,今起羌兵十万攻打西平关。身奉君命,不敢怠慢。军士们。

二丑小卒 有话说罢。

差官 快趱行。

(差官倒退跳走\textbf{干\textless{}度柳翠\textgreater{}}\protect\hyperlink{fn227}{\textsuperscript{227}},\textless{}\textbf{挑子}\textgreater{}下)

{[}第三场{]}

(报子上)

报子 马来。

报子 (念)胆量天生就,应变广机谋。探访邻邦事,名称夜不收。

报子
吾乃西蜀远探是也,探得曹丕发兵五十万,五路进兵攻取西川。探得真实,不免连夜飞报丞相知道便了。

(报子下)

{[}第四场{]}

(生扮诸葛亮上)

诸葛亮
【西皮\textbf{原}板】天命归人心归天时地利,一朝君一朝臣争夺华夷。西川地到而今虽归我主,普天下皆王土汉室地基。

诸葛亮 (念)铺谋设计正朝纲,国事纷纷费心肠。

诸葛亮
山人诸葛亮,字孔明,道号卧龙。只因\protect\hyperlink{fn228}{\textsuperscript{228}}先皇伐吴失利败入白帝城,气忿成疾,晏了圣驾。蒙托孤之重,扶保幼主登了龙位,安稳民心。可恨曹丕篡了汉位,更改国号。本当发兵问罪,怎奈我兵新败,不敢轻举妄动,待等兵精粮足再去发兵问罪。正是:(念)只为托孤恩义重,披肝沥胆报国恩。

(报子上)

报子 (念)探听北魏军情事,报与西蜀丞相知。

报子 来此已是府门,里面哪位在?

(副扮听事官上)

听事官 (念)侯门深似海,不许外人来。

听事官 什么人?

探子\protect\hyperlink{fn229}{\textsuperscript{229}} 探子要见丞相。

听事官 候着。探子求见丞相。

诸葛亮 传。

听事官 探子,丞相传,小心了。

探子 啊。探子叩头。

诸葛亮 探子,你探听哪路军情,一一讲来。

探子
相爷容禀:\textless{}\textbf{五字赞}\textgreater{}探子禀军情,相爷在上听:曹丕人马踴,五路起雄兵:中原兵十万,曹真攻阳平;上庸发人马,孟达取汉中;孙权入峡口,大兵攻涪城;西平羌兵众,国王轲比能;南蛮名孟获,四郡恶交锋。五路貔貅猛,十万虎狼兵。声如地裂山摇动,要把西川一扫平。

诸葛亮
赏你银牌一面,休叫成都\protect\hyperlink{fn230}{\textsuperscript{230}}军民知觉,不可走漏我的消息。

探子 谢相爷。

(探子下)

诸葛亮
这厮好生\protect\hyperlink{fn231}{\textsuperscript{231}}可恶,,明知我国老王驾崩,新君年幼,乘我国丧,兵发五路来克我国,如此猖狂,我自有道理。听事官。

听事官 在。

诸葛亮 传四路旗牌进府听令。

听事官 是。相爷有令:传四路旗牌进府听令。

(外扮北路旗牌,副扮西路旗牌,末扮南路旗牌,净扮东路旗牌;四旗牌内应,上)

四旗牌 (念)丞相来呼唤,忙步到府堂。

四旗牌 四路旗牌参见丞相。

诸葛亮 你等免参,听我分派。

四旗牌 愿听丞相军令。

诸葛亮
今有曹丕兵发五路攻取西川,要你等飞递边关,不叫成都军民知晓,莫要泄漏我的机关。

四旗牌 谨遵丞相军令。

诸葛亮 北路旗牌听令。

北路旗牌 在。

诸葛亮
命你赶到阳平关通知赵云,叫他暗设人马,不可交锋,待贼粮尽自退,督兵追杀,不得违误。

北路旗牌 得令。

(北路旗牌下)

诸葛亮 西路旗牌听令。

西路旗牌 在。

诸葛亮
命你赶到西平关急报马超,叫他虚立自己旗号,羌兵必不敢战,容他自退,不得违令。

西路旗牌 得令。

(西路旗牌下)

诸葛亮 南路旗牌听令。

南路旗牌 在。

诸葛亮
我有令箭一支,内有柬贴封好,赶到南郡交付魏延,叫他依计而行,不可错误。

南路旗牌 得令。

(南路旗牌下)

诸葛亮 东路旗牌听令。

东路旗牌 在。

诸葛亮 命你前去急调关兴、张苞各带汉中人马三万,四面接应,不得违误。

西路旗牌 得令。

(西路旗牌下)

诸葛亮
那上庸兵乃孟达督帅,不用劳动军卒,管叫他不战自退。我料东吴孙权必要兵扎三江口,虚作人情,静观两家胜败,就中攻取,事虽如此。怎奈我先皇昭烈兵伐东吴,结下仇怨,并未和解。我若兴兵伐魏,吴必攻取西蜀;昼夜思想,不得其人入吴和好\protect\hyperlink{fn232}{\textsuperscript{232}}。若得吴、蜀和好,结为唇齿,然后兴兵伐魏,也免我忧虑东吴之患也。

诸葛亮
【西皮\textbf{原}板】平生恨篡国贼欺君万恶,心想要灭贼子枉自揣摩。我本该去问罪天不容我,一桩桩一件件国事阻隔。到如今曹丕贼心威赫赫,乘国丧兵五路侵佔我国。西蜀中现有我区区诸葛,岂肯容贼猖獗奏唱凯歌。参想想和东吴长久计策,缺少个能言士前去说说。叹先皇心愿事敕命于我,灭国贼尽人力天意如何。居相位守臣节日日思索,我怎能负先皇临危重托。

(诸葛亮下)

{[}第五场\protect\hyperlink{fn233}{\textsuperscript{233}}{]}

(四旗牌同上)

北路旗牌 列位请了。

三路旗牌 请了。

北路旗牌
今有曹丕兵发五路,攻取西川,你我奉了丞相军令,通知各路依令而行。军情紧急,分路投递。正是:(念)将军不下马,

三路旗牌 (念)各自奔前程。

{[}第六场{]}

(四文堂、一中军站门,孟达上)

孟达 {[}引子{]}只为一着错,满盘棋势空。

孟达
(念)昔侍蜀君今侍魏(或:昔仕蜀君今仕魏),俱是三呼称万岁。叹想原郡故乡土,谁到坟前化纸灰。

孟达
俺,孟达,昔在汉中称臣,为事不平弃蜀投魏,命俺镇守上庸等处。日前圣旨到来,命俺起兵十万攻取汉中。我想永安宫乃李严镇守,我若攻打,有碍生死之交;如不攻打,又恐魏王识破疑我,好不两难也。

(副扮下书人上)

下书人 (念)四季关银饷,一年走慌忙。

下书人 来此已是,营门有人么?

中军 什么人?

下书人 永安宫李严差人下书。

中军 候着。启禀帅爷:永安宫李严差人下书。

孟达 传他进帐。

中军 是。下书人里面传你,小心了。

下书人 是。下书人叩头。

孟达 你奉何人所差?

下书人 奉永安宫李老爷所差,有书呈上。

孟达 后营用饭。

下书人 领爷赏赐。

(下书人下)

孟达 待我看来。

(孟达看信,起\textless{}\textbf{牌子}\textgreater{})

孟达 来,传下书人。

中军 下书人。

(下书人上)

下书人 (念)后营用罢饭,帐下听回音。

下书人 谢爷的酒饭。

孟达 回覆你爷:我这里修书不及,照书行事。

下书人 是,小人记下了。

(下书人下)

孟达
我正忧疑之间,李严有书到来,我岂忘了生死之交。不免假装重病,中军传令:帅爷偶得重病,暂将人马撤回,再听调用。

中军 得令。下面听者:

(内应介)

中军 元帅偶得重病,暂将人马撤回,再听调用。

众 (内)传令。

孟达 (念)谁人不思故乡土,洛阳虽好不如家。

孟达 唉哟哟,好不痛死人也。

(众搀扶孟达领下)

{[}第七场{]}

(四文堂、四水军众站门上,小生扮陆逊\textless{}\textbf{牌子}\textgreater{}上)

陆逊
吾乃东吴水军都督陆逊,今有北魏曹丕五路攻川,许以割地为约,令起大兵十万出峡口,攻打涪城。吾想吴、魏两国皆非诸葛之敌手,万难取胜。是我奏明主公,用两全之计,虚作人情,兵扎三江口,坐观胜败,就中取事。众将官------

(众应介)

陆逊 兵发三江口去者。

(\textless{}\textbf{牌子}\textgreater{}众原场)

众 前面已到三江口。

陆逊 安营下寨。

(众应介,同下)

{[}第八场{]}

(四朝官上,末扮许靖,白髯;外扮董允,黪髯;副扮杜琼,黑三;生扮邓芝,黑三)

许靖 (念)金钟响罢禁门开,

董允 (念)雨露恩深拜龙台。

杜琼 (念)常思汉鼎三分在,

邓芝 (念)灭魏伐吴待时来。

许靖 下官,司徒许靖。

董允 下官,黄门侍郎董允。

杜琼 下官,谏议大夫杜琼。

邓芝 下官,户部尚书邓芝。

许靖 请了。

董允、杜琼、邓芝 请了。

许靖 今日早朝圣驾登殿,必有国政议论。

董允、杜琼、邓芝 金钟三响,想是圣驾临朝。

许靖、董允、杜琼、邓芝 请。

(许靖、董允、杜琼、邓芝分班站介,四太监、一大太监站门,小生扮刘禅上)

刘禅 {[}引子{]}诏书赐孤王,驾坐成都称帝邦。

刘禅
(念)父皇白帝驾殡天,众卿扶保坐江山。但得吴、魏干戈定,永守西蜀心也安。

刘禅
孤刘禅,国号建兴,只因父皇兵伐东吴失利,兵退白帝城;圣驾殡天,托孤与诸葛丞相。扶孤登基,内理国政,外治民情,皆赖丞相之奇才也。今日早朝,内侍,展放龙帘。

大太监 领旨。

(黄门官上)

黄门官 (念)忙将动地惊天事,奏与君王御驾知。

黄门官 臣黄门官见驾,吾皇万岁。

刘禅 卿有何本奏?

黄门官 今有北魏曹丕兵发五路啊!( \textless{}\textbf{牌子}\textgreater{})

刘禅 既有此事,就命卿相府召丞相入朝理事。

黄门官 领旨。

黄门官 (念)五路雄兵起,三国战不息。

(黄门官下)

刘禅 适才黄门奏道:曹丕五路进兵,攻取我国。众卿,

众 万岁。

刘禅 有何良谋可退贼兵?

众
万岁圣意宽怀,暂请放心。\protect\hyperlink{fn234}{\textsuperscript{234}}待丞相入朝,必有良谋妙策。

刘禅 孤亦想到如此。

黄门官 (内)走啊!

(黄门官上)

黄门官 启奏万岁:丞相有病在府,不容进见,特来交旨。

刘禅 卿家暂退。

黄门官 领旨。

(黄门官下)

董允、杜琼 臣董允、杜琼同到相府求计,看有何说。

刘禅 二卿愿去,速来回奏。

董允、杜琼 领旨。

(董允、杜琼下)

众
诸事(已)毕\protect\hyperlink{fn235}{\textsuperscript{235}},请驾回宫。

刘禅 退班。

(众分班下)

{[}第九场{]}\protect\hyperlink{fn236}{\textsuperscript{236}}

(丑扮门官上,\textless{}\textbf{普贤歌}\textgreater{}干牌子)

门官
(念)剑戟峥嵘将相门,谁敢杂踏与高声。常闻细柳营,天子按辔行,何况文官与武臣。

门官
咱家诸葛丞相府下门吏便是。可怪我家相爷向来真正霄晓勿遑\protect\hyperlink{fn237}{\textsuperscript{237}},日理万机。近日何故,终朝不出内阁,一切政务不理。慢说羽书雪片,多官请事。就是方才圣明来召,也是推病不起,你想这样身价,可是亘古罕有的。闲言少说,只恐又有人来请见,我且坐守等候。

(董允、杜琼、二青衣扮随侍上)

董允 (念)宰正百官才独称,

杜琼 (念)仪型四海圣王尊。

董允 (念)仪门台下下了马,

杜琼 (念)好向阁内问安宁。

董允、杜琼 回避。

(二随侍下)

董允 看仪门肃静,人寂无声,同到门上。

杜琼 请。

董允 (念)月照牙旂肃,

杜琼 (念)风吹画角寒。

董允、杜琼 门官。

门官 原来二位大人,请坐。

董允、杜琼 请。

董允 我来问你,丞相往常勤于政事,近日不见升堂,是何缘故?

门官
小官不知(何)故\protect\hyperlink{fn238}{\textsuperscript{238}}。(念)宅门高挂止步牌,一切杂物何禀来。

门官 二位大人不曾见么:(念)门外朝事与边事,谁敢进内去相催。

董允 这却(为)何也?\protect\hyperlink{fn239}{\textsuperscript{239}}

杜琼 丞相连日起居如何,饮食可曾加减?

门官 这小官也曾打听:(念)起居倒觉不甚衰,肥肉三餐不吃斋。

董允、杜琼 既然身健食壮,缘何不出堂理事?

门官
据小官想来,相爷多管害心病\protect\hyperlink{fn240}{\textsuperscript{240}}。

董允、杜琼 甚么心病?

门官
二位大人,从来出将入相之家,不言歌童舞女成群,即便那娇妻美妾也却无数。可怜我们相爷就是一位黄夫人,心性却有姜嫄之德,其颜却如嫫母之陋。\protect\hyperlink{fn241}{\textsuperscript{241}}恁教相爷耐得住呢。

董允、杜琼 胡说。

门官 世情如此,不是小官妄说。

董允、杜琼 你可进去禀知说董允、杜琼特来问候金安,还有大事面启。

门官
哎呀呀\ldots{}\ldots{}相爷数日传谕:所有一应大小官员,勿得擅入禀事。方才圣命来召,尚尔\protect\hyperlink{fn242}{\textsuperscript{242}}辞去,何况大人。

董允、杜琼 我等今奉圣命而来,定要请见。

门官 既然如此,小官传禀便了。

(门官下)

董允
(念)似此葫芦闷\protect\hyperlink{fn243}{\textsuperscript{243}}难审,

杜琼 (念)只须等待听好音。

(门官上)

门官
回禀二位大人,丞相说:知道了,请二位大人不必进见,有甚军国大事,等待病体稍可,改日自出都堂会议。请回罢。

(门官下)

董允 哎呀,军情至急,哪还等得改日会议。

杜琼 量来难以进见,且回奏圣上再处。

董允、杜琼 请。

董允 (念)召命尚安难通问,

董允 带马。

(二随侍左右上,应)

杜琼 (念)何况区区僚佐臣。

(董允、杜琼急下)

{[}第十场{]}

(二宫女、一大太监引正旦扮吴后上)

(吴后
【二黄慢板】老王爷祖居在大树楼桑,在桃园三结义万世名扬。伐东吴中奸计全军俱丧,梦魂里白帝城痛断肝肠。)

吴后 {[}引子{]}珠帘高卷似蓬莱,追思先帝心痛哀。

吴后 (念)
老王祖居在楼桑,桃园结义万古扬。兵伐东吴全军丧,梦魂白帝断肝肠。

吴后
哀家吴后,先皇昭烈帝与二君侯报仇心切,兵伐东吴,连营失利,败入白帝,恸想二弟,思念桃园,气忿成疾,晏了圣驾,托孤与诸葛丞相。扶保皇儿,登了龙位,内修国政,外治民情,依赖丞相之贤也。正是:(念)谋猷\protect\hyperlink{fn244}{\textsuperscript{244}}人钦敬,调和鼎鼐臣。

刘禅 (内)摆驾。

(四太监一字引刘禅上)

刘禅
【西皮摇板/散板\protect\hyperlink{fn245}{\textsuperscript{245}}】内臣宰无计策孤心烦躁,老丞相不出府所为哪条。

内监 万岁朝罢回宫,与国太请安。

吴后 请。

内监 请驾进宫。

刘禅 参见母后千岁。

吴后
哀家\textbf{平善如常}\protect\hyperlink{fn246}{\textsuperscript{246}},坐了讲话。

刘禅 谢母后。唉!

吴后 王驾叹息为了何事?

刘禅 启母后:大事不好了!

吴后 有何大事如此惊慌?

刘禅 北魏曹丕今发大兵五十万,攻打西川,怎不惊怕?

吴后 众文武岂无退兵之策?

刘禅
文武虽多,惶惶无策\protect\hyperlink{fn247}{\textsuperscript{247}}。

吴后 诸葛丞相必有退兵之策,召来一问。

刘禅
母后有所不知,儿也曾诏他上殿,怎奈\protect\hyperlink{fn248}{\textsuperscript{248}}推病在府,不容使臣入见;又命董允、杜琼同到相府问计,未见回奏如何。

(董允、杜琼上)

董允 【西皮摇板/散板】忙步踉跄走御道,

杜琼 【西皮摇板/散板】气喘嘘嘘滚油浇。

董允 【西皮摇板/散板】丞相忠心改变了,

杜琼 【西皮摇板/散板】他把托孤火化消。

董允 【西皮摇板/散板】你我何言回奏好,

杜琼 【西皮摇板/散板】须将实言奏当朝。

董允、杜琼 原来老承奉在此,烦劳转奏:董允、杜琼回奏交旨。

内监 二位老大人回来了。

董允、杜琼 回来了。

内监 万岁在延寿宫与国太等候回奏,待咱家与你二人请驾。

董允、杜琼 有劳老承奉。

内监 是咧,交给咱家。启奏万岁:董允、杜琼宫外候旨。

刘禅 儿启母后:董允、杜琼回奏,待儿问明,回奏母后。

吴后
董允、杜琼乃旧日老臣,国事紧急,暂止肃避之条,宣进延寿宫,哀家面前回奏。

刘禅 母后之言极是。内侍,宣董允、杜琼进宫,在国太驾前回奏。

内监 董允、杜琼进宫,在国太驾前回奏。

董允、杜琼 领旨。臣董允、杜琼愿国太千岁。

吴后 二卿平身。

董允、杜琼 千千岁。

吴后 二卿同到相府求计,丞相有何良策回奏?

董允、杜琼 臣启国太:丞相推病在府,不容臣等轻入,特来回奏。

刘禅
丞相推病为辞\protect\hyperlink{fn249}{\textsuperscript{249}},并不入朝理事,又无良谋回奏,不如小王趁早死了罢。

吴后
王驾休得如此,我想老王曾将大事托孤与丞相,皇儿拜他以为相父\protect\hyperlink{fn250}{\textsuperscript{250}}。人臣之中位至极矣。今曹丕明知老王殡天,皇儿年幼,乘我国新丧,人心未定,五路进兵夺取西川,社稷危急之际,假以推病为辞,一谋不设。哀家亲到相府求计,看有何说。

刘禅 二卿意下如何?

董允、杜琼
万岁,依臣等愚昧之见,国太不可轻往。料丞相不出府门,必有奇谋妙策。暂请主公御驾亲往求计\protect\hyperlink{fn251}{\textsuperscript{251}},如有怠慢,再请国太召丞相入太庙对老王御影问之可也。

吴后 二卿所奏有理,皇儿速往,哀家立听回奏。

吴后
【西皮摇板】西川地到如今我蜀帝基,恨曹贼兴人马五路告急。却为何老相爷坐视不理,这内中必有那妙算神机。

(吴后\textless{}\textbf{小锣打下}\textgreater{})\protect\hyperlink{fn252}{\textsuperscript{252}}

刘禅 摆驾。

(四大铠两边上,四太监引刘禅上辇\textless{}\textbf{牌子}\textgreater{},当场见门官)

众 圣驾到。

门官 小臣接驾。

(刘禅下辇介,\textless{}\textbf{牌子}\textgreater{}停)

门官 小臣叩见万岁。

刘禅 起来。

门官 万万岁。

刘禅 前去通禀。

门官 丞相令出森严,不准小官通禀。

刘禅 丞相今在何处?

门官 臣亦不知。只有丞相钧谕:挡住百官,勿得擅入。

刘禅 起来。

门官 谢万岁!

刘禅 相父既有此谕,众臣府外等候,毋得喧哗。

众 领旨。

(众分下)

刘禅 门官引起。

门官 领旨。

刘禅
【二黄摇板】龙离潭凤离巢论礼不雅\protect\hyperlink{fn253}{\textsuperscript{253}},为的是平五路君到臣家。

{[}第十一场{]}

(诸葛亮持竹杖上)

诸葛亮
【二黄三眼】报国家报不过黎元为大,扭人心扭不过事理无差。恨曹丕受禅台惨行强霸,叛逆贼终有日报应相加。我本当去问罪发动人马,怎奈我兵新败难以去杀。哭献帝恸先皇淋漓泪洒,好教人肝胆碎心乱如麻。

(诸葛亮大边外场观鱼,门官引刘禅上)

刘禅
【二黄原板】入相府(或:进相府)穿廊厦肃静幽雅,过几层曲湾处倒也可夸。进花园见相父闲坐潇洒,

(刘禅指门官退下,门官作揖退下)

刘禅 【二黄原板】孤这里走近前侧耳听他。

诸葛亮 【二黄原板】汉高皇创基业治平天下,至孝平方五载丧了邦家。

(诸葛亮站)

诸葛亮
【二黄原板】光武兴白水村【转二黄三眼】重整人马,访邓禹、收岑彭到处征伐。剐王莽、诛苏献神惊鬼怕,洛阳城修宫殿一统中华。四百载东西汉六元七甲,传至在献帝朝国乱如麻(或:群寇如麻)。十常侍乱宫闱董卓强霸,许田射猎曹孟德把主欺压。曹丕贼篡汉位万民叫骂(或:万民怒发),吾主爷恨贼子咬碎齿牙。白帝城受血诏遗言留下,承受那托孤重(或:托孤情)怎敢有差。哭一声先帝爷在九泉之下,保佑臣增寿算扶保汉家。

(诸葛亮看鱼指介)

诸葛亮
【二黄摇板】这鱼儿\protect\hyperlink{fn254}{\textsuperscript{254}}比陆逊行兵狡诈,有此计无此人怎能退他。猛回头站身边当今圣驾,

(诸葛亮跪介,刘禅搀介)

诸葛亮 【二黄摇板】老孤臣轻慢君罪当重加。

刘禅 相父。

刘禅
【二黄摇板】相父病孤王我放心不下,因此上孤亲自来看卿家。见相父观鱼跃闲情潇洒,这几天小王我心乱如麻(或:这几天相父病我心乱如麻)。\textless{}\textbf{行弦}\textgreater{}

诸葛亮 陛下何事忧心?

刘禅
【二黄摇板】曹丕无故兴人马,五路大兵来战杀(或:来征杀)。文武百官心惊怕,相父有病又在家。西蜀倾危在眼下,求取良谋\protect\hyperlink{fn255}{\textsuperscript{255}}去退他。

诸葛亮 哦!

诸葛亮 【西皮导板】尊我主请正坐容臣参驾,

(刘禅正坐,诸葛亮拜,刘禅扶)

刘禅 相父免礼,请坐。

诸葛亮 谢座。

诸葛亮
【西皮原板】容忍老臣奏根芽:曹丕国贼多奸诈,贿买羌、蛮辅助他。乱臣贼子人叫骂,谁肯真心死战杀。先皇在日(或:先皇在时)常怒发,本当问罪去征伐。乘人之丧毒手下,就是百万何惧他。臣非妄奏言虚假,望我主稳听捷报奏国家。\textless{}\textbf{小拉子}\textgreater{}

刘禅 听相父之言,曹兵五路,如此容易退却?

诸葛亮 陛下只请放心(或:陛下只管放心,呃),且免忧虑。

刘禅 (呃,)望相父明言与孤,所调都是哪路人马?

诸葛亮 陛下。

诸葛亮
【西皮原板】臣不奏为的是行兵密法,怕的是成都民惊走天涯。非是臣瞒陛下事有虚假,都只为安人心保国保家。马孟起守西平威名颇大,魏文长疑兵计俱按兵法。赵子龙阳平关督理人马,一封书差人去赚走孟达。退吴兵臣已把良谋想下(或:东吴兵臣已罢良谋想下),缺少个能言士每日详查。

刘禅 呀。

刘禅
【西皮原板】孤王亲入相府地,君臣二人论军机。欺君篡位贼曹丕,兵发五路\protect\hyperlink{fn256}{\textsuperscript{256}}取川西。派将三员贼退去,孤王心内自猜疑。彼众我寡非容易,片纸岂退(或:片纸怎退)上庸敌?丞相在府观鱼戏,东吴怎肯卷旌旗。越思越想心忧虑,(收腿)孤必得拔树搜根仔细提。

诸葛亮 (啊,)万岁思索何事?

刘禅
孤王所虑,彼众我寡,孤闻贼兵五十万,五路攻川,相父所派蜀将三员,能否退敌。孟达、孙权何人抵挡?

诸葛亮
陛下!老王将大事托与老臣,臣怎敢不竭力?报答老王知遇之恩。况成都臣宰,不知兵法之妙论。若用成都人马,人民震动\protect\hyperlink{fn257}{\textsuperscript{257}},不能安稳,机关泄漏,大事去矣!臣身居相府之中,心在边关之外\protect\hyperlink{fn258}{\textsuperscript{258}}(呀)。知己知彼,略韬\protect\hyperlink{fn259}{\textsuperscript{259}}因人而使,量才择用。那马超祖居西土,声名远振。羌人称他为``神威将军''。羌人见是马超,必自退去。此西路之兵不必忧矣。

诸葛亮
【西皮二六】那羌王柯比能兴兵犯境,那马超继祖先世居西平。他父祖自从来声名远振,那羌人称马超神威将军。臣命他用奇兵四下伏定,那羌王见而丧胆决不敢前来相争。

刘禅 哦(哦),这一路退得妙!那蛮王孟获,闻他骁勇无比,谁可敌他?

诸葛亮
那南蛮孟获兵犯益州四郡,臣使魏延用疑兵之计。孟获虽勇,多生疑心,必要自退,我兵随后追杀,必然大获全胜。此南路之兵------陛下------心勿忧矣!

诸葛亮
【西皮快板】南蛮贼他夙习智量浅近,哪知晓孙、吴法机谋实深。忽见我左右军无数隐隐,管教他神魂不定不敢交兵。

刘禅
【西皮摇板】我相父素昔来谋猷谨慎,知己知彼着着胜人。还有那第三路汉中要紧,第四路恐难退贼将曹真。

诸葛亮
汉中是李严把守,孟达与李严有生死之交。臣已暗作一书,如严亲笔,着人潜递\protect\hyperlink{fn260}{\textsuperscript{260}}孟达,孟达见之,必然推病不出。况孟达并非李严对手,臣回成都,留严镇守永安宫,正为此也。东路之兵,万岁,不足忧也。

刘禅 哦哦哦\ldots{}\ldots{}

诸葛亮
那曹真领中原大兵十万,攻取阳平。那阳平本非用武之地,山岭险峻\protect\hyperlink{fn261}{\textsuperscript{261}},道路崎岖。行运粮草不便,臣命赵云暗设人马,坚守勿战。待彼粮尽,一战成功。曹真必败于赵云之手,此北路人马不足忧矣。呃,臣尚恐不能全保,又秘调关兴、张苞呵------

诸葛亮
【西皮摇板】恐四路有哪处力不胜任,故命他分要口结寨安营。有不虞急提兵前往救应,\textbf{此密事故未便先使人闻}。

刘禅 如此说来,五路贼兵,(呃,)已退四路了。

诸葛亮 正是。

刘禅 孤恐孙权必怀伐吴之恨,借此而入,当之如何?

诸葛亮 (唉,)陛下呀!

诸葛亮
【西皮摇板】量孙权必观望兵未出境,止需用一能士说彼连横。臣观鱼非潇洒寻思线引,看机变好得个这舌辩的苏秦呐。

刘禅 如相父之言,那五路大兵不日全退了?

诸葛亮 然。

刘禅
呵哈哈\ldots{}\ldots{}哎呀呀相父,你真有移星换斗之智,神鬼不测之机。使寡人万虑皆释矣。

诸葛亮 陛下既释怀疑,可请驾快快回宫,奏知太后要紧呐。

刘禅 既有此万全之策(或:万全之计),何必这等着忙,定要赶奏太后。

诸葛亮
不是呵,陛下若不及早回奏\protect\hyperlink{fn262}{\textsuperscript{262}},诚恐太后已向太庙召臣,那时教老臣何以担当得起?

刘禅 (呃,)此话(\ldots{}\ldots{}呃,)相父何以知之?

诸葛亮 (呃,)不过是推情度理(而已)。

刘禅 (唉,)真正羞煞群僚也(或:真正愧煞群僚也)。

刘禅 【西皮摇板】你谋猷真使人钦敬,不愧调和鼎鼐臣。

诸葛亮
【西皮摇板】鞠躬尽瘁臣之分,敢忘先帝委托恩(或:怎敢忘却先帝恩)。请主回宫心安定,

(\textless{}\textbf{牌子}\textgreater{}刘禅、诸葛亮出门,众上,刘禅上辇介)

诸葛亮 老臣送驾。

刘禅 啊哈哈哈\ldots{}\ldots{}

(刘禅笑介,邓芝点头,诸葛亮望)

诸葛亮 邓大夫暂留一步。

(众、刘禅下,留邓芝,\textless{}\textbf{牌子}\textgreater{}停)

诸葛亮 【西皮摇板】请留大夫说分明。

诸葛亮 请。

邓芝 请。

诸葛亮 大夫请坐。

邓芝 谢座。

(诸葛亮右上,邓芝左偏,坐)

邓芝 不知丞相有何面谕?

诸葛亮 挽留大夫,非为别事,有一桩国事难心领教。

邓芝 何事难心?

诸葛亮 今有魏、蜀、吴,鼎分三国,欲讨二国,一统中兴,请问大夫,先讨哪国?

邓芝
(这\ldots{}\ldots{})若以愚意论之,魏虽汉贼,占据中原,其势甚大,急难摇动(呵),合当徐徐缓图;今当主上新登宝位,民心未安。当与东吴连合\protect\hyperlink{fn263}{\textsuperscript{263}},结为唇齿之邦,永结盟好,一洗先帝之怨,此乃长久之计。未审丞相钧意若何?

诸葛亮 呵哈哈哈哈\ldots{}\ldots{}(笑介)吾亦思之久矣,无奈不得其人。

邓芝 其人何用?

诸葛亮
吾欲使人往结东吴,大夫既明此意,必能不辱君命。这一大任,非大夫不可。

邓芝 邓芝才疏智浅,诚恐有负丞相所托。

诸葛亮 明日我便奏知天子,大夫休要谦让,有负吾意。

(诸葛亮
【西皮快板\protect\hyperlink{fn264}{\textsuperscript{264}}】休谦让莫推辞听我言讲,我和你作臣宰同侍君王。须念在先皇爷恩如海样,谈国政量人才非比寻常。同受过托孤重遗命曾降,也是你明此意劳苦应当。与东吴结唇齿好言讲上,灭汉贼(或:灭国贼)报国仇美名远扬。伯苗你若推辞【转西皮摇板】(你)不肯前往,笑西蜀无能将\textbf{志}成风霜。况先皇待臣宰手足一样,秉赤胆方显你干国忠良。)

(邓芝 丞相。)

邓芝 【西皮摇板】先皇爷托孤情恩深海样,去东吴见孙权自有主张。

邓芝 谨遵台命,邓芝告退。

诸葛亮 (且慢,)书房小酌,聊佐行色。

邓芝 多谢丞相。

诸葛亮 请。

诸葛亮 【西皮摇板】我与你到书房饮酒欢畅,到明天同入朝启奏君王。

诸葛亮 大夫请。

邓芝 丞相请。

诸葛亮 正是:(念)但愿仲谋纳君训,

邓芝 (念)得统中华贺升平。

诸葛亮 啊,

邓芝 啊,

诸葛亮、邓芝 哈哈哈哈\ldots{}\ldots{}(笑介)

诸葛亮 大夫请。

邓芝 丞相请。

诸葛亮、邓芝 呵呵哈哈哈\ldots{}\ldots{}(笑介)

(诸葛亮、邓芝下)

{[}第十二场{]}

(\textless{}\textbf{牌子}\textgreater{}旦扮祝融夫人\textless{}\textbf{点绛唇}\protect\hyperlink{fn265}{\textsuperscript{265}}\textgreater{}上)

祝融夫人
(念)自幼生长在南方,喜读战策演刀枪。上阵能斩千员将,谁人敢犯我边疆。

祝融夫人
咱家乃洞府都蛮王孟获之妻祝融夫人是也。只因中原皇帝曹丕兵督五路,攻取西蜀。遣臣前来聘请咱家大王起蛮兵十万攻打川南四郡,去之日久,不见回来。是咱家放心不下,为此催办粮草,置买水牛、菜蟒,咱家亲身押赴军营。嘟,众蛮兵------

众 啊。(众应)

祝融夫人 咱家命你们所办粮草等物可曾齐备?

众 齐备多时。

祝融夫人 随咱家解送军营。

众 啊。(众应)

祝融夫人 【西皮导板】汉室三分争江山,

祝融夫人
【西皮原板】北魏使臣把兵搬。大王率领兵十万,攻打四郡夺西川。咱家算来日期远,不见大王转回还。解押粮草日夜赶,到军营花沾雨露续团圆。

{[}第十三场{]}

(四文堂,净扮魏延上)

魏延 {[}引子{]}奉命守边关,敌将心胆寒。

魏延
(念)少年英勇走天涯,杀死韩玄献长沙。弃暗保定刘先主,占据西川定邦家。

魏延
某乃西蜀大将魏延,奉了军师将令,命俺挡住蛮王孟获,不要临阵交锋;又道贼心性多疑,必要自退。我不免照书行事。

(报子上)

探子 报,蛮兵退去。

魏延 再探。

报子 得令。

(报子下)

魏延 且住,果然不出军师妙算,趁此追杀前去。众将官,杀!

(魏延下)

{[}第十四场{]}

(\textless{}\textbf{牌子}\textgreater{}众、净扮孟获上)

孟获
孤,都蛮王孟获,今有北魏皇帝聘请孤家帮助,因此领了蛮兵十万攻打川南四郡。孤自安营以来,蜀将并不出马交锋,见他军马每日左出右入,右入左出,不知是何缘故。孤家素闻诸葛亮诡计多端,不要入他圈套,孤不免将人马撤回,暂归蛮洞,再作计较。

(报子上)

报子 报,蜀兵追杀前来。

孟获 再探。

报子 得令。

(报子下)

孟获 嘟,众蛮兵,迎上前去。

(魏延、众人会阵上)

孟获 蜀将通名。

魏延
听者,某乃西蜀大将魏延,尔知道某家厉害\protect\hyperlink{fn266}{\textsuperscript{266}},快些下马受死。

孟获 魏延,孤家开恩,饶尔不死,竟敢大胆追赶孤王,前来送死。

魏延
住了,蛮贼无故兴兵,助逆侵犯边界,占据疆土。要想逃走,留下尔的人头。

孟获 少要多言,看枪(或:看刀)。

(魏延、孟获二人开打,下)

{[}第十五场{]}

(四手下站门上,关兴、张苞上)

张苞 俺,张苞。

关兴 俺,关兴。

张苞 贤弟请了。

关兴 请了。

张苞
你我奉了军师将令,带领人马四路接应。适才探马报道:魏延追赶蛮王孟获,不知胜败如何。

关兴 你我前去接应。前去接应杀退那贼。

张苞 好,就此迎上前去。

张苞、关兴 众将官,杀上前去。

(众全下)

{[}连场{]}

(众围魏延,关兴、张苞上,挑开起打;众围孟获,孟获败下,关兴、张苞、魏延追下)

{[}第十六场{]}

(众、祝融夫人上,报子报上)

报子 报,大王遭了围困。

祝融夫人 再探。

报子 得令。

(报子下)

祝融夫人 蛮兵们。

众 有。

祝融夫人 杀上前去。

(众全下)

{[}第十七场{]}

(众围孟获、孟优,旦上救下;关兴、张苞追下;祝融夫人与魏延架住,磕开)

魏延 杀来杀去,杀出一个蛮婆来了。呔,那蛮婆少要送死,老爷开恩饶你去罢。

祝融夫人
住着。我乃都蛮王孟获之妻祝融夫人是也。你们知道咱家厉害,就在马前磕头饶你们不死。

魏延 休得胡言,放马过来。

(魏延、祝融夫人开打,关兴、张苞上战介;蛮众败下,蜀众追下;祝融夫人上)

祝融夫人 这厮们果然厉害,待咱家飞刀伤他便了。

(张苞等追上)

祝融夫人 看咱家飞刀取你。

(张苞坠马介,众救下;祝融夫人拉孟获,蛮众随下;魏延、众上)

魏延 张小将军怎么样了?

张苞 末将身无伤损,可惜战马被她杀死。

魏延、关兴 此乃万幸。谢天谢地。

张苞 快快换马。待俺追上蛮妇,好报杀马之仇。

魏延 将军不必如此,天色已晚,道路不明,趁此收兵。

关兴 老将军之言甚是。

张苞 便宜了蛮婆。

魏延、关兴、张苞 众将官,收兵。

(\textless{}\textbf{牌子}\textgreater{}众下)

{[}第十八场{]}

(祝融夫人搀孟获上)

孟获
孤自兴兵以来,从无如此大败,似这等狼狈不堪,有何颜面回见各家洞主。我不免碰死了罢。

(祝融夫人拉孟获)

祝融夫人
大王不要如此短见,自古道:军家胜败乃古之常理。依咱主意,暂将人马撤回蛮洞,养足锐气,平整人马,再来报仇不迟。

孟获 \textless{}\textbf{叫头}\textgreater{}诸葛亮,孔明!

孟获 孤家与你誓不两立也。

祝融夫人 走了的好。

孟获 悄悄地收兵。

(孟获、祝融夫人同下)

{[}第十九场{]}

(四太监、一大太监站门,净扮孙权上)

孙权 {[}引子{]}坐镇江东,三分鼎;半壁山河。

孙权
(念)陆逊年幼智超群,蜀兵百万尽皆焚。看来孤王有福分,刘备命丧白帝城。

孙权
孤,孙权,今有曹丕聘孤兵伐西蜀,孤王难作决策,也曾命人探听各路消息,未见回报。

虞翻 (内)走啊。

(虞翻上)

虞翻
【西皮摇板\protect\hyperlink{fn267}{\textsuperscript{267}}】奉使连朝暗徵听,不道诸葛果然能。忙上银安将情禀,见了主公说分明。

虞翻 虞翻参见主公。

孙权 命你徵听曹兵各路进取西蜀消息怎么样了?

虞翻
主公容启:(念)曹兵四路寇蜀,诸葛调军相迎:马超西平退羌兵,疑兵孟获远遁;孟达推病不出,子龙拒走曹真。眼见四路尽解纷,谁与西蜀争胜。

孙权 啊,听你所言,那西蜀四路的曹兵,竟多被孔明暗调兵马,全皆逐退了。

虞翻 便是主公,幸喜听了陆逊之言,未曾动兵,不然今日也要羞归江东矣。

孙权 果然。咳,孔明啊\ldots{}\ldots{}你真有神通也。

孙权 【西皮摇板】似此神通令人敬,堪笑曹丕枉用心。兴衰此际难拟定,

(薛综上)

薛综 【西皮摇板】狂儒胆敢来批鳞。

薛综 臣薛综启事:今有西蜀邓芝特来请见主公。

孙权 他来见孤何意?

薛综 此定是孔明遣他来作说客,退我第五路兵耳。

孙权 他来何以答之?

薛综
依臣鄙意,可于殿前\protect\hyperlink{fn268}{\textsuperscript{268}}设一大大油鼎,贮油数百斤,下用木炭烧得烈烈腾沸;再选身长面大\protect\hyperlink{fn269}{\textsuperscript{269}}武士千人,执利刃从宫门直排至殿角,后唤邓芝入见。他自胆裂魂飞,彼若开言责以郦食其说齐故事,可效那田广旧例而烹之。看他怕也不怕。

孙权 甚好。你就去传旨安排者。

虞翻 领旨。

(虞翻下)

孙权 你去候殿廷排列齐整,然后引那邓芝来见。

薛综 领旨。

(薛综应下)

孙权
【西皮原板】建邺王气承天运,黄武称号顺人心。虽云蜀、魏峙如鼎,谁似孤江东群俊英。武略初仗周公瑾,他壮猷深谋谁比伦。曹兵百万犯吾境,势如压卵好惊人。天意存吴东风趁,火炬烧他无几存、若不是关公释曹命,魏家何能鼎足分。近日里全仗小陆逊,年幼智广独超群。刘备妄想报仇恨,连络七百结下营。猇亭用计火攻盛,蜀兵百万尽皆焚。看来孤王有福分,刘备命丧白帝城。眼前刘禅遣使命,孤岂做齐王烹郦生。銮杖此际排齐整,

(吹打,四值殿、四小太监执銮杖左右上,往内抄;孙权上高台,众在高台后;四校尉抬油鼎上,放前中间,虞翻、薛综上)

虞翻、薛综 万岁。

孙权
【西皮快板】器杖刀剑亮如银。殿角之下设油鼎,汤沸火烈焰腾腾。教那蜀使来认一认,方知东吴不虚名。

孙权 来,

孙权 【西皮快板】众卿替孤传一令,速宣邓芝来觐至尊。

众 宣邓芝上殿。

邓芝 (内)来也。

(邓芝上)

邓芝 啊。

邓芝
【西皮摇板】我自谓钦承皇王命,到此连合去灭魏人。只见他列杖设油鼎,这其间教我解不明。

邓芝
且住,我今奉命而来,原想与他连合伐魏。看他不以礼待,反而列杖设鼎召我。呜哈哈哈\ldots{}\ldots{}(笑介)

邓芝 (念)孤身到虎穴,气节足凌云。从来试威武,岂能屈儒生。

邓芝
【西皮快板】孤身入了虎穴境,志谋气节贯凌云。撩袍端带金殿进,看那吴王怎样行。

邓芝 啊大王,邓芝奉揖了。

孙权
呃嗯------你既奉使来朝,缘何见孤长揖不拜\protect\hyperlink{fn270}{\textsuperscript{270}}?

邓芝 吾乃上国天使,来此小邦,汝不倒履相迎,便为不恭,何必拜汝。

孙权
哦,想尔\protect\hyperlink{fn271}{\textsuperscript{271}}欲掉三寸之舌,意效郦生说齐王么?但孤非田广者比,汝却作了食其之惨。速赴鼎镬,毋得饶舌。

邓芝 哈哈哈\ldots{}\ldots{}(笑介)

邓芝 人人皆言东吴是个名贤之邦,今日一见,竟惧怕一儒生,何其鄙哉。

孙权
寡人富有半壁山河,强有兵将数百余万,何惧一匹夫乎?量汝不过为诸葛亮来做说客,欲孤绝魏向蜀,可是么?

邓芝
嗯。吾实奉大汉丞相诸葛孔明钧旨,特为汝来陈说利害。请问近日还是向蜀乎,还是向魏乎?

孙权 而今曹丕据有中原,磐石之坚,尔那西蜀刘禅焉能与魏抗衡乎?

邓芝
唉,鄙哉斯言也。夫魏虽窃据中原\protect\hyperlink{fn272}{\textsuperscript{272}},实为篡贼苗裔;蜀虽暂处西川,实为大汉帝脉。自上古以来,几见篡贼之后而能长享之理。

邓芝
【西皮原板】何故出言直不慎,全无高下枉批评。试问那王莽移汉鼎,他能可\protect\hyperlink{fn273}{\textsuperscript{273}}遗享与子孙?一朝事败身家尽,惨祸波及家满门。曹丕贼目前似侥幸,我量他不久祸临身。炎汉昭穆承天运,不日旧业重复兴。凡有疑惑不自省,则恐怕前车覆而后车跟。

孙权 啊。

孙权
【西皮摇板】他言出金石令人信,不似张、苏说连横\protect\hyperlink{fn274}{\textsuperscript{274}}。

孙权 邓芝。

孙权 【西皮摇板】兴亡大势孤且不问,如今之势怎样行。

邓芝
【西皮快板】承王问,芝直禀,非敢虚谬论世情:大王诚算命世主,吾丞相诚算佐命臣。一边仗有山川险,一边仗有江河奫\protect\hyperlink{fn275}{\textsuperscript{275}}。若能连合为唇齿,兼併山河可二分。如或弃蜀归魏贼,我川兵不久顺流征。那时节蜀、魏同临境呃,凭便是铁桶的山河也保不成。

孙权 哦\ldots{}\ldots{}罢!

孙权 【西皮摇板】孤便绝魏从伊请,谁为介绍两通情。

邓芝 喏。

邓芝 【西皮摇板】王欲使臣臣从命,若还疑臣便烹臣。

邓芝 大王。

邓芝 【西皮摇板】如不信臣即赴油鼎,

邓芝 罢!

(邓芝扑鼎介)

孙权 快快拉住。

(众应)

孙权 【西皮摇板】先生自是有信人。

孙权 将油鼎抬下,请邓先生上殿相见。

(众应;吹打,孙权下位,吹打住;校尉抬鼎下)

邓芝 大王。

孙权 先生。

孙权 【西皮摇板】你浩然之气令人敬,莫哂小量待高人。

邓芝 哎呀,大王过谦了。

孙权 先生一番金石之言,使孤顿开茅塞,从此吴、蜀连合,协力伐魏。

邓芝 (念)承蒙金诺予,看太平有待。

孙权 哈哈哈\ldots{}\ldots{}(笑介)

孙权 吩咐摆酒,与先生接谈。

(众应)

邓芝 多谢大王。

孙权 先生:(念)从此蜀、吴连合定,孤与先生叙主宾。

(孙权拉邓芝下,众下)

\textbf{天水关 之
诸葛亮}\protect\hyperlink{fn276}{\textsuperscript{276}}

{[}第一场{]}

{[}引子{]}带砺山河\protect\hyperlink{fn277}{\textsuperscript{277}},掌丝纶,运筹帷幄。(或:威权秉正,掌丝纶,运筹帷幄。或:运筹帷幄,掌丝纶,威权秉正。)

(念)先王晏驾白帝城,托孤之事嘱孔明。身受皇恩无可报,忠心耿耿扶后君。(或:汉室纷纷齐弄谋,军中战马不停留。北魏东吴不扫定,黎民涂炭何日休。)

山人复姓诸葛名亮,字孔明,道号卧龙。蒙先帝厚恩,三顾茅庐聘请下山,恢复汉室。老王宴驾,辅保幼主继位,重整汉室基业。今以魏、蜀、吴鼎足三分,昨日修下出师表章。启奏后主,兵出祁山,征服中原。(或:老夫诸葛亮。蒙先帝三顾之恩,托孤之重。一要扫灭孙吴,二要恢复汉室。前者平定孙、曹五路之兵,刻今\protect\hyperlink{fn278}{\textsuperscript{278}}蜀中安稳,为此写就出师表章。呈请幼主,准予兵出祁山。)

童儿,捧了表本、朝服,带路朝房去者。(或:来,带好朝服,金殿见驾去者。)

【二黄摇板】蒙先主三顾请才下山林,白帝城托付事谨记在心。诸葛亮出祁山汉室整顿,怎能够辜负了托孤厚恩。(或:蒙圣恩三顾请才把山下,承受那托孤恩怎敢有差。此一番上金殿参王见驾,但愿得扫北魏重整汉家。)

{[}第二场{]}

嗯哼!

(念)手捧出师表,把本奏当朝。

臣,诸葛亮见驾(或:老臣见驾)。吾主万岁。

万万岁。

谢座。

为臣修下出师表章,启奏万岁,龙目御览。

(刘禅 \ldots{}\ldots{}寡人怎生舍得。)

陛下!(或:万岁呀!)

【二黄慢板】先帝爷白帝城龙归海境,传口诏命老臣常挂(或:时刻)在心。命老臣辅陛下汉室重整,命老臣把孙、曹定要扫平。(或:命老臣辅我主社稷重整,命老臣把孙、曹定要扫平。)出师表并无有别端议论(或:别桩议论),望陛下准臣本臣要发兵。

【二黄原板】食君禄当报效臣把忠尽,臣怎敢劳陛下长亭饯行。

{[}第三场{]}

【二黄原板】接过了吾主爷皇封御饮,便转身祭过了旗纛尊神。

【二黄原板】我朝中有二臣忠心耿耿,蒋公琰、费文伟两大贤臣。主临朝大小事与他们议论,宫中的事阃外事依理\protect\hyperlink{fn279}{\textsuperscript{279}}而行。

【二黄散板】请我主龙驾归臣要发兵(或:臣要启程)。

【二黄散板】传一令众将官齐跨金镫,

【二黄散板】文武官且免送响炮启程(或:响炮起营)。

众将官,

兵出祁山!

{[}第四场{]}

(念)安排诓军计,三军掌握中。(或:眼观旌旗起,耳听好消息。)

老将军回来了。

胜负如何?

可曾问过来将名姓(或:此人名姓)?

姜维\ldots{}\ldots{}

他乃大孝之人,待山人略施小计,收服于他。

老将军后帐歇息。

来,传旗牌。

命你去往冀州,迎接姜维老母。一路之上,小心侍奉,不可怠慢。记下了。

众将进帐。

站立两厢。

众位将军------

今日出马,非比寻常。

听山人令下:

【西皮摇板】一枝将令往下传,

【西皮快板】镇北将军名魏延。假扮姜维关前站(或:假扮姜维去骂关),口口声声出反言。\protect\hyperlink{fn280}{\textsuperscript{280}}

【西皮摇板】再把马岱(或:西凉马岱)一声唤,

【西皮快板】山人言来听根源:若遇姜维莫交战,引他兵至(或:引他兵败)凤凰山。

【西皮摇板】龙虎二将一声唤,

【西皮快板】自古英雄出少年。二马连环来交战,杀得他四路无门跌跪马前。

【西皮摇板】人来看过四轮辇,

【西皮摇板】一战成功定中原。

{[}第五场{]}

【西皮散板】四面设下(或:四下安排)天罗网,姜维小儿无躲藏。四轮车且把(或:下得车来)山岗上,

【西皮散板】准备倭弓(或:弓弩)射虎狼。

(姜维 【西皮散板】\ldots{}\ldots{},手执长枪往上闯。)

大胆!

【西皮散板】姜维小儿莫逞强。要保来保(或:要降来降)圣明主,为何扶保篡位王。

(众 姜维归降!)

【西皮导板】玄天大数如反掌,

(众 姜维归降!)

啊------哈哈哈\ldots{}\ldots{}(笑介)

【西皮原板】事不遂心意彷徨。我不爱将军(你)的韬略广,爱将军是一个贤孝儿郎(或:行孝儿郎)。

(姜维 【西皮原板】冀州还有老萱堂。)

【西皮原板】我早已安排下令堂母,

(姜维 谢丞相。)

【西皮原板】将军不必(或:休得)挂心旁。一出祁山收此将(或:收良将),心中得意喜气洋洋。怕只怕(或:怕的是)五丈原秋风降,【转西皮二六】军国大事(或:军中大事)付与他承当。

【西皮摇板】请将军同把车辇上。

(姜维 【西皮摇板】姜维跌跪跪道旁。)

【西皮散板】将酒宴摆至在中军帐,我与将军论军机叙叙衷肠。

\newpage
\hypertarget{ux7a7aux57ceux8ba1-ux4e4b-ux8bf8ux845bux4eae}{%
\subsection{空城计 之
诸葛亮}\label{ux7a7aux57ceux8ba1-ux4e4b-ux8bf8ux845bux4eae}}

{[}第一场{]}

{[}引子{]}羽扇纶巾,四轮车,快似风云。阴阳反掌定乾坤,保汉家两代贤臣。

列位(或:众位)将军少礼。

(众 啊!)

(念)忆昔当年居卧龙,万里乾坤掌握中。扫尽狼烟扶汉统,人曰男儿大英雄。

老夫,复姓诸葛名亮,字孔明,道号卧龙。先帝爷白帝城托孤遗言,扫荡中原,保留汉室。闻得司马懿兵至岐山,必然夺取街亭。必须派一能将,前去防守。啊,众位将军,

(众 啊!)

哪位将军带领人马,镇守街亭,甘当此任。

(马谡 \ldots{}\ldots{}镇守街亭。)

那司马(懿)虽则年迈,用兵如神,将军不可轻敌。

(马谡 \ldots{}\ldots{}攻无不取\ldots{}\ldots{}何况小小街亭。)

(呃嗯------)

(马谡躬揖)

街亭虽小,干系甚重啊。

军无戏言。

(马谡 愿立军状。)

好,当帐立来。

帐外候令。

(马谡下)

众位将军,

(众 丞相。)

哪位将军愿协同马谡,镇守街亭,当帐请令。

王将军素来谨慎;此番到了(或:去至)街亭,必须靠山近水,安营扎寨。扎寨已毕,画一四至八道地理图,速报我知。

(王平 得令。)

赵老将军听令:带领三千人马,镇守列柳城。

马岱听令:押解粮草,军中(或:军前)需用。

马谡进帐。

(马谡上)

一旁坐下。

(马谡 \ldots{}\ldots{}有何密令?)

今逢大敌,非比寻常。我有一言,将军听了:

【西皮原板】两国交锋龙虎斗,各为其主统貔貅。管带三军要宽厚,赏罚中公平莫要自由。此一番领兵去镇守,靠山近水把营守(或:把营收;把陉\protect\hyperlink{fn281}{\textsuperscript{281}}守)。

(马谡
【西皮摇板】\ldots{}\ldots{}辞别丞相出帐口,\ldots{}\ldots{}顺水去推舟。)

【西皮摇板】先帝爷白帝城叮咛就,汉诸葛扶幼主岂能无忧。但愿得此一去扫平贼寇,免得我亲自去把贼收。

{[}第二场{]}

(念)兵扎祁山地,要擒司马懿。

(旗牌上)

(旗牌 门上哪位在?)

传。

罢了。

奉何人所差?

(旗牌 王平王将军所差。)

手捧何物?

(旗牌 地理图。)

展开。

命你去到列柳城,速速将赵老将军调回营来!快去快去!

(旗牌下)

啊------?!好大胆的马谡哇。临行怎样吩咐(或:嘱咐)与你?靠山近水,安营扎寨。怎么,你偏偏要在山顶扎营?!哎呀,大略街亭难保哇。

(探子上)

(探子 报!)

(探子 \ldots{}\ldots{}失守街亭!)

再探!

(探子下)

如何,果然把街亭失守了。唉------呀!虽然马谡失守街亭,乃诸葛(亮)之罪也。

(探子上)

(探子 报!)

(探子 \ldots{}\ldots{}带兵夺取西城!)

再探!

(探子下)

呜哙呀!司马懿居然带兵夺取西城来了。唉------

(诸葛亮站起)

当初先帝爷白帝城托孤之时言过:马谡言过其实,不可大用。悔不听先帝遗言,今日错差马谡,失守街亭,悔之晚矣呀!

(探子上)

(探子 报!)

再,再探!

(探子下)

啊?!,司马懿的兵,他来得好快呀!嗯------人言司马,用兵如神,今日一见,令人可敬呐,令人可服!

哎呀且住!

(诸葛亮站起)

想这西城的将官,俱被老夫调遣在外,所剩下尽是些个老弱残兵。倘若司马兵到,难道说教我束手被擒,这束手------被擒\ldots{}\ldots{}哎呀!\textless{}\textbf{乱锤}\textgreater{}

老军们进见。

(老军甲 司马兵到,)

(老军乙 心惊肉跳。)

(老军甲 见了丞相,)

(老军乙 急忙跪倒。)

(二老军 有何吩咐?)

命尔等将四门大开,每门上二十名老军,洒扫街道。司马兵到,不可惊慌浮躁,违令者斩。

(老军甲 丞相吩咐我,)

(老军乙 准死不能活。)

天呐,天------

汉室兴败就在这空城一计也!

【西皮摇板】我用兵数十年从来谨慎,错用了小马谡无用之人。无奈何定空城计我的心神不定,望空中求先帝大显威灵。

{[}第三场{]}

【西皮摇板】恨马谡失街亭令人可恨,这时候倒教我难以调停。

呃------

【西皮摇板】老军们因何故纷纷议论,

【西皮摇板】国家事用不着尔等劳心。

【西皮摇板】这西城地原本是咽喉路径,

【西皮摇板】我城内早埋伏有十万神兵。

(老军甲 我再到里头瞧瞧去,)

(老军乙 你瞧见什么没有?)

(老军甲
什么我也没瞧见,瞧见李佩卿在那拉胡琴呢!\protect\hyperlink{fn282}{\textsuperscript{282}})

【西皮摇板】叫老军扫街道把宽心放稳(或:把宽心拿稳),

【西皮摇板】退司马保空城全仗此琴。

(司马懿上)

(司马懿 【西皮原板】大队人马往前进,\ldots{}\ldots{})

【西皮慢板】我本是卧龙岗散淡的人,评阴阳如反掌保定乾坤。先帝爷下南阳御驾三请,算就了汉家的业鼎足三分。官封到武乡侯执掌帅印,东西战南北剿博古通今。周文王访姜尚周室大振,汉诸葛怎比得前辈的先生。闲无事在敌楼我亮一亮琴音,

(诸葛亮抚琴介)

呵呵哈哈哈\ldots{}\ldots{}(笑介)

【西皮原板】我眼前缺少个知音的人。

【西皮二六】我正在城楼观山景,又听得城外乱纷纷。旌旗招展空翻影,原来是司马发来的兵。我也曾差人去打听,打听得司马领兵往西行。一来是马谡无谋少才能,二来是将帅不和(或:二将不和;两将不和)失街亭。连得三城多侥幸,贪而无厌又夺我西城。诸葛亮在敌楼把驾等,等候你到此谈、谈、谈谈心。西城的街道(或:城外的街道)打扫净,准备司马好屯兵。到此并无有别的敬,早备下羊羔美酒犒赏你的三军(临)。既到此就该把城进,为什么你犹豫不定、进退两难为的是何情。我只有琴童人两个,我是又无有埋伏又无有兵(或:我是又没有埋伏又没有兵)。你不要胡思乱想心不定,来来来,请上城来听我抚琴。

(探子 \ldots{}\ldots{}兵退四十里呐!)

(险呐!)

【西皮散板】人言司马善用兵,到此不敢进空城呐。诸葛从来永不弄险,险中又险显才能。

哎呀老将军呐!方才司马懿兵临城下,被我(或:被山人)用空城计将他哄走。必然复返,老将军速速抵挡一阵。

(赵云 得令!)

正是:(念)虎在深山人咸远,蛟龙得水又复还。

险呐!

{[}第四场{]}

【西皮摇板】算就汉家三分鼎,险些一旦化灰尘呐。

(探子 报!)

(探子 马谡、王平回营请罪!)

升帐。

有请。

带王平!

【西皮摇板】怒上心头难消恨,

(王平 丞相。)

【西皮快板】抬头只见小王平。临行再三嘱咐你,靠山近水扎大营。大胆不听我的令,失守街亭你的罪不轻。

(王平 丞相!)

(王平
【西皮快板】丞相不必怒气生,王平言来听分明:马谡不听丞相令,他在山顶扎大营。丞相若是不肯信,现有画图作证凭。)

【西皮快板】若不是画图来得紧,定与马谡同罪名。将王平责打【转西皮摇板】四十棍\protect\hyperlink{fn283}{\textsuperscript{283}},

【西皮摇板】快带马谡这无用的人呐。

(马谡 唉呀!)

【西皮快板】见马谡跪帐下,不由老夫怒气发。大胆不听我的话,失守街亭差不差。

【西皮散板】吩咐两旁刀斧手,快斩马谡正军法。

【西皮摇板】见马谡只哭得珠泪\textless{}\textbf{哭头}\textgreater{}洒,

【西皮摇板】我心中好似乱刀扎。

\textless{}\textbf{三叫头}\textgreater{}马谡!幼常!唉------参军呐!(哭介)

马谡,你临行之时,(当着满营的将官,)先立下军令状啊。如今若不将你正法,何以服众?

\textless{}\textbf{三叫头}\textgreater{}马谡!幼常!唉------参军呐!

(众 哦\ldots{}\ldots{})

来!斩!

招回来!

马谡哇,方才言道(或:方才言过):家有八旬老母,无人侍奉。你死之后,将你兵马钱粮,拨与你老母,以为养老之费。

\textless{}\textbf{三叫头}\textgreater{}马谡!幼常!唉------参军呐!(哭介)

来!斩,斩,斩,斩\ldots{}\ldots{}

【西皮散板】我哭、哭一声马参军,叫、叫、叫一声马幼常啊。未出兵先立下军令状,可叹你为国家刀下身亡。

\textless{}\textbf{哭头}\textgreater{}马谡哇!参谋啊!啊!马幼常啊!

(赵云 \ldots{}\ldots{}为何落泪?)

唉!老将军呐!(我哪里哭的是马谡啊!)当初先帝爷(白帝城)托孤之时言过:马谡言过其实,不可大用。悔不听先帝遗言,至有今日之过。我哪里哭的是马谡哇,乃深恨己之不明,追思先帝遗言呐,呃呃呃\ldots{}\ldots{}(哭介)

也罢,待山人拜本进京,奏明幼主,贬去武乡侯。整顿人马。再与司马决战。

后帐有宴,与老将军贺功。

\newpage
\hypertarget{ux6218ux5317ux539f-ux4e4b-ux8bf8ux845bux4eae}{%
\subsection{战北原 之
诸葛亮}\label{ux6218ux5317ux539f-ux4e4b-ux8bf8ux845bux4eae}}

{[}第一场{]}

{[}引子{]}掌握兵权,运奇谋,拨乱扶贤。六出祁山,取中原,扫狼烟,全图归汉。

列位将军少礼。

(念)汉室纷纷齐弄谋,军中战马不停留。司马纵有拿云手\protect\hyperlink{fn284}{\textsuperscript{284}},老夫先下钓鱼钩。

老夫,复姓诸葛名亮字孔明,道号卧龙,汉室为臣,官拜武乡侯之职。蒙先帝三顾之恩,托孤之重,一要扫灭孙、曹,二要恢复汉室。我想北原乃魏邦咽喉之要地,必需兴兵夺取。

啊,众位将军,人马可齐?

吩咐兵发北原去者。

{[}第二场{]}

【西皮慢板】想当年在隆中何等潇洒,闲无事听鸟音观看山花。先帝爷三顾请才把山下,曾受过托孤的恩怎敢有差。在蜀中奉王命统领人马,兵行在祁山地来把营扎。这几天我未曾去把仗打,司马懿他必然笑我怕他。选一个黄道日发动人马,

【西皮摇板】扫中原灭北魏重整汉家。

再探!

来,

唤众将进帐。

众位将军少礼。

适才探马报道,司马营中有一员大将,名唤郑文,前来投降。尔等可知此人?

原来如此。

来,

吩咐击鼓升帐。

传郑文进帐。

【西皮摇板】见一将跪帐下身躯高大,

【西皮摇板】两眼中含珠泪滚滚似麻。问将军因甚事反背司马,表你的名和姓哪里有家。

【西皮摇板】人道那司马懿识高才大,却为何今日里把事做差。

【西皮摇板】见过了蜀营将请坐叙话,待山人奏幼主定把功加。

再探!

郑将军,

司马营中又来一将,名唤秦朗,他的武艺如何?

哦,原来如此。待山人差一能将前去对敌。

郑将军为何阻令?

哦------好,这一头功,就让与郑将军。

郑文听令!

命你大战秦朗,不得有误。

哈哈哈哈\ldots{}\ldots{}(笑介)

【西皮摇板】看起来吾主爷洪福天大,收此将好一似锦上添花。

【西皮摇板】待山人去观阵众将退下,

【西皮摇板】看郑文与秦朗如何杀法。

【西皮摇板】他二人见了面刀枪并架,未战到三两合就把人杀。莫不是司马懿叫他来行诈,

不错,是的。

回营。

【西皮摇板】等郑文回营来仔细盘查。

{[}第三场{]}

郑将军头功,号令辕门。请坐。

郑将军,

司马营中,有几个秦朗?

哦?并无第二。呵呵呵呵\ldots{}\ldots{}(冷笑介)

郑将军,你这是何苦哇?

【西皮原板】在茅庐曾学过孙吴兵法,仗三韬与六略扶保汉家。适才间斩秦朗多多劳驾,我在那祁山上活活笑煞。

【西皮原板】你道那小秦朗武艺高大,未战到三两合就把他杀。分明是司马懿教你行诈,你可知诸葛亮料事如神半点不差。

郑文!

【西皮摇板】谁不知诸葛亮智谋广大,尔好比螳螂臂井底之蛙。区区的诈降计敢来戏耍,瞒得过诸葛亮?你瞒不了他。

大胆!

呵呵呵呵\ldots{}\ldots{}(冷笑介)

【西皮二六】在帐中说与你一派好话,谁教你自逞那满腹才华。诸葛亮兴人马谁不惧怕,我把那司马懿当作小娃。区区的诈降计敢来行诈,我心内似明镜鉴照无差。我劝你在此间讲了实话,待山人奏幼主定把功加。你若是满口中胡言乱答,顷刻间传将令尔的血染黄沙。谁不知诸葛亮能知兵法,我服尔好大胆,我服尔好大胆敢在那虎口扳牙。

大胆!

真真的大胆!

【西皮摇板】郑将军你既然讲了实话,从今后弃暗投明扶保汉家。

郑将军请坐。

郑将军你与司马懿定下何计?

何谓苦肉之计?

如此待山人与他个计上就计。

就烦郑将军修书一封,下到司马营中,教他三更时分,前来偷营劫寨。我这里埋伏停当,纵然擒不住司马懿,呃,也要杀他个片甲不归。

呃,料无推辞的了哇。

传旗牌。

郑将军有差。

转来。要说郑将军暗差。

记下了。

来,将郑文绑了。

郑将军暂受一时捆绑,待山人擒住司马懿,再来发放于你。

押了下去。

呵哈哈哈\ldots{}\ldots{}(笑介)

【西皮摇板】可笑司马少才学,做事全然不揣摩。诈降之计错又错,些小事怎瞒我南阳诸葛。

{[}第四场{]}

【西皮摇板】老夫背地笑司马,

【西皮快板】郑文诈降智谋差。打虎之计安排下,

【西皮摇板】撒出鹰鹞把兔拿。

回来了,司马懿见了书信,怎样回答?

记功一件。

传令下去,将郑文斩首。

命你将郑文首级用白绫裹好,放在食匣之内,司马兵败之后,送至他营,就说老夫念他连日用兵辛苦,送来些小小的薄礼,望祈笑纳。

来,传众将进见。

命你等整顿军马,埋伏祁山脚下,高挑红灯,上写``郑''字旗号。等候司马前来偷营劫寨,一拥杀出,司马兵败,不可追赶,老夫在祁山口等候。

司马懿呀司马懿!这是你自讨其祸,两军阵前,看你有何面目前来见我。

【西皮快板】自从卧龙把山下,排兵布阵从无差。四轮辇且往祁山脚下,

【西皮摇板】两军阵前取笑他。

{[}第五场{]}

【西皮摇板】四下安排天罗网,准备弩弓射虎狼。耳旁又听鸾铃响,

【西皮摇板】功劳簿上写几行。

【西皮摇板】旌旗招展龙蛇样,

司马!

呵呵哈哈哈\ldots{}\ldots{}(笑介)

【西皮摇板】原来司马到战场。自己用兵不思量,诈降之计太荒唐。今日损兵又折将,我看你何颜回转营房。

住口!

【西皮摇板】你人困马乏打什么仗,败阵之将敢逞强。

【西皮摇板】回头派出一员将,

马岱,杀!

且慢。

呵呵哈哈哈\ldots{}\ldots{}(笑介)

【西皮摇板】我这一只虎能挡你一群羊。

【西皮摇板】坐在车辇把令降,大小三军听端详:人马扎在祁山上,选一个黄道吉日动刀枪。

司马!

【西皮摇板】我少陪------少陪,我要收兵将,

收兵。

【西皮摇板】你父子回营去养养伤再摆战场。

请。

\textbf{七星灯}\protect\hyperlink{fn285}{\textsuperscript{285}}

\textbf{{[}第一场{]}}

\textbf{诸葛亮 (念)万事不由人作主,一心难与命争衡。}

\textbf{旗牌 参见丞相。}

\textbf{诸葛亮 你回来了?}

\textbf{旗牌 回来了。}

\textbf{诸葛亮 司马看了衣服、书信有何举动?}

\textbf{旗牌
司马受了巾帼女衣、看了书信,并不嗔怒,呃,反请小人饮宴,席中只问丞相寝食及事之烦简如何,饭间未提军旅之事。}

\textbf{诸葛亮 尔如何抗对?}

\textbf{旗牌
呃,小人言道:``丞相夙兴夜寐,罚二十以上皆亲览焉。所啖之食,不过数升。''}

\textbf{诸葛亮 司马何言?}

\textbf{旗牌 呃,司马并无言语。}

\textbf{诸葛亮 呃------下去。}

\textbf{诸葛亮 唉,司马深知吾也!}

\textbf{诸葛亮
【二黄原板】仰面朝天【转二黄慢板】自己嗟叹,司马懿可算得将中魁元。送脂粉和钗裙不恼不怨,反与那旗牌官酒食来餐。有刚有柔是好汉,我诸葛比司马难上加难。先帝爷下南阳君臣相见,感恩}深重要扭转汉室河山。博望坡烧夏侯(或:博望坡烧曹兵;博望坡烧贼兵)初次交战,借东风助周郎火烧战船。烧藤甲用火攻孟获丧胆(或:蛮夷丧胆),谁不知诸葛亮智能扭天。到如今与司马两下会战,葫芦峪设地雷安排机关。我料他父子们必遭此难,又谁知天不遂也是枉然。一时里心血涌浑身是汗\protect\hyperlink{fn286}{\textsuperscript{286}},

\textbf{诸葛亮 【二黄摇板}】传姜维和魏延速到帐前。

\textbf{旗牌 姜维、魏延进帐。}

\textbf{姜维、魏延 来也!}

\textbf{魏延 【二黄摇板】帐上一声唤,}

\textbf{姜维 【二黄摇板】上前问根源。}

\textbf{姜维、魏延 参见丞相,有何将令?}

\textbf{诸葛亮 魏延。}

\textbf{魏延 在!}

\textbf{诸葛亮 命你巡营瞭哨,司马叫阵,不可出兵,紧守大营,不得有误!}

\textbf{魏延 得令。}

\textbf{诸葛亮 姜维。}

\textbf{姜维 在。}

\textbf{诸葛亮 命你随我后帐安排七星祭坛,一干人等,勿得入内。}

\textbf{姜维 遵令}(或:\textbf{遵命)。}

\textbf{诸葛亮 唉!}

\textbf{诸葛亮
【二黄摇板】设坛拜星求北斗,但愿天意早回头。三寸气在千般用,一旦无常万事休。扫荡中原难回首,}

\textbf{诸葛亮 【二黄摇板】怕的是天意不遂不自由。}

\textbf{{[}第二场{]}}

\textbf{司马懿 【二黄导板】谯楼鼓打罢了初更时分,}

\textbf{司马懿 【回龙】静悄悄出魏营观看天星。}

\textbf{司马懿
【二黄原板】叫人来前引路高岗来进,司马懿观天象细算详情:观东方甲乙木木能生火,观南方丙丁火火能克金。正西方庚辛金金能生水,观北方壬癸水水遇土屯。佔中央戊己土仔细看定,}

\textbf{司马懿 【二黄摇板】五行生克观不清。}

\textbf{司马懿
【二黄摇板】北斗星垣来观看,主星暗淡光不明。看罢天象心拿稳,}

\textbf{司马懿 回营!}

\textbf{司马懿 【二黄摇板】安排巧计擒孔明。}

\textbf{{[}第三场}\protect\hyperlink{fn287}{\textsuperscript{287}}\textbf{{]}}

\textbf{姜维 有请丞相!}

\textbf{诸葛亮 (内)先王呀!}

\textbf{(}正场小座,``七星灯''不摆在正场桌,而是摆在下场门斜场,诸葛亮拄宝剑上\textbf{)}

\textbf{诸葛亮
【二黄慢板】为汉家把我的心血用尽,都只为先帝爷托孤之恩。执法剑进祭坛(或:执宝剑上坛台)实难扎挣,}

\textbf{诸葛亮 【二黄原板】险些儿把老夫跌倒埃尘。}

\textbf{诸葛亮
(念)亮,谨书尺素,上告穹苍:伏望天慈,俯垂鉴听:亮生于乱世,甘老林泉;承昭烈皇帝三顾之恩,托孤之重,誓讨国贼,永延汉祚}\protect\hyperlink{fn288}{\textsuperscript{288}}\textbf{。上求北斗,曲延臣算,非敢妄祈,实由------唉------情切。诶,呃\ldots{}\ldots{}(哭介)}

\textbf{(\textless{}小开门\textgreater{}烧符箓)}

\textbf{诸葛亮 上苍呐!}

\textbf{诸葛亮
【二黄原板】诸葛亮不敢扭天行,为的是我主锦乾坤。拜南斗和北斗}\protect\hyperlink{fn289}{\textsuperscript{289}}\textbf{赐我阳寿,掌簿官执笔吏留下人情。佔中央戊己土深深拜定,}

\textbf{(诸葛亮叩头,拜后下坛台,踱步至上场门,回身看星灯)}

\textbf{诸葛亮 【二黄摇板】见将星比往常显见光明。}

\textbf{诸葛亮 【二黄摇板】虽然是星明亮吉凶未定,}

\textbf{(诸葛亮归小座)}

\textbf{诸葛亮 【二黄散板】怕的是(或:怕只怕)天意难违大事难成。}

\textbf{魏延 【二黄摇板】司马来踏营,近前说分明。}

\textbf{诸葛亮
【二黄摇板】这是我大限有一定,魏延扑熄我的本命灯。将本命灯撇在尘埃地,}

\textbf{姜维 【二黄摇板】丞相发怒为何情。}

\textbf{诸葛亮
【二黄摇板】我拜斗今日六天整,堪堪}\protect\hyperlink{fn290}{\textsuperscript{290}}\textbf{七天大功成。恨魏延他把我本命灯扑熄,我性命就要哇一旦倾。}

\textbf{姜维 啊?!}

\textbf{姜维
【二黄摇板】听一言来怒气生,魏延贼子起反心。手执宝剑将尔斩,}

\textbf{魏延 你要斩哪个?}

\textbf{姜维 要杀你。}

\textbf{魏延 你杀不得。}

\textbf{诸葛亮 将军!}

\textbf{诸葛亮 【二黄摇板】将军息怒且消停。}

\textbf{诸葛亮 魏延。}

\textbf{魏延 在。}

\textbf{诸葛亮 莫非司马前来踏营?}

\textbf{魏延 正是。}

\textbf{诸葛亮 前去抵挡,出帐去罢!}

魏延 遵命!

\textbf{魏延 (念)堪堪孔明不长久,管教蜀营众将休!}

\textbf{魏延 哼!}

\textbf{诸葛亮 姜维搀我出坛!}

\textbf{诸葛亮 【二黄摇板】姜维后营一声请,快快请出李大人。}

\textbf{姜维 有请李大人!}

\textbf{李福 (内)来也!}

\textbf{李福 【二黄摇板】忽听帐上一声请,急忙进帐看分明。}

\textbf{李福 参见丞相!}

\textbf{诸葛亮 李大人。}

\textbf{李福 在。}

\textbf{诸葛亮 这有表章一轴,连夜送往成都,替吾转奏,请吾主龙目御览。}

\textbf{李福 遵命。}

\textbf{诸葛亮 搀扶!}

\textbf{诸葛亮
【二黄摇板】远望成都忙跪定,拜谢我主爵禄恩。羞愧难见刘先主,李大人速速转奏快快登程。}

\textbf{李福 【二黄摇板】辞别丞相忙登程,不分昼夜奔都城。}

\textbf{诸葛亮 姜维!}

\textbf{姜维 在!}

\textbf{诸葛亮 听我吩咐!}

\textbf{姜维 啊!}

\textbf{诸葛亮 【二黄碰板三眼】我和你虽为将帅倒有那师徒之义,}

\textbf{诸葛亮
【二黄原板】必须要秉忠心扶保华夷。一封锦囊交与你,内藏着妙算与神机。我死后三件大事托与你,一桩桩一件件莫要泄机:第一件我死后休得挂孝,第二件必须要缓缓移营。第三件我死后那魏延必反,}

\textbf{姜维 啊?!}

\textbf{诸葛亮
【二黄散板】我自有妙计除此人。我将这奇门遁甲传授你,阵阵不离此图形。这一弩能发十条箭,九伐中原你担承。将军与我传将令,快传那杨仪、马岱与王平。}

\textbf{姜维 杨仪、马岱、王平速速进帐!}

\textbf{杨仪、马岱、王平 【二黄摇板】丞相帐中传将令,一同上前看分明。}

\textbf{杨仪、马岱、王平 丞相醒来!}

\textbf{诸葛亮
【二黄摇板】指望}\protect\hyperlink{fn291}{\textsuperscript{291}}\textbf{霸业兴炎汉,谁知半途不周全。猛然睁开昏花眼,又只见众将官站立面前。}

\textbf{诸葛亮 杨仪!}

\textbf{杨仪 在。(\textless{}小拉子\textgreater{})}

\textbf{诸葛亮
【二黄摇板】我死后军师大印你掌管,事事谨慎要周全。我今与你这小柬,我死之后再来观。}

\textbf{杨仪 遵命。(\textless{}住头\textgreater{})}

\textbf{诸葛亮 子均!}

\textbf{王平 在。(\textless{}小拉子\textgreater{})}

\textbf{诸葛亮
【二黄摇板】王子均近前听召唤,一封小柬带身边。事到头来}\protect\hyperlink{fn292}{\textsuperscript{292}}\textbf{再观看,内有如此与这般。}

\textbf{王平 遵命。}

\textbf{诸葛亮 马岱!}

\textbf{马岱 在。(\textless{}小拉子\textgreater{})}

\textbf{诸葛亮
【二黄摇板】西凉马岱听我言,我有言来记心间。倘若是魏延来造反,这封小柬临阵观。}

\textbf{马岱 遵命。}

\textbf{诸葛亮
【二黄摇板】众将官搀扶我吾主叩见,诸葛亮在营中拜别龙颜。叩罢头抽身起心血上泛,}

\textbf{诸葛亮 呜\ldots{}\ldots{}(吐血介)}

\textbf{诸葛亮 【二黄摇板】我面前站定了庞统士元。}

\textbf{诸葛亮
【二黄摇板】在荆州对把八字算,我二人各有不周全。我算他落凤坡前身带箭,他算我难逃五丈原。霎时间胸内痛(或:霎时间心内痛)心血上泛,}

\textbf{诸葛亮 呜\ldots{}\ldots{}(吐血介)}

\textbf{诸葛亮 【二黄摇板】昏沉沉一旦间命归九泉。}

\textbf{众 丞相啊!}

\textbf{李福 丞相钧体如何?}

\textbf{众 已归仙境。}

\textbf{李福 哎呀,误了吾主大事了!}

\textbf{诸葛亮 嗯哼\ldots{}\ldots{}}

\textbf{姜维 哦,丞相醒转!大人有何圣谕,快快禀来!}

\textbf{李福 启禀丞相:万岁问道:丞相之后,何人接替。}

\textbf{诸葛亮 蒋公琰。}

\textbf{李福 公琰之后?}

\textbf{诸葛亮 费文伟。}

\textbf{李福 文伟之后?}

\textbf{诸葛亮 三国归于\ldots{}\ldots{}}

\textbf{众 丞相啊\ldots{}\ldots{}}

\textbf{姜维 列公且免悲泪,待我打开丞相钧谕观看。}

\textbf{姜维 原来如此。}

\textbf{姜维
(丞相命我等)用沉香木塑成钧体,安放四轮车上。倘若司马踏营,将车推至阵前,司马必然不战自退。}

\textbf{众 原来如此。}

\textbf{姜维 你我后营安排,准备一切便了!}

\textbf{众 请呐!}

\newpage
\hypertarget{ux9664ux4e09ux5bb3-ux4e4b-ux738bux6d5aux5468ux5904}{%
\subsection{除三害 之
王浚、周处}\label{ux9664ux4e09ux5bb3-ux4e4b-ux738bux6d5aux5468ux5904}}

\textbf{{[}第一场{]}}

\textbf{(王浚
【西皮快二六】趁青年你莫当朝嬉夕宴}\protect\hyperlink{fn293}{\textsuperscript{293}}\textbf{,董仲舒他三载未曾窥园。幼而学壮而行经纶开展,那时节报皇家荣耀门田。若得儿洗旧污重新向善,不唯对儿那高、曾、祖,我二老亦可对湛湛青天。)}\protect\hyperlink{fn294}{\textsuperscript{294}}

\textbf{{[}第二场{]}}

\textbf{(王浚 (内)走哇。)}

\textbf{王浚} 【二黄摇板】摘去乌纱换儒巾,谁人识我大元勋。

\textbf{王浚}
老夫(或:下官)王浚,散操回衙,黎民百姓状告恶霸周处。我想周处乃周舫之子,我若将他(当真)查办,教我怎能(或:教我怎样)对得过他那去世先人?为此乔装改扮,出衙私访于他。用言语打动,若(或:倘)能改邪归正亦未可知。只是教我哪里去寻,哪里去找?

\textbf{王浚}
看那旁来一红脸大汉,想是周处。我不免在此等候。等他到来,他有来言,我有去语。

\textbf{王浚} 【二黄摇板】浑玉不琢(或:璞玉不琢)多壑陵,当头棒喝返本真。

周处 【二黄摇板】终日饮酒消愁闷,半世悠悠困风云。

\textbf{王浚} 唉!

周处 【二黄摇板】老丈缘何冲天恨,

\textbf{王浚} 唉,不成世界了!

周处 啊?!

周处 【二黄摇板】叫人心中解不明。

周处 啊,老丈------请了。

\textbf{王浚} 哦,原来是一位壮士。(这厢有礼。)

\textbf{周处 啊,老丈,为何一人在此长叹?莫非有人欺压于你?}

\textbf{王浚}
想老汉(或:老朽)乃(是)唾面自干之人,纵有人欺压于我,亦何敢较量。

\textbf{周处 既然如此,为何在此长叹?}

\textbf{王浚 唉,可叹这宜兴的百姓好不苦也。}

\textbf{周处 却是为何?}

\textbf{王浚} 皆因此地出了三害。

\textbf{周处 哦,出了三害?但不知是哪三害?}

\textbf{王浚} 壮士愿听(或:壮士愿闻)?

\textbf{周处 愿}闻\textbf{。}

\textbf{王浚} 愿闻(或:愿听)?

\textbf{周处 你且讲来。}

\textbf{王浚} 听了------

\textbf{王浚} 【二黄三眼】若提起这三害令人可恨,

\textbf{周处 你慢慢讲来。}

\textbf{王浚} 【二黄三眼】讲出来连壮士(或:连壮士闻此言)也要心惊:

\textbf{周处 第一害------}

\textbf{王浚} 【二黄三眼】第一害那南山出了猛虎,

\textbf{周处 哦,出了猛虎便怎样?}

\textbf{王浚} 【二黄三眼】它遇着(或:倘遇着)行路人骨肉全吞。

\textbf{周处 嗯,这第二害------}

\textbf{王浚} 【二黄三眼】第二害它比那猛虎还狠,

\textbf{周处 哦,那又是什么妖魔鬼怪?}

\textbf{王浚 【}二黄三眼\textbf{】长桥下又出了恶魔蛟精。}

\textbf{周处 哦,出了蛟精,它是怎样的厉害?}

\textbf{王浚}
【二黄三眼】在水中兴波浪吞舟荡\textbf{艇(或:}荡\textbf{坉}\protect\hyperlink{fn295}{\textsuperscript{295}}\textbf{)},到旱道作毒雾苦害行人。

\textbf{王浚}
【二黄三眼】第三害讲出口令人可恨,他比那南山猛虎、长桥孽蛟还狠十分。

\textbf{周处 哦,它是什么妖魔鬼怪?,又是怎样的厉害?}

\textbf{王浚}
【二黄快三眼】若论他是英雄亦非是禽兽之类,他本是有须眉、有志气、雄赳赳、气昂昂是一个有志的能人。

\textbf{周处 哦,既然并非禽兽之类,为何被列为``三害''之内?}

\textbf{王浚}
【二黄快三眼】都只为他父丧早无人教训,因此上习下流做了歹人。

\textbf{周处 怎样为害?}

\textbf{王浚
【}二黄快三眼\textbf{】仗势力在宜兴习为光棍,欺贫贱、诈富贵苦害良民。}

\textbf{周处 哦,怎样地不法?}

\textbf{王浚
【}二黄快三眼\textbf{】有钱的还则可苦苦地曲奉,只可怜(或:实可怜)那无钱的人儿典了庄田、鬻了妻儿、也难少他的半分。}

\textbf{周处 哦,何不去县衙状告于他?}
\protect\hyperlink{fn296}{\textsuperscript{296}}

\textbf{王浚
【}二黄快三眼\textbf{】也有那被害的家与他来理论}\protect\hyperlink{fn297}{\textsuperscript{297}}\textbf{(或:议论),怎奈他膂力过人、力能扛鼎、有势有财,大小的衙门谁敢哼声。}

\textbf{周处 哇呀呀\ldots{}\ldots{}(周扔扇子)}

\textbf{王浚 【二黄摇板】都只为宜兴城出了恶棍,害得那众黎民难度光阴。}

\textbf{周处 老丈!}

\textbf{周处
【二黄摇板】听一言来怒气生,不由豪杰动无名。快快说出他的名和姓。}

\textbf{周处 我要剥了他的皮。【接二黄摇板】抽了他的筋。}

\textbf{王浚 【二黄摇板】我若是讲出了他人名姓,怕的是我老命要活不成。}

\textbf{周处 老丈!}

\textbf{周处 【二黄摇板】有俺在此何足论。}

\textbf{周处 任凭他铜金刚、铁罗汉,【接二黄摇板】难近某的身。}

\textbf{王浚} 壮士愿听?

\textbf{周处 愿听。}

\textbf{王浚} 愿闻?

\textbf{周处 愿闻。}

\textbf{王浚} 两厢看来。

\textbf{周处 讲来。}

\textbf{王浚 听了------}

\textbf{王浚} 【二黄摇板】他姓周名处

\textbf{周处 啊。}

\textbf{王浚} 【接二黄摇板】字子隐,

\textbf{王浚} 嘿嘿!

\textbf{王浚} 【二黄摇板】壮士闻言你惊不惊。

\textbf{周处 哎呀。}

\textbf{周处 【}二黄摇板\textbf{】好似霹雷当头震,周处做了不义人。}

\textbf{王浚} 【二黄摇板】问声壮士名和姓,

\textbf{周处 【}二黄摇板\textbf{】周处就是我的名。}

\textbf{王浚} 哎呀,饶命呐!

\textbf{周处
【}二黄摇板\textbf{】老丈在此等一等。\textless{}扫头\textgreater{}}

\textbf{(周处下)}

\textbf{王浚} 哈哈哈\ldots{}\ldots{}(笑介)

\textbf{王浚}
【二黄摇板】他好似酒醉方才醒,一言惊起懵懂人。但愿三害俱除尽(或:早除尽),

(王浚捡扇子)

\textbf{王浚} 【二黄摇板】黎民百姓享太平。

\textbf{审刺客 之 闵觉}\protect\hyperlink{fn298}{\textsuperscript{298}}

{[}第一场{]}

搀扶!

【西皮原板】自那日朝驾归精神不爽(或:身体不爽),因此上染重病倒卧在床。这几天我未曾(或:并未曾)朝见皇上,宫闱中出刺客搅乱朝纲。

老夫闵觉。晋王驾前为臣,官居刑部尚书之职。只因我主在粉宫楼前,拿住刺王杀驾之徒。万岁命六部审问。想老夫身染重病,不知哪部大臣代审。今日身体稍愈,不免去至朝房观看。

家院,

带定老爷朝服、官诰,朝房去者(或:带路朝房)。

【西皮原板】晋王爷坐山河人称有道,普天下众黎民快乐逍遥。到如今吾主爷国运衰了(或:看起来吾主爷国运不好),宫闱中出刺客搅乱当朝(或:搅乱九朝)。叫家院忙带路向前引道(或:叫家院你与爷向前引道;或:叫家院你与爷忙登御道),

【西皮摇板】看一看审问官是怎样开销。

{[}第二场{]}

(贺道庵\protect\hyperlink{fn299}{\textsuperscript{299}}、贾昱上\textless{}\textbf{小锣打上}\textgreater{},靠后站;四朝官搭轿上,靠台前站)

(朝官甲 (念)世事不由人机变,)

(朝官乙 (念)宰相专权贺道庵。)

(朝官丙 (念)列位不信抬头看,)

(朝官丁 (念)谁是忠来谁是奸。)

(四朝官 啊,大丞相。

(贺道庵 审得的?)

(贺道庵 问得的?)

(贺道庵 有僭了。)

(谢四\protect\hyperlink{fn300}{\textsuperscript{300}}上,闵觉暗上)

(皂隶 刺客当堂有刑。)

(贺道庵 松刑。)

(贺道庵 刺客。)

(谢四 有。)

(贺道庵 那日\ldots{}\ldots{}一一招来。)

(谢四
丞相容禀:小人名叫史龙。宫中史娘娘乃是小人的姑母。\ldots{}\ldots{}这就是我亲口所招。)

(贺道庵 那日\ldots{}\ldots{}一一招来。)

(贺道庵 来,叫他画押。)

(贺道庵 众位大人,请来画押。)

(四朝臣 闵大人未到,我等不敢画押。)

(贾昱 你们不肯画押,待咱家替你都画上罢!)

(且慢呐!)

怪道啊,怪道------

(贾昱 哎呦,我也先别画了。)

【西皮原板】站立在朝房下用目观看,

【西皮原板】看一看晋朝中文武两班。

【西皮原板】大丞相他那里颜色改变,审刺客这内中有他牵连。我岂肯把先人门墙辱玷,既到此我就该舍命向前。

罢!

【西皮摇板】怒冲冲将大丞相抓下公案,

(闵拉贺出桌,到台中间,贾站起)

【西皮摇板】我问你审刺客哪部官员。(或:审刺客何劳你宰相专权。)

你待怎讲?

哈哈。

呵呵!

(贺道庵 \ldots{}\ldots{}轻慢\ldots{}\ldots{}当朝宰相。)

啊------呵呵哈哈哈哈!(笑介)

(大丞相,)你道我轻慢你当朝首相(或:当朝宰相),想(或:有道是)这宰相之家,表率天下文武百官,为内外大小群臣之领袖,燮理阴阳,调和鼎鼐。兼放天下主考,开科取士,考取文章;翰林之家,选拔天下人材。晋朝之中,圣上设立六部乃是:吏、户、礼,兵、工、刑。这六部大堂,各有专司,各有专责。这吏部:掌管天下百官职爵,升迁调补,提选咨留,参革简放,考功稽勋,验封文选,征辟选举,论秀书升,才能参见当今万岁;户部:掌管内务府库金银,天下各省,丁漕赋税,各项钱粮;礼部:掌礼仪祭祀,祭坛祭庙,庵观寺院,陵寝庙宇。天地三界,十方万灵,春夏秋冬,祀祭典礼;兵部:掌管副参游都守,千把外委,五营四哨,兵丁将士,兵马钱粮;那工部:掌管三宫六院,五府六部,九卿科道,城池营垒,河工粮道,道路桥梁,天下大小工程。惟有我这(小小的)刑部:督理刑名,执掌生杀之大权,凡有人命攸关,私杀、擅杀、逼杀、格杀、谋杀、斗杀、故杀、误杀,各种案件,俱是我刑部所管。想这刺客,乃弑君之徒,朝廷要犯,理当我刑部所审,理当我刑部所问,何劳大丞相你来审问?

大丞相你要审问,却也不难,你我手挽手同上金殿。到晋王驾前,奏上一本:这晋朝之中,有几部大臣。圣上言道:六部。你就奏道:晋朝之中,不要六部,只要五部。圣上必然问将下来:哪部可裁,哪部可减?那时大丞相你须奏道:刑部可裁,刑部可减。圣上准了你的本章,裁了我这刑部,你方可审得,方能问得;圣上准不了你的本章,裁不了我这刑部,你便审不得,你也问不得。此乃刑部法堂之地,非是你大丞相议事之所,你要端端正正坐定了!(或:此乃王法所在,你与我站下些,你与我退后些,你与我坐下了!)

(闵推贺到边上,贺坐,贾上前到台口)

(贾昱 难道你还不念这同朝的情面吗?)

呀呸!

【西皮摇板】非是我不念在同朝情面(或:同朝脸面),审刺客何劳你宰相专权(或:理当我刑部掌权)。

【西皮摇板】对列公施一礼忙上公案(或:忙登公案),

(闵入大座。贾到闵身边)

(贾昱 你要还审不了?)

【西皮摇板】审不清问不明你启奏龙颜(或:审不公问不明面奏龙颜)。

(来,列位大人再次审问。)

(四朝官 恐刺客受刑不过,牵连不便。)

(请列位大人将台座升上一步。)

(四朝官 是。)

(四朝官站,再坐下)

带刺客!

(皂隶带谢四上)

(皂隶 当面上刑。)

(松刑。)

(贺道庵 史龙。)

(啊?大丞相,你怎么知道他叫史龙?)

(贺道庵 呃,这这这\ldots{}\ldots{}适才供状他叫史龙,故而叫他史龙。)

(哼,他乃弑君之徒,天子重犯,叫不得史龙。)

(贺道庵 要叫什么?)

(叫------要叫刺客。要叫刺客!)

(贾昱
咳,要叫刺客就叫刺客。刺客,忙将言语\ldots{}\ldots{}还要开脱于你。)

刺客,

(谢四 有。)

刺客!

刺客,你是何人将你带进宫去,藏在粉宫楼前,刺王杀驾,被御前侍卫拿住?你要从实招来,免受\textbf{五刑}之苦!

你待怎讲?

(谢四 \ldots{}\ldots{}亲口所招。(或:句句实言))

尔就该掌嘴!

【西皮原板】朝房中比不得荒野小县(或:晋朝中比不得旷野小县),本部堂岂容你信口胡言(或:信口诬陷)。

(刺客!)

(谢四 有。)

【西皮原板】你不过受他人些许情面,为什么把性命付与九泉。

教他招来。

唗!

【西皮散板】好一个小刺客真个大胆,四十板管叫你吐露实言。

有招无招?(或:来,问他有招无招。)

(皂隶 有招无招。)

(谢四 无有什么招的。)

(皂隶 无招。)

唗!。

(【西皮散板】骂一声小刺客真个大胆,不招供管教你鲜血不干(或:四十板管教你口吐真言)。)

(来,打!)

(皂隶把谢按倒在台中间)

(朝官甲 【西皮原板】好一个小刺客真个大胆,)

(皂隶 一十。)

(朝官乙 【西皮原板】责打他四十板鲜血不干。)

(皂隶 二十。)

(朝官丙 【西皮原板】贺道庵在一旁颜色改变,)

(皂隶 三十。)

(朝官丁 【西皮原板】倒教我审问官一体为难。)

(皂隶 四十打完。)

(谢四 哎呀。)

(谢四
【西皮散板】上堂来责打我四十大板,只打得小豪杰鲜血不干。自幼儿出娘胎未遭此难,你不敢把某家送上刀山。)

(住口!)

【西皮散板】任凭你就是那铜打铁炼,铜夹棒管教你尸不周全(或:尸骨不全)。(或:任尔是铜打铁炼,少时节(或:少时间)管教你尸骨不全。)

夹起来!(或:来,夹了起来)

(朝官甲 【西皮原板】铜夹板夹得他一声呐喊,)

(朝官乙 【西皮原板】吓得我战兢兢不敢胡言。)

(朝官丙 【西皮原板】他若是招实了你我不便,)

(朝官丁 【西皮原板】咱三人这性命全凭老天。)

有招无招? (或:来,问他有招无招。)

(皂隶 有招无招。)

(收。)

(皂隶 \ldots{}\ldots{}晕刑。)

(松刑。)

(谢滚堂,摸腿,爬起)

(谢四 【西皮导板】铜夹棒夹得我皮开肉烂,\ldots{}\ldots{})

(谢四 嘿!好一个刑部大人\ldots{}\ldots{}我愿招。)

(皂隶 他有招。)

教他画供。

(皂隶 招上来画供。)

(谢看贺,贺示意谢不招)

(谢四 哎呀,招不得呀招不得。闵觉,方才俺招的俱是实言,还教我招得什么?)

哎呀!

【西皮散板】小刺客不招承浑身是汗(或:小刺客不招供难以判断),

(闵觉吐血介)

【西皮散板】不由人一阵阵心血上泛(或:不由我一阵阵血往上泛)。

【西皮散板】到如今好教我难以审断,用尽了百般刑也是枉然。

【西皮散板】无奈何下位去将他来劝(或:我只得下位去将他来赚),

(闵觉出座,坐台口椅)

【西皮散板】慢慢地相劝他好吐实言(或:好言诓哄他好漏实言)。

刺客,

(谢四 有哇!诶呀\ldots{}\ldots{})

刺客!我看你乃是一条英雄好汉,并非真心刺王杀驾,不过是受了哪部大人之托,不想被御前侍卫拿住,圣上命六部审问。方才你胡说史娘娘是你的姑母,你是史娘娘的内侄。想那史娘娘,自从进宫以来,并无三亲六眷,哪有你这样的内侄?你若信口胡言,害得史娘娘身遭斩首,你岂不怕天下人咒骂于你?你若招出真情,请列位大人,作一本首,老夫作一本尾,保你在朝,做上大官,你是何等不喜,哪些不乐?你若执意不招,一刀将你斩首,想你乃天下英雄,岂能做这刀头之鬼?

(刺客!)

(谢四 有!)

有道是:(念)要学天下奇男子(或:为人学得梁鸿志),方显男儿大丈夫!(你要再思啊再想。哈哈,哈哈,啊,哈哈哈\ldots{}\ldots{}(笑介))

【西皮原板】你本是天下的英雄好汉,

(谢四 诶,本来是英雄好汉。)

【西皮原板】为什么当刺客下贱不堪。招实供史娘娘感你的恩德匪浅,晋王爷他必然封你在帘外为官(或:圣天子龙心喜必封你帘外为官)。

【西皮原板】如不然我和你把帖来换,

(谢四 我不敢呐。)

【西皮原板】我为兄你为弟同列朝班。

(谢四 越发地不敢。)

【西皮原板】你好比错行路大大地弯转,这件事何须我替你为难(或:替你忧烦)。

(来,劝他招了口供,尔等皆有赏。)

(闵觉归大座)

(皂隶 朋友,我家大人下得位来百般相劝与你,叫你招了实供,邀请列位大人作一
本头,我家大人作一本尾,保你在晋朝之中,做一员大官。是何等儿不喜,
哪些儿不乐?朋友,你想做官的好,是挨刀的好呢。)

(谢四 待我思忖思忖。)

(谢四 好一个闵大人,好一位闵青天!下得位来百般相劝与俺,教俺招了实供,
邀请列位大人作一本头,闵大人作一本尾,保俺在这晋朝之中做一员大官。
是何等儿不喜,是哪些儿不乐?我为报他人之恩,断送俺的性命。哎,看在做官的份上,我有招,我有招,招了上来。)

教他画供。

(谢四
诶呀!招不得呀招不得!俺若招了实情,俺的恩人岂不是一刀两断,哪里还有什么官做?咳,我想世上恩只将恩报,哪有恩将仇报之理!
呔,闵大人!你为史娘娘乃是一忠,俺为俺主乃是一义,你做你的忠臣,俺做俺的义士!你教我招的什么?!)

将他吊了起来。(或:与我吊了起来。)

(问他有招无招。)

(皂隶 有招无招。)

(谢四 \ldots{}\ldots{}无有招对。)

敲牙一颗。

(谢四 呜\ldots{}\ldots{})

(问他有招无招。)

(皂隶 有招无招。)

(谢四 \ldots{}\ldots{}无有招对。)

再敲!

(谢四 呜\ldots{}\ldots{})

(问他有招无招。)

(谢四 无有什么招的。)

(贾昱 割他的舌头!)

且慢呐!

哎呀!

【西皮散板】听说是割舌根心惊肉颤,吓得我魂灵儿飞上九天。下位来我这里用目观看,(或:听说是割舌尖神魂散乱,吓得我一阵阵胆战心寒。无奈何下位去刺客来看,)

【西皮散板】尊一声众大人细听我言(或:见他口内吐鲜血我心不安)。

列位大人,下官有意,将他带回衙去,审问明白,再奏龙颜。不知列位大人,意下如何?(或:列位大人,且慢启奏龙颜,待下官将刺客带回衙去,审明回奏。)

(四朝官 但凭大人。)

带回衙去。(或:搭了下去。)

(皂隶搀谢下)

告辞了。

【西皮摇板】辞别了众大人忙回衙转(或:辞列公施一礼抽身回转) ,

(闵觉出来,四朝官下,贾昱(或:贺道庵)过去)

(贾昱 拿来我看。(或:贺道庵 拿来我看。))

(呀呸!)

【西皮摇板】这件事你二人定有牵连 (早知道这内中有你牵连) 。

哈哈,哈哈,啊------

呜\ldots{}\ldots{}(门子搀闵下)

(贾昱 请呐。)

(贺道庵 请呐。)

\newpage
\hypertarget{ux6851ux56edux5bc4ux5b50-ux4e4b-ux9093ux4f2fux9053}{%
\subsection{桑园寄子 之
邓伯道}\label{ux6851ux56edux5bc4ux5b50-ux4e4b-ux9093ux4f2fux9053}}

{[}第一场{]}

{[}引子{]}家道兴隆,训子嗣,早成功名。

(念)人生在世几度秋,好似杨花水上浮。有朝一日狂风落,大限来时一笔勾。

老汉邓伯道。兄弟伯俭,不幸中年丧命,今当他周年之期,我不免带领两个孩儿,去至坟前一祭。

家院,

有请二位少爷。

罢了。

你二人坐下\protect\hyperlink{fn301}{\textsuperscript{301}}。

今当你叔父周年之期,为父备定祭礼,带领你们去往坟前一祭。

祭礼走上。

{[}第二场{]}

唉,难得见的兄弟呀!呃,呃,呃\ldots{}\ldots{}(哭介)

【二黄慢板】叹兄弟遭不幸一旦丧命,丢下了年幼儿好不伤情。眼望着孤坟台珠泪难忍,见坟台不见人刀割我心。

\textless{}\textbf{叫头}\textgreater{}伯俭!兄弟!

今当你周年之期,愚兄带领两个孩儿,前来祭奠于你。来来来,受愚兄一陌纸钱。也不枉你我手足一场\protect\hyperlink{fn302}{\textsuperscript{302}}。

\textless{}\textbf{三叫头}\textgreater{}兄弟!伯俭!唉,兄弟呀!

【二黄导板】见坟台不由人珠泪滚滚,

\textless{}\textbf{三叫头}\textgreater{}伯俭!兄弟! 呜哙呀难得见的兄弟呀!

【回龙】叫一声同胞弟细听兄云。

【二黄快三眼】曾记得弟在世何等的侥幸,兄与弟同商议家道隆兴。料不想身得病一旦丧命,兄弟丧------命,兄弟呀!此黄土埋却了无价宝珍。\protect\hyperlink{fn303}{\textsuperscript{303}}

哦,儿要儿的叔父么?

这里面就是儿的叔父哇,

你要叫啊。

哦,儿要儿的爹爹么?

这里面就是儿的爹爹,你去叫他起来,同我们回去呀。呃,呃\ldots{}\ldots{}(哭介)

\textless{}\textbf{叫头}\textgreater{}伯俭!

你可曾听见呐,这两个孩儿,一个问我要他的叔父,一个问我要他的爹爹,你可曾听见呐!你聋了?你哑了?你,你,你睡死了哇\ldots{}\ldots{}(哭介)

【二黄散板】这一个要叔父我的心酸难忍,\protect\hyperlink{fn304}{\textsuperscript{304}}

【二黄散板】那一个要天伦刀割我心。

【二黄散板】哭一声同胞弟慢慢相等。

何事惊慌?

唉------呀,这才是福无双至,祸不单行!

儿呀,犹恐你母亲在家悬望,我们快快地回去呀!

{[}第三场{]}

回来了。

弟妹,大事不好了!

今有黑水国石勒造反,逢州夺州,遇县抢县,堪堪杀到我庄来了!

带领两个孩儿后面收拾收拾。

家丁们,看衣改换!

家丁们请上,受我全家------唉,一拜!(哭介)

【二黄散板】一家人跪草堂珠泪滚滚,叫一声众家丁细听分明:但愿得贼兵退此地安靖\protect\hyperlink{fn305}{\textsuperscript{305}},但愿得贼兵退也好回程。

【二黄散板】诉不尽衷肠苦急忙投奔。

{[}第四场{]}

【二黄慢板】走青山望白云------

【二黄慢板】山又高水又深难以忍耐,

【二黄慢板】手攀藤带娇儿忙登山界,忙登山------\protect\hyperlink{fn306}{\textsuperscript{306}}

【二黄慢板】眼望着白茫茫但不知何方地界,

【二黄散板】眼观得旌旗飘把我吓坏,

【二黄散板】又只见众贼兵蜂拥而来。

【二黄散板】手挽手带娇儿忙下山界。

{[}第五场{]}

吓煞我也,吓煞我也!

嗯。

儿啊,你的婶母呢?

儿啊,你的母亲呢?

怎么讲?

哎呀!

【二黄散板】听说是贼兵抢三魂不在,

\textless{}\textbf{三叫头}\textgreater{}弟妹!金氏!唉,弟妹呀!

【二黄散板】眼见得一家人四散分开。

儿啊,量他们走得不远,我们速速地赶上呃。

儿啊,你这是怎么样了?

这个\ldots{}\ldots{}

那旁有一土台,儿且站了上去,待为父的背儿一程就是。

不妨,儿只管的上去。

诶,儿呀,待我背了你哥哥,然后再来背你呀。

哦,来了。

哎!

【二黄散板】前世里欠下了冤孽魔债,老的老、小的小,好不伤怀。

哎呀!

我儿醒来。

儿啊,不要啼哭,你也站了上去,待为伯的也背儿一程就是。

为伯么\ldots{}\ldots{}

呃,不老,不老!

你,你,你只管的上去呀,呃\ldots{}\ldots{}(哭介)

诶,你大,他小,你还是哥哥呢。

哦,来了!

【二黄散板】小娇儿年纪小聪明可爱,可怜他无父的儿珠泪满腮。

你怎么又不走了?!

哎呀!

且住,看这两个小冤家,挨挨蹭蹭,行走不动。倘若贼兵到此,将我儿杀死,唉,倒也罢了哇;若是将我侄儿这一刀杀死------教我怎样对得过我那亡故的兄弟。这,这,这\ldots{}\ldots{}(搓手介)

有了!

看那旁有一桑园,我不免将我儿绑在树上,与他留下血书一道,然后背着我侄儿逃走,纵死九泉,也能对得过我那亡故的兄弟。我就是这个主意呀,我、我就是这个主意呀,呃\ldots{}\ldots{}(哭介)

你们站了起来!

行走半日,可曾饥饿?

抬头观看!

那旁有一桑园,桑葚长得茂盛,你们哪个上去,摘将下来,也好充饥呀。

呃,慢来慢来。

你小哇,看摔下来呀。

着哇,原要你上去呀,呃呃\ldots{}\ldots{}(哭介)

儿啊,脸朝外站。

儿啊!非是为父的心狠呐。儿来看------只因你兄弟年小啊,行走不动。倘若贼兵到此,将我儿杀死,为父的倒也落得个干净;若是将你兄弟这一刀杀死,教为父的怎样对得过你那亡故的叔父病床之上托孤之情?万般无奈,将儿绑在树上。与儿留下血书一道,然后背着你兄弟逃走。但愿贼兵不打此经过,有那仁人君子将儿救下。儿啊!你就有了活命了。你我父子日后还有相逢之日;倘若贼兵到此将我儿这一刀------杀死,儿啊,这也是儿遭劫在数,大数难逃。你我父子今生今世再若相逢,只怕是万万不得能够哇,呃呃\ldots{}\ldots{}(哭介)

\textless{}\textbf{三叫头}\textgreater{}邓方!我儿!唉,儿啊!

【二黄散板】此时间顾不得父子恩爱,眼见得亲骨肉两下分开。

【二黄散板】急忙忙扯下了这衣襟一块呀,

【二黄散板】咬指尖腹内痛珠泪满腮。

【二黄散板】我家住在太原府文水县界,我的名叫邓伯道逃难此来。舍亲生救侄儿流传后代,也免得旁人骂我年老无才。

【二黄散板】将儿的年庚月血书上载,仁君子你、你、你\ldots{}\ldots{}救了去,我佛如来。

哎呀!

【二黄散板】血迹干书不尽恩深似海呀。

儿啊,走哇。

为何?

哎呀!

儿啊,你的母亲来了!

呃,在这里呃------

\textless{}\textbf{三叫头}\textgreater{}弟妹!金氏!儿啊\ldots{}\ldots{}(哭介)

罢!

\item
  \leavevmode\hypertarget{fn136}{}%
  ``奉行''是``遵照执行''之意;``头戴乌纱奉行先''陈宫身为县令,为一县表率;\protect\hyperlink{fnref136}{↩}
\item
  \leavevmode\hypertarget{fn137}{}%
  ``开可''是``许可''的意思;\protect\hyperlink{fnref137}{↩}
\item
  \leavevmode\hypertarget{fn138}{}%
  ``家邑''本意为``采地'',这里表示陈宫管辖之中牟县;

  ``循吏''见于《史记》的《循吏列传》,一般指实施、推行善政、口碑声好的州、县级地方官;\protect\hyperlink{fnref138}{↩}
\item
  \leavevmode\hypertarget{fn139}{}%
  ``水地天''乃``尧天、舜地、禹水''之意。

  吴小如先生学的定场诗作``\textbf{头戴乌纱奉孝先},\textbf{慈祥恺悌万民欢}。\textbf{嘉言犹如湖中地},\textbf{得配汪洋水底天}。''后两句源自《后汉书·黄宪传》:郭林宗评黄宪(叔度)``汪汪若千顷之陂,澄之不清,淆之不浊,不可量也。''\protect\hyperlink{fnref139}{↩}
\item
  \leavevmode\hypertarget{fn140}{}%
  段公平君建议作``王升'',此处从``中国京剧戏考''网站《戏考》第一册本。\protect\hyperlink{fnref140}{↩}
\item
  \leavevmode\hypertarget{fn141}{}%
  夏行涛君建议作``定妥''。\protect\hyperlink{fnref141}{↩}
\item
  \leavevmode\hypertarget{fn142}{}%
  陈超老师介绍:``宿店''一场,陈宫坐小边,【二黄慢板】也坐小边虎头椅。\protect\hyperlink{fnref142}{↩}
\item
  \leavevmode\hypertarget{fn143}{}%
  陈超老师介绍:起二更时陈宫有个身段:搭左腿,左手水袖搭在左膝上,右手扶座。\protect\hyperlink{fnref143}{↩}
\item
  \leavevmode\hypertarget{fn144}{}%
  ``\textbf{陶恭祖望救兵营门立等,备哪有闲心肠来饮杯巡。}''也可以唱\textbf{【西皮二六】,刘曾复先生说戏时亦作了示范。}\protect\hyperlink{fnref144}{↩}
\item
  \leavevmode\hypertarget{fn145}{}%
  夏行涛君建议作``迎门立等''。\protect\hyperlink{fnref145}{↩}
\item
  \leavevmode\hypertarget{fn146}{}%
  夏行涛君建议作``大功易成''。\protect\hyperlink{fnref146}{↩}
\item
  \leavevmode\hypertarget{fn147}{}%
  刘曾复先生录音中念的是``占得天子'',但按文意``天时''更通顺。\protect\hyperlink{fnref147}{↩}
\item
  \leavevmode\hypertarget{fn148}{}%
  《论语·子罕》:子贡日:``有美玉于斯,韫匮而藏诸?求善贾而沽诸?''子日:``沽之哉!沽之哉!我待贾者也。''\protect\hyperlink{fnref148}{↩}
\item
  \leavevmode\hypertarget{fn149}{}%
  段公平君指出,``作幕''疑作``作掾'',因形似``作录''以致讹作。掾,后为副官佐或官署属员的通称。《全唐文》杜牧有``作掾京兆'';《元好问集》有``先夫人每以作掾为讳''。\protect\hyperlink{fnref149}{↩}
\item
  \leavevmode\hypertarget{fn150}{}%
  ``中国京剧戏考''网站《戏考》第一册本作``把贼扫''。\protect\hyperlink{fnref150}{↩}
\item
  \leavevmode\hypertarget{fn151}{}%
  一般俗作``褴衫''。李楠君按:``蓝衫''是职位低下的官吏的职服,考诸剧情,祢衡先着蓝衫觐见曹操,继而换破衣褴衫,最后赤身裸体,当是。\protect\hyperlink{fnref151}{↩}
\item
  \leavevmode\hypertarget{fn152}{}%
  吴焕老师整理的剧本记作``将身来在东廊道'',并注:``刘老云,老本旧词此句唱`将身来在西廊道',下面所接锣鼓为\textless{}\textbf{快长锤}\textgreater{},而并非\textless{}\textbf{双楗子}\textgreater{}''。\protect\hyperlink{fnref152}{↩}
\item
  \leavevmode\hypertarget{fn153}{}%
  旧谓子女的身体为父母所生,因称子女的身体为父母的``遗躰''。《大戴礼记·曾子大孝》:``身者,亲之遗躰也。''一本作``遗体''。\protect\hyperlink{fnref153}{↩}
\item
  \leavevmode\hypertarget{fn154}{}%
  此句李楠君从刘曾复先生学作``有朝大展经纶手''。\protect\hyperlink{fnref154}{↩}
\item
  \leavevmode\hypertarget{fn155}{}%
  祖考,泛指父祖之辈。\protect\hyperlink{fnref155}{↩}
\item
  \leavevmode\hypertarget{fn156}{}%
  ``祧''为远祖之庙。宗祧即宗庙,引申为祖业。\protect\hyperlink{fnref156}{↩}
\item
  \leavevmode\hypertarget{fn157}{}%
  ``志平生''夏行涛君建议作``助平升'',此处从《京剧汇编》第八十五集
  马连良藏本;录音中刘曾复先生念``天子''应作``夫子'',据夏行涛君告,``英雄几见称夫子,豪杰如斯乃圣人''是清代中叶理学家夏力恕为湖北孝感关帝庙作的对联。\protect\hyperlink{fnref157}{↩}
\item
  \leavevmode\hypertarget{fn158}{}%
  据樊百乐君告,刘曾复先生曾言,前辈艺人沿袭``尊崇关帝''的旧俗,台上往往将``青龙刀''或``青龙偃月''的``龙''念``铜(tóng)''音,也可视为一种``避讳''。刘先生学戏时``青龙刀''念法即此路数。\protect\hyperlink{fnref158}{↩}
\item
  \leavevmode\hypertarget{fn159}{}%
  刘曾复先生录音中似作``天日行'',参考《京剧汇编》第八十五集
  马连良藏本,此处作``实望汉室天日倾'',故``誓挽汉室天日倾''文意更确。\protect\hyperlink{fnref159}{↩}
\item
  \leavevmode\hypertarget{fn160}{}%
  刘曾复先生钞本注:``汪桂芬、王凤卿、余叔岩、贾洪林
  派,与富(连成)社不同''。

  (戏中所有鲁肃下场为王凤卿授)

  剧本中有关人物、场次和调度由段公平君协助整理。\protect\hyperlink{fnref160}{↩}
\item
  \leavevmode\hypertarget{fn161}{}%
  刘曾复先生钞本注:``此句可不念''。\protect\hyperlink{fnref161}{↩}
\item
  \leavevmode\hypertarget{fn162}{}%
  陈超老师介绍:这是鲁肃的第二个上场。\protect\hyperlink{fnref162}{↩}
\item
  \leavevmode\hypertarget{fn163}{}%
  此处不念《三国演义》原文中``伏路把关饶子敬,临江水战有周郎。''两句。\protect\hyperlink{fnref163}{↩}
\item
  \leavevmode\hypertarget{fn164}{}%
  刘曾复先生在为樊百乐君说戏时详细介绍了藏书的做工和细节。\protect\hyperlink{fnref164}{↩}
\item
  \leavevmode\hypertarget{fn165}{}%
  这个对儿是``真假难凭信,好歹问知音。''\protect\hyperlink{fnref165}{↩}
\item
  \leavevmode\hypertarget{fn166}{}%
  刘曾复先生钞本此处记为``蝼蚁尚生'',系脱漏,据录音补正。\protect\hyperlink{fnref166}{↩}
\item
  \leavevmode\hypertarget{fn167}{}%
  陈超老师介绍:此处鲁肃不端酒杯,更没有把酒泼在自己脸上的表演。\protect\hyperlink{fnref167}{↩}
\item
  \leavevmode\hypertarget{fn168}{}%
  陈超老师介绍:贾洪林说谭鑫培不允许(诸葛亮饮酒)。\protect\hyperlink{fnref168}{↩}
\item
  \leavevmode\hypertarget{fn169}{}%
  段公平君注``雄虎''亦作``熊虎''。\protect\hyperlink{fnref169}{↩}
\item
  \leavevmode\hypertarget{fn170}{}%
  夏行涛君建议作``起首''。姜骏按:``起首''为开始、起先之意;``起手''有动手、下手之意,亦有开始之意。\protect\hyperlink{fnref170}{↩}
\item
  \leavevmode\hypertarget{fn171}{}%
  段公平君建议作``荣辱事''。\protect\hyperlink{fnref171}{↩}
\item
  \leavevmode\hypertarget{fn172}{}%
  \textbf{原来老路子是单起霸,与《群英会》黄盖、甘宁同。}\protect\hyperlink{fnref172}{↩}
\item
  \leavevmode\hypertarget{fn173}{}%
  \textbf{金少山戴大镫,后改倒缨盔。}\protect\hyperlink{fnref173}{↩}
\item
  \leavevmode\hypertarget{fn174}{}%
  双起霸黄忠、魏延分着念黄忠单起霸的词。如单起霸,魏延的念为``(念)威风凛凛杀气飘,万马军中逞英豪。丹心一片扶社稷,深谢皇恩保汉朝。''\protect\hyperlink{fnref174}{↩}
\item
  \leavevmode\hypertarget{fn175}{}%
  刘曾复先生多次强调,黄忠、魏延对韩玄的称呼,应该称呼``都督''或``太守'',不称呼``元帅''。\protect\hyperlink{fnref175}{↩}
\item
  \leavevmode\hypertarget{fn176}{}%
  陈超老师介绍:这段\textbf{【西皮二六】设计得}很别致。七个字一句\textbf{【二六】},按十个字一句\textbf{【二六】}唱。\protect\hyperlink{fnref176}{↩}
\item
  \leavevmode\hypertarget{fn177}{}%
  此为北京谭派、汪派打法。上海有``四门斗''、\textless{}\textbf{柳青娘\textgreater{}}唢呐牌子打法。\protect\hyperlink{fnref177}{↩}
\item
  \leavevmode\hypertarget{fn178}{}%
  \textbf{陈超老师介绍:黄忠落马,念时,关羽始终举刀纹丝不动。}\protect\hyperlink{fnref178}{↩}
\item
  \leavevmode\hypertarget{fn179}{}%
  \textbf{陈超老师介绍:}关羽念到``换马再战'',右手一拄刀,左手慢捋髯,很威严。\protect\hyperlink{fnref179}{↩}
\item
  \leavevmode\hypertarget{fn180}{}%
  夏行涛君注:``旌旗起''当作``旌节旗(或:旌捷旗)''------《金瓶梅》第十二回作``眼望旌节旗'';《琵琶记》作``眼望旌捷旗''。\protect\hyperlink{fnref180}{↩}
\item
  \leavevmode\hypertarget{fn181}{}%
  此处按老路子不转【西皮快板】。\protect\hyperlink{fnref181}{↩}
\item
  \leavevmode\hypertarget{fn182}{}%
  此处及\textbf{以上几句一般没有唱,谭派、余派有唱。}\protect\hyperlink{fnref182}{↩}
\item
  \leavevmode\hypertarget{fn183}{}%
  \textbf{表示悬挂韩玄人头示众。}\protect\hyperlink{fnref183}{↩}
\item
  \leavevmode\hypertarget{fn184}{}%
  ``性情有''也有唱``性情拗''的。\protect\hyperlink{fnref184}{↩}
\item
  \leavevmode\hypertarget{fn185}{}%
  此处刘曾复先生录音疑似有缺失,\textbf{根据《马连良演出剧本选集》添加。}\protect\hyperlink{fnref185}{↩}
\item
  \leavevmode\hypertarget{fn186}{}%
  段公平君注:这两句刘曾复先生另有作:\textbf{鲁子敬再不能旁观袖手,望都督三思行另定良谋。}\protect\hyperlink{fnref186}{↩}
\item
  \leavevmode\hypertarget{fn187}{}%
  段公平君据2005年5月29日刘曾复先生与段公平、樊百乐谈话录音整理(非正式说戏录音)\protect\hyperlink{fnref187}{↩}
\item
  \leavevmode\hypertarget{fn188}{}%
  \textbf{末句刘曾复先生未能忆起。}

  \textbf{陈超老师介绍:他随刘曾复先生学的词句是``曹贼兴动人和马,进犯荆州把孤拿。四弟之言非虚假,瞒哄郡主及早还家。''尤其最后一句是八个字,挺特别。}\protect\hyperlink{fnref188}{↩}
\item
  \leavevmode\hypertarget{fn189}{}%
  刘曾复先生说戏时说明:此处\textbf{后来多不唱,刘备直接\textless{}水底鱼\textgreater{}上。}\protect\hyperlink{fnref189}{↩}
\item
  \leavevmode\hypertarget{fn190}{}%
  ``衿''的本意为正装,同``襟'';也可指系衣裳的带子。\protect\hyperlink{fnref190}{↩}
\item
  \leavevmode\hypertarget{fn191}{}%
  《京剧汇编》第十三集
  马连良藏本作``花瓣落''。\protect\hyperlink{fnref191}{↩}
\item
  \leavevmode\hypertarget{fn192}{}%
  《京剧汇编》第十三集
  马连良藏本作``以势力''。\protect\hyperlink{fnref192}{↩}
\item
  \leavevmode\hypertarget{fn193}{}%
  ``公厅''是官衙的意思。\protect\hyperlink{fnref193}{↩}
\item
  \leavevmode\hypertarget{fn194}{}%
  《京剧汇编》第十三集
  马连良藏本作``移奔''。\protect\hyperlink{fnref194}{↩}
\item
  \leavevmode\hypertarget{fn195}{}%
  该戏的相关场次的身段表演参阅李舒先生遗作《涉艺所得》所录的《刘曾复谈话、书信摘录》部分。\protect\hyperlink{fnref195}{↩}
\item
  \leavevmode\hypertarget{fn196}{}%
  《京剧汇编》第一百零七集作``一来''、``二来''。\protect\hyperlink{fnref196}{↩}
\item
  \leavevmode\hypertarget{fn197}{}%
  段公平君建议作``功得胜''。\protect\hyperlink{fnref197}{↩}
\item
  \leavevmode\hypertarget{fn198}{}%
  据樊百乐君告知,刘曾复先生强调,``牙车''是``牙床''的意思。\protect\hyperlink{fnref198}{↩}
\item
  \leavevmode\hypertarget{fn199}{}%
  陈超老师介绍,他随刘曾复先生学的此句是``抖擞精神上山道''。\protect\hyperlink{fnref199}{↩}
\item
  \leavevmode\hypertarget{fn200}{}%
  据《三国志·蜀书》载:``黄忠、赵云强挚壮猛,并作爪牙,其灌、滕之徒欤?''

  \textbf{陈超老师按}:

  《伐东吴》如果带``小桃园'',则是黄忠与吴班同上念此对儿。

  \textbf{陈超老师介绍带``小桃园''演法如下}:

  刘备封黄忠``以为随军副帅'',封吴班``前站先行'',二人领旨下;

  刘备再传令``哪位将军愿领副先锋?''关兴、张苞争功,比武、折箭后,再接黄忠``忆昔当年''。

  一般演出都不带``小桃园``。\protect\hyperlink{fnref200}{↩}
\item
  \leavevmode\hypertarget{fn201}{}%
  此处据《京剧汇编》第一百零一集
  马连良藏本增补。\protect\hyperlink{fnref201}{↩}
\item
  \leavevmode\hypertarget{fn202}{}%
  此处据《京剧汇编》第一百零一集
  马连良藏本增补。\protect\hyperlink{fnref202}{↩}
\item
  \leavevmode\hypertarget{fn203}{}%
  此处据《京剧汇编》第一百零一集
  马连良藏本增补。\protect\hyperlink{fnref203}{↩}
\item
  \leavevmode\hypertarget{fn204}{}%
  此处据《京剧汇编》第一百零一集
  马连良藏本作``吴班有言来禀告,破敌须防战马劳。老将军威风谁不晓,何妨饶他这一遭。''\protect\hyperlink{fnref204}{↩}
\item
  \leavevmode\hypertarget{fn205}{}%
  此处刘曾复先生只念``暂回师'',据上下文增补。\protect\hyperlink{fnref205}{↩}
\item
  \leavevmode\hypertarget{fn206}{}%
  此处刘曾复先生唱的是``\textbf{\ldots{}\ldots{}}斩将论英豪'',似欠通,此处从《京剧汇编》第一百零一集
  马连良藏本。\protect\hyperlink{fnref206}{↩}
\item
  \leavevmode\hypertarget{fn207}{}%
  此处据《京剧汇编》第一百零一集
  马连良藏本增补。\protect\hyperlink{fnref207}{↩}
\item
  \leavevmode\hypertarget{fn208}{}%
  《京剧汇编》第一百零一集
  马连良藏本此处作``旌旗飞龙影,干戈耀日明''。\protect\hyperlink{fnref208}{↩}
\item
  \leavevmode\hypertarget{fn209}{}%
  此处据《京剧汇编》第一百零一集
  马连良藏本增补。\protect\hyperlink{fnref209}{↩}
\item
  \leavevmode\hypertarget{fn210}{}%
  此句刘曾复先生录音不清楚,据文意添加。存疑。\protect\hyperlink{fnref210}{↩}
\item
  \leavevmode\hypertarget{fn211}{}%
  陈超老师介绍,他跟刘曾复先生学的此句``谋虑远''唱,``平吴不及定中原''一句【散板】。\protect\hyperlink{fnref211}{↩}
\item
  \leavevmode\hypertarget{fn212}{}%
  此处至本剧结尾,刘曾复先生只是大致示范,能听清个别词句。因此剧本中词句据《京剧汇编》第一百零一集
  马连良藏本增补。\protect\hyperlink{fnref212}{↩}
\item
  \leavevmode\hypertarget{fn213}{}%
  陈超老师介绍:《连营寨》前半出西皮,后半出昆腔,是这出戏的特点,陆逊先唱\textless{}\textbf{粉蝶儿}\textgreater{}、\textless{}\textbf{醉太平}\textgreater{},然后再唱九支曲子。\textbf{陆逊唱北曲},\textbf{其他人唱南曲}。\protect\hyperlink{fnref213}{↩}
\item
  \leavevmode\hypertarget{fn214}{}%
  段公平君建议作``旗偃戈收''。\protect\hyperlink{fnref214}{↩}
\item
  \leavevmode\hypertarget{fn215}{}%
  夏行涛君建议作``乃一''\protect\hyperlink{fnref215}{↩}
\item
  \leavevmode\hypertarget{fn216}{}%
  夏行涛君建议作``亘古流标''。\protect\hyperlink{fnref216}{↩}
\item
  \leavevmode\hypertarget{fn217}{}%
  据《三国志·吴书》载,陆逊是九江都尉陆骏之子。\protect\hyperlink{fnref217}{↩}
\item
  \leavevmode\hypertarget{fn218}{}%
  ``四至八道''是旧时标志土地界域的用语。表示四面八方所到之处及通往的道路。\protect\hyperlink{fnref218}{↩}
\item
  \leavevmode\hypertarget{fn219}{}%
  据李元皓君告知,``启祚'',是发祥、开创帝业之意。``中国京剧戏考''网站《戏考》第六册本作``起坐'',似非。\protect\hyperlink{fnref219}{↩}
\item
  \leavevmode\hypertarget{fn220}{}%
  段公平君建议作``路当阳''。\protect\hyperlink{fnref220}{↩}
\item
  \leavevmode\hypertarget{fn221}{}%
  根据刘曾复先生钞本整理,刘曾复先生说戏``平五路''是本剧的``观鱼遣邓''部分。刘曾复先生的钞本与《清车王府藏曲本(全印本)》\textsuperscript{{[}14{]}}第二册所收录的``安五路(总讲)''基本一致,但``安五路(总讲)''没有最后的``邓芝扑油鼎''部分。\protect\hyperlink{fnref221}{↩}
\item
  \leavevmode\hypertarget{fn222}{}%
  刘曾复先生钞本注``小生戴髯'',即曹丕归小生行应工。\protect\hyperlink{fnref222}{↩}
\item
  \leavevmode\hypertarget{fn223}{}%
  刘曾复先生钞本作``令起起羌兵十万'',似欠通;此处从``安五路(总讲)''。\protect\hyperlink{fnref223}{↩}
\item
  \leavevmode\hypertarget{fn224}{}%
  刘曾复先生钞本作``合好''。\protect\hyperlink{fnref224}{↩}
\item
  \leavevmode\hypertarget{fn225}{}%
  进位是进升爵位,封号的意思。\protect\hyperlink{fnref225}{↩}
\item
  \leavevmode\hypertarget{fn226}{}%
  刘曾复先生钞本作``还班'',此处从``安五路(总讲)''原文。\protect\hyperlink{fnref226}{↩}
\item
  \leavevmode\hypertarget{fn227}{}%
  刘曾复先生钞本与``安五路(总讲)''均作``差官倒退跳赶\textless{}\textbf{度柳翠}\textgreater{}'',经何毅老师指教,\textless{}\textbf{度柳翠}\textgreater{}是牌子名,``干''表示是``干牌子''。\protect\hyperlink{fnref227}{↩}
\item
  \leavevmode\hypertarget{fn228}{}%
  刘曾复先生钞本作``自因''。\protect\hyperlink{fnref228}{↩}
\item
  \leavevmode\hypertarget{fn229}{}%
  刘曾复先生钞本以下``报子''均作``探子'',应可通。\protect\hyperlink{fnref229}{↩}
\item
  \leavevmode\hypertarget{fn230}{}%
  刘曾复先生钞本和``安五路(总讲)''所有``成都''均作``城都''。\protect\hyperlink{fnref230}{↩}
\item
  \leavevmode\hypertarget{fn231}{}%
  刘曾复先生钞本``好生''二字不确认,疑作``如此''或``着实'',此处从``安五路(总讲)''。\protect\hyperlink{fnref231}{↩}
\item
  \leavevmode\hypertarget{fn232}{}%
  刘曾复先生钞本作``合好''。\protect\hyperlink{fnref232}{↩}
\item
  \leavevmode\hypertarget{fn233}{}%
  刘曾复先生钞本注``带`扑油鼎'可考虑不要此场。''\protect\hyperlink{fnref233}{↩}
\item
  \leavevmode\hypertarget{fn234}{}%
  刘曾复先生钞本作``万岁圣宽怀,暂且放心。'',此处从``安五路(总讲)''。\protect\hyperlink{fnref234}{↩}
\item
  \leavevmode\hypertarget{fn235}{}%
  刘曾复先生钞本作``诸事毕''``安五路(总讲)''此处补``已'',此处从之。\protect\hyperlink{fnref235}{↩}
\item
  \leavevmode\hypertarget{fn236}{}%
  这一场是``安五路(总讲)''没有的。\protect\hyperlink{fnref236}{↩}
\item
  \leavevmode\hypertarget{fn237}{}%
  ``霄晓勿遑''即不分昼夜之意。\protect\hyperlink{fnref237}{↩}
\item
  \leavevmode\hypertarget{fn238}{}%
  刘曾复先生钞本作``小官不知故''。\protect\hyperlink{fnref238}{↩}
\item
  \leavevmode\hypertarget{fn239}{}%
  刘曾复先生钞本作``这却何地'',文意欠通。\protect\hyperlink{fnref239}{↩}
\item
  \leavevmode\hypertarget{fn240}{}%
  段公平君注:``多管'',即多半,大概之意。多见于元明小说、话本等。。\protect\hyperlink{fnref240}{↩}
\item
  \leavevmode\hypertarget{fn241}{}%
  姜嫄是传说中上古农神``后稷''之母,非常贤德,后世尊为``圣母'';嫫母是传说中的丑女,是黄帝的次妃。\protect\hyperlink{fnref241}{↩}
\item
  \leavevmode\hypertarget{fn242}{}%
  段公平君注:``尚尔'':即尚且之意。如纪昀《阅微草堂笔记·滦阳消夏录五》:``对神尚尔,对人可知''。\protect\hyperlink{fnref242}{↩}
\item
  \leavevmode\hypertarget{fn243}{}%
  刘曾复先生钞本作``胡卢闷''。``闷葫芦''比喻极难猜透或令人纳闷的事或话。\protect\hyperlink{fnref243}{↩}
\item
  \leavevmode\hypertarget{fn244}{}%
  谋猷为计谋,谋略之意。\protect\hyperlink{fnref244}{↩}
\item
  \leavevmode\hypertarget{fn245}{}%
  刘曾复先生钞本未注明板式,下同。\protect\hyperlink{fnref245}{↩}
\item
  \leavevmode\hypertarget{fn246}{}%
  刘曾复先生钞本作``加常'',系误;此处从``安五路(总讲)''。\protect\hyperlink{fnref246}{↩}
\item
  \leavevmode\hypertarget{fn247}{}%
  刘曾复先生钞本与``安五路(总讲)''均作``慌慌无策'',此处从《三国演义》原文。\protect\hyperlink{fnref247}{↩}
\item
  \leavevmode\hypertarget{fn248}{}%
  刘曾复先生钞本作``岂不等'',此处从``安五路(总讲)''。\protect\hyperlink{fnref248}{↩}
\item
  \leavevmode\hypertarget{fn249}{}%
  刘曾复先生钞本与``安五路(总讲)''均作``推病为词''。\protect\hyperlink{fnref249}{↩}
\item
  \leavevmode\hypertarget{fn250}{}%
  段公平君注:刘曾复先生钞本此句作``皇儿拜他以为相父称之'',文意欠通,疑是``皇儿拜他以为相父''和``皇儿以相父称之''两句错杂而成。考``安五路(总讲)''原亦作``皇儿拜他以为相父称之'',后删去``称之'',作``皇儿拜他以为相父'',此处从``安五路(总讲)''。\protect\hyperlink{fnref250}{↩}
\item
  \leavevmode\hypertarget{fn251}{}%
  ``安五路(总讲)''此处原作``求计'',改为``问计''。\protect\hyperlink{fnref251}{↩}
\item
  \leavevmode\hypertarget{fn252}{}%
  刘曾复先生钞本注``以下`观鱼遣邓'''。\protect\hyperlink{fnref252}{↩}
\item
  \leavevmode\hypertarget{fn253}{}%
  ``安五路(总讲)''本作``礼论不雅'',旁注``也是无法'';李元皓君注``不雅'',犹言``君不登臣门''之意。\protect\hyperlink{fnref253}{↩}
\item
  \leavevmode\hypertarget{fn254}{}%
  刘曾复先生说戏录音作``这鱼你'',此处从``安五路(总讲)''。\protect\hyperlink{fnref254}{↩}
\item
  \leavevmode\hypertarget{fn255}{}%
  刘曾复先生钞本与
  ``安五路(总讲)''均作``求条良谋''。\protect\hyperlink{fnref255}{↩}
\item
  \leavevmode\hypertarget{fn256}{}%
  刘曾复先生钞本作``兵伐五路''。\protect\hyperlink{fnref256}{↩}
\item
  \leavevmode\hypertarget{fn257}{}%
  刘曾复先生钞本作``人民振动''。\protect\hyperlink{fnref257}{↩}
\item
  \leavevmode\hypertarget{fn258}{}%
  刘曾复先生说戏录音中似作``神在边关之外'';段公平君认为说戏录音误作``身在边关之外''。\protect\hyperlink{fnref258}{↩}
\item
  \leavevmode\hypertarget{fn259}{}%
  刘曾复先生说戏录音作``韬略''。\protect\hyperlink{fnref259}{↩}
\item
  \leavevmode\hypertarget{fn260}{}%
  刘曾复先生说戏录音作``潜送''。\protect\hyperlink{fnref260}{↩}
\item
  \leavevmode\hypertarget{fn261}{}%
  刘曾复先生说戏录音作``山岭峻险''。\protect\hyperlink{fnref261}{↩}
\item
  \leavevmode\hypertarget{fn262}{}%
  刘曾复先生钞本作``急早回奏''。\protect\hyperlink{fnref262}{↩}
\item
  \leavevmode\hypertarget{fn263}{}%
  刘曾复先生钞本作``连和'',此处从《三国演义》原文。\protect\hyperlink{fnref263}{↩}
\item
  \leavevmode\hypertarget{fn264}{}%
  刘曾复先生钞本注,此段可不唱。\protect\hyperlink{fnref264}{↩}
\item
  \leavevmode\hypertarget{fn265}{}%
  刘曾复先生钞本此处径写``点将诗''。疑是\textless{}\textbf{点绛唇}\textgreater{}牌子,后接\textless{}\textbf{定场诗}\textgreater{}四句。\protect\hyperlink{fnref265}{↩}
\item
  \leavevmode\hypertarget{fn266}{}%
  刘曾复先生钞本中``厉害''均作``利害''。\protect\hyperlink{fnref266}{↩}
\item
  \leavevmode\hypertarget{fn267}{}%
  刘曾复先生钞本未注明板式。\protect\hyperlink{fnref267}{↩}
\item
  \leavevmode\hypertarget{fn268}{}%
  刘曾复先生钞本作``可于光殿前''(``光''字不确认,疑此字误衍或脱漏,如作``光明''),此处从《三国演义》原文。\protect\hyperlink{fnref268}{↩}
\item
  \leavevmode\hypertarget{fn269}{}%
  刘曾复先生钞本作``身长大面'',此处从《三国演义》原文作``身长面大''。\protect\hyperlink{fnref269}{↩}
\item
  \leavevmode\hypertarget{fn270}{}%
  刘曾复先生钞本作``常揖不拜'',此处从《三国演义》原文。\protect\hyperlink{fnref270}{↩}
\item
  \leavevmode\hypertarget{fn271}{}%
  刘曾复先生钞本作``尔想'',似欠通。\protect\hyperlink{fnref271}{↩}
\item
  \leavevmode\hypertarget{fn272}{}%
  刘曾复先生钞本作``窃𢫑中原'',``𢫑''同``據''。\protect\hyperlink{fnref272}{↩}
\item
  \leavevmode\hypertarget{fn273}{}%
  刘曾复先生钞本疑``为''或``不''字,段公平君注:``他能可'',系``他可能''颠倒。全句为反诘语气。\protect\hyperlink{fnref273}{↩}
\item
  \leavevmode\hypertarget{fn274}{}%
  ``连横''亦作``连衡''。\protect\hyperlink{fnref274}{↩}
\item
  \leavevmode\hypertarget{fn275}{}%
  奫,水深广的样子。刘曾复先生钞本注``奫(音
  氲)''。\protect\hyperlink{fnref275}{↩}
\item
  \leavevmode\hypertarget{fn276}{}%
  此戏的文字结合了樊百乐君提供的刘曾复先生说戏的实况录音整理的。刘先生为百乐君说此戏时,因为没有找到剧本,部分词句是临时回忆介绍的,因此有些小的地方与为戏曲学院说戏的词句略有出入。\protect\hyperlink{fnref276}{↩}
\item
  \leavevmode\hypertarget{fn277}{}%
  ``带砺山河''亦作``带厉山河'',这里借指诸葛亮的忠贞之心恒久不变。\protect\hyperlink{fnref277}{↩}
\item
  \leavevmode\hypertarget{fn278}{}%
  ``刻今''犹``刻下''之意,即现在,当下。\protect\hyperlink{fnref278}{↩}
\item
  \leavevmode\hypertarget{fn279}{}%
  段公平君建议作``一理'',并注:《出师表》有``宫中府中,俱为一体'',故选``一理''。夏行涛君建议``依例''或``依礼''也都合文意。\protect\hyperlink{fnref279}{↩}
\item
  \leavevmode\hypertarget{fn280}{}%
  这段【西皮快板】原词较长,兹录如下:

  ``镇北将军名魏延。自从长沙来降汉,跟随山人二十年。今日战比不得往日战,比不得当年大战在渭南。四更时分造战饭,要出兵来五更天。假扮姜维关前站,口口声声出反言。''\protect\hyperlink{fnref280}{↩}
\item
  \leavevmode\hypertarget{fn281}{}%
  ``陉''是山脉中断的地方,这样的地方往往是重要的关隘。这里特指街亭。\protect\hyperlink{fnref281}{↩}
\item
  \leavevmode\hypertarget{fn282}{}%
  刘曾复先生示范说戏时介绍,这是慈瑞泉临场抓的哏。\protect\hyperlink{fnref282}{↩}
\item
  \leavevmode\hypertarget{fn283}{}%
  刘曾复先生示范说戏时介绍,此句原作``叉出帐去免责问''。\protect\hyperlink{fnref283}{↩}
\item
  \leavevmode\hypertarget{fn284}{}%
  ``拿云手''比喻远大的志向。\protect\hyperlink{fnref284}{↩}
\item
  \leavevmode\hypertarget{fn285}{}%
  陈超老师介绍:刘曾复先生所传的是贾丽川家的路子。\protect\hyperlink{fnref285}{↩}
\item
  \leavevmode\hypertarget{fn286}{}%
  刘曾复先生钞本作``遍身是汗''。\protect\hyperlink{fnref286}{↩}
\item
  \leavevmode\hypertarget{fn287}{}%
  陈超老师介绍了这一场相关的舞台布局及调度。\protect\hyperlink{fnref287}{↩}
\item
  \leavevmode\hypertarget{fn288}{}%
  《三国演义》原文为``\textbf{永延汉}祀''。\protect\hyperlink{fnref288}{↩}
\item
  \leavevmode\hypertarget{fn289}{}%
  刘曾复先生钞本作``南斗合北斗''。\protect\hyperlink{fnref289}{↩}
\item
  \leavevmode\hypertarget{fn290}{}%
  刘曾复先生钞本作``看看''。\protect\hyperlink{fnref290}{↩}
\item
  \leavevmode\hypertarget{fn291}{}%
  刘曾复先生钞本作``只望''。\protect\hyperlink{fnref291}{↩}
\item
  \leavevmode\hypertarget{fn292}{}%
  刘曾复先生钞本作``事到临头''。\protect\hyperlink{fnref292}{↩}
\item
  \leavevmode\hypertarget{fn293}{}%
  李楠君认为作``朝喜夕厌''。\protect\hyperlink{fnref293}{↩}
\item
  \leavevmode\hypertarget{fn294}{}%
  这一场戏一般省去,是王浚教训自己的儿子所唱。\protect\hyperlink{fnref294}{↩}
\item
  \leavevmode\hypertarget{fn295}{}%
  ``\textbf{坉''的意思是}用草袋装土筑墙或堵水。\protect\hyperlink{fnref295}{↩}
\item
  \leavevmode\hypertarget{fn296}{}%
  夏行涛君建议此句作``何不写状告于他?''\protect\hyperlink{fnref296}{↩}
\item
  \leavevmode\hypertarget{fn297}{}%
  此处原来唱``议论'',刘曾复先生听从吴小如先生建议,改唱``理论'',唱词文意更通顺。\protect\hyperlink{fnref297}{↩}
\item
  \leavevmode\hypertarget{fn298}{}%
  此戏的文字也结合了刘曾复先生两次为樊百乐君说戏的实况录音整理完成的。\protect\hyperlink{fnref298}{↩}
\item
  \leavevmode\hypertarget{fn299}{}%
  贺道庵也有本记作``贺道安''或``何道安''的。此处从《戏考》。\protect\hyperlink{fnref299}{↩}
\item
  \leavevmode\hypertarget{fn300}{}%
  刘曾复先生记忆中刺客本名作``谢二'',但在为樊百乐君说戏时所本作``谢四''。

  据段公平君告知:《曲海总目摘要》载清传奇《九莲灯》,丞相名``霍道南'',刺客名``獬儿''。\protect\hyperlink{fnref300}{↩}
\item
  \leavevmode\hypertarget{fn301}{}%
  此句陈超老师作``两旁坐下''。据陈超老师介绍:``两边坐下''是坐垫子,由检场的扔。\protect\hyperlink{fnref301}{↩}
\item
  \leavevmode\hypertarget{fn302}{}%
  陈超老师介绍,此处先烧纸,再上香。\protect\hyperlink{fnref302}{↩}
\item
  \leavevmode\hypertarget{fn303}{}%
  陈超老师介绍,此段王荣山教王又宸坐着唱,王又宸后改为站着唱。夏行涛君建议``此黄土''作``似黄土''。\protect\hyperlink{fnref303}{↩}
\item
  \leavevmode\hypertarget{fn304}{}%
  陈超老师介绍,唱此句场面起\textless{}\textbf{快扭丝}\textgreater{},一定站着唱。\protect\hyperlink{fnref304}{↩}
\item
  \leavevmode\hypertarget{fn305}{}%
  李舒先生遗作《涉艺所得》录《刘曾复修润剧本四篇》中《\textless{}御碑亭\textgreater{}及其他》一文中此句作``但愿得贼兵退此地安静''。\protect\hyperlink{fnref305}{↩}
\item
  \leavevmode\hypertarget{fn306}{}%
  陈超老师介绍此处老生表演为:

  左手往外转一个水袖,一抓藤,背身右手往外转一个水袖,一抓藤,滑步,倒步,把旦角和俩小孩挤到台口。旦角坐地,邓方扶旦角,老生扶邓元,再上山石片。

  \begin{quote}
  陈超老师按:这是老谭的身段。
  \end{quote}

  \protect\hyperlink{fnref306}{↩}
