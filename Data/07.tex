\addcontentsline{toc}{section}{\hfill[\hei 年代失考]\hfill}
\newpage
\chead{年代失考} % 页眉中间位置内容
\hypertarget{ux72b6ux5143ux8c31-ux4e4b-ux9648ux4f2fux611a}{%
\subsection{状元谱 之
陈伯愚}\label{ux72b6ux5143ux8c31-ux4e4b-ux9648ux4f2fux611a}}

{[}第一场{]}

(念)向阳门第春常在,积善之家庆有余。

陈植何事?

说我出迎。

啊,公道兄,

请到里面。

请坐。

罢了。

这都是谁家的孩儿?

陈植,带他们后面用饭。

请问公道兄你好哇?

我么?还好哇。

公道兄,你的生意如何?

此话怎讲?

哦,领教了。

不消。

请问公道兄到此何事?

陈植,取八人的口粮,棉布二匹。

连你夫妻二人在内,才得八人呐。

就是生产也不过是九人呐。

你是怎样晓得?

此话怎讲?

如此说来一定?

有准?

啊,呵呵哈哈哈哈\ldots{}\ldots{}(笑介)

取笑了。

陈植,取十人的口粮,棉布二匹。

岂敢。

啊,公道兄,不敢动问,你贵庚几何?

令正\protect\hyperlink{fn625}{\textsuperscript{625}}呢?

你夫妻二人今年俱才三十五岁,就有六个孩儿。真是好造化呀,好福气!

啊,呵呵哈哈哈\ldots{}\ldots{}(笑介)

我么?

唉,痴心妄想!

啊,呵呵哈哈哈\ldots{}\ldots{}(笑介)

陈植代送。

陈植,张公道带领谁家的孩儿前来冒领粮米?

他夫妻今年今年才得三十五岁,怎么就有六个孩儿?

哦,这是他祖上阴功积下来的?

唉!陈门的祖先呐!

【西皮慢板】张公道三十五六子有靠,陈伯愚年半百无有后苗。为儿女我也曾朝山拜庙,为儿女我也曾补路修桥。怕将来老天爷无有果报,眼睁睁有何人去把纸烧。

{[}第二场{]}

(念)无钱有子终有靠,有钱无子亦枉然。

何事?

哪个大相公回来了?

哦,大官儿他来了么?快快唤他进来。

罢了。

儿啊,你好哇?

为叔的么,还好,还好哇。

几载未见,儿在外面光景如何?

(惊介)你是陈大官,陈敏生?

儿为何这等模样?

你这做什么?

你快快讲来。

唗!大胆!

下站!

你,你,你,你,你讲来。

哦,儿在外面不习上进,当抵家财,失落功名。如今直落得这乞讨之途。

哈哈------哈哈------啊,呵呵哈哈哈哈\ldots{}\ldots{}

呵呵呵呵呵呵\ldots{}\ldots{}(冷笑介)

你既然落在乞讨之途,不在长街讨要,来到为叔的家中则甚呐?

哦,儿也是前来领取粮米的么?

是啊,我陈门的粮米,外人领得,难道说自家的孩儿还领不得么?

儿啊,近前来,为叔的有话对你讲啊。

唗!大胆!

放肆!

下站!

儿啊,你只管地前来。

儿是陈大官?陈敏生?

好奴才!

呀呸!

【西皮散板】提起了二爹娘要掌儿的嘴,陈门中岂要你这不肖的人。这样的奴才中何用,

(\textbf{陈伯愚打陈大官不是一前一后追着打}:老生走里圈、小生走外圈,彼此都看得见,打时是老生往右上方出板子,往左甩胡子,在脸前上方用板子头画一个圈,下来板子头打地,紧跟着一撤,小生是单袖(第一板右、第二板左、第三板右)盖头、起坐子,第四板第一番是小生接板跪,第二番是老生把陈植勾到中间板隔着陈植碰到小生头上、小生倒地、老生扔板,陈植拾板过去)\protect\hyperlink{fn626}{\textsuperscript{626}}

【西皮散板】活活打死你这败家的根。

安人来了。

谁家的孩儿要我来拷打?

你自己去看呐!

哼,好一个陈大官,好一个陈敏生。

呀呸!像这样的败家子弟,你与我养上几个,我打你再来心痛。不是你我半百夫妻,定要掌嘴!

呃------我不准你哭!

呃------我不准你嚎!

我看你们哪个大胆的敢哭哇!(哭介)

陈大官呐!

小奴才!

曾记得,儿一双爹娘染病在床,十分沉重。将我二老唤近床前,叫道:兄弟啊,弟妇哇,我二老病入膏肓,不久于人世,但死之后,别无挂念呐。惟有大官孩儿年小哇,放心不下。望你二老好好照看。可怜我那兄嫂,道罢此言,就双双一场------大梦啊。呃\ldots{}\ldots{}(哭介)

那时儿才将将的七岁呀,送往南学攻书八载,一十五岁身入黉门,是何等的侥幸呐?儿不该在外面听信旁人的闲言碎语,回得家来,与为叔的是朝吵暮闹。问起情由,儿要分门另过。本当不分,又恐外人不明者,说我欺压于他,道为叔的以大压小,以叔压侄呀。

万般无奈,将儿的亲身娘舅请至家中,说明此事。将这上等的家私,分与儿一大半呐。既已分居,儿就该在外面发愤攻书,求取上进的才是。怎么儿在外面不习上进,当抵家财,失落功名。如今直落得这,这,这乞讨之途!

安人,这个奴才今年多大年纪了?

着哇!二十一岁,还是什么小孩子吗?

陈门的祖先呐!不知哪辈先人在外面为官,错断民辞,检点不到,如今才有这样的败家子弟。

真真的气,气,气煞我也!

啊?!

呀呀呸!

(念)亦非愚呆并痴懵,不该败坏我门庭。叔侄好比黄粱梦,你是谁来我何人!

陈植过来,这奴才死了便罢,倘若不死,与我轰呃,

赶!

轰了出去。

嗯哼!

【西皮原板】老来无子甚悲惨,陈门中出了个不肖儿男。一步儿来在前庭院,

唉!

【西皮原板】见安人只哭得珠泪不干。

大官儿呢?

你可曾把他什么?

陈植,你呢?

唉!不会办事啊!

唤他回来。

唉!这个奴才,乘兴而来,败兴而去呀。

不是安人提起,我倒忘怀了。

陈植,准备祭礼,去往大员外的坟前一祭。

安人,请。

【西皮原板】陈大官辜负了青春年少,

(安人
【西皮原板】他不该\protect\hyperlink{fn627}{\textsuperscript{627}}在外面浪荡逍遥(或:放荡逍遥)。)

【西皮原板】怕的是到将来你我的大限来到,

(安人 【西皮原板】有何人到坟前把纸化烧。)

\textless{}\textbf{哭头}\textgreater{}陈大官,

(安人 \textless{}\textbf{哭头}\textgreater{}陈敏生啊,)

\textless{}\textbf{哭头}\textgreater{}啊,大官我的儿啊。

{[}第三场{]}

【西皮原板】艳阳天气正清明,

(安人 【西皮原板】家家户户上坟茔。)

【西皮原板】叩罢了祖先爷站立不稳,

(安人
【西皮原板】上前去搀夫君年迈之人\protect\hyperlink{fn628}{\textsuperscript{628}}。)

将祭礼赏与朱仓。

年年如此,有何兴趣。

由你。

彼此。

啊,啊,呵呵哈哈\ldots{}\ldots{}(笑介)

安人请。

安人,你来看,这青的------

(安人 是松。)

绿的------

(安人 是柏。)

阳关大道------

(安人 车马来行。)

好个``车马来行''。

你我不来,还有哪个前来?

不错,唤朱仓。

我来问你,大员外的坟前是哪个上了去了?

不像话。

唗!陈大官也是你这个奴才随便叫得的吗?

快快地唤他前来。

不像话。

唗!你为何将他推倒在地?

还不下去?!

唗!奴才,敢是偷盗树木来了?

哎呀安人呐!你我二老下世,这个奴才是一定不来的了哇!

哎呀,儿啊\ldots{}\ldots{}(哭介)

【西皮散板】大官儿说出了伤心话,

(安人 【西皮散板】倒教我年迈人珠泪如麻。)

就依安人。

不必拜了。

陈植,取衣帽前来与你大相公更换。

安人一同回去。

正是:(念)我儿改邪来归正,败子回头金不换呐。

着哇。

安人先行。

儿啊,从今以后,为父的有口不来骂你,有手不来打你。万贯家财付儿掌管。成人也在你,不成人也在你呀。

着哇,好个``成人还要自成人''。

儿啊,随为父的来呀。

呵呵哈哈哈\ldots{}\ldots{}(笑介)

你这是怎么样了?

呃------

  \leavevmode\hypertarget{fn625}{}%
  旧以嫡妻为正室,因用``令正''作称对方嫡妻的敬词。\protect\hyperlink{fnref625}{↩}
\item
  \leavevmode\hypertarget{fn626}{}%
  陈大官见陈伯愚,等到陈伯愚把陈大官招过来之后,是陈伯愚坐着把双袖搭在陈大官双肩之上,陈大官随着就劲跪,(\textbf{没有抓陈大官领子的动作}),只是用右袖把陈大官往里推倒,陈伯愚回身站起来抢过陈植要拿走的板子之前\textbf{没有陈大官要跑}、\textbf{陈伯愚关门把陈大官扔一个抢背的身段},是陈大官作揖等着受责。

  \textbf{王凤卿云:``《状元谱》陈伯愚打陈大官是长辈教训晚辈,不是打架。''}\protect\hyperlink{fnref626}{↩}
\item
  \leavevmode\hypertarget{fn627}{}%
  段公平君建议作``大不该''。\protect\hyperlink{fnref627}{↩}
\item
  \leavevmode\hypertarget{fn628}{}%
  夏行涛君建议作``上前去搀扶起年迈之人''。\protect\hyperlink{fnref628}{↩}
