\newpage
\phantomsection %实现目录的正确跳转
\section*{\large\hei {打鼓骂曹~{\small 之}~祢衡}}
\addcontentsline{toc}{section}{\hei 打鼓骂曹~{\small 之}~祢衡}

\hangafter=1                   %2. 设置从第1⾏之后开始悬挂缩进  %
\setlength{\parindent}{0pt}{

{\centerline{{[}{\hei 第一场}{]}}}\vspace{5pt}

{[}{\akai {\akai 引}子}{]}天宽地阔,运机谋,智广才多。

({\akai 念})口似悬河语似流,全凭舌尖运机谋。男儿若得擎天手,自然谈笑觅封侯。

卑某({\akai 或}:~卑末)姓祢名衡,字正平,乃平原孝义村人氏。幼读诗书,深通战策。(虽怀王佐之才,)少游北海,偶遇孔融。他将我荐与曹府门下作幕\footnote{段公平{\scriptsize 君}指出,``作幕''疑作``作掾'',因形似``作录''以致讹作。掾,后为副官佐或官署属员的通称。《全唐文》杜牧有``作掾京兆'';《元好问集》有``先夫人每以作掾为讳''。}。我想那曹操,名为汉相,实为汉贼,焉能敬贤礼士?此番进得相府({\akai 或}:~去至相府),必须见机而行。

正是:~({\akai 念})未遇圣命主,又愧栋梁才。({\akai 或}:~未逢圣明主,又愧栋梁才。)

\setlength{\hangindent}{56pt}{【{\akai 西皮快三眼}】平生志气运未通,似蛟龙困在浅水中。有朝一日春雷动,得会风云上九重。 }

\vspace{3pt}{\centerline{{[}{\hei 第二场}{]}}}\vspace{5pt}

(来也。)

\setlength{\hangindent}{56pt}{【{\akai 西皮快板}】相府门前杀气高,密密层层排枪刀。画阁雕梁双凤绕,亚赛天子九龙朝。 }

丞相在上,卑某({\akai 或}:~卑末)礼到。

卑某({\akai 或}:~卑末)姓祢名衡,字正平,乃平原孝义村人也。

呜哙呀!人言曹操轻贤慢士,今日一见果然名不虚传({\akai 或}:~果然话不虚传)。孔大夫,你把我错荐了。

\setlength{\hangindent}{56pt}{【{\akai 西皮快板}】人言曹贼多奸巧,果然亚似秦赵高。欺君误国非正道,全凭势力压当朝。站在丹墀微微笑,哪怕虎穴与笼牢。 }

呵呵哈哈哈$\cdots{}\cdots{}$({\hwfs 冷笑介})

吾笑天地宽阔,并无一人。

你({\akai 或}:~丞相)道你帐下,文能安国武能定邦({\akai 或}:~文能安邦,武能定国)。请问丞相,帐下文有谁能,武有谁高?

卑某({\akai 或}:~卑末)愿闻一二。

呵,呵,呵呵呵呵呵$\cdots{}\cdots{}$({\hwfs 冷笑介})

你道你帐下,尽是英雄豪杰。依卑某({\akai 或}:~卑末)看来,尽是些无用之辈呀。

听道:~荀彧、荀攸,可使吊丧问疾;

郭嘉、程昱,可使看墓守坟;

乐进、李典,可使牧羊放马;

许褚、张辽,

哎,也只可使击鼓鸣金呐。

曹子孝,呼为要钱太守;

夏侯惇,人称完肤将军。

余下者,尽是些衣架、饭囊,酒桶、肉袋,碌碌之辈,何足道哉?

区区不才,幼读诗书,深通战策。天文地理之书,无所不读;三教九流之事,无所不晓。上,可以致君为尧、舜;下,可以配德与孔、颜。吾乃天下名士,岂与你这奸贼同党。孔大夫,你把我错荐了。

\setlength{\hangindent}{56pt}{【{\akai 西皮快板}】平生志气与天高,不愿金钱结富豪。我本是堂堂青史表,岂与犬马共同槽。 }

(量你也不敢呐。)

这$\cdots{}\cdots{}$

愿为鼓吏。

呵,呵,呵呵呵$\cdots{}\cdots{}$({\hwfs 冷笑介})

\setlength{\hangindent}{56pt}{【{\akai 西皮二六}】丞相委用恩非小,用为鼓吏怎敢辞劳。背转身来微微笑,孔融做事也不高。明知曹贼多奸巧,全凭势力【{\footnotesize 转}{\akai 西皮快板}】压当朝。我越思越想心头恼,安排巧计骂奸曹。罢罢罢暂且忍下了,明天自有我的巧妙高。}

\vspace{3pt}{\centerline{{[}{\hei 第三场}{]}}}\vspace{5pt}

\setlength{\hangindent}{56pt}{【{\akai 西皮导板}】适才与贼一席话,}

\setlength{\hangindent}{56pt}{【{\akai 西皮散板}】气得我正平乱如麻。 }

({\akai 念})酒逢知己千杯少,语不投机半句多。

适才进得相府,与那贼深施一礼,他坐在上面,昂然不动,倒还罢了哇({\akai 或}:~还则罢了),反道我的礼貌不周。明日大宴群臣,将我用为鼓吏。分明是羞辱于我哇。我不免明日当着满朝文武,将贼({\akai 或}:~将他)辱骂一回。纵然将我斩首,也落得个青史名标!正是:~

({\akai 念})明知山有虎,偏向虎山行。

\setlength{\hangindent}{56pt}{【{\akai 西皮快板}】昔日里韩信受胯下,英雄落魄({\akai 或}:~落魄英雄)走天涯。到后来登台把帅挂,扶保汉室锦邦家。到明天进帐把贼骂,拚着一命染黄沙。纵然将我的头割下,落一个骂贼的名儿扬天涯。 }

\vspace{3pt}{\centerline{{[}{\hei 第四场}{]}}}\vspace{5pt}

来也!

({\akai 内})【{\akai 西皮导板}】谗臣当道谋汉朝,

\setlength{\hangindent}{56pt}{【{\akai 西皮原板}】楚汉相争动枪刀。汉王爷咸阳登大宝,一统山河乐唐尧。到如今出了个奸曹操,上欺天子下压群僚。我有心替主爷把贼讨\footnote{``中国京剧戏考''网站《戏考》第一册本作``把贼扫''。},掌中缺少杀人的刀。陪席坐定 }

【{\footnotesize 转}{\akai 西皮快板}】奸曹操,左右文武众群僚。元旦节与贼个不祥兆,假装疯魔骂奸曹。我把蓝衫\footnote{一般俗作``褴衫''。李楠{\scriptsize 君}按:~``蓝衫''是职位低下的官吏的职服,考诸剧情,祢衡先着蓝衫觐见曹操,继而换破衣褴衫,最后赤身裸体,当是。}来脱掉,

\setlength{\hangindent}{56pt}{【{\akai 西皮原板}】破衣褴衫摆摆摇。怒气不息登甬道,帐下的儿郎闹吵吵。 }

\setlength{\hangindent}{56pt}{【{\akai 西皮快板}】你二人休得呵呵笑,有辈古人听根苗:~昔日太公曾垂钓,张良进履在圯桥。为人受得苦中苦,脱去蓝衫换紫袍。 }

呸!

\setlength{\hangindent}{56pt}{【{\akai 西皮快板}】你二人把话讲差了,休把虎子当狸猫。有朝一日时运到,拔剑要斩海底鳌。 }

\setlength{\hangindent}{56pt}{【{\akai 西皮快板}】休道我白日梦颠倒,顷刻就要上青霄。我把破衣也脱掉, }

\setlength{\hangindent}{56pt}{【{\akai 西皮快板}】赤身露体逞英豪。耀武扬威往上跑, }

\setlength{\hangindent}{56pt}{【{\akai 西皮快板}】你丞相降罪有我承招。 }

\setlength{\hangindent}{56pt}{【{\akai 西皮快板}】将身来在西廊道,\footnote{吴焕老师整理的剧本记作``将身来在东廊道'',并注:~``刘老云,老本旧词此句唱`将身来在西廊道',下面所接锣鼓为\textless{}\!{\bfseries\akai 快长锤}\!\textgreater{},而并非\textless{}\!{\bfseries\akai 双楗子}\!\textgreater{}''。} }

\setlength{\hangindent}{56pt}{【{\akai 西皮散板}】看奸贼他把我怎开销。 }

曹操。

你叫得我祢衡,我就叫得你曹操!

我露父母之遗躰\footnote{旧谓子女的身体为父母所生,因称子女的身体为父母的``遗躰''。《大戴礼记·曾子大孝》:~``身者,亲之遗躰也。''一本作``遗体''。},方显我是清洁的君子。

你就是混浊的小人!

听道:~你不识贤愚,眼浊也;不纳忠言,耳浊也;不读诗书,口浊也;({\akai 或}:~不读诗书,口浊也;不纳忠言,耳浊也;)常怀篡逆,乃是心浊也!

我乃天下名士,将我用为鼓吏,犹如臧仓毁孟子,阳货轻仲尼。曹操啊,奸贼!(你)真乃匹夫之辈也!

\setlength{\hangindent}{56pt}{【{\akai 西皮快板}】开言怒发三千丈,大骂曹操听比方:~昔日文王访姜尚,亲临渭水请栋梁。臣坐君辇联辔往,为国求贤理所当。我本是堂堂奇男子,把我当作小儿郎。枉在朝中为首相,狗奸贼不识臭和香({\akai 或}:~不知臭和香)。 }

\setlength{\hangindent}{56pt}{【{\akai 西皮散板}】曹操把话错来讲,无水怎把蛟龙藏。 }

\setlength{\hangindent}{56pt}{【{\akai 西皮散板}】鼓打一通天地响,}

\setlength{\hangindent}{56pt}{【{\akai 西皮散板}】鼓打二通震朝纲。}

\setlength{\hangindent}{56pt}{【{\akai 西皮散板}】鼓打三通灭奸党,}

\setlength{\hangindent}{56pt}{【{\akai 西皮散板}】鼓打四通国安康。}

\setlength{\hangindent}{56pt}{【{\akai 西皮散板}】鼓伐一阵连声响,}

\setlength{\hangindent}{56pt}{【{\akai 西皮散板}】管教你狗奸贼死无下场。}

列公啊。

\setlength{\hangindent}{56pt}{【{\akai 西皮二六}】未曾开言我的心头恨,尊一声列公大人听详情:~家住在平原孝义村,姓祢名衡字表正平。我胸中颇有安邦论,曾与孔融当过了幕宾。他将我荐与曹奸佞,贼有眼不识宝和珍。我宁做那忠良门下客,不愿做奸贼帐下的人。 }

\setlength{\hangindent}{56pt}{【{\akai 西皮快板}】贼道我正平舌辩徒,舌辩之徒有张、苏。苏秦六国为相首,全凭舌辩压诸侯。有朝大展昆仑手\footnote{此句李楠{\scriptsize 君}从刘曾复先生学作``有朝大展经纶手''。},要把奸贼一笔勾。 }

\setlength{\hangindent}{56pt}{【{\akai 西皮快板}】贼那里道我井底蛙,井底下蛙也不差。有朝一日风云驾,要把奸贼一把拿({\akai 或}:~一把抓)。 }

\setlength{\hangindent}{56pt}{【{\akai 西皮散板}】狗奸贼他那里故意问道,尊一声列公卿细听根苗:~自幼儿举孝廉官职卑小,他本是夏侯子过继姓曹。到如今做高官忘了祖考\footnote{祖考,泛指父祖之辈。}({\akai 或}:~忘了宗祧\footnote{``祧''为远祖之庙。宗祧即宗庙,引申为祖业。}),全不怕骂名儿万载笑嘲。 }

量尔也不敢呐。({\akai 或}:~哼,我量你也不敢呐。)

住了! ({\akai 或}:~呀呸!)

\setlength{\hangindent}{56pt}{【{\akai 西皮散板}】要往荆州不能够,岂与奸贼作马牛。 }

(哦。)

\setlength{\hangindent}{56pt}{【{\akai 西皮二六}】列公大人齐来劝我,犹如方醒({\akai 或}:~犹如推醒)梦南柯。自古道未曾责人先要责己过,手摸胸膛自揣摩。罢罢罢暂息我的心头火, }

\setlength{\hangindent}{56pt}{【{\akai 西皮快板}】学一个陆贾与随何。丞相有事交与我,顺说刘表做定夺。 }

\setlength{\hangindent}{56pt}{【{\akai 西皮摇板}】丞相宽心安闲坐,披星戴月奔江河({\akai 或}:~渡江河)。顺说事儿若不妥, }

\setlength{\hangindent}{56pt}{【{\akai 西皮散板}】愿死他乡做鬼魔。\hspace{10pt}~ }

}
