\newpage
\subsubsection{\large\hei {回荆州~{\small 之}~鲁肃}}
\addcontentsline{toc}{subsection}{\hei 回荆州~{\small 之}~鲁肃}

\hangafter=1                   %2. 设置从第1⾏之后开始悬挂缩进  %}
\setlength{\parindent}{0pt}{

{(众将\hspace{30pt}得令!~)}

{慢------慢,慢(,慢)$\cdots{}\cdots{}$慢着!}

\setlength{\hangindent}{56pt}{【{\akai 西皮散板}】美人计成画饼早已料就,到此时切莫要另结冤仇哇。鲁子敬怎能够旁观袖手啊,劝都督三思行再定良谋({\akai 或}:~劝都督再思行另定良谋)}\footnote{段公平{\scriptsize 君}注:~这两句刘曾复先生另有作:~{鲁子敬再不能旁观袖手,望都督三思行另定良谋。}}{。}

{(哎呀)都督哇!(想)那郡主(随)同刘备回转荆州,乃是正理。你为何要将她赶回({\akai 或}:~你为何拦阻),是何意也?}

{哎呀,使不得,使不得呀。}

{那刘备乃是东吴的娇客呀。}

{(唉,难为你献那美人之计,诓哄刘备过江招亲,谁想以假成真。)}

{太后在甘露寺中面相刘备({\akai 或}:~太后做主),将郡主招赘刘备,(呃,那)岂不是东吴的娇客吗?}

{呃,不难,不难呐,}

{都督,再备美人,连那张飞也诓了前来!({\akai 或}:~只要都督,再备美人,漫说是那刘备,就是那张飞,嘿,他也是要来的呀。)}

{太后未必依你。}

{荆州兴兵?}

{哎呀,我怕呀------}

{(唉,)那诸葛亮的诡计,实在地厉害呀!({\akai 或}:~是厉害得很呐!)}

{(啊)都督,难道你就忘怀了?}

{(周瑜\hspace{30pt}忘怀了什么?)}

{(想)当年赤壁鏖兵,他在(那)南屏山上祭借东风,都督派了丁奉、徐盛,刺杀于他,尚且被他逃走({\akai 或}:~他尚且逃走)。}

{(都督又派他二人驾舟追赶,又被赵云箭射篷索而回。)}

{(哼,那时)几乎将你气死啊。}

{哼,难道这不是孔明的诡计么?({\akai 或}:~难道这不是孔明的诡计吗?)}

{啊都督,(你)不要生气呀。}

{这生气的日子还在后头呢。({\akai 或}:~那生气的日子还在后头呢。)}

{呃都督$\cdots{}\cdots{}$}

{(周瑜\hspace{30pt}不要管我的闲事。)}

{啊呀,这可不是闲事啊。}

{国家大事,唉,我不能不管呐。({\akai 或}:~军国大事,我是不能不问呐。)}

{呃呃呃,少弟,少弟。({\akai 或}:~呃呃,不敢不敢。)}

{不敢,不敢。({\akai 或}:~少弟。)}

{我本是个老实人,老实人才说这老实话呀!({\akai 或}:~唉,老实人才讲这老实话呀!)}

{哦哦,我,我,我吃醉了?({\akai 或}:~呃呃,呃,呃,呃呃,我,我$\cdots{}\cdots{}$我醉了。呃呃,我醉了?)}

{呃,我不曾吃酒,怎么我醉了?}

{诶------}

{({\akai 念})此时不听我言语,损兵折将后悔迟!}
}

\newpage
{\bfseries\hwfs 附:~}
\vspace{20pt}
\subsubsection{\hei {回荆州~{\small 之}~刘备}~\protect\footnote{段公平{\scriptsize 君}据{\textrm{2005}年{\textrm 5}月{\textrm 29}日刘曾复先生与段公平、樊百乐谈话录音整理(非正式说戏录音)}}}
\hangafter=1                   %2. 设置从第1⾏之后开始悬挂缩进  %}
\setlength{\parindent}{0pt}{

{\centerline{{[}{\hei 第一场}{]}}}\vspace{5pt}

\setlength{\hangindent}{56pt}{【{\akai 西皮原板}】深宫无处不飞花,年老得配女娇娃。朝欢暮乐无牵挂,愿把东吴当故家。}

\setlength{\hangindent}{56pt}{【{\akai 西皮散板}】听说曹操发人马,攻破荆州把孤拿。四弟之言并非假,想一良谋$\cdots{}\cdots{}$}\footnote{末句刘曾复先生未能忆起。陈超老师介绍:~他随刘曾复先生学的词句是``曹贼兴动人和马,进犯荆州把孤拿。四弟之言非虚假,瞒哄郡主及早还家。''尤其最后一句是八个字,挺特别。}

{郡主,备要逃走了。}

\setlength{\hangindent}{56pt}{【{\akai 西皮散板}】本当在此多潇洒,失却荆州无有家。见郡主难说分别\textless{}\!{\bfseries\akai 哭头}\!\textgreater{}话,}

\setlength{\hangindent}{56pt}{【{\akai 西皮散板}】花言巧语瞒哄她。}

\setlength{\hangindent}{56pt}{【{\akai 西皮散板}】根深哪怕狂风大,树正何惧日影斜。}

\vspace{3pt}{\centerline{{[}{\hei 第二场}{]}}}\vspace{5pt}

{(刘备{\hwfs 穿箭衣上})}

\setlength{\hangindent}{56pt}{【{\akai 西皮散板}】郡主进宫辞太后,为何一去不回头。四弟且站宫门口,准备鳌鱼脱金钩。}

{认得也!}

\setlength{\hangindent}{56pt}{【{\akai 西皮散板}】多蒙太后恩德厚,此去只怕孙仲谋。}

\setlength{\hangindent}{56pt}{【{\akai 西皮散板}】四弟与孤带走兽,}

{(赵云{\akai 接唱}{\hwfs 收腿},刘备{\hwfs 下})}

\vspace{3pt}{\centerline{{[}{\hei 第三场}{]}}}\vspace{5pt}

{(刘备{\hwfs 上})}

\setlength{\hangindent}{56pt}{【{\akai 西皮导板}】\textcolor{red}{身躬步臃}路途远,}

\setlength{\hangindent}{56pt}{【{\akai 西皮散板}】$\cdots{}\cdots{}$往前赶,怕的吴兵追赶还。}

\setlength{\hangindent}{56pt}{【{\akai 西皮散板}】你姑老爷要走呃你们谁敢拦。}

{(刘备{\hwfs 下})}

{\vspace{3pt}{\centerline{{[}{\hei 第四场}{]}}}\vspace{5pt}

{(刘备{\hwfs 上})}

\setlength{\hangindent}{56pt}{【{\akai 西皮散板}】急急赶来如风涌,插翅难飞到九重({\akai 或}:~上九重)。}\footnote{刘曾复先生说戏时说明:~此处{后来多不唱,刘备直接\textless{}\!{\bfseries\akai 水底鱼}\!\textgreater{}{\hwfs 上}。}}

{四弟,前有大江,后有追兵,如何是好?}

{(赵云\hspace{30pt}待我望来。)}

{$\cdots{}\cdots{}$}

搭了扶手。

{(诸葛亮\hspace{20pt}诸葛亮接驾。)}

{$\cdots{}\cdots{}$}
}
