\newpage
\subsubsection{\large\hei {取荥阳}}
\addcontentsline{toc}{subsection}{\hei 取荥阳}

\hangafter=1                   %2. 设置从第1⾏之后开始悬挂缩进  %
\setlength{\parindent}{0pt}{
{\centerline{{[}{\hei 第一场}{]}}}\vspace{5pt}

\setlength{\hangindent}{56pt}{钟离眜\footnote{《京剧汇编》第七集~马连良~藏本作``钟离昧'',钟离眜多误为``钟离昧''或``钟离眛''。}\hspace{12pt} ({\akai 念})并吞六国秦始皇, }

\setlength{\hangindent}{56pt}{季布\hspace{30pt}({\akai 念})修造\footnote{《京剧汇编》第七集 马连良~藏本作``修筑''。}长城建阿房。 }

\setlength{\hangindent}{56pt}{钟离眜\hspace{20pt}({\akai 念})义帝有道江山掌, }

\setlength{\hangindent}{56pt}{季布\hspace{30pt}({\akai 念})楚汉分兵定咸阳。 }

\setlength{\hangindent}{56pt}{钟离眜\hspace{20pt}俺,钟离眜, }

\setlength{\hangindent}{56pt}{季布\hspace{30pt}季布。 }

\setlength{\hangindent}{56pt}{钟离眜\hspace{20pt}请了。 }

\setlength{\hangindent}{56pt}{季布\hspace{30pt}请了。 }

\setlength{\hangindent}{56pt}{钟离眜\hspace{20pt}大王发兵,夺取荥阳,两厢伺候!  }

\setlength{\hangindent}{56pt}{季布\hspace{30pt}请!  }

\setlength{\hangindent}{56pt}{项羽\hspace{30pt}\textless{}\!{\bfseries\akai 点绛唇}\!\textgreater{}浩荡旌旗,战鼓擂齐,军雄威;要战如雷,催马龙虎退。 }

\setlength{\hangindent}{56pt}{项羽\hspace{30pt}({\akai 念})二将壮军威,曾交战如雷\footnote{李元皓{\scriptsize 君}指此句可能是``阵脚站如垒''即古代作战讲究的阵脚不能动,此处定场诗是项羽阅兵,头句言将军善於练兵,二句言军威之壮,三句言军容之整,,末句概写。}。似雪倒缨盔,照耀映光辉。 }

\setlength{\hangindent}{56pt}{项羽\hspace{30pt}孤,西楚项羽,自江淮起义以来,百战百胜。昨日探马报道:~今有韩信领兵燕赵去了,荥阳必定空虚,为此带领人马夺取荥阳,以令天下。 }

\setlength{\hangindent}{56pt}{项羽\hspace{30pt}来,传钟、季二将进帐。 }

\setlength{\hangindent}{56pt}{众\hspace{40pt}二将进帐。 }

钟离眜\\
季布\hspace{30pt}\raisebox{5pt}{来也!  }

\setlength{\hangindent}{56pt}{钟离眜\hspace{20pt}({\akai 念})万马军中气概雄, }

\setlength{\hangindent}{56pt}{季布\hspace{30pt}({\akai 念})人似飞虎马似龙。 }

钟离眜\\季布\hspace{30pt}\raisebox{5pt}{参见大王。 }

\setlength{\hangindent}{56pt}{项羽\hspace{30pt}免。 }

\setlength{\hangindent}{56pt}{项羽\hspace{30pt}二位将军,人马可齐?  }

钟离眜\\季布\hspace{30pt}\raisebox{5pt}{俱已齐备。 }

\setlength{\hangindent}{56pt}{项羽\hspace{30pt}兵发荥阳。}

钟离眜\\季布\hspace{30pt}\raisebox{5pt}{得令!众将官,兵发荥阳!}

\setlength{\hangindent}{56pt}{众\hspace{40pt}啊!}

\vspace{3pt}{\centerline{{[}{\hei 第二场}{]}}}\vspace{5pt}

\setlength{\hangindent}{56pt}{张良\hspace{30pt}({\akai 念})口似悬河语似流,}

\setlength{\hangindent}{56pt}{陈平\hspace{30pt}({\akai 念})舌似钢锋运机谋。}

\setlength{\hangindent}{56pt}{张良\hspace{30pt}({\akai 念})楚汉交兵何日息,}

\setlength{\hangindent}{56pt}{陈平\hspace{30pt}({\akai 念})灭却重瞳方罢休。}

\setlength{\hangindent}{56pt}{张良\hspace{30pt}山人张良,}

\setlength{\hangindent}{56pt}{陈平\hspace{30pt}下官陈平。}

\setlength{\hangindent}{56pt}{张良\hspace{30pt}请了,主公升帐,两厢伺候。}

\setlength{\hangindent}{56pt}{陈平\hspace{30pt}请。}

\setlength{\hangindent}{56pt}{刘邦\hspace{30pt}{[}{\akai {\akai 引}子}{]}保义安天命,但愿得,早定乾坤。}

张良\\陈平\hspace{30pt}\raisebox{5pt}{臣等见驾,主公千岁。}

\setlength{\hangindent}{56pt}{刘邦\hspace{30pt}平身。}

张良\\陈平\hspace{30pt}\raisebox{5pt}{千千岁。}

\setlength{\hangindent}{56pt}{刘邦\hspace{30pt}赐坐。}

张良\\陈平\hspace{30pt}\raisebox{5pt}{谢座。}

\setlength{\hangindent}{56pt}{刘邦\hspace{30pt}({\akai 念})提剑斩蛇聚英雄,干戈一起定关中。重瞳不遵怀王约,强霸虎踞冒吾功。}

\setlength{\hangindent}{56pt}{刘邦\hspace{30pt}孤刘邦,字季子。自沛丰起义以来,扫灭嬴秦,先定关中。可恨项羽不遵怀王之约,强霸为王,将孤贬封蜀中。幸得萧何、子房等,保荐韩信,登台拜帅,暗渡陈仓,复夺三秦。大兵已扎荥阳,暂为犄角之势。前者霸王被韩信车战,败回彭城。如今韩信兵伐燕赵去了,彭越去往东京,九江王染病在床。又恐霸王乘虚来夺荥阳,也曾命人前去打探,未见回报。}

\setlength{\hangindent}{56pt}{大太监\hspace{20pt}报!启大王:~今有霸王前来夺取荥阳,请大王敌楼答话。}

\setlength{\hangindent}{56pt}{刘邦\hspace{30pt}再探。}

\setlength{\hangindent}{56pt}{大太监\hspace{20pt}啊!}

\setlength{\hangindent}{56pt}{刘邦\hspace{30pt}二位先生:~霸王兵困荥阳,教孤敌楼答话,孤去也不去?}

\setlength{\hangindent}{56pt}{张良\hspace{30pt}此乃范增之谋,知韩信兵发燕赵去了,趁此虚空,来取荥阳。臣等保定主公,敌楼观看动静,回来再作道理。}

\setlength{\hangindent}{56pt}{刘邦\hspace{30pt}言之有理。带马------}

\setlength{\hangindent}{56pt}{刘邦\hspace{30pt}【{\akai 西皮摇板}】荥阳城外摆战场,将士纷纷马蹄忙。韩信领兵燕赵往,无有能将敌霸王。君臣且把敌楼上,}

\setlength{\hangindent}{56pt}{刘邦\hspace{30pt}【{\akai 西皮摇板}】旌旗不住空中扬。重重叠叠兵和将,}

\setlength{\hangindent}{56pt}{刘邦\hspace{30pt}哎呀!}

\setlength{\hangindent}{56pt}{刘邦\hspace{30pt}【{\akai 西皮摇板}】刀枪剑戟似秋霜。孤王一见心胆丧({\akai 或}:~魂胆丧),只恐难以保荥阳({\akai 或}:~难保这荥阳)。}

\setlength{\hangindent}{56pt}{项羽\hspace{30pt}({\akai 内})【{\akai 西皮导板}】忆昔当年渡淮江,}

\setlength{\hangindent}{56pt}{项羽\hspace{30pt}【{\akai 西皮原板}】百战百胜威名扬。多少能将枪尖丧,能征惯战鞭下亡。坐至在雕鞍用目望,城楼站的小刘邦。左有陈平狗奸党,右边站立张子房。勒住马头【{\footnotesize 转}{\akai 西皮快板}】把话讲,开言大骂小刘邦:~你本是沛县一亭长,你敢与孤夺家邦。快快开城来打仗,看看谁弱哪家强。}

\setlength{\hangindent}{56pt}{刘邦\hspace{30pt}【{\akai 西皮二六}】站立在敌楼把话讲:~开言尊声楚霸王。休提起沛县一亭长,提起了当年孤就怒满胸膛。同扶怀王把业创({\akai 或}:~基业创),楚汉分兵进咸阳({\akai 或}:~定咸阳)。先到咸阳为皇上,后到咸阳辅保朝堂。你不遵王约太狂妄,反将({\akai 或}:~反把)孤刘邦赶出了咸阳。若不是韩信他的韬略广,孤的人马也不能暗渡陈仓。复夺三秦军威壮,岂不知强中自有强中强?}

\setlength{\hangindent}{56pt}{项羽\hspace{30pt}【{\akai 西皮快板}】闻言怒发三千丈,开口大骂小刘邦。说什么同把江山创,楚汉合兵定咸阳。孤王出兵谁敢挡,纵横天下楚霸王。鸿门宴将你放,放虎归山反逞强。你道韩信韬略广,雪霜焉能见太阳。}

\setlength{\hangindent}{56pt}{刘邦\hspace{30pt}【{\akai 西皮快板}】霸王不必({\akai 或}:~休得)夸口讲,肉眼不识紫金梁。自从起义在沛上,分茅裂土古之常。你不该强把诸侯抗({\akai 或}:~挡),你不该举火焚阿房。你不该斩杀楚降将({\akai 或}:~归降将),你不该逼死楚怀王。明知韩信燕赵往,乘虚兴兵取荥阳。任你兵多将又广,休教刘邦({\akai 或}:~休劝孤王)来归降,(你)枉费心肠。}

\setlength{\hangindent}{56pt}{项羽\hspace{30pt}【{\akai 西皮快板}】你父太公在罗网,吕氏夫人笼中藏。你若不肯来归降,太公、吕后刀下亡。父子不能同欢畅,结发夫妻两分张。手摸胸膛想一想,看你归降不归降。}

\setlength{\hangindent}{56pt}{刘邦\hspace{30pt}【{\akai 西皮导板}】刘邦闻言心欢畅,}

\setlength{\hangindent}{56pt}{刘邦\hspace{30pt}呵呵呵哈哈哈$\cdots{}\cdots{}$({\hwfs 笑}{\hwfs 介})}

\setlength{\hangindent}{56pt}{刘邦\hspace{30pt}【{\akai 西皮快板}】开言尊声楚霸王。说什么({\akai 或}:~讲什么)吾父太公入罗网,(讲什么)吕氏妻子笼中藏。当初结义情谊长({\akai 或}:~义气长),犹如同父共同娘。老父、妻室蒙君养,生死存亡你主张。你父我父俱一样,忍心杀害也无妨。劝你早退兵和将,韩信到来你又着忙。}

\setlength{\hangindent}{56pt}{项羽\hspace{30pt}【{\akai 西皮摇板}】匹夫出言太无状,你拿韩信压孤王。回头再叫二员将,孤王言来听端详:~快将荥阳齐围上,休要放走小刘邦。}

\setlength{\hangindent}{56pt}{刘邦\hspace{30pt}不好了!}

\setlength{\hangindent}{56pt}{刘邦\hspace{30pt}【{\akai 西皮摇板}】一言激怒楚霸王,犹如倒海似翻江。君臣且归黄罗帐,}

\setlength{\hangindent}{56pt}{刘邦\hspace{30pt}【{\akai 西皮摇板}】再与二卿作商量。}

\setlength{\hangindent}{56pt}{刘邦\hspace{30pt}二位先生,霸王攻打荥阳甚急,有何良策?}

\setlength{\hangindent}{56pt}{张良\hspace{30pt}大王修书一封,去往燕赵,调韩信回来,以挡霸王之勇。}

\setlength{\hangindent}{56pt}{刘邦\hspace{30pt}待孤修书。}

\setlength{\hangindent}{56pt}{陈平\hspace{30pt}且慢。想荥阳与燕赵相隔甚远,一时焉能得到?荥阳乃弹丸之地,倘有人献计,将荥河之水从上而下冲灌前来,荥阳化为齑粉矣!}

\setlength{\hangindent}{56pt}{刘邦\hspace{30pt}不好了!}

\setlength{\hangindent}{56pt}{刘邦\hspace{30pt}【{\akai 西皮摇板}】层层围困小荥阳,倒教孤王无主张。插翅不能出罗网,有何良策保孤王?}

\setlength{\hangindent}{56pt}{陈平\hspace{30pt}【{\akai 西皮摇板}】君忧臣愁心惆怅,倒教陈平少主张。回头便对先生讲,}

\setlength{\hangindent}{56pt}{陈平\hspace{30pt}先生,}

\setlength{\hangindent}{56pt}{陈平\hspace{30pt}【{\akai 西皮摇板}】有何妙计救君王。}

\setlength{\hangindent}{56pt}{张良\hspace{30pt}【{\akai 西皮快板}】一片丹心扶汉王,全凭韬略定封疆。汉王闷坐\footnote{《京剧汇编》第七集~马连良~藏本作``稳坐''。}黄罗帐,这一旁难坏张子房。我也曾背剑把韩信访,我也曾赚楚定咸阳。三分天下有二项,一时无计救汉王。低下头来暗思想,}

\setlength{\hangindent}{56pt}{张良\hspace{30pt}有了!}

\setlength{\hangindent}{56pt}{张良\hspace{30pt}【{\akai 西皮快板}】猛然一计在心旁。回头我对大王讲,君臣宽怀慢商量。}

\setlength{\hangindent}{56pt}{张良\hspace{30pt}启主公:~臣有一计可保我主出得荥阳。}

\setlength{\hangindent}{56pt}{刘邦\hspace{30pt}有何妙计?}

\setlength{\hangindent}{56pt}{张良\hspace{30pt}臣思得东周列国,一段忠义故事。臣回至馆驿。约请文武,倘有忠义之臣,替主赴难,也未可知。}

\setlength{\hangindent}{56pt}{刘邦\hspace{30pt}但凭先生。}

\setlength{\hangindent}{56pt}{张良\hspace{30pt}领旨。}

\setlength{\hangindent}{56pt}{张良\hspace{30pt}【{\akai 西皮摇板}】主公但把宽心放,圣主驾前有栋梁。霸王纵有天罗网,管保困龙上天堂。}

\setlength{\hangindent}{56pt}{陈平\hspace{30pt}【{\akai 西皮摇板}】躬身施礼出宝帐,但愿我主离荥阳。}

\setlength{\hangindent}{56pt}{刘邦\hspace{30pt}【{\akai 西皮摇板}】张良、陈平出宝帐,再叫帐前众儿郎:~免战牌高挂敌楼上,滚木擂石要谨防({\akai 或}:~要提防)。}

\vspace{3pt}{\centerline{{[}{\hei 第三场}{]}}}\vspace{5pt}

\setlength{\hangindent}{56pt}{张良\hspace{30pt}【{\akai 西皮摇板}】适才离了黄罗帐,回到馆驿想良方。}

\setlength{\hangindent}{56pt}{张良\hspace{30pt}山人张良,今有霸王攻打荥阳甚急,山人思得一计,以解此危\footnote{夏行涛{\scriptsize 君}建议作``以解此围'',此处从《京剧汇编》第七集~马连良~藏本。}。来。}

\setlength{\hangindent}{56pt}{童儿\hspace{30pt}有。}

\setlength{\hangindent}{56pt}{张良\hspace{30pt}拿我名帖邀集文武,明日齐至馆驿饮宴,共议军机。}

\setlength{\hangindent}{56pt}{童儿\hspace{30pt}是。}

\setlength{\hangindent}{56pt}{张良\hspace{30pt}正是:~({\akai 念})设计救君难,开筵画图悬。}

\vspace{3pt}{\centerline{{[}{\hei 第四场}{]}}}\vspace{5pt}

\setlength{\hangindent}{56pt}{曹参\hspace{30pt}下官曹参。}

\setlength{\hangindent}{56pt}{周勃\hspace{30pt}下官周勃。}

\setlength{\hangindent}{56pt}{纪信\hspace{30pt}下官纪信。}

\setlength{\hangindent}{56pt}{随何\hspace{30pt}下官随何。}

\setlength{\hangindent}{56pt}{曹参\hspace{30pt}列位大人请了。}

\setlength{\hangindent}{56pt}{众\hspace{40pt}请了。}

\setlength{\hangindent}{56pt}{曹参\hspace{30pt}军师有帖相邀,不知为了何事?}

\setlength{\hangindent}{56pt}{众\hspace{40pt}齐至馆驿,便知明白。}

\setlength{\hangindent}{56pt}{众\hspace{40pt}来,}

\setlength{\hangindent}{56pt}{众军士\hspace{20pt}有!}

\setlength{\hangindent}{56pt}{众\hspace{40pt}打道馆驿!}

\setlength{\hangindent}{56pt}{众军士\hspace{20pt}啊!}

\setlength{\hangindent}{56pt}{随从甲\hspace{20pt}({\akai 念})奉了军师命,堂上挂图形。}

\setlength{\hangindent}{56pt}{随从乙\hspace{20pt}奉了军师之命:~打扫中堂,悬挂图形。}

\setlength{\hangindent}{56pt}{随从甲\hspace{20pt}就此悬挂起来!}

\setlength{\hangindent}{56pt}{随从乙\hspace{20pt}悬挂已毕!}

\setlength{\hangindent}{56pt}{张良\hspace{30pt}门外伺候!}

\setlength{\hangindent}{56pt}{众军士\hspace{20pt}(来至)馆驿!\footnote{根据《京剧汇编》第七集~马连良~藏本补齐。}}

\setlength{\hangindent}{56pt}{众\hspace{40pt}回避了!}

\setlength{\hangindent}{56pt}{曹参\hspace{30pt}门上哪位在?}

\setlength{\hangindent}{56pt}{童儿\hspace{30pt}什么人?}

\setlength{\hangindent}{56pt}{曹参\hspace{30pt}我等到齐。}

\setlength{\hangindent}{56pt}{童儿\hspace{30pt}众位大人稍候,有请军师。}

\setlength{\hangindent}{56pt}{张良\hspace{30pt}何事?}

\setlength{\hangindent}{56pt}{童儿\hspace{30pt}列位大人到。}

\setlength{\hangindent}{56pt}{张良\hspace{30pt}有请。}

\setlength{\hangindent}{56pt}{童儿\hspace{30pt}有请。}

\setlength{\hangindent}{56pt}{众\hspace{40pt}请------}

\setlength{\hangindent}{56pt}{众\hspace{40pt}军师请上,受我等参拜。}

\setlength{\hangindent}{56pt}{张良\hspace{30pt}山人也有一拜。}

\setlength{\hangindent}{56pt}{张良\hspace{30pt}请坐------}

\setlength{\hangindent}{56pt}{众\hspace{40pt}告坐。}

\setlength{\hangindent}{56pt}{张良\hspace{30pt}不知列位驾到,有失远迎,望乞恕罪。}

\setlength{\hangindent}{56pt}{众\hspace{40pt}我等来得鲁莽,军师海涵。}

\setlength{\hangindent}{56pt}{张良\hspace{30pt}岂敢。}

\setlength{\hangindent}{56pt}{众\hspace{40pt}军师相邀,有何见谕?}

\setlength{\hangindent}{56pt}{张良\hspace{30pt}君忧臣愁,古之常理。列公乃忠义之士,我主被困荥阳,列公有何高见,解君之忧?}

\setlength{\hangindent}{56pt}{众\hspace{40pt}我等才疏学浅,全仗军师。}

\setlength{\hangindent}{56pt}{童儿\hspace{30pt}宴齐。}

\setlength{\hangindent}{56pt}{张良\hspace{30pt}看宴。}

\setlength{\hangindent}{56pt}{张良\hspace{30pt}待我把盏。}

\setlength{\hangindent}{56pt}{众\hspace{40pt}不敢,摆下就是。}

\setlength{\hangindent}{56pt}{张良\hspace{30pt}众位大人请------}

\setlength{\hangindent}{56pt}{张良\hspace{30pt}大人请------}

\setlength{\hangindent}{56pt}{众\hspace{40pt}军师请!}

\setlength{\hangindent}{56pt}{众\hspace{40pt}请问军师:~上面挂的画图,是哪朝故事?}

\setlength{\hangindent}{56pt}{张良\hspace{30pt}此乃东周列国齐晋交兵,一段忠义故事,大有可谓\footnote{《京剧汇编》第七集~马连良~藏本``可谓''作``可为''。}。}

\setlength{\hangindent}{56pt}{众\hspace{40pt}军师请道其详。}

\setlength{\hangindent}{56pt}{张良\hspace{30pt}少时饮罢再叙。}

\setlength{\hangindent}{56pt}{众\hspace{40pt}我等酒已够了。}

\setlength{\hangindent}{56pt}{张良\hspace{30pt}将宴撤去。}

\setlength{\hangindent}{56pt}{众\hspace{40pt}军师请道其详。}

\setlength{\hangindent}{56pt}{张良\hspace{30pt}此乃齐晋交兵,战于曲阳,晋兵强盛\footnote{《京剧汇编》第七集~马连良~藏本``强盛''均作``强胜''。},齐军大败。}

\setlength{\hangindent}{56pt}{众\hspace{40pt}车内敢莫就是齐顷公?}

\setlength{\hangindent}{56pt}{张良\hspace{30pt}非也。此乃参军逄丑父\footnote{``逄''字音``庞(páng)'',因古字``逄''与``逢''通假,故``逄丑父''亦作``逢丑父''。旧时艺人误念``冯(féng)'',辗转因袭,几成定例。此处刘曾复先生遵从传统习惯,下同。}。}

\setlength{\hangindent}{56pt}{众\hspace{40pt}哦,逄丑父。}

\setlength{\hangindent}{56pt}{张良\hspace{30pt}见齐军大败,齐顷公吓得面如土色。丑父奏道:~事已危急,大王可将衣帽脱下,与臣穿戴,坐于车中,大王林中藏躲。顷公言道:~我虽然逃难\footnote{《京剧汇编》第七集~马连良~藏本``逃难''作``脱难''。},卿家必定遭擒,存亡难定,吾不忍也。丑父奏道:~食王之禄,当报君恩。臣一命好比大树林中落下一叶耳;若存大王,称为万姓之主,使天下受福不小也。}

\setlength{\hangindent}{56pt}{张良\hspace{30pt}【{\akai 西皮摇板}】人说光阴重黄金,我把光阴比浮云。舍身救主忠义尽,留得美名万古存。}

\setlength{\hangindent}{56pt}{张良\hspace{30pt}顷公见丑父忠心耿耿,遂将衣帽脱下与丑父穿戴,自己向林中藏躲呵。}

\setlength{\hangindent}{56pt}{众\hspace{40pt}那林中藏躲的就是齐顷公么?}

\setlength{\hangindent}{56pt}{张良\hspace{30pt}正是。}

\setlength{\hangindent}{56pt}{众\hspace{40pt}那逄丑父后来呢?}

\setlength{\hangindent}{56pt}{张良\hspace{30pt}晋军赶上,将丑父擒住,只道他是齐顷公,献与了晋侯,查其来历,知是丑父,彼时推出斩首,丑父大笑。晋公曰:~不避死而代君难,君得生,全其忠也;杀之不祥,当赦其罪。遂将丑父释放回国。顷公见其忠义,封以爵位。今汉王被困荥阳,我等空食君禄,竟无一人学那逄丑父之故事耳。}

\setlength{\hangindent}{56pt}{众\hspace{40pt}哦!}

\setlength{\hangindent}{56pt}{张良\hspace{30pt}【{\akai 西皮快板}】昔日丑父身代君,舍身救主不畏刑。若得一人效丑父,假扮汉王诓楚君。忠义凛凛鬼神敬,封妻荫子标美名。}

\setlength{\hangindent}{56pt}{曹参\hspace{30pt}【{\akai 西皮摇板}】军师悬图({\akai 或}:~挂图)表忠论,}

\setlength{\hangindent}{56pt}{周勃\hspace{30pt}【{\akai 西皮摇板}】方显我等不忠臣。}

\setlength{\hangindent}{56pt}{随何\hspace{30pt}【{\akai 西皮摇板}】堂堂君王遭围困,}

\setlength{\hangindent}{56pt}{纪信\hspace{30pt}呵!}

\setlength{\hangindent}{56pt}{纪信\hspace{30pt}【{\akai 西皮摇板}】愿学前辈古贤人。}

\setlength{\hangindent}{56pt}{众\hspace{40pt}师爷,自古道:~父有难子当代,君有难臣当替。我等愿效丑父舍身救主。}

\setlength{\hangindent}{56pt}{张良\hspace{30pt}公等愿舍生救主,真乃难得。但只一件$\cdots{}\cdots{}$}

\setlength{\hangindent}{56pt}{众\hspace{40pt}哪一件?}

\setlength{\hangindent}{56pt}{张良\hspace{30pt}必须与我主容颜相似,年貌相当,假扮汉王,方可救得吾主。}

\setlength{\hangindent}{56pt}{众\hspace{40pt}我等站齐,师爷请看。}

\setlength{\hangindent}{56pt}{张良\hspace{30pt}待山人看来。}

\setlength{\hangindent}{56pt}{张良\hspace{30pt}我观列位大人,各有不同;惟有纪将军与主龙颜相似,替主赴难,非纪将军不可。}

\setlength{\hangindent}{56pt}{纪信\hspace{30pt}哦!}

\setlength{\hangindent}{56pt}{纪信\hspace{30pt}【{\akai 西皮二六}】纪信闻言心不定({\akai 或}:~心不稳),背转身来自思忖。汉王荥阳遭围困,好似({\akai 或}:~犹如)孔子困至在蔡、陈。韩信领兵燕、赵境,无有能将退楚兵。师爷悬图({\akai 或}:~师爷挂图)【{\footnotesize 转}{\akai 西皮快板}】表忠论,逄丑父救主标美名。与主同貌是纪信,要学先贤贯古今。走向前来把话论,纪信替主无二心。}

\setlength{\hangindent}{56pt}{纪信\hspace{30pt}纪信情愿替主赴难。}

\setlength{\hangindent}{56pt}{张良\hspace{30pt}将军替主赴难,可解荥阳之危。但霸王性如烈火,此去存亡难定,犹恐将军难舍一死。}

\setlength{\hangindent}{56pt}{纪信\hspace{30pt}师爷说哪里话来?慢说替主之难,就是赴汤蹈火,万死不辞。}

\setlength{\hangindent}{56pt}{张良\hspace{30pt}将军可是实言?}

\setlength{\hangindent}{56pt}{纪信\hspace{30pt}焉有二意?!}

\setlength{\hangindent}{56pt}{张良\hspace{30pt}如此将军请上,受我一拜。}

\setlength{\hangindent}{56pt}{纪信\hspace{30pt}岂敢!}

\setlength{\hangindent}{56pt}{张良\hspace{30pt}【{\akai 西皮摇板}】忠心贯日纪将军,古往今来少见闻。但愿解得荥阳困,青史名标万古存。}

\setlength{\hangindent}{56pt}{纪信\hspace{30pt}师爷------}

\setlength{\hangindent}{56pt}{纪信\hspace{30pt}【{\akai 西皮快板}】说什么青史标名姓,臣报君恩子奉亲。孝当竭力忠尽命,贪生怕死岂忠臣。}

\setlength{\hangindent}{56pt}{众\hspace{40pt}【{\akai 西皮摇板}】好个仁义纪将军,为主一片忠义心。舍身救主不惜命,愧煞我等不忠臣。}

\setlength{\hangindent}{56pt}{纪信\hspace{30pt}【{\akai 西皮快板}】多蒙列位美言赠,大事全仗众将军。纪信替主解围困,功成保主({\akai 或}:~公等保主)要小心。我与霸王逞舌论,要学子牙骂纣君。纵然将我碎尸粉,留得美名万古存。}

\setlength{\hangindent}{56pt}{张良\hspace{30pt}【{\akai 西皮摇板}】家贫\footnote{《京剧汇编》第七集~马连良~藏本作``嘉臣''。}孝子令人敬,}

\setlength{\hangindent}{56pt}{张良\hspace{30pt}【{\akai 西皮快板}】国难方显忠良臣。大事全仗纪将军,明日早朝依计行。众位将军且归寝,}

\setlength{\hangindent}{56pt}{张良\hspace{30pt}【{\akai 西皮摇板}】机密大事莫漏真。}

\setlength{\hangindent}{56pt}{曹参\hspace{30pt}【{\akai 西皮摇板}】深谢将军金石论,}

\setlength{\hangindent}{56pt}{周勃\hspace{30pt}【{\akai 西皮摇板}】盖世英雄纪将军。}

\setlength{\hangindent}{56pt}{随何\hspace{30pt}【{\akai 西皮摇板}】臣替君难心情忍,}

\setlength{\hangindent}{56pt}{纪信\hspace{30pt}【{\akai 西皮摇板}】心怀忠义别先生。含悲忍泪跨金镫,}

\setlength{\hangindent}{56pt}{张良\hspace{30pt}将军请转------}

\setlength{\hangindent}{56pt}{纪信\hspace{30pt}【{\akai 西皮摇板}】师爷有话快些云。}

\setlength{\hangindent}{56pt}{张良\hspace{30pt}【{\akai 西皮摇板}】将军替主心拿稳,莫作三心二意人。倘若临时呼不应,功不就来名不成。}

\setlength{\hangindent}{56pt}{纪信\hspace{30pt}【{\akai 西皮摇板}】先生不必细叮咛,纪信岂是等闲人。忠心一片岂失信,要想回心万不能,你但放宽心。}

\setlength{\hangindent}{56pt}{张良\hspace{30pt}【{\akai 西皮摇板}】大忠大义是纪信,片言感动忠良臣。非是张良无德行\footnote{段公平{\scriptsize 君}建议作``德性''。},都只为创业兴邦的汉刘君,我不得不行。}

\setlength{\hangindent}{56pt}{{\vspace{3pt}{\centerline{{[}{\hei 第五场}{]}}}\vspace{5pt}}}

\setlength{\hangindent}{56pt}{刘邦\hspace{30pt}【{\akai 西皮摇板}】霸王日夜把城攻,}

\setlength{\hangindent}{56pt}{刘邦\hspace{30pt}【{\akai 西皮快板}】四面围困不透风。孤王心中担惊恐,}

\setlength{\hangindent}{56pt}{刘邦\hspace{30pt}【{\akai 西皮摇板}】无有良策破重瞳。}

\setlength{\hangindent}{56pt}{张良\hspace{30pt}【{\akai 西皮摇板}】一片丹心是忠信,}

\setlength{\hangindent}{56pt}{陈平\hspace{30pt}【{\akai 西皮摇板}】点破英雄众群臣。}

\setlength{\hangindent}{56pt}{张良\hspace{30pt}【{\akai 西皮摇板}】纪信可称真梁栋,}

\setlength{\hangindent}{56pt}{陈平\hspace{30pt}【{\akai 西皮摇板}】急忙进帐奏主公。}

张良\\陈平\hspace{30pt}\raisebox{5pt}{臣等见驾,大王千岁。}

\setlength{\hangindent}{56pt}{刘邦\hspace{30pt}平身。}

张良\\陈平\hspace{30pt}\raisebox{5pt}{千千岁!}

\setlength{\hangindent}{56pt}{刘邦\hspace{30pt}赐坐。}

张良\\陈平\hspace{30pt}\raisebox{5pt}{谢坐。}

\setlength{\hangindent}{56pt}{刘邦\hspace{30pt}可有忠义之臣,替孤赴难?}

\setlength{\hangindent}{56pt}{张良\hspace{30pt}今有纪信与大王容貌相似,情愿替主赴难。}

\setlength{\hangindent}{56pt}{刘邦\hspace{30pt}哦!如此宣纪信将军进帐。}

\setlength{\hangindent}{56pt}{张良\hspace{30pt}纪信将军进帐!}

\setlength{\hangindent}{56pt}{纪信\hspace{30pt}({\akai 内})来也!}

\setlength{\hangindent}{56pt}{纪信\hspace{30pt}【{\akai 西皮快板}】十年窗前习孔孟,几载又学箭和弓。不见封侯成何用,功劳出在画图中。}

\setlength{\hangindent}{56pt}{纪信\hspace{30pt}臣,纪信见驾,大王千岁!}

\setlength{\hangindent}{56pt}{刘邦\hspace{30pt}平身。}

\setlength{\hangindent}{56pt}{纪信\hspace{30pt}千千岁!}

\setlength{\hangindent}{56pt}{刘邦\hspace{30pt}赐坐。}

\setlength{\hangindent}{56pt}{纪信\hspace{30pt}谢坐。}

\setlength{\hangindent}{56pt}{刘邦\hspace{30pt}二位先生,纪将军果然与孤王面貌相似么?}

张良\\陈平\hspace{30pt}\raisebox{5pt}{与大王,相似无二。}

\setlength{\hangindent}{56pt}{刘邦\hspace{30pt}纪将军,方才二位先生言道,将军愿替孤王赴难,可是真情?}

\setlength{\hangindent}{56pt}{纪信\hspace{30pt}愿解荥阳之危。}

\setlength{\hangindent}{56pt}{刘邦\hspace{30pt}多蒙将军美意,不避刀锋之苦,此乃寡人福浅,累及卿等;只是将军如此忠义,教寡人怎能割舍?此事断然不可!}

\setlength{\hangindent}{56pt}{纪信\hspace{30pt}臣启大王:~臣替君难,理所当然,何言不可?}

\setlength{\hangindent}{56pt}{刘邦\hspace{30pt}寡人江山未定,众卿徒劳,未享一日之荣,反而连累将军,这样损人利己之事,孤心不忍也。}

\setlength{\hangindent}{56pt}{纪信\hspace{30pt}哎呀大王啊!今日事在危急,城池难保;倘若荥阳一破,玉石俱焚。那时臣死轻如鸿毛,今日臣死何惜,他日留得美名重如泰山。不要臣替君难,臣便拔剑自刎君前,以报主上之恩也。}

\setlength{\hangindent}{56pt}{纪信\hspace{30pt}【{\akai 西皮原板}】大王起义在沛丰,扫灭嬴秦定关中。招贤纳士恩义重,多少英雄反重瞳。唯愿我主成一统,岂知今日遭困中。舍身救主理当奉,打开\footnote{《京剧汇编》第七集~马连良~藏本作``放开''。}金锁走蛟龙。纪信一死成何用,保全大王数载功。}

\setlength{\hangindent}{56pt}{刘邦\hspace{30pt}将军!}

\setlength{\hangindent}{56pt}{刘邦\hspace{30pt}【{\akai 西皮二六}】孤王闻言心酸痛,怎舍得将军去尽忠。众卿创业各保重({\akai 或}:~功劳重),多亏文武众英雄。披霜戴雪真骁勇,血战疆场马蹄红。江山还未归一统,将士何曾受荣封。宁可城破孤命送,怎舍得将军遭剑锋。}

\setlength{\hangindent}{56pt}{纪信\hspace{30pt}【{\akai 西皮摇板}】大王再三不依从,回避({\akai 或}:~退避)何能表忠诚。投王麾下蒙恩宠({\akai 或}:~蒙恩重),}

\setlength{\hangindent}{56pt}{纪信\hspace{30pt}罢!}

\setlength{\hangindent}{56pt}{纪信\hspace{30pt}【{\akai 西皮摇板}】不如拔剑自尽忠。}

\setlength{\hangindent}{56pt}{刘邦\hspace{30pt}且慢。}

\setlength{\hangindent}{56pt}{刘邦\hspace{30pt}【{\akai 西皮摇板}】将军不必拔剑锋,心如铁石({\akai 或}:~心如石铁)一般同。\footnote{夏行涛{\scriptsize 君}认为``心如铁石({\akai 或}:~心如石铁)一般同''一句当由纪信接唱,此处从《京剧汇编》第七集~马连良~藏本。}}

\setlength{\hangindent}{56pt}{刘邦\hspace{30pt}先生,虽生死事急,孤实不忍;先生还是另想别策才是。}

\setlength{\hangindent}{56pt}{张良\hspace{30pt}纪将军执意效忠,大王,唉,依允了罢。}

\setlength{\hangindent}{56pt}{刘邦\hspace{30pt}先生有何妙计,你我君臣哪里逃走?}

\setlength{\hangindent}{56pt}{张良\hspace{30pt}待臣修书一封,命随何下到楚营,约定今晚大开东门纳降。主公将衣帽脱下,与纪将军穿换;再选美女数百名,随行车后。那霸王一见降书,必定深信;众楚兵观见美女,必定争攘。我君臣趁此喧哗之中,暗暗开了西门逃走,岂不是好?}

\setlength{\hangindent}{56pt}{刘邦\hspace{30pt}如此依计而行,就命先生修书。}

\setlength{\hangindent}{56pt}{张良\hspace{30pt}领旨。}

\setlength{\hangindent}{56pt}{张良\hspace{30pt}({\akai 念})兵马重围困,唤醒忠良臣。}

\setlength{\hangindent}{56pt}{刘邦\hspace{30pt}先生与孤传旨,吩咐文武,准备马摘鸾铃,美女梳妆伺候。}

\setlength{\hangindent}{56pt}{陈平\hspace{30pt}领旨。}

\setlength{\hangindent}{56pt}{陈平\hspace{30pt}({\akai 念})巧设良谋计,诈出荥阳城。}

\setlength{\hangindent}{56pt}{纪信\hspace{30pt}主公快将衣帽脱下,与臣穿戴,免得临时忙迫。}

\setlength{\hangindent}{56pt}{刘邦\hspace{30pt}唉,这是孤连累你、你$\cdots{}\cdots{}$你了!呃$\cdots{}\cdots{}$({\hwfs 哭}{\hwfs 介})}

\setlength{\hangindent}{56pt}{刘邦\hspace{30pt}卿家!}

\setlength{\hangindent}{56pt}{纪信\hspace{30pt}大王!}

\setlength{\hangindent}{56pt}{刘邦\hspace{30pt}将军!}

\setlength{\hangindent}{56pt}{纪信\hspace{30pt}我主!}

\setlength{\hangindent}{56pt}{刘邦\hspace{30pt}哎呀,卿家呀!}

\setlength{\hangindent}{56pt}{纪信\hspace{30pt}大王啊!}

\setlength{\hangindent}{56pt}{刘邦\hspace{30pt}【{\akai 西皮小导板}】楚汉年年大交锋({\akai 或}:~大交兵),}

\setlength{\hangindent}{56pt}{刘邦\hspace{30pt}\textless{}\!{\bfseries\akai 三叫头}\!\textgreater{}将军,卿家,唉,将军!呃$\cdots{}\cdots{}$({\hwfs 哭}{\hwfs 介})}

\setlength{\hangindent}{56pt}{纪信\hspace{30pt}大王啊!呃$\cdots{}\cdots{}$({\hwfs 哭}{\hwfs 介})}

\setlength{\hangindent}{56pt}{刘邦\hspace{30pt}【{\akai 西皮摇板}】不想今日【{\footnotesize 转}{\akai 西皮二六}】遭困中。丹心一片真梁栋,臣替君死第一功。江山若得归一统,在忠臣阁内画真容。}

\setlength{\hangindent}{56pt}{纪信\hspace{30pt}【{\akai 西皮快二六}】嬴秦无道社稷崩,楚汉分兵定关中。大王本是真命主,天降真龙下九重。又生我纪信容貌共,五行八字各不同。忍悲含泪({\akai 或}:~眼含珠泪)谢恩宠,}

\setlength{\hangindent}{56pt}{纪信\hspace{30pt}【{\akai 西皮摇板}】恕为臣假冒王号扮真龙。}

\setlength{\hangindent}{56pt}{刘邦\hspace{30pt}【{\akai 西皮快板}】见卿家哭得心酸痛,有辈古人听从容:~昔日重耳走西东,文武十人患难从。介子推\footnote{《京剧汇编》第七集~马连良~藏本作``介之推''。}割股把殷勤奉,黄河渡口保真龙。重耳回国归一统({\akai 或}:~成一统),却忘了({\akai 或}:~独忘了)子推他的割股的功。母子隐居不受封,绵山一旦被火焚。卿比子推功劳重,寡人非比(那)晋文公。问卿家你可有}

\setlength{\hangindent}{56pt}{刘邦\hspace{30pt}\textless{}\!{\bfseries\akai 哭头}\!\textgreater{}高堂母,}

\setlength{\hangindent}{56pt}{纪信\hspace{30pt}\textless{}\!{\bfseries\akai 哭头}\!\textgreater{}老娘亲呐,}

\setlength{\hangindent}{56pt}{刘邦\hspace{30pt}\textless{}\!{\bfseries\akai 哭头}\!\textgreater{}孤的老伯母啊,}

刘邦\\纪信\hspace{30pt}\raisebox{5pt}{\textless{}\!{\bfseries\akai 哭头}\!\textgreater{}啊,}

\setlength{\hangindent}{56pt}{纪信\hspace{30pt}\textless{}\!{\bfseries\akai 哭头}\!\textgreater{}老娘亲呐,}

\setlength{\hangindent}{56pt}{纪信\hspace{30pt}【{\akai 西皮原板}】家有年迈老慈容。臣受君恩难敬奉,未尽孝来先尽忠。}

\setlength{\hangindent}{56pt}{刘邦\hspace{30pt}【{\akai 西皮二六}】将军替孤把忠尽,可谓\footnote{《京剧汇编》第七集~马连良~藏本作``可为''。}人间第一功。卿母即是刘邦母,将伯母送至在({\akai 或}:~请至在)养老宫。生养死葬孤侍奉,金井玉葬送至在山中。问卿家你可有\textless{}\!{\bfseries\akai 哭头}\!\textgreater{}妻和子,}

\setlength{\hangindent}{56pt}{纪信\hspace{30pt}\textless{}\!{\bfseries\akai 哭头}\!\textgreater{}韩氏妻,}

\setlength{\hangindent}{56pt}{刘邦\hspace{30pt}\textless{}\!{\bfseries\akai 哭头}\!\textgreater{}孤的皇嫂,}

\setlength{\hangindent}{56pt}{纪信\hspace{30pt}\textless{}\!{\bfseries\akai 哭头}\!\textgreater{}小娇儿啊,}

\setlength{\hangindent}{56pt}{刘邦\hspace{30pt}\textless{}\!{\bfseries\akai 哭头}\!\textgreater{}皇侄啊({\akai 或}:~孤皇儿啊),}

刘邦\\纪信\hspace{30pt}\raisebox{5pt}{\textless{}\!{\bfseries\akai 哭头}\!\textgreater{}啊,}

\setlength{\hangindent}{56pt}{纪信\hspace{30pt}\textless{}\!{\bfseries\akai 哭头}\!\textgreater{}我的妻、儿啊,}

\setlength{\hangindent}{56pt}{纪信\hspace{30pt}【{\akai 西皮原板}】家有寒妻({\akai 或}:~家有贤妻)受苦穷。老母年高她侍奉,臣子还在襁褓中。}

\setlength{\hangindent}{56pt}{刘邦\hspace{30pt}【{\akai 西皮快板}】卿家替孤把忠尽,封妻荫子代代荣。卿妻就是({\akai 或}:~卿妻即是)刘邦嫂,你子我子一般同。太平年间归一统,举家满门受荣封。孤王若把良心昧,国破家亡不善终({\akai 或}:~无善终),天理不容。}

\setlength{\hangindent}{56pt}{纪信\hspace{30pt}【{\akai 西皮二六}】纪信闻言心酸痛,叩谢君王加荣封。寒门若得君王宠,死在九泉也欢荣。远望家乡珠泪滚({\akai 或}:~珠泪恸),}

\setlength{\hangindent}{56pt}{纪信\hspace{30pt}\textless{}\!{\bfseries\akai 哭头}\!\textgreater{}我的娘啊,}

\setlength{\hangindent}{56pt}{纪信\hspace{30pt}【{\akai 西皮快板}】难舍高堂老慈容。为儿不孝少侍奉,只为君王遭困中。儿替君难({\akai 或}:~儿替君死)留名重,要学丑父立奇功。妻子悬望我空\textless{}\!{\bfseries\akai 哭头}\!\textgreater{}梦,我的妻呀,}

\setlength{\hangindent}{56pt}{纪信\hspace{30pt}【{\akai 西皮快板}】难舍娇儿小孩童。父不能教儿习孔孟,父不能教儿马和弓。但愿儿成人把君奉,不枉后代受荣封。回头来奏一本,早作准备出牢笼。}

\setlength{\hangindent}{56pt}{陈平\hspace{30pt}参见主公!}

\setlength{\hangindent}{56pt}{刘邦\hspace{30pt}随何去献降书一事,怎么样了?}

\setlength{\hangindent}{56pt}{随何\hspace{30pt}霸王见了降书,十分欢喜。特来交旨。}

\setlength{\hangindent}{56pt}{刘邦\hspace{30pt}既然大事已成,准备女子可曾停当?}

\setlength{\hangindent}{56pt}{陈平\hspace{30pt}俱已停当。命周勃、滕公黄昏时候将东门大开,先将美女送出城去,纪将军随后出城;我君臣从西门逃走便了。}

\setlength{\hangindent}{56pt}{刘邦\hspace{30pt}如此改扮起来!}

\setlength{\hangindent}{56pt}{刘邦\hspace{30pt}【{\akai 西皮摇板}】君臣定下计牢笼,哭得天愁地也崩。打开玉笼飞彩凤,斩断金锁走蛟龙。}

\setlength{\hangindent}{56pt}{刘邦\hspace{30pt}\textless{}\!{\bfseries\akai 三叫头}\!\textgreater{}将军,卿家,唉,将军呐!}

\setlength{\hangindent}{56pt}{刘邦\hspace{30pt}罢!}

\setlength{\hangindent}{56pt}{纪信\hspace{30pt}摆驾楚营去者!}

\vspace{3pt}{\centerline{{[}{\hei 第六场}{]}}}\vspace{5pt}

\setlength{\hangindent}{56pt}{{刘邦\hspace{30pt}且住。幸喜你我君臣逃出虎口,同往燕赵去者。}}

\vspace{3pt}{\centerline{{[}{\hei 第七场}{]}}}\vspace{5pt}

\setlength{\hangindent}{56pt}{{项羽\hspace{30pt}【{\akai 西皮摇板}】孤王兴兵威风大,}}

\setlength{\hangindent}{56pt}{{项羽\hspace{30pt}【{\akai 西皮快板}】杀得鬼哭并神哀。刘邦小儿无可奈,}}

\setlength{\hangindent}{56pt}{{项羽\hspace{30pt}【{\akai 西皮摇板}】快写降表称孤怀。}}

\setlength{\hangindent}{56pt}{{项羽\hspace{30pt}困住小荥阳,捉拿汉刘邦。}}

{钟离眜\\季布\hspace{30pt}\raisebox{5pt}{启主公:~今有刘邦亲来投降,又带来无数美女。臣等俱已拿到。请主公发落。}}

\setlength{\hangindent}{56pt}{{项羽\hspace{30pt}将美女带上。}}

{钟离眛\\季布\hspace{30pt}\raisebox{5pt}{下面听者,主公有旨:~美女押上。}}

\setlength{\hangindent}{56pt}{{众女\hspace{30pt}叩见大王。}}

\setlength{\hangindent}{56pt}{{项羽\hspace{30pt}抬起头来。}}

\setlength{\hangindent}{56pt}{{众女\hspace{30pt}有罪不敢抬头。}}

\setlength{\hangindent}{56pt}{{项羽\hspace{30pt}恕你无罪。}}

\setlength{\hangindent}{56pt}{{众女\hspace{30pt}谢大王!}}

\setlength{\hangindent}{56pt}{{项羽\hspace{30pt}三军听令!}}

\setlength{\hangindent}{56pt}{{众\hspace{40pt}啊!}}

\setlength{\hangindent}{56pt}{{项羽\hspace{30pt}将美女分派各营。}}

\setlength{\hangindent}{56pt}{{众\hspace{40pt}得令!}}

\setlength{\hangindent}{56pt}{{项羽\hspace{30pt}沛君何在?}}

\setlength{\hangindent}{56pt}{{众\hspace{40pt}现在帐外。}}

\setlength{\hangindent}{56pt}{{项羽\hspace{30pt}吩咐有请。}}

\setlength{\hangindent}{56pt}{{众\hspace{40pt}有请千岁。}}

\setlength{\hangindent}{56pt}{{纪信\hspace{30pt}【{\akai 西皮快板}】荥阳替主代了命,假扮汉王诓楚君。含悲忍泪下车轮,}}

\setlength{\hangindent}{56pt}{{纪信\hspace{30pt}【{\akai 西皮快板}】豪杰心下暗思忖:~此番进帐把话论,霸王必定问真情。我若表出真名姓,霸王一定问斩刑。拚着一死留名姓({\akai 或}:~幽冥进),臣替君来({\akai 或}:~臣替君难)也甘心。纵死黄泉无怨恨,堂上哭坏老娘亲。妻儿悬望无音信,怀中舍了小娇生。指望阖家同欢庆,谁知({\akai 或}:~哪知)今日丧楚营。大胆且把宝帐进,}}

\setlength{\hangindent}{56pt}{{纪信\hspace{30pt}【{\akai 西皮快板}】上面坐的是楚君。两旁儿郎威风凛,刀枪剑戟杀气腾。大摇大摆且站定,各人怀揣一片心。}}

\setlength{\hangindent}{56pt}{{项羽\hspace{30pt}【{\akai 西皮快板}】灯光之下来观定,面前站定刘沛君。从先劝你来归顺,倚仗韩信会用兵。城楼你把巧言论,左有张良右陈平。好汉与孤来站定}\footnote{夏行涛{\scriptsize 君}建议作``战定'',此处从《京剧汇编》第七集~马连良~藏本。}{,孤家与你定输赢。草苗烧灰不必论,贪生怕死岂为人。韩信就该来救应,张良为何不用兵。劝主归降自惜命,方食爵禄保朝廷}\footnote{夏行涛{\scriptsize 君}建议作``报朝廷'',此处从《京剧汇编》第七集~马连良~藏本。}{。看来孤王有福分,}}

\setlength{\hangindent}{56pt}{{项羽\hspace{30pt}啊,呵呵哈哈哈$\cdots{}\cdots{}$({\hwfs 笑}{\hwfs 介})}}

\setlength{\hangindent}{56pt}{{项羽\hspace{30pt}【{\akai 西皮快板}】孤王本是有道君。宽洪量大人人敬,些许小仇哪在心。从前之事不究问}\footnote{段公平{\scriptsize 君}建议作``咎问'',此处从《京剧汇编》第七集~马连良~藏本。}{,还念当年结拜情。今日既然来归顺,前后之事我问清。见孤为何不跪定,佯瞅不睬为何情。开言便把刘邦问,难道还有两般心。}}

\setlength{\hangindent}{56pt}{{项羽\hspace{30pt}唗!胆大刘邦,见了孤王为何不跪?}}

\setlength{\hangindent}{56pt}{{纪信\hspace{30pt}住了,你乃一君,我乃一王,岂肯跪你?既知孤王到此,就该下位迎接才是。}}

\setlength{\hangindent}{56pt}{{项羽\hspace{30pt}啊?!听他声音不像刘邦,左右,掌灯待孤看来。}}

\setlength{\hangindent}{56pt}{{项羽\hspace{30pt}嘿嘿,中了他人之计也!}}

\setlength{\hangindent}{56pt}{{项羽\hspace{30pt}【{\akai 西皮快板}】看罢相貌怒气生,又中他人巧计行。开言便把儿郎问:~假扮刘邦哄寡人。好汉说出真名姓,你是他驾下什么人。}}

\setlength{\hangindent}{56pt}{{纪信\hspace{30pt}【{\akai 西皮快板}】霸王解开其中情,浑身上下冷汗淋。本当上前通名姓,必定斩首在辕门。本当不说真名姓,岂肯放我转回程。拚着一死把忠尽,叫声霸王听分明:~我的名字叫纪信,臣替君死标芳名({\akai 或}:~替主一死标美名)。}}

\setlength{\hangindent}{56pt}{{项羽\hspace{30pt}【{\akai 西皮摇板}】听说来了小纪信,不由孤王动无名。左右将他推出斩,}}

\setlength{\hangindent}{56pt}{{纪信\hspace{30pt}呵呵哈哈哈$\cdots{}\cdots{}$({\hwfs 笑}{\hwfs 介})}}

\setlength{\hangindent}{56pt}{{项羽\hspace{30pt}招回来!}}

\setlength{\hangindent}{56pt}{{项羽\hspace{30pt}【{\akai 西皮摇板}】纪信发笑为何情。}}

\setlength{\hangindent}{56pt}{{项羽\hspace{30pt}纪信为何发笑?}}

\setlength{\hangindent}{56pt}{{纪信\hspace{30pt}霸王,我来问你:~这君有难?}}

\setlength{\hangindent}{56pt}{{项羽\hspace{30pt}臣当替。}}

\setlength{\hangindent}{56pt}{{纪信\hspace{30pt}着哇!君有难臣当替。只因你兵困荥阳,攻打甚急,因此替主赴难,以解君忧也。}}

\setlength{\hangindent}{56pt}{{项羽\hspace{30pt}难道你不怕死?}}

\setlength{\hangindent}{56pt}{{纪信\hspace{30pt}忠臣不怕死,怕死不忠臣。昔日齐晋交兵,齐顷公领兵伐晋,谁想敌兵强盛,齐兵大败。杀得尸横遍野,血流成河,只剩顷公一人坐在车中。晋兵紧紧追赶,他驾前有一臣子名叫逄丑父,替主坐于车内,顷公才得脱难。彼时将他推出斩首,那丑父言道:~今日将我斩首,可惜只恐后来无人再替主难。晋公闻言将丑父放回,后来反加封赏。今吾主有难,臣当替主,你将我斩首,有何难哉?大王乃仁义之君,细心思忖!}}

\setlength{\hangindent}{56pt}{{项羽\hspace{30pt}呀!}}

\setlength{\hangindent}{56pt}{{项羽\hspace{30pt}【{\akai 西皮快板}】听他言语一番讲,可算得大义一忠良。本当将他来斩首,孤家岂是铁心肠。前朝君王遭魔障,留与后世标名芳。开言便叫二员将,快劝纪信把孤降。}}

{钟离眛\\季布\hspace{30pt} \raisebox{5pt}{【{\akai 西皮摇板}】黄罗宝帐领将令,开言叫声纪将军。}}

{钟离眛\\季布\hspace{30pt} \raisebox{5pt}{纪将军,归顺吾主,何愁封王爵位。}}

\setlength{\hangindent}{56pt}{{纪信\hspace{30pt}俺纪信生为汉臣,死为汉鬼。你劝我归顺,痴心妄想!}}

\setlength{\hangindent}{56pt}{{纪信\hspace{30pt}【{\akai 西皮摇板}】纪信忠心把主替,岂肯背主把降归。}}

\setlength{\hangindent}{56pt}{{项羽\hspace{30pt}纪信,孤家有意放你回去。}}

\setlength{\hangindent}{56pt}{{纪信\hspace{30pt}恐大王三心二意。}}

\setlength{\hangindent}{56pt}{{项羽\hspace{30pt}孤家焉有二意!}}

\setlength{\hangindent}{56pt}{{纪信\hspace{30pt}如此谢大王!}}

\setlength{\hangindent}{56pt}{{项羽\hspace{30pt}纪信!}}

\setlength{\hangindent}{56pt}{{项羽\hspace{30pt}【{\akai 西皮快板}】孤家今日将你放,后来且莫把恩忘。回营与我多拜上,拜上沛君小刘邦。趁早降书来呈上,免得孤家动刀枪。若有半字不停当,管教你君臣马前亡。}}

\setlength{\hangindent}{56pt}{{项羽\hspace{30pt}回去罢!}}

\setlength{\hangindent}{56pt}{{纪信\hspace{30pt}多谢大王!}}

\setlength{\hangindent}{56pt}{{纪信\hspace{30pt}【{\akai 西皮摇板}】谢过大王抽身往,}}

\setlength{\hangindent}{56pt}{{纪信\hspace{30pt}【{\akai 西皮快板}】隆恩似海福寿康。上钩金鳌脱了网,犹如会过五阎王。替主死难未斩丧,}}

\setlength{\hangindent}{56pt}{{纪信\hspace{30pt}【{\akai 西皮摇板}】也落得美名后世扬。}}

\setlength{\hangindent}{56pt}{{范增\hspace{30pt}【{\akai 西皮摇板}】听说吾主将他放,忙上宝帐问端详。}}

\setlength{\hangindent}{56pt}{{范增\hspace{30pt}范增见驾,愿主公千岁!}}

\setlength{\hangindent}{56pt}{{项羽\hspace{30pt}亚父平身。}}

\setlength{\hangindent}{56pt}{{范增\hspace{30pt}千千岁!}}

\setlength{\hangindent}{56pt}{{项羽\hspace{30pt}亚父进帐何事议论?}}

\setlength{\hangindent}{56pt}{{范增\hspace{30pt}老臣闻知,吾主已擒刘邦,问悉情由,乃是纪信替死。这样匹夫之辈,将他放回,犹恐}\footnote{段公平{\scriptsize 君}建议作``又恐''。}{生出别事。主公快快将他赶回斩首,免却后患。}}

\setlength{\hangindent}{56pt}{{项羽\hspace{30pt}嗯,亚父之言甚是。}}

\setlength{\hangindent}{56pt}{{项羽\hspace{30pt}钟、季二将!}}

{钟离眛\\季布\hspace{30pt} \raisebox{5pt}{在!}}

\setlength{\hangindent}{56pt}{{项羽\hspace{30pt}快将纪信赶回!}}

{钟离眛\\季布\hspace{30pt} \raisebox{5pt}{得令!}}

\setlength{\hangindent}{56pt}{{项羽\hspace{30pt}【{\akai 西皮摇板}】亚父暂且宽心放,要把纪信碎尸亡。}}

\setlength{\hangindent}{56pt}{{范增\hspace{30pt}【{\akai 西皮摇板}】辞别千岁下宝帐,管叫纪信一命亡。}}

\setlength{\hangindent}{56pt}{{纪信\hspace{30pt}【{\akai 西皮摇板}】适才放我得活命,赶我回来必有因({\akai 或}:~定有因)。}}

\setlength{\hangindent}{56pt}{{项羽\hspace{30pt}【{\akai 西皮摇板}】非是孤不肯饶你命,放虎归山反伤人。倒不如将你碎尸粉,作一个斩草除了根。}}

\setlength{\hangindent}{56pt}{{纪信\hspace{30pt}住口!}}

\setlength{\hangindent}{56pt}{{纪信\hspace{30pt}【{\akai 西皮快板}】纪信闻言笑盈盈,大骂匹夫听详情:~你二人昔年曾盟定,先到咸阳定为君。吾主宽宏行仁政,各路英雄({\akai 或}:~诸侯)俱归心。吾主咸阳江山定,谁像}\footnote{夏行涛{\scriptsize 君}建议作``谁想''。}{你是杀戮星。到一郡杀一郡,到一州来一州平。普天之下都怨恨,食你之肉({\akai 或}:~食尔之肉)方称心。将我主赶出咸阳郡,今日又困荥阳城。韩信带兵燕、赵境,无有能将领雄兵。纪信替主解围困,臣替君难古常情。蒙你释放将情领,赶我回来问典刑。你的言语不定准({\akai 或}:~无定准),易翻易覆无信人。纪信一死无怨恨,只恐你呀千年万载落骂名。}}

\setlength{\hangindent}{56pt}{{项羽\hspace{30pt}住口!}}

\setlength{\hangindent}{56pt}{{项羽\hspace{30pt}【{\akai 西皮摇板}】匹夫出言令人恨,要想活命万不能。}}

\setlength{\hangindent}{56pt}{{项羽\hspace{30pt}纪信,孤家本当将你斩首,有言在先,也罢!如今与你全尸。}}

\setlength{\hangindent}{56pt}{{项羽\hspace{30pt}钟、季二将,将纪信用火焚化!}}

{钟离眛\\季布\hspace{30pt}\raisebox{5pt}{啊!}}

{钟离眛\\季布\hspace{30pt}\raisebox{5pt}{$\cdots{}\cdots{}$焚化!}}

\setlength{\hangindent}{56pt}{{项羽\hspace{30pt}搭了下去。}}

\setlength{\hangindent}{56pt}{{项羽\hspace{30pt}众将官!}}

\setlength{\hangindent}{56pt}{{众\hspace{40pt}有!}}

\setlength{\hangindent}{56pt}{{项羽\hspace{30pt}随孤追赶刘邦去者!}}

\setlength{\hangindent}{56pt}{{众\hspace{40pt}啊!}}

