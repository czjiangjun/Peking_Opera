\newpage
\phantomsection %实现目录的正确跳转
\section*{\large\hei {汾河湾~{\small 之}~薛仁贵}}
\addcontentsline{toc}{section}{\hei 汾河湾~{\small 之}~薛仁贵}

\hangafter=1                   %2. 设置从第1⾏之后开始悬挂缩进  %
\setlength{\parindent}{0pt}{

\vspace{3pt}{\centerline{{[}{\hei 第一场}{]}}}\vspace{5pt}

({\hwfs 四}将{\hwfs 起霸},{\hwfs 发点},{\hwfs 四}文堂{\hwfs 站门},薛仁贵{\hwfs 上})

\textless{}\!{\bfseries\akai 点绛唇}\!\textgreater{}跨海征东,英名远震({\akai 或}:~威名远震)。军威盛,扫荡烟尘,保主锦绣春。

(薛仁贵{\hwfs 入大座})

({\akai 念})忆昔跨海去征东,拔山举鼎显异能。可恨张环行毒计,埋没英雄汗马勋。

本爵薛仁贵,(乃)山西绛州龙门县人氏。只因保定唐王跨海征东,立下十大汗马功劳,唐王见喜,封俺为平辽王之位({\akai 或}:~封我为平辽王之职)。只因我离家日久,不知妻室怎生度日,为此辞王别驾,回家探望。来此离家不远,我不免改换行装,以免惊动乡邻。

中军({\akai 或}:~左右),看衣更换。

(\textless{}\!{\bfseries\akai 合龙}\!\textgreater{},{\hwfs 吹}\textless{}\!{\bfseries\akai 牌子}\!\textgreater{},薛仁贵{\hwfs 换衣},{\hwfs 分开})

中军听令。

(中军\hspace{30pt}在。)

传令下去,(吩咐)大小三军就在此地靠山近水安营扎寨,不可踏践青苗,扰害百姓。({\akai 或}:~不可骚扰百姓,马踏青苗,违令者斩。)

(中军\hspace{30pt}$\cdots{}\cdots{}$传令已毕。)

(众{\hwfs 下})

(与爷)带马------

(薛仁贵{\hwfs 扯四门中}{\akai 唱})

\setlength{\hangindent}{56pt}{【{\akai 西皮原板}】忆昔当年去投军,张士贵是我的对头人。打虎遇着程千岁,他带我仁贵见了当今。卖弓计打破了摩天岭,三枝神箭辽东平。\footnote{刘曾复先生介绍说,《马上缘》一剧中开始薛仁贵唱与此段基本相同,只是后两句为``我的儿出兵无音信,且听探马报信音。''}前三日修下了辞王本, }

\setlength{\hangindent}{56pt}{【{\akai 西皮摇板}】回家探望妻迎春。 }

(\textless{}\!{\bfseries\akai 撤锣}\!\textgreater{}薛仁贵{\hwfs 下};~王禅老祖{\hwfs 上})

(王禅老祖\hspace{10pt}({\akai 念})$\cdots{}\cdots{}$)

(王禅老祖{\hwfs 唤}虎童,{\hwfs 下};~盖苏文魂{\hwfs 上},{\hwfs 过场})

\setlength{\hangindent}{52pt}{(盖苏文\hspace{20pt}【{\akai 西皮摇板}】$\cdots{}\cdots{}$驾起阴风朝前走,要报当年一箭仇。) }

(盖苏文魂{\hwfs 下},{\hwfs 接}柳迎春{\hwfs 上})

\vspace{3pt}{\centerline{{[}{\hei 第二场}{]}}}\vspace{5pt}

马来!

\setlength{\hangindent}{56pt}{【{\akai 西皮摇板}】催马来在汾河湾,见一顽童打弹丸。弹打,弹打南来宾鸿雁, }

枪挑{\footnotesize 呃},

\setlength{\hangindent}{56pt}{【{\akai 西皮摇板}】枪挑鱼儿水浪翻。翻身下了马雕鞍,再与顽童把话言。 }

那一顽童在此作何玩耍?

一弹上去,能打几雁落地?

为军的不信{\footnotesize 呐}。

好,你且打来。

呜哙呀,小小年纪有此本领,我若将他带回朝去,(将来)定是大大膀臂。

我自有道理。

啊,顽童。

弹打双雁落地,不足为奇。为军的(不才,呃,)我也会打雁{\footnotesize 呐}。

(薛丁山\hspace{20pt}一弹上去,能打几雁落地?)

一弹上去,我能打三雁落地。({\akai 或}:~呃呃,我能打三雁落地。)

(薛丁山\hspace{20pt}我却不信。)

(哦,)打来你看呐。

(薛丁山\hspace{20pt}你且打来。)

(呃------呃,)借弓弹一用。

且住!南山之上下来猛虎,有伤顽童之意。身旁带有袖箭,不免伤它一箭。

呔!顽童闪开,猛虎来了!看箭!\footnote{陈超老师注:~打虎的身段,老谭有两种演法。}

唉呀!

\setlength{\hangindent}{56pt}{【{\akai 西皮摇板}】打虎误伤顽童命,是非之地莫久停呐。仁贵拉马朝前奔({\akai 或}:~朝前进)。 }

\vspace{3pt}{\centerline{{[}{\hei 第三场}{]}}}\vspace{5pt}

马来!

\setlength{\hangindent}{56pt}{【{\akai 西皮快板}】适才离了({\akai 或}:~适才路过)汾河境,一马儿来在柳家村。勒住丝缰来观定, }

\setlength{\hangindent}{56pt}{【{\akai 西皮快板}】见一个妇人在窑门({\akai 或}:~见一位妇人站窑门\footnote{此处``站窑门''原来是``坐窑门''。})。布裙荆钗容貌整,看她好像柳迎春。翻身下马来询问({\akai 或}:~下了马能行), }

(薛仁贵{\hwfs 下马介})

\setlength{\hangindent}{56pt}{【{\akai 西皮快板}】躬身施礼把话云。 }

大嫂请了。

(柳迎春\hspace{20pt}还礼。军爷敢是失迷路途。)

正是失迷路途,请问大嫂,此处可是柳家村么?

(柳迎春\hspace{20pt}$\cdots{}\cdots{}$这面也是柳家村,这面也是柳家村。$\cdots{}\cdots{}$问的是哪个?)

此地有一柳氏迎春,大嫂可晓得?

(柳迎春\hspace{20pt}$\cdots{}\cdots{}$问她做甚?)

大嫂有所不知,我与他丈夫同营吃粮。与她({\akai 或}:~托我)带来万金家书,故而动问。

(我那薛大哥言道:~书信要面交本人。)

(柳迎春\hspace{20pt}不见本人呢?)

(原书带回。)

请便。

(柳迎春\hspace{20pt}啊军爷,与你打个哑谜你可晓得?)

这哑迷么?略知一二。

(柳迎春\hspace{20pt}这远------)

远在天边,不能相见。

(柳迎春\hspace{20pt}这近------)

哦!莫非你就是薛大嫂么?

哎呀呀,问来问去,问到本人的头上来了。

来来来,重见一礼呀。

(柳迎春\hspace{20pt}见过礼了。)

礼多人不怪呀。

大嫂请稍待。

哎呀且住,想我仁贵离家一十八载,不知她光景如何?

嗯,我自有道理。

啊大嫂,我实对你说了吧({\akai 或}:~我实对你讲了吧):~我那薛大哥,在营中已欠我二十两银子({\akai 或}:~在营中借了我二十两银子),将大嫂你就卖与我了。

呃,呃,呃,我有婚书为证呐。

呃,慢来慢来,我看大嫂变脸变色,婚书诓到手中,三把五把扯碎,为军的岂不落一个人财两空啊?

(柳迎春\hspace{20pt}依你只见?)

呃,你我去至前村,大户人家,请上三老四少,同拆同观。

当真。

哪个骗你呀?

\setlength{\hangindent}{52pt}{(柳迎春\hspace{20pt}【{\akai 西皮导板}】狠心的强盗啊$\cdots{}\cdots{}$) }

呵呵,她倒骂起来了哇!

哦,在哪里?

(柳迎春{\hwfs 关门})

哎,妻呀!

\setlength{\hangindent}{56pt}{【{\akai 西皮摇板}】叫声贤妻快开门,我是你丈夫薛仁贵转回程。 }

妻呀!

\setlength{\hangindent}{56pt}{【{\akai 西皮导板}】家住绛州县龙门, }

\setlength{\hangindent}{56pt}{【{\akai 西皮原板}】薛仁贵好命苦无亲无邻呐。幼年间父早亡母又丧命,丢下了仁贵无处身存。常言道姻缘一线引,柳家庄上招了亲。你的父嫌贫(他的)心太狠,将你我二人赶出了门庭\footnote{吴小如先生认为``门庭''应该是``门桯'',``桯''是门槛的意思。}。夫妻们双双【{\footnotesize 转}{\akai 西皮快板}】无投奔,破瓦寒窑\footnote{据《秋声集》\upcite{Qiusheng-Collect}载,程砚秋曾指出,陕西方言称有的窑洞为``坡洼寒窑'',即黄土坡下挖的窑洞,``坡洼''与``破瓦''谐音,艺人以讹传讹,久之就念成了``破瓦寒窑''。此处从俗。}暂安身。每日里窑中苦难尽,无奈何立志去投军。结交了兄弟们周青等,跨海征东把贼平。幸喜得狼烟俱扫尽,保定圣驾转回京。前三日修下了辞王的本,特地回来探望柳迎春。我的妻若还不肯信,来来来,算一算,连来带去十八春。

(柳迎春\hspace{20pt}$\cdots{}\cdots{}$薛郎,你好啊?)

我好,你可好啊?

你也不像从前了。这就是:~({\akai 念})少年子弟江湖老。

(柳迎春\hspace{20pt}({\akai 念})红粉佳人白了头。)

彼此?

一样。

啊,啊,哈哈哈$\cdots{}\cdots{}$({\hwfs 笑}{\hwfs 介})

(柳迎春\hspace{20pt}你临行之前,你还讲过什么言语你可记得?)

我讲过什么(言语),我倒是记不起来了哇。

(柳迎春\hspace{20pt}想必是做官回来了。({\akai 或}:~你不做官是不回来的,必定是做了官了。))

(呃,)再(也)不要提起做官呐,早去三天也好,晚去三天也好哇。

(柳迎春\hspace{20pt}$\cdots{}\cdots{}$刚刚凑巧。)

呃,凑巧倒还凑巧哇,

只是做了一名马头军\footnote{陈超老师注:~老谭晚年《汾河湾》炉火纯青,舞台状态极其放松。王凤卿叙述过很多老谭晚年演此戏的即兴身段,如``马头军''身段很特别,左手捋髯,右手勾起食指指出去,等旦角重复完``马头军'',晃晃勾起的手指,犹如马头吃草,配合微点几下头,再念``马头军''。舞台效果极佳。}。

(柳迎春\hspace{20pt}哦,马头军?)

({\akai 或}:~正是。)

(柳迎春\hspace{20pt}但不知你有多大的品级?)

呵,大得很呐!若论这品级台位么,呃,少不得,(少不得)也有它个七、八、十来品呐。

(柳迎春\hspace{20pt}呃,做官有七、八、十来品,但不知掌管什么?)

妻呀,为丈夫在家的时节,我管些什么?

(柳迎春\hspace{20pt}与人家看马。)

我如今呐,还是与人家牵马({\akai 或}:~看马)------

(柳迎春\hspace{20pt}$\cdots{}\cdots{}$看马。)

和从前是一样啊。

呃,有心胸,

(柳迎春\hspace{20pt}$\cdots{}\cdots{}$你有志气。)

有志气。

我这个志气还小吗?

(柳迎春\hspace{20pt}喂呀$\cdots{}\cdots{}$({\hwfs 哭介}))

呃,我不回来,你是盼我回来。我好容易回来了,你又是这样鼻子、脸子的。

好好好,我在家中住上三五天,呃,我还是出外啊。

呃,葬埋在龙头山。

何谓马头山?

呃,还是龙头山的受听呐。

龙头山,龙头山。({\akai 或}:~龙头山,龙头$\cdots{}\cdots{}$)

(柳迎春\hspace{20pt}$\cdots{}\cdots{}$马头山。)

啊,妻呀,我那岳父岳母百年之后,葬埋在何处啊?

呵,你看你看,到了他们家就成了凤凰山了。

依我看来,呵,不叫作凤凰山呐。

要叫作穷苦山。

你想啊,我在家的时节,你就是这样的受苦;我(如今)出外一十八载,如今回来,你还是这样的受苦({\akai 或}:~你还是住在这个破窑)。你爹娘生下你这受苦的女儿,呃,岂不是叫作穷苦山么?

呃,这也是你家的坟地里的风水呀。

呃,呵呵呵,穷苦山呐。

呵,穷苦山,穷$\cdots{}\cdots{}$

呃,这我倒不晓得呀。

哦,你是为我哇。

唉,我在外面一十八载,(省吃俭用,)受尽风霜之苦------

哦,我为的是哪个啊?

我啊,我也是为的是你呀。

我不为你,还为这座破窑不成么?

呃------我乃是受尽风霜之人,你不要呕我哇。

你不要呕我哇。

呵。

噗。

薛礼呀薛礼,你真真地岂有此理(呀)!

你今日回得家来,乃是一桩喜事,你偏偏要(来)呕她。

哎呀你看你看,把她气得这个样儿。

不妨不妨,待我取出一件东西,教她来看看,她就不生气了。

啊妻呀,为丈夫与你带回来好东西来了。

哎呀,(你也)特以地挖苦了,不是这些({\akai 或}:~这般)物件呐。

你拿去看来。

你看仔细。({\akai 或}:~仔细看来。)

你呀,拿过来吧。

这是我保定唐王跨海征东,立下十大汗马功劳,唐王见喜,封我为平辽王之位({\akai 或}:~之职)。这就是平辽王的虎头金印呐!

砷黄铜\footnote{``砷黄铜''和``砷白铜''又分别被称为``药金''和``药银'',是古代方士为实现以铜``炼制''金、银时,用砷化物(包括雄黄、雌黄、砒霜等)``点化''铜而生成的产物(砷铜合金)。砷黄铜和砷白铜的区别主要是砷含量不同,铜中含砷小于10\%,呈金黄色;当砷含量超过10\%,则呈银白色。砷元素易挥发,所谓``真金不怕火炼'',就是因为高温下砷黄铜中的砷会遇热流失,恢复铜的本质。}?!像这样的砷黄铜你见过几块呀。

呵呵呵,你(呀,)不开眼呐。

砷黄铜不要看了。

还要看看?

但要小心了。({\akai 或}:~你要看仔细。)

哎呀!

你还是拿过来吧。

你要把({\akai 或}:~要将)我这平辽王吞吃在腹内呀。

(柳迎春\hspace{20pt}$\cdots{}\cdots{}$饿怕了。)

惭愧!

啊妻呀,为丈夫一路行来,有些口渴,有什么香茶取来一用。

用些什么?

(好,与我取来。)

好好好,快些取来。

\setlength{\hangindent}{56pt}{【{\akai 西皮摇板}】在长安何曾吃白水,此水难饮泼埃尘。 }

不用啊。

(妻呀,为丈夫)腹中有些饥饿,有什么好酒好饭,取来一用呃。

用些什么?

何谓鱼羹?

好好好,与我取来。

\setlength{\hangindent}{56pt}{【{\akai 西皮摇板}】用手接过鲜鱼羹, }

呃,

\setlength{\hangindent}{56pt}{【{\akai 西皮摇板}】这样腥气实难闻。 }

不用了啊。

鞍马劳顿,身体困倦呐。

哦,怎么还有后窑?

好,快快打扫。({\akai 或}:~与我打扫)

我如今回来了。

啊?

\setlength{\hangindent}{56pt}{【{\akai 西皮摇板}】听她言来自思忖,莫非相交有情人。({\akai 或}:~察言观色详其情,教人心中解不明。)出得窑去观动静, }

\setlength{\hangindent}{56pt}{【{\akai 西皮摇板}】窑外并无一个人。 }

\setlength{\hangindent}{56pt}{【{\akai 西皮摇板}】将马拴在柳荫下, }

\setlength{\hangindent}{56pt}{【{\akai 西皮摇板}】鞍辔放在了地埃尘。 }

\setlength{\hangindent}{56pt}{【{\akai 西皮摇板}】站在窑中来观定, }

\setlength{\hangindent}{56pt}{【{\akai 西皮摇板}】这只男鞋必有因。 }

且住!怪道她变脸变色,原来她有了外遇了!

呀呀呸!(柳氏啊柳氏,)你在你丈夫跟前露出马脚来了。

贱人,你与我走了出来呀!

(呀呸!)你自己做的事还要问我?

你呀,你就是与我------

死呃。({\akai 或}:~呃!)

要赃。

要双。

呵呵,怕无有你的赃证?!这不是你的赃?这不是你的证?

你就是与我$\cdots{}\cdots{}$

唉!

(柳迎春\hspace{20pt}$\cdots{}\cdots{}$可问的这穿鞋的人$\cdots{}\cdots{}$)

呃,我不问这穿鞋的人儿,还问我这穿靴子的人么?

(柳迎春\hspace{20pt}$\cdots{}\cdots{}$比你强得多啊!)

(是啊,)自然比我强啊!

我如今有了这个讨厌的东西了。

是啊,你若是靠着我,饿啊,({\akai 或}:~你要靠着我,这一十八载,饿啊,)也把你饿干了哇。

什么新鲜的事情?({\akai 或}:~哦,什么新鲜之事。)

哎呀呀,你真是无羞无耻呀!

你不死待我来碰。

(你啊,你就是与我------)

(唉!)

嗯,有的。

也是有的。

一十七岁的孩儿------不大,不小,是正穿呐。

哎呀,她倒端起来了。

哎呀呀,你拿过来吧。

妇道人家,拿刀、动杖,呃,成什么样儿啊?

薛礼呀薛礼,难为你还是个平辽王啊,做事就是这样粗鲁。

哎呀呀,这窑前窑后,也无人前来解劝呐$\cdots{}\cdots{}$这这这$\cdots{}\cdots{}$

这这这$\cdots{}\cdots{}$这这这$\cdots{}\cdots{}$这这这$\cdots{}\cdots{}$

唉,上前赔个笑脸可也就拉倒了。

妻呀!为丈夫的不是,喏喏喏,我这厢(与你)赔礼了。

妻呀!为丈夫的不是,我这厢又赔礼了。

唉,妻呀!为丈夫这厢({\akai 或}:~这里)跪下了哇。({\akai 或}:~俱是为丈夫的不是,我这厢跪下了。)

哎呀你这是怎么样了?({\akai 或}:~呃,你这做什么?)

哎呀,耍出汗来了。

(啊)妻呀,将你我的儿子唤将出来,教他看看({\akai 或}:~教他看一看)我这不成器的老子。

哦,他往哪里去了?

(惊{\hwfs 介})

(啊)妻呀!我来问你,这窑前窑后,(可)还有别人家的孩儿会打雁?

贤妻你这里来啊!

你我的儿子出窑的时节,这头戴?

身穿?

左手?

右手?

唉呀!

\setlength{\hangindent}{56pt}{【{\akai 西皮导板}】听一言来吓掉魂, }

\textless{}\!{\bfseries\akai 三叫头}\!\textgreater{}丁山!我儿!唉!儿啊$\cdots{}\cdots{}$({\hwfs 哭介})

(柳迎春\hspace{20pt}$\cdots{}\cdots{}$儿的老子。)

唉!

\setlength{\hangindent}{56pt}{【{\akai 西皮散板}】凉水浇头怀抱冰。适才路过汾河境,见一个顽童打弹能。弹打南来宾鸿雁,枪挑鱼儿水浪分。 }

他不会来了\footnote{段公平{\scriptsize 君}建议作``他不回来了''。}!

\setlength{\hangindent}{56pt}{【{\akai 西皮摇板}】本当与她实言论,又恐吓坏这受苦的人呐。 }

唉!

\setlength{\hangindent}{56pt}{【{\akai 西皮摇板}】事到临头难瞒隐,咬定牙关说真情。 }

\textless{}\!{\bfseries\akai 叫头}\!\textgreater{}唉呀妻呀!

适才为丈夫打从汾河湾前经过,观见你我的儿子在那里射雁。南山之上,下来猛虎,身旁带有袖箭,实望将虎打走,不想这一箭呐------将你我的儿子就射死了!

射死了!

唉呀,又是一条人命呐!

醒来!

汾河湾前。

随我来呀。({\akai 或}:~一同寻找。)

丁山$\cdots{}\cdots{}$

我儿$\cdots{}\cdots{}$

}
