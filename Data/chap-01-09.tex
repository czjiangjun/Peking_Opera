\newpage
\subsubsection{\large\hei {长亭会~{\small 之}~伍员}}
\addcontentsline{toc}{subsection}{\hei 长亭会~{\small 之}~伍员}

\hangafter=1
\setlength{\parindent}{0pt}{
且住!

看那旁来了一哨人马({\akai 或}:~看前面一哨人马,),旌旗招展,斗大``申''字门旗,({\akai 或}:~门旗招展,上写斗大``申''字;门旗之上斗大``申''字;旌旗之上斗大``申''字,)想是申包胥({\akai 或}:~申贤弟)催贡还朝。俺不免勒住马头({\akai 或}:~我不免在此等候),将我的血海冤仇对他细说一遍({\akai 或}:~细表一遍)~!

贤弟若问,你且听道------

\setlength{\hangindent}{56pt}{【{\akai 西皮二六}】{未曾开言我的泪先流,尊一声贤弟听从头:~兄保平王功劳有,可叹我的忠心付水流。临潼会,兄为首,力举千斤压定了诸呃侯。双挂盟府}\footnote{ ``盟府''俗作``明辅''。承郝以鑫{\scriptsize 君}告,据元明间无名氏杂剧《临潼斗宝》:~``穆公与百里奚、秦姬辇设计,由百里奚与诸侯试文辞,秦姬辇比试武力,以争盟府地位。$\cdots{}\cdots{}$伍子胥文胜百里奚,武胜秦姬辇,夺得盟府地位,保定众诸侯。''  盟府是掌管盟约文书档案的机构,这里代指司盟之官。故``双挂盟府印二口''、``时来双挂盟府印'',均谓伍员文武双全之意。}{印二口,各国不敢统貔貅。恨平王无道贪色酒,父纳子媳礼不周。金顶轿换为银顶扣,}无祥女{改换马氏女流。我的父谏奏反遭斩首,可怜我的老娘也被刀割头}\footnote{ ``刀割头''亦可作``刀过头''。}{。多亏了家将越墙逃命走,来到了樊城细报根由。愚兄闻言怒冲牛斗,武城黑带兵围困城头。万般无奈我才下了毒手啊,也是我}认{扣搭弓}\footnote{ 张斯琦{\scriptsize 君}认为此处当作``纫扣搭弓''。}{箭射他咽喉。观只见门旗``申''字大如斗,就知晓贤弟催贡转回头。诉一诉我的含冤}:~{杀我的全家三百余口,就是那鸡犬也不留。似这等的冤仇怎忍受,不杀平王气怎休~?!} }

\setlength{\hangindent}{56pt}{【}西皮快板{】}贤弟把话错出口,愚兄言来听从头:~君无道,臣逃走,父不正来子外游。吾与贤弟共乡土\footnote{ 夏行涛{\scriptsize 君}建议此处``共乡土''似作``共相楚''亦通。},相交胜似亲骨肉。你今回朝休泄漏,不念今朝念从头。我今借兵来伐楚,不杀平王誓不休({\akai 或}:~气怎休)。

\setlength{\hangindent}{56pt}{【}西皮快板{】}申包胥与我把智斗({\akai 或}:~把誓斗)\footnote{ 钱盛{\scriptsize 君}指出,``把智斗''可能是京剧流变过程中,湖北方言``把志赌''的讹误,存此以备一说。},背转身来又加愁。如是领兵({\akai 或}:~若是领兵)来伐楚,不忠的名儿万古留。我不领兵({\akai 或}:~若不领兵)来伐楚,血海冤仇一旦休。

\setlength{\hangindent}{56pt}{【}西皮快板{】}贤弟请上兄叩首,

\setlength{\hangindent}{56pt}{【}西皮快板{】}多谢你放我吴国投({\akai 或}:~往东流)。你兴楚来我灭楚,你为公来我为仇。辞别贤弟跨马走,

马来!

\setlength{\hangindent}{56pt}{【{\akai 西皮摇板}】扬鞭打马吴国投。 }
}
