\newpage
\subsubsection{\large\hei {碰碑}}
\addcontentsline{toc}{subsection}{\hei 碰碑}

\hangafter=1                   %2. 设置从第1⾏之后开始悬挂缩进  %}
\setlength{\parindent}{0pt}{

{\vspace{3pt}{\centerline{{[}{\hei 第一场}{]}}}\vspace{5pt}}

\setlength{\hangindent}{56pt}{(\textless{}\!{\bfseries\akai 点绛唇}\!\textgreater{},杨延嗣{\hwfs 上高台})}

\setlength{\hangindent}{56pt}{杨延嗣\hspace{20pt}({\akai 念})忆昔当年赴两狼,交牙虎口摆战场。可恨潘洪行毒计,法标屈受箭锋芒。}

\setlength{\hangindent}{56pt}{杨延嗣\hspace{20pt}吾乃七郎鬼魂是也。今当我父归位之期,我不免去至宋营,托梦一番({\akai 或}:~托梦一回)。众鬼卒,驾起阴风,宋营去者。}

\setlength{\hangindent}{56pt}{(杨延嗣{\hwfs 下高台},{\hwfs 面向里})}

\setlength{\hangindent}{56pt}{杨延嗣\hspace{20pt}【{\akai 二黄导板}】叫鬼卒列两厢前把路引,}

\setlength{\hangindent}{56pt}{(杨延嗣{\hwfs 面向外})}

\setlength{\hangindent}{56pt}{杨延嗣\hspace{20pt}【{\akai 回龙}】杨七郎在空中暗自思忖:~}

\setlength{\hangindent}{56pt}{杨延嗣\hspace{20pt}【{\akai 二黄原板}\footnote{刘曾复先生示范说戏时告知,此处台上``扯四门''。}】我杨家投宋君忠心秉正({\akai 或}:~忠心耿耿),弟兄们八员将来到雁门。两狼山打一仗父子被困,可怜我搬救兵不得回程。叫鬼卒驾阴风宋营来进({\akai 或}:~叫鬼卒前引路两狼来进),此一去到宋营托梦爹尊。}

\setlength{\hangindent}{56pt}{(杨延嗣{\hwfs 下})}

\vspace{3pt}{\centerline{{[}{\hei 第二场}{]}}}\vspace{5pt}

\setlength{\hangindent}{56pt}{杨继业\hspace{20pt}({\akai 内})【{\akai 二黄导板}】金乌坠玉兔升黄昏时候,}

\setlength{\hangindent}{56pt}{(杨继业{\hwfs 上})}

\setlength{\hangindent}{56pt}{杨继业\hspace{20pt}【{\akai 回龙}】盼娇儿不由人珠泪双流,我的儿啊!}

\setlength{\hangindent}{56pt}{杨继业\hspace{20pt}【{\akai 二黄三眼}】七郎儿【{\footnotesize 转}{\akai 二黄原板}】回雁门搬兵求救,为什么此一去不见回头。唯恐那潘仁美忆起前仇,怕的是我的儿一命罢休。含悲泪进大营双眉愁皱({\akai 或}:~含悲泪进宝帐双眉愁皱),腹内饥身寒冷遍体飕飕哇。}

\setlength{\hangindent}{56pt}{(杨延昭{\hwfs 上})}

\setlength{\hangindent}{56pt}{杨延昭\hspace{20pt}【{\akai 二黄原板}】听谯楼打罢了二更时分,杨延昭倒做了巡营之人。迈虎步我且把宝帐来进,又只见老爹爹瞌睡沉沉。我这里上前去与父盖定,}

\setlength{\hangindent}{56pt}{杨延昭\hspace{20pt}【{\akai 二黄原板}】闷悠悠坐一旁好不伤情。}

\setlength{\hangindent}{56pt}{(杨延嗣{\hwfs 上})}

\setlength{\hangindent}{56pt}{杨延嗣\hspace{20pt}【{\akai 二黄原板}】听谯楼打罢了三更时分,阴曹府来了我七郎鬼魂。叫鬼卒驾阴风宋营来进,}

\setlength{\hangindent}{56pt}{杨延嗣\hspace{20pt}【{\akai 二黄原板}】又只见老爹爹瞌睡沉沉。我这里将他的灵魂唤醒,}

\setlength{\hangindent}{56pt}{(\textless{}\!{\bfseries\akai 乱锤}\!\textgreater{},杨继业{\hwfs 托髯口})}

\setlength{\hangindent}{56pt}{杨继业\hspace{20pt}【{\akai 二黄散板}】猛抬头又只见七郎娇生。我命儿回雁门搬取救应,儿为何哭啼啼身带雕翎?}

\setlength{\hangindent}{56pt}{杨继业\hspace{20pt}【{\akai 二黄散板}】我这里下位去将儿抱定呐,}

\setlength{\hangindent}{56pt}{杨延嗣\hspace{20pt}【{\akai 二黄原板}】老爹爹休贪睡细听儿云({\akai 或}:~老爹爹休贪睡儿有话云):~都只为潘仁美想起了打子仇恨,将孩儿绑法标乱箭攒身。}

\setlength{\hangindent}{56pt}{杨延嗣\hspace{20pt}【{\akai 二黄原板}】转面来再对六兄论,小弟言来你是听:~高堂老母要你孝敬,杨延嗣倒做了不孝之人\footnote{``不孝之人''也可作``不肖之人''。}。我本当与父兄再把话论({\akai 或}:~我本当与六兄再把话论),可怜我父在阳儿在阴\footnote{刘曾复先生示范说戏时介绍,裘桂仙此句唱``怕的是天明亮难回天庭''。}。}

\setlength{\hangindent}{56pt}{(杨延嗣{\hwfs 下})}

\setlength{\hangindent}{56pt}{杨继业\hspace{20pt}【{\akai 二黄导板}】方才朦胧得一梦({\akai 或}:~方才朦胧将养静呐),}

\setlength{\hangindent}{56pt}{杨继业\hspace{20pt}【{\akai 二黄散板}】梦见了七郎儿转回大营呐。}

\setlength{\hangindent}{56pt}{杨继业\hspace{20pt}【{\akai 二黄散板}】睁开了昏花眼难以扎挣,}

\setlength{\hangindent}{56pt}{杨继业\hspace{20pt}【{\akai 二黄散板}】又只见六郎儿瞌睡沉沉。}

\setlength{\hangindent}{56pt}{杨继业\hspace{20pt}(我儿)醒来。}

\setlength{\hangindent}{56pt}{杨延昭\hspace{20pt}【{\akai 二黄散板}】$\cdots{}\cdots{}$见七弟啊入梦,抬头只见老严亲。}

\setlength{\hangindent}{56pt}{杨延昭\hspace{20pt}\textless{}\!{\bfseries\akai 叫头}\!\textgreater{}哎呀,爹爹啊!}

\setlength{\hangindent}{56pt}{杨延昭\hspace{20pt}孩儿昨晚四更时分,梦见七弟浑身是血,遍体雕翎。不知是何缘故?}

\setlength{\hangindent}{56pt}{杨继业\hspace{20pt}为父也得此兆。唉呀儿啊------这``梦梦相应,必有应验''。为父意欲,命我儿回至雁门,打听你七弟的下落,为父的也好放心呐。({\akai 或}:~为父也有此兆。有道是``梦梦相应,必有应验''。为父有意命我儿回转雁门,探听你七弟的下落,为父的也好放心呐。或:~为父也有此兆。唉呀儿啊------这``梦梦相同,必有应验''。为父有意命我儿回转雁门,探听你七弟的下落,为父的也好放心呐。)}

\setlength{\hangindent}{56pt}{杨延昭\hspace{20pt}孩儿不去。}

\setlength{\hangindent}{56pt}{杨继业\hspace{20pt}为何?}

\setlength{\hangindent}{56pt}{杨延昭\hspace{20pt}爹爹年迈,孩儿放心不下。}

\setlength{\hangindent}{56pt}{杨继业\hspace{20pt}为父么------虽则年迈,倒还康健。({\akai 或}:~儿来看------为父的虽则年迈,身体倒还康健。)有道是:~虎老雄心在。儿只管地前去。}

\setlength{\hangindent}{56pt}{(杨延昭\hspace{20pt}孩儿放心不下。)}

\setlength{\hangindent}{56pt}{杨继业\hspace{20pt}你当真不去?}

\setlength{\hangindent}{56pt}{杨延昭\hspace{20pt}当真不去。}

\setlength{\hangindent}{56pt}{杨继业\hspace{20pt}果然不去。}

\setlength{\hangindent}{56pt}{杨延昭\hspace{20pt}果然不去。}

\setlength{\hangindent}{56pt}{杨继业\hspace{20pt}儿啊,为父有父子之情,难道儿就无有手足之义么?}

\setlength{\hangindent}{56pt}{杨延昭\hspace{20pt}爹爹不必如此,孩儿前去就是。}

\setlength{\hangindent}{56pt}{杨继业\hspace{20pt}好,上马去罢!}

\setlength{\hangindent}{56pt}{杨延昭\hspace{20pt}【{\akai 二黄散板}】爹爹不必泪伤淋,孩儿言来听分明:~倘若胡兵来叫阵,紧守大营莫出征。辞别爹爹足踏镫,雁门关前走一程。}

\setlength{\hangindent}{56pt}{(杨延昭\hspace{20pt}\textless{}\!{\bfseries\akai 叫头}\!\textgreater{}爹爹!~我父!)}

\setlength{\hangindent}{56pt}{杨继业\hspace{20pt}\textless{}\!{\bfseries\akai 叫头}\!\textgreater{}延昭!~我儿!}

\setlength{\hangindent}{56pt}{(杨延昭\hspace{20pt}罢!)}

\setlength{\hangindent}{56pt}{(杨延昭{\hwfs 下})}

\setlength{\hangindent}{56pt}{杨继业\hspace{20pt}\textless{}\!{\bfseries\akai 哭头}\!\textgreater{}啊$\cdots{}\cdots{}$我的儿啊!}

\setlength{\hangindent}{56pt}{杨继业\hspace{20pt}【{\akai 二黄散板}】见娇儿上了马能行,指着雁门骂一声{\footnotesize 呐}({\akai 或}:~手指潘洪骂一声{\footnotesize 呐;}~手指雁门骂一声{\footnotesize 呐})。我儿若有好和歹{\footnotesize 呀},}

\setlength{\hangindent}{56pt}{杨继业\hspace{20pt}潘洪{\footnotesize 呐}!贼------}

\setlength{\hangindent}{56pt}{杨继业\hspace{20pt}【{\akai 二黄散板}】我将老命与尔拼({\akai 或}:~拚着老命与尔拼)。}

\setlength{\hangindent}{56pt}{(杨继业{\hwfs 下})}

\vspace{3pt}{\centerline{{[}{\hei 第三场}{]}}}\vspace{5pt}

\setlength{\hangindent}{56pt}{(丑{\hwfs 扮}韩延寿{\hwfs 上})}

\setlength{\hangindent}{56pt}{韩延寿\hspace{20pt}俺,韩延寿!奉了太后之命,巡营瞭哨。}

\setlength{\hangindent}{56pt}{韩延寿\hspace{20pt}巴特鲁,巡营瞭哨者。}

\setlength{\hangindent}{56pt}{(杨延昭{\hwfs 上},{\hwfs 拿剑与}韩延寿{\hwfs 架住})}

\setlength{\hangindent}{56pt}{杨延昭\hspace{20pt}何人挡住某家去路!}

\setlength{\hangindent}{56pt}{韩延寿\hspace{20pt}六郎,前者饶尔不死,又来则甚?}

\setlength{\hangindent}{56pt}{杨延昭\hspace{20pt}一派胡言,放马过来!}

\setlength{\hangindent}{56pt}{(杨延昭、韩延寿{\hwfs 一合两合},杨延嗣{\hwfs 上},{\hwfs 挥蝇帚},韩延寿众{\hwfs 倒地}。杨延昭、杨延嗣{\hwfs 下}。韩延寿{\hwfs 起})}

\setlength{\hangindent}{56pt}{韩延寿\hspace{20pt}且住!~正要擒拿六郎下马,七郎显圣,不是马走如飞,险遭不测。}

\setlength{\hangindent}{56pt}{韩延寿\hspace{20pt}巴特鲁,收兵。}

\vspace{3pt}{\centerline{{[}{\hei 第四场}{]}}}\vspace{5pt}

\setlength{\hangindent}{56pt}{(杨延昭{\hwfs 上})}

\setlength{\hangindent}{56pt}{杨延昭\hspace{20pt}休赶{\footnotesize 呐}休赶。}

\setlength{\hangindent}{56pt}{杨延昭\hspace{20pt}【{\akai 二黄散板}】打开玉笼飞彩凤,斩断金锁走蛟龙。}

\setlength{\hangindent}{56pt}{(杨延昭{\hwfs 下})}

\setlength{\hangindent}{56pt}\vspace{3pt}{\centerline{{[}{\hei 第五场}{]}}}\vspace{5pt}

\setlength{\hangindent}{56pt}{({\hwfs 四}老军\textless{}\!{\bfseries\akai 慢长锤}\!\textgreater{}{\hwfs 引}杨继业{\hwfs 上})}

\setlength{\hangindent}{56pt}{杨继业\hspace{20pt}【{\akai 反二黄慢板}】叹杨家秉忠心大宋扶保,到如今呐只落得冰解瓦消。恨北国萧银宗打来战表,擅想夺吾主爷锦绣龙朝。贼潘洪在金殿帅印挂了,我父子倒做了马前的英豪。}

\setlength{\hangindent}{56pt}{(杨继业{\hwfs 归中间},{\hwfs 四}老军{\hwfs 坐下})}

\setlength{\hangindent}{56pt}{杨继业\hspace{20pt}【{\akai 反二黄慢板}】金沙滩双龙会一阵败呃了,只杀得血成河鬼哭神嚎。我的大郎儿【{\footnotesize 转}{\akai 反二黄快三眼}】替宋王把忠尽了,二郎儿短剑下命赴阴曹。杨三郎被马踏尸首不晓,四、八郎失番营无有下梢。杨五郎在五台学禅修道,七郎儿被潘洪箭射法标({\akai 或}:~箭射芭蕉\footnote{刘曾复先生曾告知,``芭蕉''是从俗的唱法。})。只落得杨延昭随营征讨,可怜他尽得忠、又尽孝,昼夜杀砍、马不停蹄、为国辛劳。可怜我八个子把四子丧了,把四子丧了!}

\setlength{\hangindent}{56pt}{杨继业\hspace{20pt}\textless{}\!{\bfseries\akai 哭头}\!\textgreater{}我的儿啊!}

\setlength{\hangindent}{56pt}{杨继业\hspace{20pt}【{\akai 反二黄原板}】眼见得年迈人无有下梢({\akai 或}:~眼见得一家人无有下梢)。方良臣和潘洪又生计呃巧,请我主到五台快乐逍遥。又谁知中了那奸贼的笼套,四下里众番奴犹如海潮。(耳边厢又听得一声号炮,直吓得宋王爷跌落鞍桥。)多亏了杨延昭一马来到哇,一杆枪救圣驾逃出笼牢。有老夫二次里也来赶到,害得我东西杀砍、左冲右突、虎闯羊群,被困在两狼山,里无粮、外无草,救兵不到,眼见得我这老残生就难以还朝。}

\setlength{\hangindent}{56pt}{杨继业\hspace{20pt}\textless{}\!{\bfseries\akai 哭头}\!\textgreater{}我的儿啊!}

\setlength{\hangindent}{56pt}{({\hwfs 四}老军{\hwfs 站起})}

\setlength{\hangindent}{56pt}{(老军\hspace{30pt}饿!)}

\setlength{\hangindent}{56pt}{杨继业\hspace{20pt}【{\akai 反二黄原板}】饥饿了就该把战马斩了,}

\setlength{\hangindent}{56pt}{(老军\hspace{30pt}冷呐!)}

\setlength{\hangindent}{56pt}{杨继业\hspace{20pt}【{\akai 反二黄原板}】身寒冷({\akai 或}:~天寒冷)就该把大营焚烧。}

\setlength{\hangindent}{56pt}{(老军\hspace{30pt}雁来了! 雁来了!)}

\setlength{\hangindent}{56pt}{(杨继业{\hwfs 望},{\hwfs 拿弓射雁})}

\setlength{\hangindent}{56pt}{杨继业\hspace{20pt}【{\akai 反二黄原板}】宝雕弓打不着空{\footnotesize 呃}中飞鸟,弓折弦断\footnote{刘曾复先生曾告知,``弓折弦断''唱``弓奓弦断''更准确,``奓''是``张开''的意思。}({\akai 或}:~弓开箭断)为的是哪条?}

\setlength{\hangindent}{56pt}{(老军\hspace{30pt}石虎把战马咬倒!)}

\setlength{\hangindent}{56pt}{杨继业\hspace{20pt}再探!}

\setlength{\hangindent}{56pt}{杨继业\hspace{20pt}不、不、不$\cdots{}\cdots{}$不好了!}

\setlength{\hangindent}{56pt}{杨继业\hspace{20pt}【{\akai 二黄散板}】恨石虎把我的战马咬倒\footnote{段公平{\scriptsize 君}注:~刘曾复先生曾告,``咬''字古音念作``交(\textrm{ji\=ao})''音},为大将无坐骑怎把兵交({\akai 或}:~为大将无良骑怎把兵交)?}

\setlength{\hangindent}{56pt}{杨继业\hspace{20pt}【{\akai 二黄散板}】看过了青龙刀\footnote{此处``青龙刀''也有唱``定宋刀''的。}且把路找{\footnotesize 呃},寻一个避风所再作计{\footnotesize 呃}较。}

\setlength{\hangindent}{56pt}{(杨继业{\hwfs 下})}

\vspace{3pt}{\centerline{{[}{\hei 第六场}{]}}}\vspace{5pt}

\setlength{\hangindent}{56pt}{(\textless{}\!{\bfseries\akai 点绛唇}\!\textgreater{},苏武{\hwfs 穿}红官衣,忠纱,黑三,{\hwfs 上高台},{\hwfs 坐})}

\setlength{\hangindent}{56pt}{苏武\hspace{30pt}({\akai 念})太阳一出万丈高,光阴犹如斩人刀。日月穿梭催人老,盖世忠良无下稍。}

\setlength{\hangindent}{56pt}{苏武\hspace{30pt}吾乃大汉苏武是也。今当令公归位之期,奉了玉帝敕旨,前去点化。}

\setlength{\hangindent}{56pt}{苏武\hspace{30pt}众云童,}

\setlength{\hangindent}{56pt}{(众\hspace{40pt}有。)}

\setlength{\hangindent}{56pt}{苏武\hspace{30pt}架起祥云,两狼去者。}

\setlength{\hangindent}{56pt}{苏武\hspace{30pt}\textless{}\!{\bfseries\akai 清江引}\!\textgreater{}渺渺茫茫祥云万丈高,荡荡悠悠人间不觉晓。善恶明彰报({\akai 或}:~善恶彰明报),只争晚共早({\akai 或}:~只争迟和早)。六道轮回,终须走这遭。\footnote{陈超老师注:~唱\textless{}\!{\bfseries\akai 清江引}\!\textgreater{}牌子的时候,苏武{\hwfs 不动},云童{\hwfs 一翻两翻}。过去《鸿鸾禧》一剧,天喜星也唱这个牌子,唱完后小生才{\akai 念}``好大雪''。}}

\setlength{\hangindent}{56pt}{(众\hspace{40pt}已至两狼。)}

\setlength{\hangindent}{56pt}{苏武\hspace{30pt}(好,)看衣改换。}

\setlength{\hangindent}{56pt}{(\textless{}\!{\bfseries\akai 合龙}\!\textgreater{}苏武{\hwfs 当场换衣})}

\setlength{\hangindent}{56pt}{苏武\hspace{30pt}(尔等)两厢退下。}

\setlength{\hangindent}{56pt}{(苏武{\hwfs 面向小边甩蝇帚})}

\setlength{\hangindent}{56pt}{苏武\hspace{30pt}变化一座苏武庙({\akai 或}:~幻化一座苏武庙),}

\setlength{\hangindent}{56pt}{(苏武{\hwfs 面向大边甩蝇帚})}

\setlength{\hangindent}{56pt}{苏武\hspace{30pt}变化一座李陵碑({\akai 或}:~幻化一座李陵碑)。}

\setlength{\hangindent}{56pt}{(苏武{\hwfs 将蝇帚扔向后台})}

\setlength{\hangindent}{56pt}{苏武\hspace{30pt}再变化一只老羊({\akai 或}:~再幻化一只老羊)。}

\setlength{\hangindent}{56pt}{苏武\hspace{30pt}远远望见令公来也。}

\setlength{\hangindent}{56pt}{(杨继业\hspace{20pt}走哇!)}

\setlength{\hangindent}{56pt}{(杨继业{\hwfs 上})}

\setlength{\hangindent}{56pt}{杨继业\hspace{20pt}【{\akai 二黄散板}】当年保驾五台山,智空长老对我言。他道我在两狼山前遭围困,到如今果应了那智空言。}

\setlength{\hangindent}{56pt}{杨继业\hspace{20pt}来此不知什么所在?也不知({\akai 或}:~但不知)怎样回转大营。}

\setlength{\hangindent}{56pt}{苏武\hspace{30pt}嗯哼!}

\setlength{\hangindent}{56pt}{杨继业\hspace{20pt}看那旁有一老丈,待我上前问来。}

\setlength{\hangindent}{56pt}{杨继业\hspace{20pt}啊,老丈请了。}

\setlength{\hangindent}{56pt}{苏武\hspace{30pt}请了。({\akai 或}:~还礼。军爷敢是失迷路途?)}

\setlength{\hangindent}{56pt}{杨继业\hspace{20pt}请问老丈,此处什么所在?}

\setlength{\hangindent}{56pt}{苏武\hspace{30pt}({\akai 念})此处是两狼({\akai 或}:~此地是两狼),前山是我庄。虎口交牙峪,犯者一命亡。}

\setlength{\hangindent}{56pt}{(杨继业\hspace{20pt}哦。)}

\setlength{\hangindent}{56pt}{杨继业\hspace{20pt}啊老丈,你在此则甚呐?}

\setlength{\hangindent}{56pt}{苏武\hspace{30pt}(我)在此牧羊。}

\setlength{\hangindent}{56pt}{杨继业\hspace{20pt}诶------这样的兵荒马乱,你还牧的什么羊啊?}

\setlength{\hangindent}{56pt}{苏武\hspace{30pt}({\akai 念})(我)管他兵荒不兵荒,与我却无妨。老汉无别事,在此牧老羊。}

\setlength{\hangindent}{56pt}{杨继业\hspace{20pt}难道说这只老羊还有什么贵处吗?}

\setlength{\hangindent}{56pt}{苏武\hspace{30pt}({\akai 念})休说老羊无贵处({\akai 或}:~休道老羊无贵处),他的名儿天下扬。生下几个羊羔子,轰轰烈烈在世上。今朝几个死,明朝几个亡。老汉掐指算,今日死老羊。}

\setlength{\hangindent}{56pt}{苏武\hspace{30pt}老羊,老羊,你还不死------}

\setlength{\hangindent}{56pt}{杨继业\hspace{20pt}可恼!}

\setlength{\hangindent}{56pt}{杨继业\hspace{20pt}【{\akai 二黄散板}】这老丈说话理不通啊,句句伤的是杨令公。手持宝刀将尔砍呐,}

\setlength{\hangindent}{56pt}{(苏武{\hwfs 收}杨令公{\hwfs 刀})}

\setlength{\hangindent}{56pt}{杨继业\hspace{20pt}【{\akai 二黄散板}】霎时不见我的护身龙。 }

\setlength{\hangindent}{56pt}{杨继业\hspace{20pt}且住。清风一阵,老丈不见,又将我的宝刀拿去。}

\setlength{\hangindent}{56pt}{杨继业\hspace{20pt}唉呀!有道是:~(身)为大将者,宁舍千军,不舍寸铁。待我将他赶上。}

\setlength{\hangindent}{56pt}{杨继业\hspace{20pt}``苏武庙''------呜哙呀。}

\setlength{\hangindent}{56pt}{杨继业\hspace{20pt}想汉室苏武,乃是大大忠良,死后有人替他修庙在此。待我进去看来------({\akai 或}:~想那苏武是炎汉忠良,死后何人与他修庙在此。待我进庙看来------;~想苏武乃是汉室忠良,死后有人与他修庙在此。待我进庙看来------)}

\setlength{\hangindent}{56pt}{杨继业\hspace{20pt}``李陵碑''------呀呀呸!}

\setlength{\hangindent}{56pt}{杨继业\hspace{20pt}想这李陵,乃是卖主求荣({\akai 或}:~想那李陵,乃是背主求荣)大大的奸佞,死后何人与他立这碑碣在此!
({\akai 或}:~想李陵乃是背主求荣,死后有人与他立这碑碣在此!)}

\setlength{\hangindent}{56pt}{杨继业\hspace{20pt}那旁还有几行小字,待我(上前)看来:~}

\setlength{\hangindent}{56pt}{(\textless{}\!{\bfseries\akai 阴锣}\!\textgreater{},杨继业{\hwfs 掸土})}

\setlength{\hangindent}{56pt}{杨继业\hspace{20pt}({\akai 念})庙是苏武庙,碑是李陵碑,令公来到此,这------卸甲------又丢盔!}

\setlength{\hangindent}{56pt}{杨继业\hspace{20pt}且住!哪里是苏武庙、李陵碑,分明神明点化于我------想我被困在两狼山,内无粮草,外无救应。({\akai 念})白日受饥饿,夜晚受风吹。盼兵兵不到,这盼子({\akai 或}:~看子)------子不归。难道说,我还等到冻饿而死?!}

\setlength{\hangindent}{56pt}{杨继业\hspace{20pt}也罢------我不免拜谢宋王爵禄之恩,就碰死在李陵------碑下!}

\setlength{\hangindent}{56pt}{(杨继业\hspace{20pt}\textless{}\!{\bfseries\akai 牌子}\!\textgreater{}令公跪倒苏武庙,喂呀------圣上啊$\cdots{}\cdots{}$李陵碑下丧黄泉({\akai 或}:~丧残生)。)}

\setlength{\hangindent}{56pt}{(杨继业{\hwfs 碰碑死介}\footnote{刘曾复先生说戏时说明:~过去杨令公碰碑后上韩延寿,现一般略去。})}

\setlength{\hangindent}{56pt}{(韩延寿{\hwfs 上})}

\setlength{\hangindent}{56pt}{韩延寿\hspace{20pt}呜哙呀!$\cdots{}\cdots{}$已死,尸首不可损坏。报与太后知道!}

\setlength{\hangindent}{56pt}{(韩延寿{\hwfs 下})}

}
