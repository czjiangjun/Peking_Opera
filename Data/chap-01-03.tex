\newpage
%\hypertarget{ux5b5dux611fux5929-ux4e4b-ux5171ux53d4ux6bb5}{%
\subsubsection{%\texorpdfstring
{孝感天%\protect\hyperlink{fn13}{\textsuperscript{13}}
~{\small 之}~共叔段}~\protect\footnote{根据刘曾复先生和杨绍箕先生2009年9月25日在电话里说戏录音整理。录音由杨绍箕先生托梁剑峰老师提供,刘曾复先生在电话中向杨绍箕先生主要介绍了整出戏的唱词、调度,同时介绍了小生的唱法。}}%{孝感天13 之 共叔段}}\label{ux5b5dux611fux5929-ux4e4b-ux5171ux53d4ux6bb5}}
\addcontentsline{toc}{subsection}{\hei 孝感天~\small{之}~共叔段}

\hangafter=1                   %2. 设置从第1⾏之后开始悬挂缩进  %
\setlength{\parindent}{0pt}{
{\centerline{\textrm{{[}\hei 第一场{]}}}}
\vspace{5pt}

(\textless{}\!{\bfseries\akai 大锣打上}\!\textgreater{}{\hwfs 四}龙套{\hwfs 拿云牌上},{\hwfs 站门},共叔段\footnote{共叔段由旦角应工,且所唱小生腔不能与旦角相重。}%\protect\hyperlink{fn14}{\textsuperscript{14}}
、卫云环{\hwfs 同上},{\hwfs 到台中间})

共叔段\hspace{20pt}~ ({\akai 念})丹心天地惨,黄泉路途遥。

(卫云环\hspace{15pt}~ ({\akai 念})溺爱反遭害,恩情一旦抛\footnote{刘曾复先生存本此句作``恩情有限闲'',上注``恩情有显消''。}%\protect\hyperlink{fn15}{\textsuperscript{15}}
。)

共叔段\hspace{20pt}~ 吾乃共叔段鬼魂\footnote{刘曾复先生存本此处作``灵魂''。}%\protect\hyperlink{fn16}{\textsuperscript{16}}
是也。

(卫云环\hspace{15pt}~ 吾乃卫云环鬼魂是也\footnote{刘曾复先生存本此句作``吾乃卫氏灵魂是也''。}%\protect\hyperlink{fn17}{\textsuperscript{17}}
。)

\setlength{\hangindent}{60pt}{   %3. 设置悬挂缩进量                %
	共叔段\hspace{20pt}~ 生前吾母爱子爱媳,反误我夫妻性命。闻得吾母在颖地之中,思儿想媳,不免前去探望一番。\footnote{刘曾复先生存本此句作``(生白)~段在生为姜国母爱子爱媳,不想反遭兄王之害,我夫妻不免梦中解劝一番呵。夫人请。(旦接)~夫君请。''}%\protect\hyperlink{fn18}{\textsuperscript{18}}
}

共叔段、\\
卫云环\hspace{20pt}~ \raisebox{5pt}{请。}

共叔段\hspace{20pt}~ 正是:~({\akai 念})蜀魄啼残三月雨\footnote{此句据刘曾复先生存本,当系准词。``蜀魄''是典籍中常用的杜鹃的别称。刘曾复先生在介绍此剧时可能据别本,作``树破提惨三月雨'',不确。}%\protect\hyperlink{fn19}{\textsuperscript{19}
},

(卫云环\hspace{15pt}~ ({\akai 念})梦魂惊断五更风\footnote{刘曾复先生存本``惊断''上注``凄断''。}%\protect\hyperlink{fn20}{\textsuperscript{20}}
。)

(\textless{}\!{\bfseries\akai 大锣打下}\!\textgreater{}{\hwfs 四}龙套{\hwfs 下},共叔段、卫云环{\hwfs 下},\textless{}\!{\bfseries\akai 撤锣}\!\textgreater{})

\vspace{3pt}{\centerline{\textrm{{[}{\hei 第二场}{]}}}}\vspace{5pt}

(\textless{}\!{\bfseries\akai 小锣打上}\!\textgreater{}{\hwfs 二}宫女{\hwfs 上},{\hwfs 站门},姜氏{\hwfs 上})

(姜氏\hspace{25pt}~ {[}{\akai 引子}{]} 身居颖地,思娇儿,珠泪双悲。)

(姜氏坐{\hwfs 小座})

(姜氏\hspace{25pt}~ ({\akai 念})生离最是苦,死后何伤乎!世看姜国母,有子不如无。)

\setlength{\hangindent}{60pt}{   %3. 设置悬挂缩进量                %
(姜氏\hspace{25pt}~ 哀家姜皇后,只因溺爱次子共叔段,反累夫妇双亡。可恨寤生,将我迁入颖地居住。思想起来,好不伤惨人也。)
}

(\textless{}\!{\bfseries\akai 扎多乙}\!\textgreater{})

\setlength{\hangindent}{60pt}{   %3. 设置悬挂缩进量                %
(姜氏\hspace{25pt}~ 【{\akai 二黄原板}】姜国母在颖地长吁短叹,思皇儿昼夜里珠泪不干。恨寤生他与我母子情断,但不知何日里得转故园。)
}

({\bfseries\akai 起更},姜氏{\hwfs 站起},\textless{}\!{\bfseries\akai 小拉子}\!\textgreater{},姜氏{\hwfs 脱黄帔})

(姜氏\hspace{25pt}~ 回避了。)

({\hwfs 二}宫女{\hwfs 翻下})

\setlength{\hangindent}{60pt}{   %3. 设置悬挂缩进量                %
(姜氏\hspace{25pt}~ 【{\akai 二黄慢板}】居深宫冷清清无依无伴,怕只怕睡不安月照栏杆。对银灯眼昏花思自叹,)
}

(姜氏{\hwfs 进桌子})

(姜氏\hspace{25pt}~ 【{\akai 二黄慢板}】有谁人怜念我影孤形单。)

(\textless{}\!{\bfseries\akai 长锤}\!\textgreater{},共叔段{\hwfs 上},卫氏{\hwfs 同随上},{\hwfs 同站小边台口})

共叔段\hspace{20pt}~ 【{\akai 二黄慢板}】风飘飘冷飕飕黄昏惨淡, 曾记得在生时束带顶冠。

(共叔段{\hwfs 挖门进},{\hwfs 站大边},{\hwfs 朝里})
   
\setlength{\hangindent}{65pt}{   %3. 设置悬挂缩进量                %
	(卫云环\hspace{15pt}~ 【{\akai 二黄慢板}】此时间讲什么粉消香散,进寝宫({\akai 或}:~进宫去)愿母后免去心酸。\footnote{``粉消香散''四字从刘曾复先生存本,当系准词,说戏录音作``焚烧香泛'',李楠君以为作``焚烧香饭'',香饭是佛家饭食。存本上注``霎时间说不尽风流云散,进宫来见母后心内痛酸。''}%\protect\hyperlink{fn21}{\textsuperscript{21}}
)
}

(卫云环{\hwfs 挖门进},{\hwfs 站小边},{\hwfs 朝里})

共叔段\hspace{20pt}~ \textless{}\!{\bfseries\akai 双叫头}\!\textgreater{}母亲!老娘啊!$\cdots${}$\cdots{}$({\hwfs 哭介})

(卫云环\hspace{15pt}~ \textless{}\!{\bfseries\akai 双叫头}\!\textgreater{}母后!婆母!喂呀$\cdots${}$\cdots{}$({\hwfs 哭介}))\footnote{刘曾复先生存本此处作``(生旦同白)~国母醒来。''}%\protect\hyperlink{fn22}{\textsuperscript{22}}

(姜氏\hspace{25pt}~ 【{\akai 反二黄导板}】梦儿里又听得有人呼唤,)

(姜氏\hspace{25pt}~ 【{\akai 反二黄摇板}】又只见灯光下双影闪闪。看不明抚一抚昏花老眼,)

共叔段\hspace{20pt}~ 母亲啊$\cdots${}$\cdots{}$({\hwfs 哭介})

(卫云环\hspace{15pt}~ 母后,喂呀$\cdots${}$\cdots{}$({\hwfs 哭介}))

(\textless{}\!{\bfseries\akai 凤点头}\!\textgreater{})

\setlength{\hangindent}{65pt}{   %3. 设置悬挂缩进量                %
(姜氏\hspace{25pt}~ 【{\akai 反二黄摇板}】却原来子叔段、儿媳云环。曾记得寿宴前({\akai 或}:~曾记得寿筵前)金樽酒满,你夫妻双双拜母子同欢。自从儿到京都未能相见,)
}

({\hwfs 起}{\akai 反二黄}【{\akai 回龙}】)

共叔段、\\
卫云环\hspace{20pt}~  \raisebox{5pt}{【{\akai 回龙}】尊国母且不必双目泪涟\footnote{刘曾复先生存本此句作``虽是儿丧黄泉也却心甘''。}%\protect\hyperlink{fn23}{\textsuperscript{23}}
。}

\setlength{\hangindent}{60pt}{   %3. 设置悬挂缩进量                %
	共叔段\hspace{15pt}~ 【{\akai 反二黄慢板}】悲切切尊国母魂飞魄散,悔不该图兄位自惹身残。辜负了生身母无依无伴,儿死在黄泉路也难心甘。\footnote{刘曾复先生存本此处作``悲切切尊国母魂伤魄断,悔不该图兄位自惹身残。辜负了生身母无依无伴,儿死在黄泉路瞑目心甘。''末句上注``劝国母切莫要珠泪不干''。}%\protect\hyperlink{fn24}{\textsuperscript{24}}
}

\setlength{\hangindent}{60pt}{   %3. 设置悬挂缩进量                %
	(卫云环\hspace{10pt}~ 【{\akai 反二黄慢板}】这也是天命定数有修短,劝国母切莫言终日泪涟。为报恩来世里重会慈范,千古来有谁人百岁同欢。\footnote{刘曾复先生存本此处作``这也是天命定数有修短,劝国母终日里免却伤惨。为报恩来世里重亲慈范,千古来是何人百岁同欢。''末句上注``千古来有何人百岁同欢''。}%\protect\hyperlink{fn25}{\textsuperscript{25}})
}

\setlength{\hangindent}{60pt}{   %3. 设置悬挂缩进量                %
	(姜氏\hspace{20pt}~ 【{\akai 反二黄原板\footnote{姜氏也可以唱【{\akai 反二黄慢板}】,但一般都唱【{\akai 反二黄原板}】。}%\protect\hyperlink{fn26}{\textsuperscript{26}}
}】听他言不由我心中悲惨,原来是我的儿站在面前。莫非是母子们梦里相见,纵然是梦相会娘也心甘。)
}

({\hwfs 叫散},\textless{}\!{\bfseries\akai 扭丝}\!\textgreater{}\textless{}\!{\bfseries\akai 凤点头}\!\textgreater{})\footnote{刘曾复先生存本此处有``(生白)~母后哇$\cdots${}$\cdots{}$(接【{\akai 反调摇板}】)''}%\protect\hyperlink{fn27}{\textsuperscript{27}}

\setlength{\hangindent}{60pt}{   %3. 设置悬挂缩进量                %
	共叔段\hspace{15pt}~ 【{\akai 反二黄摇板}】娘在阳儿在阴两厢隔断,母子们要相逢({\akai 或}:~要相会;要重逢)梦里团圆。\footnote{刘曾复先生存本此处作``儿在阴娘在阳两厢隔断,母子们要相逢今世却难。''}%\protect\hyperlink{fn28}{\textsuperscript{28}}
}

(共叔段{\hwfs 出门},{\hwfs 下})

(卫云环\hspace{10pt}~ 【{\akai 反二黄摇板}】天将明儿要归不必怜念,指日里郑君侯迎请凤鸾。\footnote{刘曾复先生存本此处作``天将明儿要归不尽悲惨,郑君侯指日里迎请凤鸾。''}%\protect\hyperlink{fn29}{\textsuperscript{29}}
)

(卫云环{\hwfs 出门},{\hwfs 下}。{\bfseries\akai 五更},{\bfseries\akai 亮更}\footnote{刘曾复先生介绍唱词及场次时说明,在中间可以加更次。}%\protect\hyperlink{fn30}{\textsuperscript{30}}
,{\hwfs 二}宫女{\hwfs 上})

(二宫女\hspace{15pt}~ 国太醒来。)

(姜氏\hspace{20pt}~ 【{\akai 二黄导板}】适才间见娇儿肝肠痛断,)

(姜氏\hspace{20pt}~ 儿啊$\cdots${}$\cdots{}$({\hwfs 哭介}))

(姜氏{\hwfs 出位},{\hwfs 站在桌前})

(姜氏\hspace{20pt}~ 【{\akai 二黄摇板}】醒来时却原来南柯梦间。相劝我好言语甚是悲惨,)

({\hwfs 二}宫女{\hwfs 扯斜})

(姜氏\hspace{20pt}~ 【{\akai 二黄摇板}】等候了郑君侯接我回还({\akai 或}:~接我回銮)。)

(姜氏\hspace{20pt}~ 唉,儿啊$\cdots${}$\cdots{}$({\hwfs 哭介}))

(姜氏{\hwfs 下},{\hwfs 二}宫女{\hwfs 随下})

\vspace{25pt}
{\bfseries\textrm{本戏人物扮相}}:~
\vspace{15pt}

共叔段:~\hspace{20pt}~ 戴紫金冠,穿红蟒、玉带,拿云帚,不戴鬼发、魂帕;

卫云环:~\hspace{20pt}~  戴凤冠,穿女蟒、玉带,拿云帚,不戴鬼发、魂帕;

姜氏:~\hspace{30pt}~ 外穿黄帔、内穿深紫帔、罩黑坎肩、系黄绦子,绿裙,不戴凤冠。

\vspace{15pt}
刘曾老特别说明,如果作为开场戏,四龙套可不拿云牌,共叔段可穿开氅,卫云环可穿宫装。

