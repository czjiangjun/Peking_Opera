\newpage
\subsubsection{\large\hei {上天台}}
\addcontentsline{toc}{subsection}{\hei 上天台}

\hangafter=1                   %2. 设置从第1⾏之后开始悬挂缩进  %
\setlength{\parindent}{0pt}{

\setlength{\hangindent}{56pt}{汉光武\hspace{20pt}({\akai 内白})摆驾!}

\setlength{\hangindent}{56pt}{(四太监,大太监,汉光武上,\textless{}\!{\bfseries\akai 帽子头}\!\textgreater{}起)}

\setlength{\hangindent}{56pt}{汉光武\hspace{20pt}【{\akai 二黄慢板}】金钟响玉磬鸣王出龙廷({\akai 或}:~宫廷)\footnote{``龙廷''一般作``龙庭'',此处从《京剧新序》。陈超老师注:~此句唱完后,刘秀在过门里{\hwfs 整冠}、{\hwfs 捋髯}。与一般演法刘秀在``金钟响''前面的过门里{\hwfs 整冠}不同。},有寡人({\akai 或}:~汉光武)喜的是五谷丰登。君有道民安乐风调雨顺,文安邦武定国四海升平。文凭着邓先生阴阳有准,武仗着姚皇兄\footnote{据《后汉书》,``姚期''当作``铫期'',故``姚皇兄''当作``铫皇兄'',``姚刚''当作``铫刚''。此处从《京剧新序》,下同。}扶保乾坤。内侍臣摆御驾九龙口进,}

\setlength{\hangindent}{56pt}{郭妃\hspace{30pt}({\akai 内})喂呀$\cdots{}\cdots{}$}

\setlength{\hangindent}{56pt}{(汉光武{\hwfs 进大座})\hspace{30pt}}

\setlength{\hangindent}{56pt}{汉光武\hspace{20pt}({\akai 接唱})殿角下是何人来放悲声?({\akai 或}:~又听得殿角下大放悲声。)}

\setlength{\hangindent}{56pt}{(\textless{}\!{\bfseries\akai 小锣抽头}\!\textgreater{}{\hwfs 四}宫女{\hwfs 引}郭妃{\hwfs 上})}

\setlength{\hangindent}{56pt}{郭妃\hspace{30pt}【{\akai 二黄摇板}】轻移莲步上龙廷,万岁驾前说详情。}

\setlength{\hangindent}{56pt}{郭妃\hspace{30pt}({\akai 白})喂呀万岁呀!}

({\hwfs 跪哭})

\setlength{\hangindent}{56pt}{汉光武\hspace{20pt}梓童为何这等模样?}

\setlength{\hangindent}{56pt}{郭妃\hspace{30pt}启禀万岁,今有姚刚将郭老太师剑劈府门,万岁做主哇$\cdots{}\cdots{}$}

\setlength{\hangindent}{56pt}{汉光武\hspace{20pt}呜哙呀,有这等事,梓童暂且回宫,寡人自有裁处。}

\setlength{\hangindent}{56pt}{郭妃\hspace{30pt}谢万岁!}

\setlength{\hangindent}{56pt}{郭妃\hspace{30pt}【{\akai 二黄摇板}】谢罢万岁后宫进,}

\setlength{\hangindent}{56pt}{(郭妃{\hwfs 立})\hspace{30pt}}

\setlength{\hangindent}{56pt}{郭妃\hspace{30pt}({\akai 接唱})管叫姚刚一命倾。}

\setlength{\hangindent}{56pt}{郭妃\hspace{30pt}({\akai 念})喂呀$\cdots{}\cdots{}$(\textless{}\!{\bfseries\akai 小锣打下}\!\textgreater{})}

\setlength{\hangindent}{56pt}{汉光武\hspace{20pt}内侍,宣伴驾王带子上殿。}

\setlength{\hangindent}{56pt}{内侍\hspace{30pt}万岁有旨,伴驾王带子上殿呐!}

\setlength{\hangindent}{56pt}{姚期\hspace{30pt}({\akai 内})领旨,}

\setlength{\hangindent}{56pt}{姚期\hspace{30pt}【{\akai 二黄导板}】安定府绑姚刚怒气爆发,}

\setlength{\hangindent}{56pt}{(姚刚{\hwfs 上到小边台口},姚期{\hwfs 上到台口})}

\setlength{\hangindent}{56pt}{姚期\hspace{30pt}({\akai 接唱})【{\akai 二黄散板}】只气得年迈人二目昏花。郭太师在朝中势力甚大,满朝中文武臣谁不让他。似这等王法儿全不惧怕,少时节见万岁定把儿杀。}

\setlength{\hangindent}{56pt}{姚刚\hspace{30pt}【{\akai 二黄散板}】老爹爹休得要担惊害怕,容孩儿把此事细说根芽,在府门那郭荣将儿辱骂,杀却了老奸贼不犯王法。}

\setlength{\hangindent}{56pt}{姚期\hspace{30pt}({\akai 接唱})小奴才说此话胆比天大,杀皇亲还说是不犯王法。}

\setlength{\hangindent}{56pt}{姚期\hspace{30pt}儿是好汉?}

\setlength{\hangindent}{56pt}{姚刚\hspace{30pt}儿是好汉。}

\setlength{\hangindent}{56pt}{姚期\hspace{30pt}({\akai 接唱})是好汉随为父参王见驾,}

\setlength{\hangindent}{56pt}{(姚刚{\hwfs 到大边台口},姚期{\hwfs 小边立指}刚)}

\setlength{\hangindent}{56pt}{姚期\hspace{30pt}跪下!}

\setlength{\hangindent}{56pt}{(姚刚{\hwfs 面外跪})}

\setlength{\hangindent}{56pt}{姚期\hspace{30pt}({\akai 接唱})一桩桩一件件启奏皇家。}

\setlength{\hangindent}{56pt}{(姚期{\hwfs 参拜})}

\setlength{\hangindent}{56pt}{姚期\hspace{30pt}臣姚期见驾吾皇万岁。}

\setlength{\hangindent}{56pt}{汉光武\hspace{20pt}皇兄平身。}

\setlength{\hangindent}{56pt}{(姚期{\hwfs 立},{\hwfs 大边立})}

\setlength{\hangindent}{56pt}{汉光武\hspace{20pt}姚皇兄你可知罪?}

\setlength{\hangindent}{56pt}{姚期\hspace{30pt}臣知罪,但不知罪犯何条?}

\setlength{\hangindent}{56pt}{汉光武\hspace{20pt}只因你三子姚刚将郭老太师剑劈府门,还说无罪?}

\setlength{\hangindent}{56pt}{姚期\hspace{30pt}太师也有一项大罪。}

\setlength{\hangindent}{56pt}{汉光武\hspace{20pt}太师何罪之有?({\akai 或}:~哪一项大罪?)}

\setlength{\hangindent}{56pt}{姚期\hspace{30pt}太师府外,立有禁地,文官落轿,武官离鞍,万岁可曾降旨?}

\setlength{\hangindent}{56pt}{汉光武\hspace{20pt}(这$\cdots{}\cdots{}$)寡人有意,尚未传旨。}

\setlength{\hangindent}{56pt}{姚期\hspace{30pt}如此斩者无亏。}

\setlength{\hangindent}{56pt}{汉光武\hspace{20pt}好个斩者无亏。}

\setlength{\hangindent}{56pt}{姚刚\hspace{30pt}绑坏了。}

\setlength{\hangindent}{56pt}{汉光武\hspace{20pt}殿角绑的何臣?}

\setlength{\hangindent}{56pt}{姚期\hspace{30pt}臣子姚刚。}

\setlength{\hangindent}{56pt}{汉光武\hspace{20pt}哎呀呀,不要绑坏孤的小爱卿,(内侍,)快快松绑。}

\setlength{\hangindent}{56pt}{(内侍{\hwfs 松绑},姚刚{\hwfs 立进殿里跪})}

\setlength{\hangindent}{56pt}{姚刚\hspace{30pt}谢万岁不斩之恩。}

\setlength{\hangindent}{56pt}{汉光武\hspace{20pt}非是寡人不斩于你,孤登基之日有言在先({\akai 或}:~寡人有言在先),姚不反汉,汉不斩姚,赐你三千人马发放湖北({\akai 或}:~发往湖北;发往湖广)宛子城,扶保({\akai 或}:~保定)殿下刘庄,你父子就在午门一别,领旨下殿。}

%姚期\hspace{30pt}\\姚刚\hspace{30pt}{\raisebox{5pt}{({\hwfs 同})谢万岁。}
\raisebox{0pt}[22pt][16pt]{\raisebox{8pt}{姚期}\raisebox{-8pt}{\hspace{-22pt}{姚刚}}\raisebox{0pt}{\hspace{30pt}({\hwfs 同})谢万岁。}}

\setlength{\hangindent}{56pt}{(姚期、姚刚{\hwfs 下殿},{\hwfs 大边台口})}

\setlength{\hangindent}{56pt}{姚期\hspace{30pt}【{\akai 二黄散板}】万岁爷赦了姚霸林,好似枯木又逢春。手拉娇儿下龙廷,}

\setlength{\hangindent}{56pt}{(姚期{\hwfs 拉}姚刚{\hwfs 到大边外角})}

\setlength{\hangindent}{56pt}{姚期\hspace{30pt}({\akai 接唱})为父言来听分明。此去湖北休逞性,切莫再来惹祸根。}

\setlength{\hangindent}{56pt}{姚期\hspace{30pt}儿来看,}

\setlength{\hangindent}{56pt}{姚期\hspace{30pt}({\akai 接唱})为父的两鬓如霜年耳顺,好一似风前灯、瓦上霜能活几春?}

\setlength{\hangindent}{56pt}{姚刚\hspace{30pt}({\akai 接唱})爹爹不必细叮咛,孩儿岂是不孝人。在朝为官多惊恐,不如弃职回故林。午门别父心酸痛,回府拜别老娘亲。}

\setlength{\hangindent}{56pt}{姚刚\hspace{30pt}爹爹,我父,爹爹呀!~({\hwfs 同}期)}

\setlength{\hangindent}{56pt}{姚期\hspace{30pt}姚刚,霸林,儿呀!~({\hwfs 同}刚)}

\setlength{\hangindent}{56pt}{姚刚\hspace{30pt}罢!}

\setlength{\hangindent}{56pt}{(姚刚{\hwfs 下},姚期{\hwfs 望})\hspace{20pt}}}

\setlength{\hangindent}{56pt}{姚期\hspace{30pt}姚刚,霸林,({\hwfs 哭})啊$\cdots{}\cdots{}$我的儿呀!}

\setlength{\hangindent}{56pt}{姚期\hspace{30pt}【{\akai 二黄散板}】午门外去了姚霸林,娇儿言语记在心。二次再把龙廷进,告职归林乐安宁。}

\setlength{\hangindent}{56pt}{(姚{\hwfs 再次上殿跪})}

\setlength{\hangindent}{56pt}{姚期\hspace{30pt}臣姚期二次见驾吾皇万岁。}

\setlength{\hangindent}{56pt}{汉光武\hspace{20pt}姚皇兄为何去而复返?}

\setlength{\hangindent}{56pt}{姚期\hspace{30pt}臣启万岁,臣告职归林。}

\setlength{\hangindent}{56pt}{汉光武\hspace{20pt}姚皇兄告职归林,(教)寡人怎能舍得({\akai 或}:~怎生舍得)?}

\setlength{\hangindent}{56pt}{姚期\hspace{30pt}微臣也舍不得万岁。}

\setlength{\hangindent}{56pt}{汉光武\hspace{20pt}既然如此({\akai 或}:~既然舍不得寡人;既然舍不得孤王)就该在朝奉君(的)才是。}

\setlength{\hangindent}{56pt}{姚期\hspace{30pt}万岁教臣在朝奉君,愿君依臣一项大事。}

\setlength{\hangindent}{56pt}{汉光武\hspace{20pt}皇兄奏来。({\akai 或}:~哪一项大事?)}

\setlength{\hangindent}{56pt}{姚期\hspace{30pt}愿吾主戒酒百日。}

\setlength{\hangindent}{56pt}{汉光武\hspace{20pt}只要皇兄在朝,漫说(是)戒酒百日,就是周年半载又有何妨,内侍,快快搀起孤的姚皇兄。}

\setlength{\hangindent}{56pt}{(内侍{\hwfs 扶起}姚期,姚{\hwfs 站小边})}

\setlength{\hangindent}{56pt}{汉光武\hspace{20pt}【{\akai 二黄慢板}】姚皇兄休得要告职归林,你本是擎天柱一根。汉江山多亏了皇兄所挣,叫寡人怎舍得开国元勋,你我是布衣的君臣({\hwfs 也可唱}【{\akai 顶板满江红}】)。}

\setlength{\hangindent}{56pt}{姚期\hspace{30pt}【{\akai 二黄原板}】非是臣在金殿告职归林,为的是汉室锦乾坤。愿吾皇休听宫闱本,普天下黎民享太平。}

\setlength{\hangindent}{56pt}{汉光武\hspace{20pt}【{\akai 二黄原板}】孤离了龙书案,(汉光武{\hwfs 出位站大边},内侍{\hwfs 分下})【{\footnotesize 转}{\akai 二黄慢板}】把皇兄带定,有寡人传口诏细说分明:~都只为那牛邈({\akai 或}:~牛邈贼)屡犯边境,老皇兄去征讨统领雄兵。又谁知叛逆贼甚是狂狞,用诡计将皇兄围困边庭。多亏了小爱卿少年英俊({\akai 或}:~少年英勇),一杆枪救皇兄得胜回京。孤封他平南王({\akai 或}:~靖南王)金殿畅饮,那郭太师在一旁他心怀不平。他二人在金殿结下仇恨,次日里({\akai 或}:~因此上)将太师剑劈府门。(一来是小爱卿刚强逞性,二来是郭太师误国欺君。)今早朝郭娘娘启奏一本,求寡人斩姚刚把父冤伸。孤岂肯行无道曲直不论,(孤岂肯宠嫔妃是非不明({\akai 或}:~宠爱妃虚实不明)。因此上孤传旨亲自细问,)宣皇兄上金殿细问详情({\akai 或}:~细说详情)。想当年孤({\akai 或}:~孤当年;孤登极)也曾把免死牌赠,姚不反汉,汉不斩姚万古留名。孤避难走南阳东逃西奔,白水村遇皇兄扶保寡人({\akai 或}:~老皇兄接驾在白水西村)。孤念你老伯母悬梁自尽,孤念你三年孝改三月,三月孝改三日,三日孝改三时,三时孝改三刻,三刻孝改三分,三年三月三日三时三刻三分永不戴孝未报娘恩。(孤念你为国家心血用尽,孤念你为国家费尽辛勤。)孤念你诛苏献乾坤重整,孤念你灭王莽社稷重兴。孤念你草桥关亲临大阵,孤念你镇边廷受尽辛勤({\akai 或}:~领人马威镇边廷;统雄师威镇边廷;镇边廷费尽辛勤)。孤念你昔年间东挡西除,南征北剿,昼夜杀砍,马不停蹄,到如今鬓发苍苍,你还是忠心耿耿。孤念你是一个开国的老臣。孤念你生三子({\akai 或}:~三个子)二子丧命,孤念你剩姚刚一脉后根。({\hwfs 此二句台上可不唱}:~将姚刚发湖北国法明正,事平后出赦旨再赦他回京。)劝皇兄你且把愁眉展定,劝皇兄你那里宽放忧心({\akai 或}:~但放忧心)。劝皇兄西宫去在娘娘台前把好言奉进,劝皇兄受屈膝({\akai 或}:~受屈情)为的是霸林。适才间卿奏本寡人已准({\akai 或}:~寡人皆允;孤王皆允),(有)寡人戒酒(百日我)不听谗言岂斩你这开国的元勋,孤是一个有道明君。叫一声姚皇兄、子匡、伴驾王,孤的爱卿,休流泪,免悲声,放大胆,一步一步步步随定寡人。(\textless{}\!{\bfseries\akai 长锤}\!\textgreater{}{\hwfs 下})}

\setlength{\hangindent}{56pt}{姚期\hspace{30pt}【{\akai 二黄原板}】光\footnote{《京剧新序》中``光''字误作``建''。}武爷走南阳闯荡四海,闹昆阳众文武聚首起来,宛子城收岑彭邓禹为帅,取洛阳搜云台马武奇才。自盘古哪有臣把君酒戒,这也是老姚期昔年间东荡西除,南征北剿感动了王的心怀,我还怕谁来。}

\setlength{\hangindent}{56pt}{(\textless{}\!{\bfseries\akai 大锣下}\!\textgreater{})}

}
