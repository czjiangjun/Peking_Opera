\newpage
\phantomsection %实现目录的正确跳转
\pagestyle{fancy}    %与文献引用超链接style有冲突
\chead{基本的艺术原则} % 页眉中间位置内容

\section*{\large\hei 基本的艺术原则~\protect\footnote{这是刘曾复先生为林瑞平、杨甲戌二位先生录制说戏录音时的前言,刘曾复先生扼要地介绍了学习和掌握京剧艺术的基本原则和要领,因此特别摘录出来。}}%\protect\hyperlink{fn1}{\textsuperscript{1}}
\addcontentsline{toc}{section}{\hei 基本的艺术原则}

{林、杨二兄,承嘱叫我录一点京剧的老生的说戏的录音。我想啊,我先说一点原则,然后呢,我底下再录。}

{我想啊,唱这个京剧老生跟唱别的行当一样,有所谓一些个基本的``艺术的原则'':}

{在唱、念这方面呢,要懂得四声、三级韵的用法;在用嗓方面呢,要会用``嗽音'';另外呢,在整个这个段子里头呢,要懂得什么叫``整'',就是跟锣鼓要衔接得好。我想就这么四个原则。}

{当然气口啦,像所谓尖团、上口、``十三辙''这类的,其实都包括在这个里边。特别我说一下``十三辙''的问题,你要懂得``十三辙''呢,要懂得``十三辙''跟曲韵的``二十一韵''的关系------曲韵呢,一般呢分成``二十一韵'',也有分成二十二个韵的------到了京剧呢,把这些个字呢都归共到十三个``辙''。从这个里边呢,就可以懂得所谓什么叫尖团、什么叫上口、什么叫收音。实际上都是遵从啊,这个``二十一韵''的这个规律,再加上这个方言,方言的这个习惯。}\footnote{刘曾复先生关于京剧``十三辙''和曲韵``二十一韵''流变关系的论述,具体可参见《京剧新序》第二章``京剧艺术''之第三节``《近代剧韵》''部分。%\protect\hyperlink{fnref2}{↩}
}
%\protect\hyperlink{fn2}{\textsuperscript{2}}

{当然京剧呢,要唱这个所谓谭派、余派老生呢,它这个方言呢,要尊重的,主要还是湖北的音韵,另外呢,湖北的音韵又结合北京人呢,北京话的一些个习惯。这个里面呢,当然还是尊重湖北的习惯。湖北的习惯加上曲韵的规律,这样而来呢,成为京剧的``十三辙'',成为京剧的这个``十三辙''。}

{这里边呢,四声、尖团、收音,还有这个所谓上口,这些个问题,其实要懂得了这个道理就全懂了。我想这个是音韵的问题,这是唱念的问题。}

{身段呢,所谓身段实际上跟武打是一样,``做''、``打''是一个规律。这里边第一要懂得什么叫,``子午相''------腰的问题、知道前后手的问题。另外呢,就得会拉云手,云手有``云手功'',台上呢不用这个``云手功''。云手就,就是所有身段的一个最基本的一个动作。}

{然后呢,如果说是走台步呢,就是要练``起霸'',把``起霸''练懂了,这一切也都懂了,台步也都懂了。}

{再有呢就是所谓这个武打呢,最主要的就是能打,打好了枪的``小五套'',你台上进退、地方也就全都明白了。所以呢,要练这个做、打的基本功呢,就是云手、起霸、再就是``小五套'',这要都弄、打明白了呢,基本功呢也就明白了。在这个基础之上,呃,学身段、学这个把子那就很容易了。}

{诶,我想先说这么一个开头的这个话。}

{如果要这个说,唱跟念,最基本的呢,还是念。你要念,这个字都排好了,啊,会运用了``三级韵'',当然尖团啦、上口啦,这都不说了。要会运用了``三级韵''、会运用了``嗽音'',会运用了这个------把这个``整''搞、啊,掌握得好的话,那你唱散板就不成问题了,散板唱好了的话,上板的就比较容易。}

{这个呢,好像我的印象当中呢,余叔岩,余叔岩先生啊,他这个教别人的时候,也常常先要教这个念白,特别是你像陈少霖啊、孟小冬啊,大概都受过念白的训练,先受他这个念白的训练,然后再学唱,一般呢好像是这么一个规律。}\\
\\
\vspace{7pt}
{\hei 附:~四声三级韵歌词~~~~(摘自《京剧新序》\upcite{Liu_Xinxu-I,Liu_Xinxu-II})}\\%\textsuperscript{{[}2{]}.}{)}
\vspace{3pt}
{\hei 
\\
\hspace*{50pt}~ 韵书有四声~\hspace{40pt} 没有三级韵~\hspace{40pt} 不懂三级韵~\hspace{40pt} 唱念没有味\\ 
\hspace*{50pt}~ 字正腔又圆~\hspace{40pt} 唱念才有味~\hspace{40pt} 运用三级韵~\hspace{40pt} 字正腔才圆\\ 
\hspace*{50pt}~ 四声念得准~\hspace{40pt} 念字才算正~\hspace{40pt} 再用三级韵~\hspace{40pt} 行腔才能圆\\
\hspace*{50pt}~ 阴阳平上去~\hspace{40pt} 一共声有四~\hspace{40pt} 不念入声字~\hspace{40pt} 入归阳上去\\
\hspace*{50pt}~ 四声有高低~\hspace{40pt} 阴平数第一~\hspace{40pt} 上声居其次~\hspace{40pt} 阳平声最低\\
\hspace*{50pt}~ 四声不许变~\hspace{40pt} 每声可高低~\hspace{40pt} 用好三级韵~\hspace{40pt} 四声才不变\\
\hspace*{50pt}~ 两阴上字高~\hspace{40pt} 三阴中间低~\hspace{40pt} 三阳两头低~\hspace{40pt} 阴阳阳可高\\
\hspace*{50pt}~ 二上上字高~\hspace{40pt} 也可下字高~\hspace{40pt} 三上中间低~\hspace{40pt} 也可两头低\\
\hspace*{50pt}~ 四上一三高~\hspace{40pt} 上阴阴字低~\hspace{40pt} 上阳阳可高~\hspace{40pt} 阴上都可低\\
\hspace*{50pt}~ 二去下字高~\hspace{40pt} 三去节节高~\hspace{40pt} 去阳阳可高~\hspace{40pt} 阳去阳可低\\
\hspace*{50pt}~ 四声可高低~\hspace{40pt} 还有八个字~\hspace{40pt} 抑扬与顿挫~\hspace{40pt} 轻重与疾徐\\
\hspace*{50pt}~ 声高就是扬~\hspace{40pt} 声低就是抑~\hspace{40pt} 扬效同重徐~\hspace{40pt} 抑用同轻疾\\
\hspace*{50pt}~ 八字灵活用~\hspace{40pt} 千变又万化~\hspace{40pt} 腔调自然生~\hspace{40pt} 此谓三级韵\\
\hspace*{50pt}~ 灵活妙又巧~\hspace{40pt} 巧中有规矩~\hspace{40pt} 规矩不可离~\hspace{40pt} 规矩不可泥\\
\hspace*{50pt}~ 字有千万个~\hspace{40pt} 腔有千万变~\hspace{40pt} 舍字从腔圆~\hspace{40pt} 舍腔从字正\\
\hspace*{50pt}~ 腔调不可重~\hspace{40pt} 灵活巧运用~\hspace{40pt} 重腔不显重~\hspace{40pt} 全仗三级韵}

%----------------------------------------------------------------------------------------------------------------------------------------------------%
