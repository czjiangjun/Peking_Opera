\newpage
\renewcommand{\thefootnote}{\roman{footnote}}
\setcounter{footnote}{0}
\phantomsection %实现目录的正确跳转
\pagestyle{fancy}    %与文献引用超链接style有冲突
\chead{基本的艺术原则} % 页眉中间位置内容

\section*{\large\hei 基本的艺术原则~\protect\footnote{这是刘曾复先生为林瑞平、杨甲戌二位先生录制说戏录音时的前言,刘曾复先生扼要地介绍了学习和掌握京剧艺术的基本原则和要领,因此特别摘录出来。}}%\protect\hyperlink{fn1}{\textsuperscript{1}}
\addcontentsline{toc}{section}{\hei 基本的艺术原则}

{林、杨二兄,承嘱叫我录一点京剧的老生的说戏的录音。我想啊,我先说一点原则,然后呢,我底下再录。}

{我想啊,唱这个京剧老生跟唱别的行当一样,有所谓一些个基本的``艺术的原则'':}

{在唱、念这方面呢,要懂得四声、三级韵的用法;在用嗓方面呢,要会用``嗽音'';另外呢,在整个这个段子里头呢,要懂得什么叫``整'',就是跟锣鼓要衔接得好。我想就这么四个原则。}

{当然气口啦,像所谓尖团、上口、``十三辙''这类的,其实都包括在这个里边。特别我说一下``十三辙''的问题,你要懂得``十三辙''呢,要懂得``十三辙''跟曲韵的``二十一韵''的关系------曲韵呢,一般呢分成``二十一韵'',也有分成二十二个韵的------到了京剧呢,把这些个字呢都归共到十三个``辙''。从这个里边呢,就可以懂得所谓什么叫尖团、什么叫上口、什么叫收音。实际上都是遵从啊,这个``二十一韵''的这个规律,再加上这个方言,方言的这个习惯。}\footnote{刘曾复先生关于京剧``十三辙''和曲韵``二十一韵''流变关系的论述,具体可参见《京剧新序》第二章``京剧艺术''之第三节``《近代剧韵》''部分。%\protect\hyperlink{fnref2}{↩}
}
%\protect\hyperlink{fn2}{\textsuperscript{2}}

{当然京剧呢,要唱这个所谓谭派、余派老生呢,它这个方言呢,要尊重的,主要还是湖北的音韵,另外呢,湖北的音韵又结合北京人呢,北京话的一些个习惯。这个里面呢,当然还是尊重湖北的习惯。湖北的习惯加上曲韵的规律,这样而来呢,成为京剧的``十三辙'',成为京剧的这个``十三辙''。}

{这里边呢,四声、尖团、收音,还有这个所谓上口,这些个问题,其实要懂得了这个道理就全懂了。我想这个是音韵的问题,这是唱念的问题。}

{身段呢,所谓身段实际上跟武打是一样,``做''、``打''是一个规律。这里边第一要懂得什么叫,``子午相''------腰的问题、知道前后手的问题。另外呢,就得会拉云手,云手有``云手功'',台上呢不用这个``云手功''。云手就,就是所有身段的一个最基本的一个动作。}

{然后呢,如果说是走台步呢,就是要练``起霸'',把``起霸''练懂了,这一切也都懂了,台步也都懂了。}

{再有呢就是所谓这个武打呢,最主要的就是能打,打好了枪的``小五套'',你台上进退、地方也就全都明白了。所以呢,要练这个做、打的基本功呢,就是云手、起霸、再就是``小五套'',这要都弄、打明白了呢,基本功呢也就明白了。在这个基础之上,呃,学身段、学这个把子那就很容易了。}

{诶,我想先说这么一个开头的这个话。}

{如果要这个说,唱跟念,最基本的呢,还是念。你要念,这个字都排好了,啊,会运用了``三级韵'',当然尖团啦、上口啦,这都不说了。要会运用了``三级韵''、会运用了``嗽音'',会运用了这个------把这个``整''搞、啊,掌握得好的话,那你唱散板就不成问题了,散板唱好了的话,上板的就比较容易。}

{这个呢,好像我的印象当中呢,余叔岩,余叔岩先生啊,他这个教别人的时候,也常常先要教这个念白,特别是你像陈少霖啊、孟小冬啊,大概都受过念白的训练,先受他这个念白的训练,然后再学唱,一般呢好像是这么一个规律。}
\newpage
{\centering
\vspace{7pt}
{\hei 
	\hspace*{-115pt}~ {\large 附:~四声三级韵歌词}}~~~~(摘自《京剧新序》\upcite{Liu_Xinxu-I,Liu_Xinxu-II})\\%\textsuperscript{{[}2{]}.}{)}
\vspace{10pt}
{\hei 
\hspace*{10pt}~ 韵书有四声~\hspace{40pt} 没有三级韵~\hspace{40pt} 不懂三级韵~\hspace{40pt} 唱念没有味\\ 
\hspace*{10pt}~ 字正腔又圆~\hspace{40pt} 唱念才有味~\hspace{40pt} 运用三级韵~\hspace{40pt} 字正腔才圆\\ 
\hspace*{10pt}~ 四声念得准~\hspace{40pt} 念字才算正~\hspace{40pt} 再用三级韵~\hspace{40pt} 行腔才能圆\\
\hspace*{10pt}~ 阴阳平上去~\hspace{40pt} 一共声有四~\hspace{40pt} 不念入声字~\hspace{40pt} 入归阳上去\\
\hspace*{10pt}~ 四声有高低~\hspace{40pt} 阴平数第一~\hspace{40pt} 上声居其次~\hspace{40pt} 阳平声最低\\
\hspace*{10pt}~ 四声不许变~\hspace{40pt} 每声可高低~\hspace{40pt} 用好三级韵~\hspace{40pt} 四声才不变\\
\hspace*{10pt}~ 两阴上字高~\hspace{40pt} 三阴中间低~\hspace{40pt} 三阳两头低~\hspace{40pt} 阴阳阳可高\\
\hspace*{10pt}~ 二上上字高~\hspace{40pt} 也可下字高~\hspace{40pt} 三上中间低~\hspace{40pt} 也可两头低\\
\hspace*{10pt}~ 四上一三高~\hspace{40pt} 上阴阴字低~\hspace{40pt} 上阳阳可高~\hspace{40pt} 阴上都可低\\
\hspace*{10pt}~ 二去下字高~\hspace{40pt} 三去节节高~\hspace{40pt} 去阳阳可高~\hspace{40pt} 阳去阳可低\\
\hspace*{10pt}~ 四声可高低~\hspace{40pt} 还有八个字~\hspace{40pt} 抑扬与顿挫~\hspace{40pt} 轻重与疾徐\\
\hspace*{10pt}~ 声高就是扬~\hspace{40pt} 声低就是抑~\hspace{40pt} 扬效同重徐~\hspace{40pt} 抑用同轻疾\\
\hspace*{10pt}~ 八字灵活用~\hspace{40pt} 千变又万化~\hspace{40pt} 腔调自然生~\hspace{40pt} 此谓三级韵\\
\hspace*{10pt}~ 灵活妙又巧~\hspace{40pt} 巧中有规矩~\hspace{40pt} 规矩不可离~\hspace{40pt} 规矩不可泥\\
\hspace*{10pt}~ 字有千万个~\hspace{40pt} 腔有千万变~\hspace{40pt} 舍字从腔圆~\hspace{40pt} 舍腔从字正\\
\hspace*{10pt}~ 腔调不可重~\hspace{40pt} 灵活巧运用~\hspace{40pt} 重腔不显重~\hspace{40pt} 全仗三级韵\\}
}
%----------------------------------------------------------------------------------------------------------------------------------------------------%
%\begin{figure}[!h]
%	\centering
%	\begin{tikzpicture}[
%    box-rect/.style={rectangle,draw=none,node distance=1cm,text width=15em,text centered,rounded corners,minimum height=2em,thick},
%    straightline/.style = {line width = 2pt,-},
%    arrow/.style={draw,-latex', line width = 2pt},
%		]
%		\node [box-rect, text width=2.0em, minimum height=1.5em, yshift=-3.5em, very thin] (Melody) {\fontsize{8.5pt}{2.5pt}\selectfont{唱腔}};
%		\node [box-rect, below=0.2 of Melody, fill=green!40, text width=2.0em, minimum height=1.5em, very thin] (Nature) {\fontsize{8.5pt}{2.5pt}\selectfont{自然}};
%		\node [box-rect, below=0.1 of Nature, fill=green!40, text width=20.0em, minimum height=1.5em, very thin] (Tech-Motion) {\fontsize{8.5pt}{2.5pt}\selectfont{技:~刚、实、宽、厚\hspace{5em}薄、窄、虚、柔:~情}};
%		\node [box-rect, below=0.1 of Tech-Motion, fill=green!40, text width=4.0em, minimum height=1.5em, very thin] (Tan-Xinpei) {\fontsize{8.5pt}{2.5pt}\selectfont{谭鑫培}};
%		\node [box-rect, below=0.1 of Tan-Xinpei, fill=green!40, text width=4.0em, minimum height=1.5em, very thin] (Yang-Xiaolou) {\fontsize{8.5pt}{2.5pt}\selectfont{杨小楼}};
%		\node [box-rect, below=0.85 of Tan-Xinpei, fill=green!40, text width=4.0em, minimum height=1.5em, very thin] (Tan-Xiaopei) {\fontsize{8.5pt}{2.5pt}\selectfont{谭小培}};
%		\node [box-rect, below=0.1 of Tan-Xiaopei, fill=green!40, text width=4.0em, minimum height=1.5em, very thin] (Five-Tan) {\fontsize{8.5pt}{2.5pt}\selectfont{``五坛''}};
%		\node [box-rect, below=0.1 of Five-Tan, fill=green!40, text width=4.0em, minimum height=1.5em, very thin] (Wang-Youchen) {\fontsize{8.5pt}{2.5pt}\selectfont{王又宸}};
%		\node [box-rect, below=0.1 of Wang-Youchen, fill=green!40, text width=4.0em, minimum height=1.5em, very thin] (Chen-Yanheng) {\fontsize{8.5pt}{2.5pt}\selectfont{陈彦衡}};
%		\node [box-rect, below=0.1 of Chen-Yanheng, fill=green!40, text width=10.0em, minimum height=1.5em, very thin] (Guan-Chen) {\fontsize{8.5pt}{2.5pt}\selectfont{贯大元\hspace{2em}陈秀华}};
%		\node [box-rect, below=0.1 of Guan-Chen, fill=green!40, text width=4.0em, minimum height=1.5em, very thin] (Yu-Shuyan) {\fontsize{8.5pt}{2.5pt}\selectfont{余叔岩}};
%		\node [box-rect, below=0.1 of Yu-Shuyan, fill=green!40, text width=4.0em, minimum height=1.5em, very thin] (Yan-Jupeng) {\fontsize{8.5pt}{2.5pt}\selectfont{言菊朋}};
%		\node [box-rect, below=0.1 of Yan-Jupeng, fill=green!40, text width=20.0em, minimum height=1.5em, very thin] (Double-Wang-Li-Zhang) {\fontsize{8.5pt}{2.5pt}\selectfont{王君直\hspace{2em}王荣山\hspace{2em}李适可\hspace{2em}张伯驹}};
%		\node [box-rect, below=0.1 of Double-Wang-Li-Zhang, fill=green!40, text width=10.0em, minimum height=1.5em, very thin] (Fan-Han) {\fontsize{8.5pt}{2.5pt}\selectfont{范濂泉\hspace{2em}韩慎先}};
%		\node [box-rect, below=0.1 of Fan-Han, fill=green!40, text width=4.0em, minimum height=1.5em, very thin] (Yang-Baosen) {\fontsize{8.5pt}{2.5pt}\selectfont{杨宝森}};
%		\node [box-rect, below=0.1 of Yang-Baosen, fill=green!40, text width=4.0em, minimum height=1.5em, very thin] (Meng-Xiaodong) {\fontsize{8.5pt}{2.5pt}\selectfont{孟小冬}};
%		\node [box-rect, below=0.1 of Meng-Xiaodong, fill=green!40, text width=4.0em, minimum height=1.5em, very thin] (Xi-Xiaobo) {\fontsize{8.5pt}{2.5pt}\selectfont{奚啸伯}};
%
%\path
%(Nature.west)++(-7.5, 0.0) coordinate(Nature-)
%(Nature.east)++(7.5, 0.0) coordinate(Nature+)
%(Tech-Motion.west)++(-3.8, 0.0) coordinate(Tech-Motion-)
%(Tech-Motion.east)++(3.8, 0.0) coordinate(Tech-Motion+)
%
%(Tan-Xinpei.west)++(-7.1, 0.0) coordinate(Tan-Xinpei-)
%(Tan-Xinpei.east)++(7.1, 0.0) coordinate(Tan-Xinpei+)
%(Yang-Xiaolou.west)++(-6.9, 0.0) coordinate(Yang-Xiaolou-)
%(Yang-Xiaolou.east)++(6.9, 0.0) coordinate(Yang-Xiaolou+)
%(Tan-Xiaopei.west)++(-4.0, 0.0) coordinate(Tan-Xiaopei-)
%(Tan-Xiaopei.east)++(4.0, 0.0) coordinate(Tan-Xiaopei+)
%(Five-Tan.west)++(-5.0, 0.0) coordinate(Five-Tan-)
%(Five-Tan.east)++(5.0, 0.0) coordinate(Five-Tan+)
%(Wang-Youchen.west)++(-5.5, 0.0) coordinate(Wang-Youchen-)
%(Wang-Youchen.east)++(5.5, 0.0) coordinate(Wang-Youchen+)
%(Chen-Yanheng.west)++(-6.0, 0.0) coordinate(Chen-Yanheng-)
%(Chen-Yanheng.east)++(6.0, 0.0) coordinate(Chen-Yanheng+)
%(Guan-Chen.west)++(-4.5, 0.0) coordinate(Guan-Chen-)
%(Guan-Chen.east)++(4.5, 0.0) coordinate(Guan-Chen+)
%
%(Yu-Shuyan.west)++(-7.5, 0.0) coordinate(Yu-Shuyan-)
%(Yu-Shuyan.east)++(5.8, 0.0) coordinate(Yu-Shuyan+)
%(Yan-Jupeng.west)++(-5.5, 0.0) coordinate(Yan-Jupeng-)
%(Yan-Jupeng.east)++(7.5, 0.0) coordinate(Yan-Jupeng+)
%(Double-Wang-Li-Zhang.west)++(-3.2, 0.0) coordinate(Double-Wang-Li-Zhang-)
%(Double-Wang-Li-Zhang.east)++(2.2, 0.0) coordinate(Double-Wang-Li-Zhang+)
%(Fan-Han.west)++(-4.3, 0.0) coordinate(Fan-Han-)
%(Fan-Han.east)++(5.3, 0.0) coordinate(Fan-Han+)
%(Yang-Baosen.west)++(-6.85, 0.0) coordinate(Yang-Baosen-)
%(Yang-Baosen.east)++(5.5, 0.0) coordinate(Yang-Baosen+)
%(Meng-Xiaodong.west)++(-7.2, 0.0) coordinate(Meng-Xiaodong-)
%(Meng-Xiaodong.east)++(5.5, 0.0) coordinate(Meng-Xiaodong+)
%(Xi-Xiaobo.west)++(-5.3, 0.0) coordinate(Xi-Xiaobo-)
%(Xi-Xiaobo.east)++(6.85, 0.0) coordinate(Xi-Xiaobo+);
%
%		\draw [dotted, very thick](-8, -13.5) -- (-8, -1);
%		\draw [dotted, very thick](8, -13.5) -- (8, -1);
%		\draw [straightline, dashed, thin](Nature.west) -- (Nature-);
%		\draw [straightline, dashed, thin](Nature.east) -- (Nature+);
%		\path [arrow, NavyBlue, thick](Tech-Motion.west) -- (Tech-Motion-);
%		\path [arrow, NavyBlue, thick](Tech-Motion.east) -- (Tech-Motion+);
%
%		\draw [solid, ultra thick](-8, -3.5) -- (-8, -3.9);
%		\draw [solid, ultra thick](8, -3.5) -- (8, -3.9);
%		\draw [straightline, dash dot, thin](Tan-Xinpei.west) -- (Tan-Xinpei-);
%		\draw [straightline, dash dot, thin](Tan-Xinpei.east) -- (Tan-Xinpei+);
%
%		\draw [solid, red, very thick](-7.9, -4.25) -- (-7.9, -4.65);
%		\draw [solid, red, very thick](7.9, -4.25) -- (7.9, -4.65);
%		\draw [straightline, dash dot dot, thin](Yang-Xiaolou.west) -- (Yang-Xiaolou-);
%		\draw [straightline, dash dot dot, thin](Yang-Xiaolou.east) -- (Yang-Xiaolou+);
%
%		\draw [solid, very thick](-5, -5) -- (-5, -5.4);
%		\draw [solid, very thick](5, -5) -- (5, -5.4);
%		\draw [straightline, dash dot, thin](Tan-Xiaopei.west) -- (Tan-Xiaopei-);
%		\draw [straightline, dash dot, thin](Tan-Xiaopei.east) -- (Tan-Xiaopei+);
%
%		\draw [solid, very thick](-6, -5.75) -- (-6, -6.15);
%		\draw [solid, very thick](6, -5.75) -- (6, -6.15);
%		\draw [straightline, dash dot, thin](Five-Tan.west) -- (Five-Tan-);
%		\draw [straightline, dash dot, thin](Five-Tan.east) -- (Five-Tan+);
%
%		\draw [solid, very thick](-6.5, -6.45) -- (-6.5, -6.85);
%		\draw [solid, very thick](6.5, -6.45) -- (6.5, -6.85);
%		\draw [straightline, dash dot, thin](Wang-Youchen.west) -- (Wang-Youchen-);
%		\draw [straightline, dash dot, thin](Wang-Youchen.east) -- (Wang-Youchen+);
%
%		\draw [solid, very thick](-7.0, -7.15) -- (-7.0, -7.55);
%		\draw [solid, very thick](7.0, -7.15) -- (7.0, -7.55);
%		\draw [straightline, dash dot, thin](Chen-Yanheng.west) -- (Chen-Yanheng-);
%		\draw [straightline, dash dot, thin](Chen-Yanheng.east) -- (Chen-Yanheng+);
%
%		\draw [solid, very thick](-6.8, -7.9) -- (-6.8, -8.3);
%		\draw [solid, very thick](6.8, -7.9) -- (6.8, -8.3);
%		\draw [straightline, dash dot, thin](Guan-Chen.west) -- (Guan-Chen-);
%		\draw [straightline, dash dot, thin](Guan-Chen.east) -- (Guan-Chen+);
%
%		\draw [solid, very thick](-8.5, -8.65) -- (-8.5, -9.05);
%		\draw [solid, very thick](6.8, -8.65) -- (6.8, -9.05);
%		\draw [straightline, dash dot, thin](Yu-Shuyan.west) -- (Yu-Shuyan-);
%		\draw [straightline, dash dot, thin](Yu-Shuyan.east) -- (Yu-Shuyan+);
%
%		\draw [solid, very thick](-6.5, -9.38) -- (-6.5, -9.78);
%		\draw [solid, very thick](8.5, -9.38) -- (8.5, -9.78);
%		\draw [straightline, dash dot, thin](Yan-Jupeng.west) -- (Yan-Jupeng-);
%		\draw [straightline, dash dot, thin](Yan-Jupeng.east) -- (Yan-Jupeng+);
%
%		\draw [solid, very thick](-7.5, -10.08) -- (-7.5, -10.48);
%		\draw [solid, very thick](6.4, -10.08) -- (6.4, -10.48);
%		\draw [straightline, dash dot, thin](Double-Wang-Li-Zhang.west) -- (Double-Wang-Li-Zhang-);
%		\draw [straightline, dash dot, thin](Double-Wang-Li-Zhang.east) -- (Double-Wang-Li-Zhang+);
%
%		\draw [solid, very thick](-6.4, -10.81) -- (-6.4, -11.21);
%		\draw [solid, very thick](7.5, -10.81) -- (7.5, -11.21);
%		\draw [straightline, dash dot, thin](Fan-Han.west) -- (Fan-Han-);
%		\draw [straightline, dash dot, thin](Fan-Han.east) -- (Fan-Han+);
%
%		\draw [solid, very thick](-7.8, -11.55) -- (-7.8, -11.95);
%		\draw [solid, very thick](6.5, -11.55) -- (6.5, -11.95);
%		\draw [straightline, dash dot, thin](Yang-Baosen.west) -- (Yang-Baosen-);
%		\draw [straightline, dash dot, thin](Yang-Baosen.east) -- (Yang-Baosen+);
%
%		\draw [solid, very thick](-8.2, -12.28) -- (-8.2, -12.68);
%		\draw [solid, very thick](6.5, -12.28) -- (6.5, -12.68);
%		\draw [straightline, dash dot, thin](Meng-Xiaodong.west) -- (Meng-Xiaodong-);
%		\draw [straightline, dash dot, thin](Meng-Xiaodong.east) -- (Meng-Xiaodong+);
%
%		\draw [solid, very thick](-6.3, -13.00) -- (-6.3, -13.40);
%		\draw [solid, very thick](7.8, -13.00) -- (7.8, -13.40);
%		\draw [straightline, dash dot, thin](Xi-Xiaobo.west) -- (Xi-Xiaobo-);
%		\draw [straightline, dash dot, thin](Xi-Xiaobo.east) -- (Xi-Xiaobo+);
%\end{tikzpicture}
%%	\includegraphics{<+file+>}
%%	\caption{<+caption text+>}
%	\label{fig:Broad_Spectrum-Analysis}
%\end{figure}
