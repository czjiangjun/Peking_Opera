\newpage
\subsubsection{\large\hei {镇潭州~{\small 之}~岳飞、杨璟}}
\addcontentsline{toc}{subsection}{\hei 镇潭州~{\small 之}~岳飞、杨璟}

\hangafter=1                   %2. 设置从第1⾏之后开始悬挂缩进  %}
\setlength{\parindent}{0pt}{

\vspace{3pt}{\centerline{{[}{\hei 第一场}{]}}}\vspace{5pt}

\setlength{\hangindent}{56pt}{岳飞\hspace{30pt}\textless{}\!{\bfseries\akai 点绛唇}\!\textgreater{}报国精忠,虎啸龙吟;~迎二圣,扫荡烟尘,保主锦绣春。}

\setlength{\hangindent}{56pt}{岳飞\hspace{30pt}({\akai 念})旌旗招展出禁城,武将心思汗马勋。剖心要尽凌云志,迎回二圣方称心。}

\setlength{\hangindent}{56pt}{岳飞\hspace{30pt}本帅,姓岳名飞字鹏举,宋室驾前为臣。只因奸佞当道,张邦昌陷二圣于沙漠,坐井观天。是我退归林下;~今蒙太后二次诏宣,官拜天下都招讨、兵马大元帅;~后宫娘娘恩赐五色锦旗,亲绣``精忠报国''。}

\setlength{\hangindent}{56pt}{岳飞\hspace{30pt}本帅亲承王命,统领六师,扫荡烟尘,恢复河山。今乃黄道吉日,正好兴兵。众位贤弟!}

\setlength{\hangindent}{56pt}{岳飞\hspace{30pt}人马可齐?}

\setlength{\hangindent}{56pt}{岳飞\hspace{30pt}香案伺候!}

\setlength{\hangindent}{56pt}{岳飞\hspace{30pt}({\akai 念})~祝告:~山川社稷、万里旗纛尊神:~信官岳飞,今奉圣命,扫荡九龙山杨再兴、长沙王罗延庆、洞庭湖水贼杨幺等。但愿此行,旗开得胜!}\footnote{{据陈超老师介绍:~}奠酒上丑司仪,圆纱、白四喜、青褶子(不穿官衣),上三炷香、三叩首。很特别。}

\setlength{\hangindent}{56pt}{岳飞\hspace{30pt}打道出府。}\footnote{``{打道出府}''一句,陈超老师从刘曾复先生学的是``{撤去香案}''。省事。}

\setlength{\hangindent}{56pt}{岳飞\hspace{30pt}牛皋听令。}

\setlength{\hangindent}{56pt}{岳飞\hspace{30pt}命你去到潭州晓谕节度使,命他高垒城郭,本帅大兵随后就到。}

\setlength{\hangindent}{56pt}{岳飞\hspace{30pt}转来。}

\setlength{\hangindent}{56pt}{岳飞\hspace{30pt}岳云听令。}

\setlength{\hangindent}{56pt}{岳飞\hspace{30pt}解押粮草}

\setlength{\hangindent}{56pt}{岳飞\hspace{30pt}张宪听令。}

\setlength{\hangindent}{56pt}{岳飞\hspace{30pt}催运粮草}

\setlength{\hangindent}{56pt}{岳飞}\hspace{40pt}张保{听令。}

\setlength{\hangindent}{56pt}{岳飞\hspace{30pt}命你以为总督粮官。}

\setlength{\hangindent}{56pt}{岳飞\hspace{30pt}吉青听令。}

\setlength{\hangindent}{56pt}{岳飞\hspace{30pt}随营护卫。}

\setlength{\hangindent}{56pt}{岳飞\hspace{30pt}施全听令。}

\setlength{\hangindent}{56pt}{岳飞\hspace{30pt}命你总督三军。传令下去,众将一路之上,不可马踏青苗,扰害百姓,违令者斩!}

\setlength{\hangindent}{56pt}{岳飞\hspace{30pt}就此起兵潭州。}

\vspace{3pt}{\centerline{{[}{\hei 第二场}{]}}}\vspace{5pt}

\setlength{\hangindent}{56pt}{岳飞\hspace{30pt}哦,恩师!}

\setlength{\hangindent}{56pt}{岳飞\hspace{30pt}不敢,恩师请!}

\setlength{\hangindent}{56pt}{岳飞\hspace{30pt}门生放肆。}

\setlength{\hangindent}{56pt}{岳飞\hspace{30pt}门生有何德能,敢劳恩师迎接十里之外?}

\setlength{\hangindent}{56pt}{岳飞\hspace{30pt}惶恐啊惶恐!}

\setlength{\hangindent}{56pt}{岳飞\hspace{30pt}我命牛皋前来,为何不见?}

\setlength{\hangindent}{56pt}{岳飞\hspace{30pt}哦,牛皋至此,未憩鞍马,径自立功去了?}

\setlength{\hangindent}{56pt}{岳飞\hspace{30pt}嗯------想他此去,必然是大败而归。}

\setlength{\hangindent}{56pt}{岳飞\hspace{30pt}贤弟,你与敌人交战,胜负如何?}

\setlength{\hangindent}{56pt}{岳飞\hspace{30pt}怎么样?}

\setlength{\hangindent}{56pt}{岳飞\hspace{30pt}可曾问过敌人的名姓?}

\setlength{\hangindent}{56pt}{岳飞\hspace{30pt}呃------你跟随愚兄出兵多年,还是这样粗鲁,倘若得胜而归,教愚兄怎上功劳簿。}

\setlength{\hangindent}{56pt}{岳飞\hspace{30pt}敢是那杨再兴?}

\setlength{\hangindent}{56pt}{岳飞\hspace{30pt}此人英勇无敌,你岂是他人对手,待本帅亲自会他。}

\setlength{\hangindent}{56pt}{岳飞\hspace{30pt}众位贤弟!~有所不知,想那杨再兴,乃是将门之子,名门之后,武艺高强,本帅意欲,将他收留帐下,做一膀臂。今日出马,非比寻常,众将只许观阵,不许助战,违令者斩!}

\setlength{\hangindent}{56pt}{岳飞\hspace{30pt}恩师不必拦阻,待门生先见一阵。}

\setlength{\hangindent}{56pt}{岳飞\hspace{30pt}就烦恩师谨守城池。}

\setlength{\hangindent}{56pt}{岳飞\hspace{30pt}众将官,带马迎敌者。}

\setlength{\hangindent}{56pt}{岳飞\hspace{30pt}杨将军,别来无恙!}

\setlength{\hangindent}{56pt}{岳飞\hspace{30pt}杨将军,想那年在汴梁小校场,会过一面,难道将军你就忘怀了?}

\setlength{\hangindent}{56pt}{岳飞\hspace{30pt}然也。}

\setlength{\hangindent}{56pt}{岳飞\hspace{30pt}杨将军,想你乃是将门之子,忠良之后,因甚事失身落草,岂不玷辱杨氏祖先?听本帅相劝,归顺皇朝,共灭金寇,不失封侯之位,将军三思。}

\setlength{\hangindent}{56pt}{岳飞\hspace{30pt}住口!好言相劝,执意不听,少时擒在马前,悔之晚矣!}

\setlength{\hangindent}{56pt}{岳飞\hspace{30pt}决一胜负。}

\setlength{\hangindent}{56pt}{岳飞\hspace{30pt}这个$\cdots{}\cdots{}$杨将军,俺若不胜,情愿将潭州奉让。}

\setlength{\hangindent}{56pt}{岳飞\hspace{30pt}杨将军,你我今日交战,非比寻常,必须一对一个;~两下各传将令,众将只许观阵,不许助战,违令者斩。}

\setlength{\hangindent}{56pt}{岳飞\hspace{30pt}军令不严非为丈夫也。}

\setlength{\hangindent}{56pt}{岳飞\hspace{30pt}\textless{}\!{\bfseries\akai 叫头}\!\textgreater{}众将官!}

\setlength{\hangindent}{56pt}{岳飞\hspace{30pt}只许观阵,不许助战,违令者斩!}

\setlength{\hangindent}{56pt}{岳飞\hspace{30pt}【{\akai 西皮小导板}】叫三军与爷战鼓操,}

\setlength{\hangindent}{56pt}{岳飞\hspace{30pt}【{\akai 西皮快板}】马前闪出一英豪。杨家世代把国保,因何埋名在山巢。劝你马前归顺好,封妻荫子永在朝。}

[上手(岳飞){\hwfs 大边},{\hwfs 一扯两扯},{\hwfs 幺二三往外把盖}下手(杨再兴){\hwfs 枪左转身}(下手{\hwfs 右转身}){\hwfs 到里边打一个腰封}、{\hwfs 两个腰封},{\hwfs 被往里面盖右转身}(下手{\hwfs 左转身}){\hwfs 到外边},{\hwfs 接一个腰封}、{\hwfs 两个腰封},{\hwfs 把盖撤枪},{\hwfs 撤右脚斜向上场门},{\hwfs 上左脚刺在上场边里面的}下手{\hwfs 左右两马腿},{\hwfs 左刺耳}\footnote{陈超老师注:~此处{\hei 老生左刺耳是两枪,小生是刺一枪}。刘先生当初稿里写了两枪,《戏曲艺术》发表时漏排。要是一枪小生倒不过抢来,老生刺两枪,小生枪鐏接一个,剜花,枪头接一个。},{\hwfs 被}下手{\hwfs 盖},{\hwfs 撤枪撤左腿在下场门边外面接}下手{\hwfs 两个刺马腿},{\hwfs 盖}下手{\hwfs 左刺耳},{\hwfs 搭}、{\hwfs 拉归里边面外}、下手{\hwfs 面里},{\hwfs 搭}、{\hwfs 兜转身过到小边},{\hwfs 面对过大边的}下手{\hwfs 掣肘},{\hwfs 撤枪}两人{\hwfs 对脸左右左三个刺马腿},{\hwfs 一二三绕}、{\hwfs 边绕边走从外面过到大边},{\hwfs 一二三绕},{\hwfs 边绕边走从外面过到小边},下手{\hwfs 向内侧刺上手马腿}、上手{\hwfs 挑起向里刺肚},{\hwfs 从里边左转身到外边向外侧刺}下手{\hwfs 马腿}、下手{\hwfs 挑起来刺肚},下手{\hwfs 直着过到小边},上手{\hwfs 接刺肚向右反转身从里边过到大边},二人{\hwfs 合身往里一盖两盖},上手{\hwfs 手平伸扎}下手{\hwfs 一枪左转身}(下手左手{\hwfs 拿枪滑}上手{\hwfs 扎出的枪}、{\hwfs 右转身右手掏翎},{\hwfs 送到嘴叼翎},{\hwfs 枪交右手)},上手{\hwfs 左手捋胡子}、{\hwfs 跨右腿}、{\hwfs 左转身又扎}下手{\hwfs 一枪(}下手{\hwfs 右手枪滑}上手{\hwfs 枪},{\hwfs 右转身左手掏翎子}),上手{\hwfs 转过来扔胡子}、{\hwfs 枪收回来平托}、{\hwfs 左手山膀},{\hwfs 大边里边站斜向外亮住}(下手{\hwfs 转过来右手枪剜萝卜}、{\hwfs 右手伸出}、{\hwfs 枪头斜向下},{\hwfs 左手拿翎横胸前},{\hwfs 弓箭步外边站斜向里亮住})]\footnote{这是{\hei 刘砚芳介绍的杨小楼的《镇潭州》中岳飞与杨再兴开打的``枪架子''头场的打法}。杨的岳飞学谭鑫培,所用的把子能显出大将风度,马上交锋,彼此较量,稳重沉着,棋逢对手,合乎剧情。  

陈超老师补注:~杨小楼跟梅兰芳唱过《镇潭州》,也打这套把子,是杨小楼教的。程继先、杨小楼没唱过这出戏,程继先从没搭过杨小楼班,也不单独与杨合作唱戏,只是在窝窝头会或堂会偶尔合作大义务戏。}

[{\hwfs 拉上}、{\hwfs 斜亮},{\hwfs 到台口正亮},{\hwfs 一二三夺换位亮},{\hwfs 一二三夺分开},{\hwfs 一合两合},岳飞{\hwfs 归小边},{\hwfs 幺二三岳被勾走马腰封到大边}、{\hwfs 再被一压}、{\hwfs 被漫头左转身到中间面向里一别}(杨{\hwfs 同时归中间里面向外一别}),岳飞{\hwfs 撤枪向里面斜刺}、{\hwfs 刺空}(杨{\hwfs 出枪贴}岳{\hwfs 背扎脖}),岳飞{\hwfs 左转身用枪杆把}杨{\hwfs 枪搕出去}、{\hwfs 捋胡子下},杨{\hwfs 望}岳{\hwfs 捋枪向外望}、{\hwfs 斜托枪亮住},{\hwfs 耍下场追下}]\footnote{这是岳飞失利下的打法。{\hei 特别值得一提的是:~《镇潭州》戏中会战的唱和开打里 ,传统灵活地用\textless{}\!{\bfseries\akai 战场}\!\textgreater{}锣鼓,适合人物和戏情。}}

\setlength{\hangindent}{56pt}{岳飞\hspace{30pt}绑了!}

\vspace{3pt}{\centerline{{[}{\hei 第三场}{]}}}\vspace{5pt}

\setlength{\hangindent}{56pt}{岳飞\hspace{30pt}杨将军,你我再决胜负。}

\setlength{\hangindent}{56pt}{岳飞\hspace{30pt}回营!}

\vspace{3pt}{\centerline{{[}{\hei 第四场}{]}}}\vspace{5pt}

\setlength{\hangindent}{56pt}{岳飞\hspace{30pt}将岳云绑了上来!}

\setlength{\hangindent}{56pt}{岳飞\hspace{30pt}小奴才,何人教你出马?何人教你出马?}

\setlength{\hangindent}{56pt}{岳飞\hspace{30pt}大胆奴才!想那杨再兴,乃是将门之后,为父指望收服于他,作为膀臂。故而不许旁人助战,你众位叔父都不敢违抗为父的将令,惟有你这小畜生,你敢犯我的军规吗?}

\setlength{\hangindent}{56pt}{岳飞\hspace{30pt}斩!}

\setlength{\hangindent}{56pt}{岳飞\hspace{30pt}众位贤弟,敢是与奴才讲情?}

\setlength{\hangindent}{56pt}{岳飞\hspace{30pt}可知本帅令出山岳动,这言发------神鬼惊!}

\setlength{\hangindent}{56pt}{岳飞\hspace{30pt}斩!}

\setlength{\hangindent}{56pt}{岳飞\hspace{30pt}\textless{}\!{\bfseries\akai 叫头}\!\textgreater{}岳云,奴才!}

\setlength{\hangindent}{56pt}{岳飞\hspace{30pt}怎么你要回去见你那祖母、娘亲么?}

\setlength{\hangindent}{56pt}{岳飞\hspace{30pt}掌起面来!}

\setlength{\hangindent}{56pt}{岳飞\hspace{30pt}\textless{}{\!\bfseries\akai 三叫头}\!\textgreater{}岳云,娇儿,唉,儿啊!}

\setlength{\hangindent}{56pt}{岳飞\hspace{30pt}为父今日要将儿斩首,怎么你要回去见你那祖母、娘亲么?}

\setlength{\hangindent}{56pt}{岳飞\hspace{30pt}嗯,只怕儿今生今世就不能相见了。}

\setlength{\hangindent}{56pt}{岳飞\hspace{30pt}斩!}

\setlength{\hangindent}{56pt}{岳飞\hspace{30pt}赦了。}

\setlength{\hangindent}{56pt}{岳飞\hspace{30pt}将岳云带了上来!}

\setlength{\hangindent}{56pt}{岳飞\hspace{30pt}奴才,本当将你斩首,念在你众位叔父苦苦讲情,死罪已免,活罪难容。}

\setlength{\hangindent}{56pt}{岳飞\hspace{30pt}牢子手,将奴才重责四十!}

\setlength{\hangindent}{56pt}{岳飞\hspace{30pt}张保听令!}

\setlength{\hangindent}{56pt}{岳飞\hspace{30pt}命你押解岳云,去到杨再兴营盘,对他言讲:~岳云解粮在先,本帅传令在后,不知有此军令,在阵前冒犯将军,回营就要斩首;~多亏满营将官讲情,死罪已免,活罪难容,重责四十,请将军验伤。上覆杨将军,明日还在阵前相会。掩门!}

\vspace{3pt}{\centerline{{[}{\hei 第五场}{]}}}\vspace{5pt}

\setlength{\hangindent}{56pt}{杨璟\hspace{30pt}({\akai 念})生前为大将,死后做忠魂。}

\setlength{\hangindent}{56pt}{杨璟\hspace{30pt}吾乃杨璟阴魂是也。今有孙男再兴,落草为寇。岳元帅难以收服,我不免去至宋营,梦中授他撒手金锏,助他成功。}

\setlength{\hangindent}{56pt}{杨璟\hspace{30pt}鬼卒,宋营去者。}

\setlength{\hangindent}{56pt}{杨璟\hspace{30pt}【{\akai 二黄原板}】我杨家祖居在梅花山后,老王爷锤换带才把宋投。都只为再兴儿落草为寇,岳元帅无良谋难把他收。教鬼卒前引路宋营来走,见了那岳元帅细说从头。}

\vspace{3pt}{\centerline{{[}{\hei 第六场}{]}}}\vspace{5pt}

\setlength{\hangindent}{56pt}{岳飞\hspace{30pt}【{\akai 二黄原板}】清晨起打一仗龙争虎斗,战不过杨再兴脸面惭羞。在虎帐传一令严加防守,迎二圣我才得展放眉头。}

\setlength{\hangindent}{56pt}{杨璟\hspace{30pt}【{\akai 二黄摇板}】听谯楼打罢了三更时候,到宋营见元帅细说根由。}

\setlength{\hangindent}{56pt}{岳飞\hspace{30pt}请问老先生尊姓大名,家住哪里,来到我营有何贵干?}

\setlength{\hangindent}{56pt}{杨璟\hspace{30pt}老夫祖居磁州梅花山后,杨璟是也。}

\setlength{\hangindent}{56pt}{岳飞\hspace{30pt}哦,原来是前辈老先生,失敬了。老先生有何见谕?}

\setlength{\hangindent}{56pt}{岳飞\hspace{30pt}元帅受我一礼。}

\setlength{\hangindent}{56pt}{岳飞\hspace{30pt}老先生施礼为何?}

\setlength{\hangindent}{56pt}{杨璟\hspace{30pt}只因孙儿再兴,不幸失身落草,还望元帅加以收服。}

\setlength{\hangindent}{56pt}{岳飞\hspace{30pt}本帅倒有此意,怎奈再兴武艺高强,难以收服。}

\setlength{\hangindent}{56pt}{杨璟\hspace{30pt}杨家梅花枪暗藏撒手锏,待老夫传授与你。}

\setlength{\hangindent}{56pt}{岳飞\hspace{30pt}领教了!}

\setlength{\hangindent}{56pt}{杨璟\hspace{30pt}【{\akai 二黄摇板}】我杨家梅花枪暗藏撒手,}

\setlength{\hangindent}{56pt}{岳飞\hspace{30pt}【{\akai 二黄摇板}】老先生秉忠心万古名留。}

\setlength{\hangindent}{56pt}{杨璟\hspace{30pt}【{\akai 二黄摇板}】但愿得收服他鞍前马后,}

\setlength{\hangindent}{56pt}{岳飞\hspace{30pt}【{\akai 二黄摇板}】他本是将门子啊必定封侯。}

\setlength{\hangindent}{56pt}{杨璟\hspace{30pt}【{\akai 二黄摇板}】哗喇喇打开了玲珑甲胄,}

\setlength{\hangindent}{56pt}{(众\hspace{40pt}$\cdots{}\cdots{}$醒来。)}

\setlength{\hangindent}{56pt}{岳飞\hspace{30pt}【{\akai 二黄散板}】多蒙你进帐来枪锏传授。猛然间又只见红日当头。}

\setlength{\hangindent}{56pt}{(报子\hspace{30pt}再兴讨战。)}

\setlength{\hangindent}{56pt}{岳飞\hspace{30pt}带马阵前去者。}

\setlength{\hangindent}{56pt}{岳飞\hspace{30pt}杨将军,昨日小儿阵前多有冒犯!}

\setlength{\hangindent}{56pt}{岳飞\hspace{30pt}岂敢。你我今日再决胜负。}

\setlength{\hangindent}{56pt}{岳飞\hspace{30pt}话出不悔,真丈夫也。放马过来。}

(岳飞、杨再兴{\hwfs 开打})

\setlength{\hangindent}{56pt}{岳飞\hspace{30pt}杨将军,本帅失手了。}

\setlength{\hangindent}{56pt}{岳飞\hspace{30pt}弃暗投明,真乃俊杰也。欲与将军结为金兰,万勿见却?}

\setlength{\hangindent}{56pt}{岳飞\hspace{30pt}不必推辞,你我望空一拜。}

\setlength{\hangindent}{56pt}{岳飞\hspace{30pt}不必查点,兵合一处。}

\setlength{\hangindent}{56pt}{岳飞\hspace{30pt}众将官,同进潭州!}

}
