\newpage
\subsubsection{{\hei\large 焚烟墩·挡幽}~\protect\footnote{本剧的舞台调度与人物扮相参考了吴焕老师整理的剧本(经刘曾复先生审订)。}}%\protect\hyperlink{fn8}{\textsuperscript{8}}
\addcontentsline{toc}{subsection}{\hei 焚烟墩·挡幽}

\hangafter=1                   %2. 设置从第1⾏之后开始悬挂缩进  %
\setlength{\parindent}{0pt}{
	\textrm{(}【{\akai 大锣长锤}】,{\hwfs 四绿}龙套{\hwfs 手拿月华旗},{\hwfs 两个}旗牌,{\hwfs 上},{\hwfs 站门}\textrm{)}

	(申侯{\hwfs 上},\textrm{``}{\hwfs 九龙口}\textrm{''}{\hwfs 旁},{\hwfs 起唱}\textrm{)}

\textrm{申侯\hspace{30pt}【{\akai 西皮三眼}】周天子行无道难逃我手,}

(申侯{\hwfs 至台中央},{\hwfs 接唱},\textrm{``}{\bfseries\hwfs 扯四门}\textrm{''})

\setlength{\hangindent}{56pt}{   %3. 设置悬挂缩进量                %
\textrm{申侯\hspace{30pt}【{\akai 西皮三眼}】先帝呀爷他封我申地为侯。恨昏王废太子贬了申后,因此上领人马扎至山头({\akai 或}:~扎至山口)。将人马埋伏在三岔路口哇,}
}

\textrm{申侯\hspace{30pt}【{\akai 西皮摇板}】等昏王到此来细问根由。}

\textrm{申侯\hspace{30pt}大路埋伏,小路放起烟火,君妃到此({\akai 或}:~昏王到此),速报我知。}

\textrm{众\hspace{41pt}啊!}

(申侯{\hwfs 下},{\hwfs 四个}龙套{\hwfs 拿月华旗在后场},{\hwfs 拉起成阵墙})

(幽王、褒姒{\hwfs 上})

\textrm{幽王\hspace{30pt}【{\akai 西皮导板}】直杀得气冲天如同白昼,}

\textrm{幽王\hspace{30pt}哎呀({\akai 或}:~罢)!}

\textrm{(幽王{\hwfs 和}褒姒{\hwfs 到小边},幽王{\hwfs 要摔倒},褒姒{\hwfs 搀扶})}

\textrm{褒姒\hspace{30pt}万岁怎么样了?}

\textrm{幽王\hspace{30pt}唉!}

\textrm{(褒姒{\hwfs 搀}幽王,幽王{\hwfs 接唱},{\hwfs 在月华旗前}``}{\bfseries\hwfs 扯四门}\textrm{'')}

\textrm{幽王\hspace{30pt}【{\akai 西皮原板}】遭不幸尸堆山血水倒流。}

\textrm{(幽王{\hwfs 至小边})}

\setlength{\hangindent}{56pt}{   %3. 设置悬挂缩进量                %
\textrm{幽王\hspace{30pt}【{\akai 西皮原板}】千层浪翻身转慈航搭救,王虽然得活命}龙落孤洲\textrm{。皆因是孤有错说不出口,都只为你进宫起祸根由。悔不该听奸臣【{\footnotesize 转}{\akai 西皮快板}】谗言启奏,悔不该把朝事一旦抛丢。悔不该废太子又贬申后,悔不该与国舅结下冤仇。悔不该在骊山【{\footnotesize 转}{\akai 西皮快板}】取乐饮酒,悔不该焚烟墩戏耍诸侯。看起来孤的命------}
}

\textrm{幽王\hspace{30pt}嘿------呀!}

\textrm{幽王\hspace{30pt}【{\akai 西皮摇板}】只怕是断送在你手,}

\textrm{褒姒\hspace{30pt}万岁命丧我手({\akai 或}:~万岁命丧奴手),唉!待我死了罢!}

\textrm{幽王\hspace{30pt}唉呀嚯,舍不得哟! ({\akai 或}:~慢来慢来,还舍不得哟!)}

\textrm{(幽王、褒姒{\hwfs 至小边})}

\setlength{\hangindent}{56pt}{   %3. 设置悬挂缩进量                %
\textrm{幽王\hspace{30pt}【{\akai 西皮摇板}】劝梓童休伤悲且免忧愁。被西戎杀得孤无处逃呃走,三岔口迷了路不能相投。}
}

\textrm{幽王\hspace{30pt}梓童,(你我)失迷路途,待孤探路。}

\textrm{褒姒\hspace{30pt}小心了。}

\textrm{幽王\hspace{30pt}(呃,)晓得哟。}

\setlength{\hangindent}{56pt}{   %3. 设置悬挂缩进量                %
\textrm{幽王\hspace{30pt}【{\akai 西皮摇板}】这一阵杀得孤魂飞胆丧,西戎兵一个个四路埋藏。又只见杏黄旗空呃中飘哇荡,倘若是遇仇敌孤命残伤。}
}

\textrm{(幽王{\hwfs 看``申''字大旗})}

\textrm{(幽王\hspace{30pt}啊------)}

\setlength{\hangindent}{56pt}{   %3. 设置悬挂缩进量                %
\textrm{幽王\hspace{30pt}【{\akai 西皮摇板}】上写着是申侯要除无道,拿住了小周王({\akai 或}:~小昏王)定斩不饶。}
}

\textrm{众\hspace{41pt}君王到。}

\textrm{(\textless{}\!{\bfseries\akai 急急风}\!\textgreater{},{\hwfs 四个}龙套{\hwfs 拿月华旗撤到大边},{\hwfs 斜向},{\hwfs 后面摆桌子}、{\hwfs 椅子},申侯{\hwfs 坐桌上},{\hwfs 子午相}。{\hwfs 两个}旗牌{\hwfs 分别站在桌子两旁的椅子上})}

\textrm{申侯\hspace{30pt}【{\akai 西皮导板}】听说是小昏王({\akai 或}:~听说是小周王)君妃驾到({\akai 或}:~君妃来到),}

\setlength{\hangindent}{56pt}{   %3. 设置悬挂缩进量                %
\textrm{申侯\hspace{30pt}【{\akai 西皮原板}】正要他到此来细问根苗。权作了痴呆汉我佯装不哇晓,是何方反贼兵身穿龙袍。}
}

\textrm{幽王\hspace{30pt}贤侯!}

\setlength{\hangindent}{56pt}{   %3. 设置悬挂缩进量                %
\textrm{幽王\hspace{30pt}【{\akai 西皮原板}】尊贤侯你那里未必不晓,孤就是周天子逃难荒郊。皆因是孤无道命运不好,看起来孤倒运还是孤八字不高({\akai 或}:~看起来算是孤八字不高)。}
}

\setlength{\hangindent}{56pt}{   %3. 设置悬挂缩进量                %
\textrm{申侯\hspace{30pt}【{\akai 西皮原板}】你既是周天子福分非小({\akai 或}:~八字非小;八字不小),你就该在宫中快乐逍遥。这是你人背时八字不好({\akai 或}:~命运不好),看起来天有眼报应在今朝。}
}

\setlength{\hangindent}{56pt}{   %3. 设置悬挂缩进量                %
\textrm{幽王\hspace{30pt}【{\akai 西皮原板}】尊贤侯你心下不必计较,孤有言你那里细听根苗:~望贤侯保孤回重重相报,孤封你世代公侯与孤王同掌九朝。}
}

\setlength{\hangindent}{56pt}{   %3. 设置悬挂缩进量                %
\textrm{申侯\hspace{30pt}【{\akai 西皮原板}】听他言不知羞令人呃可笑,我正要保你回同掌九朝。回宫去与奸妃把酒色贪好,你还把朝纲事一旦丢抛。}
}

\textrm{申侯\hspace{30pt}那土台之上,坐一女子,身穿大红,怀抱婴儿,她是何人?}

\setlength{\hangindent}{56pt}{   %3. 设置悬挂缩进量                %
\textrm{幽王\hspace{30pt}那就是褒姒娘娘,怀抱婴儿是孤的爱子名叫伯服。啊,梓童,上得前去端端正正见上一礼,他保你我君妃回朝也未可知。}
}

\textrm{(\textless{}\!{\bfseries\akai 行弦}\!\textgreater{}幽王{\hwfs 示意}褒姒{\hwfs 行礼})}

\textrm{(褒姒\hspace{25pt}奴惧呀。)}

\textrm{(幽王\hspace{25pt}你要去呀。)}\footnote{此句与上句褒姒念的``奴惧呀'',疑是同一句。}%\protect\hyperlink{fn9}{\textsuperscript{9}}

\textrm{(幽王{\hwfs 嗔介})}

\textrm{褒姒\hspace{30pt}是。}

\textrm{褒姒\hspace{30pt}申侯------万福!}

\textrm{申侯\hspace{30pt}好个褒娘娘正宫主{\footnotesize 呃}母({\akai 或}:~正宫国母)!}

\setlength{\hangindent}{56pt}{   %3. 设置悬挂缩进量                %
\textrm{申侯\hspace{30pt}【{\akai 西皮原板}】恨奸妃不行善你又不行好,无端地害吾妹天理自昭。每日呀里与昏王酒色欢好,你可知申正宫她与我一母同胞。}
}

\setlength{\hangindent}{56pt}{   %3. 设置悬挂缩进量                %
\textrm{褒姒\hspace{30pt}【{\akai 西皮原板}】申国母待小奴恩高义好,我岂肯将她人一旦丢抛。望贤侯看薄面把君妃来保,早烧香晚点灯答报恩高。}
}

\setlength{\hangindent}{56pt}{   %3. 设置悬挂缩进量                %
\textrm{申侯\hspace{30pt}【{\akai 西皮原板}】我自然保他回【{\footnotesize 转}{\akai 西皮快板}】看你的金面,为什么废太子上欺青天。叫人来准备下麻绳、铁链,不论君、不论妃锁拿军前。拿住了小昏王({\akai 或}:~拿住了小周王)大功来建,若有人违将令斩罪无宽。}
}

\setlength{\hangindent}{56pt}{   %3. 设置悬挂缩进量                %
\textrm{幽王\hspace{30pt}【{\akai 西皮原板}】你那里要我命不敢强辩,哭干了黄河水难保命还。依然是申国母独掌宫院,我和你郎舅情结什么仇冤。}
}

\setlength{\hangindent}{56pt}{   %3. 设置悬挂缩进量                %
\textrm{申侯\hspace{30pt}【{\akai 西皮摇板}】小昏王({\akai 或}:~教昏王)休得要巧言舌辩,待我将十条罪细表根源:~}
}

\setlength{\hangindent}{56pt}{   %3. 设置悬挂缩进量                %
	\textrm{申侯\hspace{30pt}【{\akai 西皮二六}】一条罪初登基民女挑选,二条罪老王薨作乐喧天。三条罪废国母宫闱混乱,四条罪宠尹球灭忠害贤。五条罪命石父}\footnote{石父即虢石父。}
%\protect\hyperlink{fn10}{\textsuperscript{10}}
\textrm{【{\footnotesize 转}{\akai 西皮快板}】兴兵发难,六条罪把朝纲丢在一边。七条罪弃太子纲常败乱({\akai 或}:~纲常大变),八条罪宠奸妃诡计多端。九条罪在骊山君妃饮宴,十条罪焚烟墩戏耍群贤。你为君十条罪人心涣散,死九泉见先王有何话言。}
}

\setlength{\hangindent}{56pt}{   %3. 设置悬挂缩进量                %
\textrm{幽王\hspace{30pt}【{\akai 西皮摇板}】大不该在骊山作乐饮宴,大不该焚烟墩戏耍群贤。你本是大丈夫英雄好汉,杀了我无用人你算什么能员。}
}

\setlength{\hangindent}{56pt}{   %3. 设置悬挂缩进量                %
\textrm{褒姒\hspace{30pt}【{\akai 西皮摇板}】听他言不由我心中好惨,做一朝人王主这样凄然。眼见得君妃们不能回转,}
}

\textrm{褒姒\hspace{30pt}\textless{}\!{\bfseries\akai 哭头}\!\textgreater{}万岁呀!}

\textrm{褒姒\hspace{30pt}【{\akai 西皮摇板}】我和你就死在燃眉之间。}

\textrm{褒姒\hspace{30pt}喂呀,万岁呀\ldots{}\ldots{}({\hwfs 哭介})}

\setlength{\hangindent}{56pt}{   %3. 设置悬挂缩进量                %
\textrm{申侯\hspace{30pt}【{\akai 西皮摇板}】又只见他君妃哭得好惨,杀君王犹如那子杀父般。叫三军({\akai 或}:~叫人来)放开路任他逃散({\akai 或}:~容他逃窜;容他逃散;任他逃窜),落一个忠义名万载流传。}
}

\textrm{幽王\hspace{30pt}梓童,此乃一字长蛇阵,有放你我君妃之意,我们走了罢({\akai 或}:~我们溜了罢)。}

\textrm{褒姒\hspace{30pt}(唉,)逃了罢({\akai 或}:~走了罢)。}

\textrm{幽王\hspace{30pt}溜了罢。}

\textrm{(幽王{\hwfs 在大边唱})}

\setlength{\hangindent}{56pt}{   %3. 设置悬挂缩进量                %
\textrm{幽王\hspace{30pt}【{\akai 西皮摇板}】孤回朝({\akai 或}:~孤还朝)有一日登了大宝,捉住了这申侯({\akai 或}:~捉住了小申侯)万剐千刀。}
}

\textrm{幽王\hspace{30pt}我们走了罢。}

\textrm{众\hspace{41pt}幽王逃走。}

\textrm{申侯\hspace{30pt}带马!}

\textrm{(旗牌{\hwfs 下去},{\hwfs 四绿}龙套{\hwfs 站门},{\hwfs 带马},申侯{\hwfs 上马})}

\setlength{\hangindent}{56pt}{   %3. 设置悬挂缩进量                %
	\textrm{申侯\hspace{30pt}【{\akai 西皮快板}】非是我放开路任他逃窜({\akai 或}:~与他逃窜),到前面}\footnote{此处吴焕老师整理本作``到前边''。}%\protect\hyperlink{fn11}{\textsuperscript{11}}
\textrm{遇戎兵难逃命还。教三军------}
}

\textrm{众\hspace{41pt}有!}

\textrm{申侯\hspace{30pt}【{\akai 西皮摇板}{\footnotesize 叫散}】你与爷({\akai 或}:~你与我)忙往前趱{\footnotesize 呐},}

\textrm{({\hwfs 两个}旗牌{\hwfs 在小边},申侯{\hwfs 大边})}

\textrm{申侯\hspace{30pt}【{\akai 西皮散板}】做一个{\footnotesize 哇}假人情顺{\footnotesize 呃}水推船。}

\textrm{(申侯{\hwfs 举马鞭},{\hwfs 台口亮相},{\hwfs 绕一圈},{\hwfs 打马下})}
}

\vspace{15pt}
{\bfseries\textrm{本戏人物扮相}}:~
\vspace{15pt}

申侯\hspace{30pt}绿蟒,黑三,戴侯帽,不挎宝剑。

幽王\hspace{30pt}大白粉脸,黑嘴窝,黑满,不戴帽,戴发鬏,穿黄团龙帔。\footnote{刘曾复先生注:~徐碧云排《褒姒》一剧时,萧长华以丑角饰演幽王。}%\protect\hyperlink{fn12}{\textsuperscript{12}}

褒姒\hspace{30pt}不戴凤冠,穿黄帔。

龙套\hspace{30pt}四名(一堂),俱穿绿。

旗牌\hspace{30pt}两名,俱穿黄;一个戴髯口,一个不戴髯口。

