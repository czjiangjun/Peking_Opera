\addcontentsline{toc}{section}{\hfill[\hei 元末·明·清]\hfill}
\newpage
\chead{元末·明·清} % 页眉中间位置内容
\hypertarget{ux91c7ux77f3ux77f6-ux4e4b-ux5f90ux8fbe}{%
\subsection{采石矶 之
徐达}\label{ux91c7ux77f3ux77f6-ux4e4b-ux5f90ux8fbe}}

\textbf{{[}第一场{]}}

\textless{}\textbf{点绛唇}\textgreater{}杀气冲霄,儿郎虎豹。军威浩,地动山摇,要把元朝扫\protect\hyperlink{fn544}{\textsuperscript{544}}。

站立两厢!

(念)昔日茅庐曾学艺,如今拜帅挂紫衣。六韬三略人难识,一战管取采石矶。

本帅,姓徐名达,字天德。蒙主重用,挂我为帅,先破牛渚渡,后破采石矶。

众位将军!

(众 元帅!)

此番出兵,全仗诸位,奋勇当先。功成之日,自然凌烟标名\protect\hyperlink{fn545}{\textsuperscript{545}}也!

(众 \ldots{}\ldots{}耳听\ldots{}\ldots{}愿听指挥!)

好,难得众位将军协力同心。李文忠听令!

(李文忠 在!)

命你带领三千长枪手,五百战船,先破牛渚渡,后破采石矶。听我令下!

郭英听令!

命你带领三千弓箭手,五百战船,先破牛渚渡,后破采石矶。听吾令下!

胡大海听令!

(胡大海 在!)

命你带领三千藤牌手,五百战船,先破牛渚渡,后破采石矶。听吾令下!

常遇春听令!

(常遇春 在!)

命你带领三千火炮手,五百战船,先破牛渚渡,后破采石矶。听吾令下!

分派已毕,众将官!

(众 有!)

人马往采石矶去者!

(龙套唱\textless{}\textbf{泣颜回}\textgreater{}
``\ldots{}\ldots{}看长江浪息风恬,济川人自在行舟。''\protect\hyperlink{fn546}{\textsuperscript{546}})

\textbf{{[}第二场{]}}

众将官!

(众 有!)

人马扎驻者!

【西皮导板】将人马扎驻在傍水村呐,

【西皮快板】旌旗招展似风云。六韬三略胸藏定,熟习孙武善用兵。个个相争分强胜呐,

【西皮摇板】须当努力\protect\hyperlink{fn547}{\textsuperscript{547}}要齐心。

\textbf{【}唢呐二黄原板\textbf{】向前者一个个呃有功有赠,退后者按军令插箭游营。俺奉旨掌管了虎头金印,当学那三齐王调将遣兵。众将官------}

\textbf{【唢呐二黄原板】齐}努力\textbf{鞭敲金镫,得胜时封官爵万古留名。}

\textbf{【唢呐二黄原板】统雄兵造战船千员上将,抖威风似虎狼夺功争强。狼牙槊、龙虎牌可战可挡,画角鸣、金锣响大纛飘扬。}

\textbf{【唢呐二黄原板】定营炮、催军鼓轰声震地,扫平贼卷旌旗奏凯而归。写表章功劳簿龙棚朝奏,那时节太平宴畅饮开怀。}

\textbf{且住,看采石矶山高水险,一时焉能得破!}

\textbf{呵呵有了!}

\textbf{满营将官听者------}

\textbf{(众 嚯!)}

\textbf{有人攻破采石矶,先锋大印付他执掌。}

\textbf{(众 啊!)}

\textbf{{[}第三场{]}}

\textbf{就将先锋大印付与常遇春执掌!}

\textbf{后帐摆宴,与众位将军贺功!}

\newpage
\hypertarget{ux6218ux592aux5e73-ux4e4b-ux82b1ux4e91}{%
\subsection{战太平 之
花云}\label{ux6218ux592aux5e73-ux4e4b-ux82b1ux4e91}}

\textbf{{[}第一场{]}}

\textbf{(内)回府哇!}

\textbf{可恼哇!}

\textbf{夫人有所不知,今有反贼}陈友谅兴兵犯界,攻取太平城,千岁命我全身披挂,守住札子口,只是采石矶头缺一能将,故而烦恼。

有劳夫人。正是:(念)青龙背上屯军马,

\textbf{【二黄导板】头戴着紫金盔齐眉盖鬓}\protect\hyperlink{fn548}{\textsuperscript{548}}\textbf{呐,}

\textbf{【二黄散板】为大将临阵时啊哪顾得残生呐。撩铠甲且把二堂进,}

\textbf{【二黄散板】有劳夫人点雄兵。}

\textbf{【二黄散板】接过夫人得胜饮呐,背转身来谢神灵。辞别夫人足踏镫,但愿此去扫荡烟尘(或:奏凯回程)。}

\textbf{{[}第二场{]}}

\textbf{【二黄摇板】一见贼子怒气发,不由老爷咬银牙。吾主洪福齐天大,把尔当作井底蛙。}

(\textbf{花云与陈英(友)杰头场开打``大扫琉璃灯''}:钻完烟筒花云大边,一扯两扯,半拉合到小边,回身盖下手的一刺,打下手蓬头,枪缓下来刺耳被勾走马腰封到小边,勾下手走马腰封,大转,下手到大边,花云在小边,里、外各一盖两盖,挑起来打下手腰封,往里漫下手头、下手低头左转身面朝里、背朝外,花云扎下手脖(即贴下手靠旗斗),下手小蹦子转身面朝外,右手推出花云枪,亮住。)\protect\hyperlink{fn549}{\textsuperscript{549}}

(\textbf{花云与陈英(友)杰头场另一种开打``一百零八枪头子''}(\textbf{没有幺二三的快枪}):钻完烟筒,上手大边,一扯两扯,半拉合到小边,回身盖下手的一刺,打下手蓬头,缓下来到右边一刺接腰封,绕道左边盖下手一刺打腰封,再右边一刺接腰封,再左边一盖打腰封,缓下来,下手刺耳,勾下手走马腰封,上手到大边,下手到小边,往外漫下手头,一、二、三刺下手喉,下手跟着三刨,往里漫下手头,二人别,撤下来一盖下手的一刺、打腰封,一刺下手接腰封,再一盖下手一刺、打腰封,这三个一刺腰封是边盖边打、边刺边接、边盖边打、边打边走的,上手从大边到小边,下手到了大边,绕下来,别,下手刺上手马腿,上手挑起来往里刺肚,左转身刺下手马腿,接下手刺肚,原地一盖、两盖、扎脖,下手是递把转身,再一刺转身面向里,小蹦子转身面外,推上手扎脖的枪,二人亮住(所谓扎脖实际是枪贴靠旗斗)。)\protect\hyperlink{fn550}{\textsuperscript{550}}

(
(花打陈下,面向外提枪花转身,面向外)三个提枪花,出枪,左手往左上方伸出平托枪(杆),右手往右掠枪杆往下绕过枪鐏反手扶鐏,跨左腿,踢右腿,两手不离枪,向右往里转身(/翻身),面向下场门,左手在前扶住枪杆,右手顺着鐏转过来、握住枪下端在右侧腰间,一绕枪头,平端枪,弓箭步(亮住),拧腰,亮住下。)\protect\hyperlink{fn551}{\textsuperscript{551}}

\textbf{{[}第三场{]}}

\textbf{花云!}

\textbf{来也!}

(\textbf{花云保护朱文逊杀出重围小开打后枪下场:}花打陈下,面向里矗枪,三个背花,转身面向外三个迎面花,扫左腿,扫右腿,在左边向后打右脚(右脚在左腿后反抬),枪顺过来单腿立背枪,向左上方斜着出左手,缓过枪来在脸前画三个大圈,同时左手也掏着画三圈,到第三圈左手向左斜上方伸出去,右手枪打左手,上膀子,跨左腿,右腿上步,右转身略偏小边面向外,在转身当中右手撤到胸前,左手绕枪鐏转过来面向外,右手平着出枪,左手绕到胸前按胡子弓箭步亮住,下场门下。)\protect\hyperlink{fn552}{\textsuperscript{552}}

\textbf{{[}第四场{]}}

\textbf{参见千岁!}

\textbf{千岁休得惊慌,随定为臣,杀出重围,去至金陵,搬兵取救(或:搬兵求救)。}

\textbf{唉呀,千岁呀,事到如今,你还顾得什么家眷呐!}

\textbf{诶------他为君的有家眷,我为臣的就无有家眷么?!}

\textbf{诶------我也要保护家眷去了!}

\textbf{{[}第五场{]}}

\textbf{【西皮导板】号炮一响啊惊天地呀,}

\textbf{【西皮散板】就是雀鸟也难飞。教花安与父带坐骑,舍不得妻儿两分离。}

\textbf{(【西皮散板】用手抱定娇儿体,我的儿啊,父子难免各东西。)}

\textbf{【西皮散板】夫人请上受一呃\textless{}哭头\textgreater{}礼,夫人呐!}

\textbf{【西皮散板】下官言来听端的:孙氏、娇儿托付你,这是我花家一脉系。}

\textbf{【西皮散板】狠心我把妻儿弃呀,落一个青史名标万古题。}

\textbf{{[}第六场{]}}

\textbf{千岁,你保得好家眷呐!}

\textbf{随定为臣(或:随臣马后),杀出重围!}

\textbf{{[}第七场{]}}

\textbf{【西皮导板】叹英雄失智呃入罗网,}

\textbf{【西皮原板】大将呃难免阵头亡。我主爷洪福齐呃天降,刘伯温八卦也平常呃。早知道(或:既知晓)采石矶被贼抢呃,早就该差能将前来提防。将身儿来在大街呀上,}\protect\hyperlink{fn553}{\textsuperscript{553}}

\textbf{【西皮摇板】那旁来了疯婆娘。}

\textbf{【西皮散板】这一足哇踏在你地埃尘呐(或:踏你在地埃尘呐)。你是谁家疯魔女啊,}

\textbf{【西皮快板】怀中抱定小娇生。明明认得孙侍女}\protect\hyperlink{fn554}{\textsuperscript{554}}\textbf{,假装疯魔见夫君(或:见主人。)。你若念在夫妻义,去至金陵搬救兵。你若不念夫妻义,千万莫丢小娇生。(或:你若念在主仆义,去至金陵搬救兵。拜求圣上发人马,点动我国龙虎军。)使个眼色快逃走。}

\textbf{【西皮散板】大街上去了孙侍女,我的妻呀!夫妻们相逢万不能。(或:大街上去了孙侍女,父子们见面万不能。)}\protect\hyperlink{fn555}{\textsuperscript{555}}

\textbf{{[}第八场{]}}

\textbf{【西皮摇板】虎落平川怎脱逃。}

\textbf{啊?!}

\textbf{自然是怒骂}\protect\hyperlink{fn556}{\textsuperscript{556}}\textbf{的是啊。}

\textbf{诶,怒骂的是。}

\textbf{千岁!}

\textbf{【西皮快板】千岁爷休说懦弱话,非是为臣把君压。进帐去把贼骂,拚着一命染黄沙。纵然将你我头割下,落一个骂贼的名儿扬天涯。}

\textbf{怒骂的是!}

\textbf{诶------懦弱无刚啊!}

\textbf{【西皮摇板】站的是啊你老爷将花云呐。}

\textbf{唉呀千岁呀\ldots{}\ldots{}}

\textbf{【西皮散板】哗喇喇大炮一声响,血淋淋的人头滚一旁。先前怎样对你讲,一心降顺北汉王。那贼焉有容人量,顷刻之间一命亡啊。}

\textbf{【西皮散板】我把人头打进帐}\protect\hyperlink{fn557}{\textsuperscript{557}}\textbf{呃,}

\textbf{【西皮快板】开言大骂北汉王:既是兴兵来打仗,一来一往动刀枪。暗施诡计非为上}\protect\hyperlink{fn558}{\textsuperscript{558}}\textbf{,你是人面兽心肠。}

\textbf{\textless{}叫头\textgreater{}陈友谅!}

\textbf{呀------}

\textbf{【西皮快板】陈友谅他把好言讲,背转身来自思量:我若是降了贼友谅,落得骂名天下扬。我若是不降贼友谅,顷刻之间一命亡。罢罢罢,屈膝呀我跪宝帐,}

\textbf{呃!}

\textbf{【西皮摇板】你老爷愿死我不愿降。}

\textbf{{[}第九场{]}}

\textbf{【西皮导板】盖世英雄遭毒手哇,}

\textbf{【西皮快板】好一似鳌鱼吞了钩。采石矶头入虎口,汗马功劳一笔勾。将身来在法标口,}

\textbf{【西皮摇板】为国忠良的下场头。}

\textbf{【西皮快板】大吼一声冲牛斗,大骂奸贼听从头:要我归降不能够,岂与奸贼做马牛。}

\textbf{【西皮快板】怒气填胸(或:开言怒发)三千丈,太阳头上冒火光。要我归降休妄想,}

\textbf{贼呀,贼!}

\textbf{【西皮快板】除非红日出西方。}

\textbf{{[}第十场{]}}

\textbf{杀败了哇,杀败了!}

\textbf{不想------误入罗网,身带箭伤,我命休矣!}

\textbf{\textless{}叫头\textgreater{}圣上啊,吾主!}

\textbf{臣力已竭,不能保主江山社稷了。}

\textbf{也罢!}

\textbf{我不免拜谢我主爵禄之恩,自刎疆场!免受贼辱!}

\textbf{罢!}

\textbf{*注:王凤卿的《战太平》一剧唱法与常见的谭派唱法异趣,刘曾老曾为吴小如先生留下示范录音,有关唱词兹录如下:}

\textbf{【二黄导板】受君禄当报效臣把忠尽,}

\textbf{【二黄散板】扫烟尘开疆土呃定主乾坤呐。古至今多少人不惜性命,君依礼臣尽忠哪顾宗亲。}

\textbf{【二黄散板】接过了得胜酒心中思忖,转身来祭虚空过往神灵。但愿得扫狼烟苍生之幸,方显得男儿汉盖国俊英。}

\textbf{【二黄散板】两军对垒在沙场,气得老爷怒满膛。无知匹夫敢犯上,狐群狗党一扫光。}

\textbf{【西皮导板】千岁说话真非理呀,}

\textbf{【西皮散板】把我当作小儿欺。}

\textbf{【西皮散板】花安与父带坐骑,二位夫人泪悲啼。用手抱定娇儿\textless{}哭头\textgreater{}体,我的儿啊!}

\textbf{【西皮散板】父子难免各东西。}

\textbf{【西皮散板】夫人请上受一\textless{}哭头\textgreater{}礼,夫人呐!}

\textbf{【西皮散板】这是花门一脉系。孙氏年轻全仗你,大小事体相扶持。狠心我把妻儿弃呀,}

\textbf{【西皮散板】落一个青史名标万古题。}

\textbf{【西皮导板】叹英雄失智落陷阱,}

\textbf{【西皮原板】一腔怒气贯长虹}\protect\hyperlink{fn559}{\textsuperscript{559}}\textbf{。刘伯温枉自挂帅印,用兵不到欠思忖呐。既知晓采石矶有伤损,为何不发【转西皮二六】救兵临。}

\textbf{【西皮摇板】迈虎步来在大街境,}

\textbf{【西皮摇板】那旁来了疯魔人。}

\textbf{【西皮散板】这一足踏你在街心。你是谁家疯魔女呀,}

\textbf{(孙氏 哈哈,哈哈,啊\ldots{}\ldots{}(笑介))}

\textbf{【西皮快板】怀中抱定小娇生。我若上前将妻认,泄露机关命难生。你若真心救夫命,去至金陵搬救兵。使个眼色快逃奔呐。}

\textbf{(孙氏 哈哈,哈哈,啊\ldots{}\ldots{}(笑介))}

\textbf{【西皮散板】大街上去了孙侍\textless{}哭头\textgreater{}女婢,我的妻呀!}

\textbf{【西皮散板】夫妻们重逢待来生。}

\newpage
\hypertarget{ux6c5fux4e1cux6865}{%
\subsection{江东桥}\label{ux6c5fux4e1cux6865}}

\textbf{康茂才 【西皮导板】战鼓嗵嗵下校场,}

\textbf{康茂才
【西皮二六】气得豪杰怒满胸膛。恼恨那反贼陈友谅,屡次兴兵犯边疆。朱公子不幸把命丧,花云又中乱箭身亡。怒恼了我主把旨降,命刘基挂帅坐至在大堂。头一枝将令往下降,常遇春领兵到战场。又差白袍郭英将,在那江东桥前暗地埋藏。帐中(的)将官(或:满营将官)【转西皮快板】俱差往,单单不差某架海金梁。因此我发笑在辕门上,笑得刘基脸无光。他那里把某来小量,怎知某能敌百万儿郎。中军帐,三击掌,赌头输印(或:赌头争印)两争强。我若是擒住了陈友谅,军师大印某承当。我若是擒不住陈友谅,愿将首级挂营房。兵在精,哪在广,将在谋略哪在刚强。三军带路土台上呐,}

\textbf{康茂才 【西皮摇板】等候了反贼北汉王。}

\textbf{陈友谅 【西皮摇板】得胜狸猫欢似虎,}

\textbf{陈友谅 唉!}

\textbf{陈友谅 【西皮摇板】败阵凤凰不如鸡。}

\textbf{陈友谅 且喜逃出虎口,来此什么所在?}

\textbf{陈英杰 乃是江东桥。}

\textbf{陈友谅 看前面旌旗招展,不知哪家人马。贤弟上前观看。}

\textbf{陈英杰 遵命。}

\textbf{陈英杰
【西皮摇板】豪杰打马奔土台,只见旌旗两边排。杏黄旗上写大字:无敌将军康茂才。}

\textbf{陈英杰 乃是``康''字旗号!}

\textbf{陈友谅 怎么讲?}

\textbf{陈英杰 ``康''字旗号!}

\textbf{陈友谅 哈哈,哈哈,啊,呵呵哈哈哈\ldots{}\ldots{}(笑介)}

\textbf{陈英杰 兄王,你为何发笑?}

\textbf{陈友谅
那康茂才曾许我三不死,待我向前苦苦哀求,放你我君臣逃走,也未可知。}

\textbf{众 我们杀不得了!}

\textbf{陈友谅 不用你们杀了!}

\textbf{众 我们也战不得了!}

\textbf{陈友谅 不用你们战了!你且退下!}

\textbf{众 啊!}

\textbf{陈友谅
【西皮快板】听说来了康茂才,不由得孤王笑开怀。走向前,把礼拜,问声贤弟可安泰。}

\textbf{康茂才
【西皮快板】军师将令把咱委,站立土台抖雄威。刘基不识英雄辈,他道豪杰少计策。打赌曾把牙咬碎,气得某家两眼黑。想起了前朝军对垒,好汉英雄出在三国}\protect\hyperlink{fn560}{\textsuperscript{560}}\textbf{。关公曾挡华容道,要把孟德魂魄追。战鼓儿不住咙嗵咙嗵咙嗵地打,鸣金的(或:勾命的)铜锣仓啷啷仓啷啷啷紧紧地催。前面走的陈友谅,后面跟随一伙贼。半像人呐半像鬼,个个脸上带尘灰。好一似佛爷离宝位,罗汉缺少袈裟披。美不美,乡中水,亲不亲来故乡回}\protect\hyperlink{fn561}{\textsuperscript{561}}\textbf{。康茂才便把良心昧,土台下来的尔是谁。}

\textbf{陈友谅
【西皮快板】康贤弟比我小几岁,耳不聋来眼不黑。明知友谅逃命归,缘何假装不认得?}

\textbf{康茂才 【西皮导板】康茂才站土台一声高骂,}

\textbf{康茂才
【西皮原板】骂一声陈友谅无义冤家。想当年在原郡同窗长大,你爱我,我爱你同把香插。徐寿辉他待你是真心非假,你不该用药酒将他毒杀。我国中刘军师阴阳八卦,算就了尔君臣半点不差(或:半点无差)。你好比秦赵高指鹿为马,又好比汉萧何私造律法。你好比曹丕贼称孤道寡,又好比毛延寿害娘娘【转西皮快板】怀抱琵琶。青龙刀摆一摆威风潇洒,顷刻间管教你(或:顷刻间管教尔)毁室亡家。}

\textbf{陈友谅
【西皮快板】你不修书孤不到,你教我带兵杀奔你朝。常遇春使钢鞭赛似虎豹,小郭英使长枪真乃英豪。只剩下残兵败将齐来到,望贤弟开恩把我饶。}

\textbf{康茂才
【西皮快板】扭回头来叫小校,老爷言来听根苗:养兵千日用一朝,多备麻绳不用刀。生擒活捉陈友谅,黄罗帐中报功劳。有人放走陈友谅,准备项上吃一刀,一刀一个,一个一刀定斩呐不饶。}

\textbf{陈友谅
【西皮快板】昔日挡曹华容道,你今挡孤江东桥。关公饶曹三不死,你今饶我这一遭。你要放,孤逃跑;你若是不放,你来来来,我吃你一刀,我若是逃走哇,孤是草鸡毛}\protect\hyperlink{fn562}{\textsuperscript{562}}\textbf{。}

\textbf{康茂才
【西皮快板】这几句言语讲得好,我比关公他比曹。我比关公实难比,他比奸曹不差分毫。关公饶曹三不死,我今饶他这一遭。教三军排开一字长蛇道,}

\textbf{康茂才 【西皮摇板】认识此阵放你逃。}

\textbf{陈友谅 贤弟,看看是何阵势?}

\textbf{陈英杰 乃是一字长蛇大阵。}

\textbf{陈友谅 怎么讲?}

\textbf{陈英杰 一字长蛇大阵。}

\textbf{陈友谅
昔日关公放曹,就是这一字长蛇大阵。莫非那康茂才有放你我君臣逃走之意?你我趁此机会,溜了罢!}

\textbf{陈英杰 逃了罢!}

\textbf{陈友谅 溜了罢!}

\textbf{陈友谅 【西皮摇板】打开玉笼飞彩凤,斩断金锁走蛟龙。}

\textbf{陈友谅 我走了!}

\textbf{康茂才 好走!}

\textbf{康茂才 回营交令呐!}

\textbf{康茂才
【西皮快板】悔不该与他结义好(或:金兰好),悔不该发誓将他饶。悔不该辕门来发笑,悔不该帐中论英豪。关公犯法刘备保,我今难逃这一遭。盖世英雄辜负了哇,}

\textbf{康茂才 【西皮摇板】但不知军师爷饶不饶。}

\newpage
\hypertarget{ux5fe0ux5b5dux5168-ux4e4b-ux79e6ux6d2a}{%
\subsection{忠孝全 之
秦洪}\label{ux5fe0ux5b5dux5168-ux4e4b-ux79e6ux6d2a}}

\textbf{{[}第一场{]}}

\textbf{(王振 下跪可是福建的解粮官么?)}

\textbf{正是。}

\textbf{(王振 \ldots{}\ldots{},讲!)}

\textbf{一路之上风雨阻隔,故而来迟。}

\textbf{(王振 \ldots{}\ldots{}看粮、豆如何?)}

\textbf{(众 \ldots{}\ldots{}干豆。)}

\textbf{(王振 \ldots{}\ldots{}干豆,\ldots{}\ldots{}却是为何?)}

\textbf{乃是卑府一片孝心!}

\textbf{(王振 二府?)}

\textbf{升去。}

\textbf{(王振 三府?)}

\textbf{告老。}

\textbf{(王振 \ldots{}\ldots{}为何不抬起头来?)}

\textbf{有罪不敢抬头。}

\textbf{(王振 赦你无罪。)}

\textbf{谢千岁!}

\textbf{(王振 唗!)}

\textbf{罢了哇罢了!}

\textbf{{[}第二场{]}}

\textbf{【西皮导板】犯官受刑身无主,}

\textbf{【西皮原板】二目圆睁怕煞人。中军帐好一比森罗殿,王公公好一比五殿阎君。刽子手好一比催命鬼,我好比屈死一鬼魂。再不能宛平为正印,再不能福建管黎民。再不能一家团圆庆,再不能满门享太平。忍泪含悲法场进,}

\textbf{【西皮摇板】咬定牙关等时辰。}

\textbf{(秦继龙 \ldots{}\ldots{}家住哪里?
\ldots{}\ldots{}监斩一毕,也好收你的尸首。)}

\textbf{监斩爷容禀!}

\textbf{【西皮原板】未曾开言泪先淋,}

\textbf{(秦继龙 不要啼哭,慢慢讲来。)}

\textbf{【西皮原板】尊一声监斩爷细听分明:}

\textbf{(秦继龙 家住哪里?)}

\textbf{【西皮原板】家住山东莱州郡,}

\textbf{(秦继龙 哪里家门?)}

\textbf{【西皮原板】即墨县内有门庭。}

\textbf{(秦继龙 可曾得中?)}

\textbf{【西皮原板】甲子年间得中举,}

\textbf{(秦继龙 可曾上进?)}

\textbf{【西皮原板】连科得会进士名。}

\textbf{(秦继龙 初任哪里为官?)}

\textbf{【西皮原板】初任宛平为正印,}

\textbf{(秦继龙 可曾升任?)}

\textbf{【西皮原板】钦命福建管黎民。}

\textbf{(秦继龙 身犯何罪?)}

\textbf{【西皮原板】都只为金鳌叛边境,误粮不到问典刑。}

\textbf{(秦继龙 膝下有几个儿子?)}

\textbf{【西皮原板】犯官膝下有三子,}

\textbf{(秦继龙 亲生还是螟蛉?)}

\textbf{【西皮原板】二子亲生一子螟蛉。}

\textbf{(秦继龙 亲生之子叫何名字?)}

\textbf{【西皮原板】长子继美、次继远,}\protect\hyperlink{fn563}{\textsuperscript{563}}

\textbf{(秦继龙 螟蛉何名?)}

\textbf{【西皮原板】螟蛉叫作秦继龙。}

\textbf{(秦继龙 秦继龙哪里去了?)}

\textbf{【西皮原板】都只为小娇儿不听教训,一家四口赶出了门庭呐。}

\textbf{(秦继龙 你如今可思想于他?)}

\textbf{【西皮原板】眼前若有继龙啊\textless{}哭头\textgreater{}子,}

\textbf{【西皮摇板】纵死黄泉也甘心。}

\textbf{(秦继龙 你叫何名字?)}

\textbf{【西皮摇板】监斩爷问我的名和姓,红旗上现有我犯官姓名。}

\textbf{(秦继龙 【西皮摇板】\ldots{}\ldots{}我担承。)}

\textbf{【西皮摇板】法场上绑得我昏迷不醒,}

\textbf{(秦继龙 起来。)}

\textbf{唉呀!}

\textbf{【西皮摇板】问一声监斩爷你是何人。}

\textbf{(秦继龙 爹爹。)}

\textbf{(秦继龙 【西皮摇板】\ldots{}\ldots{}把身存。)}

\textbf{儿是继龙?}

\textbf{\textless{}哭头\textgreater{}啊,我的儿啊!}

\textbf{【西皮散板】不正不正父不正,不该将儿赶出门。只说儿天涯逃性呐\textless{}哭头\textgreater{}命,我的儿啊,}

\textbf{【西皮散板】法场收父死尸灵。}

\textbf{(秦继龙 【西皮散板】\ldots{}\ldots{}老天伦。)}

\textbf{儿啊,此番将父松绑,可有圣上旨意?}

\textbf{(秦继龙 无有。)}

\textbf{元帅将令}

\textbf{(秦继龙 也无有。)}

\textbf{唉呀,不好了!}

\textbf{【西皮散板】私解法绳犯军令,知法犯法罪非轻。为父年迈当尽呐命,我的儿啊,怎肯连累小娇生。}

\textbf{(秦继龙 爹爹!)}

\textbf{(秦继龙 【西皮散板】\ldots{}\ldots{}走一程。)}

\textbf{(王振 你们都起来吧!)}

\textbf{谢千岁!}

\textbf{(王振 \ldots{}\ldots{}下去。)}

\textbf{儿啊,离家日久,哪里来的这身荣耀?}

\textbf{(秦继龙 \ldots{}\ldots{})}

\textbf{(王振 圣旨下,跪!)}

\textbf{万岁!}

\textbf{(王振 \ldots{}\ldots{}养老太师!)}

\textbf{谢主隆恩!}

\textbf{(王振 \ldots{}\ldots{}谢恩呐!)}

\textbf{万万岁。}

\textbf{(王振 恭喜老太师。)}

\textbf{谢千岁提拔之恩。}

\textbf{(王振 \ldots{}\ldots{}谁来搭救咱?)}

\textbf{呃,千岁若不嫌弃,将小儿拜在千岁名下,以为螟蛉义子,不知千岁意下如何?}

\textbf{(王振 那可使不得呀!)}

\textbf{使得的,儿啊,快快拜见你的义父。}

\textbf{(王振 \ldots{}\ldots{}忒轻呃!)}

\textbf{千岁!啊------哈哈哈哈\ldots{}\ldots{}(笑介)}

\textbf{(王振 \ldots{}\ldots{},老太师哪里为官?)}

\textbf{原任为官。}

\textbf{(王振 几时启程?)}

\textbf{即刻启程。}

\textbf{(王振 恕不远送。)}

\textbf{告辞了。}

\textbf{(王振 请------)}

\textbf{免。}

\textbf{梅龙镇}\protect\hyperlink{fn564}{\textsuperscript{564}}

{[}第一场{]}

(\textless{}\textbf{小锣}\textgreater{},正德拿扇上,台口中间)

正德 {[}引{]}离金阙暗藏珠宝,游天下察访民情。(正中小座)

正德
(念)大明一统锦山河,龙凤贤良庆笙歌(或:龙安凤逸庆笙歌)。孤王离了燕京地,闻得大同景致多。

正德
(白)孤,大明正德天子朱厚照(或:大明天子正德在位),离了江南太平庄,昨晚夜宿周家村,今日暂停梅龙镇上,店主人巡更守夜去了,临行言过(或:临行言道),用茶用酒响动木马,自有人来。今晚闲坐无事,我不免试它一回便了。

正德
【四平调】有寡人闷坐梅龙镇,思想朝中大事情。将玉玺授予了龙国太,(正德立)朝中大事(或:朝房大事)有公卿。忙将木马一声震
(立桌大边击木马,连唱),唤出提茶送酒人,啊啊啊,畅饮几巡。(内大座)

李凤姐 (内白)来了!

(凤姐\textless{}\textbf{小锣}\textgreater{}上唱,托盘)

李凤姐
【四平调】自幼儿生长在梅龙镇,兄妹卖酒度光阴。我兄巡更去守夜(或:我兄长巡更去守夜),他言说前堂有一位军人(或:他言道前店有一位军人)。我这里且把前堂进(或:我这里茶盘桌放定),

(凤姐掀帘进门到桌小边,左手挡脸,右手在左臂下递盘放桌上,正德用扇压下凤左手,二人对望,正惊,凤啐,掀帘出门)

李凤姐 【四平调】急忙转回小房门(或:绣房门)。啊啊啊,小房门(或:
绣房门)(前去拈针)。

(巾搭肩落地、落在台口中间,正德紧跟,掀帘出门望下场门李凤姐背影,站大边,``嗯喷''。凤姐回望,正招手,指巾,凤看自身、看巾、指巾、拾巾、搭巾走,正踩巾,凤未拉动,凤教正躲开,正不理,凤拜拜,正答应躲开,凤拾,正又要踩,凤不要巾走,正招凤回来教凤拿巾,正踩,凤又没拿动,凤想,指大边台口示有情况,正望,凤推正双手拾巾,左边云手转身由前往后背套上,回头啐,亮相笑下,正笑掀帘进门)

正德 哈哈哈。

正德
【四平调】好花儿出在深山内,美人出在这小地名。朕忙将木马再来震(或:忙将木马连声震),

(唱中正进位坐,击木马,凤上接正第三句,不要过门唱)

李凤姐 【四平调】后面来了送酒的人(或:后面来了卖酒的人)。

正德 酒保酒保。

(正边击木马边念,凤立桌小边)

李凤姐
酒保无有,倒有个酒\ldots{}\ldots{}(或:无有酒保,有个酒\ldots{}\ldots{})

正德 敢莫是酒大嫂?

李凤姐 啊,人家黄花幼女,说什么酒大嫂?

正德 要叫什么?

李凤姐 酒大姐。

(凤亮相,正德桌里边、向大边立背供念)

正德 哎呀呀,好大的口气。也罢,借酒为名,叫她一声,哦,酒大姐。

李凤姐 做什么?(或:军爷讲说什么?)

正德 哦哦哦,方才那一汉子他是何人?

李凤姐 他是我的哥哥。

(正德 呃,他是你的哥哥?)

(李凤姐 正是。)

正德 他叫什么名字?

李凤姐 他叫李龙。

正德 哦,他叫李龙,你呢,叫什么?

李凤姐 哦我呀,我是没有名字的呀!

正德 哦,人生天地之间哪里有没有名字的道理,你叫什么?

李凤姐 名字倒有,说出来怕人来叫(或:说出来,我怕军爷你叫啊)。

正德 你说将出来,为军(君)的不叫就是。

李凤姐 君爷不叫?

正德 不叫。

李凤姐 如此,我,

(正德 叫什么?)

李凤姐 我姓李呀。

正德 哦,我原晓得你是姓李呀,你叫什么?

李凤姐 我,我叫(,叫)李凤姐。

正德 哦,李凤姐,好一个李凤姐。

李凤姐 你把名字还我吧。(或:呀呸,拿名字来还我。)

正德 话出如风怎能还你?

李凤姐 你方才说不叫,为何又叫了起来?
(或:方才说过不叫,如今怎么又叫起来了?)

正德 (呃,我)下次不叫就是。

(李凤姐 下次不可。)

(正德 哦,是是是。)

李凤姐 你唤我出来做甚呐?(或:军爷用些什么?)

正德 哦哦,这梅龙镇上就是这样的酒饭不成?

(李凤姐 我们这里有三等酒饭。)

(正德 哪三等?)

李凤姐 有上中下三等。

正德 这上等的何人所用?

李凤姐 来往官员所用。(凤姐过大边,边念边走)

正德 中等的?

李凤姐 行路客商所用(或:买卖客商)。(凤姐边念边走回小边)

正德 这下等的?

李凤姐 这下等的呢,不说也罢。

正德 为何不讲(啊)?

李凤姐 说将出来,我怕客官你着恼哇。(或:说出来,恐怕军爷你,着恼哇。)

正德 为军的不恼(就是),你只管地讲来。

(李凤姐 啊,军爷你不恼?)

(正德 不恼。)

李凤姐 就是你们当军之人(或:吃粮当军人)所用。

正德
呜哙呀(或:哎呀且住),原来当军之人有这般苦处(或:原来这吃粮当军之人有这等苦处)。也罢,此番回得朝去,发饷银十万犒赏他们。(正德立向大边背供念,坐)

正德 (啊,)大姐,你将这上等的酒饭摆上一席,为军的一用啊。

李凤姐 啊军爷,(要用上等的酒饭么?)

(正德 是。)

李凤姐 我打个哑谜,你可愿猜? (或:我与你打个哑谜,你可晓得?)

正德 为军的最喜哑谜,你快些讲来。

李凤 这,渡船,(或:有道是:渡船------)

正德 哦,船钱。

李凤姐 这宿店,(或:住店------)

正德帝 店钱。

李凤姐 这吃酒呢?

正德 酒------后\ldots{}\ldots{}

李凤姐 啊,

正德 自然是酒后哇。

李凤姐
哎呀呀,你这人连酒钱都不会说。(或:哎呀呀,你连``酒钱''都不会讲,说什么``酒后''啊。)

正德 听你之言,是要钱呐。

李凤姐 我倒不要钱。

正德 哪个要钱?

李凤姐 我哥哥回来,问我要钱。

正德 (哦,)只要要钱就好讲话了,大姐你将帘儿卷起。

李凤姐 是。

(凤姐台口卷帘,正出位大边立唱)

正德
【四平调】好一个伶俐酒家女(或:好一个聪明李凤姐),言谈吐语真聪明(或:未曾用宴先讨钱)。在龙袍袖儿内摸一把(或:在龙袍袖儿中摸一把;或:我忙将袖内抓一把),白晃晃取出雪花银(或:
一锭银)。

正德 拿去。

(左手扇,右手从袖中取银,递银)

李凤姐 放下。

正德 放在何处呀?

李凤姐 放在桌面上。

正德 为什么要放在桌面上呢?(或:为什么要放下呀?)

李凤姐 你可晓得男女授受不亲呐?

正德 哎呀妙哇,男女授受不亲。(呃,放下。呃,放在何处呀?)(正德放银又收回)

(李凤姐 放在桌案上。)

正德 桌面是光的,银锭是圆的,滑将下来,掉在地面,那还了得。

李凤姐 不妨事,由我来拣。(或:那我会拾起。)

正德 哦我怕呀。

李凤姐 (啊,)你怕什么啊?

正德 我怕闪了大姐你的腰哇。

李凤姐 闪了我的腰与你什么相干?(放下!)

正德 好,放下就放下,拿去。

(正德放银桌上,右手用扇指银,凤姐上前要拿,正打开扇,一掩银,站桌前扇扇,凤退)

李凤姐 军爷你敢是舍不得(么)?

正德 为军的舍得,只怕大姐你舍不得。

李凤姐
哎呀,我看这个人有些不老成,我倒要诓上他一诓。(或:哎呀呀,看这军爷有些不老成,嗯,待我哄他一哄。)

(凤姐背供念,再转身念)

李凤姐 (啊)军爷,

(正德 做什么?)

李凤姐
你进得我们店中可曾看见那幅古画么?(或:进得我们店中,可曾看见一副古画?)

正德 为军的最爱古画,在哪里?

李凤姐 在那厢呢(或:在那里)。

(正德 在哪里?)

(李凤姐 在那里。)

(正德 在哪里?)

(李凤姐 军爷,在这里哟。) (凤姐指大边外面,正德向外看,凤姐拿银)

正德 (哎呀,)倒被她诓了去了。

李凤姐
【四平调】施巧计(或:用巧计)诓过银一锭,莫不是响马到来临,我这里再把他来问,问声军爷几个人?

正德 (接唱)为军的一人一骑马,

李凤姐 (接唱)一人用不了这许多(的)银。

李凤姐 银子多了。

正德 (呃,)人的饭食,马的草料。

李凤姐 人的草料,马的饭食。(或:哦,马的饭食,人的草料。)

正德 哦! 马的草料,人的饭食。

李凤姐 还多呢。

正德 (哦,还多\ldots{}\ldots{})那就送与大姐你买花儿戴吧!

李凤姐 多谢军爷,

(正德 不消。)

李凤姐 (军爷)请呐。

正德 请到何处呀?(或:请到哪里?)

李凤姐 请到客堂。

正德 哦,去到卧房?(或:(急切介)呃,我正要到你的卧房!)

李凤姐 哦,客堂呐。

(正德 哦,客堂哦\ldots{}\ldots{})

(凤姐拿灯前行出门,正德跟随出门。凤姐向下场门走圆场,正德到下场门站住唤凤姐)

正德 哦,大姐,这是什么所在呀?

李凤姐 这是我哥哥的卧房。

正德 (呃,)不整洁呀。

(凤姐前行过上场门,正德到上场门问)

正德 (啊,大姐,)这又是什么所在?

李凤姐 这是我的卧房。

正德 呃,待我来瞻仰瞻仰。

(正德要进,凤姐拦)

李凤姐 (诶------)你可晓得男女有别?

正德 (哦,)男女有别,有趣,有趣呀。

(凤姐放灯小边台口,正德台口拿灯,凤姐过大边开门状)

正德 【四平调】龙行虎步客堂进,

(正德边唱边过大边,用扇挡灯,凤姐从里边回小边,正德唱完``客堂进'',凤姐推正进门,带上门)

正德 (啊,)大姐你怎么将门(户)关闭了?

(正德左手拿灯,右手拿扇,灯扇在身左侧平放,念``关闭了'')

(李凤姐 遇着你们这样人,这门呐是不得不紧。)

正德 哎呀,这梅龙镇上的门户,大姐,好紧呐。

李凤姐 呀啐!

(正德用扇在上边打一下门,下场门下,凤姐再带门)

李凤姐 (接唱)回头来带上两扇门,啊啊啊,掸一掸灰尘。

(凤姐擦桌摆酒,坐中间学饮酒,捋须学饮,泼酒(不倒回酒壶),擦杯,出位,请出吃酒)

(李凤姐 (接唱)一霎时酒席安排定,请出军爷饮杯巡。)

李凤姐 客官(或:军爷),请出来吃酒哇,客官(或:啊,军爷),请出来吃酒哇。

(凤姐拿盘在小边向外等,正德溜上)

正德 这梅龙镇上好高的房子呀。

李凤姐 房子房子我打你一盘子。

正德 你怎么打起为军的来了?

李凤姐
你看你这个人呐,进得我们店呐,上也瞧瞧,下也看看,难道说,我们女孩儿家有什么好看不成(么)?

正德
不是呀,大姐你长得标致,生得漂亮,为军的爱看呐!(或:不是哟,大姐你生得标致,长得漂亮,为军的我爱看呐!)

李凤姐 (怎么,)军爷(你)爱看?

正德 (呃,)实实的爱看。

李凤姐 那你就看上一看。(或:好,就请你看上一看。)

(凤姐面朝外,叫正德看)

正德 她倒大方起来了,我倒要仔细地看上一看。

(正德看凤姐,在大边看,先看胸再看头再看脚,用扇击掌)

正德 好。

李凤姐 你再看看。

正德 (哦,再看看,)再看看我就再看看。

(仍在大边看,不过去,看身后,先看头再看脚再看腰,用扇击掌)

正德 好好好。

李凤姐 你再来看上一看。(或:还要你看看。)

正德 看够了不看了。

李凤姐
我若不看你是我们店中的客人呐,我就要骂你呢!(或:我若不念你是我店中的客人,我啊,就要骂你!)

正德 你何必骂我?

李凤姐 不但骂你,我还要打你呢。

正德
(呃,)这倒巧得很,为军的出世以来还不曾挨过打,今日就借大姐的一双玉手打上几下,为军的倒要尝上一尝,你来打。

(正德侧身叫凤姐打,凤姐要举盘欲打又停)

李凤姐 待我来打,(或:哦,军爷叫我打?)

(正德 诶,你来打。)

李凤姐 (如此我就\ldots{}\ldots{})哎呀我不打了。

正德 为何不打(呀)?

李凤姐 我怕军爷你着恼哇。

正德 为军的不恼,你只管地打来。

李凤姐 (啊,)军爷不恼?

正德 不恼。

李凤姐 如此我就打、打、打。

(凤姐先用盘左边打,正德用扇一盖;凤姐右边打,正德又一盖;凤姐再左边一打,正德盖下去打凤姐头上大花;凤姐躲开左手叉腰,右手举盘看正德亮;正德闻扇子头,扇横、看,噗笑看凤姐,哈哈哈。凤姐下。正德归中间唱)

正德
【四平调】这佳人(或:这丫头)气性真清爽,恰似嫦娥下天堂。将木马紧紧连声响(或:忙将这木马连声响),

(正德进位击木马,凤姐上紧接唱)

李凤姐 (接唱)想是酒寒茶又凉。

(站桌小边,正德击木马三下,边击边叫)

正德 酒保、酒保、酒保。

李凤姐 茶寒了?

正德 茶(也)不寒。

李凤姐 酒冷了?

正德 酒也不冷。

李凤姐
茶也不寒,酒也不冷,你这样(或:你将我们这桌案)敲敲打打,打坏了桌子是要赔的(或:打坏了是要你赔的)。

正德 就是这张破桌?

李凤姐 嗯。

正德 漫说这张破桌,就是大姐你。

李凤姐 什么?

正德
哦哦,你们的店房,为军的我还包得起,赔得过哇。(或:呃呃,这张桌儿,为军的我也包得起,赔得过。)

李凤姐
话要讲明白些,客官唤我前来做甚呐?(或:你要讲明白些,军爷叫我何事?)

正德 哦哦,这席酒是哪个摆的?

李凤姐 是我摆的。可好哇?

正德 好,缺少两般物件。

李凤姐 哪两般物件(或:什么两般物件)?

正德 青楼女市装罗敷,红粉佳人学嫦娥。

李凤姐
客官是说青红萝卜,我们这里不上席的,客官要用,待我去取。(或:军爷说的是红白萝卜,我们这里是不上酒席的,军爷要用,待我与你取来。)

(凤姐要出门,正德叫转来,立即回来)

正德 转来(或:慢来慢来),不是这般物件呐。

李凤姐 什么物件?

正德 就是那穿红挂绿的姐儿们。(或:就是那穿红挂绿的大姐。)

(李凤姐 诶,军爷说的是那些姐儿们么?)

(正德 正是。)

李凤姐 啊,从先倒有。

正德 如今呢?

李凤姐
被官府查禁了。漫说没有,就是有,夜晚之间叫我们女儿家哪里去寻,哪里去找哇?(或:被官府查禁了。漫说无有,纵然是有,这半夜三更,叫我们女儿家到哪里去寻,哪里去找哇?)

正德 着哇,啊大姐,我们商量商量。

李凤姐 商量什么?

正德 就烦大姐你与我斟上一杯如何哇?

李凤姐 哦!

正德 谅无推辞的了哇。

李凤姐 我们卖酒的不卖手(或:我们卖酒不管斟酒)。

正德 斟斟何妨啊?

李凤姐 不能斟。

(正德 不斟?)

(李凤姐 不斟。)

正德 你当真的不斟?(或:斟是不斟?)

李凤姐 不斟。

正德 好,这酒我(是)不吃了,把银子来还我吧。

(凤姐要出门,正德叫转来又止)

李凤姐 好,待我与你取来。

正德
慢来(或:转来),你(来)看这酒席已被我吃残了,倘若你哥哥(回来)问你要钱,你用何言答对呀?

(凤姐背供,再转身白)

李凤姐
是呀,待我哄他一哄,客官,你们那里的老鼠是什么颜色的?(或:哦,是呀,酒被他用残,银子被他拿去,我哥哥回来,我拿何言答对?嗯,有了,我再哄他一哄。啊,军爷,你们那里的老鼠是什么颜色的?)

正德 自然是灰色的呀。

李凤姐 我们这里不同。

正德 (哦,)怎样不同?

李凤姐 是白色的,

(正德 怎么是白色的?)

李凤姐 你看你看,那厢出来了。(或:哎哟,你看,它来了,在那里。)

正德 在哪里?

李凤姐 在那里,在这里呢。

(凤姐指大边外场,正德跟着立起往外看。凤姐偷着斟酒,斟完说``在这里呢''。正德坐下看念)

正德 这杯酒是哪个斟的?

李凤姐 是我斟的,可好哇?

正德 像这样的斟法呢,十杯、八杯何足道哉。

(正德把酒泼桌外边)

李凤姐 要(我)怎样的斟法呢?

正德
(嗯,)要用大姐你的手斟上一杯酒,再用大姐你的手递与为军的手,为军的手(再)递与为军的口,吃了下去,那才算得。

李凤姐 我手上有糖?

正德 无糖。

李凤姐 有蜜?

正德 无蜜。

李凤姐 无糖无蜜你要的什么?(或:无糖、无蜜,为何叫我斟酒啊?)

正德 为军的要的(或:为军的喜的)就是这个样儿。

李凤姐 我恼的就是这个样儿。(或:我啊,就恼恨这个样儿。)

正德 你斟是不斟?

李凤姐 不斟。

正德 好(或:不斟倒罢),这酒我不吃了,拿银子还我。

李凤姐 好,待我来取。

正德
转来,你可晓得我这银子是哪里来的(或:呃,慢来,你可晓得我这银子的来路)?

李凤姐 难道说是打抢响马来的(不成么)?

正德
着哇,正是打劫响马来的,不犯事便罢,若是犯了事(或:不犯事便罢,倘若犯下事来),将你兄妹二人攀扯在内,贼咬一口入骨三分,这(银子啊,我是不要了,)酒我(也)不吃了,店我也不住了,我要走了哇。

(正德立走状,凤姐摇手拦)

李凤姐 慢来慢来,商量商量。(或:容我商议商议。)

正德 你哪个商量? (或:你与哪个商议?)

(正德坐下,靠凤姐看凤姐)

李凤姐 我心与口商量。

正德 哼,快去商量,为军的还等着吃酒呢。

(凤姐背供)

李凤姐
(哎呀,)哥哥哇哥哥,(你)今日也卖酒,明日也卖酒,这就是卖酒的下场头哇!

(边唱边斟、边递酒)

李凤姐 【四平调】没奈何(或:无奈何)斟上酒一樽,尊声军爷饮杯巡。

(正德右手拿扇接酒时搔凤姐手,凤姐躲揉手;正德用扇挡饮,说``干'',面向凤姐用扇托杯;凤姐抢杯,正德用扇盖凤手;凤姐扔杯桌上,躲)

正德 【四平调】接酒时故意儿风情送,看她知情不知情。

正德 干。

李凤姐 干你娘的心肝。

正德 你为何(或:你怎么)骂起来了?

李凤姐
人家好意为你斟酒,为什么故意搔我哇?(或:我好意与你斟酒,你怎么搔了我手一下。)

正德
这,(哦,想是)为军的这些天未曾跑马射箭,指甲养得长了,搔了大姐一下(可)也是有的。

李凤姐 我的指甲也是长的(或:我的指甲也长得长了),怎么搔不着你呀?

正德
哎呀呀,原来大姐是个(或:哎呀,大姐原来是个)好占小便宜的人呐,你来看,为军的一双粗手,就请大姐你来搔上几下可好哇?

(正德伸平双手向凤姐)

李凤姐
待我来搔,把手放平些。(或:待我\ldots{}\ldots{}嘿,你将手放平些。)

正德 (哦,)放平些。

李凤姐 如此我就搔搔搔,啐!

(正德双手平放,中指上翘;凤姐放平些后搔;正德上下点凤姐手心,双手拉凤姐双手;凤姐脱开;正德拉空,笑;凤姐啐唱)

李凤姐
【西皮快板】月儿弯弯照天下,问声军爷你哪里有家(或:问声军爷你哪里是家)?

正德 (接唱)【西皮快板】大姐休得细盘查,天底下就是我的家。

李凤姐 住了!一个人不住在天底下,难道说还住在天上头不成么?

正德 我这个住处与旁人不同啊。

李凤姐 怎么不同?

正德
我就住在北京城内,里面有个(或:里面一个)大圈圈,大圈圈里面有(一)个小圈圈,小圈圈里面还有个黄圈圈,我就住在那个黄圈圈的里面呐。

李凤姐 如此说来,我倒像是认得你。

正德 (呃,你认识我)是哪一个?

李凤姐 你是我哥哥的\ldots{}\ldots{}

正德 什么?

李凤姐 大舅子呀。(叫起【西皮快板】)

正德 嗯,休得胡言。

李凤姐
【西皮快板】骂一声军爷理太差(或:军爷做事理太差),不该调戏我们好人家。

正德
(接唱)【西皮快板】好人家来歹人家,不该斜插海棠花。扭扭捏捏多俊雅(或:扭扭捏,多俊雅),风流就在这朵海棠花。

李凤姐
(接唱)【西皮快板】海棠花来海棠花,倒被军爷耻笑咱。将花不戴丢地下(或:有花不戴丢地下),从今后不戴这朵海棠花。

(凤姐摘花扔台口要踩,正德出位拣拾花)

正德
(接唱)【西皮快板】李凤姐,做事差,不该践踏(或:不该踏扁;不该踏碎)海棠花。为军的将花忙拾起,来来来我与你插\ldots{}\ldots{}

(正德大边里右手拿花举,左手扇柄指花,凤姐小边外,二人对亮相)

正德 【西皮摇板】啊啊啊\ldots{}\ldots{}

(正德大边外右手平拿花,左手前举起扇柄指花,凤姐小边里,二人对亮相)

正德 【西皮摇板】插上这朵海棠花(或:插上这枝海棠花)。

(正德边唱边从外漫凤姐头,凤姐从里边,正到小边,凤姐大边,二人回来对望,正德转身插花在左耳旁)

李凤姐 【西皮快板】一见军爷戏谑咱,去到后面(就)躲避他。

(凤姐下,正德过大边望笑。)

正德 哈哈哈!

正德 【西皮摇板】任你上天把地下,孤王赶你到天涯。

(正德追下,凤姐跑上回身望开唱,正德追上)

{[}第二场{]}

李凤姐 【西皮散板】前面走的李凤姐,

正德 (接唱)后面跟随正德君。

李凤姐 (接唱)进得房来门关定(或:进得房来将门掩),

正德 (接唱)叫声大姐快开门(呐)。

正德 开门来,开门来!

李凤姐 这门(呐)是不开的了。

正德 何时才开?

李凤姐 等我哥哥回来(我)才开(呢)。

正德 你哥哥今晚不回来。

李凤姐 今晚不开。

正德 明天不回来。

李凤姐 明天也(是)不开。

正德 一辈子不回来。

李凤姐 一辈子(也)是不开的了。

正德
哎呀,倔强得可爱呀,待我来骗上她一骗(或:呵,我倒要骗上他一骗)。啊李龙哥你回来了,(你来看,)你这个店中啊,酒是寒的、茶是凉的,我不住了,算清账目我要走了。哦请了,请了。

李凤姐
这就好了,(哎呀,)哥哥回来了,哥哥在哪里,哥哥在哪里,哥哥在\ldots{}\ldots{}

(凤姐开门出门,先望下场门;正德偷偷进屋正面小座;凤姐望上场门,进门念完``哥哥在'',站小边)

正德 在这里。

李凤姐 好赶呐好跑。(或:好跑呐好跑。)

正德 好跑呐好赶。(或:好赶呐好赶。)

李凤姐
你这个人呐,前厅赶至后院(或:前厅赶到后院),后院赶到卧房,你前来做甚(或:是何道理)?

正德 (唉,)我来求大姐你打发打发。

李凤姐
哦呀呀,原来是个要饭的花儿,待我去拿几个钱,来打发打发。(或:待我取钱来打发于你。)

正德 转来,大姐你当真的不懂么?

李凤姐 这,我怕呀。(或:懂倒是懂,只是我\ldots{}\ldots{}怕呀。)

正德 怕什么?

李凤姐 我怕我哥哥回来。

正德 你哥哥回来有我哇。

李凤姐 有你就无有我了,你还是快快出去罢!

正德 我就是不出去。

李凤姐 你,你不出去,我,那我就喊叫。(或:你要不出去,我就要与你喊叫。)

正德 你喊叫(我)什么?

李凤姐 我就说你杀了人了。(或:喊叫你杀了人了。)

正德 (你来看,)我手无利刃,焉有杀人之心?

李凤姐 哎呀你的心比刀还狠呢(或:你那心比刀还厉害呢)。快快出去罢!

正德 我(是)不出去。

李凤姐 那我去喊叫。(或:你不出去,我还是与你喊叫。)

正德 任凭你去喊叫。

李凤姐 啊!(或:啊,乡\ldots{}\ldots{})

(凤姐出门喊,正德追出拦,回来)

正德 啊慢来慢来,我们商量商量。

李凤姐 无有什么商量,怕你不出去。(或:还商量什么?)

正德
哎呀且住!她当真的喊叫起来,唤来地方人役,那时君臣相见多有不便。也罢,待我与她说明,她若相信封她一宫;若是不信,孤打马走去。(或:我心与口商量------哎呀且住!这个丫头当真喊叫起来,惊动地保、人役,那时君臣相见多有不便。也罢,我不免对她说明,她若有福,封她一宫;她若无福,打马走去。)

(正德背供念,念完正德面小座)

正德 啊,大姐,

(李凤姐 军爷。)

正德 你可认识我?

李凤姐 我认识你。(或:我啊认得你呀。)

正德 是哪个?

李凤姐 你(呀)是大户长的兄弟、三户长的哥哥,你(呀,)是(个)二混账。

正德 啊休得胡言,我就是当今正德天子。

李凤姐 起开起开!(或:躲开,躲开,躲开!)

(凤姐轰正德离座站大边,凤姐正德面小座,搭腿)

李凤姐 你可认识我? (或:你可认得我啊?)

正德 你是凤姐,小丫头。(或:你不过就是一个酒大姐。)

李凤姐
哼,我哇,我是当今正德皇上的娘呀!(或:我啊,就是正德皇帝------他的娘啊!)

(正德轰凤姐起,正德又座,凤姐站小边)

正德 (呃------)放肆,有道是龙行有宝。

李凤姐 有宝献宝。

正德 无宝呢?

李凤姐 献你的现世宝。(或:看你的现世宝。)

正德 凤姐观宝。(或:大姐观宝。)

(正德站大边,去风帽,唱)

正德 【四平调】将飞龙帽罩忙摘定,避尘珠照得满堂红。叫一声呐凤姐来观宝,

(正德站椅大边侧,左足蹬椅,露出里边黄帔;凤姐上前看,用手去摸,正德投左袖轰开,站,唱收腿)

(正德 男女有别。)

正德 【四平调】哪一个大胆敢穿龙袍,啊,五爪金龙。

(凤姐背供唱)

李凤姐 呀!
【四平调】怪不得昨晚得一梦,五爪金龙落房中(或:真龙天子睡卧房中),我这里上前忙跪定(或:我这里向前忙跪定),万岁驾前去讨封(或:等候万岁将我封)。

(凤姐里面跪)

正德 下跪何人?

李凤姐 李凤姐(呀)。

正德 跪在为君的面前(或:跪在孤王的面前)做甚呐?

李凤姐 前来讨封啊。

正德 你方才说我是你哥哥的大舅子,我是不能封的(了)呀。

李凤姐
若是封了我,我哥哥岂不是你的大舅子了?(或:你若封了我,我哥哥就是你的大舅子了。)

正德 就是不封。

(李凤姐 当真不封?)

(正德 当真不封。)

(李凤姐 果然不封?)

(正德 果然不封。)

李凤姐 唉不封就罢。

(欲起,正德拦)

正德 慢来慢来,焉有不封之理,凤姐听封。

李凤姐 万岁!

正德
【四平调】那三宫六院俱封尽,封你闲游戏耍宫(或:封你的这龙游戏耍宫)。

李凤姐 (接唱)叩罢头谢主龙恩重。(或:叩罢了头来隆恩谢。)

(凤姐起,正德搀,正德大边、凤姐小边站)

正德 (接唱)用手搀起爱梓童。

李凤姐 (接唱)低声问万岁,因何无侍从?(或:我低声问万岁,倒要何往?)

正德 (接唱)孤王(或:为王)打马奔大同。

李凤姐 (接唱)就在这店中来巡幸? (或:就在这梅龙镇住一晚。)

(凤姐边唱边从外过大边,正德从里过小边)

正德 (接唱)游龙落在凤巢中。

李凤姐 万岁请呐。

正德 请到何处?

李凤姐 请到(我的)卧房。

正德 哦,我怕呀。

李凤姐 (啊,你)怕的什么?

正德 我怕你哥哥回来。

李凤姐 我哥哥回来有我。(或:我哥哥回来,有娘娘保驾。)

正德 如此(说来)凤姐。

李凤姐 我主。(或:君爷。)

正德 梓童。

李凤姐 万岁。

正德 噤声。

(正德打开扇阻回身向右一望,用扇遥推凤姐,二人合身,\textless{}\textbf{大锣打下}\textgreater{},凤姐前、正德后)

\newpage
\hypertarget{ux6cd5ux95e8ux5bfa-ux4e4b-ux8d75ux5ec9}{%
\subsection{法门寺 之
赵廉}\label{ux6cd5ux95e8ux5bfa-ux4e4b-ux8d75ux5ec9}}

\textbf{{[}第一场{]}}

\textbf{臣不敢,赵廉。}

\textbf{有罪不敢抬头。}

\textbf{谢千岁!}

\textbf{【西皮散板】小傅朋他本是啊杀人凶犯,}

\textbf{千岁!}

\textbf{【西皮散板】臣问他}\protect\hyperlink{fn565}{\textsuperscript{565}}\textbf{口供词件件招全。在公堂未动刑他自己招认,因此上臣将他拿问在监。}

\textbf{公公!}

\textbf{惭愧。}

\textbf{是(是是),二甲进士出身,焉有不识字的道理?}

\textbf{是是是。}

\textbf{``告状民女宋氏巧娇\ldots{}\ldots{}''}

\textbf{啊?这``巧娇''二字,好像在哪里会过(或:见过),怎么一时想它不起\ldots{}\ldots{}}

\textbf{哦,你就是宋国士之女,名唤巧娇的么?}

\textbf{你为何告此刁状?}

\textbf{先前为何不告?}

\textbf{如今呢?}

\textbf{着啊!}

\textbf{哦------}

\textbf{【西皮散板】谁知道小刘彪是杀人的凶犯呐,却原来这内中有许多牵连。在庙堂恕为臣呐才学\textless{}哭头\textgreater{}浅,千岁爷呀!}

\textbf{【西皮散板】望千岁开大恩限臣三天。}

\textbf{谢千岁!}

\textbf{{[}第二场{]}}

\textbf{你们都来了?}

\textbf{将刘媒婆带好,带马打道孙家庄去者。}

\textbf{{[}第三场{]}}

\textbf{锁了!}

\textbf{带了!}

\textbf{两厢搜来。}

\textbf{钢刀入库,绣鞋放下。}

\textbf{带刘媒婆!}

\textbf{勾奸卖奸,可是此物?}

\textbf{(带)下去。}

\textbf{带刘彪!}

\textbf{你在大街怎样讹诈傅朋,从实讲来。}

\textbf{下去。}

\textbf{带刘公道!}

\textbf{刘彪在大街讹诈傅朋,可有你解劝过来(或:你可曾解劝过来)?}

\textbf{下去。}

\textbf{带刘彪!}

\textbf{讹诈是实。孙家庄一刀连伤二命,定(然)是你这个奴才所做的了。}

\textbf{(来,)打!}

\textbf{怎么样?}

\textbf{抓了回来!}

\textbf{一刀连伤二命,还说初犯?!}

\textbf{我来问你,男尸有头,女尸无头,这人头往哪里去了?}

\textbf{带刘公道!}

\textbf{(唗,)身当乡约(或:身作乡约),隐藏人头不报,该当何罪?}

\textbf{来,打!}

\textbf{打!}

\textbf{打道硃砂井。}

\textbf{打捞人头。}

\textbf{人头有了,就好落案了。}

\textbf{哦------还有死尸一口!}

\textbf{快快地打捞上来!}

\textbf{上前验来。}

\textbf{带刘公道!}

\textbf{这井内的死尸是哪里来的?}

\textbf{来,打!}

\textbf{他叫什么名字?}

\textbf{何事喧哗?(或:什么喧哗?)}

\textbf{(这)死尸呢?}

\textbf{唉,本县的对头到了(或:唉,本县的对头来了)!}

\textbf{啊,宋先生------}

\textbf{抱尸痛哭,敢是相认?}

\textbf{啊?!既不相认,前来搅乱尸场?!}

\textbf{左右,轰了下去!}

\textbf{唤他回来。}

\textbf{无用的奴才!}

\textbf{宋先生(请转),宋先生请转\ldots{}\ldots{}}

\textbf{好奴才! (或:好狗才!)}

\textbf{【西皮散板】骂声公道老禽兽,身作乡约隐人头。硃砂井边下毒手哇,活活打死呃你这老蠢牛。}

\textbf{怎样打不得?}

\textbf{依你之见?}

\textbf{好,回衙有赏!}

\textbf{将一干人犯带好(或:带妥),与爷带马!}

\textbf{【西皮慢板】郿坞县在马上心神不定,这几天为人犯哪得安宁。劝世人休为官务农为本,可怜我七品的官不如黎民。实指望做清官【转西皮二六】高升一品,又谁知孙家庄起下祸根。孙玉姣习针黹(在)门前站定,有傅朋起下了爱慕之情。假意儿买雄鸡来把话论,就有个刘媒婆老不正经。他二人婚姻事自有媒证,何用你诓绣鞋在暗地里勾情?只骂得老乞婆羞愧难忍,转面来骂刘彪大胆畜生。黑夜里你一刀连伤二命,将人头胡乱丢移祸旁人。刘公道在衙门充当里正(或:在衙门身为里正),你为何见人头不打报呈?你三人}\protect\hyperlink{fn566}{\textsuperscript{566}}\textbf{问典刑(或:你三人问清楚)休来怨恨,这就是自作自受、王法森严难以徇情。教衙役将人犯一齐带定,}

\textbf{【西皮摇板】放大胆闯虎穴去见上呃人。}

\textbf{{[}第四场{]}}

\textbf{何事?}

\textbf{与我打!}

\textbf{这\ldots{}\ldots{}}

\textbf{也罢,将老爷的马与他乘骑。}

\textbf{(也)只好是步行的了哇。}

\textbf{呃------}

\textbf{【西皮快板】刘公道做事真胆大,身作乡约犯王法。打死兴儿反}讹诈,绝了那宋国士后代根芽。此一番去见千岁爷的驾,老无才准备下钢\textbf{呃}刀把你的头来杀。

\textbf{{[}第五场{]}}

\textbf{何事?}

\textbf{可曾递上?}

\textbf{呃,无用的奴才!}

\textbf{公公!}

\textbf{(来了。)}

\textbf{哦,带齐了。(或:哦,多谢公公。)}

\textbf{哦,是是是,有劳公公。}

\textbf{呃,有劳公公!}

\textbf{不是这样的投法,(还)要怎样投法呢?}

\textbf{呃\ldots{}\ldots{}}

\textbf{这里的(或:这里边)的银票呢?}

\textbf{哼,无用的奴才,还不取了过来!(或:呃,拿了过来!)}

\textbf{啊,公公!}

\textbf{呃,公公收下。}

\textbf{莫非嫌轻。}

\textbf{呃\ldots{}\ldots{}不敢不敢。}

\textbf{呃,有劳公公。}

\textbf{多谢公公。}

\textbf{呵,有劳了,有劳了。}

\textbf{啊,公公\ldots{}\ldots{}}

\textbf{呃,有劳了,有劳了。}

\textbf{是是是。}

\textbf{参见千岁!}

\textbf{千岁容禀:}

\textbf{【西皮摇板】一干人犯俱带妥,望求千岁作定呐夺。}

\textbf{谢千岁!}

\textbf{千岁在上(或:千岁在此),哪有为臣的座位?}

\textbf{哦,是是是,如此谢坐。}

\textbf{啊,公公请坐。}

\textbf{全仗千岁!}

\textbf{(遵命。)}

\textbf{带刘彪。}

\textbf{一刀连伤二命,按律凌迟。千岁开恩,问他斩罪。}

\textbf{千岁开恩。}

\textbf{带刘公道。}

\textbf{身当乡约,隐藏人头不报,打伤人命,按律当斩。千岁开恩,问他绞罪。}

\textbf{千岁恩德。}

\textbf{带刘媒婆。}

\textbf{且慢千岁,有道是``子大不由母''啊!}

\textbf{千岁开恩。}

\textbf{带傅朋。}

\textbf{带孙玉姣。}

\textbf{她乃黄花幼女,可以见得。}

\textbf{带宋巧娇。}

\textbf{呃,有来头。}

\textbf{大大的有来头。}

\textbf{啊!}

\textbf{多谢千岁。}

\textbf{她么\ldots{}\ldots{}呃,也见得。}

\textbf{见得,见得,见得\ldots{}\ldots{}}

\textbf{(比作何来?)}

\textbf{谢千岁!}

\textbf{(多谢公公。)}

\textbf{带马!}

\newpage
\hypertarget{ux4e00ux6367ux96ea}{%
\subsection{一捧雪}\label{ux4e00ux6367ux96ea}}

{[}第一场{]}

(龙套、校尉、严世藩上,汤勤迎,归坐)

严世藩 可恼哇可恼!

汤勤 大人下朝为何这等烦恼?

严世藩 适才在金殿与夏言老儿顶本,被他一本参倒,你道恼是不恼?

汤勤 这有何难,明日早朝老太师上殿奏本,稍带他几句,也就够他受用的了。

严世藩 言之有理,这几日不在府中,你往哪里去了?

汤勤,小人过衙谢官。

严世藩 看起来你倒是个有良心的。

汤勤 小人本来是有良心的,眼前有一人他无有良心呐。

严世藩 哪个无有良心?

汤勤 就是那莫大老爷。

严世藩 我那莫仁兄怎见得无有良心?

汤勤 那日莫大老爷献与大人的玉杯,是真是假?

严世藩 自然是真杯,焉有假杯之理。

汤勤 乃是假的。

严世藩 怎见得?

汤勤 那日酒席筵前,他将真杯显露出来,被小官看见了。

严世藩 依你之见。

汤勤 大人过府搜杯。

严世藩 搜杯得出。

汤勤 莫大老爷之罪。

严世藩 搜杯不出。

汤勤 小人之罪。

严世藩 起过了。

(汤勤下,严世藩立)

严世藩 【二黄散板】汤勤说话如刀切,舌尖杀人哪见血。人来顺轿把队列,

(严上轿,众领下)

严世藩 (接唱)过府去搜一捧雪。

{[}第二场{]}

(莫怀古、雪艳上)

莫怀古 【二黄散板】昨夜一梦大不祥,

雪艳 (接唱)老爷言来妾身详。

(八字小座)

莫成 (内白)走哇,

(莫成\textless{}\textbf{水底鱼}\textgreater{}上,进门,立小边禀报)

莫成 禀老爷:严爷到。

莫怀古 知道了,外厢伺候。

(莫怀古、雪艳下,莫成出门立大边外场,严众上,校尉小边一字)

严世藩 【二黄散板】来在莫府下了轿,

(严世藩下轿)

严世藩 (接唱)会会当年故旧交。

莫成 禀老爷,严爷到。(或:有请老爷,严爷下轿。)

(莫成下场门上)

莫怀古 【二黄散板】听说严爷到府门,整整衣冠礼相迎。

(莫怀古出门迎,严世藩众挖门进站门,严立大边,莫挖进,莫成跟进,站小边)

莫怀古 (接唱)莫不是升官少谢忱?

严世藩
你做\protect\hyperlink{fn567}{\textsuperscript{567}}的是嘉靖皇上的官,谢我何来?

莫怀古 (接唱)大人发怒为何因?

莫怀古 大人怒气不息,为着谁来?

严世藩 就为你来。

莫怀古 为下官何来?

严世藩 一捧雪进与不进但凭于你,为何假杯哄我?

莫怀古 杯子就有一只进与大人并无第二。

莫成 着哇。

严世藩 住了。

(莫成上场门暗下)

严世藩
【二黄散板】听一言来怒气生,骂声怀古太欺心。进京未满一月整,保你太常寺正卿。人来与我忙搜定,

(校尉两抄,莫成上场门上到小边台口又下)

严世藩 (接唱)掘土三尺再搜寻。

(又两抄,莫成下场门上到大边台口又下)

严校尉 玉杯无有。

严世藩 起过了。

严世藩 (接唱)搜杯不出面带红,失却当年故旧情。

(雪艳上场门暗上,挡脸)

严世藩 仁兄身后何人?

莫怀古 贱妾雪艳。

严世藩 请来相见。

莫怀古 夫人见过严爷。

(雪艳见礼)

雪艳 参见大人。

严世藩 仁嫂。

莫怀古 回避。

(雪艳上场门下)

严世藩 仁兄,真杯也罢,假杯也罢,拿将出来小弟一观,不要你的就是。

莫怀古 方才也曾言过,杯子一只进与大人并无第二。

严世藩 有人得见。

莫怀古 何人得见?

严世藩 汤勤得见。

莫怀古
汤勤,是了,那日汤勤过府谢官,在酒席宴前得罪于他也是有的,他在大人台前搬动是非,有道是傍耳之言不可深信。

严世藩 你待怎讲?

莫怀古 不可深信。

严世藩 住口!

严世藩
【二黄散板】听罢言来怒气生,我有一言听分明。朝里朝外问一问,严家岂是省油灯。人来与爷把轿顺,

(严世藩众出门、严上轿,莫怀古随出门送)

严世藩 (接唱)三日定要灭满门。

(严世藩下,莫怀古进门归大边立,雪艳上立小边)

莫怀古 夫人,莫成这个奴才往哪里去了?

雪艳 妾身不知。

莫怀古 两厢唤来。

(莫怀古、雪艳抄过、莫小边、雪大边,向里叫)

莫怀古 莫成。

雪艳 掌家。

(莫怀古、雪艳回身,莫正场、雪大边立)

莫成 (内白)走哇!

(莫成\textless{}\textbf{水底鱼}\textgreater{}上,进门,立小边侧)

莫成 老爷受惊了。

莫怀古 我受的什么惊?

莫成 哎呀老爷呀,可记得前日吃酒之事么?

莫怀古 动不动就是你老爷的酒,你敢戒你老爷的酒,待我打死你这个奴才。

(莫怀古打、莫成遮、雪艳拦)

莫成 老爷,纵然打死小人,可容小人讲个明白。

莫怀古 快快讲来。

莫成
容禀:小人方才见严爷下轿之时,气色有些不正,想必为的是一捧雪而来。是小人去至上房,扭开箱锁,揣了一捧雪,打从前门而走,

(双手外指)

莫成
不想前门有兵丁把守(双手阻介),小人也只得打从后门而逃(双手内指),又有严府校尉阻拦(双手阻介)。小人无奈,打从犬洞逃出。在鬻食棚躲避,见严爷上轿已走,才得回来。进得府门,老爷不问青红皂白,开口就骂,举手就打,看起来,像我这为奴的呀,唉,好难办的事呀!(哭介)

莫怀古 有了一捧雪还则罢了,若无一捧雪,夫人不必阻拦,待我打死这个奴才。

(莫怀古打,莫成遮,雪艳拦)

雪艳 待妾身向前,掌家!

莫成 夫人。

雪艳 你可知你家老爷发怒为了何事?

莫成 小人不知。

雪艳 就是为那一捧雪。

莫成 啊,一捧雪,在、在\ldots{}\ldots{}这里。

(摸左袖,摸右袖,摸胸掏出杯,双手举杯呈献,雪艳指杯,莫怀古看)

莫怀古 哦。

莫怀古 【二黄散板】一见玉杯果是真,好个伶俐小莫成。走向前来与掌家论,

莫成 老爷,小人挨不起了哇。(哭介)

莫怀古 (接唱)错打几下莫挂心。

莫怀古 将玉杯收好。

莫成 是。

(揣杯)

莫怀古 有了一捧雪,拿稳作官,怕他何来。

莫成 着哇!
有了一捧雪,拿稳作官,还怕他何来。啊老爷,但不知严爷上轿之时(或:临行之时),讲些什么?

莫怀古 讲了两句淡话。

莫成 哪两句话?

莫怀古 三日定要灭满门。

莫成
三日要灭满门(或:``三日之后,灭却尔满门''),哎呀老爷呀,这三日要灭满门(或:这三日之后),是灭夫人的满门,还是灭小人的满门,一定是灭老爷的满门呐。

莫怀古 怎么讲?

莫成 老爷的满门呐。

莫怀古 哪个?

莫成 呵,(是)老爷呀。

莫怀古 哎呀!

(昏坐,莫成、雪艳左右)

莫成 老爷醒来。

雪艳 老爷醒来。(同莫成)

莫怀古 【二黄导板】听说三日灭满门,

(立,\textless{}\textbf{三叫头}\textgreater{})

莫怀古 夫人。

雪艳 老爷。

莫怀古 莫成。

莫成 老爷。

莫怀古 夫人呐!

雪艳 老爷呀!

莫怀古 【二黄散板】吓得三魂少二魂。向前忙对掌家论,快想良谋救我身。

莫成 这,有道是不做他人官,不受他人管,不如弃官走(或:弃官逃走)了罢。

莫怀古 走得的。

莫成 啊,走得的。

莫怀古 如此回转钱塘。

莫成 且慢,哪个不知老爷是钱塘人氏,他必然往钱塘追赶。

莫怀古 那往何处而去。

莫成
小人跟随老爷进京之时打从海岱门前经过,遇着一位穿红袍的官员他姓什么戚。

莫怀古 敢是那戚继光?

莫成 不错,正是那戚继光戚大老爷,他今在何处为官?

莫怀古 蓟州总镇。

莫成 你我主仆(就)往蓟州而逃。

莫怀古 如此吩咐外厢车马伺候。

莫成
哎呀老爷呀,事到如今还要什么车马呀,必须换了亵衣小帽,混出城去再作道理。

雪艳 喂呀\ldots{}\ldots{}

莫怀古 下官连累你了。

(\textless{}尾声\textgreater{}前段,雪艳下)

莫怀古
\textless{}\textbf{叫头}\textgreater{}莫成你老爷进京未满一月,这身荣耀怎能舍得。

莫成
\textless{}\textbf{叫头}\textgreater{}哎呀老爷呀,事到如今舍不得也要(翻袖内转身,莫怀古一望)舍,丢不得也要(翻袖外转身,莫怀古两望)丢,舍、丢了、请。(一轰两轰,拱手,莫怀古下)咳!

(投袖、抓袖,\textless{}尾声合头\textgreater{}莫成下)

{[}第三场{]}

(严世藩众上、汤勤迎,严下轿,进门,严坐中间,汤勤小边立,校尉站门)

严世藩 将汤勤绑了。

汤勤
且慢!启大人,他若是真杯献与大人必然拿稳做官,他若假杯献与大人必然弃官逃走,且听校尉一报,再绑不迟。

严世藩 且听校尉一报。

(张龙、郭义进门,站拜,立大边)

张龙、郭义 参见大人,莫怀古弃官逃走。

汤勤 大人如何?

(严世藩立)

严世藩
莫仁兄呐,真杯也罢,假杯也罢,只管拿稳做官,不该弃官逃走,来,顺轿。

汤勤 大人哪里去?

严世藩 追赶莫仁兄回来做官。

汤勤 他如今做不得官了。

严世藩 依你之见。

汤勤 必须行文各处将他捉拿,大小治他一个罪名。

严世藩 待我修文。

汤勤 小人溶墨。

(严世藩大坐修文,汤勤立小边磨墨)

严世藩
太子少保兵部左侍郎严,票行阃外事,为犯官一名莫怀古怀带皇家印信弃官逃走,有欺君误国之罪,命马上二校尉沿途追赶,不论文武大小衙门拿获者\ldots{}\ldots{}

汤勤 斩头解京。

严世藩 我那莫仁兄哪有这么大的罪过?

汤勤 这是他自作自受,哪个混账王八羔子害他不成。

严世藩 咳,斩头解京。来,公文一角,沿途追赶,不得有误。

汤勤 你们必须往蓟州追赶。

张龙、郭义 遵命。

(张龙、郭义出门)

严世藩 转来。

(张龙、郭义回来)

严世藩 莫怀古事小,一捧雪事大。

(张龙、郭义出门,汤勤跟过去)

汤勤 二位上差,一捧雪事小,雪娘子事大。

张龙、郭义 哼!

(张龙、郭义下,汤勤回来)

严世藩 汤勤随我饮酒来。

(严世藩、汤勤同下)

{[}第四场{]}

(\textless{}\textbf{水底鱼}\textgreater{}张龙、郭义上)

张龙、郭义 俺。

张龙 张龙。

郭义 郭义。

张龙 请了。

郭义 请了。

张龙 奉了严爷之命,追赶莫怀古夫妇,就此马上加鞭。

(\textless{}\textbf{撤锣}\textgreater{}同下)

{[}第五场{]}

莫成 (内白)趱行。

(\textbf{小锣}\textless{}\textbf{柳青娘}\textgreater{}莫成背包袱上扎犄角,莫怀古、雪艳跟上,九龙口雪艳哭,坐下)

雪艳 喂呀\ldots{}\ldots{}

莫怀古 夫人为何不走?

雪艳 两足疼痛难以行走。

莫怀古 莫成。

(莫成回身)

莫成 老爷何事?

莫怀古 夫人两腿疼痛难以行走,如何是好?

(莫成望)

莫成 此地离蓟州西门不远,待小人去至前面,雇乘小轿前来

(莫成望)

莫成 老爷事要小心呐。

(莫成包袱交莫怀古,莫怀古望)

莫怀古 你要小心呐,柳林相会。

莫成 是。

(莫成下)

莫怀古 夫人,掌家前去雇轿,待下官搀扶于你,柳林躲藏。

(莫怀古搀雪艳到大边里面,雪艳坐,莫怀古立,张龙、郭义上小边立)

张龙 他们在前面走,我们在后面赶,赶至此地为何不见?

郭义 想必柳林躲藏,你我冒叫一声。

张龙、郭义 (同叫)里面可是莫大人?

雪艳 老爷外面有人唤你。

莫怀古 是哪一位?

张龙、郭义 你是莫怀古,锁了。

(张龙、郭义押莫怀古、雪艳下,郭义拿莫怀古包袱)

{[}连场{]}

(张龙、郭义、莫怀古、雪艳上叫城同下,张龙上击堂鼓,四役旗牌引戚继光上)

戚继光 辕门鼓角声高,想是公文来到。

(戚继光上高台,两层桌,桌后两椅背对背,戚跨椅上站桌后,张龙立大边)

张龙 大人请了。

戚继光 请了。

张龙 上司行文大人观看。(递公文)

戚继光 当堂拆封,人犯可曾带齐?

张龙 人犯带上。

(郭义、莫怀古、雪艳上,莫怀古大边,郭义、雪艳小边立)

雪艳 喂呀\ldots{}\ldots{}

莫怀古 夫人不必害怕,来此戚贤弟衙料也无妨,上面敢是戚\ldots{}\ldots{}

戚继光 嗯,本镇点名哪怕你们不齐,听点。

旗牌 犯官一名莫怀古。

莫怀古 有。

旗牌 女犯无名。

戚继光 带下去。

莫怀古 夫人,事到如今连戚贤弟也不认你我了。

雪艳 这都是你交的好朋友。

(莫怀古、雪艳同下)

戚继光 他二人哪里拿获的?(戚继光记录)

张龙 西门以外柳林之下。

戚继光 什么时候?

张龙 黄昏时候。

戚继光 怎样进城?

张龙 叫开城门,批了箚子,来见大人。

戚继光 本镇看来,此事重大,必须做一个两下担待。

张龙 何谓两下担待?

戚继光
头门以里,仪门以外,有军牢小房,里面有火,外面有锁,锁上加封,将你等四人锁在一起,待等五鼓天明,看着绑,看着斩,人头打入木桶,回复严爷。

张龙 好却好,只是我等二人辛苦。

戚继光 自有你二人的下程。

(张龙、郭义四役同下)

戚继光
且住,想我那莫年兄不知为了何事冒犯严府,想他有一掌家名叫莫成,颇能办事,为何不跟随前来,是了,想是人烟众多挨挤不上也未可知,不免去至大街之上寻找便了,来,掌灯。(更)。

戚继光 【二黄散板】人来掌灯大街进,前去寻找小莫成。

(\textless{}\textbf{撤锣}\textgreater{}旗牌、戚继光下,更夫上)

更夫
为人莫当差,当差不自在。风里也得去,雨里也得来。我更夫便是。只因夜间拿住犯官莫怀古,五更天明就要开刀,街前严禁。就此巡更去者。

(\textless{}\textbf{水底鱼}\textgreater{}更夫走,莫成上相撞)

莫成 (内白)走哇。

更夫  拿住了。

莫成 拿住了什么?

更夫 拿住犯夜的了。

莫成 我不是犯夜的呀。

更夫 你不是犯夜的,你是谁呀?

莫成 我是乡下人呐。

更夫 乡下人不犯夜。难道说城里的人才犯夜吗?

莫成 呵大哥,我是前来交粮米的呀。

更夫 交钱粮的,文官衙门去交,你怎么跑这儿来拉?

莫成 这是哪个衙门?

更夫 这是戚大人的衙门。

莫成 为何这样的热闹?

更夫 你不知道,今夜拿住犯官一名叫莫怀古,五更天明就要开刀问斩。

莫成 咳,老爷呀。(哭介)

更夫 你哭哪一门子呀?

莫成 人人都说这位莫大老爷为官清正,我故而心中酸痛呐(或:在此叹息呀)。

更夫 你这不是看兵书落泪,替古人担忧吗。

莫成 本来是清官呐。

更夫 你有住处吗?

莫成 无有哇。

更夫
这么办吧,你先到我更房里去,到天明亮了你再去交钱粮,哎呀可是我的地方窄小哇。

莫成 方便方便吧。

(二人摸)

更夫 你在哪里呢?

莫成
呵呵呵,这里来,(我)在这里。(或:呃呃呃,在这里,这里来,呃,在这里。)

更夫 你多大岁数了?

莫成 我四十六岁了。

更夫 老头子呀,你那边去,我这边,(进更棚坐)替我听着点儿更呐。

(二更\protect\hyperlink{fn568}{\textsuperscript{568}},旗牌、戚继光上)

戚继光 【二黄散板】听谯楼打罢了二更时分,八台官倒做了巡更之人。

莫成 (唉,)老爷呀\ldots{}\ldots{}

戚继光 (接唱)啼哭之人哪一个?

(莫成出来)

莫成 我是莫\ldots{}\ldots{}

戚继光 噤声。(拉莫成下)

更夫 到时辰了,哎!。打更锣锤不见了,拿脑袋撞,不怕不怕。

(更夫下,旗牌、戚继光、莫成上)

戚继光 (接唱)来在二堂问分明

莫成 叩见大人。

(莫成跪)

戚继光 罢了,你家老爷来了。

莫成 怎么我家老爷也来了么,可容我主仆一见?

戚继光 容你主仆一见。

莫成 谢大人。

戚继光 下面伺候。

莫成 是,这就好了。(莫成下)

戚继光 来。

旗牌 在。

戚继光 看看严府校尉可曾睡着?

(旗牌望上场门,三更)

旗牌 睡着了。

戚继光 悄悄启开封锁,有请莫大老爷。

(旗牌开门)

旗牌 有请莫大老爷。

(【小拉子】莫怀古、雪艳上,旗牌带门,雪艳哭)

莫怀古 【反二黄散板】夫人啼哭莫高声,休要惊动严府人。悲切切且把二堂进。

(莫怀古、雪艳进门,戚继光、旗牌大边,莫中间,雪小边,均立)

戚继光 (接唱) 【二黄散板】披枷带锁为何情。(白)仁兄你掌家莫成来了。

莫怀古 在哪里?

戚继光 请莫掌家。

旗牌 莫掌家,你家老爷来了。

莫成 来了。(上,进门小边立)

莫成  老爷在哪里,老爷受惊了。

莫怀古 你这奴才办得好事呀。

莫成 事到如今埋怨小人可也是枉然(的)了哇。

戚继光 是呀,事到如今 掌家也是枉然了。但不知仁兄因何冒犯严府?

莫怀古 就为的是一捧雪。

戚继光 一捧雪乃是小事,为何有紧急公文到来?

莫成 怎么,这紧急公文(或:这行事的公文也)到了么?

莫怀古 来得好快,待我一观。

戚继光 不看也罢。

莫怀古 看了也好做一准备。

莫成 是呀,看了(或:看过)也好做一准备呀。

戚继光 旗牌掌灯,仁兄请看。。

(旗牌掌灯,众台中间前边看文)

莫怀古
太子少保兵部左侍郎严,票行阃外事,为犯官莫怀古怀带皇家印信,弃官逃走有欺君误国之罪,命马上校尉沿途追赶,不论文武大小衙门拿获者\ldots{}\ldots{}

(戚继光抢文)

莫怀古 为何不教我看了?

莫成 呵大人,为何不教我家老爷观看呐?

戚继光 恐怕仁兄看了心惊。

莫怀古 看了也好做一准备。

莫成 是呀,看了也好做一准备呀。

戚继光 如此仁兄看来。

莫怀古 拿获者斩头解京。哎呀!

莫成 哎呀!

(戚继光拿公文交旗牌,旗牌下场门下,莫怀古吓昏,气椅)

戚继光 仁兄醒来。

莫成 老爷醒来。(同戚)

莫怀古 【二黄导板】听说斩头要解京,

(众哭介)

莫怀古 (接唱)【二黄散板】好似钢刀刺我心。回头再对贤弟论,

(莫怀古揖\textless{}\textbf{乱锤}\textgreater{},戚继光扶)

莫怀古 (接唱)快想良谋救我生。

戚继光 仁兄,有道是不做他人官,不受他人管,不如弃官逃走了罢。

莫成 哦,走得的(么)。

戚继光 走得的。

莫成 如此说来,(我们)走哇!

(莫成、莫怀古、雪艳、戚继光出门一串往下场门走,站住,莫成拦)

莫成 走不得,走\ldots{}\ldots{}不得。

(莫成翻回来领众进门,站原地)

莫成
哎呀大人呀!想我家老爷(就是)为的弃官逃走才惹下这场杀身大祸,如今又要弃官逃走,岂不连累戚、戚\ldots{}\ldots{}大人,走、走\ldots{}\ldots{}不得。

戚继光
\textless{}\textbf{叫头}\textgreater{}也罢,不如你我点动人马反了罢!

莫成 反得的。

戚继光 反得的。

莫成 如此就反呐。

(莫成举拳领众往上场门走,站住,莫成拦)

莫成 反不得,反、反\ldots{}\ldots{}不得。

戚继光 怎么反不得?

莫成 请问大人,这蓟州有多少人马?

戚继光 三千人马,五百守城军。

莫成
着哇,这三千人马,五百守城军,离乱年间尚可抵挡一阵,这太平年间慢说是交锋打仗,就是垫马蹄也是不、不\ldots{}\ldots{}够哇,反、反\ldots{}\ldots{}反不得。

戚继光 咳!
【二黄散板】叫你反来你不反,叫你逃走你不行。待等五鼓天明亮,我坐法堂你受刑。

雪艳 (哭介)喂呀\ldots{}\ldots{}

(莫怀古、戚继光、雪艳分坐中,大、小两边)

莫成 \textless{}\textbf{叫头}\textgreater{}老爷,大人,哎夫人呐!

(面里正中)【二黄导板】一家人只哭得如酒哇

(撕领子,转身向外亮住)

莫成 (接唱)醉呀。

莫成 \textless{}\textbf{叫头}\textgreater{}老爷,大人,哎夫人呐!

(\textless{}\textbf{四击头}、\textbf{撕边一锣}\textgreater{}莫成转身到大边台口弓箭步,右手推胡子望雪艳,雪艳哭)

莫成 (接唱)【二黄原板】那一边哭坏了雪氏夫人。

(``那一边''往里一指,到唱``夫人''时画圈里指外甩胡须,三番,\textless{}\textbf{大锣搓锤}\textgreater{},三搓手边向小边走,摆袖、投袖、到小边台口翻袖望戚继光)

莫成 (接唱)
【二黄原板】戚大人八台官救不了家主爷的命,家主爷的命,老爷呀,实实的难坏了小莫成。

莫成
(白)且住!事到如今我倒想起一桩心事(或:一桩事故)来了,曾记得我家老爷进京之时,大夫人在钱塘府中,手举(或:手执)斗酒叫道:莫成呐掌家,此番跟随你家老爷进京,
事要正办,酒要少饮,要办上他几桩好事小心回来,慢说是你家老爷要另眼看待于你,就是夫人在钱塘也要好好照看你那文禄孩儿喏\ldots{}\ldots{}如今我家老爷进京,得罪了汤勤狗男女,害得我家老爷有这样的杀身大祸,事到如今,难道说教我这为奴的呀,看水流舟不成。哎呀!

\begin{quote}
(\textless{}\textbf{乱锤}\textgreater{}手比画往里半转身)咳!(再起\textless{}\textbf{乱锤}\textgreater{}搓手,想)事到如今,我又想起一桩心事来了,曾记得跟随我家老爷,去到长街拜客,遇着一位相面的先生与我家老爷看了一相,然后又与我觑了一觑,那(相面的)先生说道:莫大哥哇莫掌家,你的好贵相呵,这相,你有(或:倒有)你(家)老爷之相,可惜无有老爷之福哇,老爷日后有一桩大事,必然应在掌家你的身上。呜哙呀,想那相面的先生他说之无意,我倒是听之有记呀,莫非今日就应在这蓟\ldots{}\ldots{}州堂上?(看莫怀古,托胡子看自己,\textless{}\textbf{乱锤}\textgreater{},走到莫怀古座左边单腿跪托莫怀古胡子看,右手挡向外托自己胡子看,握着双手回到台口小边,\textless{}\textbf{叫头}\textgreater{})哎呀!想我莫成生在世上,无非是奴仆而已,怎能(或:焉能)有个出头之日,不如今晚(或:今日)在这蓟州堂上替我家老爷一死,日后也落得个青史名标,(这)万古流芳,我就是这个主意呀,(右拳击左掌)哦,我\ldots{}\ldots{}就是这个主\ldots{}\ldots{}意呀\ldots{}\ldots{}

【二黄散板】走向前来忙跪定,(走过去,跪戚继光面前偏右些)
\end{quote}

莫成 (接唱)【二黄散板】我家老爷有救星。(白)禀大人,我家老爷有救。

戚继光 救在哪里?

莫成 只要大人开天高地厚之恩,我家老爷就有救。

戚继光 起来。

莫成 谢大人。

(起回小边,戚继光向莫怀古)

戚继光 仁兄醒来。

(莫怀古、雪艳站起,莫怀古中间,戚继光、雪艳左右)

莫怀古 贤弟何事?

戚继光 仁兄有救了。

莫怀古 救在哪里?

戚继光 莫成言道仁兄有救。

(莫怀古向莫成)

莫怀古 莫成,你家老爷救在哪里?

莫成 哎呀老爷呀!事到如今还有什么救哇,小人情愿替老爷一死。

莫怀古 呵,想世上哪有人替人死之理,你有这两句话可也就够了。

莫成 老爷若是不信,小人有一辈古人说与老爷、大人、夫人听。

莫怀古、戚继光、雪艳 你且讲来。

莫成
容禀:昔日有一杨生最好育犬,一日杨生酒醉睡卧在荒山,(右手枕状,身侧卧状)彼时有一牧童他不晓得事务,他就放火(双翻袖过大边,面斜向外举双手,右先左后轰袖)烧荒,看看那火就烧在杨生的身上,(双手下指,边指边回小边)那黄犬是慌忙无计(或:忙中无计),只得跳下涧去呀,(双手胸前平向右侧画圈指)滚草\protect\hyperlink{fn569}{\textsuperscript{569}}救火(或:湿透毛衣,滚火救主;或:湿透毛衣,滚草救火)(右手左侧身边投袖,左手右侧身边投袖,双投)。那杨生醒来,只见那黄犬就累死在身旁(双手下指),彼时杨生捶胸顿足(或:捶胸跌足)(比画),仰天叹曰(或:对天叹曰),马有垂缰之德,羊有跪乳之恩,乌鸦有反哺之义,犬有救主之心,何况小人乃是人乎,老爷今日不教小人替死,我就碰(台口翻右袖挡脸,三人拦)死在这蓟州堂上。

莫怀古、戚继光、雪艳 不必如此。

(换巾子,莫成带链子,扶莫成里边正坐昏介,莫怀古中、戚继光、雪艳左右三人跪朝里拜)

莫怀古、戚继光、雪艳 【二黄导板】莫成请上礼恭敬,拜你如同拜先人。

莫成 【二黄导板】未曾犯法先上刑,

(莫成立,推莫怀古正坐,成到小边台口面外)

莫成
(接唱)【二黄散板】我、我\ldots{}\ldots{}犹如去到(或:犹如来到)哇那枉死城。眼望呵,钱塘哭文禄,喂呀我的儿啊!苦命的娇儿无靠承。走向前来把话禀(或:走向前来忙告禀),(往里走到莫怀古左侧跪)恕小人有言要禀明(或:有话要禀明)。

莫怀古 你且讲来。

莫成
哎呀老爷呀!小人今日替老爷一死,并无牵挂,只有一个文禄孩儿在钱塘伺候大相公在学中攻书,那大相公是性情不好哇,比不得老爷待小人这样恩厚,倘有不到之处,那大相公开口就骂,举手就打(或:举手就打,开口就骂),想我那文禄孩儿他是三岁丧母,今日小人替老爷一死。看看他是七岁丧父哇。望求老爷另眼照看我那薄命的孩儿,小人纵死九泉,唉,也就瞑目了哇。

莫怀古 莫成,日后我若错待你那孩儿,叫我天诛地灭。(发誓一跪)

莫成
谢老爷。(立)【二黄散板】文禄孩儿有了靠,纵死九泉也甘心呐。水流,千遭,归大海呀\textless{}\textbf{哭头}\textgreater{},(掏杯)原物交还旧主人。

(交杯,莫怀古接)

莫怀古
(接唱)【二黄散板】玉杯本是起货根,为你伤了小莫成。一怒将杯来倾碎,

(莫怀古摔杯,戚继光拦,接过杯)

戚继光 (接唱)【二黄散板】倾杯犹如欺先人。

莫怀古 将此杯寄在贤弟衙内,日后见杯如见愚兄一般。

雪艳 喂呀\ldots{}\ldots{}

莫怀古 贤弟请上受兄一拜。

戚继光 施礼为何?

莫怀古 将贱妾雪艳寄在贤弟衙内,休当贱妾看待,当作使女丫鬟。

戚继光 不敢,仁嫂看待。

莫怀古 多谢贤弟。

莫成 老爷,还有小人呢。(哭介)

莫怀古 贤弟受愚兄全礼。

戚继光 全礼为何?

莫怀古
待到五更天明将我恩人斩首,赏下棺木一口,将他成殓起来,埋在西门以外柳林之下,立一碑碣上写明故太常寺正卿莫公之墓,日后我那儿孙也好与他烧钱化纸。

莫成 谢老爷。

莫怀古 【二黄散板】三件大事托付你,

(打四更,莫成数)

莫成
(接唱)【二黄散板】又听得谯楼打四更。(白)哎呀老爷呀!谯楼已打四更(或:谯楼已打四鼓),看看五鼓天明,难道说这蓟州堂上还有两个莫怀古不成。

戚继光
\textless{}\textbf{乱锤}\textgreater{}\textless{}\textbf{叫头}\textgreater{}仁兄!小弟有一好友在古北为官,待弟修书一封,仁兄去往那里躲避躲避。

莫怀古 多谢修书。

戚继光 仁兄更衣。

(莫成 老爷后面更衣。)

(莫怀古、雪艳下)

莫成 小人磨墨。

(莫成磨墨,戚继光立桌大边修书)

戚继光
【二黄散板】上写顿首三顿首,拜上古北魏参谋。怀古本是我好友,还望仁兄好收留。一封书信忙修就,仁兄快快离蓟州。

(莫怀古换装,雪艳同上)

(莫怀古下。戚继光写书信。排子。莫怀古上,戚继光交书信。)

莫怀古
多谢了。【二黄散板】多谢贤弟施恻隐,搭救愚兄命残生。回头便对夫人论,下官言来听分明。五鼓天明时刻到,你向莫成叫夫君。

莫成 小人不敢。

莫怀古 (接唱)辞别贤弟足踏镫,

(旗牌带马,莫怀古扶马)

莫成 老爷请转呐\ldots{}\ldots{}

莫怀古 (接唱)莫成起下追悔心?

(莫怀古回来,旗牌带马一边)

莫怀古 莫成敢是有追悔之意,来来来将刑具与我戴上。

莫成
哎呀老爷呀!小人焉有追悔之意,(想)老爷此番去至古北(或:去至湖广),事要正办,酒要少饮,当交的交上他几个,不要像那汤勤狗男女,害得老爷这样(的)杀身大祸。今日在这蓟州堂上,有我这样一个不怕死的莫成替老爷一死,老爷日后,再有这样的祸事,再想这第二个莫成呐,只怕是无有了。

莫怀古 话是好话,可惜你讲迟了。

莫成 也还不迟,老爷快快上马去吧。

(旗牌带马,莫怀古上马)

(同\textless{}\textbf{三叫头}\textgreater{},莫怀古下,旗牌下,戚继光、雪艳、莫成进门)

雪艳 (哭)喂呀!

戚继光 【二黄散板】仁嫂休要珠泪汪。

雪艳 【二黄散板】全凭大人作主张。

(雪艳下)

莫成 大事全仗戚总镇呐,

戚继光 (白)莫成,

戚继光
【二黄散板】你的美名天下扬。(白)莫成,少时五鼓天明,去到法场,不要胡言,不要乱语,你老爷性命,本镇前程,俱在你身上。

莫成
大人,少时五鼓天明,去至法场,小人我一不、不敢胡言,二不、不敢乱语,只求大人与小人一个快。

(跪求,戚继光搀,退下,莫成转身面里中间)

莫成
\textless{}\textbf{叫头}\textgreater{}文禄,我儿,儿呀!(转身面外)儿在钱塘,今日也盼为父回来,明日也盼为父回去,盼来盼去将为父盼到枉死城中来了哇\ldots{}\ldots{}\textless{}\textbf{叫头}\textgreater{}文禄,我儿,难得见儿呀!(哭介)呀呀呸!想我莫成今日替我家老爷一死,日后纵然不能青史名标,也会落得个流传百世,此乃是一桩喜事呀,我是痛的什么,诶,我\ldots{}\ldots{}又哭\ldots{}\ldots{}的什么,(或:我是哭的什么,诶,我\ldots{}\ldots{}又痛\ldots{}\ldots{}的是什么?)我\ldots{}\ldots{}必须要笑哇,嗯哼,我是要笑哇。\textless{}\textbf{叫头}\textgreater{}哈\ldots{}\ldots{},哈\ldots{}\ldots{}呵\ldots{}\ldots{}儿呀!(打五更,比势做右手扶脖,坐子,起来)罢。

(下)

{[}连场{]}

(起牌子\textless{}吹打\textgreater{}四龙套,张龙、郭义骑马,禁卒拿桶,旗牌,戚继光骑马过场。四校尉,二刀斧手,莫成披红官衣、纱帽,招子绑上,扎犄角,回来引雪艳上,张龙、郭义、戚继光同上高台,牌子停)

戚继光 (念)本镇坐法堂,摆下杀人场,若有冤枉事,全仗一炉香。

\begin{quote}
本镇,戚继光。奉了严大人之命,监斩莫怀古,刀斧手,将莫怀古绑上来,
\end{quote}

(莫成原人上,成面里)

戚继光 二位看得清,验得明。

(戚继光点招子,刀斧手去莫成乌纱、官衣,成面外)

莫成 天呐天,想我莫\ldots{}\ldots{}

戚继光 刀斧手,将莫怀古绑好了!

雪艳 老爷你那心中要放明白些呀!

莫成 莫\ldots{}\ldots{}怀古,死得好不瞑目哇\ldots{}\ldots{}

(起\textless{}千秋岁\textgreater{}圆场,莫成蹉步、雪艳跪蹉,押成下,牌子完。二刀斧手上,头交雪,雪哭、咬头毁容,禁卒抢头、打入木桶交张龙、郭义,雪、禁下)

戚继光
二位公差,此案已毕,文书一轴回复严爷,外有手本问候大人金安,二位辛苦。

张龙、郭义 谢大人。(张龙、郭义下)

戚继光 军士们,回衙门。

(\textless{}\textbf{尾声}\textgreater{}同下)

\textbf{审头刺汤}\protect\hyperlink{fn570}{\textsuperscript{570}}

\textbf{{[}第一场{]}}

\textbf{陆炳 仁兄啊!}

\textbf{陆炳
【四平调}\protect\hyperlink{fn571}{\textsuperscript{571}}\textbf{】在金殿领了万岁命,总理天下冤枉情。最可叹仁兄死得苦,天网恢恢不差毫分。}

\textbf{陆炳
老夫陆炳,官居锦衣卫正堂(或:官居锦衣卫大堂之职)。适才朝罢而归,圣上命我审问莫怀古的人头,我想这个人头分明是真,若是问成假的(或:若是断成假的),我那故友死在九泉(之下),也是不能甘心(或:不能瞑目)。思想此事,好不教我为难也!}

\textbf{内 汤老爷到!}

\textbf{门子 启老爷:汤老爷到。}

\textbf{陆炳
且住,我正在为难之际,听说汤老爷驾到。我想汤勤乃是严府的耳目,他此番前来,我必须要留心在意。来,}

\textbf{门子 有。}

\textbf{陆炳
传话下去:说老夫有王命在身,二堂不能叙话,请汤老爷到大堂一叙。}

\textbf{门子 是。}

\textbf{陆炳 吩咐开门!}

\textbf{(陆炳下)}

\textbf{门子
大人传话下来:有王命在身,二堂不能叙话,请汤老爷到大堂一叙。}

\textbf{内 啊!}

\textbf{门子 开门。}

\textbf{(门子下)}

\textbf{{[}第二场{]}}

\textbf{(四红文堂、刽子手两边上,门子上,陆炳上,入座)}

\textbf{陆炳 来,有请汤老爷!}

\textbf{门子 有请汤老爷。}

\textbf{汤勤
(念)}心中只想美佳人(或:\textbf{心中只为美佳人),费尽三毫七孔心。但愿她心称我意}\protect\hyperlink{fn572}{\textsuperscript{572}}\textbf{,人头是假也是真。}

\textbf{汤勤 小官告进。}

\textbf{汤勤 小官参见(或:小官叩见)老大人。}

\textbf{陆炳 汤老爷过衙来了。}

\textbf{汤勤 (是,)小官过衙来了。}

\textbf{陆炳 啊?!清早过得衙来,莫非拿老夫的什么弊病来了么?}

\textbf{汤勤 小官告退。}

\textbf{陆炳 转来。(或:回来。)}

\textbf{汤勤 在。}

\textbf{陆炳 为何去心太急(或:为何去心忒急)?}

\textbf{汤勤
启禀老大人,小官上得堂来一言未发,老大人就说``弊病''二字,小官吃罪不起。}

\textbf{陆炳 老夫乃是笑谈。}

\textbf{汤勤 哦,老大人乃是笑谈?小官我吃了一大惊呐!}

\textbf{陆炳 如此汤老爷请上坐。}

\textbf{汤勤 呃,且慢,乃是老大人的法堂,小官不敢坐。}

\textbf{陆炳 哦,你也晓得法堂?}

\textbf{汤勤 怎么不晓得?}

\textbf{陆炳 如此旁设一座。}

\textbf{汤勤 多谢老大人。}

\textbf{陆炳 呃,往上些,呃,往上些。}

\textbf{汤勤 是是是。}

\textbf{陆炳 汤老爷,过衙何事?}

\textbf{汤勤 奉了严大人之命,前来会审人头。}

\textbf{陆炳 汤老爷来得好哇!若是不来,弟具一名帖,请驾到此,会审人头。}

\textbf{汤勤 小官不,不敢当啊!}

\textbf{陆炳 来,}

\textbf{门子 有。}

\textbf{陆炳 带人犯。}

\textbf{门子 带人犯!}

\textbf{(张龙、郭义、戚继光、雪艳上)}

\textbf{张龙、郭义、戚继光、雪艳 叩见大人。}

\textbf{门子 张龙、郭义,}

\textbf{张龙、郭义 有。}

\textbf{门子 戚继光,}

\textbf{戚继光 有。}

\textbf{门子 雪艳。}

\textbf{雪艳 有。}

\textbf{陆炳 戚继光、雪艳下去。}

\textbf{戚继光、雪艳 是。}

\textbf{(戚继光、雪艳分下)}

\textbf{陆炳 张龙、郭义。}

\textbf{张龙、郭义 有。}

\textbf{陆炳 (想)莫怀古夫妇,(还)是蓟州的兵丁拿获的,还是尔等拿获的?}

\textbf{张龙、郭义 乃是小人们拿获的。}

\textbf{陆炳 在哪里拿住的(或:在哪里拿获的)?}

\textbf{张龙、郭义 在蓟州西门之外(或:在蓟州西门以外),柳林之下。}

\textbf{陆炳 什么时候?}

\textbf{张龙、郭义 黄昏时候。}

\textbf{陆炳 怎么进城(或:怎样进城)?}

\textbf{张龙、郭义 叫开城门,批了劄子,击了戚大人堂鼓,才见得戚大人。}

\textbf{陆炳 戚大人怎样吩咐?}

\textbf{张龙、郭义
大人言道:此事大了,要作两家担待(或:要和我等两家担待)。}

\textbf{陆炳 何谓``两家担待''?}

\textbf{张龙、郭义
``头门以里、仪门以外,有一军牢小房,将我等并锁一处(或:将我等锁在一处),里面有灯有火,外面有封有锁(或:外面有锁有封),锁上加封,等到五鼓天明,看着绑,看着斩,人头打入木桶,回覆严大人。''}

\textbf{陆炳 可是真情?}

\textbf{张龙、郭义 俱是真情。}

\textbf{陆炳 带下去。}

\textbf{门子 下去。}

\textbf{陆炳 带雪艳。}

\textbf{门子 带雪艳。}

\textbf{(雪艳上,跪)}

\textbf{雪艳 雪艳叩见大人。}

\textbf{陆炳 雪艳。}

\textbf{雪艳 有。}

\textbf{陆炳
你夫妇(或:你夫妻),还是蓟州的兵丁拿获的,还是严府校尉(或:严府人役)拿获的?}

\textbf{雪艳 乃是严府校尉拿获的(或:乃是严府人役拿获的)。}

\textbf{陆炳 在哪里拿获的(或:在哪里拿住的)?}

\textbf{雪艳 在蓟州西门以外,柳林之下。}

\textbf{陆炳 什么时候?}

\textbf{雪艳 黄昏时候。}

\textbf{陆炳 怎样进城?}

\textbf{雪艳 叫开城门,批了劄子,击了戚大人堂鼓,才(得)见戚大人。}

\textbf{陆炳 戚大人是怎样吩咐?}

\textbf{雪艳 戚大人言道:此事大了,(必须)要作两家担待。}

\textbf{陆炳 何谓``两家担待''?}

\textbf{雪艳
头门以里、仪门以外,有一军牢小房,将我等并锁(在)一处,里面有灯有火,外面有封有锁(或:外面有锁有封),锁上加封,等到五鼓天明,看着绑,看着斩,人头打入木桶,严府校尉回复严大人去了。}

\textbf{陆炳 可是真情?}

\textbf{雪艳 俱是真情。}

\textbf{陆炳 带下去。}

\textbf{门子 下去。}

\textbf{雪艳 是。}

\textbf{(雪艳下)}

\textbf{陆炳 汤老爷,汤老爷。哎!}

\textbf{(汤勤看雪艳)}

\textbf{汤勤 呃,老大人,老大人。}

\textbf{陆炳 呃,呃,我在这里哟。}

\textbf{汤勤 哎呀,老大人。}

\textbf{陆炳
汤老爷。我想戚继光乃是阃外总兵,他犯法于朝廷,不能犯法于你我,我有意赏他一个矮座,你意下如何?}

\textbf{汤勤 但凭老大人。}

\textbf{陆炳 来,}

\textbf{门子 有。}

\textbf{陆炳 带戚继光。}

\textbf{门子 带戚继光。}

\textbf{(戚继光上)}

\textbf{戚继光 犯官参见大人。}

\textbf{陆炳 汤老爷赏你一矮座,(还不)过去谢谢汤老爷。}

\textbf{戚继光 多谢汤老爷。}

\textbf{(汤勤 不消。)}

\textbf{陆炳 戚继光,莫怀古夫妇,(是)在哪里拿获的?}

\textbf{戚继光 严府校尉在蓟州西门以外,柳林之下拿获莫怀古夫妇的。}

\textbf{陆炳 什么时候?}

\textbf{戚继光 黄昏时候。}

\textbf{陆炳 怎样进城?}

\textbf{戚继光
叫开城门,劈开劄子(或:批了劄子),击动了犯官的堂鼓,才见犯官。}

\textbf{陆炳 你是怎样吩咐?}

\textbf{戚继光
犯官(也曾)言道:此事重大,要作两家担待(或:必须要两家担待)。}

\textbf{陆炳
何谓}\protect\hyperlink{fn573}{\textsuperscript{573}}\textbf{``两家担待''?}

\textbf{戚继光
头门以里、仪门以外,有一军牢小房,将他等并锁一处(或:将他们四人锁在一处),里面有灯有火,外面有封有锁(或:有锁有封),锁上加封,等到五鼓天明,犯官看着绑,看着斩,人头打入木桶,回复严大人。}

\textbf{陆炳 可是真情?}

\textbf{戚继光 俱是真情。}

\textbf{陆炳 带下去。}

\textbf{门子 下去。}

\textbf{戚继光 是。}

\textbf{(戚继光下)}

\textbf{陆炳 汤老爷,这个人头是真的了。}

\textbf{汤勤 呃,怎见得是真的呢?}

\textbf{陆炳 他四人的口供俱是一样,怎么不是真的了啊?}

\textbf{汤勤 唉呀,老大人,想这个人头,唔,定是假的。}

\textbf{陆炳 怎么是假的?}

\textbf{汤勤 想他四人一路同行,同宿旅店,串通口供,蒙哄老大人。}

\textbf{陆炳 哦,是蒙混老夫的么?}

\textbf{汤勤 不错,是蒙混老大人。}

\textbf{陆炳 汤老爷,老夫,呃,我有个决断。}

\textbf{汤勤 老大人有何高才(或:老大人必有高才。)}

\textbf{陆炳
岂敢。想我日前斩了几个人头,不曾示众,我意欲将那几个人头排在堂口(或:摆在堂口),连莫怀古的人头也摆在其内,教那雪艳前去相认(或:上前相认)。她认真便真,认假便假。汤老爷,意下如何?}

\textbf{汤勤 呃,乃是老大人的高才。但凭老大人。}

\textbf{陆炳
来,听吩咐:将我日前斩得几个人头,俱都排在堂口(或:俱都摆在堂口),连莫怀古的人头也排在其内,排了起来。}

\textbf{(四青袍两边上,拿人头排在堂口介)}

\textbf{陆炳 排齐了?}

\textbf{四青袍 排齐了。}

\textbf{陆炳 起过。}

\textbf{四青袍 是。}

\textbf{陆炳 来,带雪艳。}

\textbf{门子 带雪艳。}

\textbf{(雪艳上)}

\textbf{雪艳 叩见大人。}

\textbf{陆炳 雪艳。}

\textbf{雪艳 有。}

\textbf{陆炳 我道人头是真,汤老爷说这个人头是假。老夫如今有这个决断。}

\textbf{雪艳 大人有何高才?}

\textbf{陆炳
想我日前斩了几个人头,不曾示众,排在堂口,连你丈夫的人头也摆在其内,命你前去相认。你认真便真,你认假便假。哪个是你丈夫的人头------抱来见我!}

\textbf{雪艳 谢大人!}

\textbf{雪艳
【二黄散板】今日领了大人命,堂口之上认夫君。走向前,把夫认,}

\textbf{雪艳 喂呀\ldots{}\ldots{}(哭介)}

\textbf{雪艳 【二黄散板】血淋淋人头认不真。这厢不是那厢认,}

\textbf{雪艳 喂呀\ldots{}\ldots{}(哭介)}

\textbf{雪艳 【二黄散板】手捧人头见大人。}

\textbf{陆炳 可是你丈夫的人头?}

\textbf{雪艳 (正)是。}

\textbf{陆炳 下去。}

\textbf{雪艳 是。喂呀\ldots{}\ldots{}(哭介)}

\textbf{(雪艳下,汤勤看介)}

\textbf{陆炳 衙役们,将人头搬下堂口。}

\textbf{四青袍 是。}

\textbf{(四青袍搬下头介,下)}

\textbf{陆炳 汤老爷。}

\textbf{汤勤 呃呃,老大人。}

\textbf{陆炳 汤老爷,这个人头可是真的了么?}

\textbf{汤勤 怎么,是真的?}

\textbf{陆炳 你看那雪艳抱着他丈夫莫怀古的人头痛哭流泪,岂不是真的么?}

\textbf{汤勤 哎呀老大人,我把那雪艳好有一比呀。}

\textbf{陆炳 比作何来?}

\textbf{汤勤 ``猫儿哭耗子------''}

\textbf{陆炳 此话怎讲?}

\textbf{汤勤 假慈悲。}

\textbf{(众手下哭介)}

\textbf{众 喂诶\ldots{}\ldots{}(哭介)}

\textbf{陆炳
啊,汤老爷,你(来)看这两班人役(或:这两班的衙役),他们都掉起泪来了。}

\textbf{汤勤 我把他们也好有一比呀。}

\textbf{陆炳 比作何来?}

\textbf{汤勤 ``看兵书,哼,掉眼泪------''}

\textbf{陆炳 此话怎讲?}

\textbf{汤勤 替古人担忧哇。}

\textbf{(陆炳 怎么讲?)}

\textbf{(汤勤 替古人担忧。)}

\textbf{陆炳 啊,他们是替古人担忧?}

\textbf{汤勤 是。}

\textbf{陆炳 汤老爷,你怎么不担忧呢?}

\textbf{汤勤
(唉,)老大人说哪里话来,小官与他一不沾亲,二不带故。呃,我担的什么忧啊?}

\textbf{陆炳
汤老爷,你苦苦道那莫怀古的人头是假,难道说他的人头还有什么质对}\protect\hyperlink{fn574}{\textsuperscript{574}}\textbf{么?}

\textbf{汤勤 哼,大大的有质对!}

\textbf{陆炳 哦,我今日单单地与你要个质对,与我讲!}

\textbf{汤勤
我那莫大老爷前有梅花额}\protect\hyperlink{fn575}{\textsuperscript{575}}\textbf{,后有三台骨。}

\textbf{陆炳
汤老爷,我且问你,想这梅花额,生在面前,你可以常常得见;想这三台骨,长在暗处,你是怎能得见的哟!}

\textbf{汤勤
哎呀老大人(你)有所不知呀,小官乃是钱塘人氏,与那莫大老爷乃是同乡。小官不得时的时节,乃是卖字画为生呐。那天小官在陈桥打睡,我那莫大老爷拜客回来(或:拜客而归),将我唤醒,看我的字(是)真、草、隶、篆,看我的画,水墨丹青,将我让到他家,做了一个同窗的幕宾。我与那莫大老爷清晨起来(或:清早起来)同盆净面,同席吃饭,到晚来同榻而眠。这三台骨啊,是常常得见的很哦!}

\textbf{陆炳
你虽然是常常得见,呃,你可晓得(当初莫大老爷待你是好是不好啊?)这人死则变!}

\textbf{汤勤 呃,老大人,这梅花额烂去了,这三台骨,嗯,是变不了啊。}

\textbf{陆炳 哦,他的骨头变不了?!}

\textbf{汤勤 骨头变不了。}

\textbf{陆炳 汤老爷,那莫大老爷待你如何?}

\textbf{汤勤 那莫大老爷待我是恩重如山。}

\textbf{陆炳 我看将起来令人可恨!}

\textbf{汤勤 老大人,敢莫是恨着小官不成么?}

\textbf{陆炳 哎呀呀,我怎敢恨着汤老爷呀。}

\textbf{汤勤 老大人,(你)恨着哪个?}

\textbf{陆炳
我恨(只恨)那莫大老爷,他虽然是个进士出身,他就大大地失了眼力呀。}

\textbf{汤勤 怎么他大大地失了眼力?}

\textbf{陆炳
(不是哟,)想汤老爷当初不得第的时节,在钱塘卖画,莫大老爷得中了进士公,在长街拜客而归,看见汤老爷在那里卖画,看你的画,乃是水墨丹青;看你的字(是)真、草、隶、篆,他乃是个读书之人,心中焉能不喜爱字画呀,故而带你进府。他(纵然)带你进府,就(不该)带你进京;(他纵然)带你进京,就(不该)将你荐与严府,严大人这才重用于你。你就做了官,你就得了经历司(或:你就做了经历司)。常言道得好:(念)不是渔夫引,怎得(或:怎能)见波涛。当初莫大老爷就不该带你进府(或:想当初莫大老爷将你带进府去),(他就)不该将你带进京来;他纵然带将你带进京来,他就不该将你荐与严府。如今也(就)用不着你这个铁板的干证呐!依我看起来,哼,人头是真也是真,是假也是真,就是这样的落案!}

\textbf{汤勤 告辞!}

\textbf{陆炳 回来。}

\textbf{汤勤 在。}

\textbf{陆炳 你(往)哪里去?}

\textbf{汤勤 回覆严大人。}

\textbf{陆炳 你见了严大人,讲些什么?}

\textbf{汤勤 我就说陆大人糊里糊涂地就是这样落了案。}

\textbf{陆炳 啊?!那严大人是狼?}

\textbf{汤勤 嗯,不是狼。}

\textbf{陆炳 是虎?}

\textbf{汤勤 嗯,不是虎。}

\textbf{陆炳 他(又)不是狼,又不是虎,难道他还要吞吃我陆炳不成么?}

\textbf{汤勤 吞吃不了老大人。}

\textbf{陆炳 哈哈,哈哈,啊呵呵哈哈哈\ldots{}\ldots{}(陆炳三笑介)}

\textbf{(陆炳 呵呵呵\ldots{}\ldots{}(冷笑介))}

\textbf{汤勤 呃,老大人为何发笑啊?}

\textbf{陆炳 我笑你这两句话癫而又狂,尊而又大。}

\textbf{汤勤 小官怎么(或:小官怎见得)癫而又狂,尊而又大?}

\textbf{陆炳
我问道汤老爷:那严府是狼?你讲道他不是狼;我问他是虎?你又讲道他不是虎。汤老爷,慢说他不是狼,不是虎。纵然他是狼是虎,我也有打狼的汉子,还有擒虎的英雄!我陆炳做官,一不欺君,
二不傲上(或:二不枉上),三不贪赃,四不无理以公报私(或:四不无理,以公报公),毫无私弊呀!我做官,做的是嘉靖皇上的官,我没有做(他)严府的官,我又不是(他)严府家人、小子,我又不是他严府使用的奴才(或:严府的使用奴才)!我又怕他何来?!我也曾中过皇榜,我乃二甲进士出身,我是个科举(或:科甲)出身呐!我奉了天子之命,在此审问莫怀古的人头。你不过是奉了严爷(或:你不过是奉了严大人)的一句话呀,过得衙来,照看而已。我与那严大人一殿为臣,不能无礼,这才赏了你一个座位,这叫作``敬其上而爱其下'',才有此举呀。你就该过得衙来,规规矩矩,坐在一旁,耳闻目睹,听其自然,才是你的道理;你不规规矩矩,坐在一旁,反倒飞扬浮躁,信口乱言(或:信口胡言),忽然人头是真,又讲人头是假,真假不定,反复无常,出乎反乎,真乃是个无耻的小人!我又不买你的字画,你倒要做什么啊?(或:我可不买你的字画呀,你到此来则甚么?)哼,好大的一个汤老爷!}

\textbf{陆炳 来,将座位与我撤了!}

\textbf{汤勤
哎呀呀,你看这个陆大人,虽然年迈,嗯,倒有些傲性。嗯,待我讲上几句好话。}

\textbf{汤勤
嗯,呃呵呵呵\ldots{}\ldots{}哎呀\ldots{}\ldots{}老大人,小官今早吃了几杯糟酒,诶,一言冒犯,诶,老大人,小官与老大人叩头赔礼了。}

\textbf{陆炳 诶呵呵\ldots{}\ldots{}汤老爷请起,请坐。}

\textbf{汤勤 多谢老大人。}

\textbf{陆炳
汤老爷,虽然是笑谈,与这命案大有关系(或:与这人头是大有关系)。}

\textbf{汤勤 是是是。}

\textbf{陆炳 啊汤老爷,你看此案如何发落(或:你有何高见呐?)}

\textbf{汤勤
呃呃呃,老大人,有道是:``抄手问贼贼不招,用棒打犬犬必逃''。不动大刑,料他们不招。}

\textbf{(陆炳 哦,不动大刑,料他们不招?)}

\textbf{陆炳 汤老爷,(你看)这上------}

\textbf{汤勤 青天。}

\textbf{陆炳 这下------}

\textbf{汤勤 浮土。}

\textbf{陆炳 你我为官的------}

\textbf{汤勤 全凭``良心''二字。}

\textbf{陆炳 若是无有良心呢?}

\textbf{汤勤 呃,呃,教天狗吃了他们。}

\textbf{陆炳
好个``天狗吃了他们''。想这无有良心的事,别人做得出来,难道我陆炳就做不出来么?}

\textbf{陆炳 来,将严府二公差与我带上来!}

\textbf{门子 带张龙、郭义。}

\textbf{(张龙、郭义上)}

\textbf{张龙、郭义 叩见大人!}

\textbf{陆炳
唗!我把你这大胆的狗才,奉了严大人一张票,就是这样遮天盖地而来,不知耽误人家(或:误了人家)多少好事,误了人家(或:人家耽误)多少功名?我今天(或:我如今)若不打你们几下,又恐怕惯坏了你们的下次!}

\textbf{陆炳 来,扯下去打!}

\textbf{汤勤 哎呀呵,且慢,且慢。哎呀老大人,他二人打不得。}

\textbf{陆炳 难道说我就打不得他们?}

\textbf{汤勤 不是哦,他二人乃是(或:他们乃是)牵连在内。}

\textbf{陆炳 什么牵连在内,分明(是)与他二人讲情!}

\textbf{汤勤 呵,小官不敢,老大人开恩。}

\textbf{陆炳
哼!本当要打你们几下,汤老爷讲情,恐怕打了尔的腿,伤了汤老爷的脸面(或:面子)。记打,记责。下去!}

\textbf{张龙、郭义 是,多谢大人。}

\textbf{(张龙、郭义下)}

\textbf{陆炳 来,带戚继光!}

\textbf{门子 带戚继光。}

\textbf{(戚继光上)}

\textbf{戚继光 犯官参见大人!}

\textbf{陆炳
唗!身为八台总兵,斩了个人头不清不白,难免自羞自愧。来,扯下去打!}

\textbf{朝官 (内)黑诏下。}

\textbf{门子 启大人:黑诏下。}

\textbf{陆炳 带下去。}

\textbf{门子 是。}

\textbf{(戚继光下)}

\textbf{陆炳
汤老爷,我正要用刑,耳闻黑诏下。还是审头事大,还是接诏事大呀?}

\textbf{汤勤 自然(是)接诏事大。}

\textbf{陆炳 (汤老爷)请到下面待茶。}

\textbf{汤勤 小官告退。}

\textbf{陆炳 请。}

\textbf{汤勤 请。}

\textbf{(汤勤下)}

\textbf{陆炳 来,请诏。}

\textbf{(四青袍、一朝官上)}

\textbf{朝官 黑诏到,下跪!}

\textbf{陆炳 万岁。}

\textbf{朝官
听宣读,诏曰:``今有圣上发下犯官四名,一十三名江洋大盗,命刑部正堂监斩,刑部正堂染病在床,命锦衣卫陆炳到平则门外监斩。''黑诏读罢,望诏谢恩!}

\textbf{陆炳 万万岁!香案供奉!}

\textbf{门子 是。}

\textbf{陆炳 有劳大人请诏前来,后堂留宴。}

\textbf{朝官 有朝命在身(或:有王命在身),不敢久停。告辞!}

\textbf{(四青袍带马下)}

\textbf{陆炳 送大人。}

\textbf{朝官 请------}

\textbf{(朝官下)}

\textbf{(汤勤上)}

\textbf{汤勤 老大人,方才黑诏到来,但不知为了何事?}

\textbf{陆炳
方才黑诏到来,有四员(或:有四名)犯官,一十三名江洋大盗,命刑部监斩,刑部染病在床,命老夫监斩。啊,还是斩头事大,还是审头事大?}

\textbf{汤勤 自然是斩头事大。}

\textbf{陆炳 王命为尊。}

\textbf{汤勤 是是是。}

\textbf{陆炳
汤老爷,老夫奉旨斩头,意欲将雪艳吊在西廊,委派汤老爷,审问她的口供,(汤老爷)意下如何?}

\textbf{汤勤 哦,小官如何审得?}

\textbf{陆炳 诶,你是奉了严大人的委派前来的,怎么审不得。}

\textbf{汤勤 哦,是是是。小官当得效劳。}

\textbf{陆炳 来,带雪艳。}

\textbf{门子 雪艳。}

\textbf{(雪艳上)}

\textbf{雪艳 叩见大人。}

\textbf{陆炳 雪艳
,老夫奉旨斩头,将你吊在西廊,委派汤老爷审问你的口供,若是翻供,吃罪不起。}

\textbf{(雪艳 是。)}

\textbf{陆炳 来,将她吊在西廊。}

\textbf{(众吊雪艳介)}

\textbf{(雪艳 喂呀\ldots{}\ldots{}(哭介))}

\textbf{陆炳 来,带张龙、郭义。}

\textbf{(张龙、郭义上)}

\textbf{张龙、郭义 叩见大人。}

\textbf{陆炳 命你二人,看守雪艳。若是卖、放,打折两腿。起过!}

\textbf{张龙、郭义 是。}

\textbf{陆炳 来,带戚继光。}

\textbf{门子 带戚继光。}

\textbf{(戚继光上)}

\textbf{戚继光 犯官叩见大人(或:犯官参见大人)!}

\textbf{陆炳
身为八台总兵,斩了(几)个人头,不明不白(或:不清不白)。用小轿一乘,随定我的轿后。待我斩两个人头,与你看上一看。起过!}

\textbf{戚继光 是。}

\textbf{陆炳 汤老爷,}

\textbf{汤勤 老大人。}

\textbf{陆炳 这个雪艳可是交与你了!}

\textbf{汤勤 哦,小官代劳。}

\textbf{陆炳 多多地辛苦了。}

\textbf{汤勤 是岂敢,呵,岂敢,。}

\textbf{陆炳 来,外厢开道。}

\textbf{(众倒领由上场门下)}

\textbf{汤勤 哎呀妙啊妙哇,想这雪艳一案,(命我代审。)我就代审代审。}

\textbf{张龙、郭义 汤老爷,}

\textbf{汤勤 哎呀,呃,二位,适才多有受惊了(或:方才多受惊了)。}

\textbf{张龙、郭义 多谢汤老爷讲情。}

\textbf{汤勤 岂敢岂敢,二位在此则甚?}

\textbf{张龙、郭义 我二人奉了陆大人之命,看守雪艳。}

\textbf{汤勤 二位歇息去罢。}

\textbf{张龙、郭义 哎呀,我二人不敢远离。}

\textbf{汤勤 呀呀哎,(这)分明是(跟我)要银子啊。}

\textbf{汤勤 哎呀二位,我这里有点小意思,呃,带着吃杯茶(去)罢。}

\textbf{张龙、郭义 呃------我二人不敢收呀。}

\textbf{汤勤
只管收下,料也无妨}\protect\hyperlink{fn576}{\textsuperscript{576}}\textbf{。}

\textbf{张龙 (如此)多谢汤老爷。}

\textbf{郭义
哎呀,看来汤老爷是个好人呐。(或:哎呀汤老爷,哼,看起来你是好人呐!)}

\textbf{汤勤 哎呀,本来(的)是个好人。}

\textbf{张龙
啊伙计,这个人头原本(或:这个人头原来)是真的,也不知哪个坏种、嘎杂子,他(偏偏)说是假的。伙计,你在里面访,我在外面寻(或:我在外面访),访着此人,将他吊在树梢儿上(面),揭开他的天灵盖,里面装上火药,安上捻子,当徽州炮放这个嘎杂子啊!汤老爷!你是个好人呐。呵呵哈哈哈\ldots{}\ldots{}(笑介)}

\textbf{汤勤 (\ldots{}\ldots{}什么,)哎呀,这两个狗头!}

\textbf{汤勤
哎呀,雪娘子,陆大人命我背审,我说人头是真的,定是真的;我说是假的,定是假的。哎呀雪娘子,你把那心要放明白了呀!(或:哎呀,雪娘子,陆大人命我审问人头,我说人头是真,就是真;我说是假的,定是假的。雪娘子,诶呀,你的心呀,哎呀,要放明白呀!)}

\textbf{雪艳
(暗)哎呀且住,听贼之言有戏奴之意,此时不应儿夫冤仇何日得报?!哦,有了,待我假意应下就是。}

\textbf{雪艳
哎呀汤老爷(或:啊汤老爷),自那日上船之时,我这心中就有了你了\ldots{}\ldots{}}

\textbf{汤勤 哎呀,我的亲娘------}

\textbf{内 噢------}

\textbf{(汤勤急下)}

\textbf{{[}第三场{]}}

\textbf{(陆炳人全上)}

\textbf{陆炳
【二黄散板】大炮一声响(或:号炮一声响)人头落,世人休犯(或:世人莫犯;为人莫犯)律萧何。}

\textbf{(陆炳下轿入位,张龙、郭义暗上,汤勤上)}

\textbf{汤勤 老大人一路之上,多有辛苦(或:多受辛苦)了哇。}

\textbf{陆炳 为国勤劳,何言``辛苦''二字?}

\textbf{汤勤 老大人适才监斩,但不知斩的(都)是什么案件?}

\textbf{陆炳
有(一)十三名江洋大盗,呃,问他们什么罪过(或:问他个什么罪过)?}

\textbf{汤勤 呃,当问斩罪。}

\textbf{陆炳 不错的。汤老爷,我有一事不明,要在汤老爷面前领教。}

\textbf{汤勤 呃,老大人有话请讲,何言``领教''二字?}

\textbf{陆炳
呃,有一官员(或:有一员犯官),他奉旨领兵,不料他临阵脱逃,呃,该当问(他个)什么罪过?}

\textbf{汤勤 呃呃,也问他(个)斩罪。}

\textbf{陆炳 嗯,斩也不亏他。}

\textbf{汤勤 呃老大人,呃,还有什么案件呢?}

\textbf{陆炳
呃------呃呃,还有这么一员小官(或:还有这么一名小官)呐,当初他不得第(的时节),多亏他的恩主,将他提拔起来了,他在大官面前搬动是非,害死他的恩主,一家四散。呃,像这样的人呐,该当问他(个)什么罪过?呃,我要领教,汤老爷。}

\textbf{汤勤
哎呀,老大人,此案要犯在小官手内么(或:此案若犯在小官手内么),嗯嗯,把他提上堂来(或:将他提上堂来),打他五个手简子}\protect\hyperlink{fn577}{\textsuperscript{577}}\textbf{,不要惯(坏)了他的下次,诶,也就够他受用的了哇。}

\textbf{陆炳 什么(或:怎么),打他几个手简子? (或:啊?打他五个手简子?)}

\textbf{汤勤 打他五个手简子。}

\textbf{陆炳 怎么,依我看将起来,定要将他问一个凌迟碎剐!}

\textbf{汤勤 哎呀,老大人,太重了哇(或:忒重了哇)。}

\textbf{陆炳
他是舌剑------杀人,最是可恶(或:最是可恨)的呀。(或:舌剑杀人,最是可恶的了哇。)}

\textbf{汤勤 重了,重了。}

\textbf{陆炳 重了?}

\textbf{汤勤 重了。}

\textbf{陆炳
啊,呵呵哈哈哈\ldots{}\ldots{}(笑介)啊汤老爷,你可(曾)背审雪艳呐,人头是真是假呐?}

\textbf{汤勤 这个人头么?是真的。}

\textbf{陆炳 是真的?}

\textbf{汤勤 呃,(是)真的呀。}

\textbf{陆炳 汤老爷,你上得堂来,就说了这么一句有良心的话呀。}

\textbf{汤勤 诶,小官最是有良心的(或:小官我是最有良心的)。}

\textbf{陆炳 张龙、郭义?}

\textbf{汤勤 销票无事。}

\textbf{陆炳 戚继光?}

\textbf{汤勤 原任蓟州八台。}

\textbf{陆炳 这雪艳------}

\textbf{汤勤 但凭老大人。}

\textbf{陆炳 哦,但凭老夫\ldots{}\ldots{}将她发往钱塘。}

\textbf{汤勤 路远。}

\textbf{陆炳 将她送到蓟州。}

\textbf{汤勤 无有亲人。}

\textbf{陆炳
呃,这样罢,就在老夫(的)衙门(或:就在老夫衙内)暂住,你意如何?}

\textbf{汤勤 哦,将雪艳要暂住老大人(的)衙内?}

\textbf{陆炳 诶,暂且寄住而已(或:暂住而已)。}

\textbf{汤勤 人头定是假的,还要再审再审。}

\textbf{(汤勤下)}

\textbf{陆炳
汤勤呐,好奸贼!我听他之言,有要纳雪艳为妾之心(或:分明要纳雪艳为妾)。我若将雪艳断与那贼为妻(或:我若将雪艳断与那汤勤为妾),慢说(是)满朝文武道我无才,就是我这两班的衙役,也是(要)道我无才!(这这这\ldots{}\ldots{})好不难坏我也\ldots{}\ldots{}}

\textbf{雪艳 唉呀,好一位(或:好一个)不明白的陆大人呐!(哭介)}

\textbf{陆炳
且住!老夫正在为难之际,雪艳言道:``好个不明白的陆大人!''哦哦哦,是了!
我那莫仁兄也曾对我讲过,他讲道(或:我那莫仁兄也曾对我言讲):雪艳(虽然是个妓者出身,倒)有些仁义节烈,她莫非是有心,与我那莫仁兄报仇?!}

\textbf{陆炳 \textless{}叫头\textgreater{}雪娘子啊,莫仁嫂!}

\textbf{陆炳
你若是有心与我那莫仁兄报仇,拚着我陆炳这顶乌纱不要,我就与你担待、担待。正是:(念)清官暂把赃官做,聪明反做懵懂人。}

\textbf{(汤勤上)}

\textbf{汤勤 老大人为何在背地沉吟?}

\textbf{陆炳
非是老夫背地沉吟。方才又问了雪艳一遍,(这个)人头本来是真的。}

\textbf{汤勤 哦,是真的,是真的。}

\textbf{陆炳
张龙、郭义,销票无事;戚继光,原任总兵(或:原任八台);仔细想来,将雪艳送回钱塘(或:将雪艳送往钱塘),钱塘路远呐。}

\textbf{汤勤 本来路远。}

\textbf{陆炳 送到蓟州(或:送往蓟州),无有亲人。}

\textbf{汤勤 (是)无有亲人。}

\textbf{陆炳 寄在老夫(的)衙内(或:寄住老夫的衙内),出入有些不便呐。}

\textbf{汤勤 是啊,有些不便,诶,二来,于老大人的名气,有些不好啊。}

\textbf{陆炳 (唉,)莫若将雪艳寄在汤老爷的衙内?}

\textbf{汤勤
老大人说哪里话来,雪艳她又不是什么货物,今日寄在东家,明日寄在西家。老大人要办么,就办一个``水落石出''。}

\textbf{陆炳 (哦,)要办个``水落石出''(或:要断个``水落石出'')?}

\textbf{汤勤 嗯,办个``水落石出''。}

\textbf{陆炳 汤老爷,你可有宝眷呐?}

\textbf{汤勤 呃,(我)无有家眷。}

\textbf{陆炳 老夫作主,将雪艳断与汤老爷(为妾)。}

\textbf{汤勤 哪个为媒?}

\textbf{陆炳 老夫为媒。}

\textbf{汤勤
哦,老大人为媒?哎呀,犹如我重生父母,再造的爹娘。我这里多谢老大人,多谢\ldots{}\ldots{}。}

\textbf{陆炳 呃呃,起来,起来。}

\textbf{汤勤 多谢老大人。}

\textbf{陆炳 请坐。}

\textbf{汤勤 呃,谢座。}

\textbf{陆炳 呃,我还有话问你,}

\textbf{汤勤 呃,老大人有话请讲。}

\textbf{陆炳 人头是真的?}

\textbf{汤勤 呃,真的。}

\textbf{陆炳 张龙、郭义?}

\textbf{汤勤 销票无事。}

\textbf{陆炳 戚继光?}

\textbf{汤勤 原任八台总镇。}

\textbf{陆炳 严大人见罪?}

\textbf{汤勤 有小官担待。}

\textbf{陆炳 原是要你担待。}

\textbf{陆炳 来,带张龙、郭义。}

\textbf{(张龙、郭义上)}

\textbf{张龙、郭义 叩见大人。}

\textbf{陆炳 人头是真,有文书一轴回覆严大人,外有手本,问候金安,下去!}

\textbf{张龙、郭义 谢大人。}

\textbf{(张龙、郭义下)}

\textbf{陆炳 来,将雪艳放了下来。}

\textbf{(公差放雪艳介)}

\textbf{雪艳 叩见大人。(或:谢大人!)}

\textbf{陆炳
雪艳,老夫为媒,将你断与汤老爷。汤老爷可比不得莫大老爷,你必须要殷勤早晚伺(谐音:刺)(陆炳用扇遮挡介)------伺候!伺候!}

\textbf{雪艳 多谢大人!}

\textbf{雪艳
【二黄散板】好一位大人断得妙,奴与汤勤配缘交。耐等三更时分到,难躲奴家这一刀。}

\textbf{雪艳 喂呀\ldots{}\ldots{}(哭介)}

\textbf{(雪艳下)}

\textbf{陆炳 汤老爷,请回衙理事。}

\textbf{汤勤 多谢大人!小官告辞了哇。}

\textbf{汤勤 【二黄摇板】辞别大人下大堂,汤勤今晚做新郎。}

\textbf{汤勤 哈哈哈\ldots{}\ldots{}(笑介)}

\textbf{(汤勤下)}

\textbf{陆炳 转堂!}

\textbf{(起\textless{}牌子\textgreater{})}

\textbf{陆炳 来,有请戚大人。}

\textbf{门子 有请戚大人。}

\textbf{(戚继光上)}

\textbf{戚继光 【二黄摇板】忽听仁兄一声请,来在二堂问详情。}

\textbf{陆炳 贤弟请坐。}

\textbf{戚继光 有座。}

\textbf{陆炳 贤弟(你)受惊了哇。}

\textbf{戚继光 有劳仁兄挂心了。}

\textbf{陆炳 岂敢。}

\textbf{陆炳 恭喜贤弟,贺喜贤弟。}

\textbf{戚继光 喜从何来?}

\textbf{陆炳 贤弟还是八台原任。}

\textbf{戚继光 此乃仁兄提拔。}

\textbf{陆炳 岂敢。}

\textbf{戚继光 那雪艳怎样落案(或:怎样发落)?}

\textbf{陆炳 将那雪艳断与汤勤了。}

\textbf{戚继光 唉,仁兄你真真(的)大大无才!}

\textbf{陆炳 贤弟,我这叫作``不得已而为之''!}

\textbf{陆炳
【四平调】自古道人亏天不亏,过往神灵饶过谁。请贤弟暂且回衙去,最可叹仁兄死得苦,三日之内自有信回。(或:戚贤弟暂且回衙去,三日之后自有信回。)}

\textbf{戚继光
【四平调】仁兄做事大有才,胸中韬略弟怎解开。辞别仁兄出府外,三日等候报马来。}

\textbf{(戚继光下)}

\textbf{陆炳 来,衙役们进见。}

\textbf{(门子 是。)}

\textbf{(衙役上)}

\textbf{众 参见大人。}

\textbf{陆炳
衙役们(或:衙役的),听我吩咐。每人领银三分(或:每人用银三分),庆贺汤勤。在洞房之内,尔等(们)只管劝酒,闹出祸来(或:闯出祸来),有老夫担待,尔等记下了。}

\textbf{众 遵命。}

\textbf{(衙役下)}

\textbf{陆炳 来,吩咐掩门。}

\textbf{门子 掩门。}

\textbf{(陆炳下)}

\textbf{{[}第四场{]}}

\textbf{雪艳 (内)苦哇\ldots{}\ldots{}}

\textbf{雪艳 【二黄导板】听谯楼打罢了初更时候,}

\textbf{(雪艳上)}

\textbf{雪艳
\textless{}叫头\textgreater{}老爷,夫君!喂呀\ldots{}\ldots{}(哭介)}

\textbf{雪艳
【二黄正板】上房内来了我雪艳女流。想当年身落在烟花巷口,多亏了莫老爷将我收留。我二人在钱塘荣华不受,一心心要进京去把官求。那日里拜客时大街行走,汤勤贼卖字画十字街头。我老爷喜字画心中爱就,他那时与汤勤骨肉相投。严世藩闻此言双眉起皱}\protect\hyperlink{fn578}{\textsuperscript{578}}\textbf{,他那里带校尉来把杯搜。我老爷闻此言弃官逃走,西门外柳林下将奴锁收。叹莫成替主死蓟州堂口,叹老爷去他乡不能回头。叹夫人在钱塘不能得够,叹莫家叹得我两泪交流。听谯楼打罢了二更时候,等候了贼子到好报冤仇。}

\textbf{汤勤 (内)走哇。}

\textbf{(丑皂隶全上)}

\textbf{汤勤 【二黄摇板】人得喜事精神爽,月到中秋分外光。}

\textbf{皂隶 汤老爷啊,脑袋掉了。}

\textbf{汤勤 咳,这是怎么讲话,纱帽掉了。}

\textbf{皂隶 咳,纱帽不在脑袋上戴着么不是的。}

\textbf{汤勤 咳,捡起来。}

\textbf{皂隶 哦,捡起来。}

\textbf{汤勤 你前面带路。}

\textbf{皂隶 慢着,我不是跟着你的么,理应你在头里,我在后头。}

\textbf{汤勤 咳,你在头里。}

\textbf{皂隶 我在头里?}

\textbf{汤勤 还是你头里。}

\textbf{皂隶 我要是在头里,那岂不是你成了跟着我的么?}

\textbf{汤勤 咳,你在头里,喝道拦挡闲人。}

\textbf{皂隶 我在头里,就是这么办。}

\textbf{汤勤 前面带路。}

\textbf{皂隶 咳,我在头里。屎来啦,屎来啦。}

\textbf{汤勤 狗才,你怎么说屎来啦。}

\textbf{皂隶
人家的官大,咱们的官小,挡住道儿不教咱们过去,这一闻见屎来啦,齁臭}\protect\hyperlink{fn579}{\textsuperscript{579}}\textbf{的,人家自然而然的就``躲开罢'',``躲开他罢''\ldots{}\ldots{}今晚上你那洞房花烛,叩门回事}\protect\hyperlink{fn580}{\textsuperscript{580}}\textbf{齁臭的,不吉祥。}

\textbf{汤勤 咳,讲得有理,屎就屎,再去喝道。}

\textbf{皂隶 屎蛋来啦。}

\textbf{汤勤 你怎么又喝道说屎蛋来啦?}

\textbf{皂隶
您呐不知道,这屎蛋是接年}\protect\hyperlink{fn581}{\textsuperscript{581}}\textbf{的了,比屎还臭,那人家闻见,说``比屎还臭,快快的躲开罢。''今儿晚上洞房花烛,哪个不来贺喜?}

\textbf{汤勤 讲得有理,唉,屎蛋就屎蛋。}

\textbf{皂隶 屎蛋不是?哎,屎蛋来------屎蛋来了。}

\textbf{(皂隶、汤勤原场)}

\textbf{汤勤 哎呀,我走不动了。}

\textbf{皂隶 走不动了就算到了罢。}

\textbf{汤勤 你去叫开。}

\textbf{皂隶 哦,我去叫门------开门来。}

\textbf{雪艳 是哪个叫门?}

\textbf{皂隶 我再问问去。汤老爷,里面问何人叫门。}

\textbf{汤勤 你就说汤老爷到了。}

\textbf{皂隶 哎是啦,开门来。}

\textbf{雪艳 是哪个叫门?}

\textbf{皂隶 汤老爷到了,快来开门。}

\textbf{雪艳 这里不认得什么汤老爷。}

\textbf{皂隶 我再问问去。汤老爷,里面说了,不认得什么汤老爷。}

\textbf{汤勤 你说是汤勤汤老爷到了,快来开门罢。}

\textbf{皂隶 开门来,汤勤汤老爷到了,快快开门罢。}

\textbf{雪艳 我这里不知道什么汤老爷。}

\textbf{皂隶 哎呀,这个麻烦着,汤老爷,里头又说了,她不知道什么汤老爷。}

\textbf{汤勤 你再去言道,说是汤勤汤老爷、裱褙的汤老爷、裱字画的汤老爷。}

\textbf{皂隶
哎,这么些个啰哩啰嗦,里头听着:汤勤汤老爷、裱褙的汤老爷,裱字画的、婊子下的汤老爷。}

\textbf{雪艳 我全都不认识。}

\textbf{皂隶 哎呀汤老爷,里头说了,她全都不认识。您呐自己去罢。}

\textbf{汤勤 待我去叫门。啊雪娘子开门,汤勤汤老爷到了,快快地开门来。}

\textbf{雪艳 这里无有什么汤老爷。}

\textbf{汤勤 哎呀,她不开门如何是好?}

\textbf{皂隶
昨日我舅舅死了,我舅舅下销}\protect\hyperlink{fn582}{\textsuperscript{582}}\textbf{落下一把斧子,她不开门,我就劈。}

\textbf{汤勤 前去劈门。}

\textbf{皂隶 劈你们家的坟。}

\textbf{汤勤 咳,劈门。}

\textbf{皂隶
劈门。呔,里面听着:汤老爷说了,若不开门,可要劈你们的棺材板啦。}

\textbf{雪艳 待我开门就是了------有请汤老爷。}

\textbf{皂隶 里面有请汤老爷。}

\textbf{汤勤 哎呀。}

\textbf{(汤勤醉介,坐)}

\textbf{皂隶 汤老爷醉死了。}

\textbf{汤勤 咳,大喜了,大喜了。}

\textbf{皂隶 大喜了,小人讨赏。汤老爷您呐赏给我几吊罢。}

\textbf{汤勤 几吊钱?连给几百钱都没有。}

\textbf{皂隶 得了汤老爷,不论多少您呐赏我俩钱罢。}

\textbf{汤勤 哎,太唠叨了,连一个大钱都没有!}

\textbf{皂隶 我不要了,你留着钱买棺材罢。大爷走了。}

\textbf{(皂隶下)}

\textbf{汤勤 这个狗才什么东西,胡说八道的,真真岂有此理话啊!}

\textbf{(丑报丧上)}

\textbf{报丧人 唉呀,汤老爷,我姥姥死了汤老爷。}

\textbf{汤勤
哎呀,我这里乃是大喜的事情,你怎么跪在我这里报丧来了。这是哪里说起?滚出去。}

\textbf{报丧人 汤老爷你呐赏口棺材罢。}

\textbf{汤勤
咳,哪里来的棺材,赏你一口``狗碰头''}\protect\hyperlink{fn583}{\textsuperscript{583}}\textbf{去罢。}

\textbf{报丧人 ``狗碰头''留着装你自己罢。}

\textbf{(丑报丧下)}

\textbf{(四青袍、丑书吏上)}

\textbf{众 走哇,走哇。}

\textbf{书吏 列位请了。}

\textbf{众 请了。}

\textbf{书吏
你我奉了陆大人之命,与汤勤贺喜,教我们大家将他灌醉,大家走哇。}

\textbf{(众走原场)}

\textbf{书吏 到了,大家进去啊。}

\textbf{书吏 汤老爷在哪里。}

\textbf{汤勤 列位来了。}

\textbf{众 来了。汤老爷醉死了。}

\textbf{汤勤 咳,该死的,是什么话,乃是大喜了。}

\textbf{书吏 不错,大喜了。}

\textbf{汤勤 你们是哪里来的?}

\textbf{书吏 我等奉了陆大人之命前来道喜。}

\textbf{汤勤 有劳众位了。}

\textbf{书吏 汤老爷是要喝个喜酒儿。}

\textbf{汤勤 是要吃的啊。}

\textbf{书吏 待我(来)把敬把敬。}

\textbf{汤勤 不敢当了。}

\textbf{书吏 使得的。汤老爷,你吃了我这酒一杯,死后变乌龟。}

\textbf{汤勤 诶,这是什么讲话(或:这是怎么讲话)?咳,后来大富贵。}

\textbf{书吏 不错,(后来)大富贵。}

\textbf{书吏 汤老爷,你吃了这二杯酒,死了变黄狗(或:死后变黄狗)。}

\textbf{汤勤 诶,越发的不像话了。呃,后来子孙有。}

\textbf{书吏 不错的,后来子孙有。}

\textbf{书吏 汤老爷,这三杯酒儿入肚肠,死了之后见阎王。}

\textbf{汤勤 诶,这是怎么讲话?后来产生状元郎。}

\textbf{书吏 诶,状元郎。}

\textbf{书吏 汤老爷,还要饮三大杯(或:再饮三大杯)。}

\textbf{汤勤
诶,我这酒吃不得了哇\ldots{}\ldots{}呵呵,\ldots{}\ldots{}(足)够了。}

\textbf{书吏 诶,可以再饮三杯,连中三元,连生贵子。}

\textbf{汤勤 哦,要连生贵子。呃,再饮再饮。}

\textbf{(汤勤醉介)}

\textbf{汤勤 呃,呜呜呜。}

\textbf{书吏 列位,汤勤醉了,你我走罢。}

\textbf{众 走哇。}

\textbf{书吏 如此走哇。}

\textbf{书吏 汤老爷天不早了,我们要回去了哇。}

\textbf{汤勤 哎,列位请回来罢。}

\textbf{书吏
啊雪娘子,汤勤是醉了,与你丈夫报仇也在你,不报仇也在你。我们大家走了。}

\textbf{(书吏锁门)}

\textbf{众 走哇。}

\textbf{(众下)}

\textbf{雪艳 汤老爷夜已深了,请安歇去罢。}

\textbf{汤勤 啊雪娘子,你我安歇了罢啊。}

\textbf{(汤勤进帐,脱衣介)}

\textbf{雪艳 汤老爷,汤老爷\ldots{}\ldots{}}

\textbf{雪艳 \textless{}叫头\textgreater{}且住!}

\textbf{雪艳 看贼子已睡,此时不下手待等何时?待我动起手来罢。}

\textbf{雪艳
【二黄散板】见贼子不由我心中怒恼,蓟州城害夫君心如火烧。我身旁暗藏刀把夫仇来报,管教这狗奸贼命赴阴曹。}

\textbf{(雪艳进帐刺汤勤介)}

\textbf{汤勤 唉呀!}

\textbf{(\textless{}扑灯蛾\textgreater{}汤勤死介)}

\textbf{雪艳 且住,看贼子已死,不免逃走了罢。}

\textbf{雪艳
唉呀且住,想我女流之辈,往哪里逃走?也罢,不免行个自尽了罢。}

\textbf{雪艳 喂呀\ldots{}\ldots{}(哭介)}

\textbf{雪艳
【二黄散板】眼望钱塘忙拜定,拜谢老爷收留恩。一把宝剑拿在手,}

\textbf{雪艳 罢!}

\textbf{雪艳 【二黄散板】不如一死赴幽冥。}

\textbf{(雪艳自刎下)}

\textbf{{[}第五场{]}}

\textbf{(四青袍、书吏上)}

\textbf{众 天已亮了,大家开开门儿看一看。怎么样了哇?}

\textbf{(开门,众进门)}

\textbf{书吏
哦哟------列位,汤勤被雪艳刺死了。她乃女流之辈。四面俱是高墙,她可往哪里逃走。走是走不了的,她手拿着宝剑,她就这样自刎了哇。唉,真乃是为丈夫报仇,尽节!她死而无怨也。呃,呃,呃\ldots{}\ldots{}}

\textbf{(书吏自刎介)}

\textbf{众 得,他也死了。唉呀,咱们大家将他抬下去罢!}

\textbf{(四青袍抬书吏下)}

\newpage
\hypertarget{ux96eaux676fux5706-ux4e4b-ux83abux6000ux53e4}{%
\subsection{雪杯圆 之
莫怀古}\label{ux96eaux676fux5706-ux4e4b-ux83abux6000ux53e4}}

\textbf{{[}第一场{]}}

\textbf{马来!}

\textbf{【二黄散板】昔年仆官}\protect\hyperlink{fn584}{\textsuperscript{584}}\textbf{到帝邦,可恨汤勤起不良。堪叹义仆把命丧,怎不令人心惨伤。}

\textbf{下官莫怀古。只为``一捧雪'',得罪严府,是我弃官逃走。可恨汤勤,搬动是非,要害我一死。多亏恩人义仆莫成,在蓟州堂上替我一死,是我逃往湖北,隐姓埋名。前者戚贤弟有书信到来,教我速回蓟州,不知有何事故。只得急急赶来。}

\textbf{来此离蓟州不远,我不免马上加鞭!}

\textbf{【二黄原板】一日离家一日深,好似孤雁宿寒林。我心中只把那汤贼恨,害得我一家人两下离分。多亏了忠义仆替我的性命,埋名隐姓湖广存身。戚贤弟来了一封信,教我转回蓟州城。眼观蓟州西门近,}

\textbf{【二黄散板】扬鞭打马奔故城。}

\textbf{{[}第二场{]}}

\textbf{马来!}

\textbf{【二黄散板】催马加鞭到柳林呐,阴风惨惨好惊人。下得马来柳林进,掌家坟墓哪厢存。}

\textbf{来此柳林,但不知恩人坟墓,今在何处。}

\textbf{看那旁有一碑碣,待我上前看来?}

\textbf{``明故太常寺正卿莫公之墓''。}

\textbf{莫公之墓!}

\textbf{\textless{}三叫头\textgreater{}掌家!莫成!唉,恩人呐,呃\ldots{}\ldots{}(哭介)}

\textbf{【二黄导板】见坟台不由人珠泪滚滚,}

\textbf{\textless{}三叫头\textgreater{}掌家!莫成!
唉,恩人呐,啊\ldots{}\ldots{}(哭介)}

\textbf{【二黄散板】黄土埋定小莫成。似这等义仆啊\textless{}哭头\textgreater{}令人敬,恩人呐!}

\textbf{【二黄散板】留下美名万古存。}

\textbf{(念)小小门户两扇开,只见坟墓眼前排。若问此是谁家墓,有罪犯官在此埋。}

\textbf{\textless{}叫头\textgreater{}汤勤呐!狗奸贼!}

\textbf{有我怀古在世一日,与你势不两立也!}

\textbf{【二黄散板】恨奸贼}把我的牙咬坏,屈害忠良为何来。你生我存冤仇在,

\textbf{奸贼呀!贼!}

\textbf{【二黄散板】}你死我亡两丢开。

\textbf{此乃我家坟墓,怎说是你家的呢。}

\textbf{呜哙呀,看那一妇人好像我妻傅氏模样,待我来冒叫一声。}

\textbf{啊,那旁敢是傅氏?}

\textbf{下官怀古在此。}

\textbf{夫人不必惊慌,下官尚在。}

\textbf{不曾死。}

\textbf{夫人请看。}

\textbf{\textless{}三叫头\textgreater{}夫人!我妻!唉,妻呀,呃\ldots{}\ldots{}(哭介)}

\textbf{【二黄散板】夫妻相逢在今朝。}

\textbf{【二黄散板】两鬓不觉似银条。}

\textbf{【二黄散板】乌鸦拆散凤凰巢。}

\textbf{你来看!}

\textbf{【二黄散板】他,他\ldots{}\ldots{}他在那蓟州堂替我代劳。}

\textbf{下官在湖北避难,前者戚贤弟有书信到来,教我转回蓟州。来至此处,不想遇见夫人。夫人你缘何至此?}

\textbf{罢了。}

\textbf{这\ldots{}\ldots{}}

\textbf{我是乘马而来,你爹爹乃是步行,故而还未曾到此。}

\textbf{你去林外观看。}

\textbf{唉!这里面就是你的爹爹呀,呃\ldots{}\ldots{}(哭介)}

\textbf{就依夫人。}

\textbf{不必拜了,起来。}

\textbf{你我一同,去往蓟州,在戚贤弟那里安身便了。}

\textbf{儿啊,与你母亲带马。}

\textbf{【二黄散板】莫怀古好一似丧家犬,隐姓埋名有数年。今日夫妻重相见,可叹义仆丧黄泉。}

\textbf{打严嵩}\protect\hyperlink{fn585}{\textsuperscript{585}}

\textbf{{[}第一场{]}}

\textbf{(\textless{}小锣打上\textgreater{})}

\textbf{邹应龙
{[}引子{]}奸贼不参反成仇,不觉白了少年头。(或:奸佞当道,何日里,才把贼平。)}

\textbf{邹应龙 (念)空怀忠义胆,枉自伴君前。为参奸佞贼,只落口怨天。}

\textbf{邹应龙
下官邹应龙。嘉靖驾前为臣,官居外帘巡城御史之职。只因严嵩老贼,在朝专横(或:专权),苦害贤臣(或:忠良),是我等同年三十六人,在双塔寺中定下(或:盟下;许下)心愿,以我为首,定要参倒老贼(或:以我为首,参奏老贼)。是我与开山王府常小国公定下一计,救下邱、马二将}\protect\hyperlink{fn586}{\textsuperscript{586}}\textbf{,去至严府,假献殷勤,暗中察勘老贼过错。今当三六九日,老贼议事之期,我不免去至相府,见机而行。(或:我不免去至严府,假献殷勤,暗中察勘。)(或:下官邹应龙。嘉靖驾前为臣,官居外帘巡丞御史。只因严嵩老贼,在朝专横,苦害贤臣,我等同年三十六人,在双塔寺中盟下誓愿,定要参奏老贼。以我举首,为国除害,日前闻知,老贼要害邱、马两将,是我与开山王府常小国公定下一计,将他二人救下。今去至严府,假献殷勤,暗中察勘老贼动静。今当三六九日,老贼议事之期,我不免前去见机行事。)}

\textbf{邹应龙
正是:(念)胆大扳龙角,心雄拔虎牙。(或:胆壮攀龙角,心雄拔虎毛。或:参倒贼奸佞,方为栋梁臣。)}

\textbf{邹应龙
【西皮原板】嘉靖爷坐山河风调雨顺,信宠那严嵩贼苦害贤臣。行奸计杀夏言丧了性命,进谗言害曾铣阖家满门。(或:他不该害死了杨继盛,杀夏言、害曾铣阖家满门。)众位年兄俱议论,不灭老贼枉为人。}

\textbf{(\textless{}小锣打下\textgreater{})}

{[}第二场{]}

\textbf{(严侠上,\textless{}小锣打上\textgreater{})}

\textbf{严侠 啊哈------}

\textbf{严侠 (念)}相府门前七品官,见他容易见我难。

\textbf{严侠
我严侠是也,蒙太师信任,身当相府门官,小有威势。今当三六九日,太师议事、放官之期,恐有帘外官员前来拜谒,不免府门伺候便了。}

\textbf{(邹应龙上,\textless{}小锣打上\textgreater{})}

\textbf{邹应龙 (念)深山虎狼震,要做打猎人。}

\textbf{邹应龙
来此(已是)严府(或:相府),(看)那旁(或:厢)打坐严侠,待我向前(或:近前)------}

\textbf{邹应龙
尊官}\protect\hyperlink{fn587}{\textsuperscript{587}}\textbf{请来搭话(或:啊,尊官请过来)。}

\textbf{(严侠 是。)}

\textbf{邹应龙 下官(或:弟)这厢有礼------}

\textbf{严侠
施礼为何,请问尊姓?(或:呃,诶,呃\ldots{}\ldots{}您是哪一位?)}

\textbf{邹应龙 怎么(你)连下官你都不认识了?}

\textbf{严侠 呃,瞧您面熟,不敢下笊篱。(或:(惊介)眼熟称呼不上来。)}

\textbf{邹应龙 外帘御史邹应龙就是我哇。}

\textbf{严侠 哦,就是外帘御史邹应龙哇。(或:你就是外帘御史邹应龙。)}

\textbf{邹应龙 嗯。(或:正是。弟这厢有礼。)}

\textbf{严侠 到此何事? (或:呃,施(/此)礼为何?)}

\textbf{(邹应龙 请问亲翁,太师可曾升堂议事?)}

\textbf{(严侠 已经升堂。)}

\textbf{邹应龙 求见太师。(或:小官求见)}

\textbf{严侠 为了何事?}

\textbf{邹应龙 有好心献上。}

\textbf{严侠 太师已经升堂。}

\textbf{邹应龙 烦劳通禀。}

\textbf{严侠 拿来。}

\textbf{邹应龙
什么,哦哦哦,想是手本。尊官(请看),小官}\protect\hyperlink{fn588}{\textsuperscript{588}}\textbf{有失打点了。(烦劳通禀。)}

\textbf{严侠 哎,我说邹老爷,八成啊,您没来过这儿罢?
(或:诶,邹老爷我记得您是来过呀?)}

\textbf{邹应龙 不错,是初次。(或:乃是初次。)}

\textbf{严侠
这就难怪喽,你要见我家太师爷,可知这严府门口儿的规矩?(或:哎呀,这------那可就------难怪了,求见太师,可有个规矩。)}

\textbf{邹应龙
怎么,还有规矩?(或:府上啊\ldots{}\ldots{}府上------还有规矩?,或:哦,严府上还有什么规矩?)}

\textbf{严侠 呃,(是了,)不以规矩,是不能成方圆呐。}

\textbf{邹应龙 哦,倒要请教。}

\textbf{严侠 诶,(求见太师,)大礼三百二,小礼二百四。}

\textbf{邹应龙 有礼------}

\textbf{严侠 就见!}

\textbf{邹应龙 无礼呢?}

\textbf{严侠 就免见呗。}

\textbf{邹应龙 啊尊官,下官(或:小官)今日来得忙迫,未带大礼,改日再补。}

\textbf{严侠 那就改日再见呗!}

\textbf{邹应龙
呜哙呀,难怪(或:怪道)严嵩老贼在朝专横(或:专权),就是他门下之人(或:府下之人),都是(或:也是)这般行事。改日再见\ldots{}\ldots{}我就改日再见罢------(哎呀!)想我邹应龙若说(或:要是说)他门下之人不过,还与老贼作的什么对啊?!}

\textbf{邹应龙 这这这\ldots{}\ldots{}嗯,我自有道理。}

\textbf{邹应龙 (亲翁请)过来!}

\textbf{严侠 诶,长调门了。}

\textbf{严侠 嘿,过来就过来。干什么?}

\textbf{邹应龙 (我来问你,)太师爷(或:老太师)每日可要朝王见驾?}

\textbf{严侠 自必朝王见驾。}

\textbf{邹应龙 朔望之日可于太庙降香?}

\textbf{严侠 怎不太庙降香?}

\textbf{邹应龙
着哇,待等老太师朝王见驾、太庙降香的时节(或:倘若老太师朝王见驾、太庙降香),(下官)我就走上前去(或:我就向前),(一把)拦住了轿杆,禀告老太师就说道:那一日小官求见太师,有好心献上,(府上)有那么一位把门的官儿与我(或:小官)要什么``大礼三百二,小礼二百四'',有礼就见(或:有礼则见),无礼免见。今日呀!呵呵,我也不见了,(我们)改日再见罢!}

\textbf{严侠 哎诶诶,回来回来。}

\textbf{邹应龙 呃,改日再见罢!}

\textbf{严侠 你拿过来吧。嘿嘿嘿嘿\ldots{}\ldots{}(陪笑介)}

\textbf{严侠
我说邹老爷哟,我跟您(说句游戏之言,)闹着玩儿,怎么你就帘子脸儿------叭嗒,诶,就掉下来了?}

\textbf{邹应龙 哼!(这是邹老爷的脾气。)(或:那你不去通禀啊?!)}

\textbf{严侠
呃,通报可是通报哇。(或:得得得得了,我啊给你通禀报)这儿可是有尺寸的地方,诶,你得------往下站。}

\textbf{邹应龙 哦,我往下站。}

\textbf{严侠 诶,再往下站!}

\textbf{(邹应龙 嗯。)}

\textbf{严侠 还得再往下站!}

\textbf{邹应龙 啊?!你叫你邹老爷站在何处啊?啊?!}

\textbf{严侠 诶,得得得,您爱站哪儿就哪儿站得了。}

\textbf{邹应龙 哼,势利的小人!}

\textbf{(严侠
嘿,诚然。唉,打一早啊就觉得不大对劲儿,得,还得\ldots{}\ldots{}没法子给他通禀吧。)}

\textbf{严侠 有请太师爷。}

\textbf{严嵩 (内)【西皮导板】昔日有个王莽臣,}

\textbf{严嵩
【西皮快板】起下谋朝篡位心。巧计设下松棚会,药酒毒死亲帝君。老夫压定文共(或:和)武,要夺大明锦乾坤。三六九日放官任,}

\textbf{严嵩 【西皮摇板】威风好似五阎君。}

\textbf{严嵩
(念)}君不君来臣不臣,朝事不问嘉靖君。私造九龙冠一顶,要夺大明锦乾坤。

\textbf{严嵩
老夫,严嵩,嘉靖皇帝驾前为臣。我儿世蕃与嘉靖皇帝同年同月同日出生,嘉靖皇帝有天子之位,我儿难道就无有九五之尊?今当三、六、九日,恐有外帘官员前来拜谒,严侠,}

\textbf{严侠 有。}

\textbf{严嵩 大事通报(或:大事通禀),小事任你去办。}

\textbf{严侠 启禀太师:外帘御史邹应龙求见。}

\textbf{严嵩 什么邹应龙,老夫这道衙门,见也罢,不见也罢。}

\textbf{严侠 他言道:有好心进献(或:好心来献)。}

\textbf{严嵩 哦,有好心献上。吩咐站堂伺候。}

\textbf{(四站堂上,严嵩坐内场椅)}

\textbf{严嵩 严侠,}

\textbf{严侠 有。}

\textbf{严嵩
传话出去:教那邹应龙东角门施礼,西角门打躬,低头合目,报门而进。}

\textbf{严侠 遵命!}

\textbf{严侠 下面听者:}

\textbf{邹应龙 在!}

\textbf{严侠 太师传话:命邹应龙东角门施礼,}

\textbf{邹应龙 是!(或:有!)}

\textbf{严侠 西角门打躬。}

\textbf{邹应龙 是!}

\textbf{严侠 低头合目,报门而进呐!}

\textbf{邹应龙 是是是!}

\textbf{严侠 你要仔细,你要打点!}

\textbf{邹应龙 知、知、知、知道了!}

\textbf{邹应龙
【西皮快板】忽听严侠传我进,狐假虎威乱胡行。(或:严侠教我报门进,不由应龙怒气生。)怠慢本官多(或:莫)侥幸,老爷是尔对头人。东角门首礼施定,西角门首打一躬。整装敛容相府进(或:回廊进),}

\textbf{邹应龙 【西皮摇板】参拜皇王宠信臣。}

\textbf{邹应龙 参见太师。}

\textbf{严侠 禀太师:邹应龙到。}

\textbf{严侠 邹应龙到。}

\textbf{严侠 邹应龙到啊------}

\textbf{严嵩 邹应龙!}

\textbf{邹应龙 小官在!(或:有)}

\textbf{严嵩 (念)老夫门深似海。}

\textbf{邹应龙 (念)小官好心献上。(或:启禀老太师:小官有好心献上。)}

\textbf{严嵩 怎么有好心献上。(或:哦,你有好心献上么?)}

\textbf{邹应龙 正是。}

\textbf{严嵩 起来。}

\textbf{邹应龙 谢太师。}

\textbf{严嵩 严侠,与邹老爷(或:大人)看坐。}

\textbf{严侠 是,有坐。}

\textbf{邹应龙 且慢,太师虎威在此,哪有小官的座位?}

\textbf{严嵩 有话叙谈,焉有不坐之理?坐下。}

\textbf{邹应龙 小官告坐。}

\textbf{严嵩 哦------}

\textbf{严侠 诶,启禀太师爷,邹应龙他不敢坐呀。}

\textbf{严嵩 教他大胆坐下。}

\textbf{严侠 太师爷啊,教你坐下。}

\textbf{邹应龙 谢太师。}

\textbf{严侠 邹老爷您请坐请坐。}

\textbf{邹应龙 尊官请坐。}

\textbf{严侠 嘿嘿,我站惯了。}

\textbf{严嵩 邹应龙!}

\textbf{邹应龙 在!}

\textbf{(严嵩 你有何好心献上?}

\textbf{邹应龙
前番太师下朝命锦衣卫陆唐追赶何人?(或:启禀老太师,小官前日巡城打从开山王府经过,有那锦衣卫人员(或:有一位长官),他叫什么陆\ldots{}\ldots{})}

\textbf{(严嵩 敢是陆唐?)}

\textbf{(邹应龙
正是。太师命他追赶何人?(或:不错,正是锦衣卫陆唐,押解邱、马二将。))}

\textbf{严嵩 追赶邱、马二将。}

\textbf{邹应龙 可曾追获?}

\textbf{严嵩 未见回报。}

\textbf{邹应龙 追赶不上了。}

\textbf{严嵩 怎见得?}

\textbf{邹应龙
那日小官巡查御街,路过开山王府,偶遇常小国公拦下邱、马二将,窝藏府中。并将陆唐抓进府去,吊在头门以里,仪门以外,日间打至犬吠,晚来打到鸡鸣。(打来打去,)还有两句歹话。(或:那陆唐解压邱、马两将路过开山王府,正遇常小国公拦下邱、马两将,窝藏府中。并将陆唐抓进府去,吊在头门以里,仪门以外,日间打至犬吠,晚来打至鸡鸣。还有两句言语。)}

\textbf{严嵩 哪两句歹话?(或:呃呃,有什么言语?)}

\textbf{邹应龙 老太师台前(或:在此),小官不敢言讲。}

\textbf{严嵩 大胆讲来!}

\textbf{邹应龙 打在他人腿上,羞在老太师脸上(或:犹如记在太师爷的脸上)。}

\textbf{严嵩 怎么讲?}

\textbf{邹应龙 老太师(或:太师爷的)脸上。}

\textbf{严嵩 好奴才!(或:可恼!)}

\textbf{严嵩
【西皮摇板】老夫闻言怒气冲,开言大骂常宝童。自古常言道得好,打犬还看主人翁。}

\textbf{严嵩 邹应龙!}

\textbf{邹应龙 在。}

\textbf{严嵩 是你亲眼得见(或:还是耳闻还是亲见)?}

\textbf{邹应龙 是小官亲眼得见。}

\textbf{(邹应龙 且慢,太师意欲何往?)}

\textbf{严嵩 老夫有意上殿奏本(或:参奏)。}

\textbf{(邹应龙 倘若圣上问起,何人见证?)}

\textbf{严嵩 你可与老夫做一见证?}

\textbf{邹应龙 小官愿与(老)太师做一见证。(或:当得效劳。)}

\textbf{邹应龙 呃呃,只怕不便。}

\textbf{严嵩 为何?}

\textbf{邹应龙
(怎奈)小官官卑职小,不能上(皇王金)殿见驾(或:见君),也是枉然。}

\textbf{严嵩 在朝官居何职?}

\textbf{邹应龙 外帘巡城御史。}

\textbf{严嵩 嗯,老夫不通圣命,升你为内帘御史。}

\textbf{(邹应龙 但不知几时领凭上任?)}

\textbf{严嵩
嘉靖封官,少不得周年半载;老夫放官,即刻上任。(或:即时领凭上任。)}

\textbf{邹应龙 (多)谢太师。}

\textbf{严嵩 随同(或:随定)老夫道后。}

\textbf{邹应龙 遵命。}

\textbf{严嵩 顺轿上朝!}

\textbf{(一番两番,严嵩跪台中间,台中间摆一张桌,上摆香炉)}

\textbf{严嵩 臣,严嵩见驾,吾皇万岁!}

\textbf{嘉靖 (内)老卿家上殿,有何本奏?}

\textbf{严嵩 启奏万岁:今有常小国公窝藏邱、马两将,请旨定夺。}

\textbf{嘉靖 (内)老卿家亲眼还是耳闻?}

\textbf{严嵩 乃是外帘御史邹应龙亲眼所见,万岁圣鉴。}

\textbf{嘉靖 (内)邹应龙官卑职小,焉能上殿?}

\textbf{严嵩 老臣有一行大罪。}

\textbf{嘉靖 (内)卿家何罪之有?}

\textbf{严嵩 老臣未通圣命,放他为内帘御史。}

\textbf{嘉靖 (内)卿家放官,与朕一样。何罪之有?殿角赐座。}

\textbf{严嵩 老臣谢座。}

\textbf{嘉靖 (内)内侍,宣邹应龙上殿。}

\textbf{内侍 (内)邹应龙上殿呐。}

\textbf{邹应龙 (内)领旨!}

\textbf{邹应龙 (念)袖内藏文本,}

\textbf{严嵩 嗯------}

\textbf{邹应龙 (念)假意顺谗臣。}

\textbf{邹应龙 臣邹应龙见驾,吾皇万岁。}

\textbf{嘉靖 (内)邹应龙。}

\textbf{邹应龙 臣。}

\textbf{嘉靖 (内) 常小国公窝藏邱、马两将,可是你(或:卿)亲眼得见?}

\textbf{邹应龙 是臣亲眼得见。}

\textbf{嘉靖 (内)与卿无干,下殿。}

\textbf{邹应龙 谢万岁。}

\textbf{邹应龙 正是:(念)点起灯芯火,要烧万重山。}

\textbf{(邹应龙一指,下)}

\textbf{嘉靖 (内)太师接旨。}

\textbf{严嵩 臣。}

\textbf{嘉靖
(内)赐卿圣旨一道,校尉四十名,去至开山王府,押解常小国公上殿辩理。}

\textbf{(太监递圣旨,严嵩接旨)}

\textbf{严嵩 领旨。}

\textbf{(一翻两翻}\protect\hyperlink{fn589}{\textsuperscript{589}}\textbf{)}

\textbf{(严嵩下轿,进门坐中间,严侠上场门上,邹应龙下场门上,邹应龙跟过来,严侠站旁边)}

\textbf{邹应龙 老太师爷回府来了?}

\textbf{严嵩 回府来了。}

\textbf{邹应龙 圣上怎样传旨(下来)?}

\textbf{严嵩 圣上赐校尉四十名,去到开山王府,押解常小国公上殿辩理。}

\textbf{邹应龙 有道明君。}

\textbf{严嵩 嗯,真是有道的明君。}

\textbf{严嵩 顺轿。(或:来,外厢开道。)}

\textbf{邹应龙
且慢!小官阻道。(或:且慢,不是小官在此,老太师险些把事办错了。)}

\textbf{严嵩 为何阻道?}

\textbf{邹应龙 太师爷意欲何往?}

\textbf{严嵩 捉拿常宝童,上殿辩理。}

\textbf{邹应龙
启禀老太师:万岁(或:圣上)虽赐有校尉(四十名),去到开山王府捉拿常小国公上殿,他有金杈银档,倘若打草惊蛇,反而大事难成。(或:倘若打草惊蛇,而况他又有金杈银档,倘若他抗旨不遵,反而大事难成。)}

\textbf{严嵩 依你之见?}

\textbf{邹应龙
依小官拙见:呃,就在府内(或:相府)百里挑十,十里选一(或:百中选十,十中选一),挑选上四十名精壮的家丁,扮作校尉模样,随定(或:随同)太师爷去到(或:去往)开山王府,(前去宣读圣旨,)捉拿常小国公上殿辩理。他若上殿,也就罢了(或:常小国公遵旨上殿,倒还不讲;或:倘若他上殿,那还不讲;)倘若他不肯上殿,这四十名校尉,(或:倘若他抗旨不遵,老太师命这些家人,或:若是他抗旨不遵,老太师命随从们将他)抬么,也就把他抬上了(皇王的)金殿呐。}

\textbf{严嵩 哦,这是何计?}

\textbf{邹应龙 此乃万全之计。}

\textbf{严嵩 怎么,万全之计。呃------呵呵哈哈哈\ldots{}\ldots{}(笑介)}

\textbf{严嵩 (啊,邹大人,)看将起来,你是老夫心腹之人了。}

\textbf{邹应龙
着啊,本来是太师爷的心腹人呐。(或:太师爷夸奖了。或:正是老太师的心腹之人。)呃呃呃,心腹之人好做,就是太师爷的金面难见呐。}

\textbf{严嵩 啊,你早来早见,晚了晚见。何言难见?}

\textbf{严侠 (低声)呃,是啊。}

\textbf{邹应龙
启禀太师爷:小官求见太师爷(或:小官那日有好心献上),(或:不是哟,那日小官求见太师有好心献上。)府上有一位门官,(他)问我要什么大礼三百二,小礼二百四,有礼就见(或:有礼通禀),无礼免见。想小官为了太师爷之事,难道拿银子打点不成么?
(或:为了老太师之事,难道还要一个穷御史借(或:小官拿)银子(来)打点不成么?)}

\textbf{严嵩 啊,竟有此事?!(或:哦,有这等事?!)你可认识此人?}

\textbf{邹应龙 小官见面就认得。(或:呃,见面就认识了。)}

\textbf{严嵩 好,将他抓来见我!}

\textbf{严侠 糟糕,来了,要坏。(或:呃,坏了,不好,闹到我这儿来了。)}

\textbf{邹应龙
尊官,我看你往哪里去?(或:呃呃呃,亲翁(或:尊官)我看见你了。)}

\textbf{严侠 诶诶,邹老爷,我给您倒茶去。}

\textbf{邹应龙
不用。你当的好差呀!(或:诶,多谢多谢,你的差事当得好哇,太师爷传你呀。(或:啊,尊官,老太师唤你呀。))}

\textbf{严侠 诶,全仗您栽培。}

\textbf{邹应龙
我在太师爷台前讲了你的好话。(或:诶,一定要重重有赏,随我见太师。)}

\textbf{严侠
谢谢您的提拔。(或:呵呵,我的铺盖卷儿早就打好喽\ldots{}\ldots{})}

\textbf{邹应龙 太师爷传。}

\textbf{严侠 不是我,不是我。}

\textbf{邹应龙 随我来!}

\textbf{严侠 报应到喽!}

\textbf{邹应龙 呃,就是此人。}

\textbf{严嵩 唗------}

\textbf{严嵩
你还要多少\ldots{}\ldots{},斩了!(或:胆大严侠,帘外官员你讹诈了多少,扯下去打!)}

\textbf{严侠 留头讲话哟! (或:太师爷,恩典喏!)}

\textbf{严侠 嘿嘿,邹老爷。}

\textbf{邹应龙 嗯哼!(或:是哪一位呀?)}

\textbf{严侠 哼!端起来了。邹老爷,邹老爷!}

\textbf{严侠 邹老爷您往下瞧,我在这儿呢。}

\textbf{邹应龙 哦,原来是尊官。}

\textbf{严侠 是我,诶,是,是我。(或:我在这儿呢。)}

\textbf{邹应龙 一时不见,你怎么矮了啊?}

\textbf{严侠 邹老爷我这儿给您跪着喽!}

\textbf{邹应龙
你亲翁你跪着则甚呐?(或:哦,不错,你是跪着呢。跪在你邹老爷面前则甚呐?)}

\textbf{严侠
邹老爷,您不知道,我在门口外头不是跟您------说笑话来着么。唉,太师爷知道了,要杀我(或:老太师,他要罚我)。}

\textbf{邹应龙 哦,要杀你?}

\textbf{严侠 诶。是要杀要杀。}

\textbf{邹应龙 你就让他杀啵! (或:那就让他杀罢。)}

\textbf{严侠 哎\ldots{}\ldots{}得了得了,您给我说个情儿啵!}

\textbf{邹应龙 哪个?}

\textbf{严侠 诶,求您喽。}

\textbf{邹应龙 要我与你讲个人情?}

\textbf{严侠 是喽,非您不可!}

\textbf{邹应龙 你可晓得严府的规矩呀?(或:你可知邹老爷的规矩啊?)}

\textbf{严侠
邹老爷,咱们讲人情就甭讲规矩喽。(或:得了得了,您别提这规矩了。)}

\textbf{邹应龙
尊官,你讲过啊:``没有规矩,不能成方圆''呐。(或:``大礼六百四,小礼四百八''。)}

\textbf{严侠
哎呦,常言道得好啊``大人不计小人过'',邹老爷,我可真服了您了。}

\textbf{邹应龙 我讲个人情不难,不知你的造化如何。(或:好,看你的造化。)}

\textbf{严侠 我就得瞧您的了。(或:我的造化就看您了。)}

\textbf{邹应龙 朝上跪!}

\textbf{(严侠转身面朝里跪)}

\textbf{邹应龙
啊太师爷(或:启禀老太师),若责罚此人,小官出入多有不便呐。(或:启禀太师爷,若斩此人,于小官出入不便。)}

\textbf{严嵩 敢是与他讲情?(或:敢是与这奴才讲情?)}

\textbf{邹应龙 老太师开恩。(或:太师爷恩德。或:太师爷开恩。)}

\textbf{严嵩 老夫准情。严侠,谢过邹老爷。}

\textbf{严侠 是,谢过邹老爷。}

\textbf{邹应龙 你要谢过老太师。}

\textbf{严侠 是,多谢老太师。}

\textbf{严嵩 严侠,还不谢过邹老爷。}

\textbf{严侠 多谢,多谢邹老爷。}

\textbf{邹应龙 谢过太师爷。(或:谢过老太师。)}

\textbf{严侠 多谢太师爷。}

\textbf{严嵩 谢过邹老爷。}

\textbf{邹应龙 谢过太师爷。}

\textbf{严嵩 谢过邹老爷。}

\textbf{邹应龙 谢过太师爷。}

\textbf{严侠 谢过太师爷。}

\textbf{严嵩 嗯------还不下去?!(或:教你谢过邹老爷,还不跪下?)}

\textbf{(严侠跪)}

\textbf{邹应龙 诶,起来起来。}

\textbf{严侠 我是两边受着``夹板气''呢。}

\textbf{严嵩 心腹人。你为了老夫之事,还是乘马而来,还是坐轿而来?}

\textbf{邹应龙 小官乃是步行。(或:与太师爷办事,自然是步行而来。)}

\textbf{严嵩 岂不跑坏心腹人的两腿?(或:特以地辛苦了!)}

\textbf{邹应龙 当得效劳。}

\textbf{严嵩
圣上赐有白龙御马,老夫相赠。赐老夫穿朝御马,我赠与你乘骑了吧。}

\textbf{邹应龙 小官不敢乘骑。}

\textbf{严嵩
也罢,严侠,将万岁所赐老夫穿朝白龙御马,卸去金鞍玉辔,另备鞍韂。命与邹老爷乘骑。}

\textbf{严侠 没落到银子,落了个带马。}

\textbf{严嵩 你得罪了邹老爷,与邹老爷带马赔------礼。}

\textbf{严侠 喳!得,邹老爷您上马。}

\textbf{邹应龙 太师爷虎威在此,将马往下带。}

\textbf{严侠 哦,往下带,是。哨,哨,哨\ldots{}\ldots{}}

\textbf{严嵩 严侠,往上带。}

\textbf{严侠 喳!是。嗯、嗯\ldots{}\ldots{}}

\textbf{严侠 邹老爷。}

\textbf{邹应龙 (这是)有尺寸的地方,往下带。}

\textbf{严侠 哦,往下带,哨,哨,哨\ldots{}\ldots{}}

\textbf{严嵩 无用的奴才,往上带。}

\textbf{严侠 喳!邹老爷。}

\textbf{邹应龙 还要往下带。}

\textbf{严嵩 往\ldots{}\ldots{}}

\textbf{邹应龙 往\ldots{}\ldots{}}

\textbf{严侠 邹老爷!够瞧老大半天的喽!}

\textbf{严嵩 上马去罢!}

\textbf{邹应龙 谢太师!}

\textbf{邹应龙 【西皮散板】月台下辞别了严太尊,}

\textbf{严侠 送邹老爷!}

\textbf{邹应龙 【西皮散板】叫声尊官你试听:}

\textbf{严侠 邹老爷您有话请讲。}

\textbf{邹应龙 【西皮散板】三百两银子有多少,}

\textbf{严侠 您没给我可也没要。}

\textbf{邹应龙 【西皮散板】有道是脸面值千金。}

\textbf{严侠 您可真有点面子。(或:您好大面子哦)}

\textbf{邹应龙 【西皮散板】适才太师对我论,}

\textbf{严侠 您可听得清楚。(或:太师爷就与您说得来。)}

\textbf{邹应龙 【西皮散板】我就是太师爷心腹的人。}

\textbf{严侠 一棵树的枣儿,就红了您这么一个。}

\textbf{邹应龙 【西皮散板】从今后不把你当尊官敬,}

\textbf{严侠 呃,您把我当什么呢?}

\textbf{邹应龙 哦------}

\textbf{严侠 我可没这小名儿。}

\textbf{邹应龙 【西皮散板】你就是邹老爷牵马坠镫------}

\textbf{严侠 手都酸喽!(或:来回不要盘缠喏。)}

\textbf{邹应龙 【西皮散板】一个势利的小人。}

\textbf{严侠 这回他可出了气喽!(或:嘿我也值得这么一骂)}

\textbf{严嵩
【西皮散板】咬牙切齿把宝童恨,窝藏邱、马罪不轻。人来与爷把道引,捉拿宝童面圣君。}

\textbf{(\textless{}抽头\textgreater{}下)}

{[}第三场{]}

\textbf{(\textless{}长锤\textgreater{}四太监上,常宝童上)}

\textbf{常宝童
【西皮摇板】先祖在朝功劳大,保定太祖定邦家。钦赐银挡与金杈,}

\textbf{(常宝童坐外场椅)}

\textbf{常宝童 【西皮摇板】官居王位第一家。}

\textbf{(\textless{}快长锤\textgreater{}邹应龙上)}

\textbf{邹应龙
【西皮快板】严府假意献殷勤,老贼把我当心腹人。暗藏金钩来拿定}\protect\hyperlink{fn590}{\textsuperscript{590}}\textbf{(或:暗藏金钩探鳌鱼),}

\textbf{(邹应龙 有劳了!)}

\textbf{邹应龙
【西皮摇板】千岁驾前(或:台前)说分明。(或:见了千岁说分明。)}

\textbf{邹应龙 参见千岁。}

\textbf{常宝童 邹官儿平身呐。}

\textbf{邹应龙 谢千岁。}

\textbf{常宝童 孩子们给邹官儿看坐。}

\textbf{邹应龙 谢坐。}

\textbf{常宝童 邹官儿(或:卿家)你发了财了?}

\textbf{邹应龙 怎见得是臣发了财了呢?}

\textbf{常宝童 你身穿大红,岂不是发了财了吗?}

\textbf{邹应龙 不错,是臣升了官了啊。}

\textbf{常宝童 是啊,升的什么官啊?}

\textbf{邹应龙 (外帘御史升为)内帘御史。}

\textbf{常宝童 可喜可贺,谁(或:何人)的保举?}

\textbf{邹应龙 (乃)严嵩(的)保举(或:保荐)。(或:严太师的保举)}

\textbf{常宝童 怎么着,变了奸臣了啊?!}

\textbf{常宝童 嘿!孩子们,撤座!(或:撤奸臣的座儿)}

\textbf{邹应龙
啊,慢来慢来,千岁,虽则是严嵩的保举(或:保荐),(臣)还是为开山王府办事啊。}

\textbf{常宝童 哦,你还是给本御办事?}

\textbf{邹应龙 是。}

\textbf{常宝童 那你就再坐下。}

\textbf{邹应龙 谢千岁!(或:谢坐。)}

\textbf{常宝童 邹官儿你可知罪?}

\textbf{邹应龙 臣知何罪?}

\textbf{常宝童 昨儿个约定与本御围棋玩耍,今儿个才到,岂不是有罪吗?}

\textbf{邹应龙 今日开山王府围不得棋了啊!
(或:哎呀,吓得微臣吃了一惊。千岁,如今这开山王府是围不得棋了啊!)}

\textbf{常宝童 怎么围不得了?}

\textbf{邹应龙 眼前就有一场大祸!}

\textbf{常宝童 哎呦,你可别吓唬我啊。}

\textbf{常宝童 我开山府欠粮?}

\textbf{邹应龙 不欠粮。}

\textbf{常宝童 缺饷?}

\textbf{邹应龙 也不缺饷。}

\textbf{常宝童
一不欠粮,二不缺饷,没什么大祸,开山府怎么围不得了棋了呢?}

\textbf{邹应龙
可恨(或:只因)严嵩老贼金殿参奏一本,要千岁与他上殿辩理。(或:道千岁隐藏邱、马二将,要千岁上朝与他辩理。)}

\textbf{常宝童 辩的何理?}

\textbf{邹应龙 隐藏邱、马二将。(或:乃是邱、马两将之事。)}

\textbf{常宝童 哎,本御将他二人献出就是嘛!}

\textbf{邹应龙 千岁此言差矣。(或:哎呀献不得,献不得。)}

\textbf{常宝童 何差?(或:为何?)}

\textbf{邹应龙
此时献了邱、马二将(或:若是将他二人献出),岂不(是)弄假成真?(或:既要献,当初就不该救啊。)}

\textbf{常宝童 那依你之见呢? (或:卿家的高见?)}

\textbf{邹应龙 这\ldots{}\ldots{}(思介)}

\textbf{邹应龙
依臣拙见:少时老贼捧旨到此,千岁用金锏挡住他的校尉,只放他一人进府。(或:也罢,少时老贼捧旨前来,千岁吩咐抬起皇挡将他一人放进府来,千岁用金锏挡住他的校尉。或:少时老贼捧旨到此,吩咐人役将皇挡抬出,将老贼一人挡进府来。)}

\textbf{常宝童 孩子们听见了没有?}

\textbf{众 听见了。}

\textbf{邹应龙
彼时他必然在银安殿上开读(或:宣读)圣旨,千岁言道:老太师不必开读,本御知罪。(或:他彼时必然宣读圣旨,千岁不要他宣读,就说``本御知罪''。)}

\textbf{常宝童 本御何罪之有?}

\textbf{邹应龙 愿将邱、马二将献上。}

\textbf{常宝童 诏罢之后?}

\textbf{邹应龙 请过圣旨。(或:将圣旨请过。)}

\textbf{常宝童 请过了圣旨?}

\textbf{邹应龙 赐他一个座位。}

\textbf{常宝童 不成!哼哼,开山王府哪有老贼的座位?}

\textbf{邹应龙
看在微臣的份上。(或:皇王金殿,二十四把金交椅,尚有老贼的座位呀。)}

\textbf{常宝童 哼,这可是瞧了你的。}

\textbf{邹应龙 多谢千岁。}

\textbf{常宝童 坐下之后?}

\textbf{邹应龙 问他是忠是奸呐。(或:千岁问他是忠臣还是奸臣。)}

\textbf{常宝童 他准得说是大大的忠臣呐。}

\textbf{邹应龙 (既然是大大的忠臣,)教他抬头观看!}

\textbf{常宝童 呃,看,看什么呀?}

\textbf{邹应龙
千岁将笼帘卷起,教他观看老皇御容、伴驾王}\protect\hyperlink{fn591}{\textsuperscript{591}}\textbf{真像(或:千岁将老皇御容、伴驾王的真像悬挂中堂),他身为大臣,见君不参,就是(或:就有)一项大罪(或:一行大罪)。}

\textbf{常宝童 他必然有辩呐。}

\textbf{邹应龙 由他(或:容他)去辩。}

\textbf{常宝童 辩罢之后?}

\textbf{(邹应龙 再赐他一个座位。)}

\textbf{邹应龙
(千岁)问他:开山府欠粮?(或:再教他坐下,问道:开山府欠粮?)}

\textbf{常宝童 不欠粮。}

\textbf{邹应龙 缺饷?}

\textbf{常宝童 不缺饷。}

\textbf{邹应龙
一不欠粮,二不缺饷,(太师爷你)来到开山王府有何贵干呐(或:太师到此何干呐)?}

\textbf{常宝童 呃,请本御上殿辩理呐?}

\textbf{邹应龙 辩得什么理啊?拿来。}

\textbf{常宝童 什么?}

\textbf{邹应龙 圣旨啊。}

\textbf{常宝童 本御方才请过去了?}

\textbf{邹应龙 原是请过了啊,千岁(你)与他个不认账啊!}

\textbf{常宝童 诶!这我成!(或:诶,我这回说回瞎话。)}

\textbf{邹应龙
千岁(就动起怒来,)言道:唗!胆大严嵩,今日在朝害文,明日在朝害武,(害来害去,害到小王(或:本御)的头上来了。你在金殿奉了圣旨,在哪厢失落,)今天竟然来到开山王府,前来讹诈本御,这回不打你两下惯坏了你的下次。来呀!脱袍打严嵩。吩咐左右。乒乒乓乓,糊里糊涂,打他一顿轰了出去,千岁的气也出了,你看此计如何?(或:就动起怒来,骂一声:胆大的严嵩,今日在朝害文,明日在朝害武,害来害去,害到本御的头上来了!今日若不打你,尤恐惯坏了你的下次。吩咐人役脱袍解带,乒乒乓乓,将他一顿暴打,糊里糊涂,就轰了出去。)}

\textbf{常宝童 打出祸来呢!}

\textbf{邹应龙 由(微)臣担待。}

\textbf{常宝童 那可就瞧你的了。}

\textbf{邹应龙 臣告便。}

\textbf{邹应龙 啊尊侍,少时老贼到此,要打在他的身上,不可打在他的脸上。}

\textbf{侍卫 却是为何?}

\textbf{邹应龙 自有妙用。}

\textbf{侍卫 哦,我等记下。}

\textbf{侍卫 圣旨下。}

\textbf{常宝童 诶,来了来了。}

\textbf{常宝童 你在哪儿?(或:那你哪里藏躲呢?)}

\textbf{邹应龙 屏风后面。}

\textbf{(邹应龙
啊列位,少时老贼到此,浑身都容你们打,面貌不要打坏。我自有用处,记下了。拜托拜托。)}

\textbf{邹应龙 老贼来了。}

\textbf{常宝童 快去回避。}

\textbf{常宝童 香案接旨。}

\textbf{(严嵩上,常宝童用锏指,严校尉下,严嵩进门)}

\textbf{严嵩 圣旨下。}

\textbf{常宝童 老太师不用宣读,小王知罪。}

\textbf{严嵩 小千岁何罪之有?}

\textbf{常宝童 愿将邱、马二将献上当今。}

\textbf{严嵩 圣旨转过。}

\textbf{常宝童 香案供奉。}

\textbf{(常宝童中间坐)}

\textbf{严嵩 小千岁请上,老臣大礼参拜。}

\textbf{常宝童 老太师您年高有德,不拜也罢。}

\textbf{严嵩 见了千岁,哪有不拜之理?}

\textbf{常宝童 哦,那就是小王我受你一拜}

\textbf{众 跪。}

\textbf{众 一个。}

\textbf{众 一个。}

\textbf{众 一个。}

\textbf{严嵩 小千岁,老臣我磕了三个响头了。}

\textbf{常宝童 老太师请起。}

\textbf{严嵩 谢千岁。}

\textbf{常宝童 孩子们,与老太师看座。}

\textbf{严嵩 千岁在此,哪有老臣的座位。}

\textbf{常宝童
哎呀,我的老太师啊,金殿之上,二十四把金交椅,都有您的座位,何况我小小的开山王府呢。您请坐请坐。}

\textbf{严嵩 谢千岁。}

\textbf{常宝童 别谢了,您请坐吧。老太师,您好啊。}

\textbf{严嵩 老臣有何德能,敢劳千岁动问。}

\textbf{常宝童 闲谈。}

\textbf{严嵩 谢千岁。}

\textbf{常宝童 他又谢。你在朝是忠臣,还是奸臣?}

\textbf{严嵩 为臣是大大的忠臣。}

\textbf{常宝童 瞧您这样,您这个扮相就是个忠臣。}

\textbf{常宝童 孩子们,将笼帘卷起!}

\textbf{严嵩
(惊介)哎呀呀!这个娃娃,领了哪个高明先生的指教,将老王的御容、伴驾王的真相悬挂中堂,老夫身为大臣者见君不参,就有一行大罪。哎呀,这这这\ldots{}\ldots{}嗯,我自有道理。}

\textbf{严嵩 啊小千岁,可容老臣一辩。}

\textbf{常宝童 孩子们!金盆打水。}

\textbf{严嵩 呃,呃\ldots{}\ldots{}打水何用呐?}

\textbf{常宝童 老太师你会变呐,呃,变个乌龟,与小王玩耍玩耍。}

\textbf{严嵩 呃,老臣乃是舌辩之辩。}

\textbf{常宝童 那你就辩!}

\textbf{严嵩 千岁,``今非朔望,闲不参君''。}

\textbf{常宝童 好,那您就再坐一坐!}

\textbf{严嵩 谢千岁。}

\textbf{常宝童 老太师,我这开山王府欠粮?}

\textbf{严嵩 不欠粮。}

\textbf{常宝童 缺饷?}

\textbf{严嵩 不缺饷。}

\textbf{常宝童 一不欠粮,二不缺饷,您来到我开山王府有何贵干呐?}

\textbf{严嵩 呃,请千岁上殿辩理。}

\textbf{常宝童 拿来。}

\textbf{严嵩 什么?}

\textbf{常宝童 圣旨啊。}

\textbf{严嵩 啊!方才千岁请过了。}

\textbf{常宝童 孩子们,你们谁请过圣旨了?}

\textbf{众人 没有。}

\textbf{严嵩 请过去了。}

\textbf{众人 没有。}

\textbf{常宝童
唗!胆大严嵩,今日在朝害文,明日在朝害武,害来害去,害在小王的头上来了!你失落圣旨,也是有的,不该前来讹诈本御。今儿不打你几下,惯了你的下次。孩子们,脱袍打严嵩!}

\textbf{严嵩 哎呀,小千岁,老臣挨不起啊。}

\textbf{常宝童
【西皮摇板】矫传皇命遭戏弄}\protect\hyperlink{fn592}{\textsuperscript{592}}\textbf{,开山王府岂能容。手持金锏将贼打,打死奸贼方称心。}

\textbf{(邹应龙踹严嵩,严嵩躺地上)}

\textbf{邹应龙 唗!唗!(你们)擅打大臣,该当何罪?!}

\textbf{众 有邹老爷在内。}

\textbf{邹应龙 哦,什么,有我(在内,呵呵哈哈)?诶,那就无有事了哇。}

\textbf{(常宝童 是你的主意。)}

\textbf{邹应龙 是我?那就无有事了哇。}

\textbf{常宝童 你可真坏啊!(或:好家伙,吓了我一跳。)}

\textbf{邹应龙
千岁,(或:怎样臣是一个坏人。坏人也罢,好人也罢,启禀千岁,)打了老贼,他必然(或:必定要)上殿奏本。千岁将老皇的御容,伴驾王的真像请入华车,抬上金殿,为臣邀请满朝文武,三十六名进士(或:再请满朝文武、众位进士)共参老贼。有何难哉?}

\textbf{常宝童 本御记下。(或:小王记下。)}

\textbf{邹应龙
啊千岁,你打了半日,可有名堂来?(或:列位,你们打了半日,可有名堂来?}\protect\hyperlink{fn593}{\textsuperscript{593}}\textbf{)}

\textbf{常宝童
嘿,没什么名堂,乱打一锅粥哦。(或:别说他们,连我都是乱打一锅粥了。)}

\textbf{邹应龙 (微)臣(要)赶至御街,要打出(或:打他)一个名堂。}

\textbf{常宝童 什么名堂?}

\textbf{邹应龙 这就是(或:这叫作):(念)文武齐喝彩,应龙闹京街。}

\textbf{常宝童 我却不信。}

\textbf{邹应龙
【西皮摇板】千岁但把心放稳,管叫老贼(或:打老贼)领人情。有劳皇挡来抬定(或:千岁休得不肯信,少时便知假和真。有劳皇挡来抬定,)}

\textbf{邹应龙 有劳了!}

\textbf{邹应龙 【西皮摇板】赶至御街打谗臣(或:奸臣)。}

\textbf{常宝童
【西皮摇板】御容请在华车上,抬上金殿奏君王(或:抬上金殿奏叔王)。}

\textbf{{[}第四场{]}}

\textbf{众 咱们找太师罢。}

\textbf{严嵩 唗!唗!你们都往哪里去了?}

\textbf{众 皇挡挡住,不敢进入。}

\textbf{严嵩 哎呀,难怪你们。起来起来,搭轿!}

\textbf{众 轿子被他们打坏了。}

\textbf{严嵩 带马,带马!}

\textbf{众 马被邹老爷骑了去了。}

\textbf{严嵩
哎呀,这样罢,你们挑选一个得力之人,背了老夫回去,重重有赏。}

\textbf{众 我们商议商议。这么大的大胖子,谁背得动?!}

\textbf{众 哎呀!常宝童赶来了。}

\textbf{严嵩 哎呀,千岁!老臣挨不起了。}

\textbf{邹应龙 参见太师(爷)!}

\textbf{严嵩 你是哪个?(或:你是何人?)}

\textbf{邹应龙 (小官)邹应龙,在此。}

\textbf{严嵩 心腹人,你来了?老夫被他们打坏了哇!}

\textbf{邹应龙 太师(爷)为何这等模样?}

\textbf{严嵩 哎呀,好打好打!}

\textbf{邹应龙 哪个敢打太师?}

\textbf{严嵩 哎呀你哪里知道------}

\textbf{邹应龙 小官怎能知晓?}

\textbf{严嵩 待老夫慢慢地对你言讲呃。(喘介)}

\textbf{邹应龙 太师爷慢慢讲。}

\textbf{严嵩
老夫领了圣旨到开山王府,去捉拿常宝童上殿辩理。我到了王府,正要开读圣旨,那娃娃言道:``老太师不必开读,小王知罪。''}

\textbf{邹应龙 他知何罪?}

\textbf{严嵩 他言道:``愿将邱、马两将献与当今。''}

\textbf{邹应龙 说罢之后?}

\textbf{严嵩 他就请过了圣旨。}

\textbf{邹应龙 请过之后?}

\textbf{严嵩 那娃娃他赐了老夫一个座位。}

\textbf{邹应龙
着哇,金銮殿上(或:金殿之上)二十四把金交椅都有(或:尚有)太师爷的座位,何况他小小的开山王府啊。坐下后呢?}

\textbf{严嵩 呵!这一坐啊,可就坐出祸来了呃!}

\textbf{邹应龙 什么祸事?(或:怎见得?)}

\textbf{严嵩 那娃娃言道:``老太师,你在我朝是个忠臣,还是个奸臣呢?''}

\textbf{邹应龙 老太师乃是大大的忠臣。}

\textbf{严嵩 着哇。原是忠臣。}

\textbf{严嵩
那娃娃言道:``既是忠臣,孩子们,珠帘卷起,太师爷抬头观看。''}

\textbf{邹应龙 看些什么?}

\textbf{严嵩
那娃娃也不知听了哪个高明的坏种------(或:领了哪位先生------)}

\textbf{邹应龙 诶。}

\textbf{严嵩
的高教,将老皇御容、伴驾王真像,悬挂中堂,老夫身为大臣,见君不参,就有一行大罪。}

\textbf{邹应龙 太师这便怎处啊?}

\textbf{严嵩 我说,老臣有辩。}

\textbf{邹应龙 哦,有辩?}

\textbf{严嵩 那娃娃言道:``太师爷还会变呐?''}

\textbf{邹应龙 会辩呐。}

\textbf{严嵩 那娃娃言道:``孩子们,金盆打水!''}

\textbf{邹应龙 打水则甚呐?(或:打水何用呐?)}

\textbf{严嵩 他说:``老太师,你变个乌龟,与小王玩耍玩耍?''}

\textbf{邹应龙 老太师你变了没有?(或:太师你如何变法?)}

\textbf{严嵩 呃!老夫焉能如此,乃是舌辩之辩。}

\textbf{邹应龙 我也是问的舌辩之辩呐。}

\textbf{严嵩 着啊。}

\textbf{邹应龙 太师怎样辩法?}

\textbf{严嵩 是我言道:``亦非朔望,闲不参君。''}

\textbf{邹应龙
辩得好,辩得好------辩倒之后?(或:老太师高才!或:坐罢之后?)}

\textbf{严嵩 那娃娃言道:``老太师,我开山王府欠粮?''}

\textbf{邹应龙 不欠粮。}

\textbf{严嵩 ``缺饷?''}

\textbf{邹应龙 不缺饷。}

\textbf{严嵩 ``一不欠粮,二不缺饷,到这儿来,干嘛来了?''}

\textbf{邹应龙 请千岁上殿辩理呀------窝藏邱、马二将。}

\textbf{严嵩 ``拿来。''}

\textbf{邹应龙 (拿)什么?}

\textbf{严嵩 ``圣旨啊。''}

\textbf{邹应龙 呃,(方才)他请过去了哇。}

\textbf{严嵩
哎哟,老夫还不知道他请过去了吗?}\protect\hyperlink{fn594}{\textsuperscript{594}}\textbf{这个娃娃他与我来了个不认账啊。}

\textbf{邹应龙 那还了得?!}

\textbf{严嵩
那娃娃言道:``唗!胆大严嵩,今日在朝害文,明日在朝害武,害来害去,害到小王的头上。今天不打你几下,惯了你的下次,来啊,吩咐脱袍打严嵩!''就是这样乒乒乓乓,一顿暴打。哎哟,可打坏了,打坏了!闪开,闪开!(或:搀扶了。)}

\textbf{邹应龙 哪里去?(或:太师你欲何往?或:慢来慢来,太师往哪里去?)}

\textbf{严嵩 上殿参他一本呐。}

\textbf{邹应龙 参他一本?}

\textbf{严嵩 参他一本。}

\textbf{邹应龙 (万岁若问)有何伤痕?}

\textbf{严嵩 浑身是伤。}

\textbf{邹应龙 怎样验伤?(或:怎样见君?)}

\textbf{严嵩 脱袍验伤。}

\textbf{邹应龙
脱袍验伤------嗯------险呐!(或:哎呀太师,险呐!若不是小官在此,你把事可又办错了。)}

\textbf{严嵩 怎么?}

\textbf{邹应龙
太师爷身为大臣,脱袍见君岂不有欺君之罪!(或:你身为大臣,脱袍见君论律当斩。)}

\textbf{严嵩
哎呀是啊,``要交部严加议处''。哎呀呀心腹人,这便如何是好啊!(或:这便怎是呢?)}

\textbf{邹应龙
这\ldots{}\ldots{}依小官拙见,在这文武两班,寻一心粗胆壮之人,在脸面之上做一伤痕,上殿奏本,参倒那常宝童。(或:这\ldots{}\ldots{}依小官拙见,必须在脸面之上做一、两处伤痕,上殿奏本,一本就准。或:必须要打一面伤,上殿奏本。)}

\textbf{(邹应龙踢起地面一块金砖)}\protect\hyperlink{fn595}{\textsuperscript{595}}

\textbf{邹应龙 这有一块金砖,太师拿在手中,自己打自己,一定成功。}

\textbf{严嵩 心腹人,漫说是你与老夫凑趣,这块金砖也与老夫凑趣来了哇。}

\textbf{邹应龙 太师爷打呀。}

\textbf{严嵩 哎,自己打自己,要直着打,狠着打!}

\textbf{严嵩 常宝童,小奴才!老夫打一面伤,上殿奏本,一本就准!}

\textbf{严嵩
嘿,一本就准。唉哟哟\ldots{}\ldots{},走走走\ldots{}\ldots{}}

\textbf{(严嵩将砖头扔地上)}

\textbf{邹应龙 哪里去?}

\textbf{严嵩 上殿走本呐。}

\textbf{邹应龙 有何伤痕呐?}

\textbf{严嵩 脚面上面了。}

\textbf{邹应龙 那怎样见君呢?}

\textbf{严嵩 脱靴子。哎呀哎呀,这不中用了啊。}

\textbf{严嵩 心腹人,自己打自己,下不了手啊。}

\textbf{邹应龙 不如去请人打。(或:就该请人来打。)}

\textbf{严嵩 好,哪里去问?(或:哦,叫老夫去请何人来打。)}

\textbf{邹应龙 文班中去问。(或:到文班去请。)}

\textbf{严嵩 好好好,文班中去问。}

\textbf{(严嵩朝下场门)}

\textbf{严嵩
列位大人,这有金砖一块,哪一个在老夫脸上做一面伤,上殿参倒常宝童,老夫重礼相谢!}

\textbf{众 我们不敢。}

\textbf{严嵩 哎呀,他们都走了!}

\textbf{邹应龙 (佯惊介)武班中去请。}

\textbf{严嵩 呃呃,武班中去问。(或:武班去请。)}

\textbf{邹应龙 他们有胆量。}

\textbf{严嵩 他们有胆量。}

\textbf{邹应龙 有力气。}

\textbf{严嵩 有力气。)}

\textbf{邹应龙 看得清,打得准呃。}

\textbf{严嵩 看得清,打得准。}

\textbf{(严嵩朝上场门)}

\textbf{严嵩
列位大人,请了。这有金砖一块,哪一位将军在老夫脸上做一面伤,上殿参奏常宝童,老夫重礼相谢!}

\textbf{众 哎呀,我们不敢,不敢呐。}

\textbf{严嵩 哎呀\ldots{}\ldots{}他们都散了(或:溜了)。}

\textbf{邹应龙 唉!教我好恨呐!}

\textbf{严嵩 难道恨着老夫不成?}

\textbf{邹应龙
小官焉敢恨着老太师,我恨只恨这两班文武,有哪个不是老太师的保举,今日用着他们,一个个袖手旁观,怎不令人好恨呐!(或:唉,常宝童啊常宝童,幸喜你是打了老太师,若是打了我邹应龙,我一定打一面伤,上殿奏本,嗯,一本就准!或:不是哟,想这文武官员,哪个不是太师的提拔,今日太师有事,一个个袖手旁观,怎不令人好恨呐!)}

\textbf{严嵩 嘿嘿!原来打老夫的人在这里(或:眼前)呢!}

\textbf{严嵩 心腹人,请上受老夫一礼。}

\textbf{邹应龙 太师,这是何意?(或:哎呀呀,折煞小官!或:太师此礼为何?)}

\textbf{严嵩
心腹人,你看满朝文武他溜的溜了,跑的跑了,只有你这心腹人在此,有劳你的贵手,与老夫打一面伤,上殿参奏常宝童。老夫是重礼相谢。}

\textbf{邹应龙
哎呀呀,(启禀太师:)小官多蒙老太师升官之恩尚未报达,焉敢下此毒手,诶,使不得,诶,使不得。(或:小官不敢,小官不敢。或:小官多蒙升官之恩尚未报得,怎么还敢打老太师,不敢呐,不敢呐。)}

\textbf{严嵩
诶,只要在老夫面上做一伤痕,上殿参倒常宝童,比报那升官之恩胜强十倍,犹如报恩一样。}

\textbf{邹应龙
报恩一样?敢是教小官报恩么?(或:怎么。犹如报了升官之恩一般?)。}

\textbf{严嵩 呃,是教你报恩,教你报恩。}

\textbf{邹应龙 如此小官报恩了。(或:如此小官(当得)效劳。)}

\textbf{严嵩 来来来,报恩呐。(或:你要用力打)}

\textbf{邹应龙 小官(这里)报恩了。}

\textbf{邹应龙 【西皮导板】手指(或:指着)严嵩贼奸佞,}

\textbf{严嵩
唗,唗!老夫教你打,胆大邹应龙,你怎么骂起老夫来了?!嘿,真真的岂有此理呀!}

\textbf{邹应龙 唉呀呀,老太师,你把小官可错怪了哇。}

\textbf{严嵩 怎么错怪了你呀?}

\textbf{邹应龙
常言说得好:举手难打笑脸人,小官与老太师夙无怨恨(或:小官与老太师远日无仇,近日无冤),况且升官之恩尚未报答,焉能下得手打太师?没奈何我指东骂西,指桑骂槐。指的是老太师,骂的是常宝童,骂起气来才好用力打呀。小官将将骂了一句,老太师就动起怒来。若是打着太师,那还了得?老太师,呃,小官得罪了,得罪了老太师,小官这厢赔礼了,小官这里请罪了。太师爷你另请高明罢。(或:常言说得好:举拳不打笑脸人,想小官与太师爷远日无仇,近日无冤,况且又有升官之恩尚未报得,焉能下得手来打太师?没奈何只得是指东骂西,指桑骂槐。指的是老太师,骂的是常宝童,骂起气来才好使力气来打。小官刚刚的骂了一句,太师爷就降下罪来。小官就不敢,不敢。你另请高明罢。)}

\textbf{严嵩
回来回来,心腹人,老夫我明白了。你是指东骂西、指桑骂槐。指的是老夫,骂的是常宝童。}

\textbf{邹应龙 是啊。(或:如此说来,太师爷你不恼?)}

\textbf{(邹应龙 你要我骂?)}

\textbf{(邹应龙 骂你这个奸贼?)}

\textbf{严嵩
骂上气来才好下手打,如此说来,心腹人!你就连打带骂}\protect\hyperlink{fn596}{\textsuperscript{596}}\textbf{。}

\textbf{邹应龙 如此严嵩!}

\textbf{严嵩 嗳!}

\textbf{邹应龙 我把你这(误国的)老贼!}

\textbf{严嵩 骂得好啊!}

\textbf{邹应龙
【西皮快板】骂一声老贼听分明:你在当朝官一品,上欺天子下压臣。满朝文武来观定(或:齐来看),我与忠良把冤申。}

\textbf{严嵩 哎呀!邹应龙打坏了!}

\textbf{邹应龙 呃------常宝童打的。}

\textbf{严嵩 呃是是是,常宝童打坏人了。走走走,上殿参本(或:参本)。}

\textbf{邹应龙 慢来慢来,看看伤痕呐。(或:待我来看看伤。)}

\textbf{严嵩 对对对对,验验伤,验验伤。}

\textbf{邹应龙 哎呀太师爷,还是不中用啊。(或:还是不成功。)}

\textbf{严嵩 怎见得?(或:怎么不成功?)}

\textbf{邹应龙 只有一伤呐。}

\textbf{严嵩 一块可也就够了。}

\textbf{邹应龙 诶,一伤乃是误伤,要两伤才算是打伤呢!还要做上一块。}

\textbf{严嵩 哎呀,老夫我挨受不起了哇。}

\textbf{邹应龙 呃,倒有个方法(或:太师,小官倒有个办法)。}

\textbf{严嵩 什么方法,快快说来。}

\textbf{邹应龙 来来来------这有金砖一块,(或:这有砖头一块)}

\textbf{严嵩 哦,一块砖头呃。}

\textbf{邹应龙 (太师)拿在手中。}

\textbf{严嵩 呃,拿在手中,}

\textbf{邹应龙 将袍角(或:袍襟)衔在口内,}

\textbf{严嵩 衔在口内。}

\textbf{邹应龙 自己打自己,}

\textbf{严嵩 自己打自己。}

\textbf{邹应龙 打一下,哼一声,(或:小官用力打,太师用力哼。)}

\textbf{严嵩 打一下,哼一声。}

\textbf{邹应龙 这还有个名堂。}

\textbf{严嵩 嘿,还有个名堂?}

\textbf{邹应龙 这叫作恨病吃药。(或:恨病------;欠债------)}

\textbf{严嵩 欠债还钱。(或:吃药;换钱。)}

\textbf{邹应龙 着啊,试上一试。}

\textbf{严嵩
诶,砖头拿在手中,袍角衔在口内,自己打自己,打一下,哼一声,恨病吃药。}

\textbf{严嵩 砖头啊砖头,你不是砖头------}

\textbf{邹应龙 是什么?}

\textbf{严嵩 你是老夫的对头。}

\textbf{邹应龙 呃,打呀!}

\textbf{严嵩 往哪里打?}

\textbf{邹应龙 面上打。}

\textbf{严嵩 面上啊------噢,哎呦嚯,打着了打着了 !}

\textbf{邹应龙 在哪里?(或:怎么样了?)}

\textbf{严嵩 诶,在脚面之上。}

\textbf{邹应龙 唉呀,还是不中用呃。}

\textbf{严嵩 嘿,白挨了一下打呀!}

\textbf{严嵩 得了,我啊,脱靴子见君罢。}

\textbf{邹应龙 那还了得。}

\textbf{严嵩 真真打不下手啊,唉!一客不烦二主啊,还是心腹人代劳罢。}

\textbf{邹应龙 还要我报恩?(或:怎么,还要心腹人代劳?)}

\textbf{严嵩 嗯,一定我要你报恩呐。}

\textbf{邹应龙 (如此说来,)太师爷,你要忍呐!}

\textbf{严嵩 心腹人,你要狠呐!}

\textbf{邹应龙 打青了脸------(或:打伤了脸------)}

\textbf{严嵩 好奏本。}

\textbf{邹应龙 严嵩!}

\textbf{严嵩 嗳!}

\textbf{邹应龙 (我把你这卖国的)奸贼!(或:老贼------)}

\textbf{严嵩 骂得好啊!(或:嗯------)}

\textbf{邹应龙
【西皮快板】听罢言来喜气生,不由得应龙抖精神。开言大骂贼奸佞(或:开言便把老贼问),苦害忠良为何情。欺君误国心太狠,应龙心中恨难平。(或:欺君误国实可恨,管教老贼命归阴。或:欺君误国心太狠,管教老贼两眼青。)}

\textbf{严嵩 哎哟,哎哟!)}

\textbf{严嵩 邹应龙,你打坏了!}

\textbf{邹应龙 常宝童打伤人了。(或:呃,常宝童打的呢。)}

\textbf{严嵩 常宝童打伤人了。}

\textbf{严嵩 邹应龙,你来看看伤痕。}

\textbf{邹应龙 还要修补修补。(或:是浮伤啊,还轻啊。)
(或:待小官来看看------打得倒还好啊。当中还有一条缝儿。两伤中间,有一条缝------还要找补找补!)}

\textbf{严嵩 诶,眼睛都看不见了。}

\textbf{严嵩 唉呀!将就了罢,将就了罢!搀扶了------}

\textbf{邹应龙 遵命!}

\textbf{严嵩 唉,慢来慢来,心腹人,老夫我倒想起一桩心事来了。}

\textbf{邹应龙 什么大事!(或:什么心事?)}

\textbf{严嵩
老夫在开山王府挨打的时节,屏风后面有那么一位穿红袍的官儿闪将出来,呃,他还踢了老夫一靴尖,也不知他是何人,弄什么诡计!(或:常宝童这个娃娃,不知领了何人的高见,将老夫弄得这副模样。)}

\textbf{邹应龙
这有何难?(老)太师在朝内(或:京内)访,小官朝外(或:京外)访,访着此人,便知分晓(或:定不与他干休)。}

\textbf{严嵩 嗯,我定要灭他的九族!}

\textbf{邹应龙 要灭贼的满门!}

\textbf{严嵩 搀扶了!}

\textbf{(邹应龙 来了!)}

\textbf{邹应龙 哎呀,常宝童来了!}

\textbf{严嵩 嘿呀!}

\textbf{严嵩 心腹人!劳您驾。}

\textbf{邹应龙 不成敬意!}

\newpage
\hypertarget{ux5fa1ux7891ux4ead-ux4e4b-ux738bux6709ux9053}{%
\subsection{御碑亭 之
王有道}\label{ux5fa1ux7891ux4ead-ux4e4b-ux738bux6709ux9053}}

\textbf{{[}第一场{]}}

\textbf{{[}引子{]}磨穿铁砚,这襟怀,不让前贤。}

\textbf{(念)读尽诗书身世寒,满腹文章不为官。月宫}\protect\hyperlink{fn597}{\textsuperscript{597}}\textbf{丹桂相攀易,金殿鳌头独占难。}

\textbf{卑人王有道,浙东人氏,寄居京华。不幸父母早逝,娶妻孟氏月华,十分贤德。胞妹淑英,年方二九,尚未许字。想我苦读寒窗,功名未能上达。今当大比之年,会试之期,我不免将孟氏、妹子唤出堂前,嘱咐门户之事,也好入场。}

\textbf{啊,娘子、贤妹哪里?}

\textbf{啊娘子,}

\textbf{贤妹,}

\textbf{一同坐下。}

\textbf{非为别事,今当大比之年,会试之期,唤你们出来,好好看守门户,我好入场(或:我要入场)会试。}

\textbf{但愿如此。}

\textbf{足见盛情,我当立饮三杯。}

\textbf{(孟月华 待奴把盏。)}

\textbf{【西皮原板】承谢你贤德的心喜之不尽,但愿得此一去鱼跳龙门。}

\textbf{【西皮原板】贤德妹礼爱我手足情份,猛想起父母恩无限伤情。}

\textbf{【西皮原板】但愿得这一科功名有份,终不愧王有道苦读诗文。施一礼辞贤妹又别闺阃,}

\textbf{【西皮摇板】赴科场好一似平步登云}\protect\hyperlink{fn598}{\textsuperscript{598}}\textbf{。(或:沐洪恩蟾宫考会会文衡}\protect\hyperlink{fn599}{\textsuperscript{599}}\textbf{。)}

\textbf{{[}第二场{]}}

\textbf{呵呵呵哈哈哈\ldots{}\ldots{}(笑介)}

\textbf{【西皮摇板】考罢了第三场文章高兴,笑盈盈喜孜孜出了龙门。回家去细说与阖家欢庆,转大街(或:穿大街)过小巷到了家门。}

\textbf{开门来。}

\textbf{是我回来了。}

\textbf{【西皮摇板】你嫂嫂因何故她不来开门。}

\textbf{你嫂嫂往哪里去了?}

\textbf{哦,她病了,得何病症呐?}

\textbf{既然中途路遇(或:中途遇见)大雨,就该寻个所在,躲避躲避,又何必冒雨而归,呃,成什么样儿啊。}

\textbf{在何处避雨?}

\textbf{哦,御碑亭。}

\textbf{不错,有一个御碑亭。}

\textbf{后来呢?}

\textbf{啊?她,她就该走了出来呀。}

\textbf{后来便怎么样啊?}

\textbf{暧昧不明,有何为证?}

\textbf{怎么还有诗词么?}

\textbf{你可记得?}

\textbf{念来我听------}

\textbf{哪一句?}

\textbf{呀呸!}

\textbf{【西皮散板】听一言不由我火上两鬓,诗词说宿碑亭隐有内情呐。我待要(或:我倒要)到里面将她查问,}

\textbf{唉。}

\textbf{【西皮散板】这桩事闹起来呀脸}面何存\textbf{呐}。

\textbf{罢了哇罢了。我想此事,闹将起来,不成体面,隐忍不言。王有道啊王有道,岂不成了(此道)\ldots{}\ldots{}}

\textbf{也罢,我不免暗写休书一封,就说她爹娘身染重病,将她送回娘家,免得两下出丑。}

\textbf{苍头快来,雇乘车辆送你大娘到孟家庄去。}

\textbf{快去。}

\textbf{啊贤妹,想女儿家谨守闺训,当讲则讲,不当讲不可胡言乱语。方才是你多口,才闹出这样的事啊!从今以后,要谨守莫言,那才是我的好妹子啊。}

\textbf{好,将笔砚端正好了。}

\textbf{唉!羞愧人也!}

\textbf{【西皮导板】王有道提笔泪难忍,}

\textbf{【西皮原板】实难舍夫妻结发情。实指望同生共到老,又谁知半途风波生。非是我一旦多薄幸,实难容留下贱的人呐。只得闭口【转西皮快板】牙咬定,字字行行写分明:那一日避雨在御碑亭,其中暧昧事不明。男女授受理不应,七出之条事有因。从今任你嫁别姓,割断了丝萝两离分。写罢休书打手印,}

\textbf{(苍头 车辆到。)}

\textbf{晓得。}

\textbf{外厢伺候。}

\textbf{【西皮摇板】密密封好待她行。}

\textbf{【西皮摇板】贤妹将你嫂嫂请,说孟家差人到来临。}

\textbf{回来了。}

\textbf{文章么,倒还得意呀。}

\textbf{(呃,) (急介)只是有一事替你着急呀。}

\textbf{我方才出场的时节,遇见你家小厮德禄,慌慌张张言道:员外安人,因你不辞而归,二老吵闹一场,双双病倒在床,十分地沉重,故而(或:故此)急急赶来接你回去。}

\textbf{呃,德禄么,我恐其家中无人,先打发他回去。我已吩咐苍头,雇来车辆。你就该速速前往,安慰他二老一回,那才是你的孝道啊。}

\textbf{【西皮摇板】车在门首趁早行。}

\textbf{转来。}

\textbf{【西皮摇板】朋友托带一封信,带回交与你严亲。}

\textbf{【西皮摇板】从先恩爱一时尽,要想相逢恐不能呐。}

\textbf{【西皮摇板】万般已是皆有哇定,}

\textbf{(王淑英 【西皮摇板】你又何必假泪淋。)}

\textbf{【西皮摇板】这是我家门遭不幸,毒意休妻心不宁。思想恩爱}泪难忍,孤孤凄凄\protect\hyperlink{fn600}{\textsuperscript{600}}愁煞人\textbf{呐}。

\textbf{(报子 啊哈------)}

\textbf{(报子 报报报,喜来到。)}

\textbf{(报子 报禄的来喽。)}

\textbf{你们是做什么的?}

\textbf{哦,报禄的。报的是哪一家呀?}

\textbf{啊?王有道他中了么?}

\textbf{就是你老爷呀。}

\textbf{进来,进来。}

\textbf{可有报单?}

\textbf{呈上来。}

\textbf{``因报贵府第王老爷、印有道,取中甲辰科第六名进士。''}

\textbf{呵呵哈哈哈哈\ldots{}\ldots{}(笑介)}

\textbf{贴在门首。}

\textbf{我好侥幸呐\ldots{}\ldots{}}

\textbf{辛苦你们了,}

\textbf{难为你们了。}

\textbf{喜钱今日不便呐,改日多多重赏。}

\textbf{我不免去到里面(或:去至后面),说与妹子知道,再打点谒师便了。}

\textbf{【西皮摇板】十载寒窗今方信,皇天不负我读书的人呐。}

\textbf{呵呵哈哈哈\ldots{}\ldots{}(笑介)}

\textbf{{[}第三场{]}}

\textbf{(念)昨日寒儒谁问姓,今朝显贵便知名。}

\textbf{老师在上,门生等大礼参拜。}

\textbf{(念)桃李公门姓氏香,荐拔之恩日月长。}

\textbf{老师在上,门生等侍立听教。}

\textbf{谢座。}

\textbf{请}

\textbf{皆赖老师提拔。}

\textbf{年兄但讲何妨?(或:年兄何妨直言?)}

\textbf{啊柳\ldots{}\ldots{}}

\textbf{啊,柳年兄,你在御碑亭避过雨来?那亭内,可有个妇人先在其内呀?}

\textbf{你可晓得那妇人的姓氏?}

\textbf{哦,哦,你并未交谈。}

\textbf{你实未交谈?!}

\textbf{哎呀,老师啊,}

\textbf{唉呀,年兄啊,那亭内的妇人不是外人呐,}

\textbf{乃是门生的,唉,拙荆哦。}

\textbf{不敢呐,}

\textbf{不敢呐。}

\textbf{哎呀,老师啊(或:年兄啊),}

\textbf{哎呀,年兄啊(或:老师啊),}

\textbf{也是门生一时不明,``莫须有''三字,竟将她休弃了。}

\textbf{老\ldots{}\ldots{}}

\textbf{老师啊------}

\textbf{【西皮摇板】门生一时做事蠢,疑她暧昧事不明。年兄说明亭中景,不该疑心退婚姻。}

\textbf{门生等告退。(或:门生等遵命。)}

\textbf{正是:(念)但愿文章依宇下。}

\textbf{门生等告退。}

\textbf{{[}第四场{]}}

\textbf{【西皮摇板】赴罢了琼林宴过庄陪罪,见岳父和岳母劝妻回归。此一番必须要准备跌跪,到此时才知呀自惹是非。}

\textbf{(念)读书侥幸已成名,到底糊涂心不精。琴瑟失调乖礼仪,方知宋}弘\protect\hyperlink{fn601}{\textsuperscript{601}}\textbf{是高人。}

\textbf{呃,哦,德禄你来了。}

\textbf{哦,员外、安人(也)来了。}

\textbf{哦,你家姑娘也来了。快快有请呐!}

\textbf{哼,休得胡言。(快快有请。)}

\textbf{啊,岳父。}

\textbf{岳母,呃,呃\ldots{}\ldots{}}

\textbf{啊,娘\ldots{}\ldots{}}

\textbf{呃,糟糕!}

\textbf{(啊,娘\ldots{}\ldots{})}

\textbf{德禄,你在此则甚?}

\textbf{有我就用不着你了。}

\textbf{出去,}

\textbf{教你出去。}

\textbf{近前来------}

\textbf{你出去吧。}

\textbf{啊娘子,千不是,万不是,俱是卑人的不是。喏喏喏,我这厢赔礼了。}

\textbf{(啊娘子,)我这厢又赔礼了。}

\textbf{啊娘子,俱是卑人的不是,我这厢跪下了。(或:啊娘子,卑人这里跪下了。)}

\textbf{【西皮快板】男儿志气三千丈,污秽之言岂能当。也是卑人太孟浪,一时性急我未推详。}

\textbf{【西皮快板】世间万事有原谅,何况丈夫与妻房。从先的事儿莫追想,还念昔日情义长。}

\textbf{呵呵哈哈哈\ldots{}\ldots{}(笑介)}

\textbf{【西皮摇板】你是我的贤妻房。}

\textbf{{[}第五场{]}}

\textbf{有请。}

\textbf{有劳众位年兄远路而来(或:有劳众位年兄前来),弟当面谢过。}

\textbf{啊众位年兄,弟有一言,不好启齿。}

\textbf{闻得柳年兄尚乏中馈}\protect\hyperlink{fn602}{\textsuperscript{602}}\textbf{,弟有一妹,名唤淑英,年方二九。呃,倒还伶俐。欲与柳年兄永结丝萝,幸勿见却。}

\textbf{(年兄休得见却。)}

\textbf{今日即是良辰黄道,(就在舍下结拜花烛,)就请二位年兄赞礼上来。}

\textbf{娘子,搀扶小妹(或:搀扶贤妹)。}

\textbf{且慢,后面备得酒饭,就烦二位年兄,陪一陪我们的新姑老爷呀。}

\textbf{呵呵哈哈哈\ldots{}\ldots{}(笑介)}

\textbf{请------}

\newpage
\hypertarget{ux4e09ux5a18ux6559ux5b50-ux4e4b-ux859bux4fdd}{%
\subsection{(三娘)教子 之
薛保}\label{ux4e09ux5a18ux6559ux5b50-ux4e4b-ux859bux4fdd}}

\textbf{(念)厨下黄粱熟,开门望东人。}

\textbf{东人回来了。}

\textbf{来来来,随老奴前去用饭呐。}

\textbf{现在机房织绢。}

\textbf{你就要来呀。}

\textbf{呵呵哈哈\ldots{}\ldots{}(笑介)}

\textbf{【二黄原板】小东人下学归言必有错,如不然母子吵闹为何。}

\textbf{【二黄原板】见三娘在机房珠泪双落,回头来问一声东人倚哥。}

\textbf{东人!}

\textbf{【二黄原板】你的母教训你非为过错(或:非为之过),为什么好言当作恶说。东人呐!}

\textbf{【二黄原板】这才是养不教,父之过,教不严来师之惰。}

\textbf{【二黄原板】老薛保进机房双膝跪呃落,双膝跪落,三娘啊,问三娘发雷霆却是为何?}

\textbf{【二黄原板】劝三娘休得要珠泪垂掉,老奴言来听根苗:千看万看,看东人的年纪小,望三娘,念东人,去世早,只生有这根苗,还须要轻打轻责,饶他这遭(或:必须要轻打轻责,饶恕这遭),下次里不饶。}

\textbf{【二黄散板】见三娘她把那机头割断,吓得我老薛保胆战心寒。走向前来良言奉劝,尊一声贤主母细听奴言:遭不幸老东人外乡命染,可怜我千山万水搬尸回还。}

\textbf{老奴好恨!}

\textbf{【二黄散板】恨只恨大娘、二娘把心肠改变,一个个反穿裙另嫁夫男。}

\textbf{老奴好喜!}

\textbf{【二黄散板】喜的是贤主母发下誓愿,从今后抚养东人呐后代流传。}

\textbf{老奴我明、明\ldots{}\ldots{}明白了!}

\textbf{【二黄散板】莫不是你也把心肠改变,你也要反穿裙另嫁夫男?}

\textbf{你要走,只管去走。}

\textbf{要嫁,只管去嫁呀,呃\ldots{}\ldots{}(哭介)}

\textbf{【二黄散板】撇下我(或:撇下了)老的老,小的小长街讨要(或:挨门乞讨),我也要抚养东人呐。啊,不明白的三娘啊!}

\textbf{不该责打。(或:就该责打。)}

\textbf{【二黄散板】怪不得三娘发怨恨,回头埋怨小东人。}

\textbf{东人,这就是你的不是了。}

\textbf{今朝}\protect\hyperlink{fn603}{\textsuperscript{603}}\textbf{下学回来,言语冒犯母亲}就该前去领责。

就是挨打呀。

挨打焉能不痛?

唉呀东人呐,去与不去,全在于你。你将老奴推倒尘埃,倘有不测,我看你是怎生得了啊!呃\textbf{\ldots{}\ldots{}}(\textbf{哭介})

\textbf{这便才是。}

待我来教导于你:来来来,这有家法在此,顶在头上,跪在你母亲的面前,说道:``孩儿今朝下学回来,言语冒犯母亲。现有家法在此,望母亲高高举起,轻轻落下;打一下,如十下;打十下,如百下;打在儿身,痛在娘心。母亲有爱子之意,呃,教训儿罢!''

呃,你母亲有爱子之意,就不打你了。

啊东人,莫道老奴讲的。

那个自然。

三娘,老奴也跪下了哇,呃\textbf{\ldots{}\ldots{}}(哭介)

\textbf{【二黄散板】老奴上前忙遮拦。你要打,将老奴责打几下,你,你\ldots{}\ldots{}你打我的东人呐老奴心酸。}

\textbf{【二黄散板】求取上进(或:要做高官)有何难。}

\textbf{三娘!}

\textbf{【二黄散板】教子的名儿万古传。}

\textbf{三娘!}

\textbf{东人。}

\textbf{哦,来了!}

\textbf{呵呵哈哈哈\ldots{}\ldots{}(笑介)}

\newpage
\hypertarget{ux5927ux4fddux56fd-ux4e4b-ux6768ux6ce2}{%
\subsection{大保国 之
杨波}\label{ux5927ux4fddux56fd-ux4e4b-ux6768ux6ce2}}

\textbf{(内)阻------诏}\protect\hyperlink{fn604}{\textsuperscript{604}}!

\textbf{(内)}兵部杨波。

\textbf{(内)}不但不押,还要上殿面奏。

\textbf{(内)}千岁慢走!

\textbf{(内)}兵部杨波。

\textbf{(内)}来也!

【二黄摇板】太师朝房用奸谋,传旨国太让龙楼。本当上殿把本奏,

唉!

【二黄摇板】怎奈我官职小啊不敢出头。

参见千岁。

国太要将江山让与太师。千岁可曾画押?

学生敢莫是吃了熊心豹胆,焉敢与那贼同谋?

官卑职小,难以出头。

只要千岁作主,拚着一腔热血洒落金阶,我也要落一个青史名标。

遵命!

【二黄原板】为的是大明朝锦绣家邦。

【二黄原板】殿角下坐定了谋朝篡位奸贼李良。

【二黄原板】老王爷赐铜锤上打昏君、下打谗臣、压定了满朝的文武、哪一个不尊定国王开国的忠良。

呵哈哈哈\ldots{}\ldots{}(笑介)

【二黄原板】臣愿国太福寿绵长。

【二黄原板】三跪九叩谢皇娘。

【二黄原板】大明江山共作商量。

臣等不押。

(徐延昭
【二黄原板】唐僖宗坐江山天心不顺,他驾前有一臣梁王朱温。臣弑君子弑父弟霸兄嫂,【垛板】君不君、臣不臣、父不父、他子不子、就败坏人伦。\ldots{}\ldots{})

(徐延昭
【二黄原板】\ldots{}\ldots{}北海内现铜桥渡过旗人。到如今才能得干戈宁静,为什么将社稷让与他人。)

千岁。

遵命。

臣兵部侍郎杨波,有一道太平表章,可容臣启奏?

容奏:

(念)忆自元朝居华地,世上多少古今奇。山崩地裂江河啸,风起云会星斗移。

容奏:

【二黄原板\protect\hyperlink{fn605}{\textsuperscript{605}}】臣不奏前朝中历代帝君,臣启奏机密事出在大明:太祖爷晏了驾龙归海境,大明朝无一人执掌龙庭。满朝中文武臣袖手不问,马皇后扶建文立帝为君。普天下众宗室无有议论,唯有那四太子燕山发兵。姚广孝为军师暗传将令,六月天冻黄河兵困金陵。直逼得马皇后火焚丧命,直逼得那建文帝削发为僧。亲叔侄尚且来争竞,何况那李呃太师是外姓之人。

臣不能全知。

徐千岁开国元勋,必然知晓。

谢国太!

【二黄原板】汉高皇曾起义(或:初起义)路过那硭砀山,偶遇白蟒把路拦。执宝剑将蟒斩为两段,兴人马灭秦楚一统河山。到后来出王莽又出苏献,松棚会害平帝命染黄泉。夺玉玺搜宫院王莽谋篡,把一个王国母逼死井前。前朝的事迹当为鉴,也免得学国母跌足怨天。

田子裕在你府中常来常往,算得是心腹之人。

算得。

【二黄原板】残唐五代乱纲常,各路诸侯霸一方。宋太祖陈桥披黄蟒,十八载马上为帝王。三下河东基业创,归来染病在龙床。御弟进宫将兄望,烛影摇红祸起萧墙。亲手足尚且这等样,

【二黄摇板】何况太师与娘娘。

你乃皇亲国戚,焉能发得高墙?

呃,发不得。

【二黄散板】劝国太江山休要让。

唉呀!

【二黄散板】国太金殿把贼宠,徐、杨保本一场空。回头忙把千岁请,倚老卖老打奸臣。

千岁虎老雄心在,撒得疯,打得动奸贼。

打得动。

功劳簿在此!

功劳簿并无奸贼的名字。

呃!

且慢,自从盘古以来,哪有臣打君的道理?

上殿请罪。

臣启国太:徐、杨有欺君之罪,国太降旨。

谢国太!

国太传下旨意:从今以后,朝中有事无事,不与徐、杨二大奸党相干。

敢夺社稷!

\newpage
\hypertarget{ux4e8cux8fdbux5bab-ux4e4b-ux6768ux6ce2}{%
\subsection{二进宫 之
杨波}\label{ux4e8cux8fdbux5bab-ux4e4b-ux6768ux6ce2}}

\textbf{千岁请!}

\textbf{【}二黄摇板\textbf{】宫门上锁贼李良。}

\textbf{【}二黄摇板\textbf{】四郎我儿击宫呃墙。}

\textbf{我儿休要动手,此乃徐家小姐,上前见过。}

\textbf{领旨。}

\textbf{千岁。}

\textbf{全仗千岁。}

\textbf{千岁------}

\textbf{【}二黄慢板\textbf{】千岁爷进寒宫休要慌忙,站宫门听学生细说比方:昔日里楚汉两争强,鸿门设宴要害汉王。张子房背宝剑把韩信来访,九里山前摆下战场。只逼得楚项羽乌江命丧,到后来封韩信三齐王。他朝中有一个萧何丞相,后宫院有一位吕后皇娘。君臣们设下了天罗地网,三宣韩信斩首在未央。九月十三严霜降,盖世忠良不得久长。千岁爷进寒宫学生不往,}

\textbf{【}二黄慢板\textbf{】怕的是辜负了十载寒窗、九州遨游、八月科场、七篇的文章}\protect\hyperlink{fn606}{\textsuperscript{606}}\textbf{,才落得个兵部侍郎,无有下场。}

\textbf{【}二黄原板\textbf{】我好比鱼闯过千层罗网,受了些惊恐着些慌忙。}

\textbf{【}二黄原板\textbf{】千岁爷保学生满门无恙,拚着一死闯进昭阳。}

\textbf{【}二黄原板\textbf{】后面跟随兵部杨侍郎。}

\textbf{【}二黄原板\textbf{】观则见龙国太怀抱太子两泪汪汪,口口声声哭的是先皇。}

\textbf{【}二黄原板\textbf{】摆一摆手儿切莫要承当。}

\textbf{【}二黄原板\textbf{】学一个文站东,}

\textbf{【}二黄原板\textbf{】各自分班站立在两厢。}

\textbf{【}二黄原板\textbf{】为什么恨天怨地、颊带惆怅、所为哪桩。}

\textbf{【}二黄原板\textbf{】有什么大祸从天降,}

\textbf{(徐延昭 【}二黄原板\textbf{】你就该请太师父女商量。)}

\textbf{【}二黄原板\textbf{】他未必一旦无情起下了谋位的心肠,太师爷忠良呃。}

\textbf{【}二黄原板\textbf{】臣七月十三日三本奏上,龙国太偏偏要让啊。}

\textbf{(徐延昭 【}二黄原板\textbf{】你言道大明朝\ldots{}\ldots{})}

\textbf{(徐延昭
【}二黄原板\textbf{】龙国太慢把懿旨降,老臣言来听端详:臣难学赵廉颇列国老将,臣难学汉马援大战昆阳。臣难学尉迟恭八寨来抢,臣难学老吴祯保驾百凉。臣年迈难把疆场上,臣年迈难挽马丝缰。臣年迈听不见金鼓声响,臣年迈眼昏花观不见阵头枪。老臣我年迈如霜降,要保国还有那兵部侍郎。)}

\textbf{【}二黄原板\textbf{】吓得臣低头不敢望,战战兢兢启奏皇娘:臣愿学严子陵垂钓矶上,臣愿学钟子期砍樵山岗。臣愿学诸葛亮躬耕垄上,臣愿学吕蒙正苦读寒窗。春来桃李齐开放,夏至荷花满呃池塘。到秋来菊桂花开金钱样,冬至腊梅带雪霜。弹一曲高山流水琴音亮,下一局残棋消遣闷愁肠。书几幅法书精神爽,巧笔丹青悬挂草堂。}\protect\hyperlink{fn607}{\textsuperscript{607}}\textbf{臣昨晚修下了辞王表章,今日里带进宫叩别皇娘。望国太开恩将臣放,落一个清闲自在、散淡逍遥、无忧无虑、无是无非,做什么兵部侍郎,臣告职还乡。}

\textbf{(徐延昭 【二黄原板】吓坏了定国王,)}

\textbf{【二黄原板】兵部的侍郎。}

\textbf{(徐延昭 【二黄原板】自从盘古立帝邦,)}

\textbf{【二黄原板】君跪臣来臣怎敢当。}

\textbf{(徐延昭 【二黄原板】锦家邦来锦呢家邦,)}

\textbf{【二黄原板】臣有一本启奏皇娘。}

\textbf{(徐延昭 【二黄原板】昔日里有一个李文、李广,)}

\textbf{【二黄原板】弟兄双双扶保朝廊。}

\textbf{(徐延昭 【二黄原板】李文北门带箭丧,)}

\textbf{【二黄原板】万家山前又收李刚。}

\textbf{(徐延昭 【二黄原板】收了一将损伤一将,)}

\textbf{【二黄原板】一将倒比一将强。}

\textbf{(徐延昭 【二黄原板】到后来保太子登龙位上,)}

\textbf{【二黄原板】反把那李广斩首庆阳。}

\textbf{(徐延昭 【二黄原板】这都是前朝的忠臣良将,)}

\textbf{【二黄原板】哪个忠良又有下场。}

\textbf{(徐延昭 【二黄原板】困龙思想长呃江浪,)}

\textbf{【二黄原板】虎落平阳想奔山岗。}

\textbf{(徐延昭 【二黄原板】事到头来想一想,)}

\textbf{【二黄原板】谁是忠良哪个是奸党。}

\textbf{(徐延昭 【二黄摇板】铜锤一举王请上,)}

\textbf{【二黄摇板】老杨波搀扶起定国王。}

\textbf{【二黄摇板】用手接过龙一条,二目睁睁把臣瞧。趁此机会生计呃巧}\protect\hyperlink{fn608}{\textsuperscript{608}}\textbf{,}

\textbf{哎呀!}

\textbf{【二黄摇板】浑身上下似水浇,难以保朝。}

\textbf{臣!}

\textbf{【二黄摇板】叩罢头来谢罢恩呐,}

\textbf{(徐延昭
【二黄摇板】徐延昭代驾}\protect\hyperlink{fn609}{\textsuperscript{609}}\textbf{且平身。)}

\textbf{【二黄摇板】一文、}

\textbf{(徐延昭 【二黄摇板】一武,)}

\textbf{【二黄摇板】出宫门,}

\textbf{【二黄摇板】仗着太子叫皇兄呃。}

\textbf{【二黄摇板】大明江山全在呃你呀,}

\textbf{(徐延昭 【二黄摇板】保国家全仗你杨家父子兵。)}

\newpage
\hypertarget{ux5357ux5929ux95e8-ux4e4b-ux66f9ux798f}{%
\subsection{南天门 之
曹福}\label{ux5357ux5929ux95e8-ux4e4b-ux66f9ux798f}}

\textbf{{[}第一场{]}}

\textbf{(曹玉莲 (内)【西皮导板】急急忙忙走得慌,)}

\textbf{【西皮散板】点点珠泪洒胸膛啊。(或:逃出主仆人一双。)}

\textbf{【西皮散板】虎口内逃出了两只羊。}

\textbf{啊小姐,且喜逃出虎口,待老奴随同小姐慢慢行走。}

\textbf{是。}

\textbf{(曹玉莲 【西皮原板】\ldots{}\ldots{}魏忠贤\ldots{}\ldots{})}

\textbf{【西皮原板】我朝中出谗臣搅乱家邦。}

\textbf{(曹玉莲 【西皮原板】天启爷坐山河\ldots{}\ldots{})}

\textbf{【西皮原板】太老爷做天官吏部大堂。}

\textbf{(曹玉莲 【西皮原板】\ldots{}\ldots{})}

\textbf{【西皮原板】丢了官罢了职贬回故乡。}

\textbf{(曹玉莲 【西皮原板】\ldots{}\ldots{})}

\textbf{【西皮原板】有司羽}\protect\hyperlink{fn610}{\textsuperscript{610}}\textbf{领人马暗地埋藏。}

\textbf{(曹玉莲 【西皮原板】\ldots{}\ldots{})}

\textbf{\textless{}哭头\textgreater{}太老爷,啊,太老爷!}

\textbf{【西皮原板】最可叹忠良臣无有下场。}

\textbf{(曹玉莲 【西皮原板】我的母花井内也把命丧,)}

\textbf{\textless{}哭头\textgreater{}太夫人呐啊!}

\textbf{(曹玉莲 喂呀,儿的娘啊\ldots{}\ldots{})}

\textbf{【西皮原板】就是那铁石人也要悲伤。}

\textbf{(小姐因何不走哇?)}

\textbf{这\ldots{}\ldots{}}

\textbf{(小姐,)老奴走得慌忙(或:走得忙迫),分文未带,如何是好?}

\textbf{是。}

\textbf{在。}

\textbf{遵命。}

\textbf{这里掌柜的有么?}

\textbf{这有金耳环一对,照市价合来。}

\textbf{掌柜的请看。}

\textbf{金子,自然是黄色的呀。}

\textbf{唉! 人不在时中,金子也变成铜了。}

\textbf{待我那厢去问(或:那厢去换)。}

\textbf{啊,这里掌柜的有么?}

\textbf{(掌柜 何事?)}

\textbf{这有金耳环一对,照市价合来。}

\textbf{掌柜的请看。}

\textbf{掌柜的好眼力呀(或:真真好眼力呀)!}

\textbf{是。}

\textbf{请问掌柜的,这里可有大米饭食无有啊?}

\textbf{呃,面食也好哇,掌柜的取来。}

\textbf{少刻把你。}

\textbf{啊小姐,金耳环一对,三钱重,他们这里是十四换呐。三兑三}\protect\hyperlink{fn611}{\textsuperscript{611}}\textbf{,三四一两二,文银四两,外找大钱二百。小姐收下。}

\textbf{是。}

\textbf{啊小姐,他们这里无有(或:没有)大米饭食,现有面食在此,小姐请用。}

\textbf{啊,小姐为何不用啊?}

\textbf{是。(或:哦,是是是。)}

\textbf{唉,呃\ldots{}\ldots{}(哭介)}

\textbf{唉,老奴也是思念太老爷、太夫人,吞吃不下呀,呃\ldots{}\ldots{}(哭介)}

\textbf{是。}

\textbf{在。}

\textbf{遵命。}

\textbf{啊掌柜的,面食未用,但不知要把多少钱?}

\textbf{哎呀,处处俱有好人呐。}

\textbf{请问掌柜的,此处可有往大同去的脚程无有哇?}

\textbf{哦,有劳了!}

\textbf{诶------这是哪个的脚程呐?}

\textbf{正是。}

\textbf{我们要到(或:我们要往)大同去。}

\textbf{大道怎说,小道怎讲啊?}

\textbf{自然是走近不走远呐。}

\textbf{多把银钱与你呀。}

\textbf{呀呸!吃了我这人,是小事;吃了你这脚程,倒是大事。(或:吃了你的脚程,是大事;吃了我这人,倒是小事。)难道说我这人还不如你那畜类么?}

\textbf{(哼!)真真地岂有此理呀!}

\textbf{(思介)也罢,待我蒙哄小姐,前面去雇。}

\textbf{啊小姐,此处无有往大同去的脚程,待老奴随同小姐,前面去雇。}

\textbf{(是。)}

\textbf{(呃!)}

\textbf{【西皮快板】恨贼子把我的牙咬坏,又埋怨太老爷做事无才。孤雁儿失落在天边外,连累得夫人花井埋。}

\textbf{【西皮快板】有一日拿奸贼与国除害,剥尔的皮、挖尔眼方称心怀,太老爷------你快显灵来。}

\textbf{哦,来了!}

\textbf{{[}第二场{]}}

\textbf{啊小姐,你为何不走哇?}

\textbf{哦!}

\textbf{【西皮快板】小姑娘啼哭坐土台,珠泪点点洒下来。自幼未出闺门界,鞋弓袜小步难捱}\protect\hyperlink{fn612}{\textsuperscript{612}}\textbf{。思想爹娘心放开,头上取下金钗来。缠足带,松放解,轻轻刺破红绣花鞋,好把路捱。}

\textbf{(哎呀!)}

\textbf{【西皮散板】霎时天气变得快,大雪鹅毛飘下来。荒郊俱被冰雪盖,处处楼阁似银台。}

\textbf{啊小姐,你、你\ldots{}\ldots{}你为何又不走啊?}

\textbf{噢!}

\textbf{【西皮快板】小姑娘啼哭坐山边,大雪纷纷遍地漫。腹内无食身寒冷,哪个身穿(或:哪个多穿)几件棉。咬定牙关朝前趱呐,}

\textbf{【西皮散板】小姑娘只哭得实可怜。无奈何脱下了衣一【回龙】件,}

\textbf{【西皮散板】赠与小姐来遮寒。}

\textbf{【西皮散板】男子头上有三昧火,}

\textbf{(我冷!)}

\textbf{(呀!唔\ldots{}\ldots{})}

\textbf{【西皮散板】我比小姐胜十番。}

\textbf{【西皮散板】送姑娘到大同完成婚嫁,留老奴吃一碗闲饭安茶。(或:小姑娘说的是哪里话,讲什么与老奴戴孝披麻。)}

\textbf{【西皮散板】小姑娘说的是哪里话,讲什么与老奴戴孝披麻呀。奴死后四块板}\protect\hyperlink{fn613}{\textsuperscript{613}}\textbf{高岗埋下}\protect\hyperlink{fn614}{\textsuperscript{614}}\textbf{,胜似你姑娘家修庙造塔。(或:送姑娘到大同完成婚嫁,留老奴吃一碗闲饭安茶。)}

\textbf{\ldots{}\ldots{}不才。}

\textbf{【西皮散板】小姑娘说的是一派的疯话,讲什么与太老爷半点不差。}

\textbf{(【西皮散板】奴欺主我就该天雷报打}\protect\hyperlink{fn615}{\textsuperscript{615}}\textbf{,怕的是(或:怕只怕)五阎君差鬼来拿。)}

\textbf{【西皮散板】行一步来至在深山野洼}\protect\hyperlink{fn616}{\textsuperscript{616}}\textbf{,}

\textbf{(哎呀!)}

\textbf{【西皮散板】见一座独木桥把我吓煞。}

\textbf{啊小姐,看前面有一独木朽桥,待老奴将桥垫稳,也好行走。}\protect\hyperlink{fn617}{\textsuperscript{617}}

\textbf{(曹玉莲 【西皮散板】\ldots{}\ldots{})}

\textbf{【西皮散板】似这等冰雪天实难招架,四下里风不顺飘落雪花。}

\textbf{【西皮散板】小姑娘休得要心中害怕(或:担惊害怕),有老奴在身边万无一差。}

\textbf{(曹玉莲 【西皮散板】\ldots{}\ldots{}行路难。)}

\textbf{【西皮散板】你道我行走不方便,紧走几步姑娘观。}

\textbf{【西皮散板】迈开大步朝前趱,}

\textbf{【西皮散板】险些跌倒在深渊。}

\textbf{【西皮散板】四肢无力身寒战,}

\textbf{【西皮散板】不觉来到哇广花山}\protect\hyperlink{fn618}{\textsuperscript{618}}\textbf{。}

\textbf{【西皮导板】耳边厢又听得有人呼唤,}

\textbf{唉!小姐呀,呃\ldots{}\ldots{}(哭介)}

\textbf{【西皮二六】尊一声小姑娘细听我言:实指望保小姐脱离此难,又谁知在中途不得周全。倘若是到不了大同地面,舍下了小姑娘,这样的寒天、大雪纷飞、孤单单你好不可怜。(我的小姑娘啊!}\protect\hyperlink{fn619}{\textsuperscript{619}}\textbf{)}

\textbf{【西皮摇板】忽然抬头来观看,}

\textbf{来了!(来了!)}

\textbf{【西皮摇板】半空中又来了八洞神仙:}

\textbf{【西皮摇板】汉钟离会同着李铁拐呀,}

\textbf{【西皮摇板】曹国舅搀扶着果老仙。}

\textbf{【西皮摇板】蓝采和、吕纯阳在空中显见,}

\textbf{【西皮摇板】韩湘子、何仙姑离了广寒(或:暂离广寒)。}

\textbf{【西皮摇板】那王母娘娘在莲台坐,}

\textbf{【西皮摇板】有金童和玉女随侍两边。}

\textbf{【西皮摇板】东南角下观一眼,}

\textbf{(呃,来了!又来了!)}

\textbf{【西皮摇板】又来了白发苍苍一洞老神仙(或:一个老神仙)。}

\textbf{【西皮摇板】手执鲜花(或:手持鲜花)呵呵笑,}

\textbf{【西皮摇板】他笑我保主不周全。}

\textbf{【西皮摇板】你也笑来我也笑,}

\textbf{哈哈,哈哈,啊,唔\ldots{}\ldots{}}

\textbf{【西皮散板】三魂渺渺归九泉。}

\textbf{参见金母!}

\textbf{圣寿无疆!}

\textbf{\textless{}叫头\textgreater{}小姐!}

\textbf{此处离大同不远,少刻就有人前来迎接于你。恕老奴不能远送了!}

\textbf{且住!想我曹福,一世为奴,今登仙界,怎不令人(或:好不令人)可笑哇!}

\textbf{哈哈,哈哈,}

\textbf{啊------}

\textbf{唉,只是苦了你了哇,呃\ldots{}\ldots{}(哭介)}

\textbf{罢!}

\textbf{附注:}

\textbf{戏中曹玉莲之父曹正邦的原型为曹定邦。据《怀安县文史资料》载:曹定邦(?-1624),生年无考,明代宣府怀安卫魏宁庄(今魏家山)人。万历十九年(1591年)举乡试第一,次年中进士,授江苏淮安府推官,专管一府刑狱。后以治行高第,授吏部给事中,钞发章疏,}稽察违误,权力颇重\textbf{。以法疏劾两京兵部尚书田乐、邢介及云南巡抚陈用宾,田、邢遂引去。吏部郎中赵邦清被诬,定邦疏雪之。后拜谒告归,僦屋以居,不蔽风日。光宗登位,始以太常少卿召,至则改为大理少卿、迁左佥都御史,佐赵南星京察,事竣进左副都御史。天启三年(1623年)秋,吏部缺右侍郎,中旨特用定邦,定邦四辞不得,遂引疾归。天启四年(1624年),启用其为南京都御史,仍辞不拜。时逆臣王绍徽、乔应甲附宦官魏忠贤,必欲害定邦,嘱其党石三畏以东林党劾之,遂予削夺,归乡途中,被东厂派人格杀,并予抄家。其女曹玉莲在家得讯,由老管家陪同往大同逃亡,投奔其未婚翁李总兵,不敢走大路,而于大雪纷纷中自四十里崎岖艰险之桦木山冒寒奔逃,行至大同近郊白登堡,老管家冻馁而死。当地绅民钦其忠烈,为之建庙奉祀。定邦笃志正学,操履刚直,立朝守正不阿,崇奖名教,有古大臣风。1949年前,魏家山曹氏宗祠还设有曹定邦的牌位,供桌上陈列其生前官帽、朝服、传略等。}

\newpage
\hypertarget{ux5c71ux6d77ux5173}{%
\subsection{山海关}\label{ux5c71ux6d77ux5173}}

\textbf{{[}第一场{]}}

\textbf{马龙 (念)朝为田舍郎,暮登天子堂。将相本无种,男儿当自强。}

\textbf{马龙
俺,马龙。吴元帅帐下为将,奉命监造刀枪,刀枪造齐,今乃操演之期。}

\textbf{马龙 众将官,}

\textbf{众 有。}

\textbf{马龙 打道辕门!}

\textbf{马龙 两厢退下。}

\textbf{马龙 嘚儿,开门!}

\textbf{{[}第二场{]}}

\textbf{吴三桂
\textless{}点绛唇\textgreater{}奉命镇边关,灭胡儿,保主江山。}

\textbf{吴三桂
(念)家住赤州在辽阳}\protect\hyperlink{fn620}{\textsuperscript{620}}\textbf{,乌鸦展翅赴钱塘。山海关前为总镇,胡儿不敢犯边疆。}

\textbf{吴三桂
本帅吴三桂,明室为臣。奉命镇守山海关一带等处。恼恨胡儿屡犯边界。也曾命马龙,监造刀枪,不知可曾造齐。每逢三、六、九日,操演人马。站堂军------}

\textbf{站堂军 有!}

\textbf{吴三桂 传马龙进帐。}

\textbf{站堂军 马龙进帐。}

\textbf{马龙 来也!}

\textbf{马龙 参见元帅!}

\textbf{吴三桂 将军少礼。}

\textbf{马龙 谢元帅!}

\textbf{吴三桂 命你监造刀枪,可曾造齐?}

\textbf{马龙 请元帅验刀!}

\textbf{吴三桂 入鞘。}

\textbf{马龙 啊!}

\textbf{吴三桂 马龙听令!}

\textbf{马龙 在。}

\textbf{吴三桂 传令下去,满营大小将官,全身披挂,齐至校场听点、操演。}

\textbf{马龙 得令。令出:下面听者!}

\textbf{众 啊!}

\textbf{马龙 元帅有令:大小三军,齐下校场操演。}

\textbf{李虎 圣旨下!}

\textbf{马龙 何人押旨?}

\textbf{李虎 九门提督李。}

\textbf{马龙 请稍待。}

\textbf{马龙 启禀元帅:圣旨下!}

\textbf{吴三桂 何人押旨?}

\textbf{马龙 九门提督李。}

\textbf{吴三桂 嚯------九门提督李。}

\textbf{马龙 正是。}

\textbf{吴三桂 稍站。}

\textbf{吴三桂
且住,我想九门提督李虎乃闯贼耳目,此番押旨前来,莫非江山有失?}

\textbf{吴三桂 嗯嗯,我自有道理。}

\textbf{吴三桂 马龙!}

\textbf{马龙 在。}

\textbf{吴三桂 少刻李虎前来,教拿就拿,教绑就绑。}

\textbf{马龙 喳。}

\textbf{吴三桂 高搭龙棚,}

\textbf{马龙 喳。}

\textbf{吴三桂 免操接旨。}

\textbf{马龙 喳喳\ldots{}\ldots{}香案接旨。}

\textbf{李虎 啊吴大人!}

\textbf{吴三桂 大人!}

\textbf{李虎 请过圣命。}

\textbf{吴三桂 大人为何不来开读。}

\textbf{李虎 有密言相告。}

\textbf{吴三桂 请坐。}

\textbf{吴三桂 不知大人驾到,未曾远迎,当面恕罪。}

\textbf{李虎 大人为国勤劳,多受风霜之苦。}

\textbf{吴三桂 岂敢。请问大人,金殿领旨,还是午门接旨?}

\textbf{李虎 呃,乃是金殿接旨。}

\textbf{吴三桂 圣旨到来,为何不来开读。}

\textbf{李虎 呃,内印外封,不敢开读。}

\textbf{吴三桂 调本镇进京为了何事?}

\textbf{李虎 这个\ldots{}\ldots{}}

\textbf{吴三桂 讲!}

\textbf{李虎 呃,不过是加官授爵。}

\textbf{吴三桂 哦!加官授爵。}

\textbf{李虎 正是。}

\textbf{吴三桂 我有一言,大人听呐了------}

\textbf{吴三桂 【西皮导板】吴三桂在大堂一言告禀呐,}

\textbf{吴三桂
【西皮原板】尊一声李大人细听分明:崇祯帝坐八载放我出任,山海关倒做了十年总兵。奉圣命镇边关哪得安静,好一个马将军阵阵出征。望大人回朝去启奏一本,再放我三五载回朝奉君。}

\textbf{李虎
【西皮摇板】有李虎在大堂一言奉告,尊一声吴元帅细听根苗:都只为你的父年纪衰了,圣旨到你就该随我还朝。}

\textbf{吴三桂 哦!}

\textbf{吴三桂
【西皮快板】听一言不由我心头焦躁,好一似烈火上把油来浇。我有心传将令将他斩了,}

\textbf{李虎 啊?!}

\textbf{吴三桂 【西皮快板】又恐怕一家人性命难逃。去愁眉换笑脸急忙赔笑,}

\textbf{吴三桂 大人!}

\textbf{吴三桂 【西皮摇板】料理了军务事随你还朝。}

\textbf{李虎 这便才是。}

\textbf{吴三桂 大人请至迎宾馆。}

\textbf{李虎 是是是,暂时别。}

\textbf{吴三桂 少刻陪。}

\textbf{吴三桂 升堂。}

\textbf{吴三桂 马龙听令!}

\textbf{马龙 在。}

\textbf{吴三桂 命你备酒一席,纹银十两,款待李虎。}

\textbf{马龙 得令。}

\textbf{吴三桂 辕门伺候。}

\textbf{众 啊。}

\textbf{吴义 (内)走啊。}

\textbf{站堂军 呔,做什么的?}

\textbf{吴义 元帅家报到。}

\textbf{站堂军 候着。}

\textbf{站堂军 禀元帅:家报到。}

\textbf{吴三桂 传。}

\textbf{站堂军 上面传你,你要小心了!}

\textbf{吴义 是是是。}

\textbf{吴义 三爷!}

\textbf{吴三桂 掩门!}

\textbf{马龙
且住,家报到来,一言未发,将他带入后堂。其中定有缘故!待俺去至二堂,探听动静。正是呃:(念)要知心腹事,但听口中言。}

\textbf{{[}第三场{]}}

\textbf{吴三桂 吴义,随我来!}

\textbf{吴三桂 呃------}

\textbf{吴三桂 胆大吴义,上得关来,见了本帅为何不跪?}

\textbf{吴义 有太夫人书信在怀,老奴不敢下跪。}

\textbf{吴三桂 呈上来。}

\textbf{吴义 三爷请上,老奴参拜。}

\textbf{吴三桂 罢了。}

\textbf{吴三桂 啊吴义,这封书信奉何人所差?}

\textbf{吴义 呃,是太夫人交与老奴。}

\textbf{吴三桂 太老爷呢?}

\textbf{吴义 呃,上朝未归。}

\textbf{吴三桂 哦,上朝未归\ldots{}\ldots{}呃,你在路上行了几日?}

\textbf{吴义 老奴不分昼夜而来。}

\textbf{吴三桂 啊?!为何这样紧急?}

\textbf{吴义 呃,太夫人吩咐老奴,不敢怠慢。}

\textbf{吴三桂 原来如此。后面见过少夫人去罢。}

\textbf{吴义 遵命。}

\textbf{吴义 唉,太老爷,太夫人呐,呃\ldots{}\ldots{}(哭介)}

\textbf{吴三桂
且住!吴义上得关来,变脸变色,其中定有缘故,不免拆开书信一观便了!}

\textbf{吴三桂 爹娘在上,恕孩儿不孝罪也!}

\textbf{吴三桂
【西皮原板】看吴义上关来气色不正,倒教我吴三桂心内着惊。对北京施一礼拆开书信,书上相逢父子情。上写着为娘修书信,三桂我儿看分明:都只为李自成燕山犯境,他要夺吾主爷锦绣龙庭。只逼得崇祯爷在煤山丧命,}

\textbf{吴三桂
\textless{}哭头\textgreater{}崇祯爷,臣的主啊,啊,崇祯爷,}

\textbf{吴三桂
【西皮原板】实可叹周遇吉乱箭攒身。儿的父也不容三拷六问,铜夹棍夹死了儿的天伦。}

\textbf{吴三桂 呀呸!}

\textbf{吴三桂
【西皮快板】看罢书信怒气生,指着燕山骂贼人。本帅二堂无计定,}

\textbf{马龙 【西皮快板】后面来了马大平。迈步且把二堂进,}

\textbf{马龙 啊?!}

\textbf{马龙 【西皮快板】元帅为何两泪淋。}

\textbf{吴三桂
【西皮快板】本帅正在无计定,后面来了马大平。本当实言对他论,马龙是个杀人精。左难右难难坏我,山海关难坏吴总兵。}

\textbf{马龙 【西皮快板】朝中有事天子宣,阃外将军听令行。}

\textbf{吴三桂 【西皮快板】将军有所不知情,本帅大祸临了身。}

\textbf{马龙 【西皮快板】有什么大祸临了身,快对马龙说分明。}

\textbf{吴三桂
【西皮快板】都只为李自成燕山犯境,他要夺吾主爷锦绣龙庭。只逼得崇祯爷煤山丧命,铜夹棍夹死了我的天伦。}

\textbf{马龙
【西皮快板】听说君父丧了命,气得我心头似火焚。元帅快发人和马,杀上燕山莫进城。}

\textbf{吴三桂 【西皮快板】本当点动人和马,帐下缺少一先行。}

\textbf{马龙 【西皮快板】元帅发动人和马,马龙情愿做先行。}

\textbf{吴三桂 【西皮快板】你说此话我不信,}

\textbf{马龙 【西皮快板】愿对苍天把誓呃盟。}

\textbf{吴三桂 你且盟来!}

\textbf{马龙 遵命!}

\textbf{马龙
【西皮快板】马龙二堂忙跪定,过往神灵听分明:马龙保主有假意,死在千军万马营。}

\textbf{吴三桂 【西皮摇板】一见马龙把誓盟,用手搀起马将军。}

\textbf{吴三桂 马龙听令!}

\textbf{马龙 在!}

\textbf{吴三桂 吩咐站堂伺候!}

\textbf{马龙 下面听者:众将站堂伺候!}

\textbf{{[}第四场{]}}

\textbf{吴三桂 (念)恼恨奸贼计,谋害我爹尊。}

\textbf{吴三桂 将李虎绑了上来!}

\textbf{吴三桂 唗!胆大李虎,在朝中谋害我父,推出斩了!}

\textbf{马龙 且慢!}

\textbf{吴三桂 将军为何拦阻。}

\textbf{马龙 若将此贼斩首,岂不便宜了他?}

\textbf{吴三桂 依将军之见?}

\textbf{马龙
将他狗皮剥了,背后刻下战表,与那反贼约定日期鏖战,岂不是好?}

\textbf{吴三桂 好哇!将他犬皮剥了!}

\textbf{吴三桂
(念)上写吴三桂,下写李闯王:逼主煤山丧,令人心惨伤。约定九月九,通州摆战场。来者是君子,不来小儿郎。上写吴三桂题。}

\textbf{吴三桂 押出帐去!}

\textbf{吴三桂 马龙听令!}

\textbf{吴三桂 吩咐满营大小将官:俱穿素白,齐至校场听点。}

\textbf{{[}第五场{]}}

\textbf{李虎
画虎不成反类其犬。吴三桂呀吴三桂,我此番回朝,若不杀你,非为人也!唉呀,痛煞我也!}

\textbf{{[}第六场{]}}

\textbf{马龙 【西皮摇板】宝帐领了元帅令,}

\textbf{马龙 【西皮快板】军前点动众三军。}

\textbf{马龙 【西皮摇板】鞭梢一举校场等,}

\textbf{马龙 【西皮摇板】且听元帅将令行。}

\textbf{吴三桂 【西皮导板】白盔白甲白旗号,}

\textbf{吴三桂 【西皮快板】白旗招展似雪飘。三军与爷催前哨哇,}

\textbf{吴三桂 【西皮摇板】再与众将说根苗。}

\textbf{马龙 参见元帅!}

\textbf{吴三桂 马龙!}

\textbf{马龙 在!}

\textbf{吴三桂 本帅兴兵,众人可服?}

\textbf{马龙 众将俱服!}

\textbf{吴三桂 好哇!坐在雕鞍听我令下------}

\textbf{吴三桂
【西皮快板】坐在雕鞍传令号,大小三军听根苗:一非兴兵把反造,二非兴兵谋当朝。只为反贼行霸道,害死君父恨难消。本帅兴兵把仇报,全仗尔等马后劳。鞭梢一举催前哨哇,}

\textbf{吴三桂 【西皮摇板】老天爷助我哇成功劳。}

\newpage
\hypertarget{ux6d17ux6d6eux5c71-ux4e4b-ux8d3aux5929ux4fdd}{%
\subsection{洗浮山 之
贺天保}\label{ux6d17ux6d6eux5c71-ux4e4b-ux8d3aux5929ux4fdd}}

\textbf{{[}第一场{]}}

(内)马来!

【西皮摇板】大英雄在世间义气为本呐,恨草寇劫皇粮苦及黎民。

适才营中,众家英雄议论剿平浮山之策,我想浮山山高水险,不明地势,只恐遭贼暗算。是我安抚众人不可轻易行动,为此私出大营,暗地查看山势,以便作好准备,就此走遭也!
(或:适才营中同众商议破敌之策,我想浮山山高水险,不明虚实,恐遭暗算。为此安抚众人不可出征,是我单人独骑私出大营,去至浮山,探勘虚实,以作准备也!)

【西皮摇板】暗地里出大营虚实探听,山口外悬人头所为何情?

且住!山口(之外)高悬人头,是何缘故?

呜哙呀!莫非黄贤弟私自探山(或:私探浮山),遭贼毒手?!

嗯,我不免速回大营(或:回转大营),观看动静,再作定夺便了(或:再作计较也)。

呃!(或:且住!)

果然黄贤弟不在营中,定是私探浮山,命丧贼手(或:遭贼毒手)。

哎呀!想我弟兄自结江南(或:结拜江南),患难相顾\protect\hyperlink{fn621}{\textsuperscript{621}}。他今已死,我焉能独生?!

也罢!

俺不免闯进浮山,杀却余六、余七,与黄贤弟报仇雪恨(呐)!

呔!浮山草寇休得猖狂(或:休得逞强),贺爷来也!

\textbf{{[}第二场{]}}

【二黄摇板】想当日在绿林江湖闯荡,转瞬间黄粱梦昙花一场。

吾乃贺天保阴魂是也。只因私探浮山(或:夜探浮山),命丧飞抓之下。我儿仁杰,跟随黄贤弟前来搬运尸灵,今晚夜宿馆驿之中,不免托梦与他,了此一段夙缘也!

【二黄摇板】到如今不能够风流话讲,蜀魄啼梦魂惊心神惨伤。

【二黄摇板】谯楼鼓三更催二更鼓尽,秋气寒月无光惨淡孤魂。

唉,俺贺天保死得好苦也!

\textbf{【反二黄慢板】站馆中(或:站店中)悲切切魂伤魄断,曾记得四霸天结义江南。我这里走向前忠魂呼唤,叫一声黄贤弟细听兄言:实指望洗浮山扫平贼乱,料不想飞抓下一旦生残。望贤弟领人马埋伏湖畔,施巧计擒贼子与兄报冤。诉不尽心头恨咽喉气短,咽喉气呃短,贤弟呀!}

\textbf{【反二黄原板】转面来与我儿再把话言:儿在阳呃、父在阴两厢隔断,可怜儿未长成幼年孤单。老娘亲全靠儿甘旨奉献,但愿儿立大志忠孝当先。天将明父要归不尽悲惨,不尽悲惨,}

\textbf{【反二黄散板】父子们要重逢今世却难。}

\textbf{(\textless{}三叫头\textgreater{}仁杰,我儿,唉,儿啊\ldots{}\ldots{}(哭介))}

\textbf{(罢!)}

\textbf{落马湖·访褚}\protect\hyperlink{fn622}{\textsuperscript{622}}

(刘曾复 饰 褚彪、家院; 朱家溍 饰 黄天霸)

\textbf{褚彪 (内)嗯哼!}

\textbf{(褚彪上)}

\textbf{褚彪
【西皮摇板】数十年为响马扶济危困,可怜我年高迈无有后根。我女儿配关泰良缘天定,朱光祖、李公然二人主婚。}

\textbf{黄天霸 (内)走哇!}

\textbf{(黄天霸上)}

\textbf{黄天霸 【西皮摇板】适才庄前得音信,特拜年高老绿林。}

\textbf{黄天霸
来此已是``侠义结交''------想必就是此处,待我冒叫一声:门上哪位在?}

\textbf{家院 是哪一位?}

\textbf{黄天霸 烦劳通禀,就说黄天霸求见。}

\textbf{家院 稍待。}

\textbf{家院 褚爷,黄天霸求见。}

\textbf{褚彪 哦,黄爷到了。说我出迎。}

\textbf{家院 家爷出迎。}

\textbf{黄天霸 哦,老丈------}

\textbf{褚彪 黄爷------}

\textbf{黄天霸 老丈在哪里?}

\textbf{褚彪 (黄爷)在哪里?}

\textbf{褚彪 黄爷。}

\textbf{黄天霸 晚生特来拜府。}

\textbf{褚彪 请------}

\textbf{褚彪 请坐。}

\textbf{黄天霸 (有)座。}

\textbf{褚彪 不知黄爷驾到,未曾远迎,当面恕罪。}

\textbf{黄天霸 岂敢!愚下来得鲁莽,老丈海涵。}

\textbf{褚彪 黄爷不去保护大人,来到鄙庄何事?}

\textbf{黄天霸 唉!老丈啊------}

\textbf{黄天霸
【西皮快板】我与殷洪两交战,大人改扮离官船。小舟夤夜渡过了岸,平白江下起祸端。}

\textbf{褚彪 大人又起了什么祸端?}

\textbf{黄天霸 大人又被水寇擒去了。}

\textbf{褚彪 就该前去搭救才是。}

\textbf{黄天霸
晚生等自离官船,三日三夜,并无大人踪影。久闻老丈乃是前辈老英雄,定然知晓这水路之中首领何人,隐藏何地------但愿救出忠良,哎呀老丈啊------晚生天霸,恩当重报!}

\textbf{褚彪 请坐。}

\textbf{褚彪
想是旱路英雄,有老汉在此,吩咐一声,量他们不敢造次。这水路英雄------}

\textbf{黄天霸 哦\ldots{}\ldots{}}

\textbf{褚彪 这\ldots{}\ldots{}哦哦,我倒想起一家来了。}

\textbf{黄天霸 哪一家?}

\textbf{褚彪
离此不远有一落马湖,为首之人名叫李佩------就是那``铁臂猿猴''。大人若被他人擒去,大略凶多吉少。}

\textbf{黄天霸 就烦老丈前去走走,不知意下如何?}

\textbf{褚彪 黄爷若是来迟,老汉就不在家下了。}

\textbf{黄天霸 到哪里去?}

\textbf{褚彪 去探望一好友。}

\textbf{黄天霸 难道老丈的故友就胜似施大人不成?}

\textbf{褚彪 提起此人,大大有名。}

\textbf{黄天霸 哪一家?}

\textbf{褚彪 就是那万君兆。}

\textbf{黄天霸 敢莫是``八臂哪吒''?}

\textbf{褚彪 正是。}

\textbf{黄天霸 我与他有八拜之交,但不知他如今住在哪里?}

\textbf{褚彪
他现在徐州夹沟驿}\protect\hyperlink{fn623}{\textsuperscript{623}}\textbf{。}

\textbf{黄天霸 我同老丈前去拜访。}

\textbf{褚彪
且慢,夹沟驿在北,落马湖在南}\protect\hyperlink{fn624}{\textsuperscript{624}}\textbf{,还是搭救大人要紧。}

\textbf{黄天霸
这个------呃,呃,言得极是。就烦老丈传达问候,倘他夫妻问起我的情由,切不可提及大人之事。}

\textbf{褚彪 却是为何?}

\textbf{黄天霸 恐怕替我担惊。告辞了!}

\textbf{褚彪 送。}

\textbf{黄天霸 【西皮摇板】多蒙老丈指明路,}

\textbf{黄天霸 【西皮摇板】披星戴月赶程途!}

\textbf{黄天霸 请。}

\textbf{(黄天霸下)}

\textbf{褚彪 【西皮摇板】侠义英名传千古,}

\textbf{褚彪 【西皮摇板】凌烟高阁标画图。}

\textbf{(褚彪下)}


\item
  \leavevmode\hypertarget{fn544}{}%
  《京剧汇编》第九十八集
  王连平藏本作``要把大元扫''。\protect\hyperlink{fnref544}{↩}
\item
  \leavevmode\hypertarget{fn545}{}%
  《京剧汇编》第九十八集
  王连平藏本作``凌烟阁标名''。\protect\hyperlink{fnref545}{↩}
\item
  \leavevmode\hypertarget{fn546}{}%
  据樊百乐君告知,此处龙套唱的\textless{}\textbf{泣颜回}\textgreater{}
  ``牌子''用的是《连环计·起布问探》的词句:

  ``羽檄会诸侯,运神机,阵拥貔貅。同心勠力,斩奸臣,拂拭吴钩。叹蒙尘冕旒,起群雄,云绕夸争斗。看长江浪息风恬,济川人自在行舟。''\protect\hyperlink{fnref546}{↩}
\item
  \leavevmode\hypertarget{fn547}{}%
  姜骏建议作``须当勠力'',此处从《京剧汇编》第九十八集
  王连平藏本。\protect\hyperlink{fnref547}{↩}
\item
  \leavevmode\hypertarget{fn548}{}%
  ``齐眉盖鬓''俗作``齐眉盖顶'',据钱盛君告,后者是地方戏和鼓书词中常见``水词儿''。\protect\hyperlink{fnref548}{↩}
\item
  \leavevmode\hypertarget{fn549}{}%
  谭鑫培、李寿山设计。《战太平》花云与陈英杰(一般``英''字写成``友'',《英烈传》中是``英''字)的情况与《南阳关》伍云召、韩擒虎不同。头场开打是陈想攻下太平,花是想奋力杀退陈,等候救兵。这场开打也不能太多,太多显松,不能显示花云一鼓作气把陈击退,但又不能太少,少则不足以显示花云之勇。谭鑫培这一场采用过``大扫琉璃灯'',比``灯笼泡''火炽,但又干净利落,颇为得体。陈英杰由李寿山配演。花云得胜后是想回城防卫,所以不用龙套追过场,要一个小下场下。另一场是花云急于保护朱文逊杀出重围,与陈英杰无心恋战,因此不能多打,但打陈下后,紧接耍一个大下场,表示奋力突围。\protect\hyperlink{fnref549}{↩}
\item
  \leavevmode\hypertarget{fn550}{}%
  余叔岩、钱金福设计。\protect\hyperlink{fnref550}{↩}
\item
  \leavevmode\hypertarget{fn551}{}%
  这是贯大元介绍的谭鑫培《战太平》头场花云开打后的下场。括号中隶体字标注的是《老生把子》一文与《京剧新序》中不同处。\protect\hyperlink{fnref551}{↩}
\item
  \leavevmode\hypertarget{fn552}{}%
  杨小楼晚年耍完三个迎面花后减去扫腿等,只打一下,直接就在脸前画三圈。许多戏他都用这个下场,例如他演《下河东》时第二个下场就用它。\protect\hyperlink{fnref552}{↩}
\item
  \leavevmode\hypertarget{fn553}{}%
  刘曾复先生为林瑞平、杨甲戌二位先生说戏录音接近刘先生为吴小如先生示范的王凤卿唱法:

  ``【西皮原板】\textbf{早就该差能将}【转西皮二六】\textbf{前来提防}。【西皮摇板】将身儿来在大街呀上,''。\protect\hyperlink{fnref553}{↩}
\item
  \leavevmode\hypertarget{fn554}{}%
  刘曾复先生专门介绍,在《英烈传》小说中是``孙侍女'',后来戏台上变成了``二夫人''。\protect\hyperlink{fnref554}{↩}
\item
  \leavevmode\hypertarget{fn555}{}%
  陈超老师注:这一场,括号内的词句也是他跟刘曾复先生学的原词。是三庆班的名演员冯瑞祥改的程长庚唱词,冯瑞祥与孙春恒是三庆班的``一文一武''。谭鑫培《战太平》的唱念和身段都学冯瑞祥。\protect\hyperlink{fnref555}{↩}
\item
  \leavevmode\hypertarget{fn556}{}%
  段公平君建议``怒骂''均作``辱骂''。\protect\hyperlink{fnref556}{↩}
\item
  \leavevmode\hypertarget{fn557}{}%
  夏行涛君建议作``搭进帐''。\protect\hyperlink{fnref557}{↩}
\item
  \leavevmode\hypertarget{fn558}{}%
  据李楠君告知,余派原词作``暗地设计把太平抢'',他从刘曾复先生学的是``暗地设计非为上''。\protect\hyperlink{fnref558}{↩}
\item
  \leavevmode\hypertarget{fn559}{}%
  陈超老师注:此句原是``止不住珠泪洒衣襟。''刘曾复先生认为``这句词太乏,我有时候就用《打登州》代替,马马虎虎''。王凤卿的传承的这套《战太平》唱词是程长庚的词句。给吴先生的录音中孙侍女身份已改成二夫人,这属于王凤卿的``过渡时期''的词句,王到后来索性就唱谭派词了。\protect\hyperlink{fnref559}{↩}
\item
  \leavevmode\hypertarget{fn560}{}%
  ``国''字系入声字,此处保留湖北方言念法。\protect\hyperlink{fnref560}{↩}
\item
  \leavevmode\hypertarget{fn561}{}%
  樊百乐君告知,此处另一作``故乡客'',因为``客''也属于应``归来''之属;吴小如先生此处作``故相违'';郝以鑫君则建议作``故相随'';另,据钱盛君介绍,在湘剧戏本中,此句作``亲不亲来旧相识''(``识''字在湘剧中归入``灰堆辙''),鉴于此戏各地方戏中唱词均大同小异,因此不排除此处可能作``故相识''解;聊备一说。\protect\hyperlink{fnref561}{↩}
\item
  \leavevmode\hypertarget{fn562}{}%
  ``草鸡毛''是根据何毅老师与樊百乐君的建议整理的。\protect\hyperlink{fnref562}{↩}
\item
  \leavevmode\hypertarget{fn563}{}%
  继美、继远的名字从《传统京剧汇编》第八集
  范叔年藏本。\protect\hyperlink{fnref563}{↩}
\item
  \leavevmode\hypertarget{fn564}{}%
  括号中的内容,主要参照刘曾复先生与梁小鸾合作(李斌植 司鼓、屠楚材
  操琴)演出录音整理。\protect\hyperlink{fnref564}{↩}
\item
  \leavevmode\hypertarget{fn565}{}%
  夏行涛君建议可作``臣按他''。\protect\hyperlink{fnref565}{↩}
\item
  \leavevmode\hypertarget{fn566}{}%
  段公平君注:\textbf{吴小如先生曾撰文称``你三人''是余派准词,孟小冬、李少春皆如此。李舒先生遗作《涉艺所得》载刘老唱词,亦是``你三人''。}\protect\hyperlink{fnref566}{↩}
\item
  \leavevmode\hypertarget{fn567}{}%
  《京剧新序》中``做''字误为``坐''。\protect\hyperlink{fnref567}{↩}
\item
  \leavevmode\hypertarget{fn568}{}%
  《京剧新序》中``更''字误为``桌''。\protect\hyperlink{fnref568}{↩}
\item
  \leavevmode\hypertarget{fn569}{}%
  《京剧新序》中``草''字误为``叫''。\protect\hyperlink{fnref569}{↩}
\item
  \leavevmode\hypertarget{fn570}{}%
  根据刘曾复先生钞本整理。\protect\hyperlink{fnref570}{↩}
\item
  \leavevmode\hypertarget{fn571}{}%
  刘曾复教授钞本作``二黄平板''。\protect\hyperlink{fnref571}{↩}
\item
  \leavevmode\hypertarget{fn572}{}%
  刘曾复教授钞本作``趁我意''。\protect\hyperlink{fnref572}{↩}
\item
  \leavevmode\hypertarget{fn573}{}%
  刘曾复教授钞本作``何为''。\protect\hyperlink{fnref573}{↩}
\item
  \leavevmode\hypertarget{fn574}{}%
  刘曾复先生钞本作``执对''。\protect\hyperlink{fnref574}{↩}
\item
  \leavevmode\hypertarget{fn575}{}%
  《京剧汇编》第三十九集
  潘侠风藏本作``梅花痦'',此处从刘曾复先生钞本,下同。\protect\hyperlink{fnref575}{↩}
\item
  \leavevmode\hypertarget{fn576}{}%
  刘曾复先生钞本作``略无妨碍''。\protect\hyperlink{fnref576}{↩}
\item
  \leavevmode\hypertarget{fn577}{}%
  刘曾复先生钞本作``手剪子'',下同。\protect\hyperlink{fnref577}{↩}
\item
  \leavevmode\hypertarget{fn578}{}%
  此处刘曾复先生钞本作``双眉皱''。\protect\hyperlink{fnref578}{↩}
\item
  \leavevmode\hypertarget{fn579}{}%
  刘曾复先生钞本作``乎臭的'',下同。\protect\hyperlink{fnref579}{↩}
\item
  \leavevmode\hypertarget{fn580}{}%
  此处刘曾复先生钞本文字不确认,段公平君建议作``那么回事''。\protect\hyperlink{fnref580}{↩}
\item
  \leavevmode\hypertarget{fn581}{}%
  ``接年'',老北京话,即``隔年''的意思。李楠君按:``接''字的正字应是``间'',因音近讹。\protect\hyperlink{fnref581}{↩}
\item
  \leavevmode\hypertarget{fn582}{}%
  ``下销'',指死人入殓后,在棺材上钉入销钉(销钉一般是七根,俗称``子孙钉'')。\protect\hyperlink{fnref582}{↩}
\item
  \leavevmode\hypertarget{fn583}{}%
  ``狗碰头''是北京俗话,形容棺材非常简易,野狗用头就能撞开棺材板子,把尸体掏出来吃掉。\protect\hyperlink{fnref583}{↩}
\item
  \leavevmode\hypertarget{fn584}{}%
  莫怀古曾任太仆寺卿。此处段公平君建议作``补官''。\protect\hyperlink{fnref584}{↩}
\item
  \leavevmode\hypertarget{fn585}{}%
  此戏的文字整理是在刘曾复先生两次为吴小如先生说《打严嵩》的邹应龙的基础上,结合刘先生为樊百乐君说戏总讲的实况录音整理完成的,因为两个本子略有出入,文本尽量兼顾两方面。刘曾复先生说戏时专门说明,《打严嵩》是本戏《玉夔龙》的头一出。\protect\hyperlink{fnref585}{↩}
\item
  \leavevmode\hypertarget{fn586}{}%
  一般戏中都作``邱、马二匠'',邱、马二人是两位匠人;据陈超老师告知,刘曾复先生传授他的是``邱、马二将'',此处从之,下同。\protect\hyperlink{fnref586}{↩}
\item
  \leavevmode\hypertarget{fn587}{}%
  ``尊官''也可称``亲翁''。\protect\hyperlink{fnref587}{↩}
\item
  \leavevmode\hypertarget{fn588}{}%
  也可自称``弟''。\protect\hyperlink{fnref588}{↩}
\item
  \leavevmode\hypertarget{fn589}{}%
  陈超老师介绍:刘曾复先生教授时说过,严嵩接旨后,``一翻两翻''是现在的演法,应该是唱四句``金钟三响王退殿,文武有怒不敢言,别驾离朝回府转,见了应龙说根源。''\protect\hyperlink{fnref589}{↩}
\item
  \leavevmode\hypertarget{fn590}{}%
  ``暗藏金钩来拿定,千岁驾前说分明。''
  \textbf{两句,}陈超老师从刘曾复先生学的是``暗放金钩海鳌引,千岁驾前问安宁。''\protect\hyperlink{fnref590}{↩}
\item
  \leavevmode\hypertarget{fn591}{}%
  明太祖和常遇春(录音中作徐达)的画像。\protect\hyperlink{fnref591}{↩}
\item
  \leavevmode\hypertarget{fn592}{}%
  此句从陈超老师的建议当作``矫造皇命来戏弄''。\protect\hyperlink{fnref592}{↩}
\item
  \leavevmode\hypertarget{fn593}{}%
  邹应龙问话时不能直接问常宝童。\protect\hyperlink{fnref593}{↩}
\item
  \leavevmode\hypertarget{fn594}{}%
  此处严嵩念京白。\protect\hyperlink{fnref594}{↩}
\item
  \leavevmode\hypertarget{fn595}{}%
  刘曾复先生在两次说《打严嵩》一戏的时候,严嵩拿砖打脚面的情节安排得不完全一样,为吴小如先生介绍的时候此情节在最后。具体可参阅录音。\protect\hyperlink{fnref595}{↩}
\item
  \leavevmode\hypertarget{fn596}{}%
  此处严嵩念京白。\protect\hyperlink{fnref596}{↩}
\item
  \leavevmode\hypertarget{fn597}{}%
  《余派戏词钱氏辑粹》\textsuperscript{{[}23{]}.}载孟小冬说戏本作``月中''。\protect\hyperlink{fnref597}{↩}
\item
  \leavevmode\hypertarget{fn598}{}%
  刘曾复先生曾回忆\textsuperscript{{[}10{]}.},此句余叔岩传孟小冬的词句为``赴科场贡士试平步登云''。\protect\hyperlink{fnref598}{↩}
\item
  \leavevmode\hypertarget{fn599}{}%
  文衡,旧指以文章取士的标准来取舍权衡。又指科举制度下的主考官。段公平君注:这是王荣山唱法。\protect\hyperlink{fnref599}{↩}
\item
  \leavevmode\hypertarget{fn600}{}%
  吴小如先生曾撰文\textsuperscript{{[}24{]}.}指出旧时艺人因``凄凄''的异体``恓恓''讹传为``洒洒'';王端璞先生改唱为``孤孤零零'',吴从之。\protect\hyperlink{fnref600}{↩}
\item
  \leavevmode\hypertarget{fn601}{}%
  东汉宋弘对妻忠诚,光武帝要他改尚寡姐湖阳公主时,宋弘有名言:``贫贱之交不可忘,糟糠之妻不下堂!''\protect\hyperlink{fnref601}{↩}
\item
  \leavevmode\hypertarget{fn602}{}%
  中馈,古时指妇女在家中主持饮食等事,后引申指妻室。\protect\hyperlink{fnref602}{↩}
\item
  \leavevmode\hypertarget{fn603}{}%
  段公平君建议``今朝''均作``今早''。\protect\hyperlink{fnref603}{↩}
\item
  \leavevmode\hypertarget{fn604}{}%
  段公平君作``住------喳'';吴焕老师整理本作``住------者''。\protect\hyperlink{fnref604}{↩}
\item
  \leavevmode\hypertarget{fn605}{}%
  这段唱腔,目前通行唱法是依据李适可传承的【二黄快三眼】,刘曾复先生示范的词句如下:

  【二黄快三眼】臣不奏前三皇后代五帝,奏的是我大明一段机密:太祖爷在南京称孤立帝,各路的烟尘起来夺华夷。四川省明玉珍把兵来起,领人马从蜀东杀到蜀西。方国珍在浙江自立为帝,苏州城张士诚僭位登极。湖广的陈友谅兴兵起义,南京城夺取那采石矶。玉山城设下了诓君之计,在鄱阳湖边火莲炎焰血染征衣。只杀得有庄有田无人耕地,只杀得贸易经商客旅稀。只杀得弟唤兄来兄不能顾弟,只杀得父在东来子在西。先皇爷坐江山并非容易,十八年改国号臣不能全知。

  又,据李楠君告知,``火莲炎焰''四字他学的是``火焰连连''。\protect\hyperlink{fnref605}{↩}
\item
  \leavevmode\hypertarget{fn606}{}%
  柴俊为老师出示湘剧《二进宫》老唱片,``七篇文章''一段有``怕的是效韩信,辜负我十载寒窗、九载遨游、八月科场、七篇锦绣、鹿鸣筵欢、五经魁首、四杆彩旗、三杯御酒、两朵金花、鳌头独占、独占鳌头,好容易个兵部侍郎。''近于谑而虐了。\protect\hyperlink{fnref606}{↩}
\item
  \leavevmode\hypertarget{fn607}{}%
  \textbf{关于这段唱词的
  ``渔樵耕读''、``琴棋书画''、``四季花名''词句,各有所本,自成体系。}兹再举三例:

  \begin{quote}
  吴小如先生记录的张伯驹先生的余派``渔樵耕读''、``四季花名''词\textsuperscript{{[}17{]}.}为:

  臣要学姜子牙钓鱼渭上,臣要学钟子期采樵山岗,臣要学诸葛亮躬耕陇上,臣要学吕蒙正苦读寒窗。春来百花齐开放,夏至荷花满池塘,秋来菊花金钱样,冬至腊梅戴雪霜。

  王庾生先生的``渔樵耕读''词\textsuperscript{{[}17{]}.}为:

  臣不学兴周灭纣姜吕望,臣不学管仲相齐邦,臣不学三国中诸葛丞相,臣要学隐居山林的张子房。

  宋湛清先生转述言派的``渔樵耕读''、``琴棋书画''、``四季花名''词\textsuperscript{{[}25{]}.}为:

  臣不学兴周的姜公吕望,臣愿学钟子期砍樵山岗。臣不学尉迟恭种田在庄上,臣愿学吕蒙正苦读文章。抚一曲高山流水声嘹亮,闲无事对棋盘散心肠。看一本古书精神爽,巧笔丹青挂在两旁。春兰发花王者之相,夏时莲花满池塘。秋后的菊花高士样,冬日的红梅雪上添香。
  \end{quote}

  \protect\hyperlink{fnref607}{↩}
\item
  \leavevmode\hypertarget{fn608}{}%
  吴小如先生告知,此句夏山楼主改为``莫不是嫌为臣官卑职小''。\protect\hyperlink{fnref608}{↩}
\item
  \leavevmode\hypertarget{fn609}{}%
  吴小如先生早年曾有专文\textsuperscript{{[}17{]}.}指出,此处徐延昭原来唱的是``杨波带驾'',从文意上亦更通顺。\protect\hyperlink{fnref609}{↩}
\item
  \leavevmode\hypertarget{fn610}{}%
  此处《京剧汇编》第四集
  陈少霖藏本作``刘司羽''。按剧情,天启年间,吏部尚书曹正邦得罪魏忠贤被贬,携眷返故里;魏忠贤遣心腹刘司羽中途埋伏,杀曹正邦全家;仅女儿曹玉莲在老仆曹福掩护下得以逃脱。\protect\hyperlink{fnref610}{↩}
\item
  \leavevmode\hypertarget{fn611}{}%
  《京剧汇编》第四集
  陈少霖藏本作``三得三''。\protect\hyperlink{fnref611}{↩}
\item
  \leavevmode\hypertarget{fn612}{}%
  ``步难捱'',《京剧汇编》第四集
  陈少霖藏本作``步难挨'';类似地``好把路捱''作``好把路挨''。\protect\hyperlink{fnref612}{↩}
\item
  \leavevmode\hypertarget{fn613}{}%
  ``四块板''是棺材的别称,因为最简陋的棺材没有头、足板(略强于芦席裹尸安葬)。\protect\hyperlink{fnref613}{↩}
\item
  \leavevmode\hypertarget{fn614}{}%
  段公平君作``刚刚埋下''。\protect\hyperlink{fnref614}{↩}
\item
  \leavevmode\hypertarget{fn615}{}%
  李元皓君建议作``天雷爆打'';夏行涛君建议作``天雷暴打''。\protect\hyperlink{fnref615}{↩}
\item
  \leavevmode\hypertarget{fn616}{}%
  夏行涛君建议作``深山野凹''。\protect\hyperlink{fnref616}{↩}
\item
  \leavevmode\hypertarget{fn617}{}%
  陈超老师注:此处刘曾复先生强调,老生身段很细腻:老生搬两次石头,石头凉,用衣裳垫上,再试。过桥后老生在大边外角``踹鸭''后倒,旦角到小边里角,二人斜着一条线。舞台调度特别。\protect\hyperlink{fnref617}{↩}
\item
  \leavevmode\hypertarget{fn618}{}%
  一般作``广华山'',此处从《京剧汇编》第三集
  陈少霖藏本。\protect\hyperlink{fnref618}{↩}
\item
  \leavevmode\hypertarget{fn619}{}%
  段公平君注:余叔岩通常唱此句,老唱法时常不唱,与之相似的如``好把路捱''句。\protect\hyperlink{fnref619}{↩}
\item
  \leavevmode\hypertarget{fn620}{}%
  李元皓君建议作``甲卒驰骤在辽阳''。\protect\hyperlink{fnref620}{↩}
\item
  \leavevmode\hypertarget{fn621}{}%
  夏行涛君建议作``患难相共''。\protect\hyperlink{fnref621}{↩}
\item
  \leavevmode\hypertarget{fn622}{}%
  根据1996年春朱家溍、刘曾复先生在出席北京市劲松社区京剧茶座的``新春京剧演唱会''时合作录音整理。二老仿学杨小楼、鲍吉祥唱片《落马湖·访褚》。落马湖亦作``骆马湖'',骆马湖位于今江苏宿迁,亦名乐马湖、洛马湖。\protect\hyperlink{fnref622}{↩}
\item
  \leavevmode\hypertarget{fn623}{}%
  杨小楼、鲍吉祥百代、高亭唱片均作``侠沟驿''。\protect\hyperlink{fnref623}{↩}
\item
  \leavevmode\hypertarget{fn624}{}%
  杨小楼、鲍吉祥百代、高亭唱片均作``\textbf{侠沟驿在南,落马湖在北}''。

  据陈超老师介绍,江苏徐州旧有``夹沟驿'',位于徐州城东北;落马湖位于徐州和宿迁之间。因此以地理位置考,当以``夹沟驿在北,落马湖在南''为是。\protect\hyperlink{fnref624}{↩}
\item
