\newpage
\phantomsection %实现目录的正确跳转

\section*{\large\hei {摘缨会}}
\addcontentsline{toc}{section}{\hei 摘缨会}
\hangafter=1                   %2. 设置从第1⾏之后开始悬挂缩进  %
\setlength{\parindent}{0pt}{ 
{\centerline{{[}{\hei 第一场}{]}}}\vspace{5pt}

({\hwfs 打朝}众官{\hwfs 上}\textless{}\!{\bfseries\akai 点绛唇}\!\textgreater{}{\hwfs 报名})

\setlength{\hangindent}{52pt}{
	众报名\hspace{20pt}大司马申无畏(台顶,红三块瓦,黪满)。上大夫苏从(夫子盔,黪三)。下大夫虞邱(荷叶盔,黑三)。上将军熊负羁(帅盔,黑三)。车左将军潘尪(踏镫,紫三块瓦,黑满)。车右将军乐伯(倒缨盔,武生)。左殿将军公子婴齐(虎头盔,武生)。右殿将军公子侧(虎头盔、元宝脸、黑一字)。左队先锋养由基(硬扎巾、大额子、武生)。右队先锋襄老(狮子盔、老丑)。({\hwfs 左右分站})}

申无畏\hspace{20pt}列公请了。

众\hspace{40pt}请了。

申无畏\hspace{20pt}大王设朝,两厢伺候。

众\hspace{40pt}请。

({\hwfs 左右分下},\textless{}\!{\bfseries\akai 大锣打下}\!\textgreater{}{\hwfs 接}\textless{}\!{\bfseries\akai 打上}\!\textgreater{})

\vspace{3pt}{\centerline{{[}{\hei 第二场}{]}}}\vspace{5pt}

({\hwfs 四}太监{\hwfs 站门}、大太监、庄王\textless{}\!{\bfseries\akai 大锣打上}\!\textgreater{},庄王九龙冠,黄开氅)

\setlength{\hangindent}{52pt}{
	庄王\hspace{30pt}({\akai 引})征战戎夷,立功勋,息卷旌旗(另本:~盛世兴隆,图霸业,一统江洪)。}

({\hwfs 正面小座}{\akai 念})

\setlength{\hangindent}{52pt}{
	庄王\hspace{30pt}为征陆浑({\akai 或}:~为征陆戎)动兵机,越椒无理({\akai 或}:~越椒无故)把孤欺。幸有众卿逞雄力,才得凯歌马停蹄。(另本:~楚国逞强有数秋,龙争虎斗统貔貅。图霸中原孤为首,山河一统乐无忧。)孤,熊旅。坐镇荆州,承先王基业,复立楚国,三年以来号令未申。大夫申无畏、苏从直谏,重整先王原例,是孤欲图中原。适逢川贼陆戎兄妹造反,孤御驾亲征,在朝摘去斗越椒兵权印信({\akai 或}:~兵权信印),不想奸贼妒贤嫉能({\akai 或}:~嫉贤妒能),司马蒍贾满门尽丧,斗贼兴兵劫驾,多亏养由基灭贼除患,保孤回朝。寡人今日设宴渐台,犒赏群臣({\akai 或}:~犒赏三军),以表孤(王)爱戴贤臣之意。内侍,与孤传旨,不论裨、副牙将大小官员({\akai 或}:~不论偏、副牙将,大小将官),满朝文武齐上渐台领宴共赏。}

太监\hspace{30pt}大王有旨,不论裨副牙将大小官员,满朝文武齐上渐台领宴共赏。

内众\hspace{30pt}领旨。

({\hwfs 两边分上},{\hwfs 四}牙将{\hwfs 左右跟上},牙将软靠,唐狡红靠、扎巾盔,{\hwfs 站大边外角})

众\hspace{40pt}臣等参见大王。

庄王\hspace{30pt}众卿平身。

众\hspace{40pt}千千岁。

({\hwfs 分站左右})

申无畏\hspace{20pt}宣臣等上殿有何旨意?

庄王\hspace{30pt}孤今得胜还朝({\akai 或}:~孤王今日得胜回朝),在渐台设宴,与众卿同饮。

申无畏\hspace{20pt}臣等领旨。

庄王\hspace{30pt}内侍。

太监\hspace{30pt}有。

庄王\hspace{30pt}酒宴可曾齐备?

太监\hspace{30pt}俱已齐备。

庄王\hspace{30pt}带路渐台。

太监\hspace{30pt}带路渐台呀。

\vspace{3pt}{\centerline{{[}\hei 连场第二场{]}}}\vspace{5pt}

(``{\bfseries\akai 吹打}''众{\hwfs 转场},庄王{\hwfs 最后下},众{\hwfs 又领上},{\hwfs 上渐台},庄{\hwfs 上正面小帐子高台大座},大太监{\hwfs 大边椅子上},众{\hwfs 分坐两边大座},{\hwfs 唐狡坐大边最外椅},\textless{}\!{\bfseries\akai 牌子}\!\textgreater{}{\hwfs 停})

庄王\hspace{30pt}看宴。(众卿请。)

众\hspace{40pt}千岁请。

(\textless{}\!{\bfseries\akai 玉芙蓉}\!\textgreater{}前段,{\hwfs 饮酒},{\hwfs 饮完})

庄王\hspace{30pt}内侍。

太监\hspace{30pt}有。

庄王\hspace{30pt}宣许娘娘上渐台。

(太监{\hwfs 下椅子})

太监\hspace{30pt}大王有旨,宣许娘娘上渐台。

内众\hspace{30pt}领旨。

({\akai 西皮}\textless{}\!{\bfseries\akai 小开门}\!\textgreater{},{\hwfs 四}宫女{\hwfs 引}许姬{\hwfs 上}{\akai 念})

许姬\hspace{30pt}深感隆恩君锡宠,轻移莲步上渐台。

(宫女{\hwfs 挖门站高台后侧两边},许姬{\hwfs 参拜})

许姬\hspace{30pt}妾妃见驾,大王千岁。

庄王\hspace{30pt}梓童平身。

许姬\hspace{30pt}千千岁。(许姬{\hwfs 上小边椅})宣妾妃上渐台不知有何旨意?

\setlength{\hangindent}{52pt}{
	庄王\hspace{30pt}孤王今日设宴犒赏功臣,梓童在众臣({\akai 或}:~在众将)席前行酒一巡,以表孤王爱戴贤臣之意。}

许姬\hspace{30pt}妾闻男女不渎,何况君臣?

%申无畏、\\
%苏从\hspace{30pt}\raisebox{5pt}{臣等叩沐恩宠,何敢再劳娘娘赐酒。}
\raisebox{0pt}[22pt][16pt]{\raisebox{8pt}{申无畏}\raisebox{-8pt}{\hspace{-32pt}{苏从}}\raisebox{0pt}{\hspace{30pt}臣等叩沐恩宠,何敢再劳娘娘赐酒。}}

\setlength{\hangindent}{52pt}{
	庄王\hspace{30pt}孤王今日设宴,原为君臣共欢,众卿不必谦逊({\akai 或}:~不必过谦),孤王言重如山,岂能反悔,梓童巡酒。}

许姬\hspace{30pt}领旨。

(许{\hwfs 下椅唱}【{\akai 西皮二六}】,太监{\hwfs 随下端壶})

\setlength{\hangindent}{52pt}{
许姬\hspace{30pt}【{\akai 西皮二六}】征夷戎立功勋天时顺应,诛越椒回朝转河晏海清。捧香醪献君王虔诚恭敬(给庄王斟酒),愿吾主乐陶陶福寿康宁。我这里奉王命把酒侍饮(给大边臣斟),在宴前依次序来敬杯巡(给小边臣斟)。似这等有道君万民之幸(给小边牙将斟),辅国家秉忠义万世留名(给唐狡连斟三杯)。耳听得风声响阶檐震动(壶递太监), }

许姬\hspace{30pt}【{\akai 西皮散板}】霎时间阖台上黑暗不明。

(\textless{}\!{\bfseries\akai 乱锤}\!\textgreater{},唐狡,许姬{\hwfs 推磨},许{\hwfs 摘}唐狡{\hwfs 缨},{\hwfs 上小边椅与}庄王{\hwfs 语}、{\hwfs 递缨即面牌},狡{\hwfs 坐下})

%申无畏、\\
%苏从 从\hspace{20pt}\raisebox{5pt}{快快点烛。}
\raisebox{0pt}[22pt][16pt]{\raisebox{8pt}{申无畏}\raisebox{-8pt}{\hspace{-32pt}{苏从}}\raisebox{0pt}{\hspace{30pt}快快点烛。}}

庄王\hspace{30pt}慢、慢$\cdots{}\cdots{}$掌灯。

(众{\hwfs 应})

庄王\hspace{30pt}众卿。

众\hspace{40pt}大王。

\setlength{\hangindent}{52pt}{
	庄王\hspace{30pt}今日此宴原为君臣共饮({\akai 或}:~君臣共欢),诸事无忌,孤王出一酒令,众卿可将盔缨摘下,丢过席前,作一绝缨大会,但有不遵者,孤王不以法度治之,只罚酒三巨觥。}

众\hspace{40pt}臣等遵旨。

(\textless{}\!{\bfseries\akai 冲头}\!\textgreater{}{\hwfs 中}太监{\hwfs 拿盘两边收缨},{\hwfs 锣中}唐狡{\hwfs 摸地找缨不见仍坐下},太监{\hwfs 收完端盘上椅放盘在桌上})

庄王\hspace{30pt}可曾摘齐?

众\hspace{40pt}俱已摘齐。

庄王\hspace{30pt}吩咐掌灯。

(太监、众{\hwfs}应)

(庄王{\hwfs 左右两望})

庄王\hspace{30pt}哈$\cdots{}\cdots{}$许姬回宫。

(\textless{}\!{\bfseries\akai 小开门}\!\textgreater{},许姬{\hwfs 下椅},{\hwfs 拜辞},宫女{\hwfs 领下})

\setlength{\hangindent}{52pt}{
许姬\hspace{30pt}【{\akai 西皮散板}】暗中牵袂醉中情,玉手如风已绝缨。 }

(许姬{\hwfs 下})

庄王\hspace{30pt}众卿再饮一回({\akai 或}:~再饮几回)。

众\hspace{40pt}臣等酒足,请驾回宫。

庄王\hspace{30pt}退班。

(\textless{}\!{\bfseries\akai 玉芙蓉}\!{\bfseries\akai 合头}\!\textgreater{},庄王众{\hwfs 窝下}。{\hwfs 留}唐狡,狡、{\hwfs 比势}、{\hwfs 摸颈},{\hwfs 怕介},{\hwfs 下})

\vspace{3pt}{\centerline{{[}{\hei 第三场}{]}}}\vspace{5pt}

(\textless{}\!{\bfseries\akai 水底鱼}\!\textgreater{}公子宋{\hwfs 上},{\hwfs 着}武生巾、黑三、白箭衣、黑马褂,{\hwfs 持}马鞭)

公子宋\hspace{20pt}晋国去求兵,要把楚邦平。俺郑国大夫公子宋是也,只因楚国屡与郑国不和,横行天下,奉主钧旨去往晋国借兵,共敌楚邦,马不宜迟火速躜行。

(\textless{}\!{\bfseries\akai 水底鱼}\!\textgreater{}{\hwfs 下})

\vspace{3pt}{\centerline{{[}{\hei 第四场}{]}}}\vspace{5pt}

({\hwfs 四}宫女{\hwfs 站门},许姬{\hwfs 上})

\setlength{\hangindent}{52pt}{许姬\hspace{30pt}【{\akai 西皮慢板}】吾主爷在渐台庆功犒赏,料不想无知辈酒后癫狂。摘盔缨奏大王未把罪降, }

(许姬{\hwfs 小座})

许姬\hspace{30pt}({\akai 接唱})岂可容乱礼法败坏纲常。

太监\hspace{30pt}({\akai 内念})大王回宫啊。

({\hwfs 四}小太监{\hwfs 上小边一字},大太监{\hwfs 引}庄王{\hwfs 上})

\setlength{\hangindent}{52pt}{庄王\hspace{30pt}【{\akai 西皮摇板}】饮罢了功臣宴神清气爽, }

(\textless{}\!{\bfseries\akai 小开门}\!\textgreater{}许姬{\hwfs 出门接驾},庄王{\hwfs 进门正面小座})

许姬\hspace{30pt}(\textless{}\!{\bfseries\akai 小开门}\!\textgreater{}{\hwfs 中}{\akai 念})大王千岁。

庄王\hspace{30pt}平身。

许姬\hspace{30pt}千千岁。

庄王\hspace{30pt}赐座。

许姬\hspace{30pt}谢座。({\hwfs 坐大边})

许姬\hspace{30pt}喂呀$\cdots{}\cdots{}$(\textless{}\!{\bfseries\akai 小开门}\!\textgreater{}{\hwfs 停})

庄王\hspace{30pt}({\akai 接唱})【{\akai 西皮摇板}】问梓童因何故面带惆怅({\akai 或}:~心意彷徨;面带彷徨)。

庄王\hspace{30pt}梓童为何面带愁容?

\setlength{\hangindent}{52pt}{
	许姬\hspace{30pt}启奏大王,大王使妾妃献觞于众臣以示敬意,牵妾之袂,王不加察,何以肃上下之礼,正男女之别也?}

\setlength{\hangindent}{52pt}{
	庄王\hspace{30pt}哈哈哈$\cdots{}\cdots{}$梓童非所知也。孤王犒赏功臣原为君臣共饮({\akai 或}:~君臣共乐),不该白昼连夜,酒后癫狂乃人情之常,孤若查而罪之,一来众臣必然心神沮丧;二者道孤君妃有陷害贤臣之意,三则外邦闻之不雅,故以酒令掩盖({\akai 或}:~以酒令遮掩)岂不三全齐美,毛皮小事({\akai 或}:~此乃小事)梓童何必挂怀。哈哈哈$\cdots{}\cdots{}$}

许姬\hspace{30pt}大王啊,

\setlength{\hangindent}{52pt}{许姬\hspace{30pt}【{\akai 西皮原板}】吾主爷有道君皇恩浩荡,沧海量宽宏度福寿绵长。似尧舜统大业千秋以上,畜鳞鱼忌流水太过清香。 }

\setlength{\hangindent}{52pt}{庄王\hspace{30pt}【{\akai 西皮慢板}】劝梓童休得要把本奏上,听孤王把前情细说端详。都只为斗越椒欺君罔上,他父子掌兵权搅乱家邦。摘去了司马印蒍贾执掌,又谁知那老儿心怀不良。孤兴兵灭陆戎狼烟扫荡,中途路竟叛逆与孤争强。杀司马搜宫院带兵对仗,楚山河险些儿被贼称王。天生来养由基英雄良将,  【{\akai 西皮二六}】只杀得他父子鼠窜獐狂。({\hwfs 立})斗越椒生得来性情倔强,清河桥比箭法老贼身亡。才能得阖朝中清平欢畅,江水静郢都宁重整朝纲。因此上在渐台论功行赏,命梓童斟御酒面带彷徨({\akai 或}:~命梓童代孤王赐过了琼浆)。又谁知霎时节狂风天降,吹熄了华堂上银烛无光。文武臣坐端然四无声响,竟有那无知徒酒后癫狂。孤若是查明了把罪来降({\akai 或}:~孤本当查明了把罪来降;或:~孤本当查明了把罪下降),怕只怕文武官意沮神伤。论国法本不该行令发放({\akai 或}:~行令放荡),也是孤做此事自有主张({\akai 或}:~也是孤一时里失了主张)。劝梓童把此事休挂心上,劝梓童把此事付与(了)汪洋。劝梓童与孤王同欢同畅,劝梓童与孤王同酌同觞。宫娥女掌银灯引归罗帐,}

(宫女{\hwfs 斜门},庄王{\hwfs 收腿})

\setlength{\hangindent}{52pt}{庄王\hspace{30pt}【{\akai 西皮摇板}】孤与你({\akai 或}:~孤和你)同偕老地久天长。 }

(庄王、许姬{\hwfs 下},宫女{\hwfs 随下}。\textless{}\!{\bfseries\akai 小锣打下}\!\textgreater{})

\vspace{3pt}{\centerline{{[}{\hei 第五场}{]}}}\vspace{5pt}

(\textless{}\!{\bfseries\akai 小锣打上}\!\textgreater{}{\hwfs 四}太监{\hwfs 引}晋成王{\hwfs 上},{\hwfs 勾}蓝三块瓦,{\hwfs 戴}黑满、草王盔,{\hwfs 着}绿蟒)

晋成王\hspace{20pt}{[}{\akai 引}{]}周室东迁,恨楚庄独霸横行。

({\hwfs 正面大座})

晋成王\hspace{20pt}({\akai 念})周室衰微中原丧,举都东迁移洛阳。群雄并起刀兵攘,楚庄横行霸一方。

\setlength{\hangindent}{52pt}{
	晋成王\hspace{20pt}孤晋侯是也。坐镇绛州,边邦平静,可恨楚庄欲霸中原,为此孤王每日操兵演将,与楚相斗雌雄,今当接报之期,设朝御览。内待展放龙棚。}

太监\hspace{30pt}展放龙棚。

先蔑\hspace{30pt}({\akai 内})呵吓。

(\textless{}\!{\bfseries\akai 四击头}\!\textgreater{}先蔑{\hwfs 上},{\hwfs 勾}黑花三块瓦,{\hwfs 着}黑满,紫金盔、翎,{\hwfs 着}黑硬靠、黑蟒,{\hwfs 持}牙笏,枪,{\akai 或}:~红三块瓦、红靠蟒)

\setlength{\hangindent}{52pt}{
	先蔑\hspace{30pt}({\akai 念})郑国请兵将,把本奏丹墀。({\hwfs 进门参拜})臣先蔑见驾,大王千岁。}

晋成王\hspace{20pt}平身。

先蔑\hspace{30pt}千千岁!

晋成王\hspace{20pt}赐座。

先蔑\hspace{30pt}谢座。({\hwfs 坐大边})

晋成王\hspace{20pt}上殿有何本奏?

先蔑\hspace{30pt}今有郑大夫公子宋前来请兵征伐楚邦,朝门候旨。

晋成王\hspace{20pt}呵呀妙哇!孤正欲伐楚,郑国使臣到来合孤意也。宣来见孤。

先蔑\hspace{30pt}领旨。(先蔑{\hwfs 立})宣郑国大夫上殿。

\setlength{\hangindent}{52pt}{
	公子宋\hspace{20pt}({\akai 内})领旨。(公子宋{\hwfs 上})为救倒悬危,求请上国兵。(宋{\hwfs 进门参拜})臣公子宋见驾,大王千岁。}

晋成王\hspace{20pt}大夫平身。

公子宋\hspace{20pt}千千岁。

晋成王\hspace{20pt}看座。

公子宋\hspace{20pt}告坐。

(公子宋\,{\hwfs 坐大边},先蔑\,{\hwfs 过去坐小边})

晋成王\hspace{20pt}来到我邦有何见谕?

\setlength{\hangindent}{52pt}{
	公子宋\hspace{20pt}只为楚王图霸要灭陈、郑二邦,臣奉主命恳请大王起兵伐楚,小国愿为后队,未知大王意下如何?}

\setlength{\hangindent}{52pt}{
	晋成王\hspace{20pt}孤久有伐楚之心,晋郑二国同体相关,大夫回去上复你主,孤王提兵伐楚,倘有不胜再来接应。}

公子宋\hspace{20pt}如此告退。感谢君金诺,同心伐楚邦。

(公子宋{\hwfs 下},先蔑{\hwfs 送},{\hwfs 回来坐大边})

\setlength{\hangindent}{52pt}{
	晋成王\hspace{20pt}先卿,孤命你为上将军元帅,统领公子凯、公子有,全军人马,兵伐楚邦,即日兴师。下殿。}

先蔑\hspace{30pt}领旨。({\akai 念})统领虎豹士,扫荡楚强兵。(先蔑{\hwfs 下})

\setlength{\hangindent}{52pt}{
	晋成王\hspace{20pt}内侍,传孤旨意,命荀林父解押粮草,军前使用。正是:~({\akai 念})两国同心争社稷,何愁海鳌不吞钩!}

(晋成王众{\hwfs 下})

\vspace{3pt}{\centerline{{[}{\hei 第六场}{]}}}\vspace{5pt}

(公子凯、公子有{\hwfs 着}硬靠、扎巾盔,{\hwfs 一}武生,{\hwfs 一}花脸,{\hwfs 双起霸})

公子凯\hspace{20pt}杀气腾腾挂铁衣,单枪匹马谁敢欺!

公子有\hspace{20pt}钢刀一举无人敌,保定大晋锦华夷。

%\raisebox{-5pt}{公子凯、\hspace{66pt}左军先锋公子凯。}\\
%公子有\hspace{20pt}\raisebox{15pt}{({\hwfs 报名})某,}右军先锋公子有。
\raisebox{0pt}[22pt][16pt]{\raisebox{8pt}{公子凯}\raisebox{-8pt}{\hspace{-32pt}{公子有}}\raisebox{0pt}{\hspace{20pt}({\hwfs 报名})某,\raisebox{8pt}{左军先锋公子凯。}\raisebox{-8pt}{\hspace{-84pt}{右军先锋公子有。}}}}

公子凯\hspace{20pt}大司马升帐发兵,你我两厢伺候!

公子有\hspace{20pt}请。

({\hwfs 四}军士{\hwfs 打上}、{\hwfs 站门}、先蔑{\hwfs 上},\textless{}\!{\bfseries\akai 点绛唇}\!\textgreater{}{\hwfs 上高台},{\hwfs 二}将{\hwfs 参})

公子凯\hspace{20pt}末将打躬。

先蔑\hspace{30pt}免,站立两厢。

先蔑\hspace{30pt}({\akai 念})凛凛雄师统貔貅,将令一出鬼神愁。号炮一声惊天地,两军对垒凭机谋。

\setlength{\hangindent}{52pt}{
	先蔑\hspace{30pt}某,晋国大司马先蔑。统领全军对敌楚王。啊众将官,此番出兵非比寻常,听本帅令下(\textless{}\!{\bfseries\akai 三枪}\!\textgreater{}~{\hwfs 牌子})}

%公子凯、\\
%公子有\hspace{20pt}\raisebox{5pt}{元帅令出如山,末将等自然奋勇当先。}
\raisebox{0pt}[22pt][16pt]{\raisebox{8pt}{公子凯}\raisebox{-8pt}{\hspace{-32pt}{公子有}}\raisebox{0pt}{\hspace{20pt}元帅令出如山,末将等自然奋勇当先。}}

先蔑\hspace{30pt}公子凯、公子有听令。

%公子凯、\\
%公子有\hspace{20pt}\raisebox{5pt}{在。}
\raisebox{0pt}[22pt][16pt]{\raisebox{8pt}{公子凯}\raisebox{-8pt}{\hspace{-32pt}{公子有}}\raisebox{0pt}{\hspace{20pt}在。}}

先蔑\hspace{30pt}命你二人打探楚兵虚实动静,不得有误。

%公子凯、\\
%公子有\hspace{20pt}\raisebox{5pt}{得令。马来!~(公子凯、公子有{\hwfs 上马下})}
\raisebox{0pt}[22pt][16pt]{\raisebox{8pt}{公子凯}\raisebox{-8pt}{\hspace{-32pt}{公子有}}\raisebox{0pt}{\hspace{20pt}得令。马来!~(公子凯、公子有{\hwfs 上马下})}}

先蔑\hspace{30pt}众将官,起兵前往。

(先蔑{\hwfs 下高台},{\hwfs 脱蟒},{\hwfs 拿枪}。\textless{}\!{\bfseries\akai 小朱奴}\!\textgreater{}{\hwfs 牌子},众{\hwfs 领}先蔑{\hwfs 下})

\vspace{3pt}{\centerline{{[}{\hei 第七场}{]}}}\vspace{5pt}

(\textless{}\!{\bfseries\akai 大锣打上}\!\textgreater{}{\hwfs 四}龙套、潘尪、伯乐、公子婴齐、公子侧{\hwfs 四}将{\hwfs 着}硬靠,{\hwfs 站门},庄王{\hwfs 上})

庄王\hspace{30pt}({\akai 引子})统领雄师,要把那晋国扫平。(庄{\hwfs 正面小座})

\setlength{\hangindent}{52pt}{
	庄王\hspace{30pt}({\akai 念})可恨晋邦礼不端,勾结陈、郑起狼烟。孤王领兵({\akai 或}:~孤王带兵)来征战,但愿齐奏凯歌还。}

\setlength{\hangindent}{52pt}{
	庄王\hspace{30pt}孤,楚王熊旅,只为图霸王室,扫荡中原。可恨晋邦反复无常,勾结陈、郑,兴兵犯境,为此命苏从、养由基护理国政,孤王亲统大兵({\akai 或}:~孤王御驾亲征),先伐晋国,后灭陈、郑。今命襄老以为前站先行,众位将军,人马可齐?}

众\hspace{40pt}俱已齐备。

庄王\hspace{30pt}吩咐文武免送,众将随营调遣,起兵前往。

潘尪\hspace{30pt}起兵前往。(\textless{}\!{\bfseries\akai 泣颜回}\!\textgreater{}{\hwfs 上马},众{\hwfs 领下})

\vspace{3pt}{\centerline{{[}{\hei 第八场}{]}}}\vspace{5pt}

(\textless{}\!{\bfseries\akai 长锤}\!\textgreater{}武小生唐狡{\hwfs 上},{\hwfs 着}大叶巾,黑箭衣、红号坎)

\setlength{\hangindent}{52pt}{唐狡\hspace{30pt}【{\akai 西皮摇板}】感受君恩未曾报,不该渐台醉酕醄。楚王宽宏量非小,摘缨罪名一笔消。 }

\setlength{\hangindent}{52pt}{唐狡\hspace{30pt}俺,唐狡,棠邑人也,父母早逝,家业凋零。投在楚王驾下当裨将。前者渐台大宴公卿,俺唐狡并无寸箭之功,蒙恩犒赏有名。不想酒后失仪,掠抱君妃暗摘盔缨,自忖性命不保;岂知君王度量宽宏,传旨众臣俱将盔缨摘去,名曰绝缨大会。想我知恩不报非丈夫也。如今晋国前来犯界,楚王御驾亲征,命襄老以为前部先锋。俺不免奔往前部,讨一差使与晋兵对敌,以报君恩也。}

\setlength{\hangindent}{52pt}{唐狡\hspace{30pt}【{\akai 西皮摇板}】楚王恩德真非小,不把国法斩儿曹。如此宽宏古来少,不辞劳碌报当朝。(唐狡{\hwfs 下}) }

\vspace{3pt}{\centerline{{[}{\hei 第九场}{]}}}\vspace{5pt}

({\hwfs 四}龙套{\hwfs 拿枪引}襄老{\hwfs 上},{[}{\akai 引子}{]},襄{\hwfs 着}狮子盔、白箭衣、黑马褂、白花开氅)

襄老\hspace{30pt}{[}{\akai 引子}{]}先行是我,我是先行。

襄老\hspace{30pt}({\akai 念})老将勇猛不可当,全凭精气逞豪强。忠心耿耿扶楚室,何日凯歌转还乡?

\setlength{\hangindent}{52pt}{
	襄老\hspace{30pt}某,襄老是也。大王征战晋国,命我以为前站先行,今日黄道正好发兵。众将官,起兵前往。}

众\hspace{40pt}啊。

唐狡\hspace{30pt}({\akai 内白})住着!

众\hspace{40pt}有人阻令。

襄老\hspace{30pt}嗯,何人竟敢阻令,传他进帐。

众\hspace{40pt}阻令者进帐。

唐狡\hspace{30pt}({\akai 内白})俺来也。(唐狡{\hwfs 上})

唐狡\hspace{30pt}欲为世上奇男子,须建人间未有功。卑将唐狡参见。

襄老\hspace{30pt}噢,原来是你。大兵正欲起行,你为何阻令?

唐狡\hspace{30pt}小将自投麾下并未建功;今主将领兵伐晋,小将愿为前站立功报国。

\setlength{\hangindent}{52pt}{
	襄老\hspace{30pt}啊,楚营多少大将,尚且全扣束身\footnote{ 李元皓{\scriptsize 君}认为此处当作``钳口、束身'',即取``钳口不言、束身自好''之意。李楠{\scriptsize 君}以为此处``全扣''当作``拳扣'',并注``拳扣,又名指虎,俗称`手撑子',古时士兵所用掌上兵器''。};你一随使将校,胆敢大言阻令,本欲取斩,犹恐出兵不利。还不下去。}

唐狡\hspace{30pt}主将差矣。

\setlength{\hangindent}{52pt}{唐狡\hspace{30pt}【{\akai 西皮散板}】唐狡虽然裨将校,胸怀韬略胆气豪。食君粮饷恩当报, }

唐狡\hspace{30pt}主将!

唐狡\hspace{30pt}({\akai 接唱})要与君王扫贼巢。

襄老\hspace{30pt}嘟,

\setlength{\hangindent}{52pt}{襄老\hspace{30pt}【{\akai 西皮散板}】我国大将有多少,遵令钳口不逞豪。小小裨将胡乱道,抗吾军令绑市曹\footnote{ 市曹,指城市中商业集中之处。古代常在这样的地方处决人犯,因此``市曹''也代指行刑场所。}。 }

\setlength{\hangindent}{52pt}{唐狡\hspace{30pt}【{\akai 西皮散板}】主将何以气量小,欺压英雄为哪条。年老出令语颠倒,焉能对垒动枪刀。交锋岂论年纪小, }

唐狡\hspace{30pt}主将,

唐狡\hspace{30pt}({\akai 接唱})定把晋国化海潮。

襄老\hspace{30pt}一派胡言,无知小卒,杀之无益,将唐狡重打四十扯下去。

({\hwfs 二}卒、唐狡{\hwfs 下},{\hwfs 内打},{\hwfs 搀上},狡{\hwfs 跪念})

唐狡\hspace{30pt}谢主将责。

襄老\hspace{30pt}念你帐下多年,留一线之情发往后队,收拾锣锅帐房。下去。

唐狡\hspace{30pt}哎呀。(唐狡{\hwfs 下})

襄老\hspace{30pt}众将官,起兵前往。

(襄老{\hwfs 脱氅},{\hwfs 拿枪上马},众{\hwfs 领下})

\vspace{3pt}{\centerline{{[}{\hei 第十场}{]}}}\vspace{5pt}

(\textless{}\!{\bfseries\akai {\akai 风入松}}\!\textgreater{}{\hwfs 头段},龙套{\hwfs 引}先蔑{\hwfs 上},{\hwfs 下场门骨牌对})

先蔑\hspace{30pt}为何不行?

众\hspace{40pt}来此楚地不远。

先蔑\hspace{30pt}列开旗门。

(\textless{}\!{\bfseries\akai {\akai 风入松}}\!\textgreater{}{\hwfs 二段},众{\hwfs 站门},先蔑{\hwfs 站中间})

\setlength{\hangindent}{52pt}{
	先蔑\hspace{30pt}众将官,楚王出兵多有奸诈,闻得前锋乃是襄老,虽不足惧,但必须人人努力,将他君臣一鼓而擒。}

众\hspace{40pt}啊!

(公子凯、公子有{\hwfs 上})

%公子凯、\\
%公子有\hspace{20pt}\raisebox{5pt}{启司马,楚兵扎颖川地方,先行襄老离此不远。}
\raisebox{0pt}[22pt][16pt]{\raisebox{8pt}{公子凯}\raisebox{-8pt}{\hspace{-32pt}{公子有}}\raisebox{0pt}{\hspace{20pt}启司马,楚兵扎颖川地方,先行襄老离此不远。}}

先蔑\hspace{30pt}啊,楚王亲自出兵,真是天助人愿。众将官,杀上前去。

(\textless{}\!{\bfseries\akai {\akai 风入松}}\!\textgreater{}{\hwfs 三段},先蔑众{\hwfs 领起},襄老众{\hwfs 抄上},襄{\hwfs 龙套下},{\hwfs 留}襄{\hwfs 大边与}先{\hwfs 架住})

襄老\hspace{30pt}呔,来将通名。

先蔑\hspace{30pt}听者,某乃晋国大司马先蔑是也,你这老将通名受死。

襄老\hspace{30pt}听者,俺乃楚王驾下前站先锋襄老是也。

先蔑\hspace{30pt}哈哈$\cdots{}\cdots{}$老弱残兵,非某对手,快教楚王自受其绑。

襄老\hspace{30pt}孺子,你嫌我老,且试演试演家伙。

(先蔑{\hwfs 打}襄老{\hwfs 败下},先众{\hwfs 追下},先{\hwfs 耍下场下})

\vspace{3pt}{\centerline{{[}{\hei 第十一场}{]}}}\vspace{5pt}

(\textless{}\!{\bfseries\akai 长锤}\!\textgreater{}众{\hwfs 引}庄王{\hwfs 上},众{\hwfs 站门})

\setlength{\hangindent}{52pt}{庄王\hspace{30pt}【{\akai 西皮摇板}】旌旗招展空中飘({\akai 或}:~空飘绕;空中绕),满营将官({\akai 或}:~将士个个)逞英豪。孤王兴兵({\akai 或}:~孤王领兵)把贼扫, }

(庄王{\hwfs 正面小座})

庄王\hspace{30pt}({\akai 接唱})旗开得胜转还朝。

(襄老\textless{}\!{\bfseries\akai 长锤}\!\textgreater{}{\hwfs 上})

\setlength{\hangindent}{52pt}{襄老\hspace{30pt}【{\akai 西皮快板}】先蔑武艺果然好,一战未交我就逃。年纪衰迈精神老,奔回大营奏根苗。 }

襄老\hspace{30pt}老臣交令。

庄王\hspace{30pt}可曾会过阵来?晋国将官哪个?

襄老\hspace{30pt}晋国元帅名叫先蔑。

庄王\hspace{30pt}呵,先蔑。(胜负如何?)

襄老\hspace{30pt}老臣出马就被他一枪,哎呀$\cdots{}\cdots{}$

庄王\hspace{30pt}(呃,)敢是带了伤了?

襄老\hspace{30pt}枪回来了。

庄王\hspace{30pt}敢是败了?({\akai 或}:~哦,败了。)

襄老\hspace{30pt}败了。

庄王\hspace{30pt}后营憩息。({\akai 或}:~老将军后营歇息。)

襄老\hspace{30pt}谢大王。(襄老{\hwfs 下})

庄王\hspace{30pt}(且住,)先蔑老儿十分骁勇,必须孤王亲自会他,众将官,奋勇当先。

(众{\hwfs 领起},先蔑众{\hwfs 上},{\hwfs 二龙出水会阵})

先蔑\hspace{30pt}呔,来者敢是楚王?

庄王\hspace{30pt}正是。来者可是先蔑?

先蔑\hspace{30pt}然。

庄王\hspace{30pt}先蔑,楚邦({\akai 或}:~孤王)有何亏负你国,无故兴兵是何理也?

先蔑\hspace{30pt}昏庄!你横行天下,某奉晋君旨意,领兵扫荡。还不束手受绑?

庄王\hspace{30pt}(贼子)住口!众将官排开阵势者。

\setlength{\hangindent}{52pt}{庄王\hspace{30pt}【{\akai 西皮导板}】叫三军与孤战鼓操, }

(龙套{\hwfs 钻烟筒},{\hwfs 一合两合拉开唱})

\setlength{\hangindent}{52pt}{庄王\hspace{30pt}【{\akai 西皮快板}】先蔑老儿听根苗:~列国早已({\akai 或}:~各国俱已)结盟好,同心协力保周朝。你主不该把孤藐,平地生波为哪条。陆戎小国被孤扫,陈、郑不敢犯边辽。({\akai 或}:~陆戎小国被孤扫,陈、郑不敢犯边辽。你主若是行无道,定把晋国永勾销。或:~陈、郑二邦写降表,陆戎不敢犯边辽。你主不该行无道,无故兴兵为哪条。或:~你主不该行无道,无故兴兵为哪条。陆戎小国何足道,陈、郑不敢犯边辽。)劝你马前写降表({\akai 或}:~归顺好),免得尸首马后抛。 }

\setlength{\hangindent}{52pt}{先蔑\hspace{30pt}【{\akai 西皮摇板}】大晋明君存仁道,【{\footnotesize 转}{\akai 西皮快板}】各守疆土见识高。你图中原行霸道,称孤道寡犯天条。屡次兴兵各国扫,横行天下夺城壕。两军对垒战场道,各显奇能逞英豪。}

\setlength{\hangindent}{52pt}{庄王\hspace{30pt}【{\akai 西皮摇板}】好言说尔说不倒({\akai 或}:~好话说尔说不倒)。 }

\setlength{\hangindent}{52pt}{先蔑\hspace{30pt}【{\akai 西皮摇板}】管教昏王丧荒郊。 }

\setlength{\hangindent}{52pt}{庄王\hspace{30pt}【{\akai 西皮摇板}】三军摆开({\akai 或}:~三军排开)长蛇道。 }

(先蔑{\bfseries\akai 扫一句},{\hwfs 开打},{\hwfs 钻烟筒},{\hwfs 打枪剑},庄王{\hwfs 败下},{\hwfs 上}楚将{\hwfs 一二败下},{\hwfs 追过场},先{\hwfs 耍下场下})

\vspace{3pt}{\centerline{{[}{\hei 第十二场}{]}}}\vspace{5pt}

\setlength{\hangindent}{52pt}{
	庄王\hspace{30pt}({\hwfs 上唱})【{\akai 西皮散板}】一霎时玉石焚金山颓倒,闯东西、奔南北生路哪条({\akai 或}:~闯东西、奔南北生路何条)。}

\setlength{\hangindent}{52pt}{
	庄王\hspace{30pt}({\akai 念})且住!先蔑老儿十分骁勇,连败孤王数员大将,呜哙呀,事到如今孤王身边连一个保驾的臣子都没有了,看将起来真是成了孤家了。({\akai 或}:~连挑孤家数员上将,哎呀,孤王如今身边连一个保驾的臣子都没有了,哎呀,看将起来,真是孤家了。)}

先蔑\hspace{30pt}({\akai 内白})哪里走!

庄王\hspace{30pt}哎呀来了。

(先蔑{\hwfs 追上打}庄王{\hwfs 下},楚将{\hwfs 三四上},{\hwfs 败下},先{\hwfs 耍下场追下})

\vspace{3pt}{\centerline{{[}{\hei 第十三场}{]}}}\vspace{5pt}

(唐狡甩发、黑箭衣、{\hwfs 背}单刀、{\hwfs 手拿}梢子帽,{\hwfs 上唱})

\setlength{\hangindent}{52pt}{唐狡\hspace{30pt}【{\akai 西皮摇板}】只望立功把恩报,主将不用枉心劳。 }

\setlength{\hangindent}{52pt}{
	唐狡\hspace{30pt}({\akai 念})俺,唐狡。我主兵伐晋邦,只望先锋面前讨一前站,不想反被罚为小卒,收拾锣锅帐房与老卒同行。好不丧气人也。(\textless{}\!{\bfseries\akai 鼓架子}\!\textgreater{})且住!耳听喊杀之声,待俺登高一望。}

({\hwfs 上桌子望}。庄王{\hwfs 领}楚众{\hwfs 上},晋众{\hwfs 压队追上},庄众{\hwfs 下},先蔑众{\hwfs 追下},唐狡{\hwfs 跳下桌})

\setlength{\hangindent}{52pt}{
	唐狡\hspace{30pt}且住!前面败的我主,后面追的先蔑。此时不救,待等何时,呔,先蔑休要逞强,唐老爷来也。({\hwfs 扔帽},{\hwfs 拔刀},{\hwfs 耍下})}

\vspace{3pt}{\centerline{{[}{\hei 第十四场}{]}}}\vspace{5pt}

(庄王{\hwfs 上},先蔑{\hwfs 追上},{\hwfs 打}庄{\hwfs 抢背},唐狡{\hwfs 上挑开},襄{\hwfs 下场门上},{\hwfs 搀}庄{\hwfs 上桌子},狡{\hwfs 打}先{\hwfs 下},狡{\hwfs 单刀耍下场},庄{\hwfs 桌上云手踢腿},{\hwfs 左右一}、{\hwfs 二外望},{\hwfs 比势摸颈},{\hwfs 唱})

\setlength{\hangindent}{52pt}{庄王\hspace{30pt}【{\akai 西皮散板}】适才被贼挑下马,忽然间闪出了年少(的)娃。满营将官俱个在孤的功劳簿上跨, }

襄老\hspace{30pt}老臣我在其内。

\setlength{\hangindent}{52pt}{庄王\hspace{30pt}【{\akai 西皮散板}】这一员小将孤就不认识他。 }

襄老\hspace{30pt}您猜我呐({\hwfs 唱})【{\akai 西皮散板}】我也不认识他。

\setlength{\hangindent}{52pt}{庄王\hspace{30pt}【{\akai 西皮散板}】看起来是孤王(拍腰)洪福大,天赐良将把贼拿。 }

(先蔑{\hwfs 上})

\setlength{\hangindent}{52pt}{先蔑\hspace{30pt}【{\akai 西皮摇板}】昏王被某挑下马,猛然来了年少娃。手使钢刀迎面扎,某家不曾提防他。落地梅花耍一耍, }

(先蔑{\hwfs 提枪花大边台口落地梅花势},唐狡{\hwfs 换枪上勒马背枪单腿站})

先蔑\hspace{30pt}娃娃为何不敢前进?

唐狡\hspace{30pt}你用落地梅花暗施诡计非英雄也。

\setlength{\hangindent}{52pt}{先蔑\hspace{30pt}【{\akai 西皮摇板}】倒教娃娃耻笑咱,楚国兵将全不怕,偏遇无名小冤家。扳鞍踏镫把马跨, }

\setlength{\hangindent}{52pt}{唐狡\hspace{30pt}【{\akai 西皮摇板}】老爷擒你献皇家。抖擞精神催战马, }

\setlength{\hangindent}{52pt}{先蔑\hspace{30pt}【{\akai 西皮摇板}】这枪刺得某两眼花。多少将官丧马下,何惧小小井底蛙。 }

(唐狡{\hwfs 打}先蔑{\hwfs 下},狡{\hwfs 耍枪下场},{\hwfs 下})

\setlength{\hangindent}{52pt}{庄王\hspace{30pt}【{\akai 西皮散板}】气宇轩昂武艺佳({\akai 或}:~小将生来实可夸),能征惯战果不差。但愿先蔑早拿下,千刀万剐不饶他(\textless{}\!{\bfseries\akai 三锣}\!\textgreater{})。 }

(晋楚{\hwfs 四}将{\hwfs 开打},先、狡{\hwfs 两边上},{\hwfs 漫对方}将{\hwfs 头},唐狡{\hwfs 擒}先蔑{\hwfs 下},庄王、襄老{\hwfs 下桌椅},{\hwfs 趴地},襄{\hwfs 扶}庄{\hwfs 起},{\bfseries\hwfs 不要有逗笑动作},{\hwfs 望})

庄王\hspace{30pt}那先蔑呢?

襄老\hspace{30pt}被小将军擒住了。

庄王\hspace{30pt}你可曾看得清楚?({\akai 或}:~呃,老将军你可曾看见?)

襄老\hspace{30pt}没错,我戴着花镜呐!

庄王\hspace{30pt}这就好了,与孤带马。({\akai 或}:~哦,拿住了。呃,带马带马。)

襄老\hspace{30pt}被小将军骑了去了。

庄王\hspace{30pt}骑你的马。({\akai 或}:~呃,带你的马。)

襄老\hspace{30pt}还没有安尾巴呢!

庄王\hspace{30pt}孤王怎样回营呢?

襄老\hspace{30pt}只好开步走了。

庄王\hspace{30pt}如此摆驾。

襄老\hspace{30pt}咦。

(襄老{\hwfs 领}庄王{\hwfs 下})

\vspace{3pt}{\centerline{{[}{\hei 第十五场}{]}}}\vspace{5pt}

(\textless{}\!{\bfseries\akai 牌子}\!\textgreater{},庄王众{\hwfs 上站门},庄{\hwfs 正面大座},襄老{\hwfs 上报})

襄老\hspace{30pt}先蔑擒到。

庄王\hspace{30pt}押上帐来。({\akai 或}:~带先蔑。)

(襄老{\hwfs 拉}先蔑{\hwfs 手杻上},先{\hwfs 在}襄{\hwfs 后踹}襄,襄{\hwfs 趴下},{\hwfs 再起来})

\setlength{\hangindent}{52pt}{先蔑\hspace{30pt}【{\akai 西皮摇板}】龙入铁网难撑架, }

\setlength{\hangindent}{52pt}{先蔑\hspace{30pt}【{\akai 西皮快板}】虎落平阳被擒拿。列国英雄也有咱,遇这无名小冤家。某既被擒凭刀剐,落得忠名扬天涯。将身站立大帐下,({\hwfs 进帐}) }

\setlength{\hangindent}{52pt}{先蔑\hspace{30pt}【{\akai 西皮摇板}】看他把某怎开发。 }

\setlength{\hangindent}{52pt}{庄王\hspace{30pt}【{\akai 西皮摇板}】孤王帐中用目洒, }

\setlength{\hangindent}{52pt}{庄王\hspace{30pt}【{\akai 西皮快板}】先蔑老儿带锁枷({\akai 或}:~披锁枷)。阵前何等威风大,运败时衰被孤拿。 }

庄王\hspace{30pt}({\akai 白})先蔑。(孤王有何亏负你国,何故兴兵犯界,是何理也?)

先蔑\hspace{30pt}昏王。

({\hwfs 踢桌},庄王{\hwfs 站躲},{\hwfs 再坐下})

\setlength{\hangindent}{52pt}{
	庄王\hspace{30pt}还是如此厉害,先蔑你在两军阵前何等威风,如今被擒帐下,有何话讲?({\akai 或}:~呜哙呀,你在阵前何等威风,何等煞气,今日被擒,有何话讲?)}

先蔑\hspace{30pt}昏王何必多言。

\setlength{\hangindent}{52pt}{
	庄王\hspace{30pt}孤王何曾亏负你国,无故兴兵犯界,先斩你这老头,再擒晋侯与他辩理,来,将先蔑推出斩了。({\akai 或}:~呜哙呀,还是这等的厉害,哼,先斩你这个老头,再擒晋侯与他辩理。来,将先蔑推出斩了。)}

(襄老{\hwfs 拉}先蔑{\hwfs 下},\textless{}\!{\bfseries\akai 五锣三鼓}\!\textgreater{},襄{\hwfs 上报})

襄老\hspace{30pt}先蔑斩首,小将回营。

庄王\hspace{30pt}有请。

(庄王{\hwfs 出位}。唐狡\textless{}\!{\bfseries\akai 紧锤}\!\textgreater{}{\hwfs 上},{\hwfs 下马}。庄{\hwfs 拉}狡{\hwfs 换边},庄{\hwfs 小边}、狡{\hwfs 大边台口},襄老{\hwfs 托}狡{\hwfs 腿}{\akai 或}{\hwfs 可扳}狡{\hwfs 朝天镫})

\setlength{\hangindent}{52pt}{庄王\hspace{30pt}【{\akai 西皮快板}】一见小将到帐下,功劳({\akai 或}:~战伐)魁首第一家。孤将龙衣来脱下, }

({\hwfs 吹打}\textless{}\!{\bfseries\akai 合龙}\~\textgreater{},唐狡{\hwfs 穿}庄王{\hwfs 黄马褂}、{\hwfs 戴}武生巾,庄{\hwfs 穿}开氅,庄{\hwfs 收腿})

\setlength{\hangindent}{52pt}{庄王\hspace{30pt}【{\akai 西皮快板}】得胜御酒({\akai 或}:~得胜琼浆;功劳簿上)把功加。({\akai 或}:~得胜御酒付卿拿。) }

({\hwfs 递酒},唐狡{\hwfs 接酒谢天地},庄王{\hwfs 正面小座},狡{\hwfs 参拜})

唐狡\hspace{30pt}参见大王,救驾来迟大王恕罪。

庄王\hspace{30pt}平身。({\akai 或}:~罢了。)

唐狡\hspace{30pt}谢大王。

庄王\hspace{30pt}赐座。({\akai 或}:~一旁坐下。)

唐狡\hspace{30pt}谢座。

(唐狡{\hwfs 坐大边},襄老{\hwfs 站小边})

庄王\hspace{30pt}小将军哪里人氏,姓甚名谁,(孤王有何恩惠于你,)竟敢一人前来救驾。

唐狡\hspace{30pt}小臣唐狡,棠邑人氏。大王待小臣有天高地厚之恩,特来救驾。

庄王\hspace{30pt}啊,孤王有何恩惠于你({\akai 或}:~哦,孤有何恩典于你)?

唐狡\hspace{30pt}大王可记得绝缨会之故否?

庄王\hspace{30pt}哦,不必深言({\bfseries\hwfs 不要背供})。你今(日)救驾有功,封为上军副帅。(同孤扫晋。)

唐狡\hspace{30pt}谢大王。

(襄老``哎呀''{\hwfs 蹲下})

庄王\hspace{30pt}老将军为何如此? ~(哎呀,老将军你这是怎么样了?)

\setlength{\hangindent}{52pt}{
	襄老\hspace{30pt}大王有所不知,唐将军乃老臣帐下兵卒,老臣曾将他重责,不料他勤王救驾封官,上军副帅,正管我这个前站先行,老臣我这回可真玩不开了。}

\setlength{\hangindent}{52pt}{
	庄王\hspace{30pt}(哦,)原来如此,这样吧,从今以后将老将军拨在唐将军帐中({\akai 或}:~将军帐下),倘有差迟,(呃,)按军令施行如何?}

襄老\hspace{30pt}哎哟。

\setlength{\hangindent}{52pt}{
	唐狡\hspace{30pt}啊老将军,为将者当以军法为重。唐狡自应以德报德,以直报怨。焉有记恨之理,老将军何必挂怀?}

襄老\hspace{30pt}将军乃奇男子也。

庄王\hspace{30pt}二卿为孤不惜\footnote{ 段公平君建议也可作``不恤''。}身躯,岂能怨恨,后帐摆宴与二卿解和贺功。

(庄王{\hwfs 下},唐狡、襄老{\hwfs 互让下},众{\hwfs 下})
}
