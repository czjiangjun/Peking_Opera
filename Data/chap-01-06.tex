\newpage

\subsubsection{\large\hei {摘缨会}}
\addcontentsline{toc}{subsection}{\hei 摘缨会}
\hangafter=1                   %2. 设置从第1⾏之后开始悬挂缩进  %
\setlength{\parindent}{0pt}{ 
{\centerline{{[}{\hei 第一场}{]}}}\vspace{5pt}

(打朝众官上\textless{}\!{\bfseries\akai 点绛唇}\!\textgreater{}报名)

众报名

大司马申无畏(台顶,红三块瓦,黪满)。上大夫苏从(夫子盔,黪三)。下大夫虞邱(荷叶盔,黑三)。上将军熊负羁(帅盔,黑三)。车左将军潘尪(踏镫,紫三块瓦,黑满)。车右将军乐伯(倒缨盔,武生)。左殿将军公子婴齐(虎头盔,武生)。右殿将军公子侧(虎头盔、元宝脸、黑一字)。左队先锋养由基(硬扎巾、大额子、武生)。右队先锋襄老(狮子盔、老丑)。(左右分站)

申无畏\hspace{20pt}~

列公请了。

众\hspace{40pt}~

请了。

申无畏\hspace{20pt}~

大王设朝,两厢伺候。

众\hspace{40pt}~

请。

(左右分下,\textless{}\!{\bfseries\akai 大锣打下}\!\textgreater{}接\textless{}\!{\bfseries\akai 打上}\!\textgreater{})

\vspace{3pt}{\centerline{{[}{\hei 第二场}{]}}}\vspace{5pt}

(四太监站门、大太监、庄王\textless{}\!{\bfseries\akai 大锣打上}\!\textgreater{},庄王九龙冠,黄开氅)

庄王\hspace{30pt}~

(引)征战戎夷,立功勋,息卷旌旗(另本:

盛世兴隆,图霸业,一统江洪)。

(正面小座念)

庄王

为征陆浑({\akai 或}: 为征陆戎)动兵机,越椒无理({\akai 或}: 越椒无故)把孤欺。幸有众卿逞雄力,才得凯歌马停蹄。(另本:

楚国逞强有数秋,龙争虎斗统貔貅。图霸中原孤为首,山河一统乐无忧。)孤,熊旅。坐镇荆州,承先王基业,复立楚国,三年以来号令未申。大夫申无畏、苏从直谏,重整先王原例,是孤欲图中原。适逢川贼陆戎兄妹造反,孤御驾亲征,在朝摘去斗越椒兵权印信({\akai 或}: 兵权信印),不想奸贼妒贤嫉能({\akai 或}: 嫉贤妒能),司马蒍贾满门尽丧,斗贼兴兵劫驾,多亏养由基灭贼除患,保孤回朝。寡人今日设宴渐台,犒赏群臣({\akai 或}: 犒赏三军),以表孤(王)爱戴贤臣之意。内侍,与孤传旨,不论裨、副牙将大小官员({\akai 或}: 不论偏、副牙将,大小将官),满朝文武齐上渐台领宴共赏。

太监\hspace{30pt}~

大王有旨,不论裨副牙将大小官员,满朝文武齐上渐台领宴共赏。

内众\hspace{30pt}~

领旨。

(两边分上,四牙将左右跟上,牙将软靠,唐狡红靠、扎巾盔,站大边外角)

众\hspace{40pt}~

臣等参见大王。

庄王\hspace{30pt}~

众卿平身。

众\hspace{40pt}~

千千岁。

(分站左右)

申无畏\hspace{20pt}~

宣臣等上殿有何旨意?

庄王\hspace{30pt}~

孤今得胜还朝({\akai 或}: 孤王今日得胜回朝),在渐台设宴,与众卿同饮。

申无畏\hspace{20pt}~

臣等领旨。

庄王\hspace{30pt}~

内侍。

太监\hspace{30pt}~

有。

庄王\hspace{30pt}~

酒宴可曾齐备?

太监\hspace{30pt}~

俱已齐备。

庄王\hspace{30pt}~

带路渐台。

太监\hspace{30pt}~

带路渐台呀。

\vspace{3pt}{\centerline{{[}{{[}连场第二场}{]}}}\vspace{5pt}

(``{吹打}''众转场,庄王最后下,众又领上上渐台,庄上正面小帐子高台大座,大太监大边椅子上,众分坐两边大座,唐狡坐大边最外椅,牌子停)

庄王\hspace{30pt}~

看宴。(众卿请。)

众\hspace{40pt}~

千岁请。

(\textless{}\!{\bfseries\akai 玉芙蓉}\!\textgreater{}前段,饮酒,饮完)

庄王\hspace{30pt}~

内侍。

太监\hspace{30pt}~

有。

庄王\hspace{30pt}~

宣许娘娘上渐台。

(太监下椅子)

太监\hspace{30pt}~

大王有旨,宣许娘娘上渐台。

内众\hspace{30pt}~

领旨。

(西皮\textless{}\!{\bfseries\akai 小开门}\!\textgreater{},四宫女引许姬上念)

许姬\hspace{30pt}~

深感隆恩君锡宠,轻移莲步上渐台。

(宫女挖门站高台后侧两边,许姬参拜)

许姬\hspace{30pt}~

妾妃见驾,大王千岁。

庄王\hspace{30pt}~

梓童平身。

许姬\hspace{30pt}~

千千岁。(许姬上小边椅)宣妾妃上渐台不知有何旨意?

庄王

孤王今日设宴犒赏功臣,梓童在众臣({\akai 或}: 在众将)席前行酒一巡,以表孤王爱戴贤臣之意。

许姬\hspace{30pt}~

妾闻男女不渎,何况君臣?

申无畏、苏从 臣等叩沐恩宠,何敢再劳娘娘赐酒。

庄王

孤王今日设宴,原为君臣共欢,众卿不必谦逊({\akai 或}: 不必过谦),孤王言重如山,岂能反悔,梓童巡酒。

许姬\hspace{30pt}~

领旨。

(许下椅唱【{\akai 西皮二六}】,太监随下端壶)

许姬

\setlength{\hangindent}{60pt}{ 【{\akai 西皮二六}】征夷戎立功勋天时顺应,诛越椒回朝转河晏海清。捧香醪献君王虔诚恭敬(给庄王斟酒),愿吾主乐陶陶福寿康宁。我这里奉王命把酒侍饮(给大边臣斟),在宴前依次序来敬杯巡(给小边臣斟)。似这等有道君万民之幸(给小边牙将斟),辅国家秉忠义万世留名(给唐狡连斟三杯)。耳听得风声响阶檐震动(壶递太监), }

许姬\hspace{30pt}~

\setlength{\hangindent}{60pt}{ 【{\akai 西皮散板}】霎时间阖台上黑暗不明。 }

(\textless{}\!{\bfseries\akai 乱锤}\!\textgreater{},唐狡,许姬推磨,许摘唐狡缨,上小边椅与庄王语、递缨即面牌,狡坐下)

申无畏、苏从 快快点烛。

庄王\hspace{30pt}~

慢、慢$\cdots{}\cdots{}$掌灯。

(众应)

庄王\hspace{30pt}~

众卿。

众\hspace{40pt}~

大王。

庄王

今日此宴原为君臣共饮({\akai 或}: 君臣共欢),诸事无忌,孤王出一酒令,众卿可将盔缨摘下,丢过席前,作一绝缨大会,但有不遵者,孤王不以法度治之,只罚酒三巨觥。

众\hspace{40pt}~

臣等遵旨。

(\textless{}\!{\bfseries\akai 冲头}\!\textgreater{}中太监拿盘两边收缨,锣中唐狡摸地找缨不见仍坐下,太监收完端盘上椅放盘在桌上)

庄王\hspace{30pt}~

可曾摘齐?

众\hspace{40pt}~

俱已摘齐。

庄王\hspace{30pt}~

吩咐掌灯。

(太监、众应)

(庄王左右两望)

庄王\hspace{30pt}~

哈$\cdots{}\cdots{}$许姬回宫。

(\textless{}\!{\bfseries\akai 小开门}\!\textgreater{},许姬下椅,拜辞,宫女领下)

许姬\hspace{30pt}~

\setlength{\hangindent}{60pt}{ 【{\akai 西皮散板}】暗中牵袂醉中情,玉手如风已绝缨。 }

(许姬下)

庄王\hspace{30pt}~

众卿再饮一回({\akai 或}: 再饮几回)。

众\hspace{40pt}~

臣等酒足,请驾回宫。

庄王\hspace{30pt}~

退班。

(\textless{}\!{\bfseries\akai 玉芙蓉}\!{\bfseries\akai 合头}\!\textgreater{},庄王众窝下。留唐狡,狡比势、摸颈,怕介,下)

\vspace{3pt}{\centerline{{[}{\hei 第三场}{]}}}\vspace{5pt}

(\textless{}\!{\bfseries\akai 水底鱼}\!\textgreater{}公子宋上,着武生巾、黑三、白箭衣、黑马褂,持马鞭)

公子宋

晋国去求兵,要把楚邦平。俺郑国大夫公子宋是也,只因楚国屡与郑国不和,横行天下,奉主钧旨去往晋国借兵,共敌楚邦,马不宜迟火速躜行。

(\textless{}\!{\bfseries\akai 水底鱼}\!\textgreater{}下)

\vspace{3pt}{\centerline{{[}{\hei 第四场}{]}}}\vspace{5pt}

(四宫女站门,许姬上)

许姬

\setlength{\hangindent}{60pt}{ 【{\akai 西皮慢板}】吾主爷在渐台庆功犒赏,料不想无知辈酒后癫狂。摘盔缨奏大王未把罪降, }

(许姬小座)

许姬\hspace{30pt}~

({\akai 接唱})岂可容乱礼法败坏纲常。

太监\hspace{30pt}~

({\akai 内}念)大王回宫啊。

(四小太监上小边一字,大太监引庄王上)

庄王\hspace{30pt}~

\setlength{\hangindent}{60pt}{ 【{\akai 西皮摇板}】饮罢了功臣宴神清气爽, }

(\textless{}\!{\bfseries\akai 小开门}\!\textgreater{}许姬出门接驾,庄王进门正面小座)

许姬\hspace{30pt}~

(\textless{}\!{\bfseries\akai 小开门}\!\textgreater{}中念)大王千岁。

庄王\hspace{30pt}~

平身。

许姬\hspace{30pt}~

千千岁。

庄王\hspace{30pt}~

赐座。

许姬\hspace{30pt}~

谢座。(坐大边)

许姬\hspace{30pt}~

喂呀$\cdots{}\cdots{}$(\textless{}\!{\bfseries\akai 小开门}\!\textgreater{}停)

庄王\hspace{30pt}~

({\akai 接唱})【{\akai 西皮摇板}】问梓童因何故面带惆怅({\akai 或}: 心意彷徨;面带彷徨)。

庄王\hspace{30pt}~

梓童为何面带愁容?

许姬

启奏大王,大王使妾妃献觞于众臣以示敬意,牵妾之袂,王不加察,何以肃上下之礼,正男女之别也?

庄王

哈哈哈$\cdots{}\cdots{}$梓童非所知也。孤王犒赏功臣原为君臣共饮({\akai 或}: 君臣共乐),不该白昼连夜,酒后癫狂乃人情之常,孤若查而罪之,一来众臣必然心神沮丧;二者道孤君妃有陷害贤臣之意,三则外邦闻之不雅,故以酒令掩盖({\akai 或}: 以酒令遮掩)岂不三全齐美,毛皮小事({\akai 或}: 此乃小事)梓童何必挂怀。哈哈哈$\cdots{}\cdots{}$

许姬\hspace{30pt}~

大王啊,

许姬

\setlength{\hangindent}{60pt}{ 【{\akai 西皮原板}】吾主爷有道君皇恩浩荡,沧海量宽宏度福寿绵长。似尧舜统大业千秋以上,畜鳞鱼忌流水太过清香。 }

庄王

\setlength{\hangindent}{60pt}{ 【{\akai 西皮慢板}】劝梓童休得要把本奏上,听孤王把前情细说端详。都只为斗越椒欺君罔上,他父子掌兵权搅乱家邦。摘去了司马印蒍贾执掌,又谁知那老儿心怀不良。孤兴兵灭陆戎狼烟扫荡,中途路竟叛逆与孤争强。杀司马搜宫院带兵对仗,楚山河险些儿被贼称王。天生来养由基英雄良将, }

【{\akai 西皮二六}】只杀得他父子鼠窜獐狂。(立)斗越椒生得来性情倔强,清河桥比箭法老贼身亡。才能得阖朝中清平欢畅,江水静郢都宁重整朝纲。因此上在渐台论功行赏,命梓童斟御酒面带彷徨({\akai 或}: 命梓童代孤王赐过了琼浆)。又谁知霎时节狂风天降,吹熄了华堂上银烛无光。文武臣坐端然四无声响,竟有那无知徒酒后癫狂。孤若是查明了把罪来降({\akai 或}: 孤本当查明了把罪来降;或: 孤本当查明了把罪下降),怕只怕文武官意沮神伤。论国法本不该行令发放({\akai 或}: 行令放荡),也是孤做此事自有主张({\akai 或}: 也是孤一时里失了主张)。劝梓童把此事休挂心上,劝梓童把此事付与(了)汪洋。劝梓童与孤王同欢同畅,劝梓童与孤王同酌同觞。宫娥女掌银灯引归罗帐,

(宫女斜门,庄王收腿)

庄王\hspace{30pt}~

\setlength{\hangindent}{60pt}{ 【{\akai 西皮摇板}】孤与你({\akai 或}: 孤和你)同偕老地久天长。 }

(庄王、许姬下,宫女随下。\textless{}\!{\bfseries\akai 小锣打下}\!\textgreater{})

\vspace{3pt}{\centerline{{[}{\hei 第五场}{]}}}\vspace{5pt}

(\textless{}\!{\bfseries\akai 小锣打上}\!\textgreater{}四太监引晋成王上,勾蓝三块瓦,戴黑满、草王盔,着绿蟒)

晋成王\hspace{20pt}~

{[}引{]}周室东迁,恨楚庄独霸横行。

(正面大座)

晋成王

({\akai 念})周室衰微中原丧,举都东迁移洛阳。群雄并起刀兵攘,楚庄横行霸一方。

晋成王

孤晋侯是也。坐镇绛州,边邦平静,可恨楚庄欲霸中原,为此孤王每日操兵演将,与楚相斗雌雄,今当接报之期,设朝御览。内待展放龙棚。

太监\hspace{30pt}~

展放龙棚。

先蔑\hspace{30pt}~

({\akai 内})呵吓。

(\textless{}\!{\bfseries\akai 四击头}\!\textgreater{}先蔑上,勾黑花三块瓦,着黑满,紫金盔、翎,着黑硬靠、黑蟒,持牙笏,枪,或红三块瓦、红靠蟒)

先蔑\hspace{30pt}~

({\akai 念})郑国请兵将,把本奏丹墀。(进门参拜)臣先蔑见驾,大王千岁。

晋成王\hspace{20pt}~

平身。

先蔑\hspace{30pt}~

千千岁!

晋成王\hspace{20pt}~

赐座。

先蔑\hspace{30pt}~

谢座。(坐大边)

晋成王\hspace{20pt}~

上殿有何本奏?

先蔑\hspace{30pt}~

今有郑大夫公子宋前来请兵征伐楚邦,朝门候旨。

晋成王\hspace{20pt}~

呵呀妙哇!孤正欲伐楚,郑国使臣到来合孤意也。宣来见孤。

先蔑\hspace{30pt}~

领旨。(先蔑立)宣郑国大夫上殿。

公子宋

({\akai 内})领旨。(公子宋上)为救倒悬危,求请上国兵。(宋进门参拜)臣公子宋见驾,大王千岁。

晋成王\hspace{20pt}~

大夫平身。

公子宋\hspace{20pt}~

千千岁。

晋成王\hspace{20pt}~

看座。

公子宋\hspace{20pt}~

告坐。

(公子宋坐大边,先蔑过去坐小边)

晋成王\hspace{20pt}~

来到我邦有何见谕?

公子宋

只为楚王图霸要灭陈、郑二邦,臣奉主命恳请大王起兵伐楚,小国愿为后队,未知大王意下如何?

晋成王

孤久有伐楚之心,晋郑二国同体相关,大夫回去上复你主,孤王提兵伐楚,倘有不胜再来接应。

公子宋\hspace{20pt}~

如此告退。感谢君金诺,同心伐楚邦。

(公子宋下,先蔑送,回来坐大边)

晋成王

先卿,孤命你为上将军元帅,统领公子凯、公子有,全军人马,兵伐楚邦,即日兴师。下殿。

先蔑\hspace{30pt}~

领旨。({\akai 念})统领虎豹士,扫荡楚强兵。(先蔑下)

晋成王

内侍,传孤旨意,命荀林父解押粮草,军前使用。正是: ({\akai 念})两国同心争社稷,何愁海鳌不吞钩!

(晋成王众下)

\vspace{3pt}{\centerline{{[}{\hei 第六场}{]}}}\vspace{5pt}

(公子凯、公子有着硬靠、扎巾盔,一武生,一花脸,双起霸)

公子凯\hspace{20pt}~

杀气腾腾挂铁衣,单枪匹马谁敢欺!

公子有\hspace{20pt}~

钢刀一举无人敌,保定大晋锦华夷。

公子凯、公子有 (报名)某,左军先锋公子凯。右军先锋公子有。

公子凯\hspace{20pt}~

大司马升帐发兵,你我两厢伺候!

公子有\hspace{20pt}~

请。

(四军士打上、站门、先蔑上,\textless{}\!{\bfseries\akai 点绛唇}\!\textgreater{}上高台,二将参)

公子凯\hspace{20pt}~

末将打躬。

先蔑\hspace{30pt}~

免,站立两厢。

先蔑

({\akai 念})凛凛雄师统貔貅,将令一出鬼神愁。号炮一声惊天地,两军对垒凭机谋。

先蔑

某,晋国大司马先蔑。统领全军对敌楚王。啊众将官,此番出兵非比寻常,听本帅令下(\textless{}\!{\bfseries\akai 三枪}\!\textgreater{}牌子)

公子凯、公子有 元帅令出如山,末将等自然奋勇当先。

先蔑\hspace{30pt}~

公子凯、公子有听令。

公子凯、公子有 在。

先蔑\hspace{30pt}~

命你二人打探楚兵虚实动静,不得有误。

公子凯、公子有 得令。马来!( 公子凯、公子有上马下)

先蔑\hspace{30pt}~

众将官,起兵前往。

(先蔑下高台,脱蟒,拿枪。\textless{}小朱奴\textgreater{}牌子,众领先蔑下)

\vspace{3pt}{\centerline{{[}{\hei 第七场}{]}}}\vspace{5pt}

(\textless{}\!{\bfseries\akai 大锣打上}\!\textgreater{}四龙套、潘尪、伯乐、公子婴齐、公子侧四将着硬靠,站门,庄王上)

庄王\hspace{30pt}~

({\akai 引子})统领雄师,要把那晋国扫平。(庄正面小座)

庄王

({\akai 念})可恨晋邦礼不端,勾结陈、郑起狼烟。孤王领兵({\akai 或}: 孤王带兵)来征战,但愿齐奏凯歌还。

庄王

孤,楚王熊旅,只为图霸王室,扫荡中原。可恨晋邦反复无常,勾结陈、郑,兴兵犯境,为此命苏从、养由基护理国政,孤王亲统大兵({\akai 或}: 孤王御驾亲征),先伐晋国,后灭陈、郑。今命襄老以为前站先行,众位将军,人马可齐?

众\hspace{40pt}~

俱已齐备。

庄王\hspace{30pt}~

吩咐文武免送,众将随营调遣,起兵前往。

潘尪\hspace{30pt}~

起兵前往。(\textless{}\!{\bfseries\akai 泣颜回}\!\textgreater{}上马,众领下)

\vspace{3pt}{\centerline{{[}{\hei 第八场}{]}}}\vspace{5pt}

(\textless{}\!{\bfseries\akai 长锤}\!\textgreater{}武小生唐狡上,着大叶巾,黑箭衣、红号坎)

唐狡

\setlength{\hangindent}{60pt}{ 【{\akai 西皮摇板}】感受君恩未曾报,不该渐台醉酕醄。楚王宽宏量非小,摘缨罪名一笔消。 }

唐狡

俺,唐狡,棠邑人也,父母早逝,家业凋零。投在楚王驾下当裨将。前者渐台大宴公卿,俺唐狡并无寸箭之功,蒙恩犒赏有名。不想酒后失仪,掠抱君妃暗摘盔缨,自忖性命不保;岂知君王度量宽宏,传旨众臣俱将盔缨摘去,名曰绝缨大会。想我知恩不报非丈夫也。如今晋国前来犯界,楚王御驾亲征,命襄老以为前部先锋。俺不免奔往前部,讨一差使与晋兵对敌,以报君恩也。

唐狡

\setlength{\hangindent}{60pt}{ 【{\akai 西皮摇板}】楚王恩德真非小,不把国法斩儿曹。如此宽宏古来少,不辞劳碌报当朝。(唐狡下) }

\vspace{3pt}{\centerline{{[}{\hei 第九场}{]}}}\vspace{5pt}

(四龙套拿枪引襄老上,{[}{\akai 引子}{]},襄着狮子盔、白箭衣、黑马褂、白花开氅)

襄老\hspace{30pt}~

{[}{\akai 引子}{]}先行是我,我是先行。

襄老\hspace{30pt}~

({\akai 念})老将勇猛不可当,全凭精气逞豪强。忠心耿耿扶楚室,何日凯歌转还乡?

襄老

某,襄老是也。大王征战晋国,命我以为前站先行,今日黄道正好发兵。众将官,起兵前往。

众\hspace{40pt}~

啊。

唐狡\hspace{30pt}~

({\akai 内}白)住着!

众\hspace{40pt}~

有人阻令。

襄老\hspace{30pt}~

嗯,何人竟敢阻令,传他进帐。

众\hspace{40pt}~

阻令者进帐。

唐狡  ({\akai 内}白)俺来也。(唐狡上)

唐狡\hspace{30pt}~

欲为世上奇男子,须建人间未有功。卑将唐狡参见。

襄老\hspace{30pt}~

噢,原来是你。大兵正欲起行,你为何阻令?

唐狡\hspace{30pt}~

小将自投麾下并未建功;今主将领兵伐晋,小将愿为前站立功报国。

襄老

啊,楚营多少大将,尚且全扣束身\footnote{ 李元皓君认为此处当作``钳口、束身'',即取``钳口不言、束身自好''之意。李楠君以为此处``全扣''当作``拳扣'',并注``拳扣,又名指虎,俗称`手撑子',古时士兵所用掌上兵器''。};你一随使将校,胆敢大言阻令,本欲取斩,犹恐出兵不利。还不下去。

唐狡\hspace{30pt}~

主将差矣。

唐狡\hspace{30pt}~

\setlength{\hangindent}{60pt}{ 【{\akai 西皮散板}】唐狡虽然裨将校,胸怀韬略胆气豪。食君粮饷恩当报, }

唐狡\hspace{30pt}~

主将!

唐狡\hspace{30pt}~

({\akai 接唱})要与君王扫贼巢。

襄老\hspace{30pt}~

嘟,

襄老

\setlength{\hangindent}{60pt}{ 【{\akai 西皮散板}】我国大将有多少,遵令钳口不逞豪。小小裨将胡乱道,抗吾军令绑市曹\footnote{ 市曹,指城市中商业集中之处。古代常在这样的地方处决人犯,因此``市曹''也代指行刑场所。}。 }

唐狡

\setlength{\hangindent}{60pt}{ 【{\akai 西皮散板}】主将何以气量小,欺压英雄为哪条。年老出令语颠倒,焉能对垒动枪刀。交锋岂论年纪小, }

唐狡\hspace{30pt}~

主将,

唐狡\hspace{30pt}~

({\akai 接唱})定把晋国化海潮。

襄老\hspace{30pt}~

一派胡言,无知小卒,杀之无益,将唐狡重打四十扯下去。

(二卒、唐狡下,内打,搀上,狡跪念)

唐狡\hspace{30pt}~

谢主将责。

襄老\hspace{30pt}~

念你帐下多年,留一线之情发往后队,收拾锣锅帐房。下去。

唐狡\hspace{30pt}~

哎呀。(唐狡下)

襄老\hspace{30pt}~

众将官,起兵前往。

(襄老脱氅,拿枪上马,众领下)

\vspace{3pt}{\centerline{{[}{\hei 第十场}{]}}}\vspace{5pt}

(\textless{}\!{\bfseries\akai {\akai 风入松}}\!\textgreater{}头段,龙套引先蔑上,下场门骨牌对)

先蔑\hspace{30pt}~

为何不行?

众\hspace{40pt}~

来此楚地不远。

先蔑\hspace{30pt}~

列开旗门。

(\textless{}\!{\bfseries\akai {\akai 风入松}}\!\textgreater{}二段,众站门,先蔑站中间)

先蔑\hspace{30pt}~

众将官,

楚王出兵多有奸诈,闻得前锋乃是襄老,虽不足惧,但必须人人努力,将他君臣一鼓而擒。

众\hspace{40pt}~

啊!

(公子凯、公子有上)

公子凯、公子有 启司马,楚兵扎颖川地方,先行襄老离此不远。

先蔑\hspace{30pt}~

啊,楚王亲自出兵,真是天助人愿。众将官,杀上前去。

(\textless{}\!{\bfseries\akai {\akai 风入松}}\!\textgreater{}三段,先蔑众领起,襄老众抄上,襄龙套下,留襄大边与先架住)

襄老\hspace{30pt}~

呔,来将通名。

先蔑\hspace{30pt}~

听者,某乃晋国大司马先蔑是也,你这老将通名受死。

襄老\hspace{30pt}~

听者,俺乃楚王驾下前站先锋襄老是也。

先蔑\hspace{30pt}~

哈哈$\cdots{}\cdots{}$老弱残兵,非某对手,快教楚王自受其绑。

襄老\hspace{30pt}~

孺子,你嫌我老,且试演试演家伙。

(先蔑打襄老败下,先众追下,先耍下场下)

\vspace{3pt}{\centerline{{[}{\hei 第十一场}{]}}}\vspace{5pt}

(\textless{}\!{\bfseries\akai 长锤}\!\textgreater{}众引庄王上,众站门)

庄王

\setlength{\hangindent}{60pt}{ 【{\akai 西皮摇板}】旌旗招展空中飘({\akai 或}: 空飘绕;空中绕),满营将官({\akai 或}: 将士个个)逞英豪。孤王兴兵({\akai 或}: 孤王领兵)把贼扫, }

(庄王正面小座)

庄王\hspace{30pt}~

({\akai 接唱})旗开得胜转还朝。

(襄老\textless{}\!{\bfseries\akai 长锤}\!\textgreater{}上)

襄老

\setlength{\hangindent}{60pt}{ 【{\akai 西皮快板}】先蔑武艺果然好,一战未交我就逃。年纪衰迈精神老,奔回大营奏根苗。 }

襄老\hspace{30pt}~

老臣交令。

庄王\hspace{30pt}~

可曾会过阵来?晋国将官哪个?

襄老\hspace{30pt}~

晋国元帅名叫先蔑。

庄王\hspace{30pt}~

呵,先蔑。(胜负如何?)

襄老\hspace{30pt}~

老臣出马就被他一枪,哎呀$\cdots{}\cdots{}$

庄王\hspace{30pt}~

(呃,)敢是带了伤了?

襄老\hspace{30pt}~

枪回来了。

庄王\hspace{30pt}~

敢是败了?({\akai 或}: 哦,败了。)

襄老\hspace{30pt}~

败了。

庄王\hspace{30pt}~

后营憩息。({\akai 或}: 老将军后营歇息。)

襄老\hspace{30pt}~

谢大王。(谢襄老)

庄王\hspace{30pt}~

(且住,)先蔑老儿十分骁勇,必须孤王亲自会他,众将官,奋勇当先。

(众领起,先蔑众上,二龙出水会阵)

先蔑\hspace{30pt}~

呔,来者敢是楚王?

庄王\hspace{30pt}~

正是。来者可是先蔑?

先蔑\hspace{30pt}~

然。

庄王\hspace{30pt}~

先蔑,楚邦({\akai 或}: 孤王)有何亏负你国,无故兴兵是何理也?

先蔑\hspace{30pt}~

昏庄!你横行天下,某奉晋君旨意,领兵扫荡。还不束手受绑?

庄王\hspace{30pt}~

(贼子)住口!众将官排开阵势者。

庄王\hspace{30pt}~

\setlength{\hangindent}{60pt}{ 【{\akai 西皮导板}】叫三军与孤战鼓操, }

(龙套钻烟筒,一合两合拉开唱)

庄王

\setlength{\hangindent}{60pt}{ 【{\akai 西皮快板}】先蔑老儿听根苗: 列国早已({\akai 或}: 各国俱已)结盟好,同心协力保周朝。你主不该把孤藐,平地生波为哪条。陆戎小国被孤扫,陈、郑不敢犯边辽。({\akai 或}: 陆戎小国被孤扫,陈、郑不敢犯边辽。你主若是行无道,定把晋国永勾销。或: 陈、郑二邦写降表,陆戎不敢犯边辽。你主不该行无道,无故兴兵为哪条。或: 你主不该行无道,无故兴兵为哪条。陆戎小国何足道,陈、郑不敢犯边辽。)劝你马前写降表({\akai 或}: 归顺好),免得尸首马后抛。 }

先蔑

\setlength{\hangindent}{60pt}{ 【{\akai 西皮摇板}】大晋明君存仁道, }

【{\footnotesize 转}{\akai 西皮快板}】各守疆土见识高。你图中原行霸道,称孤道寡犯天条。屡次兴兵各国扫,横行天下夺城壕。两军对垒战场道,各显奇能逞英豪。

庄王\hspace{30pt}~

\setlength{\hangindent}{60pt}{ 【{\akai 西皮摇板}】好言说尔说不倒({\akai 或}: 好话说尔说不倒)。 }

先蔑\hspace{30pt}~

\setlength{\hangindent}{60pt}{ 【{\akai 西皮摇板}】管教昏王丧荒郊。 }

庄王\hspace{30pt}~

\setlength{\hangindent}{60pt}{ 【{\akai 西皮摇板}】三军摆开({\akai 或}: 三军排开)长蛇道。 }

(先蔑\!{\bfseries\akai 扫一句}\!,开打,钻烟筒,打枪剑,庄王败下,上楚将一二败下,追过场,先耍下场下)

\vspace{3pt}{\centerline{{[}{\hei 第十二场}{]}}}\vspace{5pt}

庄王

(上唱)【{\akai 西皮散板}】一霎时玉石焚金山颓倒,闯东西、奔南北生路哪条({\akai 或}: 闯东西、

奔南北生路何条)。

庄王

({\akai 念})且住!先蔑老儿十分骁勇,连败孤王数员大将,呜哙呀,事到如今孤王身边连一个保驾的臣子都没有了,看将起来真是成了孤家了。({\akai 或}: 连挑孤家数员上将,哎呀,孤王如今身边连一个保驾的臣子都没有了,哎呀,看将起来,真是孤家了。)

先蔑\hspace{30pt}~

({\akai 内}白)哪里走!

庄王\hspace{30pt}~

哎呀来了。

(先蔑追上打庄王下,楚将三四上,败下,先耍下场追下)

\vspace{3pt}{\centerline{{[}{\hei 第十三场}{]}}}\vspace{5pt}

(唐狡甩发、黑箭衣、背单刀、手拿梢子帽,上唱)

唐狡\hspace{30pt}~

\setlength{\hangindent}{60pt}{ 【{\akai 西皮摇板}】只望立功把恩报,主将不用枉心劳。 }

唐狡

({\akai 念})俺,唐狡。我主兵伐晋邦,只望先锋面前讨一前站,不想反被罚为小卒,收拾锣锅帐房与老卒同行。好不丧气人也。(鼓架子)且住!耳听喊杀之声,待俺登高一望。

(上桌子望。庄王领楚众上,晋众压队追上,庄众下,先蔑众追下,唐狡跳下桌)

唐狡

且住!前面败的我主,后面追的先蔑。此时不救,待等何时,呔,先蔑休要逞强,唐老爷来也。(扔帽,拔刀,耍下)

\vspace{3pt}{\centerline{{[}{\hei 第十四场}{]}}}\vspace{5pt}

(庄王上,先蔑追上,打庄抢背,唐狡上挑开,襄下场门上,搀庄上桌子,狡打先下,狡单刀耍下场,庄桌上云手踢腿,左右一、二外望,比势摸颈,唱)

庄王

\setlength{\hangindent}{60pt}{ 【{\akai 西皮散板}】适才被贼挑下马,忽然间闪出了年少(的)娃。满营将官俱个在孤的功劳簿上跨, }

襄老\hspace{30pt}~

老臣我在其内。

庄王\hspace{30pt}~

\setlength{\hangindent}{60pt}{ 【{\akai 西皮散板}】这一员小将孤就不认识他。 }

襄老\hspace{30pt}~

您猜我呐(唱)【{\akai 西皮散板}】我也不认识他。

庄王\hspace{30pt}~

\setlength{\hangindent}{60pt}{ 【{\akai 西皮散板}】看起来是孤王(拍腰)洪福大,天赐良将把贼拿。 }

(先蔑上)

先蔑

\setlength{\hangindent}{60pt}{ 【{\akai 西皮摇板}】昏王被某挑下马,猛然来了年少娃。手使钢刀迎面扎,某家不曾提防他。落地梅花耍一耍, }

(先蔑提枪花大边台口落地梅花势,唐狡换枪上勒马背枪单腿站)

先蔑\hspace{30pt}~

娃娃为何不敢前进?

唐狡\hspace{30pt}~

你用落地梅花暗施诡计非英雄也。

先蔑

\setlength{\hangindent}{60pt}{ 【{\akai 西皮摇板}】倒教娃娃耻笑咱,楚国兵将全不怕,偏遇无名小冤家。扳鞍踏镫把马跨, }

唐狡\hspace{30pt}~

\setlength{\hangindent}{60pt}{ 【{\akai 西皮摇板}】老爷擒你献皇家。抖擞精神催战马, }

先蔑\hspace{30pt}~

\setlength{\hangindent}{60pt}{ 【{\akai 西皮摇板}】这枪刺得某两眼花。多少将官丧马下,何惧小小井底蛙。 }

(唐狡打先蔑下,狡耍枪下场,下)

庄王

\setlength{\hangindent}{60pt}{ 【{\akai 西皮散板}】气宇轩昂武艺佳({\akai 或}: 小将生来实可夸),能征惯战果不差。但愿先蔑早拿下,千刀万剐不饶他(\textless{}\!{\bfseries\akai 三锣}\!\textgreater{})。 }

(晋楚四将开打,先、狡两边上,漫对方将头,唐狡擒先蔑下,庄王、襄老下桌椅,趴地,襄扶庄起,不要有逗笑动作,望)

庄王\hspace{30pt}~

那先蔑呢?

襄老\hspace{30pt}~

被小将军擒住了。

庄王\hspace{30pt}~

你可曾看得清楚?({\akai 或}: 呃,老将军你可曾看见?)

襄老\hspace{30pt}~

没错,我戴着花镜呐!

庄王\hspace{30pt}~

这就好了,与孤带马。({\akai 或}: 哦,拿住了。呃,带马带马。)

襄老\hspace{30pt}~

被小将军骑了去了。

庄王\hspace{30pt}~

骑你的马。({\akai 或}: 呃,带你的马。)

襄老\hspace{30pt}~

还没有安尾巴呢!

庄王\hspace{30pt}~

孤王怎样回营呢?

襄老\hspace{30pt}~

只好开步走了。

庄王\hspace{30pt}~

如此摆驾。

襄老\hspace{30pt}~

咦。

(襄老领庄王下)

\vspace{3pt}{\centerline{{[}{\hei 第十五场}{]}}}\vspace{5pt}

(牌子,庄王众上站门,庄正面大座,襄老上报)

襄老\hspace{30pt}~

先蔑擒到。

庄王\hspace{30pt}~

押上帐来。({\akai 或}: 带先蔑。)

(襄老拉先蔑手杻上,先在襄后踹襄,襄趴下,再起来)

先蔑\hspace{30pt}~

\setlength{\hangindent}{60pt}{ 【{\akai 西皮摇板}】龙入铁网难撑架, }

先蔑

\setlength{\hangindent}{60pt}{ 【{\akai 西皮快板}】虎落平阳被擒拿。列国英雄也有咱,遇这无名小冤家。某既被擒凭刀剐,落得忠名扬天涯。将身站立大帐下,(进帐) }

先蔑\hspace{30pt}~

\setlength{\hangindent}{60pt}{ 【{\akai 西皮摇板}】看他把某怎开发。 }

庄王\hspace{30pt}~

\setlength{\hangindent}{60pt}{ 【{\akai 西皮摇板}】孤王帐中用目洒, }

庄王

\setlength{\hangindent}{60pt}{ 【{\akai 西皮快板}】先蔑老儿带锁枷({\akai 或}: 披锁枷)。阵前何等威风大,运败时衰被孤拿。 }

庄王\hspace{30pt}~

(白)先蔑。(孤王有何亏负你国,何故兴兵犯界,是何理也?)

先蔑\hspace{30pt}~

昏王。

(踢桌,庄王站躲,再坐下)

庄王

还是如此厉害,先蔑你在两军阵前何等威风,如今被擒帐下,有何话讲?({\akai 或}: 呜哙呀,你在阵前何等威风,何等煞气,今日被擒,有何话讲?)

先蔑\hspace{30pt}~

昏王何必多言。

庄王

孤王何曾亏负你国,无故兴兵犯界,先斩你这老头,再擒晋侯与他辩理,来,将先蔑推出斩了。({\akai 或}: 呜哙呀,还是这等的厉害,哼,先斩你这个老头,再擒晋侯与他辩理。来,将先蔑推出斩了。)

(襄老拉先蔑下,\textless{}\!{\bfseries\akai 五锣三鼓}\!\textgreater{},襄上报)

襄老\hspace{30pt}~

先蔑斩首,小将回营。

庄王\hspace{30pt}~

有请。

(庄王出位。唐狡\textless{}\!{\bfseries\akai 紧锤}\!\textgreater{}上,下马。庄拉狡换边,庄小边、狡大边台口,襄老托狡腿或可扳狡朝天镫)

庄王

\setlength{\hangindent}{60pt}{ 【{\akai 西皮快板}】一见小将到帐下,功劳({\akai 或}: 战伐)魁首第一家。孤将龙衣来脱下, }

(吹打合龙,唐狡穿庄王黄马褂、戴武生巾,庄穿开氅,庄收腿)

庄王

\setlength{\hangindent}{60pt}{ 【{\akai 西皮快板}】得胜御酒({\akai 或}: 得胜琼浆;功劳簿上)把功加。({\akai 或}: 得胜御酒付卿拿。) }

(递酒,唐狡接酒谢天地,庄王正面小座,狡参拜)

唐狡\hspace{30pt}~

参见大王,救驾来迟大王恕罪。

庄王\hspace{30pt}~

平身。({\akai 或}: 罢了。)

唐狡\hspace{30pt}~

谢大王。

庄王\hspace{30pt}~

赐座。({\akai 或}: 一旁坐下。)

唐狡\hspace{30pt}~

谢座。

(唐狡坐大边,襄老站小边)

庄王\hspace{30pt}~

小将军哪里人氏,姓甚名谁,(孤王有何恩惠于你,)竟敢一人前来救驾。

唐狡\hspace{30pt}~

小臣唐狡,棠邑人氏。大王待小臣有天高地厚之恩,特来救驾。

庄王\hspace{30pt}~

啊,孤王有何恩惠于你({\akai 或}: 哦,孤有何恩典于你)?

唐狡\hspace{30pt}~

大王可记得绝缨会之故否?

庄王

哦,不必深言(不要背供)。你今(日)救驾有功,封为上军副帅。(同孤扫晋。)

唐狡\hspace{30pt}~

谢大王。

(襄老``哎呀''蹲下)

庄王\hspace{30pt}~

老将军为何如此?

(哎呀,老将军你这是怎么样了?)

襄老

大王有所不知,唐将军乃老臣帐下兵卒,老臣曾将他重责,不料他勤王救驾封官,上军副帅,正管我这个前站先行,老臣我这回可真玩不开了。

庄王

(哦,)原来如此,这样吧,从今以后将老将军拨在唐将军帐中({\akai 或}: 将军帐下),倘有差迟,(呃,)按军令施行如何?

襄老\hspace{30pt}~

哎哟。

唐狡

啊老将军,为将者当以军法为重。唐狡自应以德报德,以直报怨。焉有记恨之理,老将军何必挂怀?

襄老\hspace{30pt}~

将军乃奇男子也。

庄王

二卿为孤不惜\footnote{ 段公平君建议也可作``不恤''。}身躯,岂能怨恨,后帐摆宴与二卿解和贺功。

(庄王下,唐狡、襄老互让下,众下)
}
