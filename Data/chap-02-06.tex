\newpage
\phantomsection %实现目录的正确跳转
\section*{\large\hei 群英会~\protect\footnote{刘曾复先生钞本注:~``汪桂芬、王凤卿、余叔岩、贾洪林  派,与富(连成)社不同''。(戏中所有鲁肃下场为王凤卿授)~剧本中有关人物、场次和调度由段公平{\scriptsize 君}协助整理。}{\large\hei ~{\small 之}~鲁肃、诸葛亮}}
\addcontentsline{toc}{section}{\hei 群英会~{\small 之}~鲁肃、诸葛亮}

\hangafter=1                   %2. 设置从第1⾏之后开始悬挂缩进  %}
\setlength{\parindent}{0pt}{

\vspace{3pt}{\centerline{{[}{\hei 第一场}{]}}}\vspace{5pt}

\setlength{\hangindent}{52pt}{鲁肃\hspace{30pt}来也!}

\setlength{\hangindent}{52pt}{鲁肃\hspace{30pt}({\akai 念})运筹除汉逆,参赞保东吴。}

\setlength{\hangindent}{52pt}{鲁肃\hspace{30pt}参见都督,孔明先生到。}

\setlength{\hangindent}{52pt}{鲁肃\hspace{30pt}现在帐外。}

\setlength{\hangindent}{52pt}{鲁肃\hspace{30pt}有请诸葛先生。}

\setlength{\hangindent}{52pt}{诸葛亮\hspace{20pt}嗯哼!}

\setlength{\hangindent}{52pt}{诸葛亮\hspace{20pt}({\akai 念})不惜一身探虎穴,智高哪怕入龙潭。}

\setlength{\hangindent}{52pt}{诸葛亮\hspace{20pt}(啊)都督。}

\setlength{\hangindent}{52pt}{诸葛亮\hspace{20pt}有座。(亮来得卤莽,都督海涵。\footnote{刘曾复先生钞本注:~``此句可不念''。})}

\setlength{\hangindent}{52pt}{诸葛亮\hspace{20pt}岂敢。(都督)相召,有何见谕?}

\setlength{\hangindent}{52pt}{诸葛亮\hspace{20pt}人马未动,粮草先行。}

\setlength{\hangindent}{52pt}{诸葛亮\hspace{20pt}承都督委派,亮自当效劳。}

\setlength{\hangindent}{52pt}{诸葛亮\hspace{20pt}就请都督传令。}

\setlength{\hangindent}{52pt}{诸葛亮\hspace{20pt}得令。}

\setlength{\hangindent}{52pt}{诸葛亮\hspace{20pt}正是:~({\akai 念})明知周郎借刀计,佯装假作不知情。}

\setlength{\hangindent}{52pt}{诸葛亮\hspace{20pt}哈哈哈$\cdots{}\cdots{}$({\hwfs 笑介})}

\setlength{\hangindent}{52pt}{鲁肃\hspace{30pt}啊,都督,命孔明劫粮却是何意?}

\setlength{\hangindent}{52pt}{鲁肃\hspace{30pt}哦,是是是。}

\setlength{\hangindent}{52pt}{(鲁肃{\hwfs 出帐在大边台口略一沉思介下})}

\setlength{\hangindent}{52pt}{鲁肃\hspace{30pt}哈哈哈$\cdots{}\cdots{}$({\hwfs 笑介})\footnote{陈超老师介绍:~这是鲁肃的第二个上场。}}

\setlength{\hangindent}{52pt}{鲁肃\hspace{30pt}【{\akai 西皮摇板}】诸葛亮背地里将人嘲笑,他道那周都督用计不高。}

\setlength{\hangindent}{52pt}{鲁肃\hspace{30pt}那孔明他回到馆驿,哈哈大笑,道他水战、陆战,车战、步战,件件精通。\footnote{此处不念《三国演义》原文中``伏路把关饶子敬,临江水战有周郎。''两句。}非比都督只习水战一能耳!}

\setlength{\hangindent}{52pt}{鲁肃\hspace{30pt}然也。}

\setlength{\hangindent}{52pt}{鲁肃\hspace{30pt}是是是。}

\setlength{\hangindent}{52pt}{鲁肃\hspace{30pt}咳,原令追回。}

\setlength{\hangindent}{52pt}{(鲁肃{\hwfs 出帐在大边台口轻摊手轻叹下})}

\vspace{3pt}{\centerline{{[}{\hei 第二场}{]}}}\vspace{5pt}
\setlength{\hangindent}{52pt}{(周瑜\hspace{30pt}但不知$\cdots{}\cdots{}$)}

\setlength{\hangindent}{52pt}{鲁肃\hspace{30pt}乃是荆襄降将蔡瑁、张允。}

\setlength{\hangindent}{52pt}{鲁肃\hspace{30pt}啊,都督闻得蒋干过江,为何发笑?}

\setlength{\hangindent}{52pt}{鲁肃\hspace{30pt}哦哦哦$\cdots{}\cdots{}$}

\setlength{\hangindent}{52pt}{鲁肃\hspace{30pt}啊,都督,想那蒋干乃都督同窗契友,恐识笔迹,肃来代笔。}

\setlength{\hangindent}{52pt}{鲁肃\hspace{30pt}是是是。}

\setlength{\hangindent}{52pt}{(鲁肃{\hwfs 出帐在大边台口左手轻捋胡子}、{\hwfs 轻转腰看右手信})}

\setlength{\hangindent}{52pt}{鲁肃\hspace{30pt}噗。({\hwfs 笑介})}

\setlength{\hangindent}{52pt}{(鲁肃{\hwfs 用手中信一挡头},{\hwfs 边转身边把信送入左袖口内下})}

\setlength{\hangindent}{52pt}{(鲁肃{\hwfs 藏书},{\hwfs 有细致做工}\footnote{刘曾复先生在为樊百乐{\scriptsize 君}说戏时详细介绍了藏书的做工和细节。})}

\setlength{\hangindent}{52pt}{(鲁肃{\hwfs 藏完书出帐先往上场门边走一小步},{\hwfs 听见}蒋干{\hwfs 来了},{\hwfs 立即撤左脚往下场门边轻退},{\hwfs 边退边把灯从右手交左手},{\hwfs 右手投袖遮灯},{\hwfs 站稳偷眼一望}、{\hwfs 一点头},{\hwfs 沉着转身轻轻稳步下})}

\setlength{\hangindent}{52pt}{鲁肃\hspace{30pt}啊,蒋先生,请了请了。}

\setlength{\hangindent}{52pt}{鲁肃\hspace{30pt}都督醒来,都督醒来。}

\setlength{\hangindent}{52pt}{(周瑜{\hwfs 出帐子})}
\setlength{\hangindent}{52pt}{鲁肃\hspace{30pt}哈哈哈$\cdots{}\cdots{}$({\hwfs 笑介})}

\setlength{\hangindent}{52pt}{鲁肃\hspace{30pt}那蒋干果然他盗书逃走了。}

\setlength{\hangindent}{52pt}{鲁肃\hspace{30pt}都督请看。}

\setlength{\hangindent}{52pt}{鲁肃\hspace{30pt}呵,哈哈哈$\cdots{}\cdots{}$({\hwfs 笑介})}

\setlength{\hangindent}{52pt}{鲁肃\hspace{30pt}量他们不知。}

\setlength{\hangindent}{52pt}{鲁肃\hspace{30pt}那孔明么,哼,他也未必料到。}

\setlength{\hangindent}{52pt}{鲁肃\hspace{30pt}呵,哈哈哈$\cdots{}\cdots{}$({\hwfs 笑介})}

\setlength{\hangindent}{52pt}{鲁肃\hspace{30pt}【{\akai 西皮摇板}】周都督运机谋神鬼不觉。}

{(周瑜{\hwfs 唱完第三句}【{\akai 西皮摇板}】{\hwfs 下},鲁肃{\hwfs 跟过去到下场门边向里站住},{\hwfs 一想}、{\hwfs 轻摇头},{\hwfs 回身右手投袖}、{\hwfs 颠袖露手})}

\setlength{\hangindent}{52pt}{鲁肃\hspace{30pt}【{\akai 西皮摇板}】怕只怕瞒不过南阳诸葛。}

\setlength{\hangindent}{52pt}{({\hwfs 用右手指},{\hwfs 抓袖转身},\textless{}\!{\bfseries\akai 大锣抽头}\!\textgreater{}{\hwfs 稳步甩下摆},{\hwfs 微摇头下})}

\vspace{3pt}{\centerline{{[}{\hei 第三场}{]}}}\vspace{5pt}
\setlength{\hangindent}{52pt}{鲁肃\hspace{30pt}哈哈哈$\cdots{}\cdots{}$({\hwfs 内笑介出})}

\setlength{\hangindent}{52pt}{鲁肃\hspace{30pt}【{\akai 西皮摇板}】曹孟德果杀了蔡瑁、张允,周都督可算得第一能人。}

\setlength{\hangindent}{52pt}{鲁肃\hspace{30pt}呵哈哈哈$\cdots{}\cdots{}$({\hwfs 笑介})}

\setlength{\hangindent}{52pt}{(周瑜\hspace{30pt}$\cdots{}\cdots{}$为何发笑?)}

\setlength{\hangindent}{52pt}{鲁肃\hspace{30pt}那曹操果然中了都督借刀之计,杀了蔡瑁、张允。水军头目换了毛玠、于禁掌管了。}

\setlength{\hangindent}{52pt}{鲁肃\hspace{30pt}量他们不知。}

\setlength{\hangindent}{52pt}{鲁肃\hspace{30pt}量他也不晓。}

\setlength{\hangindent}{52pt}{鲁肃\hspace{30pt}遵命。有请诸葛先生。}

\setlength{\hangindent}{52pt}{诸葛亮\hspace{20pt}嗯哼。}

\setlength{\hangindent}{52pt}{诸葛亮\hspace{20pt}【{\akai 西皮摇板}】昨夜晚听消息早已料定({\akai 或}:~昨夜晚观天象早已算就;{\akai 或}:~昨夜晚观天象早已料定),曹孟德中巧计误杀水军。}

\setlength{\hangindent}{52pt}{诸葛亮\hspace{20pt}(啊)都督。}

\setlength{\hangindent}{52pt}{诸葛亮\hspace{20pt}有座。恭喜都督,贺喜都督。}

\setlength{\hangindent}{52pt}{诸葛亮\hspace{20pt}那曹操中了都督借刀之计,杀了蔡瑁、张允,水军头目换了毛玠、于禁。此二人不习水战,岂非一喜?}

\setlength{\hangindent}{52pt}{鲁肃\hspace{30pt}怎么他$\cdots{}\cdots{}$({\hwfs 惊异介},{\hwfs 背躬})(知道了。)}

\setlength{\hangindent}{52pt}{诸葛亮\hspace{20pt}你我不必言明,各写一字在手,看看心意如何。}

\setlength{\hangindent}{52pt}{诸葛亮\hspace{20pt}请------}

\setlength{\hangindent}{52pt}{诸葛亮\hspace{20pt}大夫请看。}

\setlength{\hangindent}{52pt}{鲁肃\hspace{30pt}嗳呀!他二人俱是``火''字。}

\setlength{\hangindent}{52pt}{(鲁肃{\hwfs 退至小边})}

\setlength{\hangindent}{52pt}{鲁肃\hspace{30pt}你二人俱是``火''字。}

\setlength{\hangindent}{52pt}{诸葛亮\hspace{20pt}未必。}

\setlength{\hangindent}{52pt}{诸葛亮\hspace{20pt}啊,哈哈哈$\cdots{}\cdots{}$({\hwfs 笑介})}

\setlength{\hangindent}{52pt}{鲁肃\hspace{30pt}哦,哈哈哈$\cdots{}\cdots{}$({\hwfs 笑介})}

\setlength{\hangindent}{52pt}{诸葛亮\hspace{20pt}军国大事焉能泄漏。}

\setlength{\hangindent}{52pt}{诸葛亮\hspace{20pt}水面交锋弓箭当先。}

\setlength{\hangindent}{52pt}{诸葛亮\hspace{20pt}愿当此任。}

\setlength{\hangindent}{52pt}{诸葛亮\hspace{20pt}但不知宽限多少日期?}

\setlength{\hangindent}{52pt}{(周瑜\hspace{30pt}三月如何$\cdots{}\cdots{}$)}

\setlength{\hangindent}{52pt}{诸葛亮\hspace{20pt}忒多了哇。}

\setlength{\hangindent}{52pt}{(周瑜\hspace{30pt}十日$\cdots{}\cdots{}$)}

\setlength{\hangindent}{52pt}{诸葛亮\hspace{20pt}倘若曹操进军,岂不误了大事?还多啊哇。}

\setlength{\hangindent}{52pt}{(周瑜\hspace{30pt}七日如何$\cdots{}\cdots{}$)}

\setlength{\hangindent}{52pt}{诸葛亮\hspace{20pt}军务紧急,还多哇。}

\setlength{\hangindent}{52pt}{(周瑜\hspace{30pt}$\cdots{}\cdots{}$)}

\setlength{\hangindent}{52pt}{诸葛亮\hspace{20pt}容山人计算,三日交箭。}

\setlength{\hangindent}{52pt}{鲁肃\hspace{30pt}啊?三日焉能造得齐十万支雕翎~({\akai 或}:~狼牙)~箭呐?~先生!}

\setlength{\hangindent}{52pt}{(周瑜\hspace{30pt}三日无箭?)}

\setlength{\hangindent}{52pt}{诸葛亮\hspace{20pt}三日无箭,甘当军令。}

\setlength{\hangindent}{52pt}{(周瑜\hspace{30pt}军无戏言。)}

\setlength{\hangindent}{52pt}{诸葛亮\hspace{20pt}愿立军状。}

\setlength{\hangindent}{52pt}{诸葛亮\hspace{20pt}亮乎?}

\setlength{\hangindent}{52pt}{鲁肃\hspace{30pt}使不得!}

\setlength{\hangindent}{52pt}{鲁肃\hspace{30pt}完了。}

\setlength{\hangindent}{52pt}{诸葛亮\hspace{20pt}大夫,这是山人的军令状,请大夫收藏。}

\setlength{\hangindent}{52pt}{诸葛亮\hspace{20pt}三日后江边搬箭。}

\setlength{\hangindent}{52pt}{鲁肃\hspace{30pt}搬你的尸灵吧。}

\setlength{\hangindent}{52pt}{诸葛亮\hspace{20pt}取笑了。}

\setlength{\hangindent}{52pt}{诸葛亮\hspace{20pt}告辞了。}

\setlength{\hangindent}{52pt}{诸葛亮\hspace{20pt}【{\akai 西皮摇板}】在帐中辞公瑾再别子敬,三日后到江边搬取雕翎。}

\setlength{\hangindent}{52pt}{鲁肃\hspace{30pt}啊,都督,那孔明莫非有逃走之意?}

\setlength{\hangindent}{52pt}{鲁肃\hspace{30pt}是。}

\setlength{\hangindent}{52pt}{鲁肃\hspace{30pt}啊,都督,此二人恐是诈降。}

\setlength{\hangindent}{52pt}{(周瑜{\hwfs 叫}鲁肃``出帐去罢'')}

\setlength{\hangindent}{52pt}{鲁肃\hspace{30pt}哦,是是是。}

\setlength{\hangindent}{52pt}{(鲁肃{\hwfs 到大边台口})

\setlength{\hangindent}{52pt}{鲁肃\hspace{30pt}({\bfseries\akai 小}\textless{}\!{\bfseries\akai 叫头}\!\textgreater{})哎呀且住!分明是诈,怎说是实?哎呀,这、这、这$\cdots{}\cdots{}$,呵,有了,我不免去至馆驿,问过孔明先生,呃,问过孔明先生。({\hwfs 不念对儿}\footnote{这个对儿是``真假难凭信,好歹问知音。''})}

\setlength{\hangindent}{52pt}{(鲁肃{\hwfs 转身扬右手}、{\hwfs 轻摇})}

\setlength{\hangindent}{52pt}{鲁肃\hspace{30pt}哈哈哈$\cdots{}\cdots{}$({\hwfs 笑介})}

\setlength{\hangindent}{52pt}{(鲁肃{\hwfs 边笑边下})\hspace{30pt}}

\vspace{3pt}{\centerline{{[}{\hei 第四场}{]}}}\vspace{5pt}

\setlength{\hangindent}{52pt}{诸葛亮\hspace{20pt}【{\akai 西皮原板}】周公瑾命鲁肃行监坐守,叫山人背地里暗笑不休。他那里要杀我怎能得够,一桩桩一件件记在心头。}

\setlength{\hangindent}{52pt}{鲁肃\hspace{30pt}【{\akai 西皮原板}】限三天造雕翎不多时候,}

\setlength{\hangindent}{52pt}{(鲁肃\hspace{30pt}嘿嘿。)}

\setlength{\hangindent}{52pt}{鲁肃\hspace{30pt}【{\akai 西皮原板}】为什么坐一旁不睬不愁。}

\setlength{\hangindent}{52pt}{(鲁肃\hspace{30pt}嗨!)}

\setlength{\hangindent}{52pt}{鲁肃\hspace{30pt}【{\akai 西皮快板}】昨日里在帐中夸下海口,这时候倒叫我替你担忧。}

\setlength{\hangindent}{52pt}{诸葛亮\hspace{20pt}我(又)没有什么大事,要大夫替我担得什么忧哇?}

\setlength{\hangindent}{52pt}{鲁肃\hspace{30pt}啊?!昨日你({\akai 或}:~你昨日)在帐中夸下海口,立了军(令)状:~三日造齐十万支狼牙箭。你算算,今天第几天了?}

\setlength{\hangindent}{52pt}{诸葛亮\hspace{20pt}怎么还有此事么?}

\setlength{\hangindent}{52pt}{鲁肃\hspace{30pt}呵?}

\setlength{\hangindent}{52pt}{诸葛亮\hspace{20pt}不错不错,不是大夫提出,我倒忘怀了。}

\setlength{\hangindent}{52pt}{鲁肃\hspace{30pt}哎呀,哎呀,你怎么忘怀了。}

\setlength{\hangindent}{52pt}{诸葛亮\hspace{20pt}来来来,(我们)算算日期吧。昨日,}

\setlength{\hangindent}{52pt}{鲁肃\hspace{30pt}一天。}

\setlength{\hangindent}{52pt}{诸葛亮\hspace{20pt}今日,}

\setlength{\hangindent}{52pt}{鲁肃\hspace{30pt}两天,}

\setlength{\hangindent}{52pt}{诸葛亮\hspace{20pt}明日。}

\setlength{\hangindent}{52pt}{鲁肃\hspace{30pt}三天。拿来。}

\setlength{\hangindent}{52pt}{诸葛亮\hspace{20pt}什么?}

\setlength{\hangindent}{52pt}{鲁肃\hspace{30pt}拿箭来呀。}

\setlength{\hangindent}{52pt}{诸葛亮\hspace{20pt}哎呀,我是一支也无有哇!}

\setlength{\hangindent}{52pt}{鲁肃\hspace{30pt}哎呀,这这$\cdots{}\cdots{}$}

\setlength{\hangindent}{52pt}{诸葛亮\hspace{20pt}(哎呦)大夫,你要救我一救(,你要救我一救)哇。}

\setlength{\hangindent}{52pt}{鲁肃\hspace{30pt}咳,事到如今,倒不如你驾一小舟逃回江夏去罢!}

\setlength{\hangindent}{52pt}{诸葛亮\hspace{20pt}呵,我奉主之命前来同心破曹。如今寸功未立,回去怎样回覆吾主?我如何走得?(呃,)走不得呀!}

\setlength{\hangindent}{52pt}{鲁肃\hspace{30pt}(哎呀)走不得$\cdots{}\cdots{}$咳,我只有这一条主意了。}

\setlength{\hangindent}{52pt}{诸葛亮\hspace{20pt}大夫有何高见。}

\setlength{\hangindent}{52pt}{鲁肃\hspace{30pt}你呀,投江死了罢。}

\setlength{\hangindent}{52pt}{诸葛亮\hspace{20pt}呵。}

\setlength{\hangindent}{52pt}{鲁肃\hspace{30pt}还落一全尸呀。}

\setlength{\hangindent}{52pt}{诸葛亮\hspace{20pt}嗳,蝼蚁尚且贪生\footnote{刘曾复先生钞本此处记为``蝼蚁尚生'',系脱漏,据录音补正。},为人岂不惜命?你救不了我,还则罢了,你不该劝我一死。这叫作什么朋友哇。}

\setlength{\hangindent}{52pt}{鲁肃\hspace{30pt}唉,叫你走你说走不得,叫你死你又舍不得。真真的教我鲁肃为难呐!}

\setlength{\hangindent}{52pt}{诸葛亮\hspace{20pt}大夫哇!}

\setlength{\hangindent}{52pt}{鲁肃\hspace{30pt}大夫哇,不能下药了。}

\setlength{\hangindent}{52pt}{诸葛亮\hspace{20pt}【{\akai 西皮摇板}】鲁大夫平日里待人宽厚,}

\setlength{\hangindent}{52pt}{鲁肃\hspace{30pt}本来的不错哇。}

\setlength{\hangindent}{52pt}{诸葛亮\hspace{20pt}【{\akai 西皮摇板}】你保我过江来无祸无忧。}

\setlength{\hangindent}{52pt}{鲁肃\hspace{30pt}是我的保荐呐。}

\setlength{\hangindent}{52pt}{诸葛亮\hspace{20pt}【{\akai 西皮摇板}】周都督要杀我你不来搭救,}

\setlength{\hangindent}{52pt}{鲁肃\hspace{30pt}我是怎样救你呀?}

\setlength{\hangindent}{52pt}{诸葛亮\hspace{20pt}【{\akai 西皮摇板}】看起来算不得好朋友哇。}

\setlength{\hangindent}{52pt}{鲁肃\hspace{30pt}呵。}

\setlength{\hangindent}{52pt}{鲁肃\hspace{30pt}【{\akai 西皮快板}】这件事本是你自作自受,为什么反把我埋怨不休。}

\setlength{\hangindent}{52pt}{鲁肃\hspace{30pt}你怎么倒埋怨起我来了?}

\setlength{\hangindent}{52pt}{诸葛亮\hspace{20pt}大夫,(大夫)你是救不了我了。}

\setlength{\hangindent}{52pt}{鲁肃\hspace{30pt}咳,我怎样救你呀?}

\setlength{\hangindent}{52pt}{诸葛亮\hspace{20pt}嗯,你救不了我,我与你借上几样物件如何哇?}

\setlength{\hangindent}{52pt}{鲁肃\hspace{30pt}什么物件啊?呵呵,我早已预备下了。}

\setlength{\hangindent}{52pt}{诸葛亮\hspace{20pt}什么?}

\setlength{\hangindent}{52pt}{鲁肃\hspace{30pt}寿衣、寿帽,大大(的)一口棺木。}

\setlength{\hangindent}{52pt}{诸葛亮\hspace{20pt}要这些物件何用呐?}

\setlength{\hangindent}{52pt}{鲁肃\hspace{30pt}事后将你盛殓起来,送回江夏。你就不必挂念了。}

\setlength{\hangindent}{52pt}{诸葛亮\hspace{20pt}哎呀呀,不是这些宝贝呐。}

\setlength{\hangindent}{52pt}{鲁肃\hspace{30pt}什么宝贝?}

\setlength{\hangindent}{52pt}{诸葛亮\hspace{20pt}乃是军中需用之物。}

\setlength{\hangindent}{52pt}{鲁肃\hspace{30pt}什么军中需用之物?}

\setlength{\hangindent}{52pt}{诸葛亮\hspace{20pt}战船二十只,军士五百名,茅草千担,青布帐幔,金鼓全份,还要备酒一席。}

\setlength{\hangindent}{52pt}{鲁肃\hspace{30pt}啊,(这)备酒一席何用呐?}

\setlength{\hangindent}{52pt}{诸葛亮\hspace{20pt}少时我与大夫同往舟中饮酒取乐哇。}

\setlength{\hangindent}{52pt}{鲁肃\hspace{30pt}明日无箭,我看你是饮酒哇,还是取乐哇。}

\setlength{\hangindent}{52pt}{诸葛亮\hspace{20pt}大夫,千万莫对人言,你去办呐。}

\setlength{\hangindent}{52pt}{鲁肃\hspace{30pt}咳,办呐。}

\setlength{\hangindent}{52pt}{鲁肃\hspace{30pt}【{\akai 西皮摇板}】十万箭焉能够({\akai 或}:~焉能得)一夜造就,为朋友我只得顺水推舟({\hwfs 右手指})。}

\setlength{\hangindent}{52pt}{(鲁肃{\hwfs 左手撩官衣左转身右手盖头},{\hwfs 上场门反下}({\hei 示背着周瑜办事之意}))}

\setlength{\hangindent}{52pt}{诸葛亮\hspace{20pt}【{\akai 西皮摇板}】这件事料鲁肃难以猜透,哪知我袖儿中({\akai 或}:~他哪知我袖中)暗藏机谋。要借箭待等到四更时候,趁大雾到曹营去把箭收。}

\vspace{3pt}{\centerline{{[}{\hei 第五场}{]}}}\vspace{5pt}
\setlength{\hangindent}{52pt}{鲁肃\hspace{30pt}【{\akai 西皮快板}】一桩桩一件件俱已办就,请先生到江边即刻登舟。}

\setlength{\hangindent}{52pt}{诸葛亮\hspace{20pt}大夫,备齐了?}

\setlength{\hangindent}{52pt}{鲁肃\hspace{30pt}备齐了。}

\setlength{\hangindent}{52pt}{诸葛亮\hspace{20pt}请。}

\setlength{\hangindent}{52pt}{鲁肃\hspace{30pt}请到何处(去)哇?}

\setlength{\hangindent}{52pt}{诸葛亮\hspace{20pt}同往舟中饮酒取乐哇。}

\setlength{\hangindent}{52pt}{鲁肃\hspace{30pt}要去你去,我不去。({\hwfs 右手手心朝上往外一挥},{\hwfs 托胡子扔出去}、{\hwfs 摇左手右转身要走同时递左手},孔明{\hwfs 左手拉住}鲁肃{\hwfs 左手},{\hwfs 右手从}鲁肃{\hwfs 左胳膊上方过去},{\hwfs 用扇柄挑}鲁肃{\hwfs 胡子})}

\setlength{\hangindent}{52pt}{诸葛亮\hspace{20pt}走走走。(孔明{\hwfs 拉}鲁肃{\hwfs 下})}

\setlength{\hangindent}{52pt}{诸葛亮\hspace{20pt}将船往江北而发。}

\setlength{\hangindent}{52pt}{鲁肃\hspace{30pt}呵,慢来慢来,曹营现在江北,那如何去得的?来来来,待我下船。}

\setlength{\hangindent}{52pt}{诸葛亮\hspace{20pt}(呃,)船已离岸,(你)下不去了。}

\setlength{\hangindent}{52pt}{鲁肃\hspace{30pt}便怎么样呀?}

\setlength{\hangindent}{52pt}{诸葛亮\hspace{20pt}你我一同吃酒({\akai 或}:~一同饮酒)哇。}

\setlength{\hangindent}{52pt}{鲁肃\hspace{30pt}诶,什么吃酒哇。}

\setlength{\hangindent}{52pt}{诸葛亮\hspace{20pt}【{\akai 西皮原板}】一霎时白茫茫漫江雾厚,顷刻间观不出在岸在舟。似这等巧机关世间少有,学轩辕造指南以制蚩尤。}

\setlength{\hangindent}{52pt}{(鲁肃\hspace{30pt}哎!)}

\setlength{\hangindent}{52pt}{鲁肃\hspace{30pt}【{\akai 西皮原板}】鲁子敬在舟中浑身战抖,把性命当儿戏全不担忧。这时候他还有心肠饮酒\footnote{陈超老师介绍:~此处鲁肃不端酒杯,更没有把酒泼在自己脸上的表演。},}

\setlength{\hangindent}{52pt}{(鲁肃\hspace{30pt}唉!)}

\setlength{\hangindent}{52pt}{鲁肃\hspace{30pt}【{\akai 西皮原板}】顷刻间到曹营难保人头。}

\setlength{\hangindent}{52pt}{诸葛亮\hspace{20pt}将船直往曹营进发。}

\setlength{\hangindent}{52pt}{鲁肃\hspace{30pt}呵,我要下船。}

\setlength{\hangindent}{52pt}{诸葛亮\hspace{20pt}船行半江,你(是)下不去了。}

\setlength{\hangindent}{52pt}{鲁肃\hspace{30pt}(呃,)便怎么样呀?}

\setlength{\hangindent}{52pt}{诸葛亮\hspace{20pt}请来吃酒哇。}

\setlength{\hangindent}{52pt}{鲁肃\hspace{30pt}呃,吃酒,吃酒,好,吃酒(、吃酒)哇。}

\setlength{\hangindent}{52pt}{诸葛亮\hspace{20pt}大夫哇,}

\setlength{\hangindent}{52pt}{诸葛亮\hspace{20pt}【{\akai 西皮摇板}】劝大夫放宽怀且自饮酒,些许事又何必这等担忧。}

\setlength{\hangindent}{52pt}{诸葛亮\hspace{20pt}擂鼓呐喊。}

\setlength{\hangindent}{52pt}{鲁肃\hspace{30pt}(哎呀,)不要擂鼓。}

\setlength{\hangindent}{52pt}{(曹操\hspace{40pt}$\cdots{}\cdots{}$吩咐放箭。)}

\setlength{\hangindent}{52pt}{(\textless{}\!{\bfseries\akai {\akai 风入松}}\!\textgreater{}{\hwfs 头段})}

\setlength{\hangindent}{52pt}{(军士\hspace{30pt}$\cdots{}\cdots{}$经受不住。)}

\setlength{\hangindent}{52pt}{诸葛亮\hspace{20pt}拨转船头,军士大喊三声:~南阳诸葛先生谢曹丞相赠箭。}

\setlength{\hangindent}{52pt}{诸葛亮\hspace{20pt}大夫,请来观看呐。}

\setlength{\hangindent}{52pt}{(鲁肃{\hwfs 先是右手翻袖盖头},{\hwfs 往台中间一望},{\hwfs 再用左袖盖头}、{\hwfs 右手撩官衣}、{\hwfs 窝下})}

\vspace{3pt}{\centerline{{[}{\hei 第六场}{]}}}\vspace{5pt}
\setlength{\hangindent}{52pt}{(\textless{}\!{\bfseries\akai {\akai 风入松}}\!\textgreater{}{\hwfs 二段},鲁肃、诸葛亮{\hwfs 上})}

\setlength{\hangindent}{52pt}{鲁肃\hspace{30pt}先生,你怎样知道({\akai 或}:~知晓)今晚有此一场大雾哇?}

\setlength{\hangindent}{52pt}{诸葛亮\hspace{20pt}为谋士者,不知天文,不晓地理,乃庸才也。}

\setlength{\hangindent}{52pt}{鲁肃\hspace{30pt}先生真乃神人也。}

\setlength{\hangindent}{52pt}{诸葛亮\hspace{20pt}岂敢。大夫看看,这令可以交得的了么?}

\setlength{\hangindent}{52pt}{鲁肃\hspace{30pt}交令呐,有我。}

\setlength{\hangindent}{52pt}{诸葛亮\hspace{20pt}大夫请。}

\setlength{\hangindent}{52pt}{鲁肃\hspace{30pt}呃,先生请转。}

\setlength{\hangindent}{52pt}{诸葛亮\hspace{20pt}何事?}

\setlength{\hangindent}{52pt}{鲁肃\hspace{30pt}我实实服了你了。}

\setlength{\hangindent}{52pt}{诸葛亮\hspace{20pt}大夫服我何来(呢)?}

\setlength{\hangindent}{52pt}{鲁肃\hspace{30pt}我服你好阴阳,好八卦。好大的胆量呐。}

\setlength{\hangindent}{52pt}{诸葛亮\hspace{20pt}我也服了你了。}

\setlength{\hangindent}{52pt}{鲁肃\hspace{30pt}你服我何来呢?}

\setlength{\hangindent}{52pt}{诸葛亮\hspace{20pt}我服你在舟中这样呵------({\hwfs 抖介})}

\setlength{\hangindent}{52pt}{鲁肃\hspace{30pt}{\fzsong 𠳶},{\fzsong 𠳶},{\fzsong 𠳶}。({\hwfs 同时}``{\hwfs 欺}''孔明、{\hwfs 三指}孔明,孔明{\hwfs 小撤步三挡下},鲁肃{\hwfs 追下})}

\vspace{3pt}{\centerline{{[}{\hei 第七场}{]}}}\vspace{5pt}
\setlength{\hangindent}{52pt}{鲁肃\hspace{30pt}参见都督。}

\setlength{\hangindent}{52pt}{(周瑜\hspace{30pt}$\cdots{}\cdots{}$)}

\setlength{\hangindent}{52pt}{鲁肃\hspace{30pt}他造齐了。}

\setlength{\hangindent}{52pt}{(周瑜\hspace{30pt}$\cdots{}\cdots{}$怎样造$\cdots{}\cdots{}$)}

\setlength{\hangindent}{52pt}{鲁肃\hspace{30pt}都督容禀:~那孔明他回到馆驿,一天也不慌,两天也不忙。到了三日也不用我国工匠人等,只用战船二十只,军士五百名,茅草千担,青布帐幔,金鼓全份。四更时分,去至曹营,擂鼓呐喊。那时满江(的)大雾,(那)曹贼闻知,吩咐水陆两寨一齐放箭。顷刻之间,借来十万支雕翎。特来交令呐。}

\setlength{\hangindent}{52pt}{鲁肃\hspace{30pt}可算得是活神仙。}

\setlength{\hangindent}{52pt}{鲁肃\hspace{30pt}有请呵,呵,活神仙。}

\setlength{\hangindent}{52pt}{诸葛亮\hspace{20pt}({\akai 念})狼牙已造就,只在险中求。}

\setlength{\hangindent}{52pt}{诸葛亮\hspace{20pt}有座。}

\setlength{\hangindent}{52pt}{诸葛亮\hspace{20pt}都督关照。}

\setlength{\hangindent}{52pt}{诸葛亮\hspace{20pt}请。}

\setlength{\hangindent}{52pt}{(周瑜\hspace{30pt}$\cdots{}\cdots{}$一百脊杖。)}

\setlength{\hangindent}{52pt}{(鲁肃\hspace{30pt}哎呀!)}

{(``{\hei 打盖}''{\hwfs 时}黄盖{\hwfs 是跪左腿面里躬身受脊杖};~诸葛亮{\hwfs 正襟危坐},{\bfseries\hwfs 不是在喝酒}\footnote{陈超老师介绍:~贾洪林说谭鑫培不允许(诸葛亮饮酒)。};~鲁肃{\hwfs 跪}黄{\hwfs 右侧},{\hwfs 双袖覆}黄{\hwfs 背}、{\hwfs 两轰}牢子手,{\hwfs 是阻周瑜},{\hwfs 欲向孔明作色},{\hwfs 不是一劲作揖})}

\setlength{\hangindent}{52pt}{(周瑜{\hwfs 下})

\setlength{\hangindent}{52pt}{鲁肃\hspace{30pt}这一下我可不服你了({\akai 或}:~我可又不服你了)。}

\setlength{\hangindent}{52pt}{诸葛亮\hspace{20pt}怎么(又)不服我了?}

\setlength{\hangindent}{52pt}{鲁肃\hspace{30pt}方才周都督怒责黄公覆,我等俱是他麾下之人,不好讲情呐。你是客位,坐在席前,一言不发,是何道理({\akai 或}:~是何理也)?呵,难倒(说)这酒就是这样(的)好吃的么?}

\setlength{\hangindent}{52pt}{诸葛亮\hspace{20pt}他二人一个愿打,一个愿挨,与我何干呐?}

\setlength{\hangindent}{52pt}{鲁肃\hspace{30pt}世间之上只有愿打,哪个愿挨?你愿挨我就来打。}

\setlength{\hangindent}{52pt}{诸葛亮\hspace{20pt}他二人又是一计呀。}

\setlength{\hangindent}{52pt}{鲁肃\hspace{30pt}呵,又是一计?呃,倒要领教。}

\setlength{\hangindent}{52pt}{诸葛亮\hspace{20pt}大夫哇,}

\setlength{\hangindent}{52pt}{诸葛亮\hspace{20pt}【{\akai 西皮摇板}】他二人定下了苦肉之计,}

\setlength{\hangindent}{52pt}{鲁肃\hspace{30pt}收蔡中与蔡和呢?}

\setlength{\hangindent}{52pt}{诸葛亮\hspace{20pt}【{\akai 西皮摇板}】收蔡中与蔡和暗通消息。}

\setlength{\hangindent}{52pt}{鲁肃\hspace{30pt}今日之事?}

\setlength{\hangindent}{52pt}{诸葛亮\hspace{20pt}【{\akai 西皮摇板}】黄公覆受苦刑俱是假意,}

\setlength{\hangindent}{52pt}{鲁肃\hspace{30pt}(哎呀,)我是哪里晓得呀!}

\setlength{\hangindent}{52pt}{诸葛亮\hspace{20pt}【{\akai 西皮摇板}】进帐去切莫说我诸葛先知呀。}

\setlength{\hangindent}{52pt}{(孔明{\hwfs 溜下},{\hwfs 胡琴}\textless{}\!{\bfseries\akai 行弦}\!\textgreater{}鲁肃{\hwfs 转身朝外}、{\hwfs 左手托右肘}、{\hwfs 右手摸脖子}({\bfseries\hwfs 手不动颈动}))}

\setlength{\hangindent}{52pt}{鲁肃\hspace{30pt}我是哪里晓得呀!}

\setlength{\hangindent}{52pt}{(鲁肃{\hwfs 回身拱手})

\setlength{\hangindent}{52pt}{鲁肃\hspace{30pt}呵,先生。({\hwfs 看}孔明{\hwfs 不见了})}

\setlength{\hangindent}{52pt}{鲁肃\hspace{30pt}先生,先生,先生!({\hwfs 同时招手边追}孔明{\hwfs 下})}

}
