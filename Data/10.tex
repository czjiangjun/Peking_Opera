\addcontentsline{toc}{section}{\hfill[\hei 附录]\hfill}
\newpage
\textbf{柴桑口}\protect\hyperlink{fn636}{\textsuperscript{636}}

{[}第一场{]}

(四文堂,刘备、诸葛亮上)

刘备 {[}引子{]}祸福惟天造,岂在人谋计巧。

诸葛亮
{[}引子{]}秦、崔玉童\protect\hyperlink{fn637}{\textsuperscript{637}}到,从此知音稀少。

诸葛亮 参见主公。

刘备 先生少礼,请坐。

诸葛亮 谢座。

刘备
(念)周战场中祸难平,岂知转福结□□。\protect\hyperlink{fn638}{\textsuperscript{638}}

诸葛亮 (念)自古辛勤有天下,不在人谋定相星。

刘备 孤,刘备。

诸葛亮 山人诸葛亮。

刘备 先生。

诸葛亮 主公。

刘备
日前周郎设下``假途灭虢''之计,被先生奇谋,只气得他喷血坠马\protect\hyperlink{fn639}{\textsuperscript{639}},此时未闻周郎吉凶如何。

诸葛亮 亮夜观天象,见将星坠落。我料周郎刻下必死无疑也。

刘备 此话难料。

旗牌 (内)走哇。

(旗牌上)

旗牌
启上主公,昨日命小人持书递进吴营,周瑜拆开一观,忽然呕吐气绝身亡。今将灵柩移至柴桑去了。

诸葛亮 如何。

刘备 知道了。

诸葛亮 退下。

(旗牌下)

刘备 果然不出先生所料,周郎已死,还当如何?

诸葛亮
我料代周郎之权必是鲁肃。亮观天象,见将星聚在东吴,我当以过江吊孝为由,好觅贤士辅佐\protect\hyperlink{fn640}{\textsuperscript{640}}主公。

刘备
且慢,东吴将士恨先生如入骨髓,此番前去,必遭其害。且莫做那披麻救火,自惹其祸。

诸葛亮
周瑜在日,亮犹不惧,今瑜(已)死\protect\hyperlink{fn641}{\textsuperscript{641}},何足惧哉?

刘备 去得的?

诸葛亮 去得的。

刘备 无妨事?

诸葛亮 无妨事。

刘备
孤放心不下,可命三弟带领铁骑一万,战船千只,跟随先生前往。孤也好放心。

诸葛亮
既蒙主公垂念,就命四将军子龙随我前往,可命三将军在江南岸上等候。山人吊祭已毕,过江之时,用羽扇一招,前来接应便了。

刘备 先生呐。

刘备
【西皮散板\protect\hyperlink{fn642}{\textsuperscript{642}}】非是孤王人安顿,东吴尽是豺狼虎群。此番前去稍有伤损,岂不教孤两离分\protect\hyperlink{fn643}{\textsuperscript{643}}。

诸葛亮
【西皮散板\protect\hyperlink{fn644}{\textsuperscript{644}}】时不到兴与衰天心造定,当治乱自有那一辈时人。【转西皮快板】天生来周公瑾吴邦英俊,偏又有诸葛亮汉室称臣。三江口协力时同把曹併,彼爱我我爱彼各无异心。他起意三次里害我性命,心未遂气得他丧了残生。在帐中施一礼主心安稳,

刘备 先生保重了。

诸葛亮 【西皮散板】但看我一叶飘如风送云。

刘备
【西皮散板】他那里坦然不虑心安稳,孤心中不定胆战心惊。但愿得此一去吉星照定,且待他无凶险也好放心。

(同下)

{[}第二场{]}

(四白文堂、二旗牌、鲁肃上)

鲁肃
【西皮摇板\protect\hyperlink{fn645}{\textsuperscript{645}}】伤我那擎天柱一旦早丧,眼见得我东吴谁是栋梁。蒙主恩虽与我兵权执掌,愧匪才怎做得治国安邦。

鲁肃
下官鲁肃,不料孔明用计,竟将公瑾气死。蒙公瑾生前已奏表章,举我统领兵权。唉,我自愧匪才,焉能掌此大权,无奈主公再三\protect\hyperlink{fn646}{\textsuperscript{646}},难以推却,只得勉力而为。今将公瑾灵柩移至柴桑,等候他子周循\protect\hyperlink{fn647}{\textsuperscript{647}}到来,成服\protect\hyperlink{fn648}{\textsuperscript{648}}丧葬也。

四将 (内)走哇。

(四将上)

四将 参见都督。

鲁肃 列公少礼。

(鲁肃看介)

鲁肃 列公怒气不息,进帐何事?

四将
启禀都督,今有孔明带了祭礼前来祭奠先帅,故而进帐请教都督:还是将孔明杀了后祭,还是祭了后杀呢?

鲁肃 哦,那孔明竟敢前来吊祭先帅么?

四将 正是。

鲁肃 他带了多少人马?

四将 只有一叶扁舟,并无人马。

鲁肃 只有一叶扁舟,并无人马?

四将 正是。

鲁肃
(冷笑介)哼,呵呵\ldots{}\ldots{}这厮又来作怪。我东吴将士恨不得食尔之肉,他偏偏驾一叶扁舟而来。孔明啊孔明,你今前来岂不是羊入虎口。不知列公心意如何?

四将
吾家先帅被他气死,恨不得手刃此贼\protect\hyperlink{fn649}{\textsuperscript{649}}。他今自寻前来,都督传令可将他剖腹挖心\protect\hyperlink{fn650}{\textsuperscript{650}},活祭先帅,以报此仇。

鲁肃
不可。他今前来,定有一番道理。若凶惧而杀之,天下人道我东吴无容人之量。且待他祭奠之后,先用言语责他,然后治死也还不迟。不知列公意下如何?

四将 只是教那贼多活一时。

赵云 (内)孔明先生到。

鲁肃 列公不可造次,当遵吾令。

四将 遵命。

鲁肃 有请。

四将 有请。

(赵云、童儿、诸葛亮上)

鲁肃 先生。

诸葛亮 都督。

鲁肃 呀,先生一别有年,使人梦想。

诸葛亮
{久未晤面}(或:久违教益),如有所亡\protect\hyperlink{fn651}{\textsuperscript{651}}。

鲁肃
荷蒙\protect\hyperlink{fn652}{\textsuperscript{652}}足下远来吊祭,足见多情。

诸葛亮 故交之谊,聊此一行,以表寸意。

鲁肃 多谢了。

诸葛亮 先灵供在何处?

鲁肃 现在后帐。待下官引路。

诸葛亮 有劳了。

(同下)

{[}第三场{]}

(又上)

诸葛亮
\textless{}\textbf{三叫头}\textgreater{}公瑾,先生,唉,都督哇,呃\ldots{}\ldots{}(哭介)

诸葛亮
【二黄导板\protect\hyperlink{fn653}{\textsuperscript{653}}】{见陵寝}(或:见灵寝)不由人泪如雨降,想俊容不由人痛断肝肠。

诸葛亮
\textless{}\textbf{三叫头}\textgreater{}公瑾,先生,唉,都督哇,呃\ldots{}\ldots{}(哭介)

诸葛亮 【二黄散板】可惜你钟山秀{春年正旺}(或:春华正旺),

诸葛亮
【反二黄原板\protect\hyperlink{fn654}{\textsuperscript{654}}】可惜你美英才一旦夭亡。可惜你空碌碌一生容让,可惜你兢业业半世奔忙。实指望併曹瞒你我安享,\textless{}\textbf{哭头}\textgreater{}都督哇\ldots{}\ldots{}

诸葛亮
【反二黄原板】又谁知黄粱梦\protect\hyperlink{fn655}{\textsuperscript{655}}昙花一场。

旗牌 进位,上香,鞠躬。跪,叩首,二叩首,三叩首。兴,鞠躬。

赵云 (念)
呜呼公瑾,不幸夭亡!寿短固天,人岂不伤!我心实痛,酬酒一觞;君其有灵,想我衷肠。

诸葛亮
公瑾,想你英雄盖世,一代风流。贯精忠于日月,秉赤胆与东吴。不幸一旦身故,未遂你胸中之志,好不遗恨人也。

诸葛亮
【反二黄慢板\protect\hyperlink{fn656}{\textsuperscript{656}}】你是个霸业的忠贞良将,你是个振东吴豪杰贤良。谁似你天生来高智雅量,谁似你文武略器宇轩昂。谁似你青年人兵权执掌,谁似你定霸业扶弱抑强。奸曹贼统雄兵如风似浪\protect\hyperlink{fn657}{\textsuperscript{657}},只吓得江南士束手要降。若非你怀大志陈兵相抗\protect\hyperlink{fn658}{\textsuperscript{658}},运机谋烧得他抛甲弃枪。到今日稍得遂太平景象\protect\hyperlink{fn659}{\textsuperscript{659}},转瞬间\protect\hyperlink{fn660}{\textsuperscript{660}}天不佑大厦断梁。抛得我故人儿将谁依傍,\textless{}\textbf{哭头}\textgreater{}公瑾呐\ldots{}\ldots{}

诸葛亮
【反二黄原板】闪得我前后事与谁商量。今日里原比作那伯仲情况,我与你又好比鸡黍范、张\protect\hyperlink{fn661}{\textsuperscript{661}}。{望阴灵鉴吾这虔诚祭享,虔诚祭享,泪纷纷捧玉樽享受烝尝。}(或:望阴灵鉴之我祭奠,知我祭奠,诸葛亮泪纷纷痛断肝肠。\protect\hyperlink{fn662}{\textsuperscript{662}})

旗牌 进位,上香,退,鞠躬。

赵云
(念)呜呼公瑾!生死永别!朴守其真,冥冥灭灭,君如有灵,乞见我心:从此天下,更无知音!呜呼痛哉,伏惟上享。

诸葛亮 公瑾呐\ldots{}\ldots{}

诸葛亮
【反二黄散板】你非是妒贤辈胸怀愚量,都只是各为主不得不防。到今日奸曹在你命身丧,\textless{}\textbf{哭头}\textgreater{}都督哇\ldots{}\ldots{}

诸葛亮
【反二黄散板】闷得我诸葛亮心意彷徨。思至此哭得我{咽喉气颡}(或:咽喉难让),

众
【反二黄散板】只哭得满营中泪洒千行。\protect\hyperlink{fn663}{\textsuperscript{663}}

鲁肃 先生且免悲伤,还当同心破曹要紧。

众 是呀,同心破曹,全仗先生。

诸葛亮
亮纵有千言万语,一时难以尽诉。列公以奸曹为念,亮当佩服,与公同心破曹,就此告辞了。

鲁肃 且慢,备得水酒,聊表地主之情。

诸葛亮 本当领受,怎奈有公务在身,告辞了。

鲁肃 有慢了。

诸葛亮 \textless{}\textbf{叫头}\textgreater{}公瑾!

诸葛亮 你若有灵,须见我心呐!

诸葛亮 【反二黄散板】心问口、口问心牢骚千状,有万篇写不尽我心哀伤。

(诸葛亮出门)

众 送先生。

诸葛亮 【反二黄散板】送千里终须别何须谦让,

鲁肃 恕不远送了。

(赵云下)

诸葛亮
【反二黄散板】试看我一帆风雨洒康庄\protect\hyperlink{fn664}{\textsuperscript{664}}。

(诸葛亮下)

鲁肃 【反二黄散板】他那里诚恳恳哀泣模样,为朋友可算得古道热肠。

鲁肃 列公,人言先帅与孔明不睦,今日一见真乃伤情。

众 看将起来,诸葛先生乃是大大的好人。

二旗牌 (内)走哇。

(二旗牌上)

二旗牌 公子到。

鲁肃 有请。

众 有请。

(周循上)

周循 伯父。

鲁肃 公子,不知公子驾到,未曾远迎,面前恕罪。

周循 岂敢。小侄来的鲁莽,伯父、众位伯父恕罪。

众 岂敢。

周循 我父灵堂今在何处?

鲁肃 现在后帐,随我来。

周循 有劳伯父。

(同一翻两翻,周循看介、哭介)

周循 唉,爹爹呀\ldots{}\ldots{}(哭介)

周循 【二黄散板】在灵位不由我死去又醒,

周循
\textless{}\textbf{三叫头}\textgreater{}爹爹,我父,唉,爹爹呀\ldots{}\ldots{}(哭介)

周循
【二黄散板】想今生要见面万万不能。为国家受尽了千般苦衷,谁信那青史上万载标名。

鲁肃 啊公子,先帅已死,不能复生,请自保重要紧。

众 是啊,请自保重要紧。

周循 诚领众位伯父之教,小侄敢不从命。啊伯父,小侄有一言不知可听否?

鲁肃 公子有话请讲。

周循
先父执掌兵权有年,不料被孔明三气而死,我与他不共戴天之仇,望伯父助小侄一膀之力,杀往荆(州\protect\hyperlink{fn665}{\textsuperscript{665}}),生擒孔明,与我父报仇雪恨。伯父料无推辞的了。

鲁肃 啊公子,那孔明乃是个好人。

周循 啊?怎见得他是好人?

鲁肃 他闻先帅已死,过得江来哭了又祭,祭了又哭,岂不是个好人?

周循 啊,那孔明几时来的?

鲁肃 方才在此。

周循 如今何在?

鲁肃 回往荆州去了。

周循 啊伯父,你要与我统领人马追杀孔明,如若不然,我就碰死在灵前。

众 公子不必如此,大家追杀孔明便了。

周循 好,快快追赶。

鲁肃 去不得。

(同下)

{[}第四场{]}

(赵云、童儿、诸葛亮上)

诸葛亮
【西皮散板】非是我笑他们无有志量,怎知我袖儿内暗有行藏。{遇不着智谋人心中惆怅}(或:将身儿来至在江边岸上),

(庞统上)

庞统 呔,你往哪里走。

庞统 【西皮散板】你纵有托天手(或:托天胆)难逃罗网。

庞统
孔明呐孔明,你用计将周郎三气而死,又假意过江吊祭,分明笑我东吴无有能人。来来来,我与你较量。

诸葛亮 原来是凤雏先生。先生平生大才,今日出此不经之言,故意吓我。

诸葛亮、庞统 哈哈,哈哈,啊哈哈哈\ldots{}\ldots{}(诸葛亮、庞统对笑介)

诸葛亮
啊先生,我此来实为吾兄,我料仲谋必不能重用足下,玄德公宽仁厚德,(不负\protect\hyperlink{fn666}{\textsuperscript{666}})公生平所学,我有草书一封,趁便即往\protect\hyperlink{fn667}{\textsuperscript{667}}荆州共扶汉室,名垂千古,岂不美哉?

庞统
承蒙美意,自得遵教\protect\hyperlink{fn668}{\textsuperscript{668}}。

(起\textless{}\textbf{鼓架子}\textgreater{},诸葛亮、庞统两望)

庞统 啊先生,看那旁人声呐喊,必是周循追赶前来。

诸葛亮 公且自回避,亮要登舟去了。

庞统 后会有期。

诸葛亮 请呐。

诸葛亮 【西皮散板】此时节说不尽话言惆怅,

庞统 【西皮散板】暂分别改日里再会荆襄。

(庞统下)

诸葛亮 哈哈哈\ldots{}\ldots{}(笑介)

诸葛亮
【西皮散板】想人生荣与枯得失难量,际风云{显奇谋}(或:显奇能)姓字名扬。望一派{白茫茫}(或:白亮亮){翻江波浪}(或:滔天波浪),

张飞 (下场门内)嘚,开船。

(四黑龙套、二船夫、张飞上)

张飞 【西皮散板】张翼德接先生来到长江。

张飞 先生,搭了扶手。

(诸葛亮、赵云、童儿上船介;四白文堂、四将、周循上)

周循 呔,船头之上,可是诸葛亮?

诸葛亮 然也,来者何人?

周循 俺乃公瑾之子周循是也。

诸葛亮 哦,原来是公子到了,敢莫是与父谢孝的么?

周循 正是。

诸葛亮 为何持戈相向,是何理也?

周循 请先生上岸,周循有话言讲。

诸葛亮
哼呵呵呵\ldots{}\ldots{}(冷笑介)我若上岸,只恐你那小性命必随儿父去也。

周循 呔,孔明你若不上岸,休怪周循无礼了。

张飞
呔,我把你这不孝的乳臭小儿,汝父既死,儿不居守灵帐,执持器械,何以成孝?你这不忠不孝、不仁不义之人,要儿何用!先生闪开,待咱老张将他射死也。

诸葛亮 不可,饶他这条小命去罢。

张飞 也罢,念尔有重孝在身,暂且饶儿不死。嘚,开船!

(张飞三笑,诸葛亮众下)

周循 苍天呐苍天,

周循
【西皮摇板\protect\hyperlink{fn669}{\textsuperscript{669}}】满江洒下青丝网,怎奈鱼儿又脱缰。

周循 罢!

(周循跳水介,四将拦介)

四将 公子不必如此,驾船追杀孔明便了。

周循 好。驾船追者!

(同下)

注:钞本中诸葛亮祭奠周瑜的祭文,个别词句与《三国演义》原文音同字异,今将《三国演义》中诸葛亮祭文附后:

\textbf{呜呼公瑾,不幸夭亡!修短故天,人岂不伤?}

\textbf{我心实痛,酹酒一觞;君其有灵,享我烝尝!}

\textbf{吊君幼学,以交伯符;仗义疏财,让舍以民。}

\textbf{吊君弱冠,万里鹏抟;定建霸业,割据江南。}

\textbf{吊君壮力,远镇巴丘;景升怀虑,讨逆无忧。}

\textbf{吊君丰度,佳配小乔;汉臣之婿,不愧当朝,}

\textbf{吊君气概,谏阻纳质;始不垂翅,终能奋翼。}

\textbf{吊君鄱阳,蒋干来说;挥洒自如,雅量高志。}

\textbf{吊君弘才,文武筹略;火攻破敌,挽强为弱。}

\textbf{想君当年,雄姿英发;哭君早逝,俯地流血。忠义之心,英灵之气;命终三纪,名垂百世,哀君情切,愁肠千结;惟我肝胆,悲无断绝。}

\textbf{昊天昏暗,三军怆然;主为哀泣;友为泪涟。亮也不才,丐计求谋;助吴拒曹,辅汉安刘;}

\textbf{掎角之援,首尾相俦,若存若亡,何虑何忧?}

\textbf{呜呼公瑾!生死永别!朴守其贞,冥冥灭灭,魂如有灵,以鉴我心:从此天下,更无知音!}

\textbf{呜呼痛哉!伏惟尚飨。}

\newpage
\textbf{铁笼山·迷当发点}\protect\hyperlink{fn670}{\textsuperscript{670}}

{[}第一场{]}

(\textless{}\textbf{水龙吟}\textgreater{}四蛮兵、二丑校尉站门;\textless{}\textbf{四击头}\textgreater{}迷当上)

迷当
\textless{}\textbf{点绛唇}\textgreater{}西羌英豪,儿郎虎豹,统雄骁,族裔三苗,灭魏蜀汉保。

(\textless{}\textbf{水龙吟}合头\textgreater{}迷当上高台,坐)

迷当
(念)三国纷纷起战争,孔明火烧藤甲兵。七擒孟获孤得见,西羌领兵到如今。

迷当
孤,西羌国王迷当,长子迷强、次子迷能俱丧陈泰之手,为此每日操练人马,以防不测。看今日天气晴和,不免去往草上坡行围射猎,孩子们、马夫们走上。

(内 马夫们走上。)

(四马夫上)

四马夫
(念)\textless{}\textbf{马夫赞}\textgreater{}生来本是西凉娃,穿山越岭骑劣马。冲锋陷阵咱不怕,途程当玩耍,途程当玩耍。(边走边念)

四马夫 参见大王。

迷当 传蛮女们进见。

(马夫下,马夫甲、乙、丙、丁先后各拉蛮女甲、乙、丙、丁先后上,分段唱\textless{}\textbf{粉蝶儿}\textgreater{})

蛮女甲 \textless{}\textbf{粉蝶儿}\textgreater{}异国异苗,

蛮女乙 \textless{}\textbf{粉蝶儿}\textgreater{}小蛮婆,异国异苗;

蛮女丙 \textless{}\textbf{粉蝶儿}\textgreater{}天生就玉容花貌,

蛮女甲
\textless{}\textbf{粉蝶儿}\textgreater{}镇日里舞剑操刀,背弯弓、发硬驽、穿杨技巧。

四蛮女
(合)\textless{}\textbf{粉蝶儿}\textgreater{}兴来时马上嬉游,弹一曲昭君宫调。

四蛮女 参见大王。

迷当 罢了。

(马夫、蛮女分站两边)

迷当 孩子们!

(众应)

迷当 草上坡去者!

(众应,众唱分段\textless{}北\textbf{泣颜回}\textgreater{})

众 \textless{}北\textbf{泣颜回}\textgreater{}驱队出西郊,

(众合龙,迷下高台,上马)

众 \textless{}北\textbf{泣颜回}\textgreater{}逐骅骝人拥哎咆哮,

(众转场,迷上大边高台)

众 \textless{}北\textbf{泣颜回}\textgreater{}貔貅簇拥人如虎生翼英豪。

(众转场,迷下高台,上小边高台)

众
\textless{}北\textbf{泣颜回}\textgreater{}旗旛耀日,韵悠悠,画角连珠炮朴咚咚。

(迷下高台)

众
\textless{}北\textbf{泣颜回}\textgreater{}紧擂鼍鼓,布围场满塞弓刀,布围场满塞弓刀。

(归正场)

\newpage
\textbf{美良川 之 秦琼}\protect\hyperlink{fn671}{\textsuperscript{671}}

{[}第一场{]}

(上)

(念)头戴金盔凤翅飘,身穿铠甲络丝绦。劈抡双锏无人抵,保定我主锦龙朝。

俺,姓秦名琼字叔宝,唐室驾前为臣。奉主之命跟随二主千岁大战刘武周,可恨那贼战又不战,降又不降。今日闲暇无事,不免到二主营中问安。

吓!进得营来为何这样静悄悄的,待我两厢问来。三军们,主公可在营中?

哪里去了?

何人保驾?

不好了!

且住!二主夜探白璧关\protect\hyperlink{fn672}{\textsuperscript{672}},咬金保驾岂是那黑贼对手?众将官,迎上前去。(下)

{[}第二场{]}

(上)

呔!尔有何本领,擅敢追杀我主?

若问你老爷的,尔且听道。

呔!尔敢是怯战?你我两厢问来。

三军的,哪里宽阔?

打道美良川。(下)

{[}第三场{]}

(上)

来到美良川,你我怎样比试?

下得马来,何以为赌?何为打鞭换锏?

如此说来老爷先打。

老爷先打。

好,你我两厢问来。

三军的,何处地界?

黑贼,乃是我唐室地界,还是老爷先打。

哼!老爷打了就无有尔的份了。让尔先打。

这作什么?

你老爷站得稳,尔只管的打。

吐什么?

你老爷焉有吐红之理?尔只管的打。

又吐什么?

你老爷方才言过焉有吐红之理,尔只管的打。

你敢有逃走之意?若要你老爷不打,除非在老爷胯下趱将过去,俺便饶尔不死。

当真要打?

果然要打?

要打?

起鼓招打。

呔!尔为何闪你老爷这头一锏?

九十斤一根,

慢说是两锏,就是这一锏也要结果尔的性命。

当真要打?

果然要打?

要打?

起鼓招打\protect\hyperlink{fn673}{\textsuperscript{673}}。

与你老爷吐,吐红。

起鼓招打。

带马。(追下)

{[}第四场{]}

(上)

前道为何不行?

人马列开。

【西皮摇板\protect\hyperlink{fn674}{\textsuperscript{674}}】秦琼生来不可当\protect\hyperlink{fn675}{\textsuperscript{675}},美良川前摆战场。三鞭打不动秦叔宝,两锏打得他吐红光。

败兵不可追赶,人马回营。(下)

\newpage
\hypertarget{ux53d6ux91d1ux9675-ux4e4b-ux66f9ux826fux81e3}{%
\subsection{取金陵 之
曹良臣}\label{ux53d6ux91d1ux9675-ux4e4b-ux66f9ux826fux81e3}}

{[}第一场{]}

(念)威震金陵谁敢犯,一片丹心保皇朝。

本帅,曹良臣。

今有红巾贼寇,兴兵犯境。自古道:兵行千里,不战自倦。今晚末将带兵前去劫营。都督大兵随后接应,大功必成也。

得令!

【西皮摇板\protect\hyperlink{fn676}{\textsuperscript{676}}】都督传令如雷吼,扫灭红巾统貔貅。三军带马出帐口,不灭红巾誓不休。

{[}第二场{]}

【西皮摇板】旌旗招展绕星斗,金枪一举鬼神愁。三军催马朝前走,抬头只见一营头。

踹营!

哎呀!

【西皮摇板】只望劫营能成就,谁知贼有巧机谋。三军随爷绕营走,

{[}第三场{]}

【西皮摇板】适才闪出红巾寇,不由怒火起心头,三军催马踹营走,

杀!

哎呀!

【西皮摇板】只望今晚擒贼首,又恐中贼奸机谋。三军随爷夺路走,

{[}第四场{]}

【西皮摇板】四面俱是红巾寇,口口叫我把降投。东杀、西挡无路走------

然!

答话者何人?

唔哙呀!闻得徐达用兵如神,果然话不虚传。此乃天教俺归降也!

【西皮摇板】人言徐达韬略有,提兵调将似武侯。甩镫离鞍卸甲胄,含羞带愧把他投。

归降来迟,死罪呀死罪。

赤福寿人马,元帅要提防一二。\protect\hyperlink{fn677}{\textsuperscript{677}}

\item
  \leavevmode\hypertarget{fn636}{}%
  根据刘曾复先生钞录本整理。刘曾复先生钞本注``马少山本,民廿九 1940
  王存''。段公平君按:据何毅老师介绍,刘曾复先生本有意为此剧本润色文辞,后因言派《卧龙吊孝》已流行,``我就别招这个讨厌了'',竟未完成。因此抄本中讹误脱漏较多,且多有刘曾复先生修改和原文并存痕迹。这次整理,在文辞通顺的基础上拟对这些痕迹适当保留。刘曾老所改录在正文,抄本原文录在脚注中。抄本整理过程中,幸得``小豆子''老师惠赐《柴桑口》余胜荪藏本扫描件,和此抄本同质性很高,故多用为参考。\protect\hyperlink{fnref636}{↩}
\item
  \leavevmode\hypertarget{fn637}{}%
  秦即地狱秦广王,专司人间夭寿;崔即判官崔钰,掌生死簿。刘曾复先生钞本中,``童''字不确认,疑``章''字;玉童,即仙童。\protect\hyperlink{fnref637}{↩}
\item
  \leavevmode\hypertarget{fn638}{}%
  刘曾复先生钞本``周''字不确认,亦欠通;``结''字不确认,后二字缺。段公平君按:据余胜荪藏本,此四句诗为``干戈有时化玉帛,蜀吴修好结姻亲。炎汉正统有天相,不须人谋定隆兴。''和此本中的诗应该颇相关。因此推断第二句所缺两字也是``姻亲''之类。第一句``难''字不能确认,或疑为``虽'',结合文意,作``难''字似乎较合理。\protect\hyperlink{fnref638}{↩}
\item
  \leavevmode\hypertarget{fn639}{}%
  段公平君按:刘曾复先生钞本作``赞马'',似欠通,今据文意及《京剧汇编》第九十三集
  余胜荪藏本改。\protect\hyperlink{fnref639}{↩}
\item
  \leavevmode\hypertarget{fn640}{}%
  刘曾复先生钞本作``扶佐''。\protect\hyperlink{fnref640}{↩}
\item
  \leavevmode\hypertarget{fn641}{}%
  刘曾复先生钞本作``今瑜死``,此处从《三国演义》原文。\protect\hyperlink{fnref641}{↩}
\item
  \leavevmode\hypertarget{fn642}{}%
  刘曾复先生钞本未注明西皮或二黄板式,今据后文及《京剧汇编》第九十三集
  余胜荪藏本推测。\protect\hyperlink{fnref642}{↩}
\item
  \leavevmode\hypertarget{fn643}{}%
  刘曾复先生钞本``教''、``两''二字不确认,据文意推断。\protect\hyperlink{fnref643}{↩}
\item
  \leavevmode\hypertarget{fn644}{}%
  刘曾复先生钞本注``散(板)转快(板)或二六''。\protect\hyperlink{fnref644}{↩}
\item
  \leavevmode\hypertarget{fn645}{}%
  刘曾复先生钞本未注明板式,今据《京剧汇编》第九十三集
  余胜荪藏本添。\protect\hyperlink{fnref645}{↩}
\item
  \leavevmode\hypertarget{fn646}{}%
  刘曾复先生钞本原文如此,文意欠通。\protect\hyperlink{fnref646}{↩}
\item
  \leavevmode\hypertarget{fn647}{}%
  刘曾复先生钞本作``周巡'',此处从《三国演义》原文,作``周循''。\protect\hyperlink{fnref647}{↩}
\item
  \leavevmode\hypertarget{fn648}{}%
  旧时死者入殓后,其亲属穿着符合各自身分的丧服,称为``成服''。\protect\hyperlink{fnref648}{↩}
\item
  \leavevmode\hypertarget{fn649}{}%
  刘曾复先生钞本作``千刃此贼'',此处据《京剧汇编》第九十三集
  余胜荪藏本改。\protect\hyperlink{fnref649}{↩}
\item
  \leavevmode\hypertarget{fn650}{}%
  刘曾复先生钞本作``破腹挖心''。\protect\hyperlink{fnref650}{↩}
\item
  \leavevmode\hypertarget{fn651}{}%
  刘曾复先生钞本原录``如有所望'',后改``如有所忘'',段公平君注:``如有所忘''文意欠通。应为``如有所亡'',即``如有所失''意。\protect\hyperlink{fnref651}{↩}
\item
  \leavevmode\hypertarget{fn652}{}%
  荷蒙,犹``承蒙''之意。\protect\hyperlink{fnref652}{↩}
\item
  \leavevmode\hypertarget{fn653}{}%
  刘曾复先生钞本注``导(板)或散(板)''\protect\hyperlink{fnref653}{↩}
\item
  \leavevmode\hypertarget{fn654}{}%
  刘曾复先生钞本注``以下散(板)或反二黄原板''\protect\hyperlink{fnref654}{↩}
\item
  \leavevmode\hypertarget{fn655}{}%
  刘曾复先生钞本作``大数到'',此处据《京剧汇编》第九十三集
  余胜荪藏本改。\protect\hyperlink{fnref655}{↩}
\item
  \leavevmode\hypertarget{fn656}{}%
  刘曾复先生钞本仅注反二黄,今据《京剧汇编》第九十三集
  余胜荪藏本添。\protect\hyperlink{fnref656}{↩}
\item
  \leavevmode\hypertarget{fn657}{}%
  刘曾复先生钞本原录``暗兵机如风波浪'',先生改为``统雄兵如风似浪''。\protect\hyperlink{fnref657}{↩}
\item
  \leavevmode\hypertarget{fn658}{}%
  刘曾复先生钞本原录``沉兵相挡'',先生改为``陈兵相抗''。\protect\hyperlink{fnref658}{↩}
\item
  \leavevmode\hypertarget{fn659}{}%
  刘曾复先生钞本原录``烧得谁太平青浪'',先生改为``稍得遂太平景象''。\protect\hyperlink{fnref659}{↩}
\item
  \leavevmode\hypertarget{fn660}{}%
  刘曾复先生钞本原录``话未了'',先生改为``转瞬间''。\protect\hyperlink{fnref660}{↩}
\item
  \leavevmode\hypertarget{fn661}{}%
  ``范张鸡黍''指范式、张劭一起喝酒食鸡。比喻朋友之间情义与深情。刘曾复先生钞本记作``稷黍范张''。\protect\hyperlink{fnref661}{↩}
\item
  \leavevmode\hypertarget{fn662}{}%
  刘曾复先生钞本原有此两句,后改用前句。\protect\hyperlink{fnref662}{↩}
\item
  \leavevmode\hypertarget{fn663}{}%
  刘曾复先生钞本注``全体同唱,鲁(肃)末句''。\protect\hyperlink{fnref663}{↩}
\item
  \leavevmode\hypertarget{fn664}{}%
  刘曾复先生钞本原录``你看我一阵风如在康庄'',先生改。\protect\hyperlink{fnref664}{↩}
\item
  \leavevmode\hypertarget{fn665}{}%
  此处据《京剧汇编》第九十三集
  余胜荪藏本添加。\protect\hyperlink{fnref665}{↩}
\item
  \leavevmode\hypertarget{fn666}{}%
  据``中国京剧戏考''网站载《戏考》本添加。\protect\hyperlink{fnref666}{↩}
\item
  \leavevmode\hypertarget{fn667}{}%
  刘曾复先生钞本原录``趁乘谢之时,去往'',欠通,后改为``趁便即往''。\protect\hyperlink{fnref667}{↩}
\item
  \leavevmode\hypertarget{fn668}{}%
  刘曾复先生钞本作``尊教''。\protect\hyperlink{fnref668}{↩}
\item
  \leavevmode\hypertarget{fn669}{}%
  刘曾复先生钞本未注明板式,今据《京剧汇编》第九十三集
  余胜荪藏本添。\protect\hyperlink{fnref669}{↩}
\item
  \leavevmode\hypertarget{fn670}{}%
  根据刘曾复先生钞录本整理。钞本作``铁龙山'',此处从《三国演义》原文。\protect\hyperlink{fnref670}{↩}
\item
  \leavevmode\hypertarget{fn671}{}%
  根据刘曾复先生钞录的秦琼``单词本''整理。

  此戏花脸唱一支《八声甘州》,据《梅兰芳回忆录:舞台生活四十年》\textsuperscript{{[}28{]}.}记载,词句为:

  ``扬威奋勇,看愁云惨惨,杀气腾腾。鞭鞘指处,鬼神尽觉惊恐。三关怒冲千里振,八寨雄兵已成空。旌旗摇,剑戟丛,将军八面展威凤。人似虎,马如龙,伫看一战使成功!''

  《铁笼山》一剧中姜维唱的《八声甘州》即出自此戏。\protect\hyperlink{fnref671}{↩}
\item
  \leavevmode\hypertarget{fn672}{}%
  刘曾复先生钞本作``北璧关'',此处从《说唐全传》;历史上李世民破刘武周麾下尉迟敬德于美良川,是其平定北方割据势力刘武周、宋金刚的关键战役(柏壁之战)的一部分,《旧唐书》、《新唐书》和《资治通鉴》均有记载。\protect\hyperlink{fnref672}{↩}
\item
  \leavevmode\hypertarget{fn673}{}%
  刘曾复先生钞本作``起鼓照打'',此处从段公平君建议,为上下文统一改。\protect\hyperlink{fnref673}{↩}
\item
  \leavevmode\hypertarget{fn674}{}%
  刘曾复先生钞本未注明板式。\protect\hyperlink{fnref674}{↩}
\item
  \leavevmode\hypertarget{fn675}{}%
  ``不可当''犹言``不得了''之意。\protect\hyperlink{fnref675}{↩}
\item
  \leavevmode\hypertarget{fn676}{}%
  刘曾复先生钞本未注明板式,下同。\protect\hyperlink{fnref676}{↩}
\item
  \leavevmode\hypertarget{fn677}{}%
  刘曾复先生钞本注明``以上《取金陵》曹良臣''。刘曾复先生钞本未注明场次,有关场次安排据《京剧汇编》第十九集
  阎庆林藏本,该藏本系阎岚秋(九阵风)生前演出本。\protect\hyperlink{fnref677}{↩}
