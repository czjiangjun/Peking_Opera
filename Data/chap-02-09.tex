\newpage
\subsubsection{\large\hei {黄鹤楼~{\small 之}~刘备}}
\addcontentsline{toc}{subsection}{\hei 黄鹤楼~{\small 之}~刘备}

\hangafter=1                   %2. 设置从第1⾏之后开始悬挂缩进  %}
\setlength{\parindent}{0pt}{

{\centerline{{[}{\hei 第一场}{]}}}\vspace{5pt}

{[}{\akai {\akai 引}子}{]}义得人和,灭孙曹,孤心安乐。

({\akai 念})日月重明照英雄,全凭卧龙建奇功。虽得土地归王化,未能遂意高祖风。

孤,刘备,大树楼桑人氏。自与关、张结义桃园,三顾茅庐,请来卧龙先生,屡建奇功。客荆虽得安顿,只为孙、曹未得安定,叫孤常忧心也。正是:~({\akai 念})苍天遂孤意,重整汉帝基。

罢了,进帐何事?

呈上来。东吴有书信到来,待孤拆开一观。

有请先生。

先生少礼,请坐。

东吴有书信到来,先生请来观看。

此番过江,那东吴是好意,还是歹意?

既然如此,孤就不去了。

先生计将安在?

四弟少礼,见过先生。

坐下,先生有差。

呃,慢来,慢来,前番去至东吴,就是我君臣二人,险些命丧周郎之手;此番又是我君臣二人。要去你去,孤是不去的了!

\setlength{\hangindent}{56pt}{【{\akai 西皮原板}】先生把话错来讲,休提起当年赴会河梁。孙、刘仇结山海样,孤岂肯把性命送与周郎。 }

\setlength{\hangindent}{56pt}{【{\akai 西皮摇板}】他二人把话一样讲,倒教孤王少主张。回头便对先生讲,孤王言来听端详。倘若孤王东吴丧,引孤的灵魂入庙廊。 }

去,孤便去。

还是多带人马才是啊。

四弟,打开一观。

哪里是不灵,分明是孤的引魂幡喏!

呃,迎孤的灵魂吧!

\setlength{\hangindent}{56pt}{【{\akai 西皮摇板}】好个大胆诸葛亮,勒逼孤王过长江。虎穴龙潭孤去闯, }

\setlength{\hangindent}{56pt}{【{\akai 西皮散板}】你分明是送孤王去见阎王。 }

\vspace{3pt}{\centerline{{[}{\hei 第二场}{]}}}\vspace{5pt}

啊,都督,备过江来了。

都督请。

四弟子龙。

\vspace{3pt}{\centerline{{[}{\hei 第三场}{]}}}\vspace{5pt}

有坐。

远隔大江,少来问安,都督海涵。

为何不见吴侯?

告便。

进宫问安。

谨遵台命。

(周瑜\hspace{30pt}({\akai 念})相逢花中锦,)

({\akai 念})知己叙衷肠。

\vspace{3pt}{\centerline{{[}{\hei 第四场}{]}}}\vspace{5pt}

呃,大夫,备过江来了。

呵呵哈哈哈$\cdots{}\cdots{}$

有劳大夫。

大夫请便。

啊,啊,啊,都督请。

都督有何金言,当面请讲。

啊$\cdots{}\cdots{}$

呃,这$\cdots{}\cdots{}$

唉,都督哇,呃$\cdots{}\cdots{}$({\hwfs 哭介})

(周瑜\hspace{30pt}又来了。)

\setlength{\hangindent}{56pt}{【{\akai 西皮原板}】周都督他那里提前情,倒教我汉刘备有话难云。借荆州取西川以为根本,望都督禀吴侯再等几春。 }

放肆。

下站。

\setlength{\hangindent}{56pt}{【{\akai 西皮摇板}】四弟做事太莽撞,恶言恶语把人伤。周都督他倒有容人量, }

都督,四弟莽撞,备这厢赔礼了。

备这厢赔礼了。

哎,都督哇!

\setlength{\hangindent}{56pt}{【{\akai 西皮摇板}】还望都督好商量。\hspace{10pt}~ }

诸葛亮啊,害死孤王也。

\setlength{\hangindent}{56pt}{【{\akai 西皮摇板}】勒逼孤王把宴饮,黄鹤楼上遇杀星。周郎苦苦要孤命, }

四弟。

\setlength{\hangindent}{56pt}{【{\akai 西皮摇板}】想一良谋好逃生。\hspace{10pt}~ }

四弟,有何妙计?

又是他那长坂坡!

四弟,在长坂坡前,你胯下有马,掌中有枪;今日在这黄鹤楼上,难道说你拳打------足踢------不成?

那是妖道的谣言呐。

``水军都督周''。

嗯哼,真乃是孤的好先生!

四弟,搀孤下楼。

告辞了。

}
