\newpage
\phantomsection %实现目录的正确跳转
\section*{\large\hei {宫门带·十道本~{\small 之}~李渊、褚遂良}}
\addcontentsline{toc}{section}{\hei {宫门带·十道本~{\small 之}~李渊、褚遂良}}

\hangafter=1                   %2. 设置从第1⾏之后开始悬挂缩进  %}
\setlength{\parindent}{0pt}{

{\vspace{3pt}{\centerline{{[}{\hei 第一场}{]}}}\vspace{5pt}}

\setlength{\hangindent}{52pt}{李渊\hspace{30pt}【{\akai 二黄慢板}】都只为御梓童命归仙境,因此上为王的染病在身。内侍臣搀扶王龙床安定,还需要却烦虑静养精神。}

\setlength{\hangindent}{52pt}{(李世民\hspace{15pt}【{\akai 二黄摇板}】内侍摆驾进龙廷,父王台前问安宁。)}

\setlength{\hangindent}{52pt}{(李世民\hspace{15pt}儿臣见驾,父王万岁!)}

\setlength{\hangindent}{52pt}{李渊\hspace{30pt}皇儿平身。}

\setlength{\hangindent}{52pt}{(李世民\hspace{15pt}万万岁!)}

\setlength{\hangindent}{52pt}{李渊\hspace{30pt}赐座。}

\setlength{\hangindent}{52pt}{(李世民\hspace{15pt}谢座。)}

\setlength{\hangindent}{52pt}{李渊\hspace{30pt}皇儿进宫为了何事?}

\setlength{\hangindent}{52pt}{(李世民\hspace{15pt}儿臣在太医院,取得太平汤药,进宫与父王熬煎。)}

\setlength{\hangindent}{52pt}{李渊\hspace{30pt}我儿真乃孝道({\akai 或}:~孝心)。}

\setlength{\hangindent}{52pt}{(李世民\hspace{15pt}内侍,金炉伺候!)}

\setlength{\hangindent}{52pt}{(李世民\hspace{15pt}【{\akai 二黄原板}】父王将息龙床养,儿臣进宫煎药汤。屈膝跪在尘埃地,拜天拜地拜三光。但愿父王身无恙,焚香顶礼谢上苍。)}

\setlength{\hangindent}{52pt}{李渊\hspace{30pt}【{\akai 二黄原板}】儿孝心感动天和地,药下咽喉病离身。空养建成、元吉子,并不进宫问安宁。日后为父({\akai 或}:~为父日后)归仙境,儿就是东宫的守阙人。谯楼鼓打三更时分,}

\setlength{\hangindent}{52pt}{李渊\hspace{30pt}【{\akai 二黄摇板}】皇儿回避({\akai 或}:~暂且)出宫廷。}

\setlength{\hangindent}{52pt}{(李世民\hspace{15pt}【{\akai 二黄摇板}】辞别父王出宫门,)}

\setlength{\hangindent}{52pt}{(李世民\hspace{15pt}【{\akai 二黄摇板}】为何还有作乐声。)}

\setlength{\hangindent}{52pt}{(李世民\hspace{15pt}$\cdots{}\cdots{}$明白便了!~)}

\setlength{\hangindent}{52pt}{(李世民\hspace{15pt}【{\akai 二黄摇板}】听谯楼鼓打三更尽,看是何人作乐声。)}

{\vspace{3pt}{\centerline{{[}{\hei 第二场}{]}}}\vspace{5pt}}

\setlength{\hangindent}{52pt}{李渊\hspace{30pt}【{\akai 西皮摇板}】宫中服药精神爽,悼念御妻神暗伤。}

%$\bigg( \begin{aligned} &\mbox{张妃}\\&\mbox{刘妃}\mbox{\raisebox{5pt}{\hspace{22pt}万岁呐!}} \end{aligned}\bigg)$
\raisebox{0pt}[22pt][16pt]{\bigg(\raisebox{8pt}{张妃}\raisebox{-8pt}{\hspace{-22pt}{刘妃}}\raisebox{0pt}{\hspace{22pt}万岁呐!}\bigg)}

\setlength{\hangindent}{52pt}{李渊\hspace{30pt}梓童为何这等模样?}

%$\bigg( \begin{aligned} &\mbox{张妃}\\&\mbox{刘妃}\mbox{\raisebox{5pt}{\hspace{22pt}今有二主秦王,二更二点进宫调戏我二人。万岁做主!}} \end{aligned}\bigg)$
\raisebox{0pt}[22pt][16pt]{\bigg(\raisebox{8pt}{张妃}\raisebox{-8pt}{\hspace{-22pt}{刘妃}}\raisebox{0pt}{\hspace{22pt}今有二主秦王,二更二点进宫调戏我二人。万岁做主!}\bigg)}

\setlength{\hangindent}{52pt}{李渊\hspace{30pt}世民素行仁孝,为王({\akai 或}:~孤王)不信。}

%$\bigg( \begin{aligned} &\mbox{张妃}\\&\mbox{刘妃}\mbox{\raisebox{5pt}{\hspace{22pt}$\cdots{}\cdots{}$玉带为证。}} \end{aligned}\bigg)$
\raisebox{0pt}[22pt][16pt]{\bigg(\raisebox{8pt}{张妃}\raisebox{-8pt}{\hspace{-22pt}{刘妃}}\raisebox{0pt}{\hspace{22pt}玉带为证。}\bigg)}

\setlength{\hangindent}{52pt}{李渊\hspace{30pt}呈上来。}

%$\bigg( \begin{aligned} &\mbox{张妃}\\&\mbox{刘妃}\mbox{\raisebox{5pt}{\hspace{22pt}万岁请看。}} \end{aligned}\bigg)$
\raisebox{0pt}[22pt][16pt]{\bigg(\raisebox{8pt}{张妃}\raisebox{-8pt}{\hspace{-22pt}{刘妃}}\raisebox{0pt}{\hspace{22pt}万岁请看。}\bigg)}

\setlength{\hangindent}{52pt}{李渊\hspace{30pt}哎呀!}

\setlength{\hangindent}{52pt}{李渊\hspace{30pt}【{\akai 西皮摇板}】一见玉带怒气生,胆大奴才乱宫廷。你二人暂且回宫禁({\akai 或}:~后宫进),}

%$\bigg( \begin{aligned} &\mbox{张妃}\\&\mbox{刘妃}\mbox{\raisebox{5pt}{\hspace{22pt}【{\akai 西皮摇板}】$\cdots{}\cdots{}$世民丧残生。}} \end{aligned}\bigg)$
\raisebox{0pt}[22pt][16pt]{\bigg(\raisebox{8pt}{张妃}\raisebox{-8pt}{\hspace{-22pt}{刘妃}}\raisebox{0pt}{\hspace{22pt}【{\akai 西皮摇板}】$\cdots{}\cdots{}$世民丧残生。}\bigg)}

\setlength{\hangindent}{52pt}{李渊\hspace{30pt}【{\akai 西皮摇板}】内侍摆驾金殿进,快宣皇儿李世民。}

\setlength{\hangindent}{52pt}{(李世民\hspace{15pt}【{\akai 西皮摇板}】忽听父王宣世民,急忙上殿问分明。)}

\setlength{\hangindent}{52pt}{(李世民\hspace{15pt}儿臣见驾,父王万岁!)}

\setlength{\hangindent}{52pt}{李渊\hspace{30pt}儿是世民?}

\setlength{\hangindent}{52pt}{(李世民\hspace{15pt}是世民。)}

\setlength{\hangindent}{52pt}{李渊\hspace{30pt}好奴才!}

\setlength{\hangindent}{52pt}{李渊\hspace{30pt}【{\akai 西皮摇板}】把儿当作擎天柱,奴才竟是忤逆人。吩咐两旁武士手,推出午门问斩刑。}

\setlength{\hangindent}{52pt}{(李世民\hspace{15pt}【{\akai 西皮摇板}】一言未发来问斩,教我有话不敢言。因何将儿推出斩,说明儿死也心甘。)}

\setlength{\hangindent}{52pt}{李渊\hspace{30pt}【{\akai 西皮摇板}】奴才不必将父问,现有玉带作证凭。}

\setlength{\hangindent}{52pt}{(李世民\hspace{15pt}【{\akai 西皮散板}】却原来为的是联珠带,)}

\setlength{\hangindent}{52pt}{(李世民\hspace{15pt}父王,父王,父王啊,呃$\cdots{}\cdots{}$({\hwfs 哭介}))}

\setlength{\hangindent}{52pt}{(李世民\hspace{15pt}【{\akai 西皮散板}】吓得三魂少二魂。本当说出二兄长,又恐伤了手足情。望父王饶了儿的\textless{}\!{\bfseries\akai 哭头}\!\textgreater{}命,父王啊,还望看在父子情。)}

\setlength{\hangindent}{52pt}{李渊\hspace{30pt}【{\akai 西皮散板}】手摸胸膛想一想,此事可行不可行。吩咐殿前({\akai 或}:~吩咐两旁)武士手,推出午门问典刑。}

\setlength{\hangindent}{52pt}{(李世民\hspace{15pt}【{\akai 西皮摇板}】含悲忍泪下龙廷,看是何人把本升。)}

\setlength{\hangindent}{52pt}{(长孙无忌\hspace{8pt}刀下留人!)}

\setlength{\hangindent}{52pt}{(武士\hspace{30pt}啊!)}

\setlength{\hangindent}{52pt}{(长孙无忌\hspace{8pt}【{\akai 西皮摇板}】迈步撩袍上龙廷,品级台前臣见君。)}

\setlength{\hangindent}{52pt}{(长孙无忌\hspace{8pt}臣长孙无忌见驾,吾皇万岁!)}

\setlength{\hangindent}{52pt}{李渊\hspace{30pt}上殿有何本奏?}

\setlength{\hangindent}{52pt}{(长孙无忌\hspace{8pt}二主秦王身犯何罪,$\cdots{}\cdots{}$午门问斩?)}

\setlength{\hangindent}{52pt}{李渊\hspace{30pt}蠢子不正,扰乱宫廷,故而问斩。}

\setlength{\hangindent}{52pt}{(长孙无忌\hspace{8pt}想秦王有十大汗马功劳,只可一赦,不可一斩。)}

\setlength{\hangindent}{52pt}{李渊\hspace{30pt}孤王龙心已定,定斩不赦。}

\setlength{\hangindent}{52pt}{(长孙无忌\hspace{8pt}万岁呀!)}

\setlength{\hangindent}{52pt}{(长孙无忌\hspace{8pt}【{\akai 西皮摇板}】当年驾坐太原省,隋炀帝无道灭人伦。二主大战王世充,才保我主坐龙廷。\nolinebreak)}

\setlength{\hangindent}{52pt}{李渊\hspace{30pt}呃!({\akai 或}:~唗!)}

\setlength{\hangindent}{52pt}{李渊\hspace{30pt}【{\akai 西皮摇板}】无忌奏本太欺情,敢在金殿藐寡人。吩咐殿前({\akai 或}:~吩咐两旁)武士手,他与奴才同罪名。}

\setlength{\hangindent}{52pt}{李渊\hspace{30pt}绑了下去!}

\setlength{\hangindent}{52pt}{(长孙无忌\hspace{8pt}【{\akai 西皮摇板}】$\cdots{}\cdots{}$,看是何人保我生。)}

{\vspace{3pt}{\centerline{{[}{\hei 第三场}{]}}}\vspace{5pt}}

\setlength{\hangindent}{52pt}{(徐勣\hspace{30pt}【{\akai 西皮摇板}】一见秦王上了刑,不由徐勣心内惊。$\cdots{}\cdots{}$忙往龙殿奔,)}

%$\bigg( \begin{aligned} &\mbox{秦琼}\\&\mbox{程咬金}\mbox{\raisebox{5pt}{\hspace{16pt}先生慢行。}} \end{aligned}\bigg)$
\raisebox{0pt}[22pt][16pt]{\bigg(\raisebox{8pt}{秦琼}\raisebox{-8pt}{\hspace{-22pt}{程咬金}}\raisebox{0pt}{\hspace{16pt}先生慢行。}\bigg)}

%$\bigg( \begin{aligned} &\mbox{秦琼}\\&\mbox{程咬金}\mbox{\raisebox{5pt}{\hspace{16pt}【{\akai 西皮摇板}】见了先生礼相迎。}} \end{aligned}\bigg)$
\raisebox{0pt}[22pt][16pt]{\bigg(\raisebox{8pt}{秦琼}\raisebox{-8pt}{\hspace{-22pt}{程咬金}}\raisebox{0pt}{\hspace{16pt}【{\akai 西皮摇板}】见了先生礼相迎。}\bigg)}

\setlength{\hangindent}{52pt}{(徐勣\hspace{30pt}二公慌慌张张为了何事?)}

%$\bigg( \begin{aligned} &\mbox{秦琼}\\&\mbox{程咬金}\mbox{\raisebox{5pt}{\hspace{16pt}二主秦王不知身犯何罪,推出午门斩首。我等上殿保本。}} \end{aligned}\bigg)$
\raisebox{0pt}[22pt][16pt]{\bigg(\raisebox{8pt}{秦琼}\raisebox{-8pt}{\hspace{-22pt}{程咬金}}\raisebox{0pt}{\hspace{16pt}二主秦王不知身犯何罪,推出午门斩首。我等上殿保本。}\bigg)}

\setlength{\hangindent}{52pt}{(徐勣\hspace{30pt}此本你我保不下来。有人来了,你我暂退朝房便了。)}

\setlength{\hangindent}{52pt}{(徐勣\hspace{30pt}【{\akai 西皮摇板}】三人一同朝房进,)}

\setlength{\hangindent}{52pt}{褚遂良\hspace{20pt}({\akai 内})先生慢走!}

\setlength{\hangindent}{52pt}{(徐勣\hspace{30pt}【{\akai 西皮摇板}】那旁来了褚先生。)}

\setlength{\hangindent}{52pt}{(褚遂良\hspace{20pt}反了哇,反了呃!)}

\setlength{\hangindent}{52pt}{褚遂良\hspace{20pt}【{\akai 西皮散板}】听说要斩二主君呐,斩断了({\akai 或}:~斩坏了)擎天柱一根。万岁不听({\akai 或}:~万岁不准)忠良本,长孙无忌问斩刑。这都是二奸妃用计狠,谁知我主假作真。}

\setlength{\hangindent}{52pt}{褚遂良\hspace{20pt}哎呀!}

\setlength{\hangindent}{52pt}{褚遂良\hspace{20pt}这,这$\cdots{}\cdots{}$,罢!({\akai 或}:~这,这,这$\cdots{}\cdots{}$哎呀!~罢!)}

\setlength{\hangindent}{52pt}{褚遂良\hspace{20pt}【{\akai 西皮散板}】歪戴乌纱斜插带,假装疯魔去见君。大摇大摆金殿进,}

\setlength{\hangindent}{52pt}{褚遂良\hspace{20pt}【{\akai 西皮散板}】与他个君不君来臣不臣。}

\setlength{\hangindent}{52pt}{褚遂良\hspace{20pt}臣,褚遂良见驾,吾主万岁,万万岁!呃呃呃,请了!}

\setlength{\hangindent}{52pt}{李渊\hspace{30pt}呃嗯------胆大褚遂良,上得殿来衣冠不整,莫非你疯了?({\akai 或}:~呃嗯------卿家莫非你疯了?)}

\setlength{\hangindent}{52pt}{褚遂良\hspace{20pt}呃,臣倒不曾疯啊,只恐({\akai 或}:~只怕)万岁你昏了。二主秦王身犯何罪,推出午门斩首?}

\setlength{\hangindent}{52pt}{李渊\hspace{30pt}奴才扰乱宫廷,因此斩首!}

\setlength{\hangindent}{52pt}{褚遂良\hspace{20pt}想二主秦王,东挡西杀,南征北剿,有十大汗马功劳。将他斩首,君心何忍,这臣心何安呐?!}

\setlength{\hangindent}{52pt}{褚遂良\hspace{20pt}【{\akai 西皮快板}】想当年驾坐太原郡,三搜晋阳才为君。二主大战王世充,瓦岗寨收下众英雄。美良川,收敬德,千秋岭下收罗成。大唐收了罗世信,才保我主坐龙廷。挣来的江山多安稳,为何要斩创业人。}

\setlength{\hangindent}{52pt}{李渊\hspace{30pt}呃嗯------胆大褚遂良,上殿言君之过,绑了!}

\setlength{\hangindent}{52pt}{褚遂良\hspace{20pt}万岁!臣有十道条陈,容臣奏完,(诶,)再斩不迟。}

\setlength{\hangindent}{52pt}{李渊\hspace{30pt}呈上龙案,寡人御览。}

\setlength{\hangindent}{52pt}{褚遂良\hspace{20pt}臣修本不及,乃是口奏。}

\setlength{\hangindent}{52pt}{李渊\hspace{30pt}奏来。}

\setlength{\hangindent}{52pt}{褚遂良\hspace{20pt}容奏:~臣这第一道条陈奏的是夏禹王坐了一十七代,四百五十八载。后出一君,名曰桀王,宠爱一妃,名唤妹喜。那桀王听信妹喜之言,以酒为池,以肉为林,忠臣良将,俱已遭害呀。}

\setlength{\hangindent}{52pt}{褚遂良\hspace{20pt}【{\akai 西皮快板}】自古道有道反无道,汤王定计安黎民。南巢岭桀王丧了命,只落得江山一旦倾。}

\setlength{\hangindent}{52pt}{李渊\hspace{30pt}大胆褚遂良,毁谤孤王。武士手,绑了!}

\setlength{\hangindent}{52pt}{褚遂良\hspace{20pt}啊,万岁!臣只奏过一道,还有九道未奏啊,容臣奏完,诶,再斩不迟。}

\setlength{\hangindent}{52pt}{李渊\hspace{30pt}奏来!}

\setlength{\hangindent}{52pt}{褚遂良\hspace{20pt}容奏:~臣这第二道条陈奏的是成汤王得了桀王天下,传至三十一代。后出一君,名曰纣王,宠爱一妃,名叫妲己。他驾前有两个谗臣,一名费仲,一名尤浑。那纣王听信妲己之言,盖一楼名曰``摘星(楼)''。造下炮烙之刑,糟害百姓({\akai 或}:~残虐百姓)。比干丞相剖心而亡,贾氏夫人坠楼而死,姜后娘娘挖目剁手({\akai 或}:~剜目剁手),东宫太子一旦逐出,黄家父子反出五关。到后来姜尚兴兵伐纣,可叹那纣王啊,只落得火焚摘星楼台而亡。万岁,你看他也是宠爱奸妃的无道昏君喏。}

\setlength{\hangindent}{52pt}{李渊\hspace{30pt}呃------}

\setlength{\hangindent}{52pt}{李渊\hspace{30pt}【{\akai 西皮摇板}】褚遂良奏本孤心恨,把孤比作无道君。寡人至德平天下,学尧舜不差半毫分。}

\setlength{\hangindent}{52pt}{李渊\hspace{30pt}再将三道条陈奏来!}

\setlength{\hangindent}{52pt}{褚遂良\hspace{20pt}容奏:~臣这道条陈奏的是周朝。那周文王得了纣王天下,后出一君,名曰幽王,宠爱一妃,名曰褒姒,生得(是)面貌如花。怎奈进宫以来,
永无笑容。那幽王无计可施,他驾前有一谗臣,名叫尹球。是他奏道:~万岁要娘娘发笑不难。在骊山设宴,火焚烟墩。那幽王听信尹球所奏,就在骊山设宴,火焚烟墩。
各路诸侯见烟墩火起,想必国家有难,一个个顶盔贯甲,兵临城下。观见他君妃在楼台饮酒取乐哇,一个个乘兴而来,(是)败兴而返呐;那褒姒一见是呵呵地大笑。后来犬戎作乱,那幽王又将烟墩点起。各路诸侯言道:~想必(又是)他君妃(又)在那里饮酒取乐啊,你我各保汛地}\footnote{{汛地是明、清时代称军队驻防地段。``汛''通``讯'',``讯地''即为军事烽火之地,以传消息,地界不大,故而汛地为基本驻防之地。}}{要紧。(一个个是按兵不动。啊)万岁,你看那幽王为褒姒一笑不值紧要啊,失落周室家邦,他还死在了乱军之中。}

\setlength{\hangindent}{52pt}{褚遂良\hspace{20pt}【{\akai 西皮快板}】幽王无道掌乾坤,骊山设宴焚烟墩。各路诸侯无救应,江山一旦化灰尘。}

\setlength{\hangindent}{52pt}{李渊\hspace{30pt}幽王无道,戏耍诸侯,提他则甚?再将四道条陈奏来。}

\setlength{\hangindent}{52pt}{褚遂良\hspace{20pt}容奏:~臣这道条陈奏的是东周列国,周惠王驾前有一家诸侯,名曰晋献公。他({\akai 或}:~那晋献公)宠爱一妃名唤骊姬。前妃所生二子,长子申生,次子重耳。那骊姬在献公面前搬动是非,要害(那)申生太子一死,那献公是执意地不听呐。骊姬一计不成,又生二计。用蜂蜜擦头,到御花园观花({\akai 或}:~采花),命申生太子保驾采花。蜜蜂围绕头上,申生太子不解其意,在后面用扇搧开。那献公在楼台之上观见,言道:~这奴才果有戏母({\akai 或}:~残母}\footnote{此处``残母''可能是``戏母''之误。}{)之心。吩咐殿前武士,将({\akai 或}:~把)申生太子推出午门问斩。来在午门,众大臣拦路言道:~
千岁(你)有满腹含冤,为何不奏知你父王?申生太子言道:~我若奏知我父王,我父王大怒,必将骊姬斩首。斩了骊姬不关紧要哇,有日我父王思想骊姬成病,岂不是小王之罪?小王只可一死,不做那不忠不孝之人。万岁,为臣看来,二主秦王与前朝申生太子一般无二。}

\setlength{\hangindent}{52pt}{(太监\hspace{30pt}着啊!)}

\setlength{\hangindent}{52pt}{(李渊\hspace{30pt}呃嗯------)}

\setlength{\hangindent}{52pt}{李渊\hspace{30pt}【{\akai 西皮摇板}】晋献公本是无道君,听信谗言斩亲生({\akai 或}:~听信谗言斩申生)。世民本是不肖子,淫乱宫闱问斩刑。}

\setlength{\hangindent}{52pt}{李渊\hspace{30pt}再将五道条陈奏来!}

\setlength{\hangindent}{52pt}{褚遂良\hspace{20pt}呃呃,臣这道条陈奏的是楚平王在临潼斗宝,多亏伍子胥力举千斤鼎,压定各国为下邦。到后来秦、楚结亲,(那)楚平王闻得({\akai 或}:~楚平王闻听)无祥女生得是天姿国色,有意纳妾,怎奈儿媳不好启齿啊。他驾前有一谗臣,名叫费无极,奉旨(前)往秦国迎亲。行至在钟离山前,用金顶轿改换银顶轿,无祥女改换马昭仪。好个伍子胥,保定皇家四口反出昭关,去往吴国借兵。可叹那平王(啊,他)死后,只落得鞭尸三百有余。}

\setlength{\hangindent}{52pt}{褚遂良\hspace{20pt}【{\akai 西皮快板}】楚平王本是无道君,父纳子妻乱人伦。子胥后来【{\footnotesize 转}{\akai 西皮摇板}】发人马,鞭尸三百留骂名。}

\setlength{\hangindent}{52pt}{李渊\hspace{30pt}那父纳子妻,乃(是)酒色昏王,提他则甚?再将六道条陈奏来。}

\setlength{\hangindent}{52pt}{褚遂良\hspace{20pt}呃,呃,臣这六道条陈奏的是姑苏吴王宠爱一妃,名曰(西施或:~名唤西施)。他驾前有一谗臣,名唤伯嚭。那吴王听信西施、伯嚭之言,起造一台,名曰姑苏台。挑选天下出色的女子,去往楼台饮酒取乐。到后来勾践兴兵前来,只杀得那吴王有家难奔,有国难投哇。}

\setlength{\hangindent}{52pt}{李渊\hspace{30pt}呃------嗯。}

\setlength{\hangindent}{52pt}{李渊\hspace{30pt}【{\akai 西皮摇板}】姑苏吴王无道君,听信谗言选红裙。越王勾践发人马,吴国从此不太平。}

\setlength{\hangindent}{52pt}{李渊\hspace{30pt}再将七道条陈奏来。}

\setlength{\hangindent}{52pt}{褚遂良\hspace{20pt}容奏:~臣这道条陈奏的是齐宣王在桑园射猎,收来一妃,名唤无盐,手持春秋大棒({\akai 或}:~春秋大棍)压定各国。只因宠爱一妃,名曰夏迎春。那无盐娘娘身怀六甲,那夏迎春讨下收生代劳的旨意,用金丝狸猫剥去皮尾。启奏大王言道:~那无盐娘娘产生妖魔鬼怪。齐宣王大怒,将无盐娘娘推出斩首,多亏满朝文武保奏,打入冷宫。后来吴起伐齐,只落得跪门求救哇。}

\setlength{\hangindent}{52pt}{褚遂良\hspace{20pt}【{\akai 西皮摇板}】齐宣王本是无道君,宠爱奸妃夏迎春。后来吴起发人马,只落得跪门去求兵。}

\setlength{\hangindent}{52pt}{李渊\hspace{30pt}齐宣王宠妃害贤,怎比孤王?将八道条陈奏来!}

\setlength{\hangindent}{52pt}{褚遂良\hspace{20pt}容奏:~臣这道条陈奏的是齐湣王宠爱一妃,名曰邹赛花。他驾前有一宦官名叫伊立。那湣王听信邹妃之言,要害东宫太子一死。后来乐毅兴兵,前来追赶湣王。赶得他有家难奔,有国难投。只落得日晒湣王,路剐邹妃({\akai 或}:~路卧}\footnote{段公平君指出,``路卧''可能是``路剐''之误。}{邹妃)。万岁!这就是前朝宠妃灭子的报应喏!}

\setlength{\hangindent}{52pt}{李渊\hspace{30pt}呃------}

\setlength{\hangindent}{52pt}{李渊\hspace{30pt}【{\akai 西皮摇板}】宠妃灭子害忠臣,他将湣王比寡人。待等奏完十道本,定与奴才同罪名。}

\setlength{\hangindent}{52pt}{李渊\hspace{30pt}再将九道条陈奏来。}

\setlength{\hangindent}{52pt}{褚遂良\hspace{20pt}容奏:~臣这道条陈奏的是前朝杨广欺娘奸妹,败坏人伦,后来亡国丧身。}

\setlength{\hangindent}{52pt}{褚遂良\hspace{20pt}【{\akai 西皮摇板}】杨广本是无道君,欺娘奸妹乱宫廷。五花棒奸王丧了命,才保我主坐龙廷。}

\setlength{\hangindent}{52pt}{李渊\hspace{30pt}将十道条陈奏完,孤王定要将你碎尸万段。}

\setlength{\hangindent}{52pt}{褚遂良\hspace{20pt}这个$\cdots{}\cdots{}$}

\setlength{\hangindent}{52pt}{(太监\hspace{30pt}我说褚先生,十道条陈奏了九道,只管奏来,自有咱家帮助于你呃。)}

\setlength{\hangindent}{52pt}{褚遂良\hspace{20pt}万岁,臣这十道条陈奏的是前朝君王与本朝皇帝一般无二。}

\setlength{\hangindent}{52pt}{李渊\hspace{30pt}唗!}

\setlength{\hangindent}{52pt}{李渊\hspace{30pt}【{\akai 西皮摇板}】褚遂良奏本孤心恨,道道条陈刺寡人。吩咐殿前武士手,推出午门问斩刑。}

\setlength{\hangindent}{52pt}{褚遂良\hspace{20pt}冤枉------}

\setlength{\hangindent}{52pt}{武士\hspace{30pt}褚遂良冤枉。}

\setlength{\hangindent}{52pt}{李渊\hspace{30pt}召回来!}

\setlength{\hangindent}{52pt}{武士\hspace{30pt}啊!}

\setlength{\hangindent}{52pt}{褚遂良\hspace{20pt}谢万岁不斩之恩。}

\setlength{\hangindent}{52pt}{李渊\hspace{30pt}非是寡人不斩于你,为何口喊冤枉?}

\setlength{\hangindent}{52pt}{褚遂良\hspace{20pt}臣有一事不明,要在万岁驾前领教!}

\setlength{\hangindent}{52pt}{李渊\hspace{30pt}何事不明?}

\setlength{\hangindent}{52pt}{褚遂良\hspace{20pt}二主秦王什么时候进宫?}

\setlength{\hangindent}{52pt}{李渊\hspace{30pt}一更一点。}

\setlength{\hangindent}{52pt}{褚遂良\hspace{20pt}什么时候煎汤熬药?}

\setlength{\hangindent}{52pt}{李渊\hspace{30pt}二更二点。}

\setlength{\hangindent}{52pt}{褚遂良\hspace{20pt}什么时候出宫?}

\setlength{\hangindent}{52pt}{李渊\hspace{30pt}三更三点。}

\setlength{\hangindent}{52pt}{褚遂良\hspace{20pt}二位娘娘({\akai 或}:~皇娘)奏道,抓袍夺带是什么时间({\akai 或}:~什么时候)?}

\setlength{\hangindent}{52pt}{李渊\hspace{30pt}这个$\cdots{}\cdots{}$}

\setlength{\hangindent}{52pt}{(太监\hspace{30pt}二更二点。)}

\setlength{\hangindent}{52pt}{褚遂良\hspace{20pt}着哇,二主秦王,一更一点进宫,二更二点煎汤熬药,三更三点才得出宫。二位娘娘({\akai 或}:~二位皇娘)奏的是二更二点,这不是({\akai 或}:~岂不是)冤枉吗?}

\setlength{\hangindent}{52pt}{李渊\hspace{30pt}现有玉带,拿去看来。}

\setlength{\hangindent}{52pt}{褚遂良\hspace{20pt}待臣看来。}

\setlength{\hangindent}{52pt}{褚遂良\hspace{20pt}万岁,我想这抓袍夺带,必定是你这一拉,我这一扯。这玉带之上并无一点伤损,岂不是大大的冤枉么?}

\setlength{\hangindent}{52pt}{李渊\hspace{30pt}寡人看来({\akai 或}:~待孤看来)。}

\setlength{\hangindent}{52pt}{褚遂良\hspace{20pt}万岁请看。}

\setlength{\hangindent}{52pt}{李渊\hspace{30pt}嘿嘿!}

\setlength{\hangindent}{52pt}{褚遂良\hspace{20pt}嘿嘿!}

\setlength{\hangindent}{52pt}{李渊\hspace{30pt}【{\akai 西皮摇板}】寡人如醉方才醒,险些错斩李世民。}

\setlength{\hangindent}{52pt}{李渊\hspace{30pt}【{\akai 西皮摇板}】孤王急忙下龙廷,手提羊毫写分明:~一赦皇儿李世民,二赦长孙无忌卿。忙将赦旨交与你,快到法场走一程。}

\setlength{\hangindent}{52pt}{褚遂良\hspace{20pt}【{\akai 西皮散板}】手捧赦旨下龙廷,笑坏了两班文武臣。文班中笑坏了徐勣先生,武班中笑坏了叔宝、咬金二位({\akai 或}:~众位)将军。都道我褚遂良不怕死。}

\setlength{\hangindent}{52pt}{褚遂良\hspace{20pt}哈哈,哈哈,啊呵呵哈哈哈$\cdots{}\cdots{}$({\hwfs 笑介})}

{\vspace{3pt}{\centerline{{[}{\hei 第四场}{]}}}\vspace{5pt}}

\setlength{\hangindent}{52pt}{(李世民\hspace{15pt}【{\akai 西皮摇板}】父王传旨斩世民,)}

\setlength{\hangindent}{52pt}{(长孙无忌\hspace{5pt}【{\akai 西皮摇板}】听信谗言斩忠臣。)}

\setlength{\hangindent}{52pt}{(李世民\hspace{15pt}【{\akai 西皮摇板}】忍泪含悲法场进,)}

\setlength{\hangindent}{52pt}{(长孙无忌\hspace{5pt}【{\akai 西皮摇板}】两眼睁睁等时辰。)}

\setlength{\hangindent}{52pt}{褚遂良\hspace{20pt}赦旨下。}

\setlength{\hangindent}{52pt}{(武士\hspace{30pt}赦旨下。)}

\setlength{\hangindent}{52pt}{(李世民\hspace{15pt}接旨。)}

\setlength{\hangindent}{52pt}{褚遂良\hspace{20pt}圣旨下。跪!)}

%$\bigg( \begin{aligned} &\mbox{李世民}\\&\mbox{长孙无忌}\mbox{\raisebox{5pt}{\hspace{4pt}万岁!}} \end{aligned}\bigg)$
	\raisebox{0pt}[22pt][16pt]{\bigg(\raisebox{8pt}{李世民}\raisebox{-8pt}{\hspace{-34pt}{长孙无忌}}\raisebox{0pt}{\hspace{6pt}万岁!}\bigg)}

\setlength{\hangindent}{52pt}{褚遂良\hspace{20pt}听宣读。诏曰:~只因孤王误听谗言,错斩皇儿李世民与国舅长孙无忌。多亏褚遂良保奏,将他二人赦回金殿加封。旨意读罢({\akai 或}:~旨意读奏),望诏谢恩。}

%$\bigg( \begin{aligned} &\mbox{李世民}\\&\mbox{长孙无忌}\mbox{\raisebox{5pt}{\hspace{4pt}万万岁!}} \end{aligned}\bigg)$
	\raisebox{0pt}[22pt][16pt]{\bigg(\raisebox{8pt}{李世民}\raisebox{-8pt}{\hspace{-34pt}{长孙无忌}}\raisebox{0pt}{\hspace{6pt}万万岁!}\bigg)}

\setlength{\hangindent}{52pt}{褚遂良\hspace{20pt}请过圣命。}

\setlength{\hangindent}{52pt}{(李世民\hspace{15pt}有劳先生保奏。)}

\setlength{\hangindent}{52pt}{褚遂良\hspace{20pt}保本来迟,千岁恕罪。}

\setlength{\hangindent}{52pt}{(李世民\hspace{15pt}岂敢。)}

\setlength{\hangindent}{52pt}{褚遂良\hspace{20pt}一同上殿交旨。}

\setlength{\hangindent}{52pt}{(李世民\hspace{15pt}请!)}

{\vspace{3pt}{\centerline{{[}{\hei 第五场}{]}}}\vspace{5pt}}

\setlength{\hangindent}{52pt}{李渊\hspace{30pt}({\akai 念})可恨奸妃做事错,平白无故起风波。}

\setlength{\hangindent}{52pt}{褚遂良\hspace{20pt}({\akai 念})忙将赦旨事,启奏万岁知。}

\setlength{\hangindent}{52pt}{褚遂良\hspace{20pt}启万岁:~二主千岁、长孙无忌宣到。}

\setlength{\hangindent}{52pt}{李渊\hspace{30pt}宣他二人冠带上殿!}

\setlength{\hangindent}{52pt}{褚遂良\hspace{20pt}二人冠带上殿!}

\setlength{\hangindent}{52pt}{(李世民\hspace{15pt}({\akai 念})法场得活命,)}

\setlength{\hangindent}{52pt}{(长孙无忌\hspace{8pt}({\akai 念})死而又复生。)}

\setlength{\hangindent}{52pt}{(李世民\hspace{15pt}儿臣李世民$\cdots{}\cdots{}$)}

\setlength{\hangindent}{52pt}{(长孙无忌\hspace{8pt}臣长孙无忌,)}

%$\bigg( \begin{aligned} &\mbox{李世民}\\&\mbox{长孙无忌}\mbox{\raisebox{5pt}{\hspace{4pt}谢万岁不斩之恩。}} \end{aligned}\bigg)$
	\raisebox{0pt}[22pt][16pt]{\bigg(\raisebox{8pt}{李世民}\raisebox{-8pt}{\hspace{-34pt}{长孙无忌}}\raisebox{0pt}{\hspace{6pt}谢万岁不斩之恩。}\bigg)}

\setlength{\hangindent}{52pt}{李渊\hspace{30pt}皇儿、国舅平身。赐座。}

\setlength{\hangindent}{52pt}{(李世民\hspace{15pt}谢座。)}

\setlength{\hangindent}{52pt}{李渊\hspace{30pt}长孙无忌为皇儿误受一绑,加升三级,免朝一月。下殿!}

\setlength{\hangindent}{52pt}{(长孙无忌\hspace{8pt}谢万岁!)}

\setlength{\hangindent}{52pt}{李渊\hspace{30pt}褚遂良上殿听封。}

\setlength{\hangindent}{52pt}{褚遂良\hspace{20pt}臣有本启奏。}

\setlength{\hangindent}{52pt}{李渊\hspace{30pt}奏来!}

\setlength{\hangindent}{52pt}{褚遂良\hspace{20pt}臣不愿加官封赠。}

\setlength{\hangindent}{52pt}{李渊\hspace{30pt}愿者何来?}

\setlength{\hangindent}{52pt}{褚遂良\hspace{20pt}请我主差哪部大臣,将宫中查明。}

\setlength{\hangindent}{52pt}{李渊\hspace{30pt}赐座。}

\setlength{\hangindent}{52pt}{褚遂良\hspace{20pt}谢座。}

\setlength{\hangindent}{52pt}{李渊\hspace{30pt}皇儿,你二姨母怎样害你({\akai 或}:~怎生害你),一一奏来!}

\setlength{\hangindent}{52pt}{(李世民\hspace{15pt}父王啊!)}

\setlength{\hangindent}{52pt}{(李世民\hspace{15pt}【{\akai 二黄原板}】未开言不由人珠泪滚滚,尊父王听儿臣细说分明:~二皇兄与姨母行事不正,儿戏君妃乱胡行。儿本当进宫细查问,又恐失了手足情。因此上将玉带宫门挂定,这就是一桩桩一件件父王详情。)}

\setlength{\hangindent}{52pt}{李渊\hspace{30pt}【{\akai 二黄原板}】劝皇儿休得要【{\footnotesize 转}{\akai 二黄快三眼}】珠泪滚滚,为父的心中明如灯:~将二妃贬至在冷宫禁,她自羞自惭自丧生。为江山儿何曾略得安静,为江山东挡西除、南征北剿未享安宁。今日里儿活命实称万幸,改日里过府去酬谢先生。武德君迈虎步忙下九重,}

\setlength{\hangindent}{52pt}{(褚遂良{\hwfs 跪})}

\setlength{\hangindent}{52pt}{李渊\hspace{30pt}【{\akai 二黄快三眼}】用手儿挽定了褚先生({\akai 或}:~搀扶起褚先生)。满朝中文武臣袖手不问,怎当得先生你赤胆忠心。为皇儿把卿家的心血用尽,为皇儿哪顾得费尽辛勤。为皇儿在朝房一番议论({\akai 或}:~一番争论),为皇儿可算得擎天柱一根。(为皇儿把卿家的心血用尽,为皇儿哪顾得费尽辛勤。)为皇儿假装作疯魔急病,为皇儿衣冠不整来见当今。为皇儿把君臣大礼全然不论,为皇儿哪顾得舍死忘生。为皇儿连奏过十道表本,为皇儿把夏桀与商纣、前朝后代历代的昏王一代一代比与孤听。加封你吏部大堂带管那都察院,太子少保伴君正卿({\akai 或}:~太子少保外加正卿;或:~太子少保陪伴寡人)。再赐你尚方剑如山压定,【{\akai 垛板}】压定了九卿四相、满朝文武、大小的官员哪一个不遵,先斩后奏启奏寡人,你是捍国}\footnote{``捍国''亦可作``扞国''。``干国良臣''是旧时戏曲中常见词汇,``干国''是治理国家之意。干的本意是盾牌,引申为捍卫、保卫之意。``干国良臣''即``保国良臣''、``治国良臣''。个人以为此处``捍国''与``干国''同意,侧重``保国''之意。}{的良臣。}

\setlength{\hangindent}{52pt}{褚遂良\hspace{20pt}【{\akai 二黄原板}】非是臣我不愿({\akai 或}:~我不爱)加官封赠,为的是我主锦乾坤。从今后主休听宫闱谗本({\akai 或}:~主休听宫中谗本),普天下众黎民乐享太平,都道你是(海不扬波是一个)有道明君。}

\setlength{\hangindent}{52pt}{李渊\hspace{30pt}【{\akai 二黄原板}】好一个孝道李世民,赤胆忠心褚先生。孤的皇儿残生性命亏你救应,命皇儿与先生结为师生。侍内臣把酒宴宫中摆定,孤与那皇儿、先生来压惊。左手带定世民子,右手带定褚先生。孤的皇儿李世民,孤的爱卿褚先生,你本是皇儿的恩人、孤的爱卿。劝皇儿休流泪、免悲声,放大胆一步一步随定寡人。({\akai 或}:~孤的皇儿李世民,孤的爱卿褚先生,你二人一步一步随定寡人。)}

}
