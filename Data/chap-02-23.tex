\newpage\hspace{30pt}~
\phantomsection %实现目录的正确跳转
\section*{\large\hei {除三害~{\small 之}~王浚、周处}}
\addcontentsline{toc}{section}{\hei 除三害~{\small 之}~王浚、周处}

\hangafter=1                   %2. 设置从第1⾏之后开始悬挂缩进  %
\setlength{\parindent}{0pt}{

\vspace{3pt}{\centerline{{[}{\hei 第一场}{]}}}\vspace{5pt}

{\setlength{\hangindent}{52pt}{(王浚\hspace{30pt}【{\akai 西皮快二六}】趁青年你莫当朝嬉夕宴\footnote{李楠{\scriptsize 君}认为作``朝喜夕厌''。},董仲舒他三载未曾窥园。幼而学壮而行经纶开展,那时节报皇家荣耀门田。若得儿洗旧污重新向善,不唯对儿那高、曾、祖,我二老亦可对湛湛青天。)\footnote{这一场戏一般省去,是王浚教训自己的儿子所唱。} }

{\vspace{3pt}{\centerline{{[}{\hei 第二场}{]}}}\vspace{5pt}}

{(王浚\hspace{30pt}({\akai 内})走哇。)}

\setlength{\hangindent}{52pt}{王浚\hspace{30pt}【{\akai 二黄摇板}】摘去乌纱换儒巾,谁人识我大元勋。 }

\setlength{\hangindent}{52pt}{王浚\hspace{30pt}老夫({\akai 或}:~下官)王浚,散操回衙,黎民百姓状告恶霸周处。我想周处乃周舫之子,我若将他(当真)查办,教我怎能({\akai 或}:~教我怎样)对得过他那去世先人?为此乔装改扮,出衙私访于他。用言语打动,若({\akai 或}:~倘)能改邪归正亦未可知。只是教我哪里去寻,哪里去找?}

{王浚}\hspace{30pt}看那旁来一红脸大汉,想是周处。我不免在此等候。等他到来,他有来言,我有去语。

\setlength{\hangindent}{52pt}{{王浚\hspace{30pt}【{\akai 二黄摇板}】浑玉不琢({\akai 或}:~璞玉不琢)多壑陵,当头棒喝返本真。 }

\setlength{\hangindent}{52pt}{周处\hspace{30pt}【{\akai 二黄摇板}】终日饮酒消愁闷,半世悠悠困风云。 }

{王浚}\hspace{30pt}唉!

\setlength{\hangindent}{52pt}{周处\hspace{30pt}【{\akai 二黄摇板}】老丈缘何冲天恨, }

{王浚}\hspace{30pt}唉,不成世界了!

周处\hspace{30pt}啊?!

\setlength{\hangindent}{52pt}{周处\hspace{30pt}【{\akai 二黄摇板}】叫人心中解不明。 }

周处\hspace{30pt}啊,老丈------请了。

{王浚}\hspace{30pt}哦,原来是一位壮士。(这厢有礼。)

{周处\hspace{30pt}啊,老丈,为何一人在此长叹?莫非有人欺压于你?}

{王浚}\hspace{30pt}想老汉({\akai 或}:~老朽)乃(是)唾面自干之人,纵有人欺压于我,亦何敢较量。

{周处\hspace{30pt}既然如此,为何在此长叹?}

{王浚\hspace{30pt}唉,可叹这宜兴的百姓好不苦也。}

{周处\hspace{30pt}却是为何?}

王浚\hspace{30pt}皆因此地出了三害。

{周处\hspace{30pt}哦,出了三害?但不知是哪三害?}

王浚\hspace{30pt}壮士愿听({\akai 或}:~壮士愿闻)?

{周处\hspace{30pt}愿闻。}

王浚\hspace{30pt}愿闻({\akai 或}:~愿听)?

{周处\hspace{30pt}你且讲来。}

{王浚}\hspace{30pt}听了------

\setlength{\hangindent}{52pt}{{王浚}\hspace{30pt}【{\akai 二黄三眼}】若提起这三害令人可恨, }

{周处\hspace{30pt}你慢慢讲来。}

\setlength{\hangindent}{52pt}{{王浚}\hspace{30pt}【{\akai 二黄三眼}】讲出来连壮士({\akai 或}:~连壮士闻此言)也要心惊:~ }

{周处\hspace{30pt}第一害------}

\setlength{\hangindent}{52pt}{{王浚}\hspace{30pt}【{\akai 二黄三眼}】第一害那南山出了猛虎, }

{周处\hspace{30pt}哦,出了猛虎便怎样?}

\setlength{\hangindent}{52pt}{{王浚}\hspace{30pt}【{\akai 二黄三眼}】它遇着({\akai 或}:~倘遇着)行路人骨肉全吞。 }

{周处\hspace{30pt}嗯,这第二害------}

\setlength{\hangindent}{52pt}{{王浚}\hspace{30pt}【{\akai 二黄三眼}】第二害它比那猛虎还狠, }

{周处\hspace{30pt}哦,那又是什么妖魔鬼怪?}

\setlength{\hangindent}{52pt}{{王浚\hspace{30pt}【}二黄三眼{】长桥下又出了恶魔蛟精。} }

{周处\hspace{30pt}哦,出了蛟精,它是怎样的厉害?}

\setlength{\hangindent}{52pt}{{王浚}\hspace{30pt}【{\akai 二黄三眼}】在水中兴波浪吞舟荡{艇({\akai 或}:~}荡{坉}\footnote{``{坉''的意思是}用草袋装土筑墙或堵水。}{)},到旱道作毒雾苦害行人。 }

\setlength{\hangindent}{52pt}{{王浚}\hspace{30pt}【{\akai 二黄三眼}】第三害讲出口令人可恨,他比那南山猛虎、长桥孽蛟还狠十分。 }

{周处\hspace{30pt}哦,它是什么妖魔鬼怪?,又是怎样的厉害?}

\setlength{\hangindent}{52pt}{{王浚}\hspace{30pt}【{\akai 二黄快三眼}】若论他是英雄亦非是禽兽之类,他本是有须眉、有志气、雄赳赳、气昂昂是一个有志的能人。 }

{周处\hspace{30pt}哦,既然并非禽兽之类,为何被列为``三害''之内?}

\setlength{\hangindent}{52pt}{{王浚}\hspace{30pt}【{\akai 二黄快三眼}】都只为他父丧早无人教训,因此上习下流做了歹人。 }

{周处\hspace{30pt}怎样为害?}

\setlength{\hangindent}{52pt}{{王浚\hspace{30pt}【}二黄快三眼{】仗势力在宜兴习为光棍,欺贫贱、诈富贵苦害良民。} }

{周处\hspace{30pt}哦,怎样地不法?}

\setlength{\hangindent}{52pt}{{王浚\hspace{30pt}【}二黄快三眼{】有钱的还则可苦苦地曲奉,只可怜({\akai 或}:~实可怜)那无钱的人儿典了庄田、鬻了妻儿、也难少他的半分。} }

{周处\hspace{30pt}哦,何不去县衙状告于他?\footnote{夏行涛{\scriptsize 君}建议此句作``何不写状告于他?''}}

\setlength{\hangindent}{52pt}{{王浚\hspace{30pt}【}二黄快三眼{】也有那被害的家与他来理论}\footnote{此处原来唱``议论'',刘曾复先生听从吴小如先生建议,改唱``理论'',唱词文意更通顺。}{({\akai 或}:~议论),怎奈他膂力过人、力能扛鼎、有势有财,大小的衙门谁敢哼声。} }

{周处\hspace{30pt}哇呀呀$\cdots{}\cdots{}$(周扔扇子)}

\setlength{\hangindent}{52pt}{{王浚\hspace{30pt}【{\akai 二黄摇板}】都只为宜兴城出了恶棍,害得那众黎民难度光阴。} }

{周处\hspace{30pt}老丈!}

\setlength{\hangindent}{52pt}{{周处\hspace{30pt}【{\akai 二黄摇板}】听一言来怒气生,不由豪杰动无名。快快说出他的名和姓。} }

{周处\hspace{30pt}我要剥了他的皮。【{\footnotesize 接}{\akai 二黄摇板}】抽了他的筋。}

\setlength{\hangindent}{52pt}{{王浚\hspace{30pt}【{\akai 二黄摇板}】我若是讲出了他人名姓,怕的是我老命要活不成。} }

{周处\hspace{30pt}老丈!}

\setlength{\hangindent}{52pt}{{周处\hspace{30pt}【{\akai 二黄摇板}】有俺在此何足论。} }

{周处\hspace{30pt}任凭他铜金刚、铁罗汉,【{\footnotesize 接}{\akai 二黄摇板}】难近某的身。}

{王浚}\hspace{30pt}壮士愿听?

{周处\hspace{30pt}愿听。}

{王浚}\hspace{30pt}愿闻?

{周处\hspace{30pt}愿闻。}

{王浚}\hspace{30pt}两厢看来。

{周处\hspace{30pt}讲来。}

{王浚\hspace{30pt}听了------}

\setlength{\hangindent}{52pt}{{王浚}\hspace{30pt}【{\akai 二黄摇板}】他姓周名处 }

{周处\hspace{30pt}啊。}

\setlength{\hangindent}{52pt}{{王浚}\hspace{30pt}【{\footnotesize 接}{\akai 二黄摇板}】字子隐, }

{王浚}\hspace{30pt}嘿嘿!

\setlength{\hangindent}{52pt}{{王浚}\hspace{30pt}【{\akai 二黄摇板}】壮士闻言你惊不惊。 }

{周处\hspace{30pt}哎呀。}

\setlength{\hangindent}{52pt}{{周处\hspace{30pt}【}二黄摇板{】好似霹雷当头震,周处做了不义人。} }

\setlength{\hangindent}{52pt}{{王浚}\hspace{30pt}【{\akai 二黄摇板}】问声壮士名和姓, }

\setlength{\hangindent}{52pt}{{周处\hspace{30pt}【}二黄摇板{】周处就是我的名。} }

{王浚}\hspace{30pt}哎呀,饶命呐!

\setlength{\hangindent}{52pt}{{周处\hspace{30pt}【}二黄摇板{】老丈在此等一等。\textless{}{\!\bfseries\akai 扫头}\!\textgreater{}} }

{(周处{\hwfs 下})}

{王浚}\hspace{30pt}哈哈哈$\cdots{}\cdots{}$({\hwfs 笑}{\hwfs 介})

\setlength{\hangindent}{52pt}{{王浚}\hspace{30pt}【{\akai 二黄摇板}】他好似酒醉方才醒,一言惊起懵懂人。但愿三害俱除尽({\akai 或}:~早除尽), }

(王浚{\hwfs 捡扇子})

\setlength{\hangindent}{52pt}{{王浚}\hspace{30pt}【{\akai 二黄摇板}】黎民百姓享太平。 }

}
