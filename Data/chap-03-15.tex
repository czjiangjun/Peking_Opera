\newpage
\phantomsection %实现目录的正确跳转
\section*{\large\hei 打金枝~{\small 之}~唐王\protect\footnote{根据剧中人物推测,此剧中的唐王应为唐代宗李豫。}}
\addcontentsline{toc}{section}{\hei 打金枝~{\small 之}~唐王}

\hangafter=1                   %2. 设置从第1⾏之后开始悬挂缩进  %}
\setlength{\parindent}{0pt}{

\vspace{3pt}{\centerline{{[}{\hei 第一场}{]}}}\vspace{5pt}

\setlength{\hangindent}{52pt}{(郭暧\hspace{30pt}【{\akai 西皮摇板}】吾主爷有道君长安驾坐,全凭着驾下臣保定山河。安禄山反河东文武胆破,我父子扫狼烟才定干戈。蒙圣恩将金枝招赘于我,父王位、子东床\footnote{刘曾复先生说戏录音中作``父公侯、子王位'',此处从《京剧汇编》第三十二集~邢威明~藏本。}扶保朝阁。今日里八旬寿群臣齐贺,奉王命回府去呀敬致三多\footnote{《京剧汇编》第三十二集~邢威明~藏本作``共贺三多''。}。) }

\vspace{3pt}{\centerline{{[}{\hei 第二场}{]}}}\vspace{5pt}

\setlength{\hangindent}{52pt}{(郭暧\hspace{30pt}【{\akai 西皮摇板}】唐君瑞\footnote{有些地方戏作``唐君蕊'',此处从《京剧汇编》第三十二集~邢威明~藏本。}失却了周公之礼,有天地有父母才有夫妻。似这等不贤妇要她何益,倒不如在府中独宿孤栖。) }

\vspace{3pt}{\centerline{{[}{\hei 第三场}{]}}}\vspace{5pt}

摆驾!

\setlength{\hangindent}{56pt}{【{\akai 西皮慢板}】金乌东升玉兔坠,景阳钟三下响王出宫闱\footnote{陈超老师介绍,唐王唱完此句后整冠捋髯。}。唐室连年遭颠沛,国乱只为杨贵妃。安禄山在河东【{\footnotesize 转}{\akai 西皮二六}】曾起反意,兵破潼关夺社稷。陈元礼兵变在马嵬驿,可怜那贵妃丧沟渠。先皇驾幸西蜀地,多亏皇兄郭子仪。血战三载狼烟息,擒住了贼子剑下劈。到如今乐享这太平世,黄河清、北海晏有凤来仪。内侍臣摆御驾九龙【{\akai 回龙}】里,}

\setlength{\hangindent}{56pt}{【{\akai 西皮摇板}】君王有道福寿齐。 }

\setlength{\hangindent}{56pt}{【{\akai 西皮快板}】一见皇儿泪悲啼,打碎珠冠扯破衣。你与驸马因何起\footnote{夏行涛{\scriptsize 君}建议作``因何气''。},一一从头奏孤知。 }

皇儿平身。

赐座。

慢慢奏来!

\setlength{\hangindent}{56pt}{【{\akai 西皮摇板}】御妻休得本奏启, }

\setlength{\hangindent}{56pt}{【{\akai 西皮摇板}】皇儿且莫泪悲啼({\akai 或}:~皇儿也莫泪悲啼)。 }

\setlength{\hangindent}{56pt}{【{\akai 西皮摇板}】你母女暂且回宫去。 }

\setlength{\hangindent}{56pt}{【{\akai 西皮摇板}】内侍与孤传旨意,快宣皇兄郭子仪。 }

\setlength{\hangindent}{56pt}{【{\akai 西皮二六}】九龙口内红光起,来了皇兄郭子仪。昨日里皇兄悬弧喜\footnote{古代习俗,生了男孩子,就在门的左首悬挂一张弓。因此用``悬弧之喜''指男性的生日。},王未曾去拜寿也曾赐过你珍奇。王坐江山全亏你,从今后赐你剑、履上丹墀。内侍臣与孤搀扶起, }

\setlength{\hangindent}{56pt}{【{\akai 西皮摇板}】君臣对坐把话提。 }

\setlength{\hangindent}{56pt}{【{\akai 西皮摇板}】殿角绑的何臣子,一一从头说孤知({\akai 或}:~一一从头奏孤知)。 }

(\setlength{\hangindent}{52pt}{郭子仪\hspace{20pt}【{\akai 西皮摇板}】请王传旨将他斩。})

\setlength{\hangindent}{56pt}{【{\akai 西皮散板}】老皇兄做事太心急。况且驸马轻年纪\footnote{``轻年纪'',即年纪尚轻之意。},公主又是少年妻。自古道清官难断家务事,他夫妻吵闹常有之。孤皇传旨不降罪,快与驸马去换朝衣。 }

皇兄平身!

赐座。

皇兄,昨晚宫中,驸马缘何与公主争论?

内侍,宣驸马冠带上殿。

平身。

驸马,昨晚为何与公主争论?

皇兄,驸马所言甚是。

从今以后,红灯撤去,只行夫妻常礼。

呃,往下奏来。

啊,皇兄,听驸马所奏,孤是明白了。

昨日皇兄八旬双寿,众家哥弟,一个个成双结对,拜寿堂前。公主不在,驸马一人拜寿,自觉孝道有亏,是与不是?

唉!皇兄啊,自古道:~不痴不聋,难做公翁。从今以后,他夫妻之事,你不必劳心呐。

听孤旨下!

\setlength{\hangindent}{56pt}{【{\akai 西皮二六}】驸马奏本孤的龙心爽,颇知三纲并五常。但愿皇兄多欢畅({\akai 或}:~臣心若得君欢畅),福寿齐眉永安康。老皇兄暂且回府往,王与驸马有商量({\akai 或}:~共商量)。 }

\setlength{\hangindent}{56pt}{【{\akai 西皮散板}】驸马近前听旨降:~忠臣孝子永留芳。孤王赐你尚方剑,命公主赔罪到汾阳。 }

\setlength{\hangindent}{56pt}{【{\akai 西皮散板}】内侍臣摆驾后宫进,见了御妻说分明。 }

}
