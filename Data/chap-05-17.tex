\newpage
\subsubsection{\large\hei {洗浮山~{\small 之}~贺天保}}
\addcontentsline{toc}{subsection}{\hei 洗浮山~\small{之}~贺天保}

\hangafter=1                   %2. 设置从第1⾏之后开始悬挂缩进  %
\setlength{\parindent}{0pt}{

{\vspace{3pt}{\centerline{{[}{\hei 第一场}{]}}}\vspace{5pt}}

({\akai 内})马来!

\setlength{\hangindent}{56pt}{【{\akai 西皮摇板}】大英雄在世间义气为本呐,恨草寇劫皇粮苦及黎民。 }

适才营中,众家英雄议论剿平浮山之策,我想浮山山高水险,不明地势,只恐遭贼暗算。是我安抚众人不可轻易行动,为此私出大营,暗地查看山势,以便作好准备,就此走遭也!

({\akai 或}:~适才营中同众商议破敌之策,我想浮山山高水险,不明虚实,恐遭暗算。为此安抚众人不可出征,是我单人独骑私出大营,去至浮山,探勘虚实,以作准备也!)

\setlength{\hangindent}{56pt}{【{\akai 西皮摇板}】暗地里出大营虚实探听,山口外悬人头所为何情? }

且住!山口(之外)高悬人头,是何缘故?

呜哙呀!莫非黄贤弟私自探山({\akai 或}:~私探浮山),遭贼毒手?!

嗯,我不免速回大营({\akai 或}:~回转大营),观看动静,再作定夺便了({\akai 或}:~再作计较也)。

呃!({\akai 或}:~且住!)

果然黄贤弟不在营中,定是私探浮山,命丧贼手({\akai 或}:~遭贼毒手)。

哎呀!想我弟兄自结江南({\akai 或}:~结拜江南),患难相顾\footnote{夏行涛{\scriptsize 君}建议作``患难相共''。}。他今已死,我焉能独生?!

也罢!

俺不免闯进浮山,杀却余六、余七,与黄贤弟报仇雪恨({\footnotesize 呐})!

呔!浮山草寇休得猖狂({\akai 或}:~休得逞强),贺爷来也!

{\vspace{3pt}{\centerline{{[}{\hei 第二场}{]}}}\vspace{5pt}}

\setlength{\hangindent}{56pt}{【{\akai 二黄摇板}】想当日在绿林江湖闯荡,转瞬间黄粱梦昙花一场。 }

吾乃贺天保阴魂是也。只因私探浮山({\akai 或}:~夜探浮山),命丧飞抓之下。我儿仁杰,跟随黄贤弟前来搬运尸灵,今晚夜宿馆驿之中,不免托梦与他,了此一段夙缘也!

\setlength{\hangindent}{56pt}{【{\akai 二黄摇板}】到如今不能够风流话讲,蜀魄啼梦魂惊心神惨伤。 }

\setlength{\hangindent}{56pt}{【{\akai 二黄摇板}】谯楼鼓三更催二更鼓尽,秋气寒月无光惨淡孤魂。 }

唉,俺贺天保死得好苦也!

\setlength{\hangindent}{66pt}{【{\akai 反二黄慢板}】站馆中({\akai 或}:~站店中)悲切切魂伤魄断,曾记得四霸天结义江南。我这里走向前忠魂呼唤,叫一声黄贤弟细听兄言:~实指望洗浮山扫平贼乱,料不想飞抓下一旦生残。望贤弟领人马埋伏湖畔,施巧计擒贼子与兄报冤。诉不尽心头恨咽喉气短,咽喉气呃短,贤弟呀!}

\setlength{\hangindent}{66pt}{【{\akai 反二黄原板}】转面来与我儿再把话言:~儿在阳呃、父在阴两厢隔断,可怜儿未长成幼年孤单。老娘亲全靠儿甘旨奉献,但愿儿立大志忠孝当先。天将明父要归不尽悲惨,不尽悲惨,}

\setlength{\hangindent}{66pt}{【{\akai 反二黄散板}】父子们要重逢今世却难。}

{(\textless{}{\!\bfseries\akai 三叫头}\!\textgreater{}仁杰,我儿,唉,儿啊$\cdots{}\cdots{}$({\hwfs 哭介}))}

{(罢!)}
