\newpage
\phantomsection %实现目录的正确跳转
\section*{\large\hei {御碑亭~{\small 之}~王有道}}
\addcontentsline{toc}{section}{\hei 御碑亭~{\small 之}~王有道}

\hangafter=1                   %2. 设置从第1⾏之后开始悬挂缩进  %
\setlength{\parindent}{0pt}{

{\vspace{3pt}{\centerline{{[}{\hei 第一场}{]}}}\vspace{5pt}}

{{[}{\akai 引子}{]}磨穿铁砚,这襟怀,不让前贤。}

{({\akai 念})读尽诗书身世寒,满腹文章不为官。月宫\footnote{《余派戏词钱氏辑粹》\upcite{Meng-Shuoxi}载孟小冬说戏本作``月中''。}丹桂相攀易,金殿鳌头独占难。}

{卑人王有道,浙东人氏,寄居京华。不幸父母早逝,娶妻孟氏月华,十分贤德。胞妹淑英,年方二九,尚未许字。想我苦读寒窗,功名未能上达。今当大比之年,会试之期,我不免将孟氏、妹子唤出堂前,嘱咐门户之事,也好入场。}

{啊,娘子、贤妹哪里?}

{啊娘子,}

{贤妹,}

{一同坐下。}

{非为别事,今当大比之年,会试之期,唤你们出来,好好看守门户,我好入场({\akai 或}:~我要入场)会试。}

{但愿如此。}

{足见盛情,我当立饮三杯。}

{(孟月华\hspace{20pt}待奴把盏。)}

\setlength{\hangindent}{56pt}{【{\akai 西皮原板}】承谢你贤德的心喜之不尽,但愿得此一去鱼跳龙门。}

\setlength{\hangindent}{56pt}{【{\akai 西皮原板}】贤德妹礼爱我手足情份,猛想起父母恩无限伤情。}

\setlength{\hangindent}{56pt}{【{\akai 西皮原板}】但愿得这一科功名有份,终不愧王有道苦读诗文。施一礼辞贤妹又别闺阃,}

\setlength{\hangindent}{56pt}{【{\akai 西皮摇板}】赴科场好一似平步登云\footnote{刘曾复先生曾回忆\upcite{Chai-DaXikao},此句余叔岩传孟小冬的词句为``赴科场贡士试平步登云''。}。({\akai 或}:~沐洪恩蟾宫考会会文衡\footnote{文衡,旧指以文章取士的标准来取舍权衡。又指科举制度下的主考官。段公平{\scriptsize 君}注:~这是王荣山唱法。}。)}

\vspace{3pt}{\centerline{{[}{\hei 第二场}{]}}}\vspace{5pt}

{呵呵呵哈哈哈$\cdots{}\cdots{}$({\hwfs 笑介})}

\setlength{\hangindent}{56pt}{【{\akai 西皮摇板}】考罢了第三场文章高兴,笑盈盈喜孜孜出了龙门。回家去细说与阖家欢庆,转大街({\akai 或}:~穿大街)过小巷到了家门。}

{开门来。}

{是我回来了。}

\setlength{\hangindent}{56pt}{【{\akai 西皮摇板}】你嫂嫂因何故她不来开门。}

{你嫂嫂往哪里去了?}

{哦,她病了,得何病症呐?}

{既然中途路遇({\akai 或}:~中途遇见)大雨,就该寻个所在,躲避躲避,又何必冒雨而归,呃,成什么样儿啊。}

{在何处避雨?}

{哦,御碑亭。}

{不错,有一个御碑亭。}

{后来呢?}

{啊?她,她就该走了出来呀。}

{后来便怎么样啊?}

{暧昧不明,有何为证?}

{怎么还有诗词么?}

{你可记得?}

{念来我听------}

{哪一句?}

{呀呸!}

\setlength{\hangindent}{56pt}{【{\akai 西皮散板}】听一言不由我火上两鬓,诗词说宿碑亭隐有内情呐。我待要({\akai 或}:~我倒要)到里面将她查问,}

{唉。}

\setlength{\hangindent}{56pt}{【{\akai 西皮散板}】这桩事闹起来呀脸面何存{\footnotesize 呐}}。

{罢了哇罢了。我想此事,闹将起来,不成体面,隐忍不言。王有道啊王有道,岂不成了(此道)$\cdots{}\cdots{}$}

{也罢,我不免暗写休书一封,就说她爹娘身染重病,将她送回娘家,免得两下出丑。}

{苍头快来,雇乘车辆送你大娘到孟家庄去。}

{快去。}

{啊贤妹,想女儿家谨守闺训,当讲则讲,不当讲不可胡言乱语。方才是你多口,才闹出这样的事啊!~从今以后,要谨守莫言,那才是我的好妹子啊。}

{好,将笔砚端正好了。}

{唉!羞愧人也!}

\setlength{\hangindent}{56pt}{【{\akai 西皮导板}】王有道提笔泪难忍,}

\setlength{\hangindent}{56pt}{【{\akai 西皮原板}】实难舍夫妻结发情。实指望同生共到老,又谁知半途风波生。非是我一旦多薄幸,实难容留下贱的人呐。只得闭口【{\footnotesize 转}{\akai 西皮快板}】牙咬定,字字行行写分明:~那一日避雨在御碑亭,其中暧昧事不明。男女授受理不应,七出之条事有因。从今任你嫁别姓,割断了丝萝两离分。写罢休书打手印,}

{(苍头\hspace{30pt}车辆到。)}

{晓得。}

{外厢伺候。}

\setlength{\hangindent}{56pt}{【{\akai 西皮摇板}】密密封好待她行。}

\setlength{\hangindent}{56pt}{【{\akai 西皮摇板}】贤妹将你嫂嫂请,说孟家差人到来临。}

{回来了。}

{文章么,倒还得意呀。}

{(呃,)

({\hwfs 急介})只是有一事替你着急呀。}

{我方才出场的时节,遇见你家小厮德禄,慌慌张张言道:~员外安人,因你不辞而归,二老吵闹一场,双双病倒在床,十分地沉重,故而({\akai 或}:~故此)急急赶来接你回去。}

{呃,德禄么,我恐其家中无人,先打发他回去。我已吩咐苍头,雇来车辆。你就该速速前往,安慰他二老一回,那才是你的孝道啊。}

\setlength{\hangindent}{56pt}{【{\akai 西皮摇板}】车在门首趁早行。}

{转来。}

\setlength{\hangindent}{56pt}{【{\akai 西皮摇板}】朋友托带一封信,带回交与你严亲。}

\setlength{\hangindent}{56pt}{【{\akai 西皮摇板}】从先恩爱一时尽,要想相逢恐不能{\footnotesize 呐}。}

\setlength{\hangindent}{56pt}{【{\akai 西皮摇板}】万般已是皆有{\footnotesize 哇}定,}

\setlength{\hangindent}{52pt}{(王淑英\hspace{20pt}【{\akai 西皮摇板}】你又何必假泪淋。)} 

\setlength{\hangindent}{56pt}{【{\akai 西皮摇板}】这是我家门遭不幸,毒意休妻心不宁。思想恩爱泪难忍,孤孤凄凄\footnote{吴小如先生曾撰文\upcite{Wu_Wenlu-II}指出旧时艺人因``凄凄''的异体``恓恓''讹传为``洒洒'';~王端璞先生改唱为``孤孤零零'',吴从之。}愁煞人{\footnotesize 呐}。}

{(报子\hspace{30pt}啊哈------)}

{(报子\hspace{30pt}报报报,喜来到。)}

{(报子\hspace{30pt}报禄的来喽。)}

{你们是做什么的?}

{哦,报禄的。报的是哪一家呀?}

{啊?王有道他中了么?}

{就是你老爷呀。}

{进来,进来。}

{可有报单?}

{呈上来。}

{``因报贵府第王老爷、印有道,取中甲辰科第六名进士。''}

{呵呵哈哈哈哈$\cdots{}\cdots{}$({\hwfs 笑介})}

{贴在门首。}

{我好侥幸{\footnotesize 呐}$\cdots{}\cdots{}$}

{辛苦你们了,}

{难为你们了。}

{喜钱今日不便呐,改日多多重赏。}

{我不免去到里面({\akai 或}:~去至后面),说与妹子知道,再打点谒师便了。}

\setlength{\hangindent}{56pt}{【{\akai 西皮摇板}】十载寒窗今方信,皇天不负我读书的人呐。}

{呵呵哈哈哈$\cdots{}\cdots{}$({\hwfs 笑介})}

{\vspace{3pt}{\centerline{{[}{\hei 第三场}{]}}}\vspace{5pt}}

{({\akai 念})昨日寒儒谁问姓,今朝显贵便知名。}

{老师在上,门生等大礼参拜。}

{({\akai 念})桃李公门姓氏香,荐拔之恩日月长。}

{老师在上,门生等侍立听教。}

{谢座。}

{请}

{皆赖老师提拔。}

{年兄但讲何妨?({\akai 或}:~年兄何妨直言?)}

{啊柳$\cdots{}\cdots{}$}

{啊,柳年兄,你在御碑亭避过雨来?那亭内,可有个妇人先在其内呀?}

{你可晓得那妇人的姓氏?}

{哦,哦,你并未交谈。}

{你实未交谈?!}

{哎呀,老师啊,}

{唉呀,年兄啊,那亭内的妇人不是外人呐,}

{乃是门生的,唉,拙荆哦。}

{不敢呐,}

{不敢呐。}

{哎呀,老师啊({\akai 或}:~年兄啊),}

{哎呀,年兄啊({\akai 或}:~老师啊),}

{也是门生一时不明,``莫须有''三字,竟将她休弃了。}

{老$\cdots{}\cdots{}$}

{老师啊------}

\setlength{\hangindent}{56pt}{【{\akai 西皮摇板}】门生一时做事蠢,疑她暧昧事不明。年兄说明亭中景,不该疑心退婚姻。}

{门生等告退。({\akai 或}:~门生等遵命。)}

{正是:~({\akai 念})但愿文章依宇下。}

{门生等告退。}

\vspace{3pt}{\centerline{{[}{\hei 第四场}{]}}}\vspace{5pt}

\setlength{\hangindent}{56pt}{【{\akai 西皮摇板}】赴罢了琼林宴过庄陪罪,见岳父和岳母劝妻回归。此一番必须要准备跌跪,到此时才知\nolinebreak{\footnotesize 呀}自惹是非。}

{({\akai 念})读书侥幸已成名,到底糊涂心不精。琴瑟失调乖礼仪,方知宋弘\footnote{东汉宋弘对妻忠诚,光武帝要他改尚寡姐湖阳公主时,宋弘有名言:~``贫贱之交不可忘,糟糠之妻不下堂~!''}是高人。}

{呃,哦,德禄你来了。}

{哦,员外、安人(也)来了。}

{哦,你家姑娘也来了。快快有请呐!}

{哼,休得胡言。(快快有请。)}

{啊,岳父。}

{岳母,呃,呃$\cdots{}\cdots{}$}

{啊,娘$\cdots{}\cdots{}$}

{呃,糟糕!}

{(啊,娘$\cdots{}\cdots{}$)}

{德禄,你在此则甚?}

{有我就用不着你了。}

{出去,}

{教你出去。}

{近前来------}

{你出去吧。}

{啊娘子,千不是,万不是,俱是卑人的不是。喏喏喏,我这厢赔礼了。}

{(啊娘子,)我这厢又赔礼了。}

{啊娘子,俱是卑人的不是,我这厢跪下了。({\akai 或}:~啊娘子,卑人这里跪下了。)}

\setlength{\hangindent}{56pt}{【{\akai 西皮快板}】男儿志气三千丈,污秽之言岂能当。也是卑人太孟浪,一时性急我未推详。}

\setlength{\hangindent}{56pt}{【{\akai 西皮快板}】世间万事有原谅,何况丈夫与妻房。从先的事儿莫追想,还念昔日情义长。}

{呵呵哈哈哈$\cdots{}\cdots{}$({\hwfs 笑介})}

\setlength{\hangindent}{56pt}{【{\akai 西皮摇板}】你是我的贤妻房。}

{\vspace{3pt}{\centerline{{[}{\hei 第五场}{]}}}\vspace{5pt}}

{有请。}

{有劳众位年兄远路而来({\akai 或}:~有劳众位年兄前来),弟当面谢过。}

{啊众位年兄,弟有一言,不好启齿。}

{闻得柳年兄尚乏中馈\footnote{中馈,古时指妇女在家中主持饮食等事,后引申指妻室。},弟有一妹,名唤淑英,年方二九。呃,倒还伶俐。欲与柳年兄永结丝萝,幸勿见却。}

{(年兄休得见却。)}

{今日即是良辰黄道,(就在舍下结拜花烛,)就请二位年兄赞礼上来。}

{娘子,搀扶小妹({\akai 或}:~搀扶贤妹)。}

{且慢,后面备得酒饭,就烦二位年兄,陪一陪我们的新姑老爷呀。}

{呵呵哈哈哈$\cdots{}\cdots{}$({\hwfs 笑介})}

{请------}

}
