\newpage
\phantomsection %实现目录的正确跳转
\section*{\large\hei 取帅印~\protect\footnote{本剧本中标注人物台上位置主要由段公平{\scriptsize 君}标注。刘曾复先生在为樊百乐{\scriptsize 君}说戏时曾说明:~《取帅印》是二十四本《龙门阵》的头一本。《龙门阵》是大老板程长庚排的。《白良关》也是《龙门阵》里的,但《凤凰山》、《独木关》、《汾河湾》等都不是《龙门阵》里的。}}
\addcontentsline{toc}{section}{\hei 取帅印}

\hangafter=1                   %2. 设置从第1⾏之后开始悬挂缩进  %}}
\setlength{\parindent}{0pt}{

\vspace{3pt}{\centerline{{[}{\hei 第一场}{]}}}\vspace{5pt}

\setlength{\hangindent}{52pt}{({\hwfs 吹}\textless{}\!{\bfseries\akai 点绛唇}\!\textgreater{})}

\setlength{\hangindent}{52pt}{徐勣\hspace{30pt}({\akai 念})朝臣待漏月坠西,}

\setlength{\hangindent}{52pt}{尉迟恭\hspace{20pt}({\akai 念})文臣武将整朝衣。}

\setlength{\hangindent}{52pt}{程咬金\hspace{20pt}({\akai 念})金钟玉磬连声响,}

%徐勣\\尉迟恭\hspace{20pt}({\hwfs 同}{\akai 念})三跪九叩拜丹墀。\\程咬金
\raisebox{0pt}[24pt][16pt]{\raisebox{12pt}{徐勣}\raisebox{0pt}{\hspace{-22pt}{尉迟恭}}\raisebox{-12pt}{\hspace{-32pt}{程咬金}}\raisebox{0pt}{\hspace{20pt}({\hwfs 同念})三跪九叩拜丹墀。}}

\setlength{\hangindent}{52pt}{徐勣\hspace{30pt}山人徐勣。}

\setlength{\hangindent}{52pt}{尉迟恭\hspace{20pt}鄂国公敬德。}

\setlength{\hangindent}{52pt}{程咬金\hspace{20pt}鲁国公咬金。}

\setlength{\hangindent}{52pt}{徐勣\hspace{30pt}列公请了!}

%尉迟恭\\程咬金\raisebox{5pt}{\hspace{20pt}请了!}
\raisebox{0pt}[22pt][16pt]{\raisebox{8pt}{尉迟恭}\raisebox{-8pt}{\hspace{-32pt}{程咬金}}\raisebox{0pt}{\hspace{20pt}请了。}}

\setlength{\hangindent}{52pt}{徐勣\hspace{30pt}今有张士贵,在绛州龙门,招军已满,有本回朝。少刻万岁登殿,一同启奏。}

%徐勣\\尉迟恭\hspace{20pt}看,香烟缭绕,圣驾临朝,分班伺候。请!\\程咬金
\raisebox{0pt}[24pt][16pt]{\raisebox{12pt}{徐勣}\raisebox{0pt}{\hspace{-22pt}{尉迟恭}}\raisebox{-12pt}{\hspace{-32pt}{程咬金}}\raisebox{0pt}{\hspace{20pt}看,香烟缭绕,圣驾临朝,分班伺候。请!}}

\setlength{\hangindent}{52pt}{李世民\hspace{20pt}{[}{\akai 引子}{]}海宴河淸,喜的是,四海升平。}

%徐勣\\尉迟恭\raisebox{5pt}{\hspace{20pt}臣等见驾,吾皇万岁!}\\程咬金
\raisebox{0pt}[24pt][16pt]{\raisebox{12pt}{徐勣}\raisebox{0pt}{\hspace{-22pt}{尉迟恭}}\raisebox{-12pt}{\hspace{-32pt}{程咬金}}\raisebox{0pt}{\hspace{20pt}臣等见驾,吾皇万岁!}}

\setlength{\hangindent}{52pt}{李世民\hspace{20pt}平身。}

%徐勣\\尉迟恭\hspace{20pt}万万岁!\\程咬金
\raisebox{0pt}[24pt][16pt]{\raisebox{12pt}{徐勣}\raisebox{0pt}{\hspace{-22pt}{尉迟恭}}\raisebox{-12pt}{\hspace{-32pt}{程咬金}}\raisebox{0pt}{\hspace{20pt}万万岁!}}

\setlength{\hangindent}{52pt}{李世民\hspace{20pt}赐座。}

%徐勣\\尉迟恭\hspace{20pt}谢座!\\程咬金
\raisebox{0pt}[24pt][16pt]{\raisebox{12pt}{徐勣}\raisebox{0pt}{\hspace{-22pt}{尉迟恭}}\raisebox{-12pt}{\hspace{-32pt}{程咬金}}\raisebox{0pt}{\hspace{20pt}谢座!}}

\setlength{\hangindent}{52pt}{李世民\hspace{20pt}({\akai 念})父王宴驾命归天,孤王接位掌江山。征扫北国回朝转,可恨辽东起狼烟。}

\setlength{\hangindent}{52pt}{李世民\hspace{20pt}孤,李世民。国号贞观在位。父王宴驾,众卿保孤登基。可恨辽东盖苏文,打来连环战表,教寡人御驾亲征。是寡人命张士贵,在绛州龙门,招军集将,王君可监造战船。二人出京,数月有余,并无本章回朝,教孤日夜忧虑也。}

\setlength{\hangindent}{52pt}{徐勣\hspace{30pt}臣启万岁:~今有张士贵有本还朝,请我主龙目御览!}

\setlength{\hangindent}{52pt}{李世民\hspace{20pt}呈上来!}

\setlength{\hangindent}{52pt}{李世民\hspace{20pt}【{\akai 西皮原板}】日出扶桑万道霞,群臣歌颂帝王家。张环有本奏陛下,请主龙目细详察。\footnote{夏行涛{\scriptsize 君}建议此四句应为一人一句接唱,此处从《京剧汇编》第九集~赵荣鹏~藏本。}奉王旨意招人马,英雄投效到王家。并无有仁贵投帐下,}

\setlength{\hangindent}{52pt}{李世民\hspace{20pt}【{\akai 西皮摇板}】再与先生把话答。}

\setlength{\hangindent}{52pt}{李世民\hspace{20pt}先生,张士贵绛州龙门招军,为查访应梦贤臣。这本章上面,并无``仁贵''二字,张环莫非有欺君之意?}

\setlength{\hangindent}{52pt}{徐勣\hspace{30pt}张环焉敢欺君。万岁此番征东,若无贤臣保驾,臣之罪也。}

\setlength{\hangindent}{52pt}{李世民\hspace{20pt}秦恩公染病在床,先生保奏何人?}

\setlength{\hangindent}{52pt}{徐勣\hspace{30pt}臣保尉迟恭挂帅,征伐辽东,一战成功。}

\setlength{\hangindent}{52pt}{程咬金\hspace{20pt}万岁,休听军师之言,尉迟恭挂不得帅印。}

\setlength{\hangindent}{52pt}{尉迟恭\hspace{20pt}程将军,军师保我挂帅,你为何拦阻?想我开国元勋不挂,谁人能挂?}

\setlength{\hangindent}{52pt}{程咬金\hspace{20pt}得了罢,动不动就是开国元勋,难道我老程就不是开国元勋吗?\footnote{刘曾复先生特别说明:~程咬金对唐王念韵白,对其他臣宰念京白。}}

\setlength{\hangindent}{52pt}{尉迟恭\hspace{20pt}你不能。}

\setlength{\hangindent}{52pt}{程咬金\hspace{20pt}哼,我不能,那你也不能啊。}

\setlength{\hangindent}{52pt}{李世民\hspace{20pt}且慢!~二卿不必争论,帅印现在秦府,就命程皇兄前去取印回来,再作定夺。}

\setlength{\hangindent}{52pt}{程咬金\hspace{20pt}臣领旨。}

\setlength{\hangindent}{52pt}{李世民\hspace{20pt}转来!}

\setlength{\hangindent}{52pt}{程咬金\hspace{20pt}臣在。}

\setlength{\hangindent}{52pt}{李世民\hspace{20pt}他乃有病之人,必须见机而行。听孤旨下!}

\setlength{\hangindent}{52pt}{李世民\hspace{20pt}【{\akai 西皮摇板}】恩公投唐功劳大,东挡西除定邦家。虽然卧病在床榻,雄心依然保唐家。卿家要说温柔话,随机应变把印拿。}

\setlength{\hangindent}{52pt}{程咬金\hspace{20pt}领旨!}

\setlength{\hangindent}{52pt}{程咬金\hspace{20pt}【{\akai 西皮摇板}】万岁叮咛一席话,为臣一一转秦家。辞王别驾把殿下,背转身来自咂牙。}

\setlength{\hangindent}{52pt}{程咬金\hspace{20pt}哎呀且住。我想这颗帅印,乃是秦二哥执掌多年;~我若取回,岂不白白地送与黑贼之手?~呃呃呃,我有了!~我不免午门闲游一番,急回谎奏,就说二哥染病在床,昏迷不醒,他不肯交印。也免得那个黑贼痴心妄想也!}

\setlength{\hangindent}{52pt}{程咬金\hspace{20pt}【{\akai 西皮摇板}】急回谎奏君王驾,痴心妄想不归他。}

\setlength{\hangindent}{52pt}{黄门官\hspace{20pt}【{\akai 西皮摇板}】春城无处不飞花,随王并无半日暇。}

\setlength{\hangindent}{52pt}{黄门官\hspace{20pt}臣黄门官见驾,吾皇万岁!}

\setlength{\hangindent}{52pt}{李世民\hspace{20pt}平身。}

\setlength{\hangindent}{52pt}{黄门官\hspace{20pt}万万岁!}

\setlength{\hangindent}{52pt}{李世民\hspace{20pt}上殿有何本奏?}

\setlength{\hangindent}{52pt}{黄门官\hspace{20pt}今有王君可,有本回朝。我主龙目御览!}

\setlength{\hangindent}{52pt}{李世民\hspace{20pt}呈上来,待孤观看!}

\setlength{\hangindent}{52pt}{李世民\hspace{20pt}【{\akai 西皮摇板}】奉王旨意到海下,王君可修本奏皇家。请主旨意发人马,扫平辽东定中华。看罢本章记心下,黄门官近前听根芽。吩咐众将免见驾,三日之后候旨发。}

\setlength{\hangindent}{52pt}{黄门官\hspace{20pt}领旨!}

\setlength{\hangindent}{52pt}{黄门官\hspace{20pt}【{\akai 西皮摇板}】辞王别驾把殿下,晓谕文武百官家。}

\setlength{\hangindent}{52pt}{程咬金\hspace{20pt}【{\akai 西皮摇板}】胸藏妙计说假话,急忙上殿我骗皇家。}

\setlength{\hangindent}{52pt}{程咬金\hspace{20pt}臣交旨。}

\setlength{\hangindent}{52pt}{李世民\hspace{20pt}赐坐。}

\setlength{\hangindent}{52pt}{程咬金\hspace{20pt}谢座。}

\setlength{\hangindent}{52pt}{李世民\hspace{20pt}秦恩公病体如何?}

\setlength{\hangindent}{52pt}{程咬金\hspace{20pt}照常一样,呕吐不止。}

\setlength{\hangindent}{52pt}{李世民\hspace{20pt}唉,恩公啊$\cdots{}\cdots{}$({\hwfs 哭介})}

\setlength{\hangindent}{52pt}{李世民\hspace{20pt}【{\akai 西皮摇板}】恩公病势不见佳,不由孤王泪如麻。东挡西除功劳大,病体缠身难挣扎。}

\setlength{\hangindent}{52pt}{程咬金\hspace{20pt}他乃久病之人,万岁何必忧虑。}

\setlength{\hangindent}{52pt}{李世民\hspace{20pt}他乃有功之臣,倘有不测,孤心不忍。}

\setlength{\hangindent}{52pt}{程咬金\hspace{20pt}万岁真乃有道明君。}

\setlength{\hangindent}{52pt}{李世民\hspace{20pt}取印之事如何?}

\setlength{\hangindent}{52pt}{程咬金\hspace{20pt}万岁休要提起取印之事。为臣走进了病房之间,言道:~二哥,你病体如何。他说:~照常一样,呕吐不止。是臣言道,万岁因你染病在床,龙心悬念,命我前来探望于你。他说真是有道明君。他又问起为臣,呃,征东一事如何。臣言:~万岁今日设立早朝,张士贵有本回朝,招军已满。万岁因你染病在床,龙心未定。军师力保尉迟恭挂帅。他听说尉迟恭挂帅,哼,是一派的好埋怨\footnote{``埋怨'',《京剧汇编》第九集~赵荣鹏~藏本皆作``瞒怨''。}啊。}

\setlength{\hangindent}{52pt}{李世民\hspace{20pt}埋怨何来?}

\setlength{\hangindent}{52pt}{程咬金\hspace{20pt}呃,呃,他言道:~想我秦琼,自投唐以来,攻无不胜,战无不取,才挣下,呃,这颗帅印。如今染病在床,倘有不测,呃,还有我儿怀玉呀。再一说,还有咬金兄弟,也是文武双全,可以挂得帅印。想那尉迟恭,与我秦琼,并无半点瓜葛之情。况且他目不识丁,何能决胜千里之外?呵呵,他一派的好埋怨呐!}

\setlength{\hangindent}{52pt}{徐勣\hspace{30pt}哼,你一派说谎。}

\setlength{\hangindent}{52pt}{程咬金\hspace{20pt}嘿,你又不曾听见,你怎么知道是谎言呢?}

\setlength{\hangindent}{52pt}{程咬金\hspace{20pt}臣见他那般光景呵!}

\setlength{\hangindent}{52pt}{程咬金\hspace{20pt}【{\akai 西皮摇板}】气力不佳难讲话,病势未减只又加。为臣一见心害怕,万岁龙心细详察。}

\setlength{\hangindent}{52pt}{李世民\hspace{20pt}【{\akai 西皮摇板}】孤王闻言头低下,心中辗转泪如麻。辽东若不去征伐,定说孤王惧怕他。}

\setlength{\hangindent}{52pt}{李世民\hspace{20pt}先生,秦恩公昏迷不醍,不肯交印,如何是好?}

\setlength{\hangindent}{52pt}{徐勣\hspace{30pt}万岁明日过府,一来探病,二取帅印。}

\setlength{\hangindent}{52pt}{程咬金\hspace{20pt}呵,万岁,休听军师之言,取印乃是一桩小事,不论差哪部大人前去也就是了,何劳御驾亲往!}

\setlength{\hangindent}{52pt}{尉迟恭\hspace{20pt}程将军,君入臣门,蓬荜生辉,你为何拦阻?}

\setlength{\hangindent}{52pt}{徐勣\hspace{30pt}你呀,真是多口!}

\setlength{\hangindent}{52pt}{程咬金\hspace{20pt}哎呀!你们俩呀,嘿嘿,打成了合同了。}

\setlength{\hangindent}{52pt}{程咬金\hspace{20pt}臣启万岁:~呃,臣好有一比。}

\setlength{\hangindent}{52pt}{李世民\hspace{20pt}比作何来?}

\setlength{\hangindent}{52pt}{程咬金\hspace{20pt}掌上的乌鸦,呵,我开不得口啊!}

\setlength{\hangindent}{52pt}{李世民\hspace{20pt}开口便怎样?}

\setlength{\hangindent}{52pt}{程咬金\hspace{20pt}开口便是祸。方才说了一句话,一个,道臣拦阻,一个,道臣多口。明日过府,呃,必须有几句言语要讲;不讲,呃,反倒得罪他们,我还是不去的为妙啊。}

\setlength{\hangindent}{52pt}{李世民\hspace{20pt}呃,你与秦恩公昔年结为好友,只管大胆,保孤前去。有什么祸事,寡人与你担待。}

\setlength{\hangindent}{52pt}{程咬金\hspace{20pt}我说三哥哟,这可是万岁教我去的。呃,我这可是奉了旨的了!}

\setlength{\hangindent}{52pt}{徐勣\hspace{30pt}真乃一张油口。}

\setlength{\hangindent}{52pt}{程咬金\hspace{20pt}哼,我又油口了。}

\setlength{\hangindent}{52pt}{李世民\hspace{20pt}听孤旨下!}

\setlength{\hangindent}{52pt}{李世民\hspace{20pt}【{\akai 西皮摇板}】昔日恩公走天涯,锏打杨广救全家。明日文武齐保驾,孤王亲自去看他。}

\setlength{\hangindent}{52pt}{程咬金\hspace{20pt}【{\akai 西皮摇板}】黑贼金殿夸大话,军师一旁暗保他。就是帅印归他挂,也教他口念活菩萨呀。}

\setlength{\hangindent}{52pt}{尉迟恭\hspace{20pt}【{\akai 西皮摇板}】军师金殿抬爱咱,咬金一旁把话答。若是帅印归我挂,寻一良谋摆布他。}

\setlength{\hangindent}{52pt}{徐勣\hspace{30pt}【{\akai 西皮摇板}】适才咬金一席话,蒙哄万岁弄巧牙。辞王别驾把殿下,明日保主到秦家。}

\setlength{\hangindent}{52pt}{李世民\hspace{20pt}【{\akai 西皮摇板}】龙楼凤阁紫雾霞,金殿祥光绕瑞华。内侍与孤摆銮驾,探望功臣到秦家。}

\vspace{3pt}{\centerline{{[}{\hei 第二场}{]}}}\vspace{5pt}

\setlength{\hangindent}{52pt}{(程咬金\hspace{20pt}({\akai 内})掌灯。)}

\setlength{\hangindent}{52pt}{({\hwfs 程}家院{\hwfs 引}程咬金\textless{}\!{\bfseries\akai 小锣打上}\!\textgreater{})}

\setlength{\hangindent}{52pt}{程咬金\hspace{20pt}【{\akai 西皮摇板}】无事关心心不乱,有事关心心不安。}

\setlength{\hangindent}{52pt}{程咬金\hspace{20pt}老夫程咬金。适才在金殿用花言巧语,蒙哄万岁。可恨那个牛鼻子老道,奏了一本,请万岁明日过府,一来探病,二来取印。哎呀,我想这件事,二哥可是一概不知呀。倘若万岁明日过府,问起情由,二哥一概不知,我岂不是有蒙君之罪么。我不免连夜过府,与二哥送上一信。倘若明日万岁问起情由,也免得二哥临时,诶,失于机变。}

\setlength{\hangindent}{52pt}{程咬金\hspace{20pt}家院。}

\setlength{\hangindent}{52pt}{程家院\hspace{20pt}有。}

\setlength{\hangindent}{52pt}{程咬金\hspace{20pt}掌灯秦府。}

\setlength{\hangindent}{52pt}{程咬金\hspace{20pt}【{\akai 西皮摇板}】谋事在人成事天,心中恼恨黑炭丸。一心要放暗中箭,摆布黑贼有何难。}

\vspace{3pt}{\centerline{{[}{\hei 第三场}{]}}}\vspace{5pt}

\setlength{\hangindent}{52pt}{({\hwfs 秦}家院、秦怀玉{\hwfs 搀}秦琼{\hwfs 上})}

\setlength{\hangindent}{52pt}{秦琼\hspace{30pt}【{\akai 西皮摇板}】投唐保国扶江山,东挡西除马上眠。自从扫北回朝转,疾病缠身整一年。眼观红日西山晚,}

\setlength{\hangindent}{52pt}{(秦琼{\hwfs 入大座})}

\setlength{\hangindent}{52pt}{秦琼\hspace{30pt}【{\akai 西皮摇板}】心中焦躁不耐烦。}

\setlength{\hangindent}{52pt}{(秦琼{\hwfs 睡介},程咬金{\hwfs 上})}

\setlength{\hangindent}{52pt}{程咬金\hspace{20pt}【{\akai 西皮摇板}】万岁金殿把旨传,晓谕文武众两班。明日过府把病探,怕的泄漏这机关。}

\setlength{\hangindent}{52pt}{程家院\hspace{20pt}来在({\akai 或}:~来到)秦府。}

\setlength{\hangindent}{52pt}{程咬金\hspace{20pt}前去通禀,鲁国公求见。}

\setlength{\hangindent}{52pt}{程家院\hspace{20pt}门上哪位在?}

\setlength{\hangindent}{52pt}{秦家院\hspace{20pt}做什么的?({\akai 或}:~什么人?)}

\setlength{\hangindent}{52pt}{程家院\hspace{20pt}鲁国公求见。}

\setlength{\hangindent}{52pt}{秦家院\hspace{20pt}候着。}

\setlength{\hangindent}{52pt}{秦家院\hspace{20pt}启少爷:~鲁国公求见。}

\setlength{\hangindent}{52pt}{秦怀玉\hspace{20pt}启爹爹:~程叔父到。}

\setlength{\hangindent}{52pt}{秦琼\hspace{30pt}怀玉迎接!}

\setlength{\hangindent}{52pt}{秦怀玉\hspace{20pt}遵命。}

\setlength{\hangindent}{52pt}{秦怀玉\hspace{20pt}【{\akai 西皮摇板}】怀玉出了府门前,见了叔父礼当先。}

\setlength{\hangindent}{52pt}{秦怀玉\hspace{20pt}迎接叔父。}

\setlength{\hangindent}{52pt}{程咬金\hspace{20pt}哎,罢了,罢了。怀玉,你父病体如何?}

\setlength{\hangindent}{52pt}{秦怀玉\hspace{20pt}照常一样,呕吐不止。}

\setlength{\hangindent}{52pt}{程咬金\hspace{20pt}你父今在何处?}

\setlength{\hangindent}{52pt}{秦怀玉\hspace{20pt}现在病房。}

\setlength{\hangindent}{52pt}{程咬金\hspace{20pt}带路病房啊!}

\setlength{\hangindent}{52pt}{程咬金\hspace{20pt}【{\akai 西皮摇板}】来在({\akai 或}:~来到)病房用目看,}

\setlength{\hangindent}{52pt}{程咬金\hspace{20pt}唉({\akai 或}:~哎呀)!}

\setlength{\hangindent}{52pt}{程咬金\hspace{20pt}【{\akai 西皮摇板}】看见二哥病容颜。重病缠身容颜变,精神恍惚非当年。}

\setlength{\hangindent}{52pt}{程咬金\hspace{20pt}二哥醒来。}

\setlength{\hangindent}{52pt}{秦琼\hspace{30pt}【{\akai 西皮摇板}】适才朦胧将合眼,耳旁又听有人言。猛然睁开昏花眼,}

\setlength{\hangindent}{52pt}{程咬金\hspace{20pt}二哥。}

\setlength{\hangindent}{52pt}{秦琼\hspace{30pt}【{\akai 西皮摇板}】只见贤弟在眼前({\akai 或}:~只见贤弟在面前)。}

\setlength{\hangindent}{52pt}{秦琼\hspace{30pt}贤弟来了,请坐。}

\setlength{\hangindent}{52pt}{程咬金\hspace{20pt}嘿,谢坐哟。请问二哥,(你)病体如何(呀)?}

\setlength{\hangindent}{52pt}{秦琼\hspace{30pt}照常一样,呕吐不止。}

\setlength{\hangindent}{52pt}{程咬金\hspace{20pt}你乃久病之人,何须忧虑。}

\setlength{\hangindent}{52pt}{秦琼\hspace{30pt}啊,贤弟连夜过府({\akai 或}:~夤夜到此)何事?}

\setlength{\hangindent}{52pt}{程咬金\hspace{20pt}哎呀二哥呀,小弟有一件为难之事({\akai 或}:~要紧之事)。二哥,你要依我才是啊。}

\setlength{\hangindent}{52pt}{秦琼\hspace{30pt}哎,贤弟,你我自结金兰,患难相顾。今日有何为难之事,愚兄依你就是。}

\setlength{\hangindent}{52pt}{程咬金\hspace{20pt}我说是啊,二哥啊,你有所不知啊。张士贵绛州龙门招军已满,有本回奏({\akai 或}:~有本回朝)。万岁见你染病在床,无人挂帅,是龙心未定啊。可恨那个牛鼻子老道,诶,他保尉迟恭挂帅呀。}

\setlength{\hangindent}{52pt}{秦琼\hspace{30pt}哦,那尉迟恭么$\cdots{}\cdots{}$}

\setlength{\hangindent}{52pt}{程咬金\hspace{20pt}嘿,是哦。}

\setlength{\hangindent}{52pt}{秦琼\hspace{30pt}哼,一介村夫,况且目不识丁,焉能(够)掌得帅印。}

\setlength{\hangindent}{52pt}{程咬金\hspace{20pt}是啊。小弟就是因为此事,与黑贼跟------诶,军师争论了几句,圣上命我前来取印。我想这颗帅印,二哥你执掌了多年,我若取回({\akai 或}:~我要取回),岂不白白地送给黑贼之手么。那时小弟,在午门闲游了一番,急回谎奏({\akai 或}:~急忙谎奏)。呃,实指望说几句言语,蒙哄万岁;诶嘿嘿,谁知那个牛鼻子老道,又启奏了一本:~明日过府,一来探病,二来取印。我想这些个事,二哥啊,你可是一概不知啊,我特来通报与你。二哥啊,要是圣驾过府,问起取印之事,二哥啊,你就说小弟我来过了,周全小弟无罪。想你我弟兄,自投唐以来嚯!}

\setlength{\hangindent}{52pt}{程咬金\hspace{20pt}【{\akai 西皮摇板}】算来倒有数十年,并未分首各一天。生死相交共患难,这件事儿要周全。}

\setlength{\hangindent}{52pt}{秦琼\hspace{30pt}贤弟。}

\setlength{\hangindent}{52pt}{秦琼\hspace{30pt}【{\akai 西皮摇板}】可笑({\akai 或}:~堪笑)万岁见识浅,听信军师入蜚言({\akai 或}:~宠信军师入蜚言)。}

\setlength{\hangindent}{52pt}{秦琼\hspace{30pt}贤弟,只管放心,些许小事,由我担承。}

\setlength{\hangindent}{52pt}{程咬金\hspace{20pt}是,谢二哥。}

\setlength{\hangindent}{52pt}{秦琼\hspace{30pt}怀玉,取印过来!}

\setlength{\hangindent}{52pt}{秦怀玉\hspace{20pt}遵命。帅印在此!}

\setlength{\hangindent}{52pt}{秦琼\hspace{30pt}放在床前。}

\setlength{\hangindent}{52pt}{秦琼\hspace{30pt}夜已经深了,贤弟回府去罢。}

\setlength{\hangindent}{52pt}{程咬金\hspace{20pt}二哥,明日那黑贼保驾前来,你必须用言语摆布于他。小弟的言语,你可要牢牢地紧记。我告辞啦!}

\setlength{\hangindent}{52pt}{程咬金\hspace{20pt}【{\akai 西皮摇板}】辞别二哥回身转,}

\setlength{\hangindent}{52pt}{秦琼\hspace{30pt}怀玉代送。}

\setlength{\hangindent}{52pt}{程咬金\hspace{20pt}【{\akai 西皮摇板}】猛然想起巧机关。}

\setlength{\hangindent}{52pt}{秦怀玉\hspace{20pt}送叔父!}

\setlength{\hangindent}{52pt}{程咬金\hspace{20pt}哎,我说怀玉,你可知你父的病体,是因何而得呢?}

\setlength{\hangindent}{52pt}{秦怀玉\hspace{20pt}唉,前者在金殿赌力而得。}

\setlength{\hangindent}{52pt}{程咬金\hspace{20pt}嘿,这可就不对啦。为叔的我要不说,你哪儿知道哇。昔年大战美良川,三鞭换两锏。你父在马鞍鞒上,气堵胸膛,故而成病,至今才发。我要是不说呀,你这一辈子也不明白({\akai 或}:~也不知道)。}

\setlength{\hangindent}{52pt}{秦怀玉\hspace{20pt}依叔父之见?}

\setlength{\hangindent}{52pt}{程咬金\hspace{20pt}依我之见呐,明日圣驾过府探病,那黑贼一定是保驾前来。你父用言语摆布于他,他必然叫骂你父啊,你听见之后啊,就只管地打他。}

\setlength{\hangindent}{52pt}{秦怀玉\hspace{20pt}哎呀,他乃是开国元勋,侄儿怎敢呐。}

\setlength{\hangindent}{52pt}{程咬金\hspace{20pt}他是开国元勋,那你父就不是开国元勋吗?~为叔我------诶,那也不是开国元勋吗?哼。}

\setlength{\hangindent}{52pt}{秦怀玉\hspace{20pt}侄儿打他不过呀。}

\setlength{\hangindent}{52pt}{程咬金\hspace{20pt}诶,小小的年纪,就说这种软弱的话。你打不过,不碍事啊,为叔父的,诶,我帮着你。}

\setlength{\hangindent}{52pt}{秦怀玉\hspace{20pt}哦,侄儿遵命。}

\setlength{\hangindent}{52pt}{程咬金\hspace{20pt}你要记下了!}

\setlength{\hangindent}{52pt}{程咬金\hspace{20pt}【{\akai 西皮摇板}】昔年大战美良城,三鞭两锏赌输赢。明日只管将他打,打出祸来有我老程({\akai 或}:~打出祸来我担承)。}

\setlength{\hangindent}{52pt}{秦怀玉\hspace{20pt}【{\akai 西皮摇板}】适才叔父对我言,为的当年旧仇冤。为子当把父仇报,暗藏心机不漏言。}

\vspace{3pt}{\centerline{{[}{\hei 第四场}{]}}}\vspace{5pt}

\setlength{\hangindent}{52pt}{({\hwfs 吹}\textless{}\!{\bfseries\akai 牌子}\!\textgreater{},大铠{\hwfs 等引}李世民、徐勣、尉迟恭、程咬金{\hwfs 上})}

\setlength{\hangindent}{52pt}{秦怀玉\hspace{20pt}怀玉接驾!}

\setlength{\hangindent}{52pt}{李世民\hspace{20pt}你父病体如何?}

\setlength{\hangindent}{52pt}{秦怀玉\hspace{20pt}照常一样,呕吐不止。}

\setlength{\hangindent}{52pt}{李世民\hspace{20pt}前去通禀,孤王前来探病。}

\setlength{\hangindent}{52pt}{秦怀玉\hspace{20pt}领旨。}

\setlength{\hangindent}{52pt}{秦怀玉\hspace{20pt}({\akai 念})君入臣门第,蓬荜又生辉。}

\setlength{\hangindent}{52pt}{(秦怀玉{\hwfs 下})}

\setlength{\hangindent}{52pt}{李世民\hspace{20pt}程皇兄,吩附銮驾,府外伺候!}

\setlength{\hangindent}{52pt}{程咬金\hspace{20pt}銮驾府外伺候哇。}

\setlength{\hangindent}{52pt}{(大铠{\hwfs 等}众{\hwfs 下})}

\setlength{\hangindent}{52pt}{李世民\hspace{20pt}尉迟皇兄!}

\setlength{\hangindent}{52pt}{尉迟恭\hspace{20pt}万岁。}

\setlength{\hangindent}{52pt}{李世民\hspace{20pt}少时秦恩公问起征东之事,孤必命卿挂帅。他乃有病之人,你必须忍耐。}

\setlength{\hangindent}{52pt}{尉迟恭\hspace{20pt}领旨。}

\setlength{\hangindent}{52pt}{秦怀玉\hspace{20pt}启万岁:~臣父叫之不应,请驾回宫。}

\setlength{\hangindent}{52pt}{李世民\hspace{20pt}孤是为你父病而来,再去通禀,孤在前厅等候。}

\setlength{\hangindent}{52pt}{秦怀玉\hspace{20pt}领旨。}

\setlength{\hangindent}{52pt}{(秦怀玉{\hwfs 下})}

\setlength{\hangindent}{52pt}{李世民\hspace{20pt}唉!~皇兄啊!}

\setlength{\hangindent}{52pt}{李世民\hspace{20pt}【{\akai 西皮摇板}】孤摆銮驾到府门,}

\setlength{\hangindent}{52pt}{(李世民{\hwfs 站起})}

\setlength{\hangindent}{52pt}{李世民\hspace{20pt}【{\akai 西皮摇板}】亭台以外柳青青。山水古画多齐整,厅前瑞草送芳馨。君臣且在前厅等,}

\setlength{\hangindent}{52pt}{(李世民{\hwfs 坐下})}

\setlength{\hangindent}{52pt}{李世民\hspace{20pt}【{\akai 西皮摇板}】等候怀玉报信音。}

\setlength{\hangindent}{52pt}{秦怀玉\hspace{20pt}【{\akai 西皮摇板}】君入臣门多侥幸,龙行一步百草生。}

\setlength{\hangindent}{52pt}{秦怀玉\hspace{20pt}臣启万岁:~臣父昏迷不醒,叫之不应,请驾回宫。}

\setlength{\hangindent}{52pt}{李世民\hspace{20pt}平身。}

\setlength{\hangindent}{52pt}{李世民\hspace{20pt}先生,秦恩公昏迷不醒,如何是好?}

\setlength{\hangindent}{52pt}{徐勣\hspace{30pt}万岁请至({\akai 或}:~请到)病房,将他唤醒。}

\setlength{\hangindent}{52pt}{李世民\hspace{20pt}怀玉,}

\setlength{\hangindent}{52pt}{秦怀玉\hspace{20pt}在。}

\setlength{\hangindent}{52pt}{李世民\hspace{20pt}带路病房!}

\setlength{\hangindent}{52pt}{秦怀玉\hspace{20pt}领旨。}

\setlength{\hangindent}{52pt}{李世民\hspace{20pt}【{\akai 西皮摇板}】恩公昏迷不得醒,去至病房看真情。}

\setlength{\hangindent}{52pt}{(\textless{}\!{\bfseries\akai 大锣打下}\!\textgreater{})}

\vspace{3pt}{\centerline{{[}{\hei 第五场}{]}}}\vspace{5pt}

\setlength{\hangindent}{52pt}{({\hwfs 秦}家院{\hwfs 搀}秦琼{\hwfs 上})}

\setlength{\hangindent}{52pt}{秦琼\hspace{30pt}【{\akai 西皮摇板}】昨日咬金来送信,果然今日驾临门。假装昏迷睡不醒,}

\setlength{\hangindent}{52pt}{(秦琼{\hwfs 入大座},{\hwfs 睡介})}

\setlength{\hangindent}{52pt}{秦琼\hspace{30pt}【{\akai 西皮摇板}】且看万岁怎样行({\akai 或}:~圣上怎样行)。}

\setlength{\hangindent}{52pt}{(秦怀玉{\hwfs 引}李世民、徐勣、尉迟恭、程咬金{\hwfs 上})}

\setlength{\hangindent}{52pt}{李世民\hspace{20pt}【{\akai 西皮摇板}】孤王亲自来探病,功劳打动帝王心。君臣且把病房进,}

\setlength{\hangindent}{52pt}{(众{\hwfs 挖门进})}

\setlength{\hangindent}{52pt}{李世民\hspace{20pt}【{\akai 西皮摇板}】只见恩公睡沉沉。}

\setlength{\hangindent}{52pt}{秦琼\hspace{30pt}【{\akai 西皮散板}】适才朦胧将睡定({\akai 或}:~适才朦胧荏苒\footnote{``荏苒''是逡巡、一刹那的意思。}动),耳旁又听有人声。猛然睁开昏花眼,抬头只见圣明君。}

\setlength{\hangindent}{52pt}{秦琼\hspace{30pt}怀玉,圣驾到此,为何不来通报?}

\setlength{\hangindent}{52pt}{秦怀玉\hspace{20pt}孩儿呼唤爹爹不醒,未敢惊动。}

\setlength{\hangindent}{52pt}{秦琼\hspace{30pt}哼------只恐儿难免欺君之罪。}

\setlength{\hangindent}{52pt}{李世民\hspace{20pt}啊皇兄,此乃寡人自进,与小爱卿无干。}

\setlength{\hangindent}{52pt}{秦琼\hspace{30pt}若非圣上开恩,定要加罪。还不谢过万岁!}

\setlength{\hangindent}{52pt}{秦怀玉\hspace{20pt}谢万岁!}

\setlength{\hangindent}{52pt}{李世民\hspace{20pt}平身。}

\setlength{\hangindent}{52pt}{秦琼\hspace{30pt}万岁驾到臣门,奈臣有病,不能接驾。万岁恕罪!}

\setlength{\hangindent}{52pt}{李世民\hspace{20pt}皇兄,你乃有病之人,谁来怪你。}

\setlength{\hangindent}{52pt}{秦琼\hspace{30pt}谢万岁!}

\setlength{\hangindent}{52pt}{李世民\hspace{20pt}孤今前来问病,但不知你病体如何?}

\setlength{\hangindent}{52pt}{秦琼\hspace{30pt}照常一样,呕吐不止。}

\setlength{\hangindent}{52pt}{李世民\hspace{20pt}你乃久病之人,且免忧虑。}

\setlength{\hangindent}{52pt}{秦琼\hspace{30pt}微臣久病,日加沉重,今见万岁一面,再不能朝见的了,呃$\cdots{}\cdots{}$({\hwfs 哭介})}

\setlength{\hangindent}{52pt}{尉迟恭\hspace{20pt}元帅,某这几日有朝事在身,少来问候。今日保驾前来,问候元帅金安。}

\setlength{\hangindent}{52pt}{秦琼\hspace{30pt}有劳台驾。}

\setlength{\hangindent}{52pt}{程咬金\hspace{20pt}二哥,您听见了没有?他说这几天有朝事在身,少来问候。今日保驾前来,捎带着给你问个好儿。这个人情,诶,你可得领他的。}

\setlength{\hangindent}{52pt}{徐勣\hspace{30pt}你又来多口!}

\setlength{\hangindent}{52pt}{程咬金\hspace{20pt}嘿,我又------哼,多口了。}

\setlength{\hangindent}{52pt}{秦琼\hspace{30pt}有劳众位皇兄前来问病,请坐。}

%徐勣\\尉迟恭\hspace{20pt}{有座。}\\程咬金
\raisebox{0pt}[24pt][16pt]{\raisebox{12pt}{徐勣}\raisebox{0pt}{\hspace{-22pt}{尉迟恭}}\raisebox{-12pt}{\hspace{-32pt}{程咬金}}\raisebox{0pt}{\hspace{20pt}有座!}}

\setlength{\hangindent}{52pt}{秦琼\hspace{30pt}万岁征东一事如何?}

\setlength{\hangindent}{52pt}{李世民\hspace{20pt}孤王昨日设立早朝,张士贵有本回朝,招军已满;王君可海下战船造齐,二人俱有本章还朝。孤见恩公染病在床,无人挂帅,孤心未定。}

\setlength{\hangindent}{52pt}{秦琼\hspace{30pt}万岁征东事大,奈臣,唉,这病$\cdots{}\cdots{}$({\hwfs 哭介})}

\setlength{\hangindent}{52pt}{秦琼\hspace{30pt}【{\akai 西皮原板}】疾病缠身整一春,吉凶二字解不明。残生难逃幽冥境,再不能替主扫烟尘。}

\setlength{\hangindent}{52pt}{李世民\hspace{20pt}【{\akai 西皮原板}】孤王闻言心酸痛({\akai 或}:~泪双淋),好似狼牙箭攒心。恩公有病难挂印,有何人替孤领雄兵。}

\setlength{\hangindent}{52pt}{秦琼\hspace{30pt}【{\akai 西皮原板}】臣子怀玉年纪轻,文韬武略智超群。胸中颇有安邦论,可命他替主({\akai 或}:~挂帅)统雄兵。}

\setlength{\hangindent}{52pt}{李世民\hspace{20pt}【{\akai 西皮原板}】怀玉虽然有本领,年幼怎能压({\akai 或}:~年幼怎能服)老臣。况且皇兄身有病,还要膝下奉晨昏。}

\setlength{\hangindent}{52pt}{秦琼\hspace{30pt}【{\akai 西皮原板}】甘罗十二为宰臣,无志空活百岁龄。怀玉年幼难挂帅,有何人替主掌权衡。}

\setlength{\hangindent}{52pt}{李世民\hspace{20pt}【{\akai 西皮原板}】昨日金殿【{\footnotesize 转}{\akai 西皮快板}】同议论,公保尉迟老将军。因此为王来取印,即日兴兵往东征。}

\setlength{\hangindent}{52pt}{秦琼\hspace{30pt}啊?!}

\setlength{\hangindent}{52pt}{秦琼\hspace{30pt}【{\akai 西皮快板}】听说尉迟挂帅印,急得我心头似火焚。尉迟恭有勇无学问,焉能挂帅统雄兵。非是为臣抗君命,军师分明你错用了人呐。}

\setlength{\hangindent}{52pt}{徐勣\hspace{30pt}【{\akai 西皮摇板}】休道万岁错用人,}

\setlength{\hangindent}{52pt}{徐勣\hspace{30pt}【{\akai 西皮快板}】病缠有力不从心({\akai 或}:~病缠身力不从心)。将军染病一年整,无人挂印掌权衡。尉迟恭暂挂元帅印,剿灭辽东盖苏文({\akai 或}:~扫平辽东盖苏文)。且等将军病安稳,再往辽东扫烟尘。我主龙心似尧、舜,岂负你开国老元勋。}

\setlength{\hangindent}{52pt}{尉迟恭\hspace{20pt}元帅。}

\setlength{\hangindent}{52pt}{尉迟恭\hspace{20pt}【{\akai 西皮摇板}】元帅息怒且消停,}

\setlength{\hangindent}{52pt}{尉迟恭\hspace{20pt}【{\akai 西皮快板}】末将言来听分明:~你双锏打下唐社稷,(某)单鞭挣下锦乾坤。末将虽然无学问,可以挂帅领雄兵。征伐辽东干戈定,元帅大印付将军({\akai 或}:~元帅大印还将军)。非是某有意来夺印,元帅还要三思行。}

\setlength{\hangindent}{52pt}{程咬金\hspace{20pt}老黑。}

\setlength{\hangindent}{52pt}{程咬金\hspace{20pt}【{\akai 西皮快板}】黑贼休要逞舌能,我等俱是有功臣。投唐国公三十六,咬金也能我统雄兵。}

\setlength{\hangindent}{52pt}{尉迟恭\hspace{20pt}你不能。}

\setlength{\hangindent}{52pt}{程咬金\hspace{20pt}我不能啊,你也不能。}

\setlength{\hangindent}{52pt}{尉迟恭\hspace{20pt}你不能。}

\setlength{\hangindent}{52pt}{程咬金\hspace{20pt}诶,你也不能。}

\setlength{\hangindent}{52pt}{李世民\hspace{20pt}且慢。}

\setlength{\hangindent}{52pt}{李世民\hspace{20pt}【{\akai 西皮快板}】二位皇兄免争论,俱是开国老元勋。尉迟皇兄能挂印,程皇兄也能领雄兵。太平原是将军定,原是将军定太平。}

\setlength{\hangindent}{52pt}{李世民\hspace{20pt}【{\akai 西皮快板}】回头来把话论,尊一声恩公听分明:~四海的狼烟俱扫尽,不伏辽东盖苏文。恩公不肯让帅印,征东的事儿({\akai 或}:~征东大事)孤去不成。}

\setlength{\hangindent}{52pt}{秦琼\hspace{30pt}【{\akai 西皮导板}】狼烟一起主亲征呐,}

\setlength{\hangindent}{52pt}{秦琼\hspace{30pt}【{\akai 西皮摇板}】怎敢违抗圣明君。}

\setlength{\hangindent}{52pt}{秦琼\hspace{30pt}\textless{}\!({\bfseries\akai 三}){\bfseries\akai 叫头}\!\textgreater{}万岁!我主!(唉,万岁啊!)}

\setlength{\hangindent}{52pt}{李世明\hspace{20pt}\textless{}\!({\bfseries\akai 三}){\bfseries\akai 叫头}\!\textgreater{}恩公!皇兄!(唉,恩公啊!)}

\setlength{\hangindent}{52pt}{秦琼\hspace{30pt}万岁此番征东,三年不知,五载不晓。主在边庭,臣在京内({\akai 或}:~阃内)。臣不能见君,君不能见臣。臣子怀玉,尚未授职,又无婚配;臣病入膏肓,不久于人世。({\akai 或}:~臣病入膏肓,不久于人世;臣子怀玉,尚未授职,又未婚配。)臣纵死九泉,唉!也是不能瞑目的了哇,呃$\cdots{}\cdots{}$({\hwfs 哭介})}

\setlength{\hangindent}{52pt}{李世民\hspace{20pt}哦!({\akai 或}:~呀!)}

\setlength{\hangindent}{52pt}{李世民\hspace{20pt}【{\akai 西皮二六}】孤王闻言心酸痛,句句言辞记龙心({\akai 或}:~记在心)。倘若是皇兄遭不幸,细听孤王加荣封:~追封王位归正品,儿孙代代({\akai 或}:~子子孙孙)入朝门。孤有公主银屏女,赐与怀玉配为婚。众家国公为媒证,即日里婚配驾登程,老皇兄请放宽心。}

\setlength{\hangindent}{52pt}{秦琼\hspace{30pt}【{\akai 西皮摇板}】叔宝闻言心安稳,纵死九泉也甘心呐。在枕边谢恩把首顿,转面再谢众公卿。}

\setlength{\hangindent}{52pt}{秦琼\hspace{30pt}怀玉。}

\setlength{\hangindent}{52pt}{秦琼\hspace{30pt}【{\akai 西皮二六}】叫怀玉近前来听父命,万岁爷的金言要谨记心。倘若是为父遭不幸,追封王位葬至在山林。我的儿子袭父职品,银屏公主配儿为婚。列位国公为媒证,即日里婚配驾启程。这就是({\akai 或}:~这都是)圣上面应允,儿要三跪九叩谢王(的)恩。}

\setlength{\hangindent}{52pt}{秦怀玉\hspace{20pt}【{\akai 西皮摇板}】怀玉床前遵父命,}

\setlength{\hangindent}{52pt}{秦怀玉\hspace{20pt}【{\akai 西皮快板}】含悲忍泪谢皇恩。恩赐子婿父极品,食王爵禄当报恩。叩罢头来抽身起,}

\setlength{\hangindent}{52pt}{秦怀玉\hspace{20pt}【{\akai 西皮快板}】转面再谢众公卿。小侄在朝受皇恩,还要叔父好看承。倘有一点不到处,休怪怀玉乱胡行。}

\setlength{\hangindent}{52pt}{程咬金\hspace{20pt}儿啊!}

\setlength{\hangindent}{52pt}{程咬金\hspace{20pt}【{\akai 西皮快板}】我儿只管任意行,凡事有我程咬金。你父身归蓬莱境,我的儿,你是这当朝的驸马,你还怕何人呐。}

\setlength{\hangindent}{52pt}{秦琼\hspace{30pt}呃------}

\setlength{\hangindent}{52pt}{秦琼\hspace{30pt}【{\akai 西皮快板}】咬金不必逞舌能,不使良言\footnote{段公平{\scriptsize 君}建议作``不识良言'';《京剧汇编》第九集~赵荣鹏~藏本作``不良之言教子孙''。}训子孙。怀玉({\akai 或}:~我儿)在朝受皇恩,大事全仗徐先生。怀玉年轻须教训,念在当年结义的情。怀玉请过元帅印,}

\setlength{\hangindent}{52pt}{秦琼\hspace{30pt}【{\akai 西皮快板}】再听为父细叮咛:~后堂酒宴安排整({\akai 或}:~安排定),先敬君来后敬臣。}

\setlength{\hangindent}{52pt}{秦怀玉\hspace{20pt}【{\akai 西皮摇板}】怀玉床前遵父命,准备酒宴好看承。}

\setlength{\hangindent}{52pt}{秦琼\hspace{30pt}【{\akai 西皮快板}】见床前({\akai 或}:~见床头)摆列元帅印,见物情伤好惨心({\akai 或}:~惨凄)。咽喉紧哽话难尽,}

\setlength{\hangindent}{52pt}{秦琼\hspace{30pt}【{\akai 西皮摇板}】再叫尉迟猛将军。}

\setlength{\hangindent}{52pt}{秦琼\hspace{30pt}尉迟将军!}

\setlength{\hangindent}{52pt}{尉迟恭\hspace{20pt}元帅。}

\setlength{\hangindent}{52pt}{秦琼\hspace{30pt}万岁此番征东,可是挂你为帅?}

\setlength{\hangindent}{52pt}{尉迟恭\hspace{20pt}正是末将为帅。}

\setlength{\hangindent}{52pt}{秦琼\hspace{30pt}既然挂你为帅,你可晓得为将帅之道?}

\setlength{\hangindent}{52pt}{尉迟恭\hspace{20pt}身为武将,焉有不知为帅之道。}

\setlength{\hangindent}{52pt}{秦琼\hspace{30pt}好,今当万岁金面,你且讲来!}

\setlength{\hangindent}{52pt}{尉迟恭\hspace{20pt}元帅容禀({\akai 或}:~元帅听了):~为帅之道,必须盔缨灿烂,铠甲鲜明;刀枪锋利,金鼓齐鸣({\akai 或}:~锣鼓齐鸣);安营巩固({\akai 或}:~安营坚固),谨守大营;擂鼓而起({\akai 或}:~擂鼓而进),鸣金收兵。战马须要强壮,上阵观看动静。众将不能取胜,某就单鞭匹马------我就杀------杀入万马营中。三合九战,方可收兵。这才是为帅的之道。}

\setlength{\hangindent}{52pt}{秦琼\hspace{30pt}(真乃是)一派的胡言!}

\setlength{\hangindent}{52pt}{程咬金\hspace{20pt}嘿,简直是放屁({\akai 或}:~如同放屁)呀!}

\setlength{\hangindent}{52pt}{秦琼\hspace{30pt}跪近床前,待本帅教导于你!}

\setlength{\hangindent}{52pt}{尉迟恭\hspace{20pt}元帅有何金言,当面请教,何必要跪!}

\setlength{\hangindent}{52pt}{秦琼\hspace{30pt}一定要跪!}

\setlength{\hangindent}{52pt}{尉迟恭\hspace{20pt}谁来跪你!}

\setlength{\hangindent}{52pt}{程咬金\hspace{20pt}要我挂帅啊,嘿,跪一辈子我也跪({\akai 或}:~跪一辈子我也干呐)!}

\setlength{\hangindent}{52pt}{李世民\hspace{20pt}啊尉迟皇兄,看孤份上,你就跪他一膝({\akai 或}:~你且屈膝)。}

\setlength{\hangindent}{52pt}{尉迟恭\hspace{20pt}嗯,臣({\akai 或}:~某家)这里跪下了。}

\setlength{\hangindent}{52pt}{程咬金\hspace{20pt}嘿,我说老黑诶,你要跪,就两条腿都跪下罢,这条腿干什么呐!}

\setlength{\hangindent}{52pt}{(程咬金{\hwfs 踹介},尉迟恭{\hwfs 跪})}

\setlength{\hangindent}{52pt}{秦琼\hspace{30pt}听道:~为将帅者,必须饱读经纶,深通战策;运筹帷幄之中,决胜千里之外呀。行兵不可马踏青苗;将士不可扰害百姓。逢高山莫先登,遇空城莫乱入。高防围困,低防水淹;森林防埋伏,芦苇防火攻。身为元帅,令不乱传。有功嘉赏,有过责罚。有道是:~({\akai 念})朝中天子三诏宣,阃外将军令出山岳动,这言发鬼神惊\footnote{《京剧汇编》第九集~赵荣鹏~藏本作``令出山摇动,严法鬼神惊''。}。渴饮刀头血,困卧马上眠。受得苦中苦,方为人上人。这都是为将帅之道({\akai 或}:~此乃是为将帅之道),须要牢牢谨记!}

\setlength{\hangindent}{52pt}{(程咬金\hspace{20pt}一点儿没病。)}

\setlength{\hangindent}{52pt}{尉迟恭\hspace{20pt}末将全记。}

\setlength{\hangindent}{52pt}{秦琼\hspace{30pt}帅印在此。}

\setlength{\hangindent}{52pt}{尉迟恭\hspace{20pt}拿来。({\akai 或}:~鲁莽了。)}

\setlength{\hangindent}{52pt}{秦琼\hspace{30pt}唗!想这帅印,乃圣上钦赐与我,理当交还万岁({\akai 或}:~交还圣上)。你是甚等样人({\akai 或}:~尔是甚等样人),竟敢前来夺取帅印,真正是无羞无耻,匹夫之辈({\akai 或}:~真正是匹夫之辈,毫不知羞耻)也!}

\setlength{\hangindent}{52pt}{秦琼\hspace{30pt}【{\akai 西皮快板}】不记得当年美良城,三鞭两锏赌输赢。不看万岁、先生面,定要叉出帅府门。}

\setlength{\hangindent}{52pt}{尉迟恭\hspace{20pt}【{\akai 西皮快板}】可恨秦琼太欺心,藐视尉迟有功臣。怒气不息到前厅,}

\setlength{\hangindent}{52pt}{程咬金\hspace{20pt}【{\akai 西皮摇板}】程咬金与你个腚后跟\footnote{《京剧汇编》第九集~赵荣鹏~藏本作``定后跟''。}。}

\setlength{\hangindent}{52pt}{秦怀玉\hspace{20pt}【{\akai 西皮摇板}】准备酒宴多齐整,特请万岁饮杯巡。}

\setlength{\hangindent}{52pt}{秦琼\hspace{30pt}【{\akai 西皮摇板}】请主后堂把宴饮,军师伴驾饮杯巡。({\akai 或}:~后堂酒宴安排整,请先生陪驾饮杯巡。)}

\setlength{\hangindent}{52pt}{(秦琼{\hwfs 作揖},{\hwfs 佯睡介})}

\setlength{\hangindent}{52pt}{李世民\hspace{20pt}【{\akai 西皮摇板}】辞别恩公把宴饮,}

\setlength{\hangindent}{52pt}{(李世民、徐勣{\hwfs 出来},李世民{\hwfs 指印},徐勣{\hwfs 抱印},李世民{\hwfs 下})}

\setlength{\hangindent}{52pt}{徐勣\hspace{30pt}【{\akai 西皮摇板}】后堂奉陪天子君。}

\setlength{\hangindent}{52pt}{(徐勣{\hwfs 下},秦琼{\hwfs 醒},{\hwfs 出座})}

\setlength{\hangindent}{52pt}{秦琼\hspace{30pt}【{\akai 西皮摇板}】好个有道圣明君,赛似({\akai 或}:~亚似)尧、舜掌龙庭。可叹({\akai 或}:~嗟叹)秦琼身染病,不能保主\footnote{《京剧汇编》第九集 赵荣鹏~藏本作``替主''}扫烟尘。}

\setlength{\hangindent}{52pt}{(秦琼{\hwfs 下})}

\vspace{3pt}{\centerline{{[}{\hei 第六场}{]}}}\vspace{5pt}

\setlength{\hangindent}{52pt}{程咬金\hspace{20pt}【{\akai 西皮摇板}】黑贼前厅怒不息,咬金正好搬是非。迈步且进二堂里,}

\setlength{\hangindent}{52pt}{秦怀玉\hspace{20pt}【{\akai 西皮摇板}】见了叔父问端的。}

\setlength{\hangindent}{52pt}{程咬金\hspace{20pt}怀玉你慌慌张张,是为了何事?}

\setlength{\hangindent}{52pt}{秦怀玉\hspace{20pt}请尉迟赴席。}

\setlength{\hangindent}{52pt}{程咬金\hspace{20pt}嘿嘿,是呀?他骂你的父可骂出理来了!}

\setlength{\hangindent}{52pt}{秦怀玉\hspace{20pt}他骂我父何来?}

\setlength{\hangindent}{52pt}{程咬金\hspace{20pt}嘿,他骂你父啊,病不死的老牛精。他说一颗帅印,让与不让,但凭于你,为何当着众家的国公,羞侮于我呀。从前你血气方刚,可以耀武扬威;如今你咽喉啊,就只有一口虚气喽,还待这样的性傲啊。嘿嘿,教你死在阴山背后,永不翻身。这场骂哟!}

\setlength{\hangindent}{52pt}{秦怀玉\hspace{20pt}叔父之言,侄儿不信。}

\setlength{\hangindent}{52pt}{程咬金\hspace{20pt}哎呀,为叔父这大的年纪,还跟你撒谎啊。}

\setlength{\hangindent}{52pt}{秦怀玉\hspace{20pt}依叔父之见?}

\setlength{\hangindent}{52pt}{程咬金\hspace{20pt}嘿,还是昨晚那话呀,忘了啊?打他呀!}

\setlength{\hangindent}{52pt}{秦怀玉\hspace{20pt}哎呀,他乃是开国元勋,侄儿不敢打呀。}

\setlength{\hangindent}{52pt}{程咬金\hspace{20pt}哼,他是开国元勋呐,嘿,如今晚儿,你可就是当朝驸马了。你只管地打他,打完了,嗯,还得教他给你赔个礼儿。}

\setlength{\hangindent}{52pt}{秦怀玉\hspace{20pt}为何与侄儿赔礼呀?}

\setlength{\hangindent}{52pt}{程咬金\hspace{20pt}嘿,你听我告诉你,他必然在前厅,叫骂你父。你悄悄地走在他的背后,给他来个饿虎扑食啊,劈拳就打。那时为叔父的去至后堂,把万岁给请来。你听我再咳嗽这么一声,诶,赶紧翻身在地,百般地你就喊叫。}

\setlength{\hangindent}{52pt}{秦怀玉\hspace{20pt}喊叫什么?}

\setlength{\hangindent}{52pt}{程咬金\hspace{20pt}你就说,诶,儿臣好好请尉迟恭赴宴,谁知他以大压小,将儿臣暴打了一顿。若非父王到此啊,孩儿性命可有亏。那时候为叔就一句话,他就得给你赔礼。}

\setlength{\hangindent}{52pt}{秦怀玉\hspace{20pt}哪一句话?}

\setlength{\hangindent}{52pt}{程咬金\hspace{20pt}我说老黑诶,你敢打当朝驸马,犹如欺君之罪。你说他给你赔礼不赔礼啊?}

\setlength{\hangindent}{52pt}{秦怀玉\hspace{20pt}自然要赔礼。}

\setlength{\hangindent}{52pt}{程咬金\hspace{20pt}嘿,你要记下了!}

\setlength{\hangindent}{52pt}{秦怀玉\hspace{20pt}遵命!}

\setlength{\hangindent}{52pt}{秦怀玉\hspace{20pt}【{\akai 西皮摇板}】叔父之言牢谨记,顷刻之间打尉迟。}

\setlength{\hangindent}{52pt}{程咬金\hspace{20pt}【{\akai 西皮摇板}】娃娃中了我的计,管教老黑儿他暗吃亏。}

\vspace{3pt}{\centerline{{[}{\hei 第七场}{]}}}\vspace{5pt}

\setlength{\hangindent}{52pt}{尉迟恭\hspace{20pt}走哇!}

\setlength{\hangindent}{52pt}{尉迟恭\hspace{20pt}哇啊啊$\cdots{}\cdots{}$}

\setlength{\hangindent}{52pt}{尉迟恭\hspace{20pt}【{\akai 西皮摇板}】恼恨秦琼太无礼,一阵火起\footnote{段公平{\scriptsize 君}建议作``一阵火气''。}往上提。将身且坐前厅椅,气坏当朝老尉迟。}

\setlength{\hangindent}{52pt}{尉迟恭\hspace{20pt}\textless{}\!{\bfseries\akai 叫头}\!\textgreater{}秦琼啊!匹夫!}

\setlength{\hangindent}{52pt}{尉迟恭\hspace{20pt}一颗帅印让与不让,但凭于你。从先血气方刚,可以耀武扬威;如今咽喉只有一口虚气,还是这样地性傲。我把你这病不死的老牛精!}

\setlength{\hangindent}{52pt}{秦怀玉\hspace{20pt}着打!}

\setlength{\hangindent}{52pt}{(尉迟恭、秦怀玉{\hwfs 扭打介})}

\setlength{\hangindent}{52pt}{程咬金\hspace{20pt}嘿嘿,老黑呀,你当着万岁,你还打他呐!}

\setlength{\hangindent}{52pt}{程咬金\hspace{20pt}起来,起来,起来,起来$\cdots{}\cdots{}$嘿,有事别哭,别哭!有什么话,你只管地讲啊。}

\setlength{\hangindent}{52pt}{秦怀玉\hspace{20pt}\textless{}\!{\bfseries\akai 叫头}\!\textgreater{}父王!}

\setlength{\hangindent}{52pt}{秦怀玉\hspace{20pt}儿臣好意请尉迟恭赴席,谁想他以大压小,将儿臣暴打一顿。若非父王到此,唉,儿的性命有亏啊$\cdots{}\cdots{}$({\hwfs 哭介})}

\setlength{\hangindent}{52pt}{尉迟恭\hspace{20pt}哎呀万岁呀,他打了老臣了!}

\setlength{\hangindent}{52pt}{程咬金\hspace{20pt}怎么着?他打了你了,诶,那你怎么在上头,他在底下呢?}

\setlength{\hangindent}{52pt}{尉迟恭\hspace{20pt}哎呀!这个$\cdots{}\cdots{}$喳,喳,喳,喳$\cdots{}\cdots{}$}

\setlength{\hangindent}{52pt}{程咬金\hspace{20pt}哎呀,我说老黑诶,你敢打当朝驸马呀,这就是欺君之罪呀!}

\setlength{\hangindent}{52pt}{李世民\hspace{20pt}着哇!}

\setlength{\hangindent}{52pt}{程咬金\hspace{20pt}你摸摸,还有脑袋吗?}

\setlength{\hangindent}{52pt}{李世民\hspace{20pt}唗!}

\setlength{\hangindent}{52pt}{李世民\hspace{20pt}【{\akai 西皮摇板}】孤王闻言怒冲起,}

\setlength{\hangindent}{52pt}{李世民\hspace{20pt}【{\akai 西皮快板}】开言大骂黑面皮。你本是堂堂国公体,全然不知高和低。怀玉本是东床婿,打他犹如把孤欺。罚俸三载赎你罪,快与驸马把罪赔。}

\setlength{\hangindent}{52pt}{尉迟恭\hspace{20pt}\textless{}\!{\bfseries\akai 叫头}\!\textgreater{}万岁。}

\setlength{\hangindent}{52pt}{尉迟恭\hspace{20pt}【{\akai 西皮快板}】万岁容臣本奏启,细听为臣辩是非:~为臣坐在前厅椅,他背后将椅往下推。就势将臣推在地,脊背打得响如雷。他见万岁来到此,翻身在地假悲啼。花言巧语奏万岁,反说为臣把他欺。臣本是堂堂的国公体,岂与他无知少年把罪替。}

\setlength{\hangindent}{52pt}{程咬金\hspace{20pt}我说怀玉呀,你可别哭,有什么话,你倒是说呀。}

\setlength{\hangindent}{52pt}{秦怀玉\hspace{20pt}\textless{}\!{\bfseries\akai 叫头}\!\textgreater{}父王。}

\setlength{\hangindent}{52pt}{秦怀玉\hspace{20pt}【{\akai 西皮快板}】父王在上容奏启,细听儿臣辩是非:~我父功劳谁能比,盖世忠良属第一。临潼山,把功立,双锏保驾把名提\footnote{录音中``把名提''疑作``压名敌'',此处从《京剧汇编》第九集~赵荣鹏~藏本。}。如今病在牙床里,来在帅府夺帅旗。兵权大印让与你,反来叫骂无礼仪。怀玉虽然小年纪,出山的猛虎抖毛衣。万岁招我东床婿,当今驸马谁不知。以大压小把我欺,你不赔礼我不依。}

\setlength{\hangindent}{52pt}{尉迟恭\hspace{20pt}【{\akai 西皮快板}】娃娃说话太无礼,花言巧语你骂谁。你父与我同一辈,论什么高来论什么低。他双锏打来唐社稷,某单鞭挣下锦华夷。你父临潼把功立,御果园单鞭救驾回。你父功劳谁能比,某的功劳也不亏。花言巧语就是你,要想我赔礼日出西。}

\setlength{\hangindent}{52pt}{程咬金\hspace{20pt}你呀,赔礼罢!}

\setlength{\hangindent}{52pt}{尉迟恭\hspace{20pt}我不能!}

\setlength{\hangindent}{52pt}{程咬金\hspace{20pt}嗨嗨$\cdots{}\cdots{}$打红眼了,我可不惹你!}

\setlength{\hangindent}{52pt}{徐勣\hspace{30pt}尉迟呀!}

\setlength{\hangindent}{52pt}{徐勣\hspace{30pt}【{\akai 西皮快板}】尉迟恭暂忍心头气,我有一言听端的:~怀玉虽然得罪你,大能容小休要提。为人休得心生气,又无烦来又无非。你执意不肯去赔礼,有道是君王有命怎能违。}

\setlength{\hangindent}{52pt}{尉迟恭\hspace{20pt}先生。}

\setlength{\hangindent}{52pt}{尉迟恭\hspace{20pt}【{\akai 西皮快板}】先生说话大有理,背转身来自猜疑。马行夹道难回避,船到江心补漏迟。罢罢罢,暂忍心头气,保主征东挂帅旗。走向前来我忙赔礼,}

\setlength{\hangindent}{52pt}{程咬金\hspace{20pt}嘿,怀玉你看呐,你看瞧尉迟恭这么大的岁数,跪在你的面前,你呀,饶了他罢!}

\setlength{\hangindent}{52pt}{秦怀玉\hspace{20pt}呵,我饶恕于你!}

\setlength{\hangindent}{52pt}{程咬金\hspace{20pt}嘿,老黑诶!你瞧瞧咯,这人情可是我讲的。}

\setlength{\hangindent}{52pt}{尉迟恭\hspace{20pt}怎么,你讲的?诶,我这儿谢谢你喽!}

\setlength{\hangindent}{52pt}{尉迟恭\hspace{20pt}【{\akai 西皮快板}】驸马爷宽宏休要提。万岁驾前告过罪\footnote{段公平{\scriptsize 君}建议作``告个罪''。},臣不该把他少年欺。}

\setlength{\hangindent}{52pt}{李世民\hspace{20pt}【{\akai 西皮摇板}】一见尉迟来赔礼,满天浮云一扫归。怀玉近前听旨意,安排花烛结光辉。孤王回到昭阳去,急送\footnote{夏行涛{\scriptsize 君}建议作``即送'',此处从《京剧汇编》第九集~赵荣鹏~藏本。}公主出宫闱。}

\setlength{\hangindent}{52pt}{(李世民、徐勣、尉迟恭、程咬金{\hwfs 分下})}

\setlength{\hangindent}{52pt}{秦怀玉\hspace{20pt}怀玉送驾!}

\vspace{3pt}{\centerline{{[}{\hei 第八场}{]}}}\vspace{5pt}

\setlength{\hangindent}{52pt}{李世民\hspace{20pt}尉迟皇兄,挂你为帅,当殿谢过。}

\setlength{\hangindent}{52pt}{尉迟恭\hspace{20pt}领旨。}

\setlength{\hangindent}{52pt}{李世民\hspace{20pt}程皇兄,你为三十六路都先锋,带领八十三万人马,去至海口扎营!}

%尉迟恭\\程咬金\raisebox{5pt}{\hspace{20pt}谢万岁!}
\raisebox{0pt}[22pt][16pt]{\raisebox{8pt}{尉迟恭}\raisebox{-8pt}{\hspace{-32pt}{程咬金}}\raisebox{0pt}{\hspace{20pt}谢万岁!}}

}
