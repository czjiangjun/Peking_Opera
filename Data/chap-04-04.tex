\newpage
\phantomsection %实现目录的正确跳转
\section*{\large\hei {贺后骂殿~{\small 之}~赵光义}}
\addcontentsline{toc}{section}{\hei 贺后骂殿~{\small 之}~赵光义}

\hangafter=1                   %2. 设置从第1⾏之后开始悬挂缩进  %}
\setlength{\parindent}{0pt}{

{{[}{\akai 引子}{]}兄亡侄幼,众文武,辅孤登极\footnote{``登极'',《京剧汇编》第一百零九集作``登基'',下同。}。}

{众卿平身。}

{({\akai 念})兄王晏驾龙归西,全凭争先一着棋。满朝文武来辅助,孤王才得立帝基。}

{孤,赵光义。今日登殿受贺,众卿!}

{(众\hspace{40pt}万岁!)}

{孤王登极,不知当降何国号?}

{依卿所奏。}

{代孤传旨\footnote{段公平{\scriptsize 君}建议作``待孤传旨''。},晓谕天下。}

{潘洪听封。}

{封你左班丞相,卿女执掌昭阳正院。}

{赵普听封。}

{封你右班丞相,代孤执掌朝政。}

{曹彬听封。}

{封卿孝义侯之位,代管禁军。}

{苗宗善听封。}

{封你护国军师,子袭父职。}

{潘疆\footnote{《京剧汇编》第一百零九集作``潘强''。}、潘豹。}

{封你二人为镇殿将军。}

{满朝文武,加升三级,大赦天下。}

{有本早奏,无本退班!}

{代孤传旨,宣杨继业上殿!}

{嗯------大胆杨继业,孤王登极,为何不来朝贺?}

{哪里是参驾来迟,分明藐视寡人。看孤登极,心中不服。}

{殿前武士。}

{推出斩了}

{呃------怎么参本是你,保本也是你?}

{依卿所奏。}

{将杨继业赦回!}

{非是孤王不斩于你,念你在朝有十大汗马功劳。今将你削职为民,赐你百亩田园;~无事不准入朝,限三日出京,三日不走,其罪还在。下殿!}

{\setlength{\hangindent}{52pt}{(赵德昭\hspace{20pt}【{\akai 二黄散板}】$\cdots{}\cdots{}$还我锦家邦。)} }

\setlength{\hangindent}{56pt}{【{\akai 二黄散板}】大皇儿来在金殿上,口口声声要家邦。}孤本当下位将国\textless{}\!{\bfseries\akai 哭头}\!\textgreater{}让,

\setlength{\hangindent}{56pt}{【{\akai 二黄散板}】}难学尧、舜------禹、商汤\footnote{《京剧汇编》第一百零九集作``难学尧舜与商汤''。}。

\setlength{\hangindent}{56pt}{【{\akai 二黄散板}】皇儿近前听叔讲,你母子宫中乐安康。}

{\setlength{\hangindent}{52pt}{(赵德昭\hspace{20pt}【{\akai 二黄散板}】$\cdots{}\cdots{}$你今不把江山让,篡位的名儿天下扬。)} }

{诶------}

\setlength{\hangindent}{56pt}{【{\akai 二黄散板}】皇儿休得多言讲,孤不封你自为王。吩咐潘豹与潘疆,}

%$\bigg( \begin{aligned} &\mbox{潘豹}\\&\mbox{潘疆}\mbox{\raisebox{5pt}{\hspace{28pt}有。}} \end{aligned}\bigg)$
\raisebox{0pt}[22pt][16pt]{\bigg(\raisebox{8pt}{潘豹}\raisebox{-8pt}{\hspace{-22pt}{潘疆}}\raisebox{0pt}{\hspace{22pt}有。}\bigg)}

\setlength{\hangindent}{56pt}{【{\akai 二黄散板}】再有人胡言绑云阳}\footnote{云阳,在今约陕西淳化县西北,是秦代监狱、刑场所在地。所以后常以``云阳''或``云阳市曹''{代指刑场。}}{。}

{(赵德昭\hspace{20pt}唉呀!)}

\setlength{\hangindent}{56pt}{【{\akai 二黄慢板}】自盘古立帝基天子为重\footnote{《京剧丛刊》第三十四集~贯大元~口述本作``自盘古立帝邦天子为重'';~《京剧汇编》第一百零九集作``自盘古开天地皇帝为重''。},老皇嫂骂孤王情理难容。论国法该将你残生断送,}

{(贺后\hspace{30pt}谁敢?!)}

{退班!}

{皇嫂!}

\setlength{\hangindent}{56pt}{【{\akai 二黄碰板三眼}】还念你与皇兄掌印正宫。先王爷晏了驾钟鼓齐动,满朝中文武臣议论孤穷\footnote{《京剧丛刊》第三十四集~贯大元~口述本作``议论皆同'';~《京剧汇编》第一百零九集作``议论不同''。}。全都道大皇儿年轻无用,一个个辅保孤驾坐九重啊。孤虽然掌山河依然大宋,并非是外姓人来坐金龙。走向前、再打一躬把皇嫂尊奉,昭阳院改作了养老宫。将皇嫂当作了太后侍奉,崇上徽号容是不容。}

\setlength{\hangindent}{56pt}{【{\akai 二黄原板}】老皇嫂说什么务农耕种,普天下尽都是老王荣封。享荣华、受富贵母子同共,并非是叔为君、侄为臣各自西东。赐皇嫂尚方剑泰山压重}\footnote{{《京剧汇编》第一百零九集作``泰山压众''。}}{,管三宫和六院,大小嫔妃若有违抗任你施行,你从也不从?}

\setlength{\hangindent}{56pt}{【{\akai 二黄原板}】赵德芳我的儿莫要悲痛,近前来听为叔将儿来封:~孤赐你金镶白玉锁,加封你一钦王、二良王、三忠、四正、五德王、六靖王,\footnote{《京剧汇编》第一百零九集作``一秦王、二梁王、三忠王、四正王、五德王、六延王,''。}上殿不参王、你下殿不辞王,再赐你凹面金锏上打昏君、下打谗臣,压定了满朝的文武,哪一个不尊,你是个八贤王,带管孤穷\footnote{《京剧丛刊》第三十四集~贯大元~口述本作``孤躬''。\\另,刘琦先生曾撰文\upcite{Liu_Jinwan}回忆,吴同宾先生亦指出,《贺后骂殿》中的``孤穷''(或``孤穹'')系讹误,当作``孤躬''(因为``躬''与``窮''字相似,旧时艺人误识,乃至因袭)。}。}

\setlength{\hangindent}{56pt}{【{\akai 二黄散板}】贤皇侄从今后莫要悲痛,老皇嫂请回养老宫。}

{\setlength{\hangindent}{52pt}{(贺后\hspace{30pt}【{\akai 二黄散板}】$\cdots{}\cdots{}$三尺龙泉不容留。)} }

{众卿!}

{孤身不爽,是何缘故?}

{(潘洪\hspace{30pt}$\cdots{}\cdots{}$休养。)}

{就命卿家,代孤办理。}

{(潘洪\hspace{30pt}领旨。)}

{退班。}

{(潘洪\hspace{30pt}$\cdots{}\cdots{}$回宫。)}

}
