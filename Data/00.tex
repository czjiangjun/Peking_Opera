%\hypertarget{ux8bf4-ux660e}

	此为个人整理的刘曾复教授说戏录音的文本稿,\textbf{主要根据刘曾复先生为中国戏曲学院提供的百余出说戏录音为底本,并结合刘曾老在其他场合的说戏录音}\upcite{Liu-Shuoxi-Record}%\textsuperscript{{[}1{]}}
\textbf{整理完成的}。其中《太平桥》、《盗宗卷》、《梅龙镇》、《辕门斩子》、《摘缨会》、《上天台》、《一捧雪》、《卖马》、《南阳关》的``总讲本''主要依据《京剧新序》\upcite{Liu_Xinxu-I,Liu_Xinxu-II}%\textsuperscript{{[}2{]}.}
中收录的刘曾复先生整理的剧本并结合说戏录音整理完成;《马鞍山》、《战长沙》的``总讲本''则参考了李舒先生遗作《涉艺所得》%\textsuperscript{{[}3{]}.}
收录的刘曾复先生手书稿和传本并结合说戏录音整理完成的。\textbf{有关剧目中的把子,主要摘录自}《京剧新序》和《京剧老生把子见闻录》%\textsuperscript{{[}4{]}.}
一文记录的开打和舞台调度。

除了上述《太平桥》等十一出剧目,其余剧目的场次安排主要参考了《京剧汇编
(1-109集)》\textsuperscript{{[}5{]}.}、《传统剧目汇编》\textsuperscript{{[}6{]}.}、《京剧丛刊
(1-50集)》\textsuperscript{{[}7{]}.}和``中国京剧戏考''网站\textsuperscript{{[}8{]}.}上的相应的剧目的安排,个别剧目的词句也参考了,``中国京剧老唱片''网站\textsuperscript{{[}9{]}.}上载的老唱片戏词。

剧目按照剧中人物年代排列,部分剧目的年代排序参考了《京剧大戏考》\textsuperscript{{[}10{]}.}和《京剧知识词典(增订版)》\textsuperscript{{[}11{]}.}中的剧目顺序。

基于全面、客观、忠实的记录原则,整理剧目文字的标记说明如下:

\begin{enumerate}
\def\labelenumi{\arabic{enumi}.}
\item
  \textbf{因为本人学识浅陋、加之录音带存年较久,因此文字中有不少存疑处。凡是存疑处,尽量用红色字体标出,}表明此处可能文辞欠通顺,或只是根据字音听写臆测的词句;
\item
  \textbf{刘曾复先生腹笥渊博,在不同的场合说戏时,即使是同一出戏,个别词句也略有出入,文本中尽量作了标注:}
\end{enumerate}

\begin{enumerate}
\def\labelenumi{\arabic{enumi}.}
\item
  每个剧目中凡有出入的唱、念词句标注为:
\end{enumerate}

\begin{quote}
{XX词1}(或:XX词2;XX词3;\ldots{}\ldots{})

{XX句1}(或:XX句2或:XX句3\ldots{}\ldots{})
\end{quote}

\begin{enumerate}
\def\labelenumi{\arabic{enumi}.}
\setcounter{enumi}{1}
\item
  每个剧目中可不念或某些衬字的唱、念标注为:
\end{enumerate}

\begin{quote}
(XX词句)
\end{quote}

\begin{enumerate}
\def\labelenumi{\arabic{enumi}.}
\setcounter{enumi}{2}
\item
  \textbf{除``总讲本''外,``单词本''中,与表演配合的其他人物唱、念(盖口)}标记为:
  (人物 唱、念词句XXX。)
\item
  \textbf{在本人的知识范围内,对一些生僻的典故、词汇作了简要的注解。}
\item
  \textbf{刘曾复先生对唱、念中的虚词非常重视,文本中的虚词标注有限,建议以先生的录音为准。}
\item
  \textbf{由于文字记录的功能有限,此书辑录的主要是说戏的文字内容,关于舞台表演过程中的唱、念关键都没有标注。}
\end{enumerate}
