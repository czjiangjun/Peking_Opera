\newpage
\subsubsection{\large\hei 法场换子\protect\footnote{刘曾复先生在说该戏总讲时,除徐策外的人都说得比较简略,整理时个别地方参考了程君谋、蒋锡康录音唱片进行了增补;刘曾复先生在《京剧书文指伪录》\upcite{ZGXJ1-57_1993}一文中介绍,徐策和夫人穿蓝帔,徐策戴员外巾。该戏有关场次调度也参照此文。}}
\addcontentsline{toc}{subsection}{\hei 法场换子}

\hangafter=1                   %2. 设置从第1⾏之后开始悬挂缩进  %
\setlength{\parindent}{0pt}{

\vspace{3pt}{\centerline{{[}{\hei 第一场}{]}}}\vspace{5pt}

\setlength{\hangindent}{56pt}{夫人\hspace{30pt}({\akai 念})夫受皇家爵,妻沾雨露恩。}

\setlength{\hangindent}{56pt}{徐策\hspace{30pt}({\akai 内})开道!}

\setlength{\hangindent}{56pt}{(\textless{}\!{\bfseries\akai 小锣六幺令}{\hwfs 前段}\!\textgreater{},徐策{\hwfs 下轿},\textless{}\!{\bfseries\akai 小锣原场}\!\textgreater{},{\hwfs 进门})\footnote{刘曾复先生在为樊百乐{\scriptsize 君}说戏时说明,\textless{}\!{\bfseries\akai 六幺令}\!\textgreater{}行路时用;~\textless{}\!{\bfseries\akai 大锣六幺令}{\hwfs 后段}\!\textgreater{}留着张泰上场时用。}}

\setlength{\hangindent}{56pt}{徐策\hspace{30pt}唉!}

\setlength{\hangindent}{56pt}{夫人\hspace{30pt}相爷今日下得朝来,为何这等长叹?}

\setlength{\hangindent}{56pt}{徐策\hspace{30pt}夫人有所不知,({\akai 或}:~夫人呐------听道:~)今日早朝,可恨张泰奸贼将薛猛夫妻调进京来,要害他二人一死,(唉,)倒也罢了哇。最可叹未满三月小薛蛟,也要受皇家一刀之苦。怎不令人长叹呐!}

\setlength{\hangindent}{56pt}{夫人\hspace{30pt}就该寻一计策,搭救忠良才是。}

\setlength{\hangindent}{56pt}{徐策\hspace{30pt}(下官正为此事回来,与夫人商议。)计策倒有哇,只是要应在夫人的身上。}

\setlength{\hangindent}{56pt}{夫人\hspace{30pt}难道说教妾身替他不成?}

\setlength{\hangindent}{56pt}{徐策\hspace{30pt}不是哟。我看金斗孩儿,面带七煞\footnote{七煞是紫微斗数中十四颗主星之一,七煞主``肃杀''。据说``面带七煞''的人往往寿命不长。},终难抚养。意欲带到法场,将薛蛟调换下来,以接薛门宗嗣。不知夫人你的意下如何?}

\setlength{\hangindent}{56pt}{夫人\hspace{30pt}相爷说哪里话来,想你我夫妻,年将半百,只有此子,若是替人------万万不能。\footnote{此句程君谋、蒋锡康录音作``欸------相爷说哪里话来,想你我二老,年过半百,只生金斗孩儿,要他替旁人去死,万万不能!''}}

\setlength{\hangindent}{56pt}{徐策\hspace{30pt}唉!~夫人呐,呃,唉$\cdots{}\cdots{}$({\hwfs 哭介})}

\setlength{\hangindent}{56pt}{夫人\hspace{30pt}万万不得能够!}

\setlength{\hangindent}{56pt}{徐策\hspace{30pt}唉!夫人呐,呃$\cdots{}\cdots{}$({\hwfs 哭介})}

\setlength{\hangindent}{66pt}{徐策\hspace{30pt}【{\akai 二黄快三眼}】恨薛刚小奴才不如禽兽,吃醉了酒全不顾满面惭羞。闯下了滔天祸一人逃走,连累他二爹娘不能到头。把一个两辽王午门斩首,樊夫人拔宝剑自刎人头。眼见得忠良臣乏嗣无后,可怜他斩草除根、寸草不留、天地含忧,怎教我看水流舟,夫人呐!}

\setlength{\hangindent}{56pt}{夫人\hspace{30pt}【{\akai 二黄原板}】老相爷说此话情理不周,听妾身把此事再说从头:~张泰贼与薛家结成仇扣,满朝中文武臣不敢出头。怕的是画虎不成反类狗,那时节船到江心倒做了逆水行舟。}

\setlength{\hangindent}{56pt}{徐策\hspace{30pt}【{\akai 二黄散板}】贤夫人舍不得娇儿金斗,眼见得小薛蛟一命罢休。为忠良我只得屈膝叩首哇,}

\setlength{\hangindent}{56pt}{夫人\hspace{30pt}【{\akai 二黄散板}】老相爷跪埃尘情理不周。}

\setlength{\hangindent}{56pt}{夫人\hspace{30pt}相爷不必如此,妾身应允就是。}

\setlength{\hangindent}{56pt}{徐策\hspace{30pt}多谢夫人。}

\setlength{\hangindent}{56pt}{徐策\hspace{30pt}家院(过来)。}

\setlength{\hangindent}{56pt}{家院\hspace{30pt}有。}

\setlength{\hangindent}{56pt}{徐策\hspace{30pt}将你家少公爷放在食盒之内,抬到法场。再拿我名帖,去见张泰,就说老夫要亲自祭奠。}

\setlength{\hangindent}{56pt}{(家院\hspace{30pt}是。)}

\setlength{\hangindent}{56pt}{(家院{\hwfs 招}丫鬟{\hwfs 抱}小孩({\hwfs 喜神}){\hwfs 同下})}

\setlength{\hangindent}{56pt}{徐策\hspace{30pt}附耳上来(,记下了)。}

\setlength{\hangindent}{56pt}{家院\hspace{30pt}遵命。}

\setlength{\hangindent}{56pt}{夫人\hspace{30pt}啊相爷,妾身也要跟随前去。}

\setlength{\hangindent}{56pt}{徐策\hspace{30pt}法场之上,耳目甚众,去之无益。}

\setlength{\hangindent}{56pt}{夫人\hspace{30pt}妾身要去。}

\setlength{\hangindent}{56pt}{徐策\hspace{30pt}夫人要去?到了法场,看下官眼色行事。}

\setlength{\hangindent}{56pt}{夫人\hspace{30pt}遵命。}

\setlength{\hangindent}{56pt}{徐策\hspace{30pt}(如此)夫人请。}

\setlength{\hangindent}{56pt}{夫人\hspace{30pt}相爷请。}

\setlength{\hangindent}{56pt}{徐策\hspace{30pt}正是:~({\akai 念})可叹薛家世代贤,}

\setlength{\hangindent}{56pt}{夫人\hspace{30pt}({\akai 念})忠良无故把刀餐({\akai 或}:~忠良无辜被刀餐)。}

\setlength{\hangindent}{56pt}{徐策\hspace{30pt}({\akai 念})苍天有灵睁开眼,}

\setlength{\hangindent}{56pt}{夫人\hspace{30pt}({\akai 念})仇报仇来冤报冤。}

\setlength{\hangindent}{56pt}{徐策\hspace{30pt}着哇!~好一个``仇报仇来冤报冤''。}

\setlength{\hangindent}{56pt}{徐策\hspace{30pt}夫人,}

\setlength{\hangindent}{56pt}{夫人\hspace{30pt}相爷。}

\setlength{\hangindent}{56pt}{徐策\hspace{30pt}随我来。}

\vspace{3pt}{\centerline{{[}{\hei 第二场}{]}}}\vspace{5pt}

\setlength{\hangindent}{56pt}{(\textless{}\!{\bfseries\akai 大锣六幺令}{\hwfs 后段}\!\textgreater{},张泰{\hwfs 上})}

\setlength{\hangindent}{56pt}{张泰\hspace{30pt}({\akai 念})树大遮天盖地,根深哪怕风狂。(任他皇亲国戚,一本斩草除根。)\footnote{据程君谋、蒋锡康唱片录音增补。}}

\setlength{\hangindent}{56pt}{张泰\hspace{30pt}老夫张泰,奉圣命监斩薛猛夫妻。刀斧手,将薛猛夫妻押了上来。}

\setlength{\hangindent}{56pt}{马氏\hspace{30pt}哎吓老爷呀,你我夫妻一死,不值要紧,可叹三月孩儿,也要受皇家一刀之苦哇$\cdots{}\cdots{}$({\hwfs 哭介})}

\setlength{\hangindent}{56pt}{马氏\hspace{30pt}【{\akai 二黄散板}】叫你反来你不反,叫你行来你不行({\akai 或}:~叫你行来逃生你不行)。你我一死不要紧,可怜那娇儿也受酷刑。\footnote{据程君谋、蒋锡康唱片录音增补。}}

\setlength{\hangindent}{56pt}{薛猛\hspace{30pt}夫人呐~!}

\setlength{\hangindent}{56pt}{薛猛\hspace{30pt}【{\akai 二黄散板}】薛家世代忠良后,$\cdots{}\cdots{}$怎做那叛逆臣。回头再把张泰论:~苦害薛家为何情?恨不得一足将尔踏,阴曹地府勾尔魂。\footnote{此处程君谋、蒋锡康唱片录音作:~

	薛猛\hspace{30pt}【{\akai 二黄散板}】夫人不必泪双淋,忠良哪怕丧残生。回头便对奸贼论:~ 

	薛猛\hspace{30pt}贼! 
	
	薛猛\hspace{30pt}【{\akai 二黄散板}】阴曹地府勾尔的魂。}}

\setlength{\hangindent}{56pt}{张泰\hspace{30pt}校尉等,将他夫妻绑上法标。有人讨祭,报我知道。\footnote{据程君谋、蒋锡康唱片录音增补。}}

\setlength{\hangindent}{56pt}{家院\hspace{30pt}({\akai 念})奉了相爷命,法场走一程。\footnote{据程君谋、蒋锡康唱片录音增补。}}

\setlength{\hangindent}{56pt}{家院\hspace{30pt}法场之上哪位听事。\footnote{据程君谋、蒋锡康唱片录音增补。}}

\setlength{\hangindent}{56pt}{校尉\hspace{30pt}做什么的?}

\setlength{\hangindent}{56pt}{家院\hspace{30pt}徐老相爷有名帖奉上,前来法场祭奠。}

\setlength{\hangindent}{56pt}{校尉\hspace{30pt}候着。}

\setlength{\hangindent}{56pt}{校尉\hspace{30pt}启相爷,徐相爷有帖拜上。}

\setlength{\hangindent}{56pt}{张泰\hspace{30pt}呈上来。}

\setlength{\hangindent}{56pt}{张泰\hspace{30pt}呜哙呀,这老儿又来多事。}

\setlength{\hangindent}{56pt}{(校尉\hspace{30pt}他的夫人也来了。)\footnote{据程君谋、蒋锡康唱片录音增加。}}

\setlength{\hangindent}{56pt}{张泰\hspace{30pt}(哦,夫人爷来了。)\footnote{据程君谋、蒋锡康唱片录音增加。}命他一祭({\akai 或}:~容他一祭),时辰一到,速报我知。}

\setlength{\hangindent}{56pt}{校尉\hspace{30pt}容你们一祭。}

\setlength{\hangindent}{56pt}{家院\hspace{30pt}祭礼走上。}

\setlength{\hangindent}{56pt}{(丫鬟{\hwfs 随}家院、{\hwfs 四}青袍{\hwfs 上},{\hwfs 中间两个}青袍{\hwfs 抬盒},家院{\hwfs 站台中间},{\hwfs 四}青袍{\hwfs 脸朝里},{\hwfs 开盒},丫鬟{\hwfs 取出}小孩{\hwfs 与}马氏{\hwfs 怀中}小孩{\hwfs 交换}。家院{\hwfs 令}众人{\hwfs 下},青袍{\hwfs 领下},丫鬟{\hwfs 随}青袍{\hwfs 后},{\hwfs 右手抱}小孩、{\hwfs 用左袖盖}小孩{\hwfs 随下},家院{\hwfs 留场上})}

\setlength{\hangindent}{56pt}{家院\hspace{30pt}有请相爷夫人。}

\setlength{\hangindent}{56pt}{徐策\hspace{30pt}({\akai 内})夫人,随我来!}

\setlength{\hangindent}{56pt}{(家院{\hwfs 下})}

\setlength{\hangindent}{56pt}{徐策\hspace{30pt}【{\akai 二黄散板}】夫妻双双到法场呃,}

\setlength{\hangindent}{56pt}{夫人\hspace{30pt}【{\akai 二黄散板}】不见忠良在哪厢。}

\setlength{\hangindent}{56pt}{徐策\hspace{30pt}\textless{}\!{\bfseries\akai 叫头}\!\textgreater{}薛猛!}

\setlength{\hangindent}{56pt}{夫人\hspace{30pt}\textless{}\!{\bfseries\akai 叫头}\!\textgreater{}马氏。}

\setlength{\hangindent}{56pt}{徐策\hspace{30pt}唉!~儿啊$\cdots{}\cdots{}$({\hwfs 哭介})}

\setlength{\hangindent}{56pt}{徐策\hspace{30pt}【{\akai 二黄散板}】他夫妻好比一张弓,}

\setlength{\hangindent}{56pt}{夫人\hspace{30pt}【{\akai 二黄散板}】万马营中抖威风。}

\setlength{\hangindent}{56pt}{徐策\hspace{30pt}【{\akai 二黄散板}】未把箭放弦又\textless{}\!{\bfseries\akai 哭头}\!\textgreater{}断,我的儿啊,}

\setlength{\hangindent}{56pt}{夫人\hspace{30pt}【{\akai 二黄散板}】一到法场一场空。}

\setlength{\hangindent}{56pt}{徐策\hspace{30pt}夫人,天色不早,先回府去吧。}

\setlength{\hangindent}{56pt}{(夫人{\hwfs 过大边},徐策{\hwfs 过小边},夫人{\hwfs 在大边面向小边})}

\setlength{\hangindent}{56pt}{夫人\hspace{30pt}待我辞别辞别。}

\setlength{\hangindent}{56pt}{夫人\hspace{30pt}\textless{}\!{\bfseries\akai 叫头}\!\textgreater{}薛猛!}

\setlength{\hangindent}{56pt}{夫人\hspace{30pt}\textless{}\!{\bfseries\akai 叫头}\!\textgreater{}马氏------我那金$\cdots{}\cdots{}$}

\setlength{\hangindent}{56pt}{(徐策{\hwfs 面向大边},{\hwfs 阻拦介})}

\setlength{\hangindent}{56pt}{徐策\hspace{30pt}噤声!}

\setlength{\hangindent}{56pt}{夫人\hspace{30pt}今生今世难得见的$\cdots{}\cdots{}$亲儿啊$\cdots{}\cdots{}$({\hwfs 哭}{\hwfs 介})}

\setlength{\hangindent}{56pt}{(夫人{\hwfs 下场门下})}

\setlength{\hangindent}{56pt}{徐策\hspace{30pt}正是:~({\akai 念})法场之上冷嗖嗖,绳拿索绑不自由。盖世忠良遭毒手,({\akai 或}:~法鼓嗵嗵打,西山月影斜。黄泉无客店,)}

\setlength{\hangindent}{56pt}{徐策\hspace{30pt}\textless{}\!{\bfseries\akai 叫头}\!\textgreater{}薛猛!}

\setlength{\hangindent}{56pt}{徐策\hspace{30pt}\textless{}\!{\bfseries\akai 叫头}\!\textgreater{}马氏!}

\setlength{\hangindent}{56pt}{徐策\hspace{30pt}({\akai 念})花开花落({\akai 或}:~花开花谢)籽未丢哇。({\akai 或}:~今晚宿谁家。)啊,呃$\cdots{}\cdots{}$({\hwfs 哭介})}

\setlength{\hangindent}{56pt}{徐策\hspace{30pt}【{\akai 反二黄慢板}】见夫人哭出了席棚以外,可怜她抛撇下十月怀胎。催命鼓响嗵嗵魂飞天界,勾命锣仓啷响魄散泉台。这壁厢绑的是薛猛元帅,那壁厢绑的是马氏裙钗。马夫人使双刀名扬四海,女将中可算得出色英才。你夫妻原本是镇守边塞,为什么一心心闯进京来。儿好无才,我的儿啊!}

\setlength{\hangindent}{56pt}{徐策\hspace{30pt}【{\akai 反二黄三眼}】千不该万不该是儿不该,大不该命薛刚私出府来。那奴才寿堂上把寿来拜,二爹娘一见娇儿,溺爱不明,把酒戒来开。三杯酒下咽喉劣性还在,酒壮胆、胆包天闯下祸来。驸马爷张登荣被他踢呃坏,太子爷紫金冠也打落尘埃。保驾的官、文武臣一齐打坏,最不该持香炉去打张泰。张泰贼奏一本将你来害,将儿的一家人捆绑御街。你夫妻尽了忠留名后代({\akai 或}:~你夫妻尽了忠留名四海;你夫妻双双死命里所在),【{\akai 垛板}】最可叹,断送了未满三月小婴孩,捆绑到御街,刀下赴泉台,儿好无才!冤哉冤哉,令人悲哀,好不伤怀,我的儿啊!}

\setlength{\hangindent}{56pt}{徐策\hspace{30pt}【{\akai 反二黄原板}】我也曾送儿的信,儿怎生不解,书信中藏密语儿解之不开。我教儿领人马反出了边塞,儿为何一心心闯进网来({\akai 或}:~闯入网来)。老徐策见此情无计可奈,舍亲生将薛蛟调换下来。待老夫替你家抚养几载,将养\footnote{``将养''即``抚养''之意。}起忠良后祭扫泉台。可怜我年半百绝了后代,绝了后代,}

\setlength{\hangindent}{56pt}{徐策\hspace{30pt}【{\akai 反二黄散板}】恨不得将张泰斧斫刀开。}

\setlength{\hangindent}{56pt}{徐策\hspace{30pt}【{\akai 反二黄散板}】这一旁搀扶起薛猛元帅,马夫人我不便搀你、你$\cdots{}\cdots{}$你自呃己起来。到九泉见先人呐把我话带,你把我舍子的情细说开怀。}

\setlength{\hangindent}{56pt}{徐策\hspace{30pt}【{\akai 反二黄散板}】悲切切哭出了法场以\textless{}\!{\bfseries\akai 哭头}\!\textgreater{}外啊,}

\setlength{\hangindent}{56pt}{徐策\hspace{30pt}【{\akai 反二黄散板}】等候了大炮响啊,收儿的尸骸。}

\setlength{\hangindent}{56pt}{徐策\hspace{30pt}\textless{}\!{\bfseries\akai 叫头}\!\textgreater{}薛猛!}

\setlength{\hangindent}{56pt}{徐策\hspace{30pt}\textless{}\!{\bfseries\akai 叫头}\!\textgreater{}马氏!}

\setlength{\hangindent}{56pt}{徐策\hspace{30pt}我那金$\cdots{}\cdots{}$({\hwfs 惊介})}

\setlength{\hangindent}{56pt}{徐策\hspace{30pt}今生今世难得见的$\cdots{}\cdots{}$唉,亲儿啊$\cdots{}\cdots{}$({\hwfs 哭介})}

\setlength{\hangindent}{56pt}{徐策\hspace{30pt}罢!}

\setlength{\hangindent}{56pt}{(徐策{\hwfs 一跺脚},{\hwfs 下})}

\setlength{\hangindent}{56pt}{张泰\hspace{30pt}校尉等,时辰可到?\footnote{以下至结尾全部据程君谋、蒋锡康录音增补。}}

\setlength{\hangindent}{56pt}{校尉\hspace{30pt}时辰已到呃。}

\setlength{\hangindent}{56pt}{张泰\hspace{30pt}拿去开刀!}

薛猛\\马氏\raisebox{5pt}{\hspace{30pt}好贼------}

\setlength{\hangindent}{56pt}{校尉\hspace{30pt}斩首已毕,现有一婴孩。}

\setlength{\hangindent}{56pt}{张泰\hspace{30pt}呈上来。}

\setlength{\hangindent}{56pt}{张泰\hspace{30pt}呜哙呀,这一婴孩,生得是眉清目秀,不免带回府去,收为义子。唉------呃,``斩草不除根,萌芽依旧生;斩草除了根,萌芽永不生''。}

\setlength{\hangindent}{56pt}{张泰\hspace{30pt}校尉等,将这婴孩,腰铡三截。}

\setlength{\hangindent}{56pt}{校尉\hspace{30pt}斩首已毕。}

\setlength{\hangindent}{56pt}{张泰\hspace{30pt}打道,上殿交旨。}

\vspace{20pt}
{\hei 按}:~此戏中徐策法场哭祭的大段``反二黄''唱腔是余叔岩根据李吉甫的《法场换子》的本子(用余自己的《焚绵山》的本子换取)设计的唱腔。刘曾复先生为樊百乐{\footnotesize 君}另外还示范了传统的谭派《法场换子》的``反二黄''唱法(可参考程君谋、蒋锡康的《法场换子》唱片录音),兹照录如下:~

\setlength{\hangindent}{69pt}{【{\akai 反二黄慢板}】见夫人哭出了席棚以外,可怜她年半百十月怀胎。催命鼓响嗵嗵魂飞天界,救生锣仓啷响魂又转来。}

\setlength{\hangindent}{78pt}{【{\akai 反二黄中三眼}】站席棚先埋怨薛猛元帅,大不该命薛刚私出府来。进什么京来把什么寿拜,二爹娘爱子心又把宴排。三杯酒下咽喉劣性还在,酒壮胆胆包天闯下祸来。御花园众神像打成土块,太子爷紫金冠也打落尘埃。探花郎张登荣也被打坏,最不该上金殿去打张泰。张泰贼奏一本将你来害,将你来害,我的儿啊!}

\setlength{\hangindent}{68pt}{【{\akai 反二黄原板}】因此上薛门中降下祸灾。这一边哭坏了薛猛元帅,转面来再埋怨马氏裙钗。在阳河你就该反出了边塞,为什么将娇儿带进京来。你夫妻双双死情理所在,【{\akai 垛板}】最可叹,小薛蛟,未满三月也被刀开,我的儿啊!}

\setlength{\hangindent}{68pt}{【{\akai 反二黄原板}】只为你薛门中绝了后代,舍金斗将薛蛟调换下来。待老夫替你家抚养几载,将养起忠良后祭扫泉台。可怜我年半百绝了后代,绝了后代,}

\setlength{\hangindent}{56pt}{(薛猛、马氏{\hwfs 跪},徐策{\hwfs 不看二人})}

\setlength{\hangindent}{56pt}{【{\akai 反二黄散板}】恨不得把张泰斧斫刀开。}

\setlength{\hangindent}{56pt}{哎呀!}

\setlength{\hangindent}{56pt}{【{\akai 反二黄散板}】这一旁搀扶起薛猛元帅,马夫人我不便搀你、你$\cdots{}\cdots{}$你自己起来。}

\setlength{\hangindent}{56pt}{【{\akai 反二黄散板}】悲切切哭出了法场以\textless{}\!{\bfseries\akai 哭头}\!\textgreater{}外啊,}

\setlength{\hangindent}{56pt}{【{\akai 反二黄散板}】等候了大炮响收儿的尸骸。}

