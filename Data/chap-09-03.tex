\newpage
\subsubsection{\textrm{美良川 {\small 之} 秦琼}~\protect\footnote{根据刘曾复先生钞录的秦琼``单词本''整理。\\
	\vspace{7pt}
\hangafter=1                   %2. 设置从第1⾏之后开始悬挂缩进  %
\setlength{\parindent}{0pt}{
	此戏花脸唱一支《八声甘州》,据《梅兰芳回忆录:舞台生活四十年》\upcite{Mei-Remember}%\textsuperscript{{[}28{]}.}
记载,词句为:~\\
	\vspace{5pt}
{\hei ``扬威奋勇,看愁云惨惨,杀气腾腾。鞭鞘指处,鬼神尽觉惊恐。三关怒冲千里振,八寨雄兵已成空。旌旗摇,剑戟丛,将军八面展威凤。人似虎,马如龙,伫看一战使成功!''}\\
	\vspace{7pt}
《铁笼山》一剧中姜维唱的《八声甘州》即出自此戏。}
}}%\protect\hyperlink{fn671}{\textsuperscript{671}}
\addcontentsline{toc}{subsection}{\hei 美良川~\small{之}~秦琼}

\hangafter=1                   %2. 设置从第1⾏之后开始悬挂缩进  %
\setlength{\parindent}{0pt}{
{\centerline{\textrm{{[}\hei 第一场{]}}}}
\vspace{5pt}

({\hwfs 上})

({\akai 念})头戴金盔凤翅飘,身穿铠甲络丝绦。劈抡双锏无人抵,保定我主锦龙朝。

俺,姓秦名琼字叔宝,唐室驾前为臣。奉主之命跟随二主千岁大战刘武周,可恨那贼战又不战,降又不降。今日闲暇无事,不免到二主营中问安。

吓!进得营来为何这样静悄悄的,待我两厢问来。三军们,主公可在营中?

哪里去了?

何人保驾?

不好了!

且住!二主夜探白璧关\footnote{刘曾复先生钞本作``北璧关'',此处从《说唐全传》;历史上李世民破刘武周麾下尉迟敬德于美良川,是其平定北方割据势力刘武周、宋金刚的关键战役(柏壁之战)的一部分,《旧唐书》、《新唐书》和《资治通鉴》均有记载。}%\protect\hyperlink{fn672}{\textsuperscript{672}}
,咬金保驾岂是那黑贼对手?众将官,迎上前去。({\hwfs 下})

\vspace{3pt}{\centerline{\textrm{{[}{\hei 第二场}{]}}}}\vspace{5pt}

({\hwfs 上})

呔!尔有何本领,擅敢追杀我主?

若问你老爷的,尔且听道。

呔!尔敢是怯战?你我两厢问来。

三军的,哪里宽阔?

打道美良川。({\hwfs 下})

\vspace{3pt}{\centerline{\textrm{{[}{\hei 第三场}{]}}}}\vspace{5pt}

({\hwfs 上})

来到美良川,你我怎样比试?

下得马来,何以为赌?何为打鞭换锏?

如此说来老爷先打。

老爷先打。

好,你我两厢问来。

三军的,何处地界?

黑贼,乃是我唐室地界,还是老爷先打。

哼!老爷打了就无有尔的份了。让尔先打。

这作什么?

你老爷站得稳,尔只管的打。

吐什么?

你老爷焉有吐红之理?尔只管的打。

又吐什么?

你老爷方才言过焉有吐红之理,尔只管的打。

你敢有逃走之意?若要你老爷不打,除非在老爷胯下趱将过去,俺便饶尔不死。

当真要打?

果然要打?

要打?

起鼓招打。

呔!尔为何闪你老爷这头一锏?

九十斤一根,

慢说是两锏,就是这一锏也要结果尔的性命。

当真要打?

果然要打?

要打?

起鼓招打\footnote{刘曾复先生钞本作``起鼓照打'',此处从段公平{\scriptsize 君}建议,为上下文统一改。}
%\protect\hyperlink{fn673}{\textsuperscript{673}}
。

与你老爷吐,吐红。

起鼓招打。

带马。({\hwfs 追下})

\vspace{3pt}{\centerline{\textrm{{[}{\hei 第四场}{]}}}}\vspace{5pt}

({\hwfs 上})

前道为何不行?

人马列开。

【{\akai 西皮摇板\footnote{刘曾复先生钞本未注明板式。}%\protect\hyperlink{fn674}{\textsuperscript{674}
}】秦琼生来不可当\footnote{``不可当''犹言``不得了''之意。}
%\protect\hyperlink{fn675}{\textsuperscript{675}}
,美良川前摆战场。三鞭打不动秦叔宝,两锏打得他吐红光。

败兵不可追赶,人马回营。({\hwfs 下})}

