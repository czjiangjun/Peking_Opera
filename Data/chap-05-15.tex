\newpage
\phantomsection %实现目录的正确跳转
\section*{\large\hei {南天门~{\small 之}~曹福}}
\addcontentsline{toc}{section}{\hei 南天门~{\small 之}~曹福}

\hangafter=1                   %2. 设置从第1⾏之后开始悬挂缩进  %
\setlength{\parindent}{0pt}{

\vspace{3pt}{\centerline{{[}{\hei 第一场}{]}}}\vspace{5pt}

{(曹玉莲\hspace{20pt}({\akai 内})【{\akai 西皮导板}】急急忙忙走得慌,)}

\setlength{\hangindent}{56pt}{【{\akai 西皮散板}】点点珠泪洒胸膛啊。({\akai 或}:~逃出主仆人一双。)}

\setlength{\hangindent}{56pt}{【{\akai 西皮散板}】虎口内逃出了两只羊。}

{啊小姐,且喜逃出虎口,待老奴随同小姐慢慢行走。}

{是。}

{\setlength{\hangindent}{52pt}{(曹玉莲\hspace{20pt}【{\akai 西皮原板}】$\cdots{}\cdots{}$魏忠贤$\cdots{}\cdots{}$)} }

\setlength{\hangindent}{56pt}{【{\akai 西皮原板}】我朝中出谗臣搅乱家邦。}

{\setlength{\hangindent}{52pt}{(曹玉莲\hspace{20pt}【{\akai 西皮原板}】天启爷坐山河$\cdots{}\cdots{}$)} }

\setlength{\hangindent}{56pt}{【{\akai 西皮原板}】太老爷做天官吏部大堂。}

{\setlength{\hangindent}{52pt}{(曹玉莲\hspace{20pt}【{\akai 西皮原板}】$\cdots{}\cdots{}$)} }

\setlength{\hangindent}{56pt}{【{\akai 西皮原板}】丢了官罢了职贬回故乡。}

{\setlength{\hangindent}{52pt}{(曹玉莲\hspace{20pt}【{\akai 西皮原板}】$\cdots{}\cdots{}$)} }

\setlength{\hangindent}{56pt}{【{\akai 西皮原板}】有司羽}\footnote{此处《京剧汇编》第四集~陈少霖~藏本作``刘司羽''。按剧情,天启年间,吏部尚书曹正邦得罪魏忠贤被贬,携眷返故里;~魏忠贤遣心腹刘司羽中途埋伏,杀曹正邦全家;~仅女儿曹玉莲在老仆曹福掩护下得以逃脱。}{领人马暗地埋藏。}

{\setlength{\hangindent}{52pt}{(曹玉莲\hspace{20pt}【{\akai 西皮原板}】$\cdots{}\cdots{}$)} }

{\textless{}\!{\bfseries\akai 哭头}\!\textgreater{}太老爷,啊,太老爷!}

\setlength{\hangindent}{56pt}{【{\akai 西皮原板}】最可叹忠良臣无有下场。}

{\setlength{\hangindent}{52pt}{(曹玉莲\hspace{20pt}【{\akai 西皮原板}】我的母花井内也把命丧,)} }

{\textless{}\!{\bfseries\akai 哭头}\!\textgreater{}太夫人{\footnotesize 呐}啊!}

{(曹玉莲\hspace{20pt}喂呀,儿的娘啊$\cdots{}\cdots{}$)}

\setlength{\hangindent}{56pt}{【{\akai 西皮原板}】就是那铁石人也要悲伤。}

{(小姐因何不走哇?)}

{这$\cdots{}\cdots{}$}

{(小姐,)老奴走得慌忙({\akai 或}:~走得忙迫),分文未带,如何是好?}

{是。}

{在。}

{遵命。}

{这里掌柜的有么?}

{这有金耳环一对,照市价合来。}

{掌柜的请看。}

{金子,自然是黄色的呀。}

{唉!

人不在时中,金子也变成铜了。}

{待我那厢去问({\akai 或}:~那厢去换)。}

{啊,这里掌柜的有么?}

{(掌柜

何事?)}

{这有金耳环一对,照市价合来。}

{掌柜的请看。}

{掌柜的好眼力呀({\akai 或}:~真真好眼力呀)!}

{是。}

{请问掌柜的,这里可有大米饭食无有{\footnotesize 啊}?}

{呃,面食也好哇,掌柜的取来。}

{少刻把你。}

{啊小姐,金耳环一对,三钱重,他们这里是十四换呐。三兑三}\footnote{《京剧汇编》第四集~陈少霖~藏本作``三得三''。}{,三四一两二,文银四两,外找大钱二百。小姐收下。}

{是。}

{啊小姐,他们这里无有({\akai 或}:~没有)大米饭食,现有面食在此,小姐请用。}

{啊,小姐为何不用啊?}

{是。({\akai 或}:~哦,是是是。)}

{唉,呃$\cdots{}\cdots{}$({\hwfs 哭介})}

{唉,老奴也是思念太老爷、太夫人,吞吃不下呀,呃$\cdots{}\cdots{}$({\hwfs 哭介})}

{是。}

{在。}

{遵命。}

{啊掌柜的,面食未用,但不知要把多少钱?}

{哎呀,处处俱有好人呐。}

{请问掌柜的,此处可有往大同去的脚程无有哇?}

{哦,有劳了!}

{诶------这是哪个的脚程呐?}

{正是。}

{我们要到({\akai 或}:~我们要往)大同去。}

{大道怎说,小道怎讲啊?}

{自然是走近不走远呐。}

{多把银钱与你呀。}

{呀呸!吃了我这人,是小事;~吃了你这脚程,倒是大事。({\akai 或}:~吃了你的脚程,是大事;~吃了我这人,倒是小事。)难道说我这人还不如你那畜类么?}

{(哼!)真真地岂有此理呀!}

{({\hwfs 思介})也罢,待我蒙哄小姐,前面去雇。}

{啊小姐,此处无有往大同去的脚程,待老奴随同小姐,前面去雇。}

{(是。)}

{(呃!)}

\setlength{\hangindent}{56pt}{【{\akai 西皮快板}】恨贼子把我的牙咬坏,又埋怨太老爷做事无才。孤雁儿失落在天边外,连累得夫人花井埋。}

\setlength{\hangindent}{56pt}{【{\akai 西皮快板}】有一日拿奸贼与国除害,剥尔的皮、挖尔眼方称心怀,太老爷------你快显灵来。}

{哦,来了!}

{\vspace{3pt}{\centerline{{[}{\hei 第二场}{]}}}\vspace{5pt}}

{啊小姐,你为何不走哇?}

{哦!}

\setlength{\hangindent}{56pt}{【{\akai 西皮快板}】小姑娘啼哭坐土台,珠泪点点洒下来。自幼未出闺门界,鞋弓袜小步难捱}\footnote{``步难捱'',《京剧汇编》第四集~陈少霖~藏本作``步难挨'';~类似地``好把路捱''作``好把路挨''。}{。思想爹娘心放开,头上取下金钗来。缠足带,松放解,轻轻刺破红绣花鞋,好把路捱。}

{(哎呀!)}

\setlength{\hangindent}{56pt}{【{\akai 西皮散板}】霎时天气变得快,大雪鹅毛飘下来。荒郊俱被冰雪盖,处处楼阁似银台。}

{啊小姐,你、你$\cdots{}\cdots{}$你为何又不走啊?}

{噢!}

\setlength{\hangindent}{56pt}{【{\akai 西皮快板}】小姑娘啼哭坐山边,大雪纷纷遍地漫。腹内无食身寒冷,哪个身穿({\akai 或}:~哪个多穿)几件棉。咬定牙关朝前趱{\footnotesize 呐},}

\setlength{\hangindent}{56pt}{【{\akai 西皮散板}】小姑娘只哭得实可怜。无奈何脱下了衣一【{\akai 回龙}】件,}

\setlength{\hangindent}{56pt}{【{\akai 西皮散板}】赠与小姐来遮寒。}

\setlength{\hangindent}{56pt}{【{\akai 西皮散板}】男子头上有三昧火,}

{(我冷!)}

{(呀!唔$\cdots{}\cdots{}$)}

\setlength{\hangindent}{56pt}{【{\akai 西皮散板}】我比小姐胜十番。}

\setlength{\hangindent}{56pt}{【{\akai 西皮散板}】送姑娘到大同完成婚嫁,留老奴吃一碗闲饭安茶。({\akai 或}:~小姑娘说的是哪里话,讲什么与老奴戴孝披麻。)}

\setlength{\hangindent}{56pt}{【{\akai 西皮散板}】小姑娘说的是哪里话,讲什么与老奴戴孝披麻呀。奴死后四块板}\footnote{``四块板''是棺材的别称,因为最简陋的棺材没有头、足板(略强于芦席裹尸安葬)。}{高岗埋下}\footnote{段公平{\scriptsize 君}作``刚刚埋下''。}{,胜似你姑娘家修庙造塔。({\akai 或}:~送姑娘到大同完成婚嫁,留老奴吃一碗闲饭安茶。)}

{$\cdots{}\cdots{}$不才。}

\setlength{\hangindent}{56pt}{【{\akai 西皮散板}】小姑娘说的是一派的疯话,讲什么与太老爷半点不差。}

{(【{\akai 西皮散板}】奴欺主我就该天雷报打}\footnote{李元皓{\scriptsize 君}建议作``天雷爆打'';~夏行涛{\scriptsize 君}建议作``天雷暴打''。}{,怕的是({\akai 或}:~怕只怕)五阎君差鬼来拿。)}

\setlength{\hangindent}{56pt}{【{\akai 西皮散板}】行一步来至在深山野洼}\footnote{夏行涛{\scriptsize 君}建议作``深山野凹''。}{,}

{(哎呀!)}

\setlength{\hangindent}{56pt}{【{\akai 西皮散板}】见一座独木桥把我吓煞。}

{啊小姐,看前面有一独木朽桥,待老奴将桥垫稳,也好行走。}\footnote{陈超老师注:~此处刘曾复先生强调,老生身段很细腻:~老生搬两次石头,石头凉,用衣裳垫上,再试。过桥后老生在大边外角``踹鸭''后倒,旦角到小边里角,二人斜着一条线。舞台调度特别。}

{(曹玉莲\hspace{20pt}\setlength{\hangindent}{56pt}{【{\akai 西皮散板}】$\cdots{}\cdots{}$)} }

\setlength{\hangindent}{56pt}{【{\akai 西皮散板}】似这等冰雪天实难招架,四下里风不顺飘落雪花。}

\setlength{\hangindent}{56pt}{【{\akai 西皮散板}】小姑娘休得要心中害怕({\akai 或}:~担惊害怕),有老奴在身边万无一差。}

{(曹玉莲i\hspace{20pt}\setlength{\hangindent}{56pt}{【{\akai 西皮散板}】$\cdots{}\cdots{}$行路难。)} }

\setlength{\hangindent}{56pt}{【{\akai 西皮散板}】你道我行走不方便,紧走几步姑娘观。}

\setlength{\hangindent}{56pt}{【{\akai 西皮散板}】迈开大步朝前趱,}

\setlength{\hangindent}{56pt}{【{\akai 西皮散板}】险些跌倒在深渊。}

\setlength{\hangindent}{56pt}{【{\akai 西皮散板}】四肢无力身寒战,}

\setlength{\hangindent}{56pt}{【{\akai 西皮散板}】不觉来到哇广花山}\footnote{一般作``广华山'',此处从《京剧汇编》第三集~陈少霖~藏本。}{。}

\setlength{\hangindent}{56pt}{【{\akai 西皮导板}】耳边厢又听得有人呼唤,}

{唉!小姐呀,呃$\cdots{}\cdots{}$({\hwfs 哭介})}

\setlength{\hangindent}{56pt}{【{\akai 西皮二六}】尊一声小姑娘细听我言:~实指望保小姐脱离此难,又谁知在中途不得周全。倘若是到不了大同地面,舍下了小姑娘,这样的寒天、大雪纷飞、孤单单你好不可怜。(我的小姑娘啊!}\footnote{段公平{\scriptsize 君}注:~余叔岩通常唱此句,老唱法时常不唱,与之相似的如``好把路捱''句。}{)}

\setlength{\hangindent}{56pt}{【{\akai 西皮摇板}】忽然抬头来观看,}

{来了!(来了!)}

\setlength{\hangindent}{56pt}{【{\akai 西皮摇板}】半空中又来了八洞神仙:~}

\setlength{\hangindent}{56pt}{【{\akai 西皮摇板}】汉钟离会同着李铁拐呀,}

\setlength{\hangindent}{56pt}{【{\akai 西皮摇板}】曹国舅搀扶着果老仙。}

\setlength{\hangindent}{56pt}{【{\akai 西皮摇板}】蓝采和、吕纯阳在空中显见,}

\setlength{\hangindent}{56pt}{【{\akai 西皮摇板}】韩湘子、何仙姑离了广寒({\akai 或}:~暂离广寒)。}

\setlength{\hangindent}{56pt}{【{\akai 西皮摇板}】那王母娘娘在莲台坐,}

\setlength{\hangindent}{56pt}{【{\akai 西皮摇板}】有金童和玉女随侍两边。}

\setlength{\hangindent}{56pt}{【{\akai 西皮摇板}】东南角下观一眼,}

{(呃,来了!又来了!)}

\setlength{\hangindent}{56pt}{【{\akai 西皮摇板}】又来了白发苍苍一洞老神仙({\akai 或}:~一个老神仙)。}

\setlength{\hangindent}{56pt}{【{\akai 西皮摇板}】手执鲜花({\akai 或}:~手持鲜花)呵呵笑,}

\setlength{\hangindent}{56pt}{【{\akai 西皮摇板}】他笑我保主不周全。}

\setlength{\hangindent}{56pt}{【{\akai 西皮摇板}】你也笑来我也笑,}

{哈哈,哈哈,啊,唔$\cdots{}\cdots{}$}

\setlength{\hangindent}{56pt}{【{\akai 西皮散板}】三魂渺渺归九泉。}

{参见金母!}

{圣寿无疆!}

{\textless{}{\!\bfseries\akai 叫头}\!\textgreater{}小姐!}

{此处离大同不远,少刻就有人前来迎接于你。恕老奴不能远送了!}

{且住!想我曹福,一世为奴,今登仙界,怎不令人({\akai 或}:~好不令人)可笑哇!}

{哈哈,哈哈,}

{啊------}

{唉,只是苦了你了哇,呃$\cdots{}\cdots{}$({\hwfs 哭介})}

{罢!}

}

\vspace{20pt}
{\hei 附注:~}

{\setlength{\parindent}{22pt}本剧中曹玉莲之父曹正邦的原型为曹定邦。据《怀安县文史资料》\upcite{Huaian_History}载:~曹定邦(?-1624),生年无考,明代宣府怀安卫魏宁庄(今魏家山)人。万历十九年(\textrm{1591}年)举乡试第一,次年中进士,授江苏淮安府推官,专管一府刑狱。后以治行高第,授吏部给事中,钞发章疏,稽察违误,权力颇重。以法疏劾两京兵部尚书田乐、邢介及云南巡抚陈用宾,田、邢遂引去。吏部郎中赵邦清被诬,定邦疏雪之。后拜谒告归,僦屋以居,不蔽风日。光宗登位,始以太常少卿召,至则改为大理少卿、迁左佥都御史,佐赵南星京察,事竣进左副都御史。天启三年(\textrm{1623}年)秋,吏部缺右侍郎,中旨特用定邦,定邦四辞不得,遂引疾归。天启四年(\textrm{1624}年),启用其为南京都御史,仍辞不拜。时逆臣王绍徽、乔应甲附宦官魏忠贤,必欲害定邦,嘱其党石三畏以东林党劾之,遂予削夺,归乡途中,被东厂派人格杀,并予抄家。其女曹玉莲在家得讯,由老管家陪同往大同逃亡,投奔其未婚翁李总兵,不敢走大路,而于大雪纷纷中自四十里崎岖艰险之桦木山冒寒奔逃,行至大同近郊白登堡,老管家冻馁而死。当地绅民钦其忠烈,为之建庙奉祀。定邦笃志正学,操履刚直,立朝守正不阿,崇奖名教,有古大臣风。\textrm{1920}年代,魏家山曹氏宗祠还设有曹定邦的牌位,供桌上陈列其生前官帽、朝服、传略等。}
