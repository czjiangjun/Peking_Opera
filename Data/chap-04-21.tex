\newpage
\phantomsection %实现目录的正确跳转
\section*{\large\hei {八大锤}}
\addcontentsline{toc}{section}{\hei 八大锤}

\hangafter=1                   %2. 设置从第1⾏之后开始悬挂缩进  %}
\setlength{\parindent}{0pt}{

\vspace{3pt}{\centerline{{[}{\hei 第一场}{]}}}\vspace{5pt}

\setlength{\hangindent}{52pt}{(\textless{}\!{\bfseries\akai 撤锣}\!\textgreater{}\textless{}\!{\bfseries\akai 帽子头}\!\textgreater{}王佐{\hwfs 上})}

\setlength{\hangindent}{52pt}{王佐\hspace{30pt}({\akai 念})若为({\akai 或}:~欲为)天下奇男子,须立人间未有功。}

\setlength{\hangindent}{52pt}{探子\hspace{30pt}陆文龙讨战。}

\setlength{\hangindent}{52pt}{岳飞\hspace{30pt}再探!}

\setlength{\hangindent}{52pt}{探子\hspace{30pt}得令!}

\setlength{\hangindent}{52pt}{岳飞\hspace{30pt}\textless{}\!{\bfseries\akai 叫头}\!\textgreater{}天呐,天!番邦出了陆文龙,此乃------天亡宋也。}

\setlength{\hangindent}{52pt}{王佐\hspace{30pt}啊元帅,想那陆文龙,敢莫是当年潞安州节度使陆登之子么?}

\setlength{\hangindent}{52pt}{岳飞\hspace{30pt}正是。}

\setlength{\hangindent}{52pt}{王佐\hspace{30pt}闻得他父命丧金人之手,如今为何反助仇人?}

\setlength{\hangindent}{52pt}{岳飞\hspace{30pt}贤弟哪里知道,当初金兵大破潞安州,此子未满三月,他怎能知晓。}

\setlength{\hangindent}{52pt}{王佐\hspace{30pt}也罢,待俺王佐,诈降番营。顺说陆文龙来降,不知元帅意下如何?}

\setlength{\hangindent}{52pt}{岳飞\hspace{30pt}唉,贤弟,画虎不成反类其犬。哎,你料理军务去吧。}

\setlength{\hangindent}{52pt}{王佐\hspace{30pt}是,告呃退。}

\setlength{\hangindent}{52pt}{岳飞\hspace{30pt}众将官,小心防守。}

\vspace{3pt}{\centerline{{[}{\hei 第二场}{]}}}\vspace{5pt}

\setlength{\hangindent}{52pt}{(王佐\hspace{30pt}唉!)}

\setlength{\hangindent}{52pt}{王佐\hspace{30pt}【{\akai 二黄导板}】听谯楼打初更玉兔东上,}

\setlength{\hangindent}{52pt}{王佐\hspace{30pt}【{\akai 回龙}】为国家、秉忠心、食君禄、报王恩,昼夜奔忙。}

\setlength{\hangindent}{52pt}{王佐\hspace{30pt}【{\akai 二黄原板}】想当年在洞{\footnotesize 呃}庭逍遥放荡,到如今食君禄未报宋王。岳大哥他待我手足一样,我王佐无功劳怎受荣光?今夜晚思一计番营去闯,落一个美名儿万载传扬。}

\setlength{\hangindent}{52pt}{王佐\hspace{30pt}想俺({\akai 或}:~想我)王佐,自投宋以来,寸功未立。今日岳元帅杀得大败。俺王佐若能思得一计,诈降番营,顺说陆文龙来降,岂不是大功一场,名垂千古?}

\setlength{\hangindent}{52pt}{王佐\hspace{30pt}【{\akai 二黄原板}】怎能够今夜晚番营得进,前后话向文龙细说真情({\akai 或}:~衷情)。}

\setlength{\hangindent}{52pt}{王佐\hspace{30pt}【{\akai 二黄原板}】前也思、后又想无有计定,倒不如上公案细看古今。}

\setlength{\hangindent}{52pt}{王佐\hspace{30pt}《前唐》?}

\setlength{\hangindent}{52pt}{王佐\hspace{30pt}不好哇!}

\setlength{\hangindent}{52pt}{王佐\hspace{30pt}《后汉》!}

\setlength{\hangindent}{52pt}{王佐\hspace{30pt}呜哙呀!想汉室卫律、苏武,同往北国催贡,一个降顺番邦,一个打入羊群,食毡饮雪,还是忠心不改,与岳大哥一般无二矣!}

\setlength{\hangindent}{52pt}{王佐\hspace{30pt}【{\akai 二黄原板}】汉室中({\akai 或}:~汉朝中)卫律声名不正,却为何那苏武一片丹心。饥食毡、渴饮雪忠心耿耿,天保护、地保佑暗有神灵。}

\setlength{\hangindent}{52pt}{王佐\hspace{30pt}《后汉》?}

\setlength{\hangindent}{52pt}{王佐\hspace{30pt}不好呃!}

\setlength{\hangindent}{52pt}{王佐\hspace{30pt}《(东周)列国(志)》。}

\setlength{\hangindent}{52pt}{王佐\hspace{30pt}还是看看《列国》罢。}

\setlength{\hangindent}{52pt}{王佐\hspace{30pt}``要离断臂刺庆忌'',``要离断臂刺庆忌''$\cdots{}\cdots{}$}

\setlength{\hangindent}{52pt}{王佐\hspace{30pt}且住,想那要离断臂,刺死公子庆忌,(此)乃大丈夫所为,俺王佐何不学他一学?}

\setlength{\hangindent}{52pt}{王佐\hspace{30pt}【{\akai 二黄散板}】那要离呀断臂行果有志量({\akai 或}:~颇有志量){\footnotesize 呐},留下了美名儿万载传扬。我王佐学断臂番营{\footnotesize 呐}去闯{\footnotesize 啊},顾不得生和死啊天作主张。}

\setlength{\hangindent}{52pt}{旗牌\hspace{30pt}(王将军)醒来!)}

\setlength{\hangindent}{52pt}{王佐\hspace{30pt}【{\akai 二黄散板}】霎时间痛得我神魂不定,好一似滚油煎乱箭攒心。睁开了昏花眼难以扎挣,为国家斩断臂要留美名。}

\setlength{\hangindent}{52pt}{旗牌\hspace{30pt}将军为何如此?}

\setlength{\hangindent}{52pt}{王佐\hspace{30pt}尔等不可声张。来来来,这有书信一封,送往大帐({\akai 或}:~送到大帐)岳元帅。就说我,呃,另有公干去了哇。}

\setlength{\hangindent}{52pt}{旗牌\hspace{30pt}遵命。}

\setlength{\hangindent}{52pt}{王佐\hspace{30pt}转来。}

\setlength{\hangindent}{52pt}{旗牌\hspace{30pt}在。}

\setlength{\hangindent}{52pt}{王佐\hspace{30pt}千万不可走漏风声。}

\setlength{\hangindent}{52pt}{旗牌\hspace{30pt}遵命。}

\setlength{\hangindent}{52pt}{王佐\hspace{30pt}且住,趁此天色朦胧,我不免诈降番营去者。}

\setlength{\hangindent}{52pt}{王佐\hspace{30pt}呼------呜$\cdots{}\cdots{}$}

\vspace{3pt}{\centerline{{[}{\hei 第三场}{]}}}\vspace{5pt}

\setlength{\hangindent}{52pt}{旗牌\hspace{30pt}有请元帅。}

\setlength{\hangindent}{52pt}{岳飞\hspace{30pt}({\akai 念})闷坐大营无良计,愁思昼夜费心机。}

\setlength{\hangindent}{52pt}{岳飞\hspace{30pt}何事。}

\setlength{\hangindent}{52pt}{旗牌\hspace{30pt}王将军书信呈上。}

\setlength{\hangindent}{52pt}{岳飞\hspace{30pt}待我看来。}

\setlength{\hangindent}{52pt}{岳飞\hspace{30pt}呜哙呀,原来王贤弟诈降番营去了。}

\setlength{\hangindent}{52pt}{岳飞\hspace{30pt}来,王贵进帐。}

\setlength{\hangindent}{52pt}{王贵\hspace{30pt}参见元帅。}

\setlength{\hangindent}{52pt}{岳飞\hspace{30pt}命你巡营瞭哨,待等王佐将军消息。需要小心!}

\setlength{\hangindent}{52pt}{王贵\hspace{30pt}得令!}

\vspace{3pt}{\centerline{{[}{\hei 第四场}{]}}}\vspace{5pt}

\setlength{\hangindent}{52pt}{金兀朮\hspace{20pt}({\akai 念})兴兵攻宋室,}

\setlength{\hangindent}{52pt}{陆文龙\hspace{20pt}({\akai 念})一战建奇功。}

\setlength{\hangindent}{52pt}{金兵\hspace{30pt}启狼主,拿住奸细一名。}

\setlength{\hangindent}{52pt}{金兀朮\hspace{20pt}押进帐来。}

\setlength{\hangindent}{52pt}{王佐\hspace{30pt}叩见狼主。}

\setlength{\hangindent}{52pt}{金兀朮\hspace{20pt}唗!大胆奸细,竟敢前来窥探。来------推出斩了!}

\setlength{\hangindent}{52pt}{王佐\hspace{30pt}(啊)慢来慢来,留头讲话呀。}

\setlength{\hangindent}{52pt}{陆文龙\hspace{20pt}啊,是啊,父王要留头讲话。}

\setlength{\hangindent}{52pt}{金兀朮\hspace{20pt}你且讲来。}

\setlength{\hangindent}{52pt}{王佐\hspace{30pt}是。难臣王佐,乃岳飞帐下一名随营参军。见他屡次杀得大败,是我劝他归降({\akai 或}:~归顺);~(不想)他是执意不肯。当时拔剑,断臣左臂,言道:~誓要扫灭金邦,迎 请二圣还朝,然后再将难臣斩首。哎呀狼主啊!如今,我死又死不了,活是活受罪呀!唉,狼主救命呐,呃$\cdots{}\cdots{}$({\hwfs 哭介})}

\setlength{\hangindent}{52pt}{金兀朮\hspace{20pt}孤家不信,一派谎言。}

\setlength{\hangindent}{52pt}{王佐\hspace{30pt}现有断臂(在此)为证。}

\setlength{\hangindent}{52pt}{金兀朮\hspace{20pt}我却不信。}

\setlength{\hangindent}{52pt}{王佐\hspace{30pt}狼主请看------}

\setlength{\hangindent}{52pt}{金兀朮\hspace{20pt}呜哙呀,岳飞呀岳飞,降与不降,但凭于你。为何下此毒手?!}

\setlength{\hangindent}{52pt}{金兀朮\hspace{20pt}罢了,你起来,孤家收留于你也就是了。}

\setlength{\hangindent}{52pt}{王佐\hspace{30pt}谢狼主!}

\setlength{\hangindent}{52pt}{金兀朮\hspace{20pt}如今归顺我国,就是我国人了,必须与你改个名字。你叫什么?}

\setlength{\hangindent}{52pt}{陆文龙\hspace{20pt}是呀,要改个名字的才是。他叫什么?}

\setlength{\hangindent}{52pt}{王佐\hspace{30pt}唉,苦------哇!呃$\cdots{}\cdots{}$({\hwfs 哭介})}

\setlength{\hangindent}{52pt}{金兀朮\hspace{20pt}有了有了,你为孤家吃了苦了。就叫作``苦人儿''罢!}

\setlength{\hangindent}{52pt}{陆文龙\hspace{20pt}``苦人儿''------呵,甚好。}

\setlength{\hangindent}{52pt}{王佐\hspace{30pt}是。}

\setlength{\hangindent}{52pt}{金兀朮\hspace{20pt}我命太医与你调治伤痕。满营之中,任你闲游。出帐调治去罢。}

\setlength{\hangindent}{52pt}{王佐\hspace{30pt}是,谢狼主。}

\setlength{\hangindent}{52pt}{王佐\hspace{30pt}呼------呜$\cdots{}\cdots{}$}

\setlength{\hangindent}{52pt}{金兀朮\hspace{20pt}啊,儿啊,为父已命人去搬取铁浮图,攻打宋营。正是:~}

\setlength{\hangindent}{52pt}{金兀朮\hspace{20pt}({\akai 念})恼恨岳飞太不仁,}

\setlength{\hangindent}{52pt}{陆文龙\hspace{20pt}({\akai 念})军中哪有断臂刑!}

\vspace{3pt}{\centerline{{[}{\hei 第五场}{]}}}\vspace{5pt}

\setlength{\hangindent}{52pt}{(乳娘\hspace{30pt}【{\akai 二黄摇板}】何日里才能得冤冤相报,思想起当年事心似火烧。撇故土到他乡谁为倚靠,屡次里想回国无路可逃。\footnote{这是刘曾复先生示范的罗福山唱法。})}

\setlength{\hangindent}{52pt}{王佐\hspace{30pt}走哇!}

\setlength{\hangindent}{52pt}{王佐\hspace{30pt}【{\akai 二黄摇板}】这几天到番营未有巧机}\footnote{夏行涛{\scriptsize 君}建议作``未有巧计''。}{,怎能够向他人来把话提({\akai 或}:~细说端的;~细说端倪)。}

\setlength{\hangindent}{52pt}{王佐\hspace{30pt}来此已是陆文龙的营盘,待我来偷觑偷觑。}

\setlength{\hangindent}{52pt}{乳娘\hspace{30pt}啊------哪里来的奸细,小番,与我拿下了。}

\setlength{\hangindent}{52pt}{王佐\hspace{30pt}啊老太太,莫要高声呐。(我不是奸细呀,)我就是狼主新收下的一个残废人,取名``苦人儿'',就是我哇。}

\setlength{\hangindent}{52pt}{乳娘\hspace{30pt}啊,不错不错。殿下言道,有一南朝将官,名唤王佐,投顺我邦,改名``苦人儿'',呃,就是足下么?}

\setlength{\hangindent}{52pt}{王佐\hspace{30pt}正是!}

\setlength{\hangindent}{52pt}{乳娘\hspace{30pt}哦,我们是幸会呀。}

\setlength{\hangindent}{52pt}{王佐\hspace{30pt}幸会呀。}

\setlength{\hangindent}{52pt}{王佐\hspace{30pt}啊老太太,听你讲话,不像此地({\akai 或}:~此处)人氏啊。}

\setlength{\hangindent}{52pt}{乳娘\hspace{30pt}本不是此地人氏。}

\setlength{\hangindent}{52pt}{王佐\hspace{30pt}哪里人氏?}

\setlength{\hangindent}{52pt}{乳娘\hspace{30pt}老身乃是湖广潭州人氏。}

\setlength{\hangindent}{52pt}{王佐\hspace{30pt}哦,老太太,你是湖广潭州人么?}

\setlength{\hangindent}{52pt}{乳娘\hspace{30pt}正是。}

\setlength{\hangindent}{52pt}{王佐\hspace{30pt}呵呵,这倒巧得紧({\akai 或}:~这倒巧得很)呐,我也是湖广潭州人呐。}

\setlength{\hangindent}{52pt}{乳娘\hspace{30pt}哦,如此说来,我们是同乡?!}

\setlength{\hangindent}{52pt}{王佐\hspace{30pt}是同乡啊。}

\setlength{\hangindent}{52pt}{(乳娘\hspace{30pt}重见一礼。)}

\setlength{\hangindent}{52pt}{王佐\hspace{30pt}好,重见一礼。}

\setlength{\hangindent}{52pt}{乳娘\hspace{30pt}({\akai 念})久旱逢甘雨,}

\setlength{\hangindent}{52pt}{王佐\hspace{30pt}({\akai 念})他乡遇故知。}

\setlength{\hangindent}{52pt}{王佐\hspace{30pt}啊,老太太你缘何至此?}

\setlength{\hangindent}{52pt}{乳娘\hspace{30pt}噤声!我与将军乃是同乡,说也无妨:~老身薛氏,当年在潞安州陆登陆大老爷府中,以为乳娘;~那年金兵打破城池,老爷、夫人尽忠、尽节而死,撇下未满三月的陆公子,被狼主捉回金邦,算来一十六载。唉,陆家的冤仇何日得报哇,啊,啊$\cdots{}\cdots{}$({\hwfs 哭介})}

\setlength{\hangindent}{52pt}{(王佐\hspace{30pt}哦哦哦,是是是$\cdots{}\cdots{}$)}

\setlength{\hangindent}{52pt}{王佐\hspace{30pt}唉!实实地可怜呐!}

\setlength{\hangindent}{52pt}{乳娘\hspace{30pt}唉!实实地可怜呐!}

\setlength{\hangindent}{52pt}{王佐\hspace{30pt}啊老太太,(但不知)那陆公还有后么?}

\setlength{\hangindent}{52pt}{乳娘\hspace{30pt}怎说无有呃,昨日在两军阵前,连挑宋将数员上将\footnote{此处念是``连挑宋营数员上将'',似亦可。},那不就是陆公子么。}

\setlength{\hangindent}{52pt}{王佐\hspace{30pt}哦?那就是陆公子么?}

\setlength{\hangindent}{52pt}{乳娘\hspace{30pt}嗯,正是。}

\setlength{\hangindent}{52pt}{王佐\hspace{30pt}呵呵,我王佐今日来的好机会也!}

\setlength{\hangindent}{52pt}{王佐\hspace{30pt}【{\akai 二黄摇板}】听罢言来喜心上,尊声安人听端详:~我断臂原本为小殿下呀,舍死忘生到番邦。}

\setlength{\hangindent}{52pt}{乳娘\hspace{30pt}如此说来,呃,你为我家公子吃了苦了哇!}

\setlength{\hangindent}{52pt}{王佐\hspace{30pt}呜------不妨啊。}

\setlength{\hangindent}{52pt}{王佐\hspace{30pt}【{\akai 二黄摇板}】这断臂的情由休声嚷啊,泄漏机关祸难当。待等殿下回营帐,全仗安人作主张。}

\setlength{\hangindent}{52pt}{(乳娘\hspace{30pt}公子已回,快快躲避。\footnote{刘曾复先生为陈超老师说戏时没有此句,此处根据《京剧丛刊》第十三集~《八大锤》增补,使文意通顺。})}

\setlength{\hangindent}{52pt}{王佐\hspace{30pt}哦,来了!}

\setlength{\hangindent}{52pt}{王佐\hspace{30pt}来了。}

\setlength{\hangindent}{52pt}{(陆文龙{\hwfs 上})}

\setlength{\hangindent}{52pt}{王佐\hspace{30pt}``苦人儿''叩见殿下。}

\setlength{\hangindent}{52pt}{陆文龙\hspace{20pt}罢了。``苦人儿'',这几日你往哪里去了?}

\setlength{\hangindent}{52pt}{王佐\hspace{30pt}这几日({\akai 或}:~这些天)被那些平章、将官们,这个请我吃酒,那个叫我({\akai 或}:~那个请我)说评书,故而未能前来,与殿下请安({\akai 或}:~与千岁请安)呐。}

\setlength{\hangindent}{52pt}{陆文龙\hspace{20pt}哦,你还会说评书么?}

\setlength{\hangindent}{52pt}{王佐\hspace{30pt}呃,我是一肚子的(评)书啊。}

\setlength{\hangindent}{52pt}{陆文龙\hspace{20pt}你且稍待。乳娘有请!}

\setlength{\hangindent}{52pt}{乳娘\hspace{30pt}殿下何事?}

\setlength{\hangindent}{52pt}{陆文龙\hspace{20pt}啊乳娘,有个``苦人儿'',他会说评书。请至出来,一同听书。}

\setlength{\hangindent}{52pt}{乳娘\hspace{30pt}好好好。}

\setlength{\hangindent}{52pt}{陆文龙\hspace{20pt}啊``苦人儿'',这就是我家乳娘,上前见过。}

\setlength{\hangindent}{52pt}{王佐\hspace{30pt}哦,这就是乳娘老太太?}

\setlength{\hangindent}{52pt}{王佐\hspace{30pt}啊老太太,你好哇。}

\setlength{\hangindent}{52pt}{乳娘\hspace{30pt}``苦人儿''你好哇!}

\setlength{\hangindent}{52pt}{陆文龙\hspace{20pt}呃,你快快说来。}

\setlength{\hangindent}{52pt}{乳娘\hspace{30pt}啊,殿下,此时要有一个座位。}

\setlength{\hangindent}{52pt}{陆文龙\hspace{20pt}你坐下说吧。}

\setlength{\hangindent}{52pt}{王佐\hspace{30pt}呃,慢来慢来,殿下在此,哪有``苦人儿''的座位呀?}

\setlength{\hangindent}{52pt}{陆文龙\hspace{20pt}咱们师兄弟,您甭客气。}

\setlength{\hangindent}{52pt}{乳娘\hspace{30pt}我们是自己人,不要客气。坐下罢。}

\setlength{\hangindent}{52pt}{王佐\hspace{30pt}谢座。}

\setlength{\hangindent}{52pt}{王佐\hspace{30pt}啊殿下,你是爱听文的呀,还是爱听武的呢?}

\setlength{\hangindent}{52pt}{陆文龙\hspace{20pt}小王习武,自然是爱听武的。}

\setlength{\hangindent}{52pt}{王佐\hspace{30pt}哦,武的。}

\setlength{\hangindent}{52pt}{乳娘\hspace{30pt}自然武的好哇({\akai 或}:~我是爱听武的)。}

\setlength{\hangindent}{52pt}{王佐\hspace{30pt}是忠的,还是奸的呢?}

\setlength{\hangindent}{52pt}{陆文龙\hspace{20pt}小王喜的是忠臣,恨的是奸佞。}

\setlength{\hangindent}{52pt}{王佐\hspace{30pt}哦------爱听忠的。呃,待我来说一段``骅骝思乡''罢。}

\setlength{\hangindent}{52pt}{陆文龙\hspace{20pt}哦,``骅骝思乡''?嗯,这个故事,呃,倒是要听上一听。}

\setlength{\hangindent}{52pt}{乳娘\hspace{30pt}是啊,``苦人儿''你且讲来。}

\setlength{\hangindent}{52pt}{(王佐{\hwfs 拍醒木})}

\setlength{\hangindent}{52pt}{陆文龙\hspace{20pt}啊,这做什么?}

\setlength{\hangindent}{52pt}{王佐\hspace{30pt}这是我们说评书的}规矩呀。

\setlength{\hangindent}{52pt}{陆文龙\hspace{20pt}哦,说书的有规矩?那唱戏的就更有规矩了。}

\setlength{\hangindent}{52pt}{王佐\hspace{30pt}是啊,({\akai 或}:~呃,)无有规矩,就不成方圆了。}

\setlength{\hangindent}{52pt}{乳娘\hspace{30pt}是啊,殿下,无有规矩,就不成方圆了。}

\setlength{\hangindent}{52pt}{陆文龙\hspace{20pt}哦,这是他们的规矩?}

\setlength{\hangindent}{52pt}{乳娘\hspace{30pt}是啊。}

\setlength{\hangindent}{52pt}{陆文龙\hspace{20pt}哎,就依你的``规矩''。}

\setlength{\hangindent}{52pt}{王佐\hspace{30pt}({\akai 念})道德三皇五帝,功名夏后商周;~英雄五霸闹春秋,顷刻兴亡过手。}

\setlength{\hangindent}{52pt}{王佐\hspace{30pt}({\akai 念})青史几行名姓,北邙无数荒丘;~前人田地后人收,说甚龙争虎斗。}

(王佐{\hwfs 拍醒木})

\setlength{\hangindent}{52pt}{陆文龙\hspace{20pt}哎,他又来了。}

\setlength{\hangindent}{52pt}{王佐\hspace{30pt}残词道罢({\akai 或}:~残词念罢),书归正传。花开两朵,各表一枝。话表:~大宋朝真宗天子在位,朝中有一家大大的忠良,名唤杨延昭。}

(王佐{\hwfs 拍醒木})

\setlength{\hangindent}{52pt}{陆文龙\hspace{20pt}嗯------杨延昭是个忠良?}

\setlength{\hangindent}{52pt}{乳娘\hspace{30pt}是啊。({\akai 或}:~忠良事啊。)}

\setlength{\hangindent}{55pt}{王佐\hspace{30pt}只因北国屡次交战,被那杨元帅杀得大败。那北国萧后就勾通了南朝一个大大的奸佞({\akai 或}:~一家大大的奸佞),名叫王钦若。}

\setlength{\hangindent}{52pt}{陆文龙\hspace{20pt}王钦若是个奸佞么?}

\setlength{\hangindent}{52pt}{王佐\hspace{30pt}嗯,大大的奸佞{\footnotesize 呐}。他与杨家旧有仇恨。一日,真宗早朝,那王钦若出班奏道,说道,北国出了一骑好马,日行千里,夜走八百,名为``日月骕骦马''。}

\setlength{\hangindent}{52pt}{陆文龙\hspace{20pt}嗯,是一骑好马!}

\setlength{\hangindent}{52pt}{乳娘\hspace{30pt}嗯,是一骑好马啊。}

\setlength{\hangindent}{52pt}{王佐\hspace{30pt}那圣上闻奏,就动了爱马之意({\akai 或}:~爱马之心)。问道:~命何人前去盗马呀?那王钦若又奏道:~非杨延昭不可哇。那圣上({\akai 或}:~宋王)就命杨延昭(前)去盗马。那杨元帅奉旨回来,是闷闷而不乐哇。他帐下有一员虎将,此人姓孟名良字佩仓------}

(王佐{\hwfs 拍醒木})

\setlength{\hangindent}{52pt}{陆文龙\hspace{20pt}啊?}

\setlength{\hangindent}{52pt}{王佐\hspace{30pt}(呃,这)三关的孟良是哪个(儿)不晓哇?!}

\setlength{\hangindent}{52pt}{乳娘\hspace{30pt}是啊,三关孟良哪个不晓。}

\setlength{\hangindent}{52pt}{陆文龙\hspace{20pt}哦,哦------嗯。}

\setlength{\hangindent}{52pt}{王佐\hspace{30pt}进帐问起情由,当时就讨下了一枝将令。呃,呃,不到一日二啊,二日三,他就混进番营去了。}

\setlength{\hangindent}{52pt}{陆文龙\hspace{20pt}他是怎样混进去的?}

\setlength{\hangindent}{52pt}{王佐\hspace{30pt}呃、呃,(呃$\cdots{}\cdots{}$)他会说({\akai 或}:~他能通)三川六国的番语呀。}

\setlength{\hangindent}{52pt}{陆文龙\hspace{20pt}哦------原来如此。嗯,后来呢?}

\setlength{\hangindent}{52pt}{王佐\hspace{30pt}嗯,(呃,呃,)不到一月呀,竟将此马盗回来了。}

\setlength{\hangindent}{52pt}{陆文龙\hspace{20pt}哦------盗回来了?!嗯,此人有能耐。}

\setlength{\hangindent}{52pt}{王佐\hspace{30pt}不错,有能耐。可惜呀!}

\setlength{\hangindent}{52pt}{陆文龙\hspace{20pt}可惜什么?}

\setlength{\hangindent}{52pt}{王佐\hspace{30pt}可惜那马,七日七夜,不食草料,眼望北番,大叫(了)三声,呵嘿,就饿死了。}

\setlength{\hangindent}{52pt}{陆文龙\hspace{20pt}哦------?这是何意呀?}

\setlength{\hangindent}{52pt}{王佐\hspace{30pt}呃,不过是思乡罢!}

\setlength{\hangindent}{52pt}{陆文龙\hspace{20pt}哦,这畜类还会思乡么?}

\setlength{\hangindent}{52pt}{乳娘\hspace{30pt}啊殿下,畜类会思乡,何况人乎?}

\setlength{\hangindent}{52pt}{王佐\hspace{30pt}啊老太太,如今的人呐,还不如个畜类呢!}

\setlength{\hangindent}{52pt}{王佐\hspace{30pt}【{\akai 二黄摇板}】那马倒有思乡意,如今的人儿不如它。父母冤仇全不管({\akai 或}:~父母冤仇抛撇下),反把仇人当自家。}

\setlength{\hangindent}{52pt}{陆文龙\hspace{20pt}好------}

(王佐{\hwfs 拍醒木})

\setlength{\hangindent}{52pt}{王佐\hspace{30pt}完了。}

\setlength{\hangindent}{52pt}{陆文龙\hspace{20pt}啊?完了么?}

\setlength{\hangindent}{52pt}{王佐\hspace{30pt}完了。}

\setlength{\hangindent}{52pt}{陆文龙\hspace{20pt}哎呀,不热闹哇$\cdots{}\cdots{}$}

\setlength{\hangindent}{52pt}{王佐\hspace{30pt}(呵,《八大锤》带``断臂'',还不热闹?)不热闹$\cdots{}\cdots{}$唉,待我来说一段本朝四狼主,当年大破潞安州的故事罢。}

\setlength{\hangindent}{52pt}{陆文龙\hspace{20pt}哦?可是我父王?}

\setlength{\hangindent}{52pt}{王佐\hspace{30pt}正是。}

\setlength{\hangindent}{52pt}{陆文龙\hspace{20pt}嗯。我倒要听上一听。呵哈,可热闹?}

\setlength{\hangindent}{52pt}{王佐\hspace{30pt}(哼,)热闹得很呐!}

\setlength{\hangindent}{52pt}{陆文龙\hspace{20pt}哦,你快些讲来。}

\setlength{\hangindent}{52pt}{王佐\hspace{30pt}呃------慢来,慢来。我这里有画图一幅,我们挂将起来({\akai 或}:~悬挂起来),照图言讲啊。}

\setlength{\hangindent}{52pt}{陆文龙\hspace{20pt}挂了起来。}

\setlength{\hangindent}{52pt}{陆文龙\hspace{20pt}啊------``苦人儿'',这画图之上,有许多的兵将,是金兵还是宋将?}

\setlength{\hangindent}{52pt}{王佐\hspace{30pt}金兵也有,宋将也有哇。}

\setlength{\hangindent}{52pt}{陆文龙\hspace{20pt}啊,``苦人儿'',有一员大将,手执宝剑,自刎而死,他是何人?}

\setlength{\hangindent}{52pt}{王佐\hspace{30pt}(呃,)这就是潞安州节度使,陆登陆老先生。只因我国狼主打破城池,他万般无奈,拔剑自刎尽忠了。}

\setlength{\hangindent}{52pt}{陆文龙\hspace{20pt}哦------尽忠了。啊``苦人儿'',那旁有一妇人,悬梁自尽,她是何人?}

\setlength{\hangindent}{52pt}{王佐\hspace{30pt}这就是陆老夫人。见她丈夫尽忠,她就悬梁自缢,尽节了哇。}

\setlength{\hangindent}{52pt}{陆文龙\hspace{20pt}哦------尽节了。哦,``苦人儿'',有一员大将,跪在尘埃,好像我父王模样,他是何人?}

\setlength{\hangindent}{52pt}{王佐\hspace{30pt}(呃,)正是我国四狼主。}

\setlength{\hangindent}{52pt}{陆文龙\hspace{20pt}哦,是我父王。他为何与他下拜呀?}

\setlength{\hangindent}{52pt}{王佐\hspace{30pt}我国四狼主念他是个忠良,故(而)与他下拜呀。}

\setlength{\hangindent}{52pt}{陆文龙\hspace{20pt}哦------我父王拜得,呃,小王也可拜得么?}

\setlength{\hangindent}{52pt}{王佐\hspace{30pt}(哦,)殿下(要下拜)么?}

\setlength{\hangindent}{52pt}{陆文龙\hspace{20pt}正是。}

\setlength{\hangindent}{52pt}{王佐\hspace{30pt}呃呃,正拜,正拜!}

\setlength{\hangindent}{52pt}{陆文龙\hspace{20pt}啊------陆老先生在上,受小王大礼参拜!}

\setlength{\hangindent}{52pt}{王佐\hspace{30pt}(啊,)陆老先生,我家殿下拜你,你要明白呀。}

\setlength{\hangindent}{52pt}{陆文龙\hspace{20pt}啊``苦人儿'',有一妇人,怀抱婴儿,躲在一旁,她是何人?}

\setlength{\hangindent}{52pt}{王佐\hspace{30pt}这就是陆$\cdots{}\cdots{}$呃,呃,这就是陆府的乳娘。见她主人一个尽忠,一个尽节,死得可怜,她在一旁落泪呀。}

\setlength{\hangindent}{52pt}{乳娘\hspace{30pt}呃------呜呜呜$\cdots{}\cdots{}$({\hwfs 哭介})}

\setlength{\hangindent}{52pt}{陆文龙\hspace{20pt}啊乳娘,你为何啼哭啊?}

\setlength{\hangindent}{52pt}{乳娘\hspace{30pt}唉!我见他一家死得可怜,故而------啼哭啊,呜呜呜$\cdots{}\cdots{}$({\hwfs 哭介})}

\setlength{\hangindent}{52pt}{陆文龙\hspace{20pt}啊``苦人儿'',我把乳娘好有一比呀。}

\setlength{\hangindent}{52pt}{王佐\hspace{30pt}比作何来?}

\setlength{\hangindent}{52pt}{陆文龙\hspace{20pt}看兵书落泪------}

\setlength{\hangindent}{52pt}{王佐\hspace{30pt}此话怎讲?}

\setlength{\hangindent}{52pt}{陆文龙\hspace{20pt}替古人担忧哇。}

\setlength{\hangindent}{52pt}{王佐\hspace{30pt}(啊------)是啊,老太太你可替古人担的什么忧哇?}

\setlength{\hangindent}{52pt}{乳娘\hspace{30pt}唉!是啊,我替古人担的什么忧。}

\setlength{\hangindent}{52pt}{陆文龙\hspace{20pt}啊``苦人儿'',他为何立尸不倒?}

\setlength{\hangindent}{52pt}{王佐\hspace{30pt}若问(他)立尸不倒么?}

\setlength{\hangindent}{52pt}{陆文龙\hspace{20pt}是啊。}

\setlength{\hangindent}{52pt}{王佐\hspace{30pt}唉!恐怕他后人,不与他父母报仇,故而立尸不倒。}

\setlength{\hangindent}{52pt}{陆文龙\hspace{20pt}哦------他还有后人么?}

\setlength{\hangindent}{52pt}{王佐\hspace{30pt}有道是``忠良不绝后''呃。}

\setlength{\hangindent}{52pt}{乳娘\hspace{30pt}是啊,忠良不绝后哇。}

\setlength{\hangindent}{52pt}{陆文龙\hspace{20pt}嗯------此子何在?}

\setlength{\hangindent}{52pt}{王佐\hspace{30pt}被我国四狼主带回来了。}

\setlength{\hangindent}{52pt}{陆文龙\hspace{20pt}哦?带回来了------他,他,他今年有多大年纪?}

\setlength{\hangindent}{52pt}{王佐\hspace{30pt}(呃,他)今年么$\cdots{}\cdots{}$呃,哦,哦$\cdots{}\cdots{}$(他今年)一十六岁了哇。}

\setlength{\hangindent}{52pt}{陆文龙\hspace{20pt}哦,一十六岁?与小王同庚呐。}

\setlength{\hangindent}{52pt}{王佐\hspace{30pt}哦,殿下也是一十六岁么?}

\setlength{\hangindent}{52pt}{陆文龙\hspace{20pt}正是。}

\setlength{\hangindent}{52pt}{王佐\hspace{30pt}呃({\akai 或}:~呵呵),这倒巧得很呐。}

\setlength{\hangindent}{52pt}{陆文龙\hspace{20pt}嗯------此子本领如何?}

\setlength{\hangindent}{52pt}{王佐\hspace{30pt}若论他的本领么,嗯------两军阵前,他,他,他能力敌万人!}

\setlength{\hangindent}{52pt}{陆文龙\hspace{20pt}哦------他、他、他------能力敌万人?}

\setlength{\hangindent}{52pt}{王佐\hspace{30pt}嗯------}

\setlength{\hangindent}{52pt}{陆文龙\hspace{20pt}呵,呵,哼------({\hwfs 冷笑介})他既然力敌万人,为何不与他父母报仇?}

\setlength{\hangindent}{52pt}{王佐\hspace{30pt}唉!不提``报仇''还则罢了,提起``报仇'',令人好恨呐!}

\setlength{\hangindent}{52pt}{陆文龙\hspace{20pt}恨者何来?}

\setlength{\hangindent}{52pt}{王佐\hspace{30pt}他非但不替他父母报仇,如今反认仇人为父!}

\setlength{\hangindent}{52pt}{陆文龙\hspace{20pt}哦------}

\setlength{\hangindent}{52pt}{陆文龙\hspace{20pt}嗯------他叫什么名字?}

\setlength{\hangindent}{52pt}{王佐\hspace{30pt}(他叫陆$\cdots{}\cdots{}$)他叫------陆文龙。({\hwfs 低声})}

\setlength{\hangindent}{52pt}{陆文龙\hspace{20pt}他叫什么名字?}

\setlength{\hangindent}{52pt}{王佐\hspace{30pt}(他)叫陆文龙。({\hwfs 含糊})}

\setlength{\hangindent}{52pt}{陆文龙\hspace{20pt}到底叫什么?}

\setlength{\hangindent}{52pt}{王佐\hspace{30pt}诶------他、他$\cdots{}\cdots{}$他叫陆文龙啊!}

\setlength{\hangindent}{52pt}{陆文龙\hspace{20pt}唗!~胆大``苦人儿'',戏耍小王,休走,看剑!}

\setlength{\hangindent}{52pt}{乳娘\hspace{30pt}唉------殿下,这就是你全家遭害的故事。}

\setlength{\hangindent}{52pt}{陆文龙\hspace{20pt}怎么讲?!}

\setlength{\hangindent}{52pt}{乳娘\hspace{30pt}遭害的故事呀!}

\setlength{\hangindent}{52pt}{陆文龙\hspace{20pt}\textless{}\!{\bfseries\akai 双叫头}\!\textgreater{}爹爹,母亲!~哎呀------}

\setlength{\hangindent}{52pt}{王佐\hspace{30pt}(公子)醒来!}

\setlength{\hangindent}{52pt}{陆文龙\hspace{20pt}【{\akai 二黄散板}】听一言来珠泪掉,}

\setlength{\hangindent}{52pt}{王佐\hspace{30pt}(公子)醒来!}

\setlength{\hangindent}{52pt}{陆文龙\hspace{20pt}\textless{}{\!\bfseries\akai 三叫头}\!\textgreater{}爹爹,母亲!唉------爹娘啊$\cdots{}\cdots{}$({\hwfs 哭介})}

\setlength{\hangindent}{52pt}{陆文龙\hspace{20pt}【{\akai 二黄散板}】不由小王怒全消。三尺龙泉出了鞘,}

\setlength{\hangindent}{52pt}{王佐\hspace{30pt}哪里去?}

\setlength{\hangindent}{52pt}{陆文龙\hspace{20pt}【{\akai 二黄散板}】斩尽金兵回宋朝。}

\setlength{\hangindent}{52pt}{王佐\hspace{30pt}\textless{}\!{\bfseries\akai 叫头}\!\textgreater{}公子!}

\setlength{\hangindent}{52pt}{王佐\hspace{30pt}【{\akai 二黄散板}】公子休要({\akai 或}:~公子不必)珠泪掉,快想良谋回南朝。}

\setlength{\hangindent}{52pt}{陆文龙\hspace{20pt}\textless{}\!{\bfseries\akai 叫头}\!\textgreater{}哎呀叔父啊!那贼见岳元帅闭门不出,明日欲用取铁浮图攻打宋营,如何是好?}

\setlength{\hangindent}{52pt}{王佐\hspace{30pt}(哎呀!)------有了,待我修下书信一封,公子用箭,射入宋营,教那岳元帅也好作一准备。}

\setlength{\hangindent}{52pt}{陆文龙\hspace{20pt}待我溶墨。}\footnote{此处王佐也可同念{``}与我溶墨{''}。}

\setlength{\hangindent}{52pt}{王佐\hspace{30pt}待我修书。}

\setlength{\hangindent}{52pt}{王佐\hspace{30pt}公子!}

\setlength{\hangindent}{52pt}{陆文龙\hspace{20pt}恩公、乳娘,受我一拜!}

\setlength{\hangindent}{52pt}{陆文龙\hspace{20pt}\textless{}{\!\bfseries\akai 三叫头}\!\textgreater{}爹爹,母亲!唉------我那$\cdots{}\cdots{}$({\hwfs 哭介})}

\setlength{\hangindent}{52pt}{王佐\hspace{30pt}噤声!}

\setlength{\hangindent}{52pt}{(陆文龙{\hwfs 下})}\hspace{30pt}

\setlength{\hangindent}{52pt}{乳娘\hspace{30pt}这一下他就明白了。}

\setlength{\hangindent}{52pt}{王佐\hspace{30pt}他明白了,我也残废了。}

}
