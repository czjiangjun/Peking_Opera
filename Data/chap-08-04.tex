\newpage
\subsubsection{\hei\large 逍遥津~{\small 之}~汉献帝$^{\ast}$}
\addcontentsline{toc}{subsection}{\hei 逍遥津~{\small 之}~汉献帝}

\hangafter=1                   %2. 设置从第1⾏之后开始悬挂缩进  %
\setlength{\parindent}{0pt}{

【{\akai 二黄导板}】苦汉帝在后宫伤心难忍,

【{\akai 回龙}】父子们悲切切好不伤情,贤御妻呀。

\setlength{\hangindent}{56pt}{   %3. 设置悬挂缩进量                %
【{\akai 二黄原板}】叹伏后此时间必定丧命,我君妃生离散惨不忍闻。二皇儿年幼小孩童天性,哭啼啼与孤王要他的娘亲。想奸贼不由孤咬牙愤恨,上欺寡人下压群臣。欺寡人贼带剑上殿孤见他不敢责问,欺寡人贼独霸朝纲、目无君王、自专自尊。欺寡人孤只得百般谨慎,欺寡人孤只得时刻留心。欺寡人贼奏本是非曲直孤不敢争论,欺寡人孤有命贼大胆妄为抗旨不遵。欺寡人贼一意孤行孤不敢过问,欺寡人孤怒不敢言、忍耐在心。欺寡人孤见他气色不正吓得孤乱了方寸,欺寡人孤见他带怒发威吓得孤胆战心惊。欺寡人蹂躏百般、万分难忍,欺寡人贼败坏朝纲、逆了五伦。欺寡人好一似【{\footnotesize 转}{\akai 二黄慢板}】奴仆受训,欺寡人好一似虐待家人。欺寡人好一似无辜良民被贼围困,欺寡人好一似冤屈罪犯无处冤申。欺寡人好一似蛇毒蝎狠,欺寡人好一似虎咽狼吞。欺寡人好一似前世冤孽今生报应,欺寡人好一似狭路相逢对头仇人。欺寡人好一似阎君索命,欺寡人好一似饿鬼孤魂。欺寡人好一似败阵残兵无投奔,反被贼困垓心难逃遁难存身,坐以待毙谁来救应,
}

【{\akai 二黄散板}】又听得一片喧哗声震乾坤。}

\vspace{15pt}
{\hei 附注}:~

\setlength{\parindent}{0pt}{
《顺天时报》曾刊《逍遥津任辰辙之唱词》一文,所载词句与刘曾复先生所传词句非常相近,照录供参考(张斯琦{\scriptsize 君}提供)
}

\vspace{10pt}
{\centerline{\textcolor{blue}{\hei《逍遥津》``任辰''辙之唱词}}}
\vspace{10pt}

\setlength{\parindent}{22pt}{     %
	{\hwfs 旧本《逍遥津》,``欺寡人''一段,俱用``由求''辙,戏中汉献帝唱``欺寡人好一比鹰抓兔胁''句,过于俚俗,殊伤大雅。刘鸿升未故时,将``由求''改``任辰'',虽亦不免俗,但较旧本,似觉雅驯。李桂芬唱《逍遥津》,亦用斯词,爰将改词录于左端:~}}

\vspace{15pt}
\setlength{\parindent}{0pt}{
	{\hei 汉献帝在后宫伤心难忍,可叹我父子们悲切切冷清清、求生不得求死不能、好不惨情。叹伏后到此时难保活命,我君妃生离散惨不忍闻。二皇儿年幼小孩童之性,哭啼啼与孤王要他的娘亲。想奸贼不由孤嚼牙愤恨,上欺天子下压群臣。欺寡人(贼)带剑上殿孤见他不敢责问,欺寡人(贼)独霸朝纲、目无君、自耑自尊。欺寡人孤只得百般谨慎,欺寡人孤只得时刻留神。欺寡人(贼)奏本是非曲直孤不敢\textcolor{red}{$\square~\square$},欺寡人孤有命贼大胆妄为抗旨不遵。欺寡人贼自由行孤不敢过问,欺寡人孤怒不敢言、忍耐在心。欺寡人孤见他气色不正嚇得孤乱了方寸,欺寡人孤见他怒发威嚇得吊胆提心。欺寡人蹂躏百般、惨忍万分,欺寡人贼败坏纲常、逆了五伦。欺寡人好一似主仆受训,欺寡人好一似虐待家人。欺寡人好一似无辜良民被贼围困,欺寡人好一似冤屈罪犯\textcolor{red}{$\square$}而受刑。欺寡人好一似蛇毒蝎狠,欺寡人好一似虎狼把孤吞。欺寡人好一似前世冤孽今生报应,欺寡人好一似夹路相逢对头仇人。欺寡人好一似阎王索命,欺寡人好一似饿鬼勾魂。欺寡人好一似败阵惨兵无投奔,反被贼困垓心、难逃命、难生存、认贼斩、恁贼擒,孤做一待毙谁来救应,又听得宫门外喧哗之声。
}}
