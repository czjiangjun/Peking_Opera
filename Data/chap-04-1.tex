\addcontentsline{toc}{section}{\hfill[\hei 五代·两宋]\hfill}

\newpage

\chead{五代·两宋} % 页眉中间位置内容

{困曹府}\footnote{ 根据刘曾复先生为吴小如先生说戏总讲录音底本整理。  陈超老师注: 刘曾复先生学的《困曹府》路子是程长庚弟子、三庆班沈三元的路子,沈是谭鑫培的把兄弟,因此《困曹府》的传承不是谭派。而另外一路是李顺亭的,再往下传就是贯大元。{413}}

{{[}第一场{]}}

{曹彬\hspace{30pt}~

({\akai 念})跳过虎穴龙潭,({\akai 或}: $\cdots{}\cdots{}$虎穴,)}

{赵匡胤\hspace{20pt}~

({\akai 念})好似凤鹤腾空。({\akai 或}: 逃出天罗地网。)}

{曹彬\hspace{30pt}~

仁兄请。}

{赵匡胤\hspace{20pt}~

请。}

{曹彬\hspace{30pt}~

请坐。}

{赵匡胤\hspace{20pt}~

有座。}\footnote{

陈超老师注: 此处台上摆骑马桌。{414}}

赵匡胤\hspace{20pt}~

适才关前多蒙贤弟阖家({\akai 或}: 全家)搭救,愚兄当面谢过。

{曹彬\hspace{30pt}~

岂敢,小弟当报前恩。}

赵匡胤\hspace{20pt}~

{好说。}

{曹彬\hspace{30pt}~

桌案现有酒食,仁兄请来压惊。}

{赵匡胤\hspace{20pt}~

讨扰了。}

{曹彬\hspace{30pt}~

仁兄请。}

{(赵匡胤 请。)}

{曹彬

\setlength{\hangindent}{60pt}{ 【{\akai 二黄摇板}】知恩不报非君子,忘恩负义是小人。({\akai 或}: 酒不醉人人自醉,色不迷人人自迷。)} }

{赵匡胤\hspace{20pt}~

二相公,贤弟。({\akai 或}: 贤弟,二相公。)}

{(赵匡胤 再饮几杯。)}

{赵匡胤\hspace{20pt}~

睡着了。}

{(起初更)}

{赵匡胤

唉!想俺玄郎,今晚被困曹府({\akai 或}: 夜困曹府),好不焦虑人也------}

{赵匡胤

\setlength{\hangindent}{60pt}{ 【{\akai 二黄慢板}】有豪杰在书房心神不爽,二相公心无事睡卧一旁。无奈何推吊窗观看月亮,真乃是中秋节({\akai 或}: 真乃是中秋夜}\footnote{ 段公平君建议作``中秋月''。{415}}{),星明月朗、轮月皎皎分外风光。} }

{赵匡胤

\setlength{\hangindent}{60pt}{ 【{\akai 二黄原板}】我本是宦门后娇生惯养,闯关东、走关西自逞豪强。思爹娘、想妹弟终朝悬望,山又高、水又深阻隔两厢。洒金桥遇苗顺曾把命讲,他算我到后来南面称王。周文王坐江山全凭姜尚,保周朝八百载国祚绵长。汉光武仗云台二十八将,文邓禹、武姚期、马武子张。小秦王收下了瓦岗诸将,有罗成锁五龙图霸称强。俺玄郎逃灾祸东西游荡,孤一身({\akai 或}: 独一身)并无有架海金梁。到如今坐江山全然不想,全然不想,登九五如南柯大梦一场。恨金鸡不报晓天光未亮,谯楼上睡着了打更儿郎。恨不得抛长枪刺落了天边月亮,用金钩钩出了红日轮光。} }

{(起二更,张氏上)}

{张氏\hspace{30pt}~

\setlength{\hangindent}{60pt}{ 【{\akai 二黄摇板}】轻移莲步出房门,窗外且听他人云。} }

{张氏\hspace{30pt}~

奴家张氏,适才关前搭救恩人,不知他有何言语,待奴细听一番。}

{赵匡胤

且住,适才关前,多蒙张氏嫂嫂,叫了我一声``丈夫'',真乃难得呀难得!({\akai 或}: 多蒙张氏嫂嫂搭救,思想起来,真是难得呀难得!

)}

{赵匡胤

哎------(呀)!说什么难得------倘若(是)我那贺氏妻子,叫(道旁)人一声``丈夫'',我就是这一刀------}

{张氏\hspace{30pt}~

唉呀!}

{赵匡胤\hspace{20pt}~

唉呀,醉了!(呃,)醉了!}

{赵匡胤

\setlength{\hangindent}{60pt}{ 【{\akai 二黄散板}】窗里窗外隔窗棂,窗里说话窗外听。窗里之人吃酒醉,} }

{赵匡胤\hspace{20pt}~

醉了哇!醉了!(呜呜呜$\cdots{}\cdots{}$(吐介))}

{赵匡胤\hspace{20pt}~

\setlength{\hangindent}{60pt}{ 【{\akai 二黄散板}】窗外休听醉汉云。} }

{赵匡胤\hspace{20pt}~

呜呜呜$\cdots{}\cdots{}$(吐介)}

{张氏

\setlength{\hangindent}{60pt}{ 【{\akai 二黄散板}】听一言来吃一惊,羞得奴家脸带红。腰间解下丝鸾带,不如一命丧残生。} }

{(起三更,华佗上}\footnote{ 陈超老师注: 华佗抱宝剑上。{416}}{)}

{华佗\hspace{30pt}~

\setlength{\hangindent}{60pt}{ 【{\akai 二黄摇板}】灵霄领了玉帝命,曹府搭救赤须龙。} }

{华佗\hspace{30pt}~

({\akai 念})

闷坐松林下,修道数百年。三国我为首,自称华佗仙。}

{华佗

今有赤须龙有难,奉了玉帝敕旨,下凡搭救。来此已是曹府,不免进府寻找。}

{华佗\hspace{30pt}~

原来星君在此,待我用起功来。}

{华佗

({\akai 念})不用急来不用愁,真龙天子百灵佑。宝剑一挥龙瘤落,丢入长江顺水流。}

{华佗\hspace{30pt}~

且喜大功成就,我不免趁此机会,讨一封号。}

{华佗\hspace{30pt}~

参见圣上。}

{赵匡胤\hspace{20pt}~

(呃------)何处妖道,在此摆来摆去?}

{华佗\hspace{30pt}~

小道三国华佗。}

{赵匡胤\hspace{20pt}~

前来则甚?}

{华佗\hspace{30pt}~

前来讨封。}

{赵匡胤\hspace{20pt}~

修炼多少年了?}

{华佗\hspace{30pt}~

千年有余。}

{赵匡胤\hspace{20pt}~

嗯,可算得一洞老神仙了。}

{华佗

谢主隆恩。(kai 正是}: ({\akai 念})不是天子隆恩}\footnote{ 段公平君建议作``天赐隆恩''。{417}}{重,焉得一洞老神仙。}

{(起四更,张氏魂上)}

{张氏\hspace{30pt}~

({\akai 念})人死如灯灭,犹如汤浇雪。若得回阳转,水底捞明月。}

{张氏

奴家张氏阴魂是也,是奴一时不明,悬梁自尽,死后方知恩人乃是当今真龙天子,不免趁此机会,前去讨一封号。}

{张氏\hspace{30pt}~

参见圣上。}

{赵匡胤\hspace{20pt}~

啊------({\akai 或}: 呃------)何处冤鬼,在此摆来摆去?}

{张氏\hspace{30pt}~

冤鬼张氏。}

{赵匡胤\hspace{20pt}~

前来则甚?}

{张氏\hspace{30pt}~

前来讨封。}

{赵匡胤

呃,前者({\akai 或}: 前番)路过华山,少一圣母({\akai 或}: 缺一圣母),封你以为华山圣母之位({\akai 或}: 封你为插花圣母}\footnote{ 段公平君注: 汉调二黄《水西门》剧本(``割瘤讨封''是其中{[}第十场{]})作``插花圣母'',豫剧似乎也是这个名字。插花圣母,不供香火,插花为供。今浙江云和龙门村瓯江边有插花殿,据传有``敕封护国插花圣母娘娘''匾。{418}}{)。}

{赵匡胤\hspace{20pt}~

金童、玉女何在?}

{金童、玉女}\footnote{ 陈超老师注: 金童、玉女上时,金童打幡,玉女捧托盘凤冠。{419}}

{有何旨意?}

{赵匡胤\hspace{20pt}~

护送圣母归位去者。({\akai 或}: 护送圣母归位,去罢。)}

{金童、玉女 领法旨。}

{张氏\hspace{30pt}~

谢主隆恩。}

{{[}第二场{]}}

{(起二更,曹小姐上)}

{曹小姐\hspace{20pt}~

({\akai 念})忙将嫂嫂事,报与兄长知。}

{曹小姐\hspace{20pt}~

兄长快些醒来,大事不好了!}

{曹彬\hspace{30pt}~

何事惊慌?}

{曹小姐\hspace{20pt}~

嫂嫂悬梁自尽了!}

{曹彬\hspace{30pt}~

哦,待我观看。}

{曹彬\hspace{30pt}~

唉,嫂嫂啊$\cdots{}\cdots{}$(哭介)}

{曹彬\hspace{30pt}~

兄长醒来!}

{赵匡胤

\setlength{\hangindent}{60pt}{ 【{\akai 二黄散板}】插花圣母归了位,三国华佗讨封回。猛然睁开丹凤眼,} }

{曹彬\hspace{30pt}~

嫂嫂哇啊$\cdots{}\cdots{}$(哭介)}

{赵匡胤\hspace{20pt}~

\setlength{\hangindent}{60pt}{ 【{\akai 二黄摇板}】贤弟缘何两泪垂?} }

{曹彬\hspace{30pt}~

唉呀兄长,嫂嫂悬梁自尽了哇$\cdots{}\cdots{}$(哭介)}

{(赵匡胤 愚兄不信。)}

{赵匡胤\hspace{20pt}~

今在何处?}

{曹彬\hspace{30pt}~

随我来!}

{赵匡胤\hspace{20pt}~

唉!嫂嫂啊$\cdots{}\cdots{}$(哭介)}

{赵匡胤\hspace{20pt}~

啊贤弟,不必悲泪({\akai 或}: 休得悲恸),嫂嫂成仙去了。}

{曹彬\hspace{30pt}~

但愿如此。}

{家人\hspace{30pt}~

报!}

{家人\hspace{30pt}~

崔龙带兵围困府门,请二老爷答话。}

{曹彬\hspace{30pt}~

起过。}

{曹彬\hspace{30pt}~

兄长暂且回避。}

{赵匡胤\hspace{20pt}~

是。}

{曹彬\hspace{30pt}~

带路。}

{曹彬\hspace{30pt}~

请了。崔将军有何见谕?}

{崔龙\hspace{30pt}~

圣上有旨: 命大人将过关人犯,带上金殿审问。}

{曹彬\hspace{30pt}~

将军人马暂退一箭之地,待弟将人犯戴上刑具,一同上殿交旨。}

{崔龙\hspace{30pt}~

大人休得迟慢,请!}

{曹彬\hspace{30pt}~

有请仁兄。}

{赵匡胤\hspace{20pt}~

贤弟何事?}

{曹彬\hspace{30pt}~

今有崔龙,带领人马,要将仁兄押上金殿见驾。}

{赵匡胤\hspace{20pt}~

就依贤弟。}

{曹彬\hspace{30pt}~

仁兄受屈了。}

{赵匡胤\hspace{20pt}~

啊贤弟,今日何日?}

{曹彬\hspace{30pt}~

中秋佳节。}

{赵匡胤\hspace{20pt}~

明日呢?}

{曹彬\hspace{30pt}~

乃是十六日。}

{赵匡胤

明日,就是我光棍家出头之日了。({\akai 或}: 明日中秋,就是我光棍家出头之日了。)}

{曹彬\hspace{30pt}~

想你这光棍家,还有什么根本不成?}

{赵匡胤\hspace{20pt}~

贤弟------}

{赵匡胤\hspace{20pt}~

\setlength{\hangindent}{60pt}{ 【{\akai 二黄碰板}】休道我光棍家根本不讲,请台座听玄郎细说端详: } }

{赵匡胤

\setlength{\hangindent}{60pt}{ 【{\akai 二黄原板}】家住在西罗县}\footnote{ 据史载,赵匡胤出生于河南洛阳夹马营。但很多地方戏曲都传说赵匡胤是``西罗县''(约在今山西洪洞县境内)。刘曾复先生有一版说戏录音中音近似``西蒙县'',可能是``西罗县''的讹误。{420}}{双龙街上,本姓赵名匡胤字表玄郎。头辈祖名赵暠家财颇广,二辈祖名赵霸({\akai 或}: 二辈祖名赵强;二辈祖名淮庆)四海名扬。三辈祖名赵强({\akai 或}: 三辈祖名赵霸)隋唐为将,子不言父的名四品黄堂}\footnote{ ``黄堂''是古代太守衙门中的正堂,可借指太守职位。  关于赵匡胤的家世,《残唐五代史演义》中称赵匡胤的祖父为赵霸,霸生弘殷,弘殷生匡胤;据《宋史·太祖本纪》载``太祖启运立极英武睿文神德圣功至明大孝皇帝,讳匡胤,姓赵氏,涿郡人也。高祖朓,是为僖祖,仕唐历永清、文安、幽都令。朓生珽,是为顺祖,历藩镇从事,累官兼御史中丞。珽生敬,是为翼祖,历营、蓟、涿三州刺史。敬生弘殷,是为宣祖。周显德中,宣祖贵,赠敬左骁骑卫上将军$\cdots{}\cdots{}$太祖,宣祖仲子也,母杜氏。后唐天成二年,生于洛阳夹马营,赤光绕室,异香经宿不散。体有金色,三日不变。既长,容貌雄伟,器度豁如,识者知其非常人。''{421}}{。生下了俺玄郎面带奇相,酒醉后杀御乐}\footnote{ 段公平君注: {据{[}清{]}吴}璿{《飞龙全传》,赵闹御院(勾栏),打女乐,后离家出逃。``杀御乐''一句即谓此。汉调二黄有``}悔不该将女乐满门杀坏{''。}{422}}{惹下祸殃。二爹娘修书信四路探望,遇柴荣和郑恩关西道旁({\akai 或}: 关西路旁)。我三人尧王庙同把香上,要学那三国中刘备、关、张。董家桥打五虎弟兄各往,柴大哥到怀庆受爵封王。好一个柴子耀不把友忘,差旗牌带书信迎接玄郎。弟兄们同饮酒花亭以上,久分手又相会畅叙衷肠。一霎时({\akai 或}: 顷刻间)鱼池内陡起风浪,吓坏了王府人俱各({\akai 或}: 吓坏了王府中个个)惊慌。大哥说鱼戏水常来常往,俺玄郎见妖魔甚是张狂。左挽弓来右搭箭照妖发放({\akai 或}: 对妖撒放;照妖撒放),谯楼上打三筹鼓角凄凉。次日里兄带我朝见皇上,郭王爷想起了梦中箭伤。顷刻间({\akai 或}: 一霎时)传旨意将我捆绑,柴大哥奏一本({\akai 或}: 保一本)刺杀刘王。悔不该({\akai 或}: 最不该)在金殿海口夸讲,不用兵不用将独下燕邦({\akai 或}: 独上燕邦)。走西门({\akai 或}: 在西门)遇见了崔龙老将,多亏你阖家搭救玄郎。昨夜晚同饮酒桌案之上({\akai 或}: 桌案以上),俺玄郎得二梦}\footnote{ 刘曾复先生有一版录音作``得一梦'',似误。{423}}{牢记心旁: 头一梦见华佗三国道长,用宝剑割龙瘤丢入长江。二贤弟你不信观看左膀,} }

{曹彬\hspace{30pt}~

\setlength{\hangindent}{60pt}{ 【{\footnotesize 接}\akai 二黄原板}】果然是无肉瘤毫无损伤。} }

{赵匡胤

\setlength{\hangindent}{60pt}{ 【{\akai 二黄原板}】第二梦见张氏嫂嫂形象,项带锁披着发珠泪汪洋。我封她为圣母在华山以上,} }

{赵匡胤\hspace{20pt}~

\setlength{\hangindent}{60pt}{ 【{\akai 二黄散板}】有金童和玉女送至山岗({\akai 或}: 送上山岗)。} }

{赵匡胤

\setlength{\hangindent}{60pt}{ 【{\akai 二黄散板}】劝贤弟在此间休要来往,搬至在怀庆府可作家乡。柴大哥是玄郎结义兄长,大小事他必然念在玄郎。天牢内二双亲劳你探望({\akai 或}: 劳你看望),说玄郎成了功即刻还乡。} }

{赵匡胤\hspace{20pt}~

\setlength{\hangindent}{60pt}{ 【{\akai 二黄散板}】二贤弟将手杻与我戴上,} }

{(\textless{}阴锣\textgreater{}崔龙人马两边上,巡查;赵匡胤换衣、戴手杻,上)}

{赵匡胤\hspace{20pt}~

\setlength{\hangindent}{60pt}{ 【{\akai 二黄散板}】此一番上金殿我杀一个倒海翻江。} }

{赵匡胤

\setlength{\hangindent}{60pt}{ 【{\akai 二黄散板}】施一礼辞贤弟出门观望({\akai 或}: 入门观望),见崔龙人和马个自逞强({\akai 或}: 个个逞强)。} }

{赵匡胤\hspace{20pt}~

\setlength{\hangindent}{60pt}{ 【{\akai 二黄散板}】真和假是与非见尔主上,俺曹仁不犯法又有何妨?} }

\newpage

\hypertarget{ux4e0bux6cb3ux4e1c-ux4e4b-ux547cux5ef6ux5bffux5ef7}{%

\subsection{下河东 之

呼延寿廷}\label{ux4e0bux6cb3ux4e1c-ux4e4b-ux547cux5ef6ux5bffux5ef7}}

{{[}第一场{]}}

{领旨!}

{({\akai 念})怀揣忠义胆,保主锦江山。}

{臣呼延寿廷见驾,吾皇万岁!}

{万万岁。}

{宣臣上殿,有何国事议论?}

{欧相乃是文职官员,焉能挂得武将帅印?}

{臣与欧相有打牙仇恨,此番到了河东,犹恐}\footnote{ 段公平君建议作``又恐''。{424}}{以公报私。}

{万岁作主。}

{万万岁。}

{参见元帅!}

{身为大将,焉能不晓军令?}

{一捆四十。}

{两捆八十。}

{这三卯------}

{呵呵呵呵$\cdots{}\cdots{}$(冷笑介)}

{也不过就是项上的人头。}

{贼呀,贼!}

{({\akai 念})身居矮檐下,怎敢不低头。}

{哎!}

{[}第二场{]}

{回府。}

{可恼!}

{今有河东打来连环战表,要我主御驾亲征。万岁命欧相挂帅,下官以为前战先行。}

{欧相奏道: 幼年习文,中年习武。({\akai 念})习就文共武,扶保帝王都。}

{好个有道明君,言道: 待等平定河东回来,与我两家解和。}

{如此有劳夫人。}

{(kai 正是}: ({\akai 念})青龙背上屯军马。}

\setlength{\hangindent}{60pt} {【{\akai 二黄导板}】这几年未出征干戈宁静,}

\setlength{\hangindent}{60pt} {【{\akai 二黄散板}】玲珑铠甲挂灰尘。}

\setlength{\hangindent}{60pt} {【{\akai 二黄散板}】迈步且把二堂进,有劳夫人点雄兵。}

\setlength{\hangindent}{60pt} {【{\akai 二黄散板}】接过夫人得胜饮,背转身来谢神灵。回头再对夫人论,下官言来你试听}\footnote{ 段公平君建议作``你是听''。{425}}{: 倘若河东遭不幸,这是呼家报仇人。}

\setlength{\hangindent}{60pt} {【{\akai 二黄散板}】辞别夫人跨金镫,}

\setlength{\hangindent}{60pt} {【{\akai 二黄散板}】但愿此去奏凯回程。}

{[}第三场{]}

{参见元帅。}

{这$\cdots{}\cdots{}$披挂来迟,元帅恕罪。}

{谢元帅!}

{圣驾到!}

{在。}

{得令。}

{令出: 圣上有旨,元帅有令: 文武百官免送。人马打从德胜门}\footnote{ 《京剧汇编》第十六集  苏连汉藏本作``得胜门''。{426}}{而出,就此响炮离京。}

{{[}第四场{]}}

{前站为何不行?}

{候令!}

{启禀元帅: 前面已到河东地界。}

{在,}

{得令。}

{令出: 下面听者: 圣上有旨,元帅有令: 就在此地,择一平阳所在,靠山近水,安营扎寨。歇兵三日,再与河东鏖战呐!}

{传令已毕。}

{{[}第五场{]}}

{({\akai 念})来到河东地,昼夜费心机。}

{带马。}

{(}呼延寿廷从上手接枪上马,左转身到上场门,右手出枪亮,小绕到小边台口双手托枪,走到下场门扎出去跺泥右手收回来举枪亮,回来到台中间右转身面向外出枪提枪花,转身,三个提枪花,枪上右手膀子,枪头向左,右手在胸前平端枪,左手向左伸出接枪杆,跨左腿,踢右腿,右脚落地,右手在脸前画圈拿枪杆下端,向右翻身面向下场门,枪交左手拿枪杆下端在左腰间平端,右手画圈握拳勒马,弓箭步亮住,下场门下{)}\footnote{ 这是{杨小楼晚年最常用的一个下场},不仅《下河东》、《阳平关》、《战宛城》等戏用它,《挑华车》、《霸王别姬》大铲枪下场也用它,《铁笼山》姜维与司马师,《贾家楼》唐璧与来护,一前一后,提枪花,并马双下场也用它。《下河东》呼延寿廷耍两个枪下场。头一个下场是呼延闻报后急于出马救欧阳芳,所以上马后只耍一个小下场表示急忙上阵。第二个下场是呼延杀败白龙太子,追击一程,用大下场。{427}}

{{[}第六场{]}}

{元帅受惊了,元帅受惊了!}

{你与白龙交战,堪堪落马,多亏末将一马当先,将你救下。来来来,请上功劳簿哇。}

{这$\cdots{}\cdots{}$无有。}

{也无有。}

{谢元帅责!}

{打的------呃,不公!}

{也不是。}

{有罪不敢抬头。}

{谢元帅。}

{是是是。}

\setlength{\hangindent}{60pt} {【{\akai 二黄散板}】奸贼做事太欺情,有功不赏反加刑。}

\setlength{\hangindent}{60pt} {【{\akai 二黄散板}】人来搀我小营进,快请姑娘到此营。}

{贤妹哪里知道,老贼与白龙交战,堪堪落马,多亏愚兄将他救下。}

{有功劳不赏,反将愚兄------唉,重责。呃$\cdots{}\cdots{}$(哭介)}

{贤妹呀!}

\setlength{\hangindent}{60pt} {【{\akai 二黄散板}】随王驾来愿王兴,食王爵禄当报恩。军权落在奸贼手哇,得自闲来且自清。}

{{[}第七场{]}}

{臣在暗地保驾。}

\newpage

\hypertarget{ux9f99ux864eux6597}{%

\subsection{龙虎斗}\label{ux9f99ux864eux6597}}

{{[}第一场{]}}

{赵匡胤\hspace{20pt}~

{[}{\akai 引子}{]}龙争虎斗,这干戈,何日罢休?}

{赵匡胤

({\akai 念})蛟龙无水困沙滩,失却明珠寻找难。奸贼诓孤河东地,不知何日返中原。}

{赵匡胤

孤,乾德王赵。可恨奸贼欧阳芳将孤诓下河东,一困七载。内无粮草,外无救应。飞报报道: 阵前来了一哨人马,杏黄旌旗招展,不知哪路救应。也曾命黄信打探,未见回报。}

{黄信\hspace{30pt}~

({\akai 内})马来!}

{黄信

({\akai 念})打探军情事,奏与万岁知}\footnote{ 刘曾复先生钞本作``报与万岁知''。{428}}{。}

{黄信\hspace{30pt}~

报,黄信告进。}

{黄信\hspace{30pt}~

参见万岁。}

{赵匡胤\hspace{20pt}~

罢了!}

黄信\hspace{30pt}~

谢万岁。

{赵匡胤\hspace{20pt}~

打探军情,有何消息?。}

{黄信

臣探得军情,乃是落伽山发来人马,昨日阵前鞭诛白龙,今日马踏御营,要万岁出马,不知是何缘故,特来启奏。}

{赵匡胤\hspace{20pt}~

赐你金牌一面,再去打探!}

{黄信\hspace{30pt}~

领旨。}

{赵匡胤

且住,适才黄信报道: 落伽山发来一哨人马,昨日阵前鞭诛白龙,今日马踏御营。要孤王出马,此事叫孤难猜难解也!}

{赵匡胤

\setlength{\hangindent}{60pt}{ 【唢呐二黄原板}】探马儿不住地飞来报,他报道落伽山兵马一标。杏黄旗不住地}\footnote{ 刘曾复先生钞本作``不住得''。{429}}{空中飘绕,吓坏了满营中大小儿曹。(}恨河东打来了连环战表,他要夺孤王我锦绣龙朝。\footnote{ 刘曾复先生钞本中有此二句。{430}}){欧阳芳在金殿帅印挂了,呼延(寿)廷倒作了马前英豪。下河东连七载须发苍了,(须发苍了,)为国家哪顾得昼夜辛劳。在头上摘下了飞龙帽罩,} }

{赵匡胤\hspace{20pt}~

\setlength{\hangindent}{60pt}{ 【唢呐二黄原板}】(身上------)在身上脱下了衮龙袍。} }

{赵匡胤

\setlength{\hangindent}{60pt}{ 【唢呐二黄散板}】孤的玲珑------玲珑铠甲丝绦绕,杀人妙计有千条。御林军且回避黄罗道}\footnote{ ``{御林军且回避黄罗道}''  一句,陈超老师从刘曾复先生学的是``{御林军且退宝帐道}''。刘曾复先生钞本亦作``御林军且退宝帐道''。{431}}{,站立辕门喊喝高。教槽头将孤的龙驹马鞍韂备好,} }

{马童\hspace{30pt}~

啊!}

{马童\hspace{30pt}~

请万岁上马!}

{赵匡胤\hspace{20pt}~

\setlength{\hangindent}{60pt}{ 【唢呐二黄散板}】这才是无良将亲把兵交。} }

{{[}第二场{]}}

{呼延赞\hspace{20pt}~

哈哈,哈哈,哇呀呀$\cdots{}\cdots{}$(笑介)}

{{[}第三场{]}}

{赵匡胤\hspace{20pt}~

\setlength{\hangindent}{60pt}{ 【唢呐二黄摇板}】催马来在战场道,看一看落伽山哪家英豪。} }

{呼延赞\hspace{20pt}~

({\akai 内})【唢呐二黄导板}】乌骓马不住得连声吼,}

{呼延赞\hspace{20pt}~

\setlength{\hangindent}{60pt}{ 【唢呐二黄原板}】打将鞭一举鬼神愁。催战马来至在双阳道口,} }

{呼延赞\hspace{20pt}~

呔!}

{呼延赞\hspace{20pt}~

\setlength{\hangindent}{60pt}{ 【唢呐二黄原板}】叫一声乾德王快出龙楼。} }

{赵匡胤\hspace{20pt}~

\setlength{\hangindent}{60pt}{ 【唢呐二黄原板}】只见他乌油盔乌油甲,} }

{赵匡胤\hspace{20pt}~

皂罗------}

{赵匡胤

\setlength{\hangindent}{60pt}{ 【唢呐二黄原板}】皂罗袍上绣团花。问一声小将名和姓,通上名来哪里有家。} }

{呼延赞

\setlength{\hangindent}{60pt}{ 【唢呐二黄原板}】家住在落伽山呼家寨,我的父就是呼延寿廷。若问少爷名和姓,} }

{呼延赞\hspace{20pt}~

呼延赞------}

{呼延赞\hspace{20pt}~

\setlength{\hangindent}{60pt}{ 【{\akai 二黄原板}】就是少爷名。} }

{赵匡胤

\setlength{\hangindent}{60pt}{ 【{\akai 二黄散板}】人道呼家无有后,哪里又来这条根。不通姓名催马走,} }

{呼延赞\hspace{20pt}~

哪里走!}

{呼延赞\hspace{20pt}~

\setlength{\hangindent}{60pt}{ 【{\akai 二黄散板}】你少爷不打无名人。} }

{赵匡胤\hspace{20pt}~

\setlength{\hangindent}{60pt}{ 【{\akai 二黄散板}】乾德王御驾宝帐坐,我是帐前一小军。} }

{呼延赞\hspace{20pt}~

住口!}

{呼延赞

\setlength{\hangindent}{60pt}{ 【{\akai 二黄散板}】蚕眉凤目龙颜相,五绺长髯飘胸膛。看你不像小军样,定是当今乾德王。} }

{赵匡胤

\setlength{\hangindent}{60pt}{ 【{\akai 二黄散板}】听一言来心火上,孤王何惧小儿郎。若问老爷名和姓呐,少打关西赵玄郎。} }

{呼延赞\hspace{20pt}~

啊?!}

{呼延赞

\setlength{\hangindent}{60pt}{ 【{\akai 二黄散板}】听说来了贼昏王,太阳头上冒火光。手持钢鞭朝下打呀。\textless{}扫一句\textgreater{}} }

{赵匡胤

\setlength{\hangindent}{60pt}{ 【{\akai 二黄散板}】小将生来实可夸,手执钢鞭似铁塔。一鞭打下难招架,震得孤王虎口麻。勒住丝缰且住马,小将到来顺说他。} }

{呼延赞

\setlength{\hangindent}{60pt}{ 【{\akai 二黄散板}】昏王休得来逃遁,快将诡计来说清: 我父身犯何条令,缘何}\footnote{ 刘曾复先生钞本作``{为何''。}{432}}{斩首在辕门。} }

{赵匡胤\hspace{20pt}~

小将!}

{赵匡胤

\setlength{\hangindent}{60pt}{ 【{\akai 二黄原板}】小将不必问详情,孤王言来听分明。误中奸贼反间计,满营喊叫反了那呼延寿廷。孤王我斩字未开口,} }

{赵匡胤\hspace{20pt}~

欧阳芳,}

{赵匡胤\hspace{20pt}~

\setlength{\hangindent}{60pt}{ 【{\akai 二黄原板}】欧阳芳拔剑杀尔天伦。} }

{呼延赞\hspace{20pt}~

呸!}

{呼延赞\hspace{20pt}~

\setlength{\hangindent}{60pt}{ 【{\akai 二黄散板}】你不传旨谁敢斩,哪有臣子乱杀人。} }

{赵匡胤\hspace{20pt}~

\setlength{\hangindent}{60pt}{ 【{\akai 二黄散板}】军权落在奸贼手,孤王也要听令行。} }

{呼延赞\hspace{20pt}~

\setlength{\hangindent}{60pt}{ 【{\akai 二黄散板}】懦弱之人怎为君,苍天再降紫微星。} }

{赵匡胤

\setlength{\hangindent}{60pt}{ 【{\akai 二黄散板}】小将休得胡言讲,孤王言来听端详。你若马前来归降,} }

{赵匡胤\hspace{20pt}~

封你为前殿王。}

{呼延赞\hspace{20pt}~

不要!}

{赵匡胤\hspace{20pt}~

后殿王。}

{呼延赞\hspace{20pt}~

不要!}

{赵匡胤\hspace{20pt}~

\setlength{\hangindent}{60pt}{ 【{\akai 二黄散板}】一十八家总兵王。} }

{呼延赞\hspace{20pt}~

\setlength{\hangindent}{60pt}{ 【{\akai 二黄散板}】不做官来不受管,一心要报我父冤。} }

{赵匡胤

\setlength{\hangindent}{60pt}{ 【{\akai 二黄散板}】小将说话太张狂}\footnote{ 刘曾复先生钞本作``{太猖狂}''。{433}}{,不由孤王怒满膛}\footnote{ 刘曾复先生钞本作``怒满腔''。{434}}{。我今与你分上下。\textless{}扫一句\textgreater{}} }

{{[}第四场{]}}

{赵匡胤\hspace{20pt}~

\setlength{\hangindent}{60pt}{ 【{\akai 二黄散板}】人困马乏难交战,} }

{赵匡胤\hspace{20pt}~

啊?!}

{赵匡胤

\setlength{\hangindent}{60pt}{ 【{\akai 二黄散板}】只见黑虎落雕鞍}\footnote{ 刘曾复先生钞本作``卧雕鞍``。{435}}{。手执金锏朝下打,} }

{赵匡胤\hspace{20pt}~

\setlength{\hangindent}{60pt}{ 【{\akai 二黄散板}】只见黑虎奔南山。} }

{呼延赞\hspace{20pt}~

\setlength{\hangindent}{60pt}{ 【{\akai 二黄散板}】梦里只见我父到,} }

{呼延赞

\setlength{\hangindent}{60pt}{ 【{\akai 二黄散板}】只见青龙卧鞍鞒。手执钢鞭朝下打,只见青龙上九霄。} }

{呼延赞\hspace{20pt}~

哈哈,哈哈,啊哈哈哈$\cdots{}\cdots{}$(笑介)}

{呼延赞\hspace{20pt}~

\setlength{\hangindent}{60pt}{ 【{\akai 二黄散板}】翻身下了乌骓马,} }

{呼延赞\hspace{20pt}~

\setlength{\hangindent}{60pt}{ 【{\akai 二黄散板}】三呼万岁把臣饶。} }

{赵匡胤\hspace{20pt}~

\setlength{\hangindent}{60pt}{ 【{\akai 西皮导板}】耳边厢又听得山呼万岁,} }

{呼延赞\hspace{20pt}~

万岁!}

{赵匡胤

\setlength{\hangindent}{60pt}{ 【{\akai 西皮原板}】是何方保驾臣来在军前。乾德王睁开了丹凤眼,只见小将 }

【{\footnotesize 转}{\akai 西皮快板}】跌跪马前。马上擒王擒不住,诓孤下马难上难。}

{呼延赞\hspace{20pt}~

\setlength{\hangindent}{60pt}{ 【{\akai 西皮快板}】万岁不必心惊怕,为臣保你坐中华。} }

{赵匡胤\hspace{20pt}~

\setlength{\hangindent}{60pt}{ 【{\akai 西皮快板}】你若保孤坐中华,可对苍天把誓发。} }

{呼延赞

\setlength{\hangindent}{60pt}{ 【{\akai 西皮快板}】跪在阵前把誓发,日月三光照某家。呼延赞保主若有假,死在千军万马踏。} }

{赵匡胤

\setlength{\hangindent}{60pt}{ 【{\akai 西皮快板}】一见小将把誓发,不由孤王笑哈哈。左思右想不下马,} }

{赵匡胤\hspace{20pt}~

\setlength{\hangindent}{60pt}{ 【{\akai 西皮散板}】金锏挑起小卿家。} }

{呼延赞

\setlength{\hangindent}{60pt}{ 【{\akai 西皮散板}】叩罢头来谢王恩,尊声万岁听分明: 我父仇人哪一个?} }

{赵匡胤\hspace{20pt}~

\setlength{\hangindent}{60pt}{ 【{\akai 西皮散板}】欧阳芳是尔对头人。} }

{呼延赞\hspace{20pt}~

\setlength{\hangindent}{60pt}{ 【{\akai 西皮散板}】老贼何处把身隐,} }

{赵匡胤\hspace{20pt}~

\setlength{\hangindent}{60pt}{ 【{\akai 西皮散板}】卿家后营去找寻。} }

{呼延赞

\setlength{\hangindent}{60pt}{ 【{\akai 西皮散板}】辞别万岁后营进。\textless{}扫一句\textgreater{}} }

{赵匡胤

\setlength{\hangindent}{60pt}{ 【{\akai 西皮快板}】喜孜孜来笑盈盈,看来孤王有福分。今日收了这员将,} }

{赵匡胤\hspace{20pt}~

\setlength{\hangindent}{60pt}{ 【{\akai 西皮摇板}】哪怕河东百万兵。} }

\newpage

\hypertarget{ux8d3aux540eux9a82ux6bbf-ux4e4b-ux8d75ux5149ux4e49}{%

\subsection{贺后骂殿 之

赵光义}\label{ux8d3aux540eux9a82ux6bbf-ux4e4b-ux8d75ux5149ux4e49}}

{{[}{\akai 引子}{]}兄亡侄幼,众文武,辅孤登极}\footnote{ ``登极'',《京剧汇编》第一{百}零九集作``登基'',下同。{436}}{。}

{众卿平身。}

{({\akai 念})兄王晏驾龙归西,全凭争先一着棋。满朝文武来辅助,孤王才得立帝基。}

{孤,赵光义。今日登殿受贺,众卿!}

{(众\hspace{40pt}~

万岁!)}

{孤王登极,不知当降何国号?}

{依卿所奏。}

{代孤传旨}\footnote{ 段公平君建议作``待孤传旨''。{437}}{,晓谕天下。}

{潘洪听封。}

{封你左班丞相,卿女执掌昭阳正院。}

{赵普听封。}

{封你右班丞相,代孤执掌朝政。}

{曹彬听封。}

{封卿孝义侯之位,代管禁军。}

{苗宗善听封。}

{封你护国军师,子袭父职。}

{潘疆}\footnote{ 《京剧汇编》第一{百}零九集作``潘强''。{438}}{、潘豹。}

{封你二人为镇殿将军。}

{满朝文武,加升三级,大赦天下。}

{有本早奏,无本退班!}

{代孤传旨,宣杨继业上殿!}

{嗯------大胆杨继业,孤王登极,为何不来朝贺?}

{哪里是参驾来迟,分明藐视寡人。看孤登极,心中不服。}

{殿前武士。}

{推出斩了}

{呃------怎么参本是你,保本也是你?}

{依卿所奏。}

{将杨继业赦回!}

{非是孤王不斩于你,念你在朝有十大汗马功劳。今将你削职为民,赐你百亩田园;无事不准入朝,限三日出京,三日不走,其罪还在。下殿!}

{(赵德昭 【{\akai 二黄散板}】$\cdots{}\cdots{}$还我锦家邦。)}

\setlength{\hangindent}{60pt} {【{\akai 二黄散板}】大皇儿来在金殿上,口口声声要家邦。}孤本当下位将国\textless{}\!{\bfseries\akai 哭头}\!\textgreater{}让,

\setlength{\hangindent}{60pt} {【{\akai 二黄散板}】}难学尧、舜------禹、商汤\footnote{ {《京剧汇编》第一百零九集}作``难学尧舜与商汤''。{439}}。

\setlength{\hangindent}{60pt} {【{\akai 二黄散板}】皇儿近前听叔讲,你母子宫中乐安康。}

{(赵德昭

\setlength{\hangindent}{60pt}{ 【{\akai 二黄散板}】$\cdots{}\cdots{}$你今不把江山让,篡位的名儿天下扬。)} }

{诶------}

\setlength{\hangindent}{60pt} {【{\akai 二黄散板}】皇儿休得多言讲,孤不封你自为王。吩咐潘豹与潘疆,}

{(潘豹、潘疆 有。)}

\setlength{\hangindent}{60pt} {【{\akai 二黄散板}】再有人胡言绑云阳}\footnote{ 云阳,在今约陕西淳化县西北,是秦代监狱、刑场所在地。所以后常以``云阳''或``云阳市曹''{代指刑场。}{440}}{。}

{(赵德昭 唉呀!)}

\setlength{\hangindent}{60pt} {【{\akai 二黄慢板}】自盘古立帝基天子为重}\footnote{ {《京剧丛刊》第三十四集  贯大元口述本作``自盘古立帝邦天子为重'';《京剧汇编》第一百零九集作``自盘古开天地皇帝为重''。此处``立帝基''也有以盘古开天劈地故事,作``立地基''的。}{441}}{,老皇嫂骂孤王情理难容。论国法该将你残生断送,}

{(贺后\hspace{30pt}~

谁敢?!)}

{退班!}

{皇嫂!}

\setlength{\hangindent}{60pt} {【{\akai 二黄碰板三眼}】还念你与皇兄掌印正宫。先王爷晏了驾钟鼓齐动,满朝中文武臣议论孤穷}\footnote{ {《京剧丛刊》第三十四集  贯大元口述本作``议论皆同'';《京剧汇编》第一百零九集作``议论不同''。}{442}}{。全都道大皇儿年轻无用,一个个辅保孤驾坐九重啊。孤虽然掌山河依然大宋,并非是外姓人来坐金龙。走向前、再打一躬把皇嫂尊奉,昭阳院改作了养老宫。将皇嫂当作了太后侍奉,崇上徽号容是不容。}

\setlength{\hangindent}{60pt} {【{\akai 二黄原板}】老皇嫂说什么务农耕种,普天下尽都是老王荣封。享荣华、受富贵母子同共,并非是叔为君、侄为臣各自西东。赐皇嫂尚方剑泰山压重}\footnote{ {《京剧汇编》第一百零九集作``泰山压众''。}{443}}{,管三宫和六院,大小嫔妃若有违抗任你施行,你从也不从?}

\setlength{\hangindent}{60pt} {【{\akai 二黄原板}】赵德芳我的儿莫要悲痛,近前来听为叔将儿来封: 孤赐你金镶白玉锁,加封你一钦王、二良王、三忠、四正、五德王、六靖王,}\footnote{ {《京剧汇编》第一百零九集作``一秦王、二梁王、三忠王、四正王、五德王、六延王,''。}{444}}{上殿不参王、你下殿不辞王,再赐你凹面金锏上打昏君、下打谗臣,压定了满朝的文武,哪一个不尊,你是个八贤王,带管孤穷}\footnote{ {《京剧丛刊》第三十四集 贯大元口述本作``孤躬''。}  \begin{quote}  另,从吴同宾先生学戏的人曾撰文回忆,吴同宾先生亦指出,《贺后骂殿》中的``孤穷''(或``孤穹'')系讹误,当作``孤躬''(因为``躬''与``窮''字相似,旧时艺人误识,乃至因袭)。  \end{quote}  {445}}{。}

\setlength{\hangindent}{60pt} {【{\akai 二黄散板}】贤皇侄从今后莫要悲痛,老皇嫂请回养老宫。}

{(贺后\hspace{30pt}~

\setlength{\hangindent}{60pt}{ 【{\akai 二黄散板}】$\cdots{}\cdots{}$三尺龙泉不容留。)} }

{众卿!}

{孤身不爽,是何缘故?}

{(潘洪\hspace{30pt}~

$\cdots{}\cdots{}$休养。)}

{就命卿家,代孤办理。}

{(潘洪\hspace{30pt}~

领旨。)}

{退班。}

{(潘洪\hspace{30pt}~

$\cdots{}\cdots{}$回宫。)}

\newpage

\hypertarget{ux91d1ux6c99ux6ee9-ux4e4b-ux6768ux7ee7ux4e1a}{%

\subsection{金沙滩 之

杨继业}\label{ux91d1ux6c99ux6ee9-ux4e4b-ux6768ux7ee7ux4e1a}}

{{[}第一场{]}}

{({\akai 念})腰悬三尺剑,保主锦江山。}

{老夫,杨继业。宋王驾前为臣,奉主之命,巡营瞭哨。今有北国鞑儿韩昌,兴兵前来,请我主敌楼答话。众儿郎,}

{回营交旨。}

{[}第二场{]}

{({\akai 念})忙将韩昌事,奏与万岁知。}

{臣,继业见驾,吾皇万岁,}

{万万岁。}

{贤爷千岁,}

{千千岁。}

{谢座。}

{今有北国鞑儿韩昌,兴兵前来,请我主敌楼答话。}

{领旨。}

{[}第三场{]}

{(韩昌\hspace{30pt}~

$\cdots{}\cdots{}$宋王请了。)}

{贤爷在此!}

{(韩昌

$\cdots{}\cdots{}$巴特鲁}\footnote{ ``{巴特鲁}'',系满语``勇士''之意。{446}}{。)}

{住口!}

{\textless{}三叫头\textgreater{}万岁!吾主!唉,万岁呃!}

\setlength{\hangindent}{60pt} {【{\akai 西皮导板}】事到如今恨奸党,}

\setlength{\hangindent}{60pt} {【{\akai 西皮原板}】潘仁美诓圣驾来到番邦。哪里是双龙会把宴上}\footnote{ 夏行涛君建议作``把宴赏''。{447}}{,明明是一座杀人战场。若要想救主爷出罗网,除非是学纪信替主荥阳。}

\setlength{\hangindent}{60pt} {【{\akai 西皮摇板}】低下头来暗思想,}

{有了!}

\setlength{\hangindent}{60pt} {【{\akai 西皮快板}】忽然一计上心旁。继业撩铠出宝帐,}

\setlength{\hangindent}{60pt} {【{\akai 西皮快板}】众家儿郎听端详: 哪个孩儿有胆量,替主赴会到番邦。}

\setlength{\hangindent}{60pt} {【{\akai 西皮摇板}】答话儿郎哪一个,}

\setlength{\hangindent}{60pt} {【{\akai 西皮摇板}】进帐来父子们再作商量。}

{儿有此胆量?}

{好哇!}

\setlength{\hangindent}{60pt} {【{\akai 西皮散板}】这才是父是英雄儿有胆量,杨家出了假宋王。}

\setlength{\hangindent}{60pt} {【{\akai 西皮散板}】父子一同进宝帐,}

\setlength{\hangindent}{60pt} {【{\akai 西皮散板}】把本启奏圣吾皇。}

{臣启万岁: 长子延平愿替主赴会。}

{看衣更换。}

{儿啊!}

\setlength{\hangindent}{60pt} {【{\akai 西皮快板}】我的儿休得悲声放,痛哭尤恐惊君王。继业二次出宝帐,}

\setlength{\hangindent}{60pt} {【{\akai 西皮快板}】众家儿郎听端详: 哪个孩儿有胆量,保你大哥赴双龙去至番邦?}

{(众\hspace{40pt}~

我等愿去呃。)}

{好呃!}

\setlength{\hangindent}{60pt} {【{\akai 西皮散板}】二郎、三郎一声叫,四、五、六、七、八郎细听端详: 去时弟兄人四对,回来还复人四双。你大哥若有好和歹,我管教尔等把命偿。}

\setlength{\hangindent}{60pt} {【{\akai 西皮散板}】众家儿郎把马上,}

\setlength{\hangindent}{60pt} {【{\akai 西皮散板}】儿在那酒席宴前见机行,谨慎提防。}

{(杨延平 【{\akai 西皮散板}】$\cdots{}\cdots{}$两泪汪。)}

{\textless{}叫头\textgreater{}延平!我儿!}

{\textless{}哭头\textgreater{}啊,我的儿啊!}

\setlength{\hangindent}{60pt} {【{\akai 西皮摇板}】众家儿郎把马上,把本启奏圣吾皇。}

{此地不是藏龙之所,请吾主驾转五台。}

{带马------}

\newpage

\hypertarget{ux78b0ux7891}{%

\subsection{碰碑}\label{ux78b0ux7891}}

{{[}第一场{]}}

{(\textless{}点绛唇\textgreater{},杨延嗣上高台)}

{杨延嗣

({\akai 念})忆昔当年赴两狼,交牙虎口摆战场。可恨潘洪行毒计,法标屈受箭锋芒。}

{杨延嗣

吾乃七郎鬼魂是也。今当我父归位之期,我不免去至宋营,托梦一番({\akai 或}: 托梦一回)。众鬼卒,驾起阴风,宋营去者。}

{(杨延嗣下高台,面向里)}

{杨延嗣\hspace{20pt}~

\setlength{\hangindent}{60pt}{ 【{\akai 二黄导板}】叫鬼卒列两厢前把路引,} }

{(杨延嗣面向外)}

{杨延嗣\hspace{20pt}~

\setlength{\hangindent}{60pt}{ 【回龙】杨七郎在空中暗自思忖: } }

{杨延嗣

\setlength{\hangindent}{60pt}{ 【{\akai 二黄原板}\footnote{ 刘曾复先生示范说戏时告知,此处{台上``扯四门''。}{448}}{】我杨家投宋君忠心秉正({\akai 或}: 忠心耿耿),弟兄们八员将来到雁门。两狼山打一仗父子被困,可怜我搬救兵不得回程。叫鬼卒驾阴风宋营来进({\akai 或}: 叫鬼卒前引路两狼来进),此一去到宋营托梦爹尊。} }

{(杨延嗣下)}

{{[}第二场{]}}

{杨继业\hspace{20pt}~

({\akai 内})【{\akai 二黄导板}】金乌坠玉兔升黄昏时候,}

{(杨继业上)}

{杨继业\hspace{20pt}~

\setlength{\hangindent}{60pt}{ 【回龙】盼娇儿不由人珠泪双流,我的儿啊!} }

{杨继业

\setlength{\hangindent}{60pt}{ 【{\akai 二黄三眼}】七郎儿 }

【{\footnotesize 转}{\akai 二黄原板}】回雁门搬兵求救,为什么此一去不见回头。唯恐那潘仁美忆起前仇,怕的是我的儿一命罢休。含悲泪进大营双眉愁皱({\akai 或}: 含悲泪进宝帐双眉愁皱),腹内饥身寒冷遍体飕飕哇。}

{(杨延昭上)}

{杨延昭

\setlength{\hangindent}{60pt}{ 【{\akai 二黄原板}】听谯楼打罢了二更时分,杨延昭倒做了巡营之人。迈虎步我且把宝帐来进,又只见老爹爹瞌睡沉沉。我这里上前去与父盖定,} }

{杨延昭\hspace{20pt}~

\setlength{\hangindent}{60pt}{ 【{\akai 二黄原板}】闷悠悠坐一旁好不伤情。} }

{(杨延嗣上)}

{杨延嗣

\setlength{\hangindent}{60pt}{ 【{\akai 二黄原板}】听谯楼打罢了三更时分,阴曹府来了我七郎鬼魂。叫鬼卒驾阴风宋营来进,} }

{杨延嗣\hspace{20pt}~

\setlength{\hangindent}{60pt}{ 【{\akai 二黄原板}】又只见老爹爹瞌睡沉沉。我这里将他的灵魂唤醒,} }

{(\textless{}乱锤\textgreater{},杨继业托髯口)}

{杨继业

\setlength{\hangindent}{60pt}{ 【{\akai 二黄散板}】猛抬头又只见七郎娇生。我命儿回雁门搬取救应,儿为何哭啼啼身带雕翎?} }

{杨继业\hspace{20pt}~

\setlength{\hangindent}{60pt}{ 【{\akai 二黄散板}】我这里下位去将儿抱定呐,} }

{杨延嗣

\setlength{\hangindent}{60pt}{ 【{\akai 二黄原板}】老爹爹休贪睡细听儿云({\akai 或}: 老爹爹休贪睡儿有话云): 都只为潘仁美想起了打子仇恨,将孩儿绑法标乱箭攒身。} }

{杨延嗣

\setlength{\hangindent}{60pt}{ 【{\akai 二黄原板}】转面来再对六兄论,小弟言来你是听: 高堂老母要你孝敬,杨延嗣倒做了不孝之人}\footnote{ ``不孝之人''也可作``不肖之人''。{449}}{。我本当与父兄再把话论({\akai 或}: 我本当与六兄再把话论),可怜我父在阳儿在阴}\footnote{ 刘曾复先生示范说戏时介绍,裘桂仙此句唱``怕的是天明亮难回天庭''。{450}}{。} }

{(杨延嗣下)}

{杨继业\hspace{20pt}~

\setlength{\hangindent}{60pt}{ 【{\akai 二黄导板}】方才朦胧得一梦({\akai 或}: 方才朦胧将养静呐),} }

{杨继业\hspace{20pt}~

\setlength{\hangindent}{60pt}{ 【{\akai 二黄散板}】梦见了七郎儿转回大营呐。} }

{杨继业\hspace{20pt}~

\setlength{\hangindent}{60pt}{ 【{\akai 二黄散板}】睁开了昏花眼难以扎挣,} }

{杨继业\hspace{20pt}~

\setlength{\hangindent}{60pt}{ 【{\akai 二黄散板}】又只见六郎儿瞌睡沉沉。} }

{杨继业\hspace{20pt}~

(我儿)醒来。}

{杨延昭

\setlength{\hangindent}{60pt}{ 【{\akai 二黄散板}】$\cdots{}\cdots{}$见七弟啊入梦,抬头只见老严亲。} }

{杨延昭\hspace{20pt}~

\textless{}叫头\textgreater{}哎呀,爹爹啊!}

{杨延昭

孩儿昨晚四更时分,梦见七弟浑身是血,遍体雕翎。不知是何缘故?}

{杨继业

为父也得此兆。唉呀儿啊------这``梦梦相应,必有应验''。为父意欲,命我儿回至雁门,打听你七弟的下落,为父的也好放心呐。({\akai 或}: 为父也有此兆。有道是``梦梦相应,必有应验''。为父有意命我儿回转雁门,探听你七弟的下落,为父的也好放心呐。或: 为父也有此兆。唉呀儿啊------这``梦梦相同,必有应验''。为父有意命我儿回转雁门,探听你七弟的下落,为父的也好放心呐。)}

{杨延昭\hspace{20pt}~

孩儿不去。}

{杨继业\hspace{20pt}~

为何?}

{杨延昭\hspace{20pt}~

爹爹年迈,孩儿放心不下。}

{杨继业

为父么------虽则年迈,倒还康健。({\akai 或}: 儿来看------为父的虽则年迈,身体倒还康健。)有道是: 虎老雄心在。儿只管地前去。}

{(杨延昭 孩儿放心不下。)}

{杨继业\hspace{20pt}~

你当真不去?}

{杨延昭\hspace{20pt}~

当真不去。}

{杨继业\hspace{20pt}~

果然不去。}

{杨延昭\hspace{20pt}~

果然不去。}

{杨继业\hspace{20pt}~

儿啊,为父有父子之情,难道儿就无有手足之义么?}

{杨延昭\hspace{20pt}~

爹爹不必如此,孩儿前去就是。}

{杨继业\hspace{20pt}~

好,上马去罢!}

{杨延昭

\setlength{\hangindent}{60pt}{ 【{\akai 二黄散板}】爹爹不必泪伤淋,孩儿言来听分明: 倘若胡兵来叫阵,紧守大营莫出征。辞别爹爹足踏镫,雁门关前走一程。} }

{(杨延昭 \textless{}叫头\textgreater{}爹爹!我父!)}

{杨继业\hspace{20pt}~

\textless{}叫头\textgreater{}延昭!我儿!}

{(杨延昭 罢!)}

{(杨延昭下)}

{杨继业\hspace{20pt}~

\textless{}哭头\textgreater{}啊$\cdots{}\cdots{}$我的儿啊!}

{杨继业

\setlength{\hangindent}{60pt}{ 【{\akai 二黄散板}】见娇儿上了马能行,指着雁门骂一声呐({\akai 或}: 手指潘洪骂一声呐;手指雁门骂一声呐)。我儿若有好和歹呀,} }

{杨继业\hspace{20pt}~

潘洪呐!贼------}

{杨继业\hspace{20pt}~

\setlength{\hangindent}{60pt}{ 【{\akai 二黄散板}】我将老命与尔拼({\akai 或}: 拚着老命与尔拼)。} }

{(杨继业下)}

{{[}第三场{]}}

{(丑扮韩延寿上)}

{韩延寿\hspace{20pt}~

俺,韩延寿!奉了太后之命,巡营瞭哨。}

{韩延寿\hspace{20pt}~

巴特鲁,巡营瞭哨者。}

{(杨延昭上,拿剑与韩延寿架住)}

{杨延昭\hspace{20pt}~

何人挡住某家去路!}

{韩延寿\hspace{20pt}~

六郎,前者饶尔不死,又来则甚?}

{杨延昭\hspace{20pt}~

一派胡言,放马过来!}

{(杨延昭、韩延寿一合两合,杨延嗣上,挥蝇帚,韩延寿众倒地。杨延昭、杨延嗣下。韩延寿起)}

{韩延寿

且住!正要擒拿六郎下马,七郎显圣,不是马走如飞,险遭不测。}

{韩延寿\hspace{20pt}~

巴特鲁,收兵。}

{{[}第四场{]}}

{(杨延昭上)}

{杨延昭\hspace{20pt}~

休赶呐休赶。}

{杨延昭\hspace{20pt}~

\setlength{\hangindent}{60pt}{ 【{\akai 二黄散板}】打开玉笼飞彩凤,斩断金锁走蛟龙。} }

{(杨延昭下)}

{[}第五场{]}

{(四老军\textless{}慢长锤\textgreater{}引杨继业上)}

{杨继业

\setlength{\hangindent}{60pt}{ 【{\akai 反二黄慢板}】叹杨家秉忠心大宋扶保,到如今呐只落得冰解瓦消。恨北国萧银宗打来战表,擅想夺吾主爷锦绣龙朝。贼潘洪在金殿帅印挂了,我父子倒做了马前的英豪。} }

{(杨继业归中间,四老军坐下)}

{杨继业

\setlength{\hangindent}{60pt}{ 【{\akai 反二黄慢板}】金沙滩双龙会一阵败呃了,只杀得血成河鬼哭神嚎。我的大郎儿 }

【{\footnotesize 转}{\akai 反二黄快三眼}】替宋王把忠尽了,二郎儿短剑下命赴阴曹。杨三郎被马踏尸首不晓,四、八郎失番营无有下梢。杨五郎在五台学禅修道,七郎儿被潘洪箭射法标({\akai 或}: 箭射芭蕉}\footnote{ 刘曾复先生曾告知,``芭蕉''是从俗的唱法。{451}}{)。只落得杨延昭随营征讨,可怜他尽得忠、又尽孝,昼夜杀砍、马不停蹄、为国辛劳。可怜我八个子把四子丧了,把四子丧了!}

{杨继业\hspace{20pt}~

\textless{}哭头\textgreater{}我的儿啊!}

{杨继业

\setlength{\hangindent}{60pt}{ 【{\akai 反二黄原板}】眼见得年迈人无有下梢({\akai 或}: 眼见得一家人无有下梢)。方良臣和潘洪又生计呃巧,请我主到五台快乐逍遥。又谁知中了那奸贼的笼套,四下里众番奴犹如海潮。(耳边厢又听得一声号炮,直吓得宋王爷跌落鞍桥。)多亏了杨延昭一马来到哇,一杆枪救圣驾逃出笼牢。有老夫二次里也来赶到,害得我东西杀砍、左冲右突、虎闯羊群,被困在两狼山,里无粮、外无草,救兵不到,眼见得我这老残生就难以还朝。} }

{杨继业\hspace{20pt}~

\textless{}哭头\textgreater{}我的儿啊!}

{(四老军站起)}

{(老军\hspace{30pt}~

饿!)}

{杨继业\hspace{20pt}~

\setlength{\hangindent}{60pt}{ 【{\akai 反二黄原板}】饥饿了就该把战马斩了,} }

{(老军\hspace{30pt}~

冷呐!)}

{杨继业\hspace{20pt}~

\setlength{\hangindent}{60pt}{ 【{\akai 反二黄原板}】身寒冷({\akai 或}: 天寒冷)就该把大营焚烧。} }

{(老军\hspace{30pt}~

雁来了!

雁来了!)}

{(杨继业望,拿弓射雁)}

{杨继业

\setlength{\hangindent}{60pt}{ 【{\akai 反二黄原板}】宝雕弓打不着空呃中飞鸟,弓折弦断}\footnote{ 刘曾复先生曾告知,``弓折弦断''唱``弓奓弦断''更准确,``奓''是``张开''的意思。{452}}{({\akai 或}: 弓开箭断)为的是哪条?} }

{(老军\hspace{30pt}~

石虎把战马咬倒!)}

{杨继业\hspace{20pt}~

再探!}

{杨继业\hspace{20pt}~

不、不、不$\cdots{}\cdots{}$不好了!}

{杨继业

\setlength{\hangindent}{60pt}{ 【{\akai 二黄散板}】恨石虎把我的战马咬倒}\footnote{ 段公平君注: 刘曾复先生曾告,``咬''字古音念作``交(jiāo)''音{453}}{,为大将无坐骑怎把兵交({\akai 或}: 为大将无良骑怎把兵交)?} }

{杨继业

\setlength{\hangindent}{60pt}{ 【{\akai 二黄散板}】看过了青龙刀}\footnote{ 此处``青龙刀''也有唱``定宋刀''的。{454}}{且把路找呃,寻一个避风所再作计呃较。} }

{(杨继业下)}

{[}第六场{]}

{(\textless{}点绛唇\textgreater{},苏武穿红官衣,忠纱,黑三,上高台,坐)}

{苏武

({\akai 念})太阳一出万丈高,光阴犹如斩人刀。日月穿梭催人老,盖世忠良无下稍。}

{苏武\hspace{30pt}~

吾乃大汉苏武是也。

今当令公归位之期,奉了玉帝敕旨,前去点化。}

{苏武\hspace{30pt}~

众云童,}

{(众\hspace{40pt}~

有。)}

{苏武\hspace{30pt}~

架起祥云,两狼去者。}

{苏武

\textless{}清江引\textgreater{}渺渺茫茫祥云万丈高,荡荡悠悠人间不觉晓。善恶明彰报({\akai 或}: 善恶彰明报),只争晚共早({\akai 或}: 只争迟和早)。六道轮回,终须走这遭。}\footnote{ 陈超老师注: {唱\textless{}清江引\textgreater{}牌子的时候,苏武不动,云童一翻两翻。过去《鸿鸾禧》一剧,天喜星也唱这个牌子,唱完后小生才念``好大雪''。}{455}}

{(众\hspace{40pt}~

已至两狼。)}

{苏武\hspace{30pt}~

(好,)看衣改换。}

{(\textless{}合龙\textgreater{}苏武当场换衣)}

{苏武\hspace{30pt}~

(尔等)两厢退下。}

{(苏武面向小边甩蝇帚)}

{苏武\hspace{30pt}~

变化一座苏武庙({\akai 或}: 幻化一座苏武庙),}

{(苏武面向大边甩蝇帚)}

{苏武\hspace{30pt}~

变化一座李陵碑({\akai 或}: 幻化一座李陵碑)。}

{(苏武将蝇帚扔向后台)}

{苏武\hspace{30pt}~

再变化一只老羊({\akai 或}: 再幻化一只老羊)。}

{苏武\hspace{30pt}~

远远望见令公来也。}

{(杨继业 走哇!)}

{(杨继业上)}

{杨继业

\setlength{\hangindent}{60pt}{ 【{\akai 二黄散板}】当年保驾五台山,智空长老对我言。他道我在两狼山前遭围困,到如今果应了那智空言。} }

{杨继业\hspace{20pt}~

来此不知什么所在?也不知({\akai 或}: 但不知)怎样回转大营。}

{苏武\hspace{30pt}~

嗯哼!}

{杨继业\hspace{20pt}~

看那旁有一老丈,待我上前问来。}

{杨继业\hspace{20pt}~

啊,老丈请了。}

{苏武\hspace{30pt}~

请了。({\akai 或}: 还礼。军爷敢是失迷路途?)}

{杨继业\hspace{20pt}~

请问老丈,此处什么所在?}

{苏武

({\akai 念})此处是两狼({\akai 或}: 此地是两狼),前山是我庄。虎口交牙峪,犯者一命亡。}

{(杨继业 哦。)}

{杨继业\hspace{20pt}~

啊老丈,你在此则甚呐?}

{苏武\hspace{30pt}~

(我)在此牧羊。}

{杨继业\hspace{20pt}~

诶------这样的兵荒马乱,你还牧的什么羊啊?}

{苏武

({\akai 念})(我)管他兵荒不兵荒,与我却无妨。老汉无别事,在此牧老羊。}

{杨继业\hspace{20pt}~

难道说这只老羊还有什么贵处吗?}

{苏武

({\akai 念})休说老羊无贵处({\akai 或}: 休道老羊无贵处),他的名儿天下扬。生下几个羊羔子,轰轰烈烈在世上。今朝几个死,明朝几个亡。老汉掐指算,今日死老羊。}

{苏武\hspace{30pt}~

老羊,老羊,你还不死------}

{杨继业\hspace{20pt}~

可恼!}

{杨继业

\setlength{\hangindent}{60pt}{ 【{\akai 二黄散板}】这老丈说话理不通啊,句句伤的是杨令公。手持宝刀将尔砍呐,} }

{(苏武收杨令公刀)}

{杨继业\hspace{20pt}~

\setlength{\hangindent}{60pt}{ 【{\akai 二黄散板}】霎时不见我的护身龙。 } }

{杨继业\hspace{20pt}~

且住。清风一阵,老丈不见,又将我的宝刀拿去。}

{杨继业

唉呀!有道是: (身)为大将者,宁舍千军,不舍寸铁。待我将他赶上。}

{杨继业\hspace{20pt}~

``苏武庙''------呜哙呀。}

{杨继业

想汉室苏武,乃是大大忠良,死后有人替他修庙在此。待我进去看来------({\akai 或}: 想那苏武是炎汉忠良,死后何人与他修庙在此。待我进庙看来------;想苏武乃是汉室忠良,死后有人与他修庙在此。待我进庙看来------)}

{杨继业\hspace{20pt}~

``李陵碑''------呀呀呸!}

{杨继业

想这李陵,乃是卖主求荣({\akai 或}: 想那李陵,乃是背主求荣)大大的奸佞,死后何人与他立这碑碣在此!

({\akai 或}: 想李陵乃是背主求荣,死后有人与他立这碑碣在此!)}

{杨继业\hspace{20pt}~

那旁还有几行小字,待我(上前)看来: }

{(\textless{}阴锣\textgreater{},杨继业掸土)}

{杨继业

({\akai 念})庙是苏武庙,碑是李陵碑,令公来到此,这------卸甲------又丢盔!}

{杨继业

且住!哪里是苏武庙、李陵碑,分明神明点化于我------想我被困在两狼山,内无粮草,外无救应。({\akai 念})白日受饥饿,夜晚受风吹。盼兵兵不到,这盼子({\akai 或}: 看子)------子不归。难道说,我还等到冻饿而死?!}

{杨继业

也罢------我不免拜谢宋王爵禄之恩,就碰死在李陵------碑下!}

{(杨继业

\textless{}牌子\textgreater{}令公跪倒苏武庙,喂呀------圣上啊$\cdots{}\cdots{}$李陵碑下丧黄泉({\akai 或}: 丧残生)。)}

{(杨继业碰碑死介}\footnote{ {刘曾复先生说戏时说明: 过去杨令公碰碑后上韩延寿,现一般略去。}{456}}{)}

{(韩延寿上)}

{韩延寿\hspace{20pt}~

呜哙呀!$\cdots{}\cdots{}$已死,尸首不可损坏。报与太后知道!}

{(韩延寿下)}

\newpage

\hypertarget{ux6e05ux5b98ux518c-ux4e4b-ux5bc7ux51c6}{%

\subsection{清官册 之

寇准}\label{ux6e05ux5b98ux518c-ux4e4b-ux5bc7ux51c6}}

{{[}第一场{]}}

{{[}{\akai 引子}{]}做清官民之父母,积阴功留与儿孙。}

{({\akai 念})读诗书智广才高,中皇榜青史名标。三杯御酒加封号,被权臣一本参掉。}

{下官寇准,陕西华州人氏。蒙圣恩得中一甲一名,不想被权臣参掉。是我在吏部效力三载,蒙八千岁提拔,才得职授霞峪县正堂。自到任以来,地方安定。今当三、六、九日,放告之期。左右,}

{将放告牌抬出。}

{有请。}

{万岁,}

{万万岁。}

{({\akai 念})即刻便登程。}

{转堂。}

{有请夫人。}

{夫人,请坐。}

{金牌调我连夜进京,不知为了何事。}

{但愿如此。}

{即刻启程。}

{有劳夫人。}

\setlength{\hangindent}{60pt} {【{\akai 二黄原板}】接过了夫人酒一樽,背转身来谢神灵。贤夫人请上受一礼,下官言来你试听: 高堂老母要你孝敬,早晚侍奉要殷勤。弓开难留弦上箭,}

\setlength{\hangindent}{60pt} {【{\akai 二黄摇板}】舟急哪顾岸上人。}

{[}第二场{]}

\setlength{\hangindent}{60pt} {【{\akai 二黄散板}】一路马踏芳草尽,日落西山小桃红。}

{罢了。}

{前途俱已用过。}

{今晚小心更鼓。}

{家院,四更时分,冠带伺候。}

{想我寇准,职授霞峪县令,为官以来,上不负君,下不亏民。圣上金牌调我连夜进京,不知为了何事。今晚独宿馆驿,好不愁闷人也------}

\setlength{\hangindent}{60pt} {【{\akai 二黄慢板}】一轮明月正东升,想起了高堂上老娘亲。伴君犹如羊伴虎,尽得忠啊来难把孝行。}

\setlength{\hangindent}{60pt} {【{\akai 二黄原板}】移星换斗二更时分,想起当年一举成名。八贤爷奏一本领凭上任,来至在霞峪县主管}\footnote{ 夏行涛君建议作``治管''或``执管''。{457}}{万民。早堂接状早堂审,午堂接状要审明。到晚来接下了无头冤状,一对红灯审到了天明。}

\setlength{\hangindent}{60pt} {【{\akai 二黄原板}】听谯楼打三更人烟静,一轮明月照街心。霞峪县我不曾亏负百姓,金牌调我所为何情。}

{看衣更换。}

\setlength{\hangindent}{60pt} {【{\akai 二黄原板}】顶冠束带四更尽,忙把家院叫一声: 我命你回衙报一信,你就说平安到了都城。倘若是太夫人将你问,你就说你老爷进都城、平步登云往上升,切莫要挂心。}

\setlength{\hangindent}{60pt} {【{\akai 二黄原板}】朝臣待漏五更冷,铁甲将军夜宿津。朝房鼓不住地嗵嗵打,文武百官列朝门。东华门前文官走,西华门前武将行。}

\setlength{\hangindent}{60pt} {【{\akai 二黄原板}】我寇准打从这东华门进,两旁文武着了惊。都看我七品小县令,小小前程也来见君。有才不在官职小,无才枉受爵禄恩。整装敛容丹墀进,}

\setlength{\hangindent}{60pt} {【{\akai 二黄散板}】品级台前见当今。}

{臣寇准见驾,吾皇万岁!}

{调臣进京,不知有何圣命?}

{臣启万岁,潘、杨两家,一家是当朝太师,一家是皇家郡马,臣官卑职小,难以审问。}

{谢主隆恩!}

{({\akai 念})捧旨下龙庭。}

{({\akai 念})叩见八贤君。}

{千岁在上。恕臣有王命在身,不能全礼。}

{贤爷千岁!}

{进京来了。}

{调臣进京,审问潘、杨两家之事。}

{蒙圣恩,七品县令升为西台御史。}

{千岁提拔。}

{臣却不知。}

{(惊介)有这等事?}

{待臣回复圣命。}

{多谢千岁!}

\setlength{\hangindent}{60pt} {【{\akai 二黄散板}】八千岁做了主大胆审问,哪怕那潘洪贼国戚皇亲。}

{[}第三场{]}

\setlength{\hangindent}{60pt} {【{\akai 二黄摇板}】一枝杏花香十里,状元归来马如飞。}

{供奉圣旨。}

{有请!}

{公公!}

{喜从何来?}

{公公提拔。}

{在敝衙审问。}

{(惊介)好一份厚礼呀!}

{啊,公公此礼为何?}

{王法森严,必须按律而断!}

{无功不受禄啊!}

{不敢,收下。}

{({\akai 念})王法不徇情!}

{且住!正要升堂理事,后宫潘娘娘送来一份厚礼,与老贼讲情。我若收下此礼,岂不学了前任刘御史;我若不收此礼,后宫娘娘降罪如何事好?!哎呀,这、这、这$\cdots{}\cdots{}$}

{有了,下殿之时,八千岁言道: 若有为难之处,可至南清宫领教。}

{左右,打道南清宫。}

{[}第四场{]}

\setlength{\hangindent}{60pt} {【{\akai 二黄散板}】急忙来在宫闱境,心有疑难问圣明。}

{来此宫门,待我叩环。}

{烦劳通禀: 寇准求见。}

{领旨!}

{臣寇准见驾,贤爷千岁。}

{谢座。}

{臣正要升堂理事,后宫潘娘娘送来一份厚礼,现有礼单在此,贤爷请看。}

{臣若收了此礼,岂不学了前任刘御史之故。}

{也罢,就暂寄南清宫,候事完毕,再作定夺。}

{千岁何出此言?}

{哎呀,臣要按律而断!}

{谢千岁}

{臣乃步行而来。}

{谢千岁。}

{呃,千岁在此,多有不便,将马往下带。}

{呃,方才言过,贤爷在此,多有不便,往下带,往下带。}

{哎呀!}

\setlength{\hangindent}{60pt} {【{\akai 二黄散板}】自盘古哇哪有君与臣带马呀,}

\setlength{\hangindent}{60pt} {【{\akai 二黄散板}】臣大胆跨龙驹足踏金镫,}

\setlength{\hangindent}{60pt} {【{\akai 二黄散板}】得意洋洋发笑声。}

{呵呵哈哈哈$\cdots{}\cdots{}$(笑介)}

{(掩口介)}

{[}第五场{]}

\setlength{\hangindent}{60pt} {【{\akai 二黄散板}】御史衙前下金镫,升坐大堂鬼神惊。}

{来,升堂。}

{今日升堂理事,刑具俱要齐备。}

{潘洪到此,教他报门而进。}

{潘洪,见了本御史为何不跪?}

{呵呵呵呵$\cdots{}\cdots{}$(冷笑介)}

{你欺我官卑职小。来,请过圣命!}

{潘洪: 圣旨在上,本御史在此,你怎样私通北国,苦害杨家,从实招来一一讲!}

{怎么讲?}

{潘洪,你这卖国的奸贼!}

{自古道: ({\akai 念})君待臣以礼,臣事君以忠。}

{想你身为当朝太师,一人之下,万万人之上,你是何等侥幸?谁想你这老贼心怀叵测: 命你子潘豹在天齐庙前摆下百日擂台,要将天下的英雄一网打尽,你这老贼也好谋篡社稷。}

{也是那杨老将军他的家规不严呐,那杨七将军私出府门,行至在天齐庙前,见你子潘豹在擂台之上是洋洋得意。那杨七将军性如烈火,焉能容得?上得擂台,三拳两足,将你子潘豹打死。}

{你这老贼就与那杨老将军抓袍掳带,面见当今。好一个有道明君,不加罪过,在龙楼之上,与你两家解和。谁想,你这老贼怀恨在心,修书一封,私通北国胡儿,教他们打来连环战表,夺取宋室天下。你这老贼在金殿之上,讨下帅印,单单就要那杨老将军以为前站先行。那杨老将军上殿,连辞数本,万岁不准。只得在金殿之上讨一名保官。圣上就命呼延老将军做了杨家的保官。你这老贼也要讨一名保官,想这满朝文武,谁来保你?呵,偏偏那贺朝进,与你这贼狼狈为奸!那杨老将军见势不祥,只得去到瓦桥三关,调他两个孩儿回营,共灭胡儿。}

{你这老贼,兵到雁门,升帐点卯。天气炎热,误了你的卯期,可也是有之啊。怎么,你这老贼就要将杨老将军斩首。那呼延老将军进帐讲情,你这老贼假意准情,又命人报道: 营中缺粮。想你这为元帅者,岂不知: 兵马未动,粮草先行。营中焉能缺粮?你故意命那呼延老将军催解粮草。想那呼延老将军乃是他杨家的保官,岂能替你这老贼前去催粮?本当不允,又恐违背你的将令。那呼延老将军出得大营,大笑了三声,气堵胸膛,就口吐鲜血而亡了!}

{那杨老将军见呼延老将军一死,犹如断了他杨家的命脉,就带他两子,怒出大营,不听你的调遣。你这老贼就命白牌请过了尚方宝剑,追赶他父子回营。那杨七将军性如烈火,打碎了白牌,扭断了令箭。那杨老将军,乃是知罪的臣子啊,就命他六子回营请罪。你也不管他是皇家的郡马,就一捆四十,叉出了大营。}

{黄道日期,你不准他父子出兵;黑道日期,反命他父子出马。偏偏他父子又是得胜而归,你就该大开城门,迎接他父子进城,才是你做元帅的道理呀。怎么,你反命那贺朝进带领五百名雁翎刀手,把守在雁门关,对那杨老将军言道: 必须将北国胡儿斩尽杀绝,方许进城。想那北国的胡儿犹如潮水一般,一时焉能斩得尽,杀得绝?他父子万般无奈,就杀一阵、败一阵,败一阵、杀一阵,败至在这两狼山下!}

{他父子被困在两狼山,那杨老将军就命那杨七将军回转雁门,搬兵取救。不想你这老贼想起了打子的仇恨,将他诓下马来,用酒灌醉,绑在法标之上,射了他一百单三箭!将他射死,你这打子的仇恨也就报了。怎么,你还是按兵不动呢?}

{那杨老将军只为放心不下,又命杨六将军杀出重围,探听下落。那杨老将军被困在两狼山,({\akai 念})盼兵兵不到,

望子子不归。白日受饥饿,夜晚受风吹。万般无奈,就碰死在李陵碑下!}

{那杨六将军闻得他父已死,进京告下御状。圣上命刘御史审问你这老贼,审得是不清不明,被八千岁金锏打死。万岁又发金牌连夜调本御史进京,审问你这老贼。你这老贼,为臣不能尽忠,为子不能尽孝。似你这等不忠不孝、卖国欺君,国法岂能容得?!}

\setlength{\hangindent}{60pt} {【{\akai 二黄散板}】老贼不信抬头看,本御史非比前任官。}

{来,打!}

{呀呸!}

\setlength{\hangindent}{60pt} {【{\akai 二黄散板}】皇亲国戚我不打,打的谋朝篡位臣。}

{打!}

{潘洪!万岁在这里问你,你是怎样私通北国,苦害杨家,速速招来!}

{呸!}

\setlength{\hangindent}{60pt} {【{\akai 二黄散板}】人来看过铜夹棍,看他招承不招承。}

{问他有招无招。}

{收!}

{松刑。}

{潘洪,万岁又在那里问你,你是怎样苦害杨家,按兵不动,谁与同谋?}

{呀呸。}

{(【{\akai 二黄散板}】人来看过红铁链,不招即刻赴幽冥。)}

{(哎呀!)}

{(且住,五刑用过,老贼并无半点口供,竟而气绝身亡,这、这、这$\cdots{}\cdots{}$)}

{(快些取来!)}

{(嗯------)}

{啊太师,不必如此,待下官将此事推在杨郡马身上,与太师无干就是。}

{搀了下去。}

{哎呀且住,五刑用尽,老贼并无半点口供,不免再去南清宫商议。}

{来,带马南清宫去者!}

{[}第六场{]}

{千岁不必惊慌,为臣这里还有------一张!}

\newpage

\hypertarget{ux8f95ux95e8ux65a9ux5b50}{%

\subsection{辕门斩子}\label{ux8f95ux95e8ux65a9ux5b50}}

(\textless{}\!{\bfseries\akai 大开门}\!\textgreater{}头,\textless{}\!{\bfseries\akai 急急风}\!\textgreater{}四文堂站门,孟良、焦赞上台口\textless{}\!{\bfseries\akai 四击头}\!\textgreater{}双亮相,杨延昭随上,孟、焦站两边,杨站台中间,\textless{}\!{\bfseries\akai 五击头}\!\textgreater{}念)

杨延昭

宗保犯将令,\textless{}\!{\bfseries\akai 大锣住头}\!\textgreater{}王法不徇情。

(\textless{}\!{\bfseries\akai 原场}\!\textgreater{}入大座,\textless{}\!{\bfseries\akai 乱锤}\!\textgreater{}左右望,坐下)

焦赞\hspace{30pt}~

二哥,这里来。

孟良\hspace{30pt}~

做什么?

焦赞\hspace{30pt}~

元帅今日升帐,与往日大不相同。你我要小心了。

(焦、孟到台口又回去)

杨延昭\hspace{20pt}~

焦、孟二将。

孟良、焦赞 在。

杨延昭\hspace{20pt}~

宗保到此叫他报门而进。

孟良、焦赞 啊!

焦赞\hspace{30pt}~

二哥,小本官不见,如何是好?

孟良\hspace{30pt}~

你我营门一望。

杨宗保\hspace{20pt}~

({\akai 内}白)马来!(在\textless{}\!{\bfseries\akai 小锣上}\!\textgreater{}中念)

杨宗保\hspace{20pt}~

离了穆柯寨,到此是宋营。

孟良、焦赞 小本官回来了。

杨宗保\hspace{20pt}~

我父帅可曾升帐?

焦赞、孟良 升帐多时了,元帅今日升帐,与往日大不相同,你要小心了。

杨宗保\hspace{20pt}~

如此待我转去。

焦赞\hspace{30pt}~

啊,大丈夫只有向前,哪有退后之理?二哥,与他报门。

孟良\hspace{30pt}~

小本官告进。

(宗保进帐,面里跪)

孟良、焦赞 宗保当面。

杨延昭\hspace{20pt}~

儿是宗保?({\akai 或}: 下跪可是宗保?)

杨宗保\hspace{20pt}~

正是。

杨延昭\hspace{20pt}~

焦、孟二将,张起面来,儿是宗保。

杨宗保\hspace{20pt}~

正是。

杨延昭\hspace{20pt}~

唗,

杨延昭

\setlength{\hangindent}{60pt}{ 【{\akai 西皮散板}】怒恼杨延昭,奴才听根苗。我命儿去巡哨,私自把亲招。枪挑穆天王,桂英下山巢。将父擒马下,笑,笑坏了众英豪。焦、孟二将一声叫,将奴才绑辕门定斩不饶。 }

(宗保绑坐大边台口,文堂下,焦赞招孟良)

焦赞\hspace{30pt}~

二哥,这里来。

孟良\hspace{30pt}~

做什么?

焦赞\hspace{30pt}~

小本官犯罪,都在你我弟兄的身上,必须进帐讲个人情。

孟良\hspace{30pt}~

这个人情,只怕讲不下来。

焦赞\hspace{30pt}~

嘿,元帅喜欢的就是我。。

孟良\hspace{30pt}~

一同进帐。

(二人进帐,左右面里跪)

孟良、焦赞 与元帅叩头!

杨延昭\hspace{20pt}~

你二人进帐何事?

焦赞

小本官犯罪,念在我弟兄二人,鞍前马后,小小的功劳,望求元帅开恩饶恕。

杨延昭\hspace{20pt}~

敢是与宗保讲情?

孟良、焦赞 不敢,元帅开恩。

杨延昭\hspace{20pt}~

哈哈哈$\cdots{}\cdots{}$(冷笑介)

(二人站)

焦赞\hspace{30pt}~

行,有门儿。

杨延昭\hspace{20pt}~

唗!

(二人又跪)

杨延昭

宗保犯罪({\akai 或}: 犯法),俱是({\akai 或}: 皆是)你二人的引诱,先斩宗保,然后再取你二人的黑头。

孟良、焦赞 (哎呀,)喳喳喳!

(二人起立出帐)

孟良\hspace{30pt}~

我说这个人情讲不下来,你说有你。

焦赞\hspace{30pt}~

你说有你。

孟良\hspace{30pt}~

呸,着打。

焦赞\hspace{30pt}~

二哥,你别着急。你看好了小本官,我去搬老太太去。

(焦赞下,搀佘太君拄拐杖上)

佘太君

\textless{}\!{\bfseries\akai 慢长锤}\!\textgreater{}【{\akai 西皮散板}】听说是斩宗保把我吓坏,险些儿一步跌倒尘埃。又只见小孙儿捆绑营外(过大边),因甚事犯将令要把刀开?

杨宗保\hspace{20pt}~

(接唱)都只为招亲在穆柯山寨,因此上绑辕门要把刀开。

佘太君

(接唱)小孙儿免悲声休要急坏,待祖母进帐去讲情来。叫焦赞你与我前把路带,

(焦赞引佘太君进帐,坐小边侧椅)

佘太君\hspace{20pt}~

(接唱)大胆的杨延昭不接娘来。

孟良\hspace{30pt}~

太君到。

杨延昭\hspace{20pt}~

\setlength{\hangindent}{60pt}{ 【{\akai 西皮导板}】恨宗保犯将令捆绑帐外。 }

焦赞\hspace{30pt}~

太君到。

(\textless{}\!{\bfseries\akai 四击头}\!\textgreater{}亮相)

孟良\hspace{30pt}~

元帅他知道啦!

焦赞\hspace{30pt}~

为的叫他知道。

杨延昭

\setlength{\hangindent}{60pt}{ 【{\akai 西皮原板}】不由人一阵阵怒满胸怀。(出帐)忽听报老娘亲驾到帐外,杨延昭下位去迎接娘来。见老娘施一礼躬身下拜, }

佘太君\hspace{20pt}~

不消。

焦赞\hspace{30pt}~

不消。

杨延昭\hspace{20pt}~

(接唱)问老娘因何故愁眉不开?

佘太君\hspace{20pt}~

\setlength{\hangindent}{60pt}{ 【{\akai 西皮原板}】娘进帐我的儿早已知解,还把这殷勤话问娘何来? }

杨延昭

\setlength{\hangindent}{60pt}{ 【{\akai 西皮原板}】老娘亲坐虎堂怒冲天界,莫不是为宗保不肖\footnote{ 《京剧新序》作``不孝''。{458}}奴才? }

佘太君\hspace{20pt}~

\setlength{\hangindent}{60pt}{ 【{\akai 西皮原板}】小孙儿他犯了何条律戒,因甚事绑在辕门要把刀开? }

杨延昭

\setlength{\hangindent}{60pt}{ 【{\akai 西皮原板}】提起来小宗保({\akai 或}: 提起来这奴才)将儿气坏,恨不得将奴才斧斫刀开。儿命他领人马巡查边塞,到山东穆柯寨配了裙钗。临阵上去招亲国法何在,问老娘儿斩他该是不该? }

佘太君\hspace{20pt}~

\setlength{\hangindent}{60pt}{ 【{\akai 西皮原板}】犯将令理应该斩首营外, }

杨延昭\hspace{20pt}~

谢母亲。

佘太君\hspace{20pt}~

且慢。

焦赞\hspace{30pt}~

且慢。

佘太君\hspace{20pt}~

(接唱)还看他年幼小无知婴孩。

杨延昭

(接唱)【{\akai 西皮原板}】娘道他年纪小孩童气概,【{\footnotesize 转}{\akai 西皮快板}】讲几个年幼人娘且听来: 秦甘罗十二岁身为太宰,石敬瑭十三岁拜将登台;三国中周公瑾名扬四海,七岁上学兵法人称将才,在赤壁用火攻鬼神难解,烧曹兵八十万无处葬埋。这都是父母生非仙下界({\akai 或}: 非神下界),难道说小宗保他不是娘怀?

佘太君

\setlength{\hangindent}{60pt}{ 【{\akai 西皮快板}】听罢言不由娘牙根咬坏,骂一声杨延昭不肖奴才: 父子们投宋君名扬四海,弟兄们一个个俱是将才。到如今剩宗保一脉后代,眼睁睁还要他祭扫坟台。倘若是小孙儿有个好歹,那时节管叫儿悔不及来。 }

杨延昭

\setlength{\hangindent}{60pt}{ 【{\akai 西皮快板}】昨日里斩八将头悬帐外,老娘亲怎不把慈悲来开。今日里斩宗保娘把儿怪,哭啼啼坐宝帐所为何来。叫焦赞呐 }

【{\footnotesize 转}{\akai 西皮摇板}】将宝剑悬挂营外,

(焦赞取剑,挂剑介)

杨延昭\hspace{20pt}~

(接唱)老娘亲再讲情儿自刎头来。

(杨进帐可以下场\footnote{ 陈超老师注: 如杨延昭``三换衣''就是此处换红蟒(之前穿绿蟒),``自刎头来''再下。{459}},八贤王进帐时再上,也可以不下场坐大座,焦赞向孟良示请八贤王,上场门下)

佘太君

\textless{}\!{\bfseries\akai 慢长锤}\!\textgreater{}【{\akai 西皮摇板}】杨延昭性倔强令人可恼,他把我年迈人不放在心梢,\textless{}\!{\bfseries\akai 哭头}\!\textgreater{}眼睁睁小孙儿性命难保。

(孟良扶太君下,焦赞牵马引赵德芳上)

赵德芳\hspace{20pt}~

\setlength{\hangindent}{60pt}{ 【{\akai 西皮摇板}】赵德芳下龙驹来到法标。 }

(赵德芳下马,过大边,焦赞接马鞭放后场,后边不用刖马足)

赵德芳\hspace{20pt}~

(接唱)御外男把什么军令犯了,因何故绑辕门项上加刀?

杨宗保\hspace{20pt}~

(接唱)都只为招亲事将令犯了,因此上绑辕门问罪开刀。

赵德芳

(接唱)我道是犯了那何条令号,却原来为的是私把亲招。叫焦赞你与我上前通报,

(焦赞、赵德芳进帐,赵坐小边)

赵德芳\hspace{20pt}~

(接唱)杨延昭不下位藐视当朝。

孟良\hspace{30pt}~

贤爷到。\footnote{

刘曾复先生为樊百乐君说戏时说明: 如``三换衣'',此处杨延昭穿白蟒暗上。{460}}

杨延昭\hspace{20pt}~

\setlength{\hangindent}{60pt}{ 【{\akai 西皮导板}】耳边厢又听得贤爷驾到。 }

焦赞\hspace{30pt}~

贤爷到。

(焦赞亮相)

孟良\hspace{30pt}~

元帅早就知道啦!

焦赞\hspace{30pt}~

为的叫他知道哇。

杨延昭

\setlength{\hangindent}{60pt}{ 【{\akai 西皮原板}】无故的离龙朝事有蹊跷。(出位)撩蟒袍端玉带整整纱帽,见过了宋天子裔派根苗。走上前({\akai 或}: 走向前)施一礼扬尘舞蹈, }

赵德芳\hspace{20pt}~

平身。

焦赞\hspace{30pt}~

平身。

杨延昭\hspace{20pt}~

(接唱)恕为臣接驾迟贤爷恕饶。

赵德芳

(接唱)赵德芳坐虎堂满脸陪笑,尊一声御妹夫细听根苗: 御外男他正在英雄年少,绑辕门犯的是军令何条?

杨延昭

(接唱)八千岁来查询此事非小,提起来不由臣({\akai 或}: 提起来这奴才)怒气难消。臣命他领将令巡营瞭哨,又谁知这奴才临阵脱逃。绑辕门皆因是不听臣教,斩宗保为的是私把亲招。

赵德芳\hspace{20pt}~

(接唱)临阵上招亲事理当斩了,

杨延昭\hspace{20pt}~

谢千岁。

赵德芳\hspace{20pt}~

且慢。

焦赞\hspace{30pt}~

且慢。

赵德芳\hspace{20pt}~

(接唱)还看在本御面将他恕饶。

杨延昭

(接唱)君有命臣当领怎敢违矫,哪有个为臣的不顺当朝,赦却了小宗保事倒还小,怕只怕宋天子斩杀不饶。

赵德芳\hspace{20pt}~

(接唱)漫说是我叔王圣旨来到,五阎君要性命本御承招。

杨延昭\hspace{20pt}~

(接唱)八千岁休把臣轻视小藐,

\setlength{\hangindent}{60pt}{ 【{\footnotesize 转}{\akai 西皮快板}】休得要在虎堂絮絮叨叨,斩不斩犯的是杨家令号,并不曾({\akai 或}: 曾不曾)犯千岁这半点律条。 }

赵德芳

\setlength{\hangindent}{60pt}{ 【{\akai 西皮快板}】听一言不由孤心头懊恼,叫一声杨元帅细听根苗: 曾记得你七弟打死潘豹,潘仁美哭啼啼启奏当朝。我叔王龙颜怒降旨一道,要将你一满门问罪开刀,那时节不是我({\akai 或}: 若不是为王的)把本来保,到如今你焉能够身挂紫袍。 }

杨延昭

\setlength{\hangindent}{60pt}{ 【{\akai 西皮快板}】曾记得天庆王打来战表,他要夺吾主爷锦绣龙朝。我大哥赴双龙({\akai 或}: 我大哥替宋王;我大哥与宋王)把忠尽了,我二哥短剑下命赴阴曹;我三哥被马踏尸骨难找,我四哥失番邦无有下梢;我五哥在五台削发修道,我七弟被潘洪箭射法标({\akai 或}: 箭射芭蕉)。我八弟被贼擒生死不晓,一家人好一似雁遇翎雕,这都是我杨家尽忠报效,凭功劳挣下了乌纱蟒袍({\akai 或}: 玉带蟒袍\footnote{ 据吴焕老师告知,``玉带蟒袍''也可作``玉带紫袍''。{461}})。 }

赵德芳

\setlength{\hangindent}{60pt}{ 【{\akai 西皮快板}】臣有功君有赏顺天行道,并不曾亏负你半点功劳。到如今做高官把前情忘了,看起来你是个无义儿曹\footnote{ 《京剧新序》误作``无义人曹''。{462}}。 }

杨延昭

\setlength{\hangindent}{60pt}{ 【{\akai 西皮摇板}】既然是南清宫势力雄浩,又何须杨家将扶保宋朝,杨延昭在虎堂帅印交了哇, }

(杨延昭取印交赵德芳,焦赞接印放回桌上)

杨延昭\hspace{20pt}~

(接唱)破天门保宋室千岁承招。

(杨延昭下)

赵德芳

\setlength{\hangindent}{60pt}{ 【{\akai 西皮摇板}】杨延昭交帅印孤心害怕({\akai 或}: 王心害怕),破天门保宋室还要仗他。\textless{}\!{\bfseries\akai 哭头}\!\textgreater{}眼见得御外男难以救下。 }

(赵德芳下,孟、焦面里分坐大帐子椅,穆瓜引穆桂英上)

穆桂英

\textless{}\!{\bfseries\akai 慢长锤}\!\textgreater{}【{\akai 西皮摇板}】穆家寨又来了女将娇娃。({东东仓},{东东仓})(\textless{}\!{\bfseries\akai 五锣三鼓}\!\textgreater{}的前\textless{}\!{\bfseries\akai 四鼓二锣}\!\textgreater{})

穆桂英

(接唱)耳边厢又听得锣鸣鼓打,辕门外列刀枪剑戟如麻。叫穆瓜你与我上前问话,是斩兵是斩将细问根芽。

穆瓜

(接唱)我姑娘她把那将令传下,宋营中来了我大将穆瓜。我这里走向前用目来洒,(白)哎呀,不好了。

(穆瓜向穆桂英唱)

穆瓜\hspace{30pt}~

\setlength{\hangindent}{60pt}{ 【{\akai 西皮摇板}】辕门外绑的是我家姑爷,你的他。 }

穆桂英\hspace{20pt}~

\setlength{\hangindent}{60pt}{ 【{\akai 西皮摇板}】听一言我这里急忙下马, }

(穆桂英下马过大边)

穆桂英\hspace{20pt}~

将军,

穆桂英\hspace{20pt}~

\setlength{\hangindent}{60pt}{ 【{\akai 西皮摇板}】叫一声杨将军恩爱冤家。 }

穆桂英\hspace{20pt}~

将军,将军呐。

穆桂英

(接唱)我这里叫将军他不应答,想必是为招亲犯了王法,整一整青丝鬓紧紧铠甲,

(回身唱)

穆桂英\hspace{20pt}~

\setlength{\hangindent}{60pt}{ 【{\akai 西皮摇板}】转面来再吩咐家将穆瓜。 }

穆瓜\hspace{30pt}~

三千担干粮草,

穆桂英\hspace{20pt}~

\setlength{\hangindent}{60pt}{ 【{\akai 西皮摇板}】堆在帐下。 }

穆瓜\hspace{30pt}~

五百名家丁,

穆桂英\hspace{20pt}~

\setlength{\hangindent}{60pt}{ 【{\akai 西皮摇板}】四下安扎。 }

穆瓜\hspace{30pt}~

降龙木,

穆桂英\hspace{20pt}~

\setlength{\hangindent}{60pt}{ 【{\akai 西皮摇板}】交与我。 }

穆瓜\hspace{30pt}~

我,

穆桂英\hspace{20pt}~

\setlength{\hangindent}{60pt}{ 【{\akai 西皮摇板}】你且退下。 }

穆瓜\hspace{30pt}~

喳。

(穆瓜交棍给穆桂英,上场门下,英进帐面向外中间跪,棍放面前地下)

穆桂英\hspace{20pt}~

\setlength{\hangindent}{60pt}{ 【{\akai 西皮摇板}】他那里问一声再把话答。 }

(杨延昭暗上大座睡介,孟良、焦赞两边站起来,转身,见穆桂英)

孟良、焦赞 女将跪帐。

杨延昭

\setlength{\hangindent}{60pt}{ 【{\akai 西皮原板}】适才间与贤爷帐中叙话,只气得杨延昭咬碎银牙。睁开了杀人眼观看帐下呀, }

焦赞\hspace{30pt}~

女将跪帐。

杨延昭\hspace{20pt}~

(接唱)白虎堂跪定了(是)女将娇娃。

杨延昭\hspace{20pt}~

焦赞!

焦赞\hspace{30pt}~

在。

杨延昭

(接唱)叫焦赞你与我将她的名姓留下({\akai 或}: 教焦赞你与我上前问话),谁家女哪家眷哪里有家(接\textless{}\!{\bfseries\akai 行弦}\!\textgreater{})。

焦赞\hspace{30pt}~

那一女子家住哪里,姓甚名谁,你一一讲来。

穆桂英\hspace{20pt}~

听了。

焦赞\hspace{30pt}~

讲啊。

穆桂英\hspace{20pt}~

听了。

穆桂英\hspace{20pt}~

\setlength{\hangindent}{60pt}{ 【{\akai 西皮原板}】家住在山东地穆家山下,我就是穆桂英来到夫家。 }

焦赞

啊,元帅,她是穆桂英,她$\cdots{}\cdots{}$来了,哇呀呀$\cdots{}\cdots{}$

(孟良、焦赞后退,杨延昭站椅左侧)

杨延昭\hspace{20pt}~

\setlength{\hangindent}{60pt}{ 【{\akai 西皮原板}】听说是穆桂英不由我(的)心中害怕, }

焦赞\hspace{30pt}~

二哥,元帅怎么嗓子眼小了?

杨延昭\hspace{20pt}~

\setlength{\hangindent}{60pt}{ 【{\akai 西皮原板}】宋营中来了个杀人的夜叉。 }

(杨延昭大座,孟良、焦赞原位站)

杨延昭\hspace{20pt}~

焦赞!

杨延昭

\setlength{\hangindent}{60pt}{ 【{\akai 西皮原板}】问小姐不在那青山({\akai 或}: 山东)潇洒,来到我宋营中有何话答(接\textless{}\!{\bfseries\akai 行弦}\!\textgreater{})? }

焦赞\hspace{30pt}~

喳,穆小姐,我家元帅问你,不在山东穆柯寨,来到这宋营中做甚呢?

穆桂英\hspace{20pt}~

听了。

焦赞\hspace{30pt}~

嗯。

穆桂英

\setlength{\hangindent}{60pt}{ 【{\akai 西皮原板}】三千担干粮草辕门堆下,五百名勇家丁四下安扎。随带来降龙木真宝非假,特地里到宋营中献与皇家\textless{}\!{\bfseries\akai 行弦}\!\textgreater{}。 }

焦赞\hspace{30pt}~

啊元帅,她是献宝来的。

杨延昭\hspace{20pt}~

哦!

杨延昭\hspace{20pt}~

\setlength{\hangindent}{60pt}{ 【{\akai 西皮二六}】听罢言来笑开怀, }

(\textless{}\!{\bfseries\akai 九锤半}\!\textgreater{},焦赞举帅印笑,杨延昭笑)

杨延昭\hspace{20pt}~

\setlength{\hangindent}{60pt}{ 【{\akai 西皮摇板}】焦赞将宝呈上来。 }

(\textless{}\!{\bfseries\akai 紧锤}\!\textgreater{}焦赞从穆桂英手中接棍,耍棍花,将棍一头递杨延昭,杨立接唱)

杨延昭

\setlength{\hangindent}{60pt}{ 【{\akai 西皮快板}】我为你终日愁眉带,为你四路把兵排,为你山东把兵败,为你要斩小奴才。焦赞将宝后营摆, }

(\textless{}\!{\bfseries\akai 快长锤}\!\textgreater{}杨延昭交棍给焦赞,焦耍棍花,扛棍下,放棍再上)

杨延昭\hspace{20pt}~

\setlength{\hangindent}{60pt}{ 【{\akai 西皮摇板}】等候五哥下山来。 }

焦赞\hspace{30pt}~

(\textless{}\!{\bfseries\akai 行弦}\!\textgreater{}中念)三千担粮草,

杨延昭\hspace{20pt}~

\setlength{\hangindent}{60pt}{ 【{\akai 西皮摇板}】三千担干粮草堆积帐外({\akai 或}: 堆积营外)。 }

焦赞\hspace{30pt}~

(\textless{}\!{\bfseries\akai 行弦}\!\textgreater{}中念)五百名勇家丁,

杨延昭\hspace{20pt}~

\setlength{\hangindent}{60pt}{ 【{\akai 西皮摇板}】五百名勇家丁改换腰牌, }

焦赞\hspace{30pt}~

(\textless{}\!{\bfseries\akai 行弦}\!\textgreater{}中念)那穆小姐?

杨延昭\hspace{20pt}~

\setlength{\hangindent}{60pt}{ 【{\akai 西皮摇板}】穆小姐暂回转穆柯山寨。 }

焦赞\hspace{30pt}~

(\textless{}\!{\bfseries\akai 行弦}\!\textgreater{}中念)那进宝的功?

杨延昭

\setlength{\hangindent}{60pt}{ 【{\akai 西皮摇板}】奏明了宋天子再接进营来\textless{}\!{\bfseries\akai 行弦}\!\textgreater{}。 }

焦赞\hspace{30pt}~

啊,穆小姐,我家元帅言道,请小姐暂回山寨,奏明天子再将你接进营来。

穆桂英\hspace{20pt}~

我还有话讲。

焦赞\hspace{30pt}~

元帅还有话讲(女声),她说的。

穆桂英

\setlength{\hangindent}{60pt}{ 【{\akai 西皮摇板}】小将军犯的是何条军法,为什么绑辕门要把头杀\textless{}\!{\bfseries\akai 行弦}\!\textgreater{}? }

焦赞\hspace{30pt}~

她是为小本官来的。

杨延昭\hspace{20pt}~

嗯。

杨延昭

\setlength{\hangindent}{60pt}{ 【{\akai 西皮摇板}】斩宗保为的是违令犯法,劝小姐这件事休要管它\textless{}\!{\bfseries\akai 行弦}\!\textgreater{}。 }

焦赞\hspace{30pt}~

元帅说,小本官犯了军令,叫你少管闲事。

穆桂英\hspace{20pt}~

还有话讲。

穆桂英

\setlength{\hangindent}{60pt}{ 【{\akai 西皮摇板}】小将军虽然是犯了军法,还念在进宝功饶恕于他\textless{}\!{\bfseries\akai 行弦}\!\textgreater{}。 }

焦赞\hspace{30pt}~

着,元帅要念在她进宝的功啊!

杨延昭\hspace{20pt}~

嗯。

杨延昭

\setlength{\hangindent}{60pt}{ 【{\akai 西皮摇板}】若不念投帐前进宝(的)功大,定将她斩辕门血染黄沙\textless{}\!{\bfseries\akai 行弦}\!\textgreater{}。 }

焦赞\hspace{30pt}~

(接唱)我去问她,一个俩仨。

焦赞

啊小姐,元帅言道若不念进宝有功,连你一块儿杀,我瞧这个事儿,光说不行,干脆,拿这个家伙(指剑)吓唬吓唬他\textless{}\!{\bfseries\akai 行弦}\!\textgreater{}。

穆桂英\hspace{20pt}~

嗳!

穆桂英\hspace{20pt}~

\setlength{\hangindent}{60pt}{ 【{\akai 西皮摇板}】老元戎若不把人情准下, }

穆桂英\hspace{20pt}~

罢,

穆桂英\hspace{20pt}~

\setlength{\hangindent}{60pt}{ 【{\akai 西皮摇板}】宋营中杀一个寸草无芽。 }

(穆桂英跪回身拔剑向大帐砍,孟良、焦赞拔腰刀架,杨延昭离位退椅后躲惊介)

杨延昭\hspace{20pt}~

\setlength{\hangindent}{60pt}{ 【{\akai 西皮摇板}】这女将赛煞神凭空降下呀, }

焦赞\hspace{30pt}~

招架不住了。

(穆桂英、孟良、焦赞撤剑、刀)

杨延昭

\setlength{\hangindent}{60pt}{ 【{\akai 西皮摇板}】似猛虎恶狠狠舞爪张牙,赦却了小宗保事倒也罢,天门阵有何人前去征杀({\akai 或}: 前去厮杀)? }

(杨延昭大座)

焦赞

(\textless{}\!{\bfseries\akai 行弦}\!\textgreater{}中白)啊穆小姐,我家元帅言道,若是将你人情准下,天门阵哪个去杀?

穆桂英\hspace{20pt}~

听了。

穆桂英

\setlength{\hangindent}{60pt}{ 【{\akai 西皮摇板}】老元戎他若是人情准下。天门阵自有我前去征杀\textless{}\!{\bfseries\akai 行弦}\!\textgreater{}。 }

焦赞\hspace{30pt}~

元帅,穆小姐言道,若是赦了小本官,天门阵有她去征杀。

杨延昭

\setlength{\hangindent}{60pt}{ 【{\akai 西皮摇板}】萧天佐摆天门阵人人惊怕,排天罡列地煞一百单八\textless{}\!{\bfseries\akai 行弦}\!\textgreater{}。 }

焦赞\hspace{30pt}~

(接唱)哎,我再问她。

焦赞\hspace{30pt}~

啊穆小姐,天门阵一百单八难道阵阵你都能杀?

穆桂英\hspace{20pt}~

听了。

穆桂英

\setlength{\hangindent}{60pt}{ 【{\akai 西皮摇板}】一千阵、一万阵何足惊怕,何况那天门阵才一百单八\textless{}\!{\bfseries\akai 行弦}\!\textgreater{}。 }

焦赞\hspace{30pt}~

嘿,真是好的。啊元帅,一千阵,一万阵她都能去杀。

杨延昭\hspace{20pt}~

哦,(怎么)她都能去杀。

焦赞\hspace{30pt}~

哎,她都能。

杨延昭\hspace{20pt}~

焦赞,你呢?

焦赞\hspace{30pt}~

我,哎我饭桶,那么元帅您呐?

杨延昭\hspace{20pt}~

嗯!

焦赞\hspace{30pt}~

这可就瞧您的啦。

杨延昭

\setlength{\hangindent}{60pt}{ 【{\akai 西皮摇板}】这女将在帐中夸口甚大\footnote{ 刘(鸿昇)派、高(庆奎)派此处有杨延昭``看天书''的唱法,兹照录如下:  \begin{quote}  杨延昭   }

【{\akai 西皮快板}】穆桂英在帐中夸口甚大,我这里取天书仔细观察。看一看九曜星临凡托化,或生男或生女报效皇家。玉皇差天魔女把凡来下,托化了穆桂英不错不差。  \end{quote}  (焦赞 你看这上头有穆桂英。)  杨延昭 

【{\akai 西皮摇板}】穆桂英,我那聪明伶俐的儿啊,  \begin{quote}  杨延昭  

【{\akai 西皮快板}】你不该将为父枪挑马下,宋营中大小将活活笑煞。罢罢罢将人情暂且准下,(坐下)念兹在进宝的功饶恕这冤家。  \end{quote}  {463}},投宋营献降龙({\akai 或}: 献降龙投宋营)报效皇家。非是我废公议把私情准下,

(杨延昭出位,焦赞挪座,杨正面小座)

杨延昭\hspace{20pt}~

\setlength{\hangindent}{60pt}{ 【{\akai 西皮摇板}】破天门难得这女将娇娃。 }

杨延昭\hspace{20pt}~

宗保赦回。({\akai 或}: 解下桩来。)

焦赞\hspace{30pt}~

元帅赦了小本官了。

穆桂英\hspace{20pt}~

\setlength{\hangindent}{60pt}{ 【{\akai 西皮摇板}】谢过了老元戎人情准下, }

(穆桂英起立)

焦赞\hspace{30pt}~

二哥快给小本官松绑。

(穆桂英拦,孟、焦退)

穆桂英\hspace{20pt}~

\setlength{\hangindent}{60pt}{ 【{\akai 西皮摇板}】拔宝剑吓退了黑红二煞。 }

(穆桂英用宝剑给宗保松绑,推宗保下,焦赞招孟良到台口中间)

焦赞\hspace{30pt}~

二哥,元帅有四字不周全。

孟良\hspace{30pt}~

哪四字不周全?

焦赞\hspace{30pt}~

不忠不孝不仁不义,你我不要管他的闲事,随我后帐饮酒去呀。

(杨延昭偷听,焦赞拉孟良,孟教杨跟焦后面走)

杨延昭\hspace{20pt}~

焦赞,哪里去吃酒哇?

焦赞\hspace{30pt}~

二哥你怎么也赚开我啦?

(焦、孟两边站,杨中间)

(杨延昭\hspace{20pt}~

唉!)

杨延昭

\setlength{\hangindent}{60pt}{ 【{\akai 西皮摇板}】他二人在帐中背地叙话,倒教我在一旁难把话答,不忠孝、不仁义人人笑骂,到此时我只得装聋作哑({\akai 或}: 我只得佯装聋哑)。 }

(杨延昭正面小座)

杨延昭

焦、孟二将,本帅收了穆桂英,赦了杨宗保,有保状者({\akai 或}: 有保状的)呈了上来。

孟良、焦赞 元帅收了穆桂英,赦了杨宗保,有保状者呈上。

({\akai 内}白:\hspace{30pt}~

贤爷的保状,太君的保状,满营将官的保状。)

(孟良、焦赞两边从后场接保状,每人两份帖,回身见杨延昭,交帖)

孟良\hspace{30pt}~

贤爷保状。

焦赞\hspace{30pt}~

太君保状。

孟良\hspace{30pt}~

满营将官的保状。

焦赞\hspace{30pt}~

这是我弟兄二人的小小帖儿。

(杨延昭分别接帖)

杨延昭\hspace{20pt}~

(啊,呵呵)哈哈哈$\cdots{}\cdots{}$

杨延昭\hspace{20pt}~

\setlength{\hangindent}{60pt}{ 【{\akai 西皮摇板}】叫焦赞将保状龙棚张挂,每日里焚清香供奉于它。 }

(焦赞持帖下又上)

杨延昭

\setlength{\hangindent}{60pt}{ 【{\akai 西皮摇板}】叫孟良传宗保({\akai 或}: 叫孟良唤宗保)速到帐下,待本帅亲自里教训与他。 }

(孟良向下场唤)

孟良\hspace{30pt}~

宗保进见。

(穆桂英推宗保下场门上)

杨宗保\hspace{20pt}~

\setlength{\hangindent}{60pt}{ 【{\akai 西皮摇板}】我这里进宝帐心中害怕, }

(穆桂英推宗保进帐,教他勿怕。宗保进帐跪大边一侧)

杨宗保\hspace{20pt}~

\setlength{\hangindent}{60pt}{ 【{\akai 西皮摇板}】从今后儿不敢再犯王法。 }

杨延昭

\setlength{\hangindent}{60pt}{ 【{\akai 西皮摇板}】从今后儿必须奉公守法,学一个奇男子名扬天涯。我这里将奴才踏至帐下({\akai 或}: 我这里将奴才一足来踏)。 }

(杨延昭欲踏,焦、孟拦,宗保坐,穆桂英搀宗保起)

穆桂英\hspace{20pt}~

\setlength{\hangindent}{60pt}{ 【{\akai 西皮摇板}】待等到破天门还要用他。 }

(穆桂英推宗保下)

杨延昭

\setlength{\hangindent}{60pt}{ 【{\akai 西皮摇板}】明日里南清宫把罪来请,到后营见老娘去献殷勤,扫将台准备着与贼会阵, }

(孟良、焦赞左右翻下,杨延昭站,到大边收腿)

杨延昭\hspace{20pt}~

\setlength{\hangindent}{60pt}{ 【{\akai 西皮散板}】等五哥下山林好破天门。 }

(\textless{}\!{\bfseries\akai 尾声}\!\textgreater{}下)

\newpage

\hypertarget{ux63a2ux6bcdux56deux4ee4-ux4e4b-ux6768ux5ef6ux8f89}{%

\subsection{\texorpdfstring{探母回令\footnote{ 陈超老师注: 谭鑫培的《四郎探母》一剧是与王君直交流最多的,并听取王君直的建议做了不少修改,做了很多独具匠心的身段设计。{464}}

之

杨延辉}{探母回令464 之 杨延辉}}\label{ux63a2ux6bcdux56deux4ee4-ux4e4b-ux6768ux5ef6ux8f89}}

{{[}第一场{]}}

(\textless{}\!{\bfseries\akai 小锣打上}\!\textgreater{}杨延辉上)

{[}{\akai 引子}{]}金井锁梧桐,长叹声随一阵风。

({\akai 念})失落番邦十五年,雁隔衡阳\footnote{ 李舒先生遗作《涉艺所得》录《刘曾复修润剧本四篇》中《\textless{}四郎探母\textgreater{}及其他》一文中此处作``雁断衡阳'';吴焕老师整理本作``雁过衡阳''。{465}}又一天,思想老母难得见,怎不教人泪涟涟。

本宫,四郎延辉,我父金刀令公,我母佘氏太君。只因十五年前沙滩大会,本宫被擒。多蒙(萧)太后不斩,反将公主匹配。

昨日韩昌奏道: 萧天佐,在九龙飞虎峪,摆下天门大阵,老娘亲统大兵,来到雁门,我有心回转宋营,见母一面,怎奈关津阻隔,插翅难飞({\akai 或}: 插翅难过)。

思想起来,好不伤感,唉,人也呀!呃$\cdots{}\cdots{}$(哭介)

\setlength{\hangindent}{60pt}{ 【{\akai 西皮慢板}】杨延辉坐宫院自思自叹,想起了当年事好不惨然。我好比笼中鸟有翅难展,我好比虎离山受了孤单。我好比南来雁失群飞散,我好比浅水龙困在沙滩。想当年双龙会 }

【{\footnotesize 转}{\akai 西皮二六}】一场血战,只杀得血成河尸骨堆山。只杀得杨家将东逃西散,只杀得众儿郎滚下马鞍。我被擒在番营身脱此难,将楊字拆木易匹配良缘。萧天佐摆天门两下会战,我的娘领人马来到北番。我有心到宋营见母一面,怎奈我无令箭不能出关。九龙峪离幽州相隔不远,看起来好一似万重高山。

\setlength{\hangindent}{60pt}{ 【{\akai 西皮摇板}】眼睁睁高堂母不能\textless{}\!{\bfseries\akai 哭头}\!\textgreater{}见,儿的老娘啊! }

\setlength{\hangindent}{60pt}{ 【{\akai 西皮摇板}】要相逢除非是梦里团圆。 }

公主来了?请坐!

免。

本宫无有心事,公主不要多疑。

这$\cdots{}\cdots{}$

本宫心事却有,慢说公主,就是大罗神仙,难以知觉。

要猜呢?

好,今日闲暇无事,我们就猜上一猜。

请------

啊公主,你这头一猜呀$\cdots{}\cdots{}$

就猜错了!

想太后,乃一国之主,慢说没有怠慢,纵有怠慢,还把她老人家怎么样吗?

着啊!

不是的。

公主,你又猜错了!

想你我夫妻,相亲相爱,说什么冷落寡欢。

(呃,)也不是的。

那秦楼楚馆是甚等地方,难道说还胜得过这皇宫内院不成?

越发的不对了!

哎呀公主啊!本宫方才言过,你我夫妻,相亲相爱,一十五载。况且又与我生下了后代,说什么(怀)抱琵琶,另想别弹。你说此话呀,唉!岂不屈煞本宫啊,呃$\cdots{}\cdots{}$(哭介)

哦!

\setlength{\hangindent}{60pt}{ 【{\akai 西皮快板}】好一个贤公主智谋广远,猜透了杨延辉袖内机关。我本当向前去求她方便, }

\setlength{\hangindent}{60pt}{ 【{\akai 西皮摇板}】还须要紧闭口({\akai 或}: 还须要谨开口)慢漏真言。 }

心事却被公主猜透({\akai 或}: 被公主猜破),不能与本宫作主,也是枉然。

公主啊!

\setlength{\hangindent}{60pt}{ 【{\akai 西皮快板}】我在南来你在番,千里的姻缘一线牵。公主对天盟誓愿,本宫方肯吐真言。 }

正是!

怎么,番邦女子连誓都不会盟么?

待本宫教导于你呀。

来来来,跪在尘埃,口称: 皇天在上,番邦女子在下,驸马爷对我说了真情实话,日后走漏消息半点,天把我怎长,地把我怎短!

唉!要你终生大誓,对天一表!

言重了。

\setlength{\hangindent}{60pt}{ 【{\akai 西皮快板}】一见公主盟誓愿,本宫才把心放宽。二次向前重把礼见, }

\setlength{\hangindent}{60pt}{ 【{\akai 西皮摇板}】我方能到宋营见母问安。 }

(啊)公主,你道本宫真姓木名易么$\cdots{}\cdots{}$

唉,非也!

哎呀!

\setlength{\hangindent}{60pt}{ 【{\akai 西皮导板}】未开言不由人泪流满面, }

啊,公主,你怎么在阿哥身上打搅啊?

唉!公主啊,呃$\cdots{}\cdots{}$(哭介)

\setlength{\hangindent}{60pt}{ 【{\akai 西皮原板}】贤公主细听我表一表家园({\akai 或}: 表叙家园): 我的父老令公官高爵显,我的母佘太君所生我弟兄七男。都只为宋王爷五台香拈,潘仁美诓圣驾来到北番。你的父设下了双龙会宴,我弟兄八员将 }

【{\footnotesize 转}{\akai 西皮快板}】就赴会在沙滩。我大哥替宋王席前遭难,我二哥短剑下命丧黄泉。我三哥被马踏尸如泥烂,我五弟在五台削发参禅({\akai 或}: 削发修禅)。我六弟镇三关威名震显,我七弟被潘洪乱箭来攒。我本是杨------

\textless{}\!{\bfseries\akai 哭头}\!\textgreater{}啊------贤公主,我的妻呀!

\setlength{\hangindent}{60pt}{ 【{\akai 西皮摇板}】我本是杨四郎名姓改换,将楊字拆木易匹配姻缘。 }

\setlength{\hangindent}{60pt}{ 【{\akai 西皮快板}】我和你好夫妻恩爱匪浅,贤公主又何必故意迁延\footnote{ 迁延,拖延,多指耽误时间之意。此外还有后退、退却之意;又可指徘徊,停留不前貌。{466}}({\akai 或}: 故意谦言;或: 过于歉言)。杨延辉有一日愁眉得展,誓不忘贤公主你恩重如山。 }

(铁境公主\hspace{10pt}~

\setlength{\hangindent}{60pt}{ 【{\akai 西皮快板}】$\cdots{}\cdots{}$有什么心腹事$\cdots{}\cdots{}$) }

\setlength{\hangindent}{60pt}{ 【{\akai 西皮快板}】非是我终日里愁眉不展,有一桩心腹事不敢明言。萧天佐摆天门两国交战,我的娘押粮草({\akai 或}: 我的娘领人马)来到北番。贤公主若容我母子相见({\akai 或}: 母子来见),到来生变犬马结草衔环。 }

\setlength{\hangindent}{60pt}{ 【{\akai 西皮快板}】虽然公主行方便,无有令箭难出关。 }

\setlength{\hangindent}{60pt}{ 【{\akai 西皮快板}】公主赐我金鈚箭,见母一面即刻还。 }

\setlength{\hangindent}{60pt}{ 【{\akai 西皮快板}】公主只管放大胆,快马加鞭一夜还。 }

哦------

\setlength{\hangindent}{60pt}{ 【{\akai 西皮快板}】公主叫我盟誓愿,屈膝跪在地平川。我若探母不回转,黄沙盖定\footnote{ 段公平君建议作``黄沙盖顶''。{467}}({\akai 或}: 黄沙盖脸)尸不全。 }

\setlength{\hangindent}{60pt}{ 【{\akai 西皮快板}】一见公主盗令箭,本宫才把心放宽。扭回头来叫小番, }

\setlength{\hangindent}{60pt}{ 【{\akai 西皮散板}】将爷的千里战马扣连环,驸马爷即刻出关。 }

{{[}第二场{]}}

\setlength{\hangindent}{60pt}{ 【{\akai 西皮快板}】在头上摘下狐腋冠,身上脱下紫罗衫。沿毡帽,齐眉掩,三尺青锋挂腰间。将身来在了宫门站,等、等$\cdots{}\cdots{}$等候了公主奔阳关。 }

公主回来了。

辛苦你了,

有劳你了。({\akai 或}: 难为你了。)

拿来------

令箭呐!

唉呀!你误了本宫的大事了。

公主请上,受我一拜!

公主啊------

\setlength{\hangindent}{60pt}{ 【{\akai 西皮快板}】虽然相隔一夜晚,为人总要礼当先。辞别公主跨雕鞍, }

马来!

\setlength{\hangindent}{60pt}{ 【{\akai 西皮摇板}】泪汪汪哭出了雁门关。 }

{{[}第三场{]}}

\setlength{\hangindent}{60pt}{ 【{\akai 西皮快板}】乔装改扮离宫院,一心回家探慈颜。 }

\setlength{\hangindent}{60pt}{ 【{\akai 西皮快板}】催马来在关前站,把关的儿郎列两边。 }

开关!

奉了太后旨意,出关另有公干。

站定了!

\setlength{\hangindent}{60pt}{ 【{\akai 西皮快板}】听说一声要令箭,翻身下了马雕鞍。用手取出金鈚箭,把关之人你要仔细观。 }

\setlength{\hangindent}{60pt}{ 【{\akai 西皮摇板}】两国不和屡交战,把守关口莫偷闲。任那南蛮巧改扮, }

马------来呃!

\setlength{\hangindent}{60pt}{ 【{\akai 西皮摇板}】无有太后的令箭,莫放他出关({\akai 或}: 莫放他过关)。 }

{{[}第四场{]}}

\setlength{\hangindent}{60pt}{ 【{\akai 西皮快板}】眼望宋营灯光影\footnote{ 据夏行涛君告知,苏雪安记录的谭派词句作``灯光隐''。{468}},刀枪剑戟似麻林。大胆且把辕门进,闯入御营见娘亲。 }

{{[}第五场{]}}

(杨延昭\hspace{20pt}~

({\akai 内})【{\akai 西皮导板}】一封战表到东京,)

(杨延昭

\setlength{\hangindent}{60pt}{ 【{\akai 西皮原板}】宋王爷御驾亲自征。萧天佐摆下无名阵,满营将官解不明。(本帅帐中修书信,天波府搬来了老娘亲。)我命宗保把兵请,中途路上遇仙人({\akai 或}: 谁知中途遇仙人)。得来兵书三卷整({\akai 或}: 拾来经书三卷整),才知番邦阵有名: 青龙堪比孟佩苍,白虎就是焦克明。玄武机关擒岳胜,朱雀竟有宗保名。天罡地煞金锁阵,阵阵都有宋将名。青龙阵下少曲水,玉皇殿前缺天灯。白虎少耳又无睛,玄武皂旗无人擎。将身且坐宝帐等,且候五哥破天门。)\footnote{ 这段杨延昭的唱词是陈超老师提供,与《四游记》中``东游记''所述天门阵故事符合。据《稀见京昆名伶抄校本丛刊  (第一辑)》\textsuperscript{{[}18{]}}中所收录``燕台菊萃第一辑`全本探母回令'(据杨宝森朱批本影印)''杨宝森批注: ``余先生吊毛下场重新勒头,以温黄酒饮场,六郎唱垫场词: $\cdots{}\cdots{}$洪林(贾洪林)老先生傍老谭唱词''。与此高度一致(括号中为杨注词句差异处)。``解不明''作``皆不明''似亦可。  经请教尹薇君,告知宋湛清先生所传杨延昭唱词为:  \begin{quote}  一封战表到东京,宋王爷御驾亲自征。萧天佐摆下了无名阵,满营内将官解不明。我命宗保打头阵,在中途路上遇仙人。得来了兵书三卷整,才知道番邦阵有名。天门阵一百单八阵,阵阵俱有我姓杨的人: 硃砂阵,孟伯苍,黑沙阵内焦克明。童子阵,杨宗保,女儿阵上穆桂英。青龙阵,宋天子,白虎阵内有本帅名。将身且坐宝帐等,五哥下山破天门。  \end{quote}  {469}} }

({\akai 内})【{\akai 西皮导板}】大吼一声呐如雷震,

\setlength{\hangindent}{60pt}{ 【{\akai 西皮快板}】杨家将令鬼神惊。大胆且把宝帐进呐, }

\setlength{\hangindent}{60pt}{ 【{\akai 西皮快板}】上面呐坐定同胞人。弟兄分别十五春,不料今日回宋营。暂且不通名和姓,问我一言答一声。 }

\setlength{\hangindent}{60pt}{ 【{\akai 西皮快板}】家住山后磁州\footnote{ 磁州在晋东南。查阅历史地图,今忻州神池西即为宋之火山军治所,火塘寨应在此附近。由于过去艺人口述辗转相传,``池州''讹为磁州(此外还有赤州、石州、慈州等),此处从俗。据《金史地理志》,池州本应名庾州,方志云: ``庾州,中宋旧火山军,大定二十二年升为火山州,后更今名。兴定二年九月改隶岚州,四年以残破徙治于黄河滩许父寨。户七千五百九十二,县一、镇一: 河曲贞元元年置,有火山、黄河。\ldots{}''火塘山考其地应在今河曲县南部,有黄土大山终年冒烟火,俗呼``火山''。{470}}郡,火塘寨上有家门。我父令公官极品,我母佘氏老太君。十五年前沙滩会,失落番邦被贼擒。六弟下位将兄认,我是四哥回宋营。 }

罢了。

这是何人?({\akai 或}: 此是何人?)

多大年纪?

呜哙呀,且喜杨家有后,待我谢天谢地。

唉!一言难尽呐------

\setlength{\hangindent}{60pt}{ 【{\akai 西皮原板}】弟兄们分别十五春呐,我和你沙滩会两离分。闻听得老娘驾到北郡,因此上乔改扮黑夜里探望娘亲呐。 }

\setlength{\hangindent}{60pt}{ 【{\akai 西皮摇板}】问贤弟老娘今何在? }

(杨延昭\hspace{20pt}~

\setlength{\hangindent}{60pt}{ 【{\akai 西皮摇板}】$\cdots{}\cdots{}$后帐未出来。) }

\setlength{\hangindent}{60pt}{ 【{\akai 西皮摇板}】有劳贤弟把路带, }

\setlength{\hangindent}{60pt}{ 【{\akai 西皮摇板}】母子们相逢痛伤怀。 }

{{[}第六场{]}}

(杨延昭\hspace{20pt}~

\setlength{\hangindent}{60pt}{ 【{\akai 西皮摇板}】$\cdots{}\cdots{}$营门外。) }

\setlength{\hangindent}{60pt}{ 【{\akai 西皮摇板}】贤弟禀告老萱台。 }

这是何人?({\akai 或}: 这是$\cdots{}\cdots{}$)

\textless{}\!{\bfseries\akai 三叫头}\!\textgreater{}母亲!老娘!唉,娘------啊$\cdots{}\cdots{}$(哭介)

\textless{}\!{\bfseries\akai 三叫头}\!\textgreater{}老娘!母亲!唉,母亲------啊$\cdots{}\cdots{}$(哭介)

\setlength{\hangindent}{60pt}{ 【{\akai 西皮导板}】老娘亲请上啊受儿啊 }

【回龙】拜,

唉,娘啊,呃$\cdots{}\cdots{}$(哭介)

\setlength{\hangindent}{60pt}{ 【{\akai 西皮二六}】千拜万拜也是折不过儿的罪来。孩儿被擒在番邦外,隐姓埋名脱祸灾({\akai 或}: 隐姓埋名躲祸灾)。多蒙太后恩似海呀,铁镜公主配和谐。儿在番邦一十五载,常把我的老娘挂在儿的心怀。胡狄衣冠懒穿戴,每年间花开 }

【{\footnotesize 转}{\akai 西皮快板}】儿的心不开。闻听得老娘征北塞,乔装改扮回营来。见母一面愁颜解,愿老娘福寿康宁、永无恙无灾。

\setlength{\hangindent}{60pt}{ 【{\akai 西皮快板}】那铁镜公主真可爱,黄金难买女裙钗。本当过营来奉拜,怎奈是两下相争呐,儿的娘啊,她不能来。 }

\setlength{\hangindent}{60pt}{ 【{\akai 西皮摇板}】六贤弟请上受兄拜,贤弟可挂忠孝牌。 }

\setlength{\hangindent}{60pt}{ 【{\akai 西皮摇板}】二贤妹请上受一拜,愧煞愚兄不将才。 }

\setlength{\hangindent}{60pt}{ 【{\akai 西皮散板}】听一言来泪满腮,心中阵阵似刀裁。问贤妹你四嫂今何在, }

\setlength{\hangindent}{60pt}{ 【{\akai 西皮散板}】有劳贤妹把路带, }

\setlength{\hangindent}{60pt}{ 【{\akai 西皮散板}】儿到后营看一看受苦的女裙钗。儿的娘啊,儿去去就来。 }

{{[}第七场{]}}

这是何人?\hspace{10pt}~

({\akai 或}: 这是$\cdots{}\cdots{}$)

\textless{}\!{\bfseries\akai 三叫头}\!\textgreater{}孟氏!我妻!唉,妻------呀,呃$\cdots{}\cdots{}$(哭介)

\textless{}\!{\bfseries\akai 三叫头}\!\textgreater{}贤妻!孟氏!唉,妻------呀,呃$\cdots{}\cdots{}$(哭介)

妻呀!

\setlength{\hangindent}{60pt}{ 【{\akai 西皮快板}】自从沙滩一阵败,隐姓埋名躲祸灾。萧后待我恩似海,铁镜公主配和谐。闻听得老娘到北塞,乔装改扮回营来({\akai 或}: 乔装改扮过营来)。一来见娘问安泰,二念贤妻挂心怀。 }

\setlength{\hangindent}{60pt}{ 【{\akai 西皮快板}】贤妻莫把夫来怪,我有言来听开怀: 不是她盗令来得快,插翅焉能转回来。临行时言语叮咛再,即刻回令莫迟捱。夫妻们只哭得肝肠啊\textless{}\!{\bfseries\akai 哭头}\!\textgreater{}坏, }

唉呀!

\setlength{\hangindent}{60pt}{ 【{\akai 西皮散板}】又听谯楼四更牌。 }

\setlength{\hangindent}{60pt}{ 【{\akai 西皮散板}】辞别贤妻出帐外呀, }

\setlength{\hangindent}{60pt}{ 【{\akai 西皮散板}】你苦苦地留我为何来? }

\setlength{\hangindent}{60pt}{ 【{\akai 西皮散板}】岂不知老娘年高迈,船到江心马停崖。 }

罢!

\setlength{\hangindent}{60pt}{ 【{\akai 西皮散板}】狠心抛妻后营寨。 }

唉呀!

{{[}第八场{]}}

\setlength{\hangindent}{60pt}{ 【{\akai 西皮散板}】辞别老娘回北塞, }

母亲!({\akai 或}: 老娘!)

儿岂不知``天地为大,忠孝当先''?

孩儿此刻若不回去,你那番邦孙男、媳妇({\akai 或}: 孙儿、媳妇)难免这一刀------之苦哇$\cdots{}\cdots{}$(哭介)

\textless{}\!{\bfseries\akai 哭头}\!\textgreater{}老娘亲呐,

\textless{}\!{\bfseries\akai 哭头}\!\textgreater{}六贤弟,

\textless{}\!{\bfseries\akai 哭头}\!\textgreater{}二贤妹呀,

\textless{}\!{\bfseries\akai 哭头}\!\textgreater{}受苦的妻呀,

\textless{}\!{\bfseries\akai 哭头}\!\textgreater{}啊,儿的娘啊。

唉呀!

\setlength{\hangindent}{60pt}{ 【{\akai 西皮散板}】谯楼鼓打五更牌, }

\setlength{\hangindent}{60pt}{ 【{\akai 反西皮散板}】辞别老娘呃出帐外呃,杨四郎心中似刀裁呀:  }

\setlength{\hangindent}{60pt}{ 【{\akai 反西皮散板}】舍不得老娘啊年高迈, }

\setlength{\hangindent}{60pt}{ 【{\akai 反西皮散板}】舍不得六贤弟将英才。 }

\setlength{\hangindent}{60pt}{ 【{\akai 反西皮散板}】舍不得二贤妹未出闺阁外, }

\setlength{\hangindent}{60pt}{ 【{\akai 反西皮散板}】实难舍结发的夫妻两分开。 }

(罢!)

\setlength{\hangindent}{60pt}{ 【{\akai 反西皮散板}】狠心肠抛一家出了帐外。\textless{}\!{\bfseries\akai 扫头}\!\textgreater{} }

哦。

(走!)

{{[}第九场{]}}

\setlength{\hangindent}{60pt} {【{\akai 西皮快板}】雁门关前来拿定,好似鱼儿把钩吞。罢罢罢,且把银安进,太后台前请罪名。}

{太后!}

\setlength{\hangindent}{60pt} {【{\akai 西皮快板}】家住在山后磁州郡,火塘寨上有家门。太后问儿的名和姓,儿本是杨------}

\setlength{\hangindent}{60pt} {【{\akai 西皮摇板}】杨四郎延辉是儿的名呐。}

{唉呀!}

\setlength{\hangindent}{60pt} {【{\akai 西皮快板}】早知道回令无性命,见母不该转回程。眼望后宫呼救哇}\textless{}\!{\bfseries\akai 哭头}\!\textgreater{}{应,}

\textless{}\!{\bfseries\akai 哭头}\!\textgreater{}{公主,我的妻呀!}

\setlength{\hangindent}{60pt} {【{\akai 西皮摇板}】夫妻们见一面死也甘心。}

\setlength{\hangindent}{60pt} {【{\akai 西皮导板}】在银安绑得我昏迷不醒,}

{唉,公主啊$\cdots{}\cdots{}$(哭介)}

{唉------}

\setlength{\hangindent}{60pt} {【{\akai 西皮快板}】又只见公主到来临。你若念在夫妻义,太后台前讲人情。你若不念夫妻义,杀了我杨延辉,你另嫁旁人。}

{太后!}

{唉!太后啊$\cdots{}\cdots{}$(哭介)}

\textless{}\!{\bfseries\akai 哭头}\!\textgreater{}{我哭,哭一声老太后,}

\setlength{\hangindent}{60pt}{ 【{\akai 反西皮散板}】{当初被擒就该斩,} }

\setlength{\hangindent}{60pt}{ 【{\akai 反西皮散板}】{杀了孩儿不要紧,} }

\setlength{\hangindent}{60pt}{ 【{\akai 反西皮散板}】{老太后哇,} }

\textless{}\!{\bfseries\akai 哭头}\!\textgreater{}{啊------丈母娘呐!({\akai 或}: 啊------老太后啊!)}

\setlength{\hangindent}{60pt} {【{\akai 西皮快板}】千层浪里翻身滚,百尺高竿又复生。见了公主礼恭敬,}

{适才多蒙公主讲情,我这里当面谢过。}

{当面谢过。}

{公主啊------}

\setlength{\hangindent}{60pt} {【{\akai 西皮摇板}】我母道你是贤德的人呐。}

{不敢,不敢! ({\akai 或}: 岂敢,岂敢!)}

\setlength{\hangindent}{60pt} {【{\akai 西皮摇板}】夫妻双双银安进,多谢太后不斩恩。}

{领旨!}

{哦,是是是!}

{众小番,带马四盘山去者!}

\newpage

\hypertarget{ux6d2aux7f8aux6d1e-ux4e4b-ux6768ux5ef6ux662dux8001ux4ee4ux516c}{%

\subsection{\texorpdfstring{洪羊洞\footnote{ 剧本参照《刘曾复京剧文存》\textsuperscript{{[}19{]}.}中收录的《\textless{}洪羊洞\textgreater{}说戏誊稿》并结合刘曾复先生说戏录音整理。{471}}

之

杨延昭、老令公}{洪羊洞471 之 杨延昭、老令公}}\label{ux6d2aux7f8aux6d1e-ux4e4b-ux6768ux5ef6ux662dux8001ux4ee4ux516c}}

{{[}第一场{]}}

{(\textless{}冲头\textgreater{}切住,起更,要稍快,用\textless{}咚哐\textgreater{}收,一场\textless{}大锣打上\textgreater{},四鬼卒引老令公魂子上,鬼卒站门,令公到台口,\textless{}大锣归位\textgreater{})}

{老令公

({\akai 念})生前为大将,死后做忠魂。(\textless{}住头\textgreater{})}

{老令公

吾乃继业灵魂({\akai 或}: 阴魂;鬼魂)是也。(\textless{}住头\textgreater{})}

{老令公

今当三星归位之期,我不免去至天波杨府托兆一番({\akai 或}: 托梦一回;托梦一番)便了。}

{老令公\hspace{20pt}~

众鬼卒,}

{(鬼卒应``呜''。)}

{老令公\hspace{20pt}~

驾起阴风,天波杨府去者。}

{(\textless{}帽子头\textgreater{}\textless{}哆啰\textgreater{}起带头子【{\akai 二黄原板}】)}

{老令公

\setlength{\hangindent}{60pt}{ 【{\akai 二黄原板}】风萧萧}\footnote{ 刘曾复先生为戏曲学院说戏录音中近于``{风飘飘}'',此处从《说戏誊稿》。{472}}{冷飕飕星稀月淡,荡悠悠飘渺渺来到人间。教鬼卒前引路风旗辗转({\akai 或}: 风旗拨转}\footnote{ 段公平君建议作``风起魄转''。{473}}{),} }

{(\textless{}大锣抽头\textgreater{}鬼卒领下)}

{老令公\hspace{20pt}~

\setlength{\hangindent}{60pt}{ 【{\akai 二黄原板}】此一去见六郎细说根源。} }

{(\textless{}大锣抽头\textgreater{}老令公下)}

{[}第二场{]}

({\textless{}大锣抽头\textgreater{}转\textless{}小锣四反正\textgreater{}家院打灯笼引杨延昭上}\footnote{ 据《京剧谈往录四编》\textsuperscript{{[}20{]}.}载刘曾复先生著《我所见过的一些京剧配角老生演员》介绍,``{\textless{}小锣四反正\textgreater{}由\textless{}抽头\textgreater{}\textless{}反带锣\textgreater{}\textless{}硬四击\textgreater{}\textless{}夺头\textgreater{}四个小锣鼓点连接合成的。''}此处的表演细节为: {``家院在\textless{}抽头\textgreater{}中掌灯上场,走到中场右转身举灯向上场门一照,这就是交代,表示让\textless{}抽头\textgreater{}收住,转\textless{}反带锣\textgreater{};杨延昭这才好在\textless{}反带锣\textgreater{}中上场,在\textless{}硬四击\textgreater{}中一亮,双投袖叫\textless{}夺头\textgreater{}起}【{\akai 二黄原板}】{听更起唱}。''{474}}{,起}【{\akai 二黄原板}】{二更})

杨延昭

\setlength{\hangindent}{60pt}{ 【{\akai 二黄原板}】为国家哪何曾半日闲空,我也曾平服了塞北西东。官封到节度使啊皇王恩重, }

(家院带门下,杨延昭归大座)

杨延昭\hspace{20pt}~

\setlength{\hangindent}{60pt}{ 【{\akai 二黄原板}】身不爽不由人瞌睡朦胧啊。 }

(杨延昭睡介,{鬼卒上站``一条边''},{老令公上,到小边台,唱}【{\akai 二黄原板}】{三更})

{老令公

\setlength{\hangindent}{60pt}{ 【{\akai 二黄原板}】黑暗暗雾沉沉人烟息静,惨戚戚悲切切来到家门。静悄悄沉寂寂天波府进,} }

{(众挖门,\textless{}大锣抽头\textgreater{},鬼卒站门,老令公到大边)}

{老令公\hspace{20pt}~

\setlength{\hangindent}{60pt}{ 【{\akai 二黄原板}】又只见六郎儿瞌睡沉沉。我这里将他的灵魂唤醒,} }

{(叫散,杨延昭醒,\textless{}乱锤\textgreater{}站,一望,桌面上向老令公双投袖,\textless{}撕边\textgreater{}\textless{}凤点头\textgreater{},起}【{二黄摇板}】{)}

杨延昭

\setlength{\hangindent}{60pt}{ 【{二黄摇板}】猛抬头又只见我父令公。({\textless{}仓\textgreater{}}) }

杨延昭

\setlength{\hangindent}{60pt}{ 【{二黄摇板}】曾记得在两狼父归仙境,哪有个人故后又能复逢。({\textless{}仓\textgreater{}}) }

杨延昭\hspace{20pt}~

\setlength{\hangindent}{60pt}{ 【{二黄摇板}】我这里({\textless{}仓\textgreater{}}) }

杨延昭

\setlength{\hangindent}{60pt}{ 【{\footnotesize 接}\akai {二黄摇板}】下位去({\textless{}顷仓\textgreater{}}) }

杨延昭

\setlength{\hangindent}{60pt}{ 【{\footnotesize 接}\akai {二黄摇板}】实难呐({\textless{}仓仓仓仓仓才仓\textgreater{})} }

杨延昭\hspace{20pt}~

\setlength{\hangindent}{60pt}{ 【{\footnotesize 接}\akai {二黄摇板}】转动。 }

{(杨延昭坐,\textless{}撕边一锣\textgreater{},\textless{}哆啰\textgreater{},起【回龙】)}

{老令公\hspace{20pt}~

\setlength{\hangindent}{60pt}{ 【回龙】我的儿休贪睡父有话云: } }

{(杨延昭睡介,\textless{}夺头\textgreater{},起}【{\akai 二黄原板}】{四更)}

{老令公

\setlength{\hangindent}{60pt}{ 【{\akai 二黄原板}】儿前番命孟良骸骨搬运,那乃是萧天佐以假为真({\akai 或}: 弄假为真)。真骸骨现在({\akai 或}: 真骸骨藏在)洪羊洞,望乡台上第三层。叮咛的言语牢牢记紧,} }

{(叫散,鬼卒领下,老令公归大边外角,\textless{}扭丝\textgreater{}切住,五更\textless{}凤点头\textgreater{})}

{老令公

\setlength{\hangindent}{60pt}{ 【{\akai 二黄散板}】待等儿临危时}\footnote{ 夏行涛君建议作``临位时'',即``归位''之时。此处从《说戏誊稿》。{475}}{父再来临。} }

{(\textless{}大锣打下\textgreater{}老令公下,亮更,家院上,推门、进门,挖到小边)}

{(家院\hspace{30pt}~

元帅醒来。)}

杨延昭\hspace{20pt}~

\setlength{\hangindent}{60pt}{ 【{二黄导板}】方才老元戎呐前来托梦, }

(杨延昭望介,{\textless{}嘟仓\textgreater{},}立{,}出位{,\textless{}快扭丝\textgreater{}}到中场)

杨延昭

\setlength{\hangindent}{60pt}{ 【{二黄散板}】醒来时不由人珠泪满胸。({\textless{}住头\textgreater{}}) }

(杨延昭小座)

杨延昭\hspace{20pt}~

有请孟二爷。

(家院\hspace{30pt}~

有请孟二爷。)

({家院下})

(孟良\hspace{30pt}~

({\akai 内})嗯喷。)

({\textless{}小锣打上\textgreater{}孟良上})

(孟良\hspace{30pt}~

({\akai 念})不听皇王三诏宣,单听杨家一令传。)

({孟良进门})

(孟良\hspace{30pt}~

参见元帅。)

杨延昭\hspace{20pt}~

贤弟少礼,请坐。

(孟良\hspace{30pt}~

谢座。)

({\textless{}台\textgreater{}}孟良坐大边)

(孟良\hspace{30pt}~

唤末将前来有何军事议论?)

杨延昭

贤弟哪里知道,昨晚三更时分,老元戎前来托梦,言道: 前番盗骨,乃是假的。

(孟良\hspace{30pt}~

真的呢?)

杨延昭

现在北国洪羊洞,望乡台第三层之上。愚兄意欲,命贤弟二下番营,盗取骸骨,不知贤弟意下如何?

(孟良

元帅说哪里话来,末将好比元帅胯下之驹,扬鞭就走,勒缰即止。就请元帅传令。)

杨延昭\hspace{20pt}~

如此贤弟听令: 

(杨延昭站,孟良站)

(孟良\hspace{30pt}~

在。)

(杨延昭拿令旗)

杨延昭\hspace{20pt}~

({\akai 念})本帅帐中把令传,

(杨延昭令旗交孟良,孟良接令旗)

(孟良\hspace{30pt}~

({\akai 念})此去哪怕路艰难。)

(杨延昭边过大边边念)

杨延昭\hspace{20pt}~

({\akai 念})但愿盗得尸骸转,

(孟良同时过小边)

(孟良\hspace{30pt}~

({\akai 念})凌烟阁上美名传。)

杨延昭\hspace{20pt}~

小心。

(孟良\hspace{30pt}~

得令。)

(杨延昭由下场门下,孟良上场门下,{\textless{}小锣打下\textgreater{}})

{(}以下{[}第三场{]}到{[}第七场{]}是孟良、焦赞二人盗骨,与他们二人之死。这几场唱【{\akai 西皮}】,从略{)}

(程宣下{\textless{}小锣打下\textgreater{}})

{[}第八场{]}

{(\textless{}小锣抽头\textgreater{},}杨延昭上,站{)}

杨延昭\hspace{20pt}~

\setlength{\hangindent}{60pt}{ 【{\akai 二黄摇板}】孟良盗骨无音信,倒教本帅挂在心。 }

{(}杨延昭坐小座,{\textless{}小锣五击头\textgreater{},}家院领程宣上到小边台口{)}

{(家院\hspace{30pt}~

候着。)}

{(}家院进门归大边{)}

{(家院\hspace{30pt}~

启禀元帅,小番求见。)}

杨延昭\hspace{20pt}~

传。

{(}家院出门{)}

{(家院\hspace{30pt}~

元帅传你,需要小心。)}

{(程宣\hspace{30pt}~

是。)}

{(家院}、{程宣}进门{,家院}大边{,程宣}边挖到小边边念{)}

{(程宣\hspace{30pt}~

元帅在哪里,元帅在哪里。)}

杨延昭\hspace{20pt}~

嗯------({\textless{}台\textgreater{}})

杨延昭\hspace{20pt}~

胆大小番,头顶何物?见了本帅,大胆不跪?

(程宣\hspace{30pt}~

来人言过: 见了元帅,去掉头上匣儿,方可下跪。)

杨延昭\hspace{20pt}~

将匣儿取去。({\akai 或}: 来,将匣儿取过。)

(家院取匣,{\textless{}台\textgreater{},大边端匣请杨延昭看})

杨延昭\hspace{20pt}~

呈上来。

(杨延昭接匣,看)

杨延昭\hspace{20pt}~

令公骸------({\textless{}仓\textgreater{}})

杨延昭\hspace{20pt}~

唉呀!

({\textless{}快扭丝\textgreater{}},杨延昭台口跪)

杨延昭

\setlength{\hangindent}{60pt}{ 【{\akai 二黄散板}】见骸骨哇不由人泪双流,({\textless{}仓\textgreater{}}) }

杨延昭

\setlength{\hangindent}{60pt}{ 【{\akai 二黄散板}】如今才见亲骨肉哇。({\textless{}仓\textgreater{}}) }

杨延昭\hspace{20pt}~

\setlength{\hangindent}{60pt}{ 【{\akai 二黄散板}】家院供奉二堂后, }

(杨延昭起身,{\textless{}扭丝\textgreater{}},匣交家院拿,放堂桌上)

杨延昭\hspace{20pt}~

\setlength{\hangindent}{60pt}{ 【{\akai 二黄散板}】再与老军说从头。 }

(杨延昭小座,{\textless{}住头\textgreater{}})

(程宣\hspace{30pt}~

叩见元帅。)

(程宣叩介)

杨延昭\hspace{20pt}~

罢了,起来。

(程宣\hspace{30pt}~

谢元帅。)

(程宣起身归小边)

杨延昭\hspace{20pt}~

你奉何人所差?

(程宣\hspace{30pt}~

孟良孟二爷所差。)

杨延昭\hspace{20pt}~

有何为证?

(程宣\hspace{30pt}~

板斧为证。)

杨延昭\hspace{20pt}~

呈上来。

(家院拿斧呈杨延昭看,{\textless{}台\textgreater{}})

杨延昭\hspace{20pt}~

收过。

(家院拿斧放堂桌上)

杨延昭\hspace{20pt}~

你叫什么名字?

(程宣\hspace{30pt}~

小人名叫程宣。)

杨延昭\hspace{20pt}~

程宣,你孟二爷他往哪里去了?

(程宣\hspace{30pt}~

哎呀,元帅呀!\textless{}\!{\bfseries\akai 台台令令台}\!\textgreater{})

(程宣\hspace{30pt}~

孟二爷前去盗骨,焦二爷暗地跟随。孟二爷一时失手,将焦二爷劈死了!

\textless{}\!{\bfseries\akai 仓}\!\textgreater{})

杨延昭\hspace{20pt}~

怎么讲?!

(程宣\hspace{30pt}~

将焦二爷劈死了!

\textless{}\!{\bfseries\akai 仓}\!\textgreater{})

杨延昭\hspace{20pt}~

(唉!)贤弟呀$\cdots{}\cdots{}$(哭介)

({\textless{}快扭丝\textgreater{}},杨延昭站)

杨延昭

\setlength{\hangindent}{60pt}{ 【{\akai 二黄散板}】听罢言来泪双淋,(\textless{}\!{\bfseries\akai 仓}\!\textgreater{}) }

杨延昭\hspace{20pt}~

\setlength{\hangindent}{60pt}{ 【{\akai 二黄散板}】可叹你为杨家命丧番营。 }

杨延昭\hspace{20pt}~

(唉!)贤弟呀,呃$\cdots{}\cdots{}$(哭介)

({\textless{}住头\textgreater{}},杨延昭坐)

杨延昭\hspace{20pt}~

焦二爷已死,那孟二爷也该来见我哇。

(程宣\hspace{30pt}~

哎呀,元帅呀!\textless{}\!{\bfseries\akai 台台令令台}\!\textgreater{})

(程宣\hspace{30pt}~

孟二爷将焦二爷劈死了,不愿回来,他就自刎在洪羊洞!)

({\textless{}快冲头\textgreater{}},杨延昭拉程宣到台口)

杨延昭\hspace{20pt}~

怎么讲?!

(程宣\hspace{30pt}~

自刎在洪羊洞!)

({\textless{}快冲头\textgreater{},}\textless{}\!{\bfseries\akai 双叫头}\!\textgreater{})

杨延昭\hspace{20pt}~

孟良!

(\textless{}\!{\bfseries\akai 顷仓}\!\textgreater{})

杨延昭\hspace{20pt}~

焦赞!

(\textless{}\!{\bfseries\akai 仓才仓}\!\textgreater{})

杨延昭\hspace{20pt}~

唉呀!

(杨延昭台口气椅坐{,\textless{}快冲头\textgreater{}家院、}程宣挡)

(家院、程宣 元帅醒来。)

杨延昭\hspace{20pt}~

\setlength{\hangindent}{60pt}{ 【{\akai 二黄导板}】听说是二将双双丧命呐, }

(杨延昭站,{\textless{}冲头\textgreater{},}\textless{}\!{\bfseries\akai 双叫头}\!\textgreater{})

杨延昭\hspace{20pt}~

焦赞!

(\textless{}\!{\bfseries\akai 顷仓}\!\textgreater{})

杨延昭\hspace{20pt}~

孟良!

(\textless{}\!{\bfseries\akai 仓才仓}\!\textgreater{})

杨延昭\hspace{20pt}~

唉,贤弟呀,呃$\cdots{}\cdots{}$(哭介)

({\textless{}扭丝\textgreater{}})

杨延昭

\setlength{\hangindent}{60pt}{ 【{\akai 二黄散板}】去掉我左右膀难以飞行。教老军(\textless{}\!{\bfseries\akai 仓}\!\textgreater{}) }

杨延昭\hspace{20pt}~

\setlength{\hangindent}{60pt}{ 【{\footnotesize 接}\akai 二黄散板}】到番营尸骸搬运, }

({\textless{}扭丝\textgreater{},程宣}出门{,}上场门下)

杨延昭

\setlength{\hangindent}{60pt}{ 【{\akai 二黄散板}】待本帅奏圣上超度阴魂。(\textless{}\!{\bfseries\akai 仓}\!\textgreater{}) }

杨延昭\hspace{20pt}~

唉呀!

({\textless{}乱锤\textgreater{},杨延昭}胸疼介{,家院}搀{,\textless{}凤点头\textgreater{}})

杨延昭\hspace{20pt}~

\setlength{\hangindent}{60pt}{ 【{\akai 二黄散板}】霎时间(\textless{}\!{\bfseries\akai 仓}\!\textgreater{}) }

杨延昭\hspace{20pt}~

\setlength{\hangindent}{60pt}{ 【{\footnotesize 接}\akai 二黄散板}】心内痛啊({\akai 或}: 腹内痛啊) }

(\textless{}\!{\bfseries\akai 顷仓}\!\textgreater{})

杨延昭

\setlength{\hangindent}{60pt}{ 【{\footnotesize 接}\akai 二黄散板}】鲜血({\akai 或}: 心血)上(\textless{}\!{\bfseries\akai 仓仓仓仓才仓}\!\textgreater{}) }

杨延昭\hspace{20pt}~

\setlength{\hangindent}{60pt}{ 【{\footnotesize 接}\akai 二黄散板}】涌啊, }

({家院}搀{,\textless{}乱锤\textgreater{},杨延昭}吐介{)}

{呜$\cdots{}\cdots{}$(}\textless{}\!{\bfseries\akai 顷仓}\!\textgreater{}{)呜$\cdots{}\cdots{}$(}\textless{}\!{\bfseries\akai 顷仓}\!\textgreater{}\textless{}\!{\bfseries\akai 叭嗒仓}\!\textgreater{}{)呜$\cdots{}\cdots{}$(}\textless{}\!{\bfseries\akai 乱锤}\!\textgreater{})

(\textless{}\!{\bfseries\akai 扭丝}\!\textgreater{}{\textless{}凤点头\textgreater{},家院}搀{杨延昭。杨延昭}倒左右手{、}左右两转身到大边)

杨延昭\hspace{20pt}~

\setlength{\hangindent}{60pt}{ 【{\akai 二黄散板}】休得要惊动年迈的太君。 }

(杨延昭左转身,右手扶家院手,{\textless{}大锣打下\textgreater{},\textless{}撤锣\textgreater{},杨延昭、家院下})

{[}第九场{]}

({\textless{}小锣打上\textgreater{},}四{太监、大太监}上{,站门,赵德芳}上{,}到台口)

(赵德芳\hspace{20pt}~

{[}{\akai 引子}{]}一片丹心,保叔王,锦绣龙庭。)

({\textless{}小锣归位\textgreater{}},赵德芳坐小座)

(赵德芳

({\akai 念})紫金冠凤翅双飘,蟒龙袍玉带围腰。上金殿扬尘舞蹈,凹面锏压定群僚。)

({\textless{}小锣住头\textgreater{}})

(赵德芳\hspace{20pt}~

本御赵德芳。\textless{}\!{\bfseries\akai 台}\!\textgreater{})

(赵德芳

下朝回宫,内侍报道: 御妹夫身染重病。本御放心不下,亲去探望。内侍。)

({\akai 内}侍应``有'')

(赵德芳\hspace{20pt}~

御林军走上。)

({\akai 内}侍\hspace{30pt}~

御林军走上。)

({\textless{}冲头\textgreater{},}四{御林军}两边上{,}合龙)

(御林军\hspace{20pt}~

参见贤爷。)

(赵德芳\hspace{20pt}~

罢了。)

(四{御林军}分站两边{,\textless{}住头\textgreater{}})

(赵德芳\hspace{20pt}~

外厢开道,天波杨府去者。)

(众应``啊'')

(赵德芳\hspace{20pt}~

带马。)

(赵德芳上马,起{\textless{}大锣长锤\textgreater{},赵德芳唱}【{\akai 二黄原板}】中众``扯四门'')

(赵德芳

\setlength{\hangindent}{60pt}{ 【{\akai 二黄原板}】我本是金枝体大宋根本,秉忠心保叔王锦绣龙庭。内侍报御妹夫身染重病,因此上为王我御驾亲临。御林军忙摆驾前把路引,) }

(叫散{,\textless{}大锣扭丝\textgreater{},众站小边。虎形下场门上,到大边台口,风声,跳介,\textless{}凤点头\textgreater{}})

(赵德芳\hspace{20pt}~

\setlength{\hangindent}{60pt}{ 【{\akai 二黄散板}】只见猛虎下山林。) }

(赵德芳\hspace{20pt}~

弓箭伺候。)

({\textless{}凤点头\textgreater{}})

(赵德芳\hspace{20pt}~

\setlength{\hangindent}{60pt}{ 【{\akai 二黄散板}】手挽弓又搭箭将虎射定。) }

({\textless{}扫头\textgreater{},虎形下,众下,\textless{}撤锣\textgreater{}})

{[}第十场{]}

杨延昭\hspace{20pt}~

({\akai 内})搀扶!

({\textless{}铙钹夺头\textgreater{},起}【{\akai 二黄慢板}】{\textless{}才\textgreater{},杨宗保在右,搀杨延昭手,上})

杨延昭

\setlength{\hangindent}{60pt}{ 【{\akai 二黄慢板}】叹杨家投宋主啊心血用尽,最可叹({\akai 或}: 真可叹)焦、孟将命丧番营。宗保儿搀为父病房来进({\akai 或}: 床榻靠枕;软榻靠枕), }

({铙钹\textless{}搓锤\textgreater{},杨延昭把杨宗保从小边带到大边,杨延昭面向堂桌正面,走,到了堂桌,右手扶桌,一滑,左转身,甩髯口,右手托身后,向右扶桌,左手在脸前从右往左画圈招手,叫杨宗保,杨综保双手扶杨延昭双手,\textless{}抽头\textgreater{},推磨,杨延昭进大座,杨宗保归小边堂桌旁边,杨延昭坐})

杨延昭\hspace{20pt}~

\setlength{\hangindent}{60pt}{ 【{\akai 二黄原板}】怕只怕难捱过({\akai 或}: 熬不过)尺寸光阴。 }

({\textless{}小锣抽头\textgreater{},赵德芳众上,御林军``一条边''})

(赵德芳\hspace{20pt}~

\setlength{\hangindent}{60pt}{ 【{\akai 二黄散板}】来至在府门外王下金镫,) }

(赵德芳下马,众下。杨宗保出门,作揖)

(杨宗保\hspace{20pt}~

迎接千岁。)

(赵德芳\hspace{20pt}~

\setlength{\hangindent}{60pt}{ 【{\akai 二黄散板}】宗保儿免礼你且平身。你父帅身染病何处安顿?) }

(杨宗保\hspace{20pt}~

现在病房。)

(赵德芳\hspace{20pt}~

带路。)

({\textless{}小锣抽头\textgreater{},赵德芳挖到大边,杨宗保归小边})

(赵德芳\hspace{20pt}~

\setlength{\hangindent}{60pt}{ 【{\akai 二黄散板}】又只见御妹夫瞌睡沉沉。) }

(赵德芳\hspace{20pt}~

醒来。)

杨延昭

\setlength{\hangindent}{60pt}{ 【{\akai 二黄导板\footnote{ 《说戏誊稿》记作{ }

【{\akai 大锣导板}】

【{\akai 散板}】,因刘曾复先生在文稿中注明这是``三个导板之一'',此处记作

【{\akai 导板}】。}{476}}】方才郊外闲游散闷,(\textless{}\!{\bfseries\akai 仓}\!\textgreater{})

杨延昭

\setlength{\hangindent}{60pt}{ 【{\akai 二黄散板}】见一官长放雕翎呐。(\textless{}\!{\bfseries\akai 仓}\!\textgreater{}) }

杨延昭

\setlength{\hangindent}{60pt}{ 【{\akai 二黄散板}】对我胸前射一箭,(\textless{}\!{\bfseries\akai 仓}\!\textgreater{}) }

杨延昭

\setlength{\hangindent}{60pt}{ 【{\akai 二黄散板}】险些儿丧了命残生呐。(\textless{}\!{\bfseries\akai 仓}\!\textgreater{}) }

杨延昭

\setlength{\hangindent}{60pt}{ 【{\akai 二黄散板}】猛然睁开昏花眼呐,(\textless{}\!{\bfseries\akai 仓}\!\textgreater{}) }

(杨延昭一望赵德芳)

杨延昭\hspace{20pt}~

哎呀!

(杨延昭站到桌左内侧,扶杨宗保,{\textless{}凤点头\textgreater{}})

杨延昭

\setlength{\hangindent}{60pt}{ 【{\akai 二黄散板}】面前站定放箭之人。(\textless{}\!{\bfseries\akai 仓}\!\textgreater{}) }

杨延昭

\setlength{\hangindent}{60pt}{ 【{\akai 二黄散板}】我和你一无冤仇,二无怨恨,你,你,你却缘何放雕翎呐射我前心? }

(杨延昭又坐睡,{\textless{}凤点头\textgreater{}})

(赵德芳

\setlength{\hangindent}{60pt}{ 【{\akai 二黄散板}】听罢言来才知情,\textless{}\!{\bfseries\akai 仓}\!\textgreater{}) }

(赵德芳

\setlength{\hangindent}{60pt}{ 【{\akai 二黄散板}】白虎是他本命星。\textless{}\!{\bfseries\akai 仓}\!\textgreater{}) }

(赵德芳

\setlength{\hangindent}{60pt}{ 【{\akai 二黄散板}】走向前来把话论: \textless{}\!{\bfseries\akai 仓}\!\textgreater{}) }

(赵德芳\hspace{20pt}~

\setlength{\hangindent}{60pt}{ 【{\akai 二黄散板}】休把我当作了放箭之人。) }

(杨延昭睡介)

(杨宗保\hspace{20pt}~

贤爷驾到。)

杨延昭\hspace{20pt}~

哦!

(赵德芳坐堂桌大边侧,{\textless{}凤点头\textgreater{}})

杨延昭

\setlength{\hangindent}{60pt}{ 【{\akai 二黄散板}】听说贤爷驾到临,(\textless{}\!{\bfseries\akai 仓}\!\textgreater{}) }

杨延昭\hspace{20pt}~

\setlength{\hangindent}{60pt}{ 【{\akai 二黄散板}】宗保儿替为父赔罪负荆。 }

(杨延昭醒,{\textless{}住头\textgreater{}})

(杨宗保\hspace{20pt}~

千岁恕罪。)

(赵德芳\hspace{20pt}~

平身,赐座。)

(杨宗保\hspace{20pt}~

谢千岁。)

(杨宗保起身,坐大边,\textless{}\!{\bfseries\akai 台}\!\textgreater{})

(赵德芳\hspace{20pt}~

御妹夫此病从何而起?)

杨延昭\hspace{20pt}~

{\textless{}小锣叫头\textgreater{}}唉!(

\textless{}\!{\bfseries\akai 台}\!\textgreater{})

杨延昭\hspace{20pt}~

贤爷呀,呃$\cdots{}\cdots{}$(哭介)

杨延昭

\setlength{\hangindent}{60pt}{ 【{\akai 二黄快三眼}】自那日朝罢归身罹疾病({\akai 或}: 身染重病),三更时梦呃见了年迈爹尊呐。臣前番({\akai 或}: 我前番)命孟良骸骨搬请,那乃是萧天佐以假为真({\akai 或}: 弄假为真)。真骸骨伊藏在\footnote{ 《说戏誊稿》作``已藏在'',据樊百乐君告知,刘曾复先生作``伊藏在''。作``匿藏在''似亦可。{477}}({\akai 或}: 藏至在)洪羊洞,望乡台第三层那才是真。二次里命孟良番营来进,又谁知焦克明他私自后跟呐。老军报他二人在洪羊洞丧命,去掉我左右膀难以飞行。为此事终日里忧愁急窘({\akai 或}: 忧成疾病),因此上臣的病重加十分。千岁爷呀! }

(赵德芳

\setlength{\hangindent}{60pt}{ 【{\footnotesize 接}\akai 二黄原板}】御妹夫休得要心中烦闷,焦、孟将他二人难以复生。宗保儿近前来听王命: ) }

(杨宗保站)

(赵德芳\hspace{20pt}~

\setlength{\hangindent}{60pt}{ 【{\akai 二黄原板}】后堂内快请出儿祖母、娘亲。) }

(杨宗保向外)

(杨宗保\hspace{20pt}~

有请祖母、娘亲。)

(赵德芳下场门下,{\textless{}撞金钟\textgreater{},佘太君、柴夫人}上场门上)

(佘太君\hspace{20pt}~

\setlength{\hangindent}{60pt}{ 【{\akai 二黄摇板}】忽听宗保一声请,) }

(柴夫人\hspace{20pt}~

\setlength{\hangindent}{60pt}{ 【{\akai 二黄摇板}】急忙前来问分明。) }

(佘太君、柴夫人挖进去,站小边,杨宗保站大边)

(佘太君\hspace{20pt}~

醒来。)

杨延昭

\setlength{\hangindent}{60pt}{ 【{\akai 二黄导板\footnote{ 《说戏誊稿》记作{ }

【{\akai 导板}】

【{\akai 散板}】,刘曾复先生在文稿中注明这是``三个导板之一'',此处记作

【{\akai 导板}】。}{478}}】我方才朦胧荏苒\footnote{ ``荏苒''是逡巡、一刹那的意思。{479}}动啊,(\textless{}\!{\bfseries\akai 仓}\!\textgreater{})

杨延昭

\setlength{\hangindent}{60pt}{ 【{\akai 二黄散板}】耳旁又听有人声。(\textless{}\!{\bfseries\akai 仓}\!\textgreater{}) }

杨延昭

\setlength{\hangindent}{60pt}{ 【{\akai 二黄散板}】睁开了昏花眼难以扎挣,(\textless{}\!{\bfseries\akai 仓}\!\textgreater{}) }

杨延昭\hspace{20pt}~

哎呀!

({\textless{}凤点头\textgreater{}})

杨延昭

\setlength{\hangindent}{60pt}{ 【{\akai 二黄散板}】抬头只见儿的老娘亲呐。(\textless{}\!{\bfseries\akai 仓}\!\textgreater{}) }

杨延昭

\setlength{\hangindent}{60pt}{ 【{\akai 二黄散板}】生下了孩儿人七个,到如今白发人反送了黑发人呐,儿的娘啊! }

(\textless{}\!{\bfseries\akai 顷仓}\!\textgreater{})

杨延昭\hspace{20pt}~

\setlength{\hangindent}{60pt}{ 【{\footnotesize 接}\akai 二黄散板}】好不伤情。 }

(众哭,{\textless{}凤点头\textgreater{}})

杨延昭

\setlength{\hangindent}{60pt}{ 【{\akai 二黄散板}】舍不得宗保儿无人教训,(\textless{}\!{\bfseries\akai 仓}\!\textgreater{}) }

杨延昭

\setlength{\hangindent}{60pt}{ 【{\akai 二黄散板}】实难舍柴夫人结发之情呐。(\textless{}\!{\bfseries\akai 仓}\!\textgreater{}) }

杨延昭\hspace{20pt}~

\setlength{\hangindent}{60pt}{ 【{\akai 二黄散板}】宗保儿喏、(\textless{}\!{\bfseries\akai 仓}\!\textgreater{}) }

杨延昭\hspace{20pt}~

\setlength{\hangindent}{60pt}{ 【{\footnotesize 接}\akai 二黄散板}】柴夫人呐(\textless{}\!{\bfseries\akai 顷仓}\!\textgreater{}) }

杨延昭

\setlength{\hangindent}{60pt}{ 【{\footnotesize 接}\akai 二黄散板}】将我(\textless{}\!{\bfseries\akai 仓仓仓仓仓才仓}\!\textgreater{}) }

杨延昭\hspace{20pt}~

\setlength{\hangindent}{60pt}{ 【{\footnotesize 接}\akai 二黄散板}】搀定, }

(检场撤桌。宗保、柴夫人搀杨延昭站,走到台口,佘太君站小边,杨延昭等跪)

杨延昭

\setlength{\hangindent}{60pt}{ 【{\akai 二黄散板}】一家人跪埃尘叩谢圣恩呐。(\textless{}\!{\bfseries\akai 仓}\!\textgreater{}) }

杨延昭

\setlength{\hangindent}{60pt}{ 【{\akai 二黄散板}】恕为臣\footnote{ ``为臣''作``微臣''似亦可。{480}}再不能社稷重整,恕为臣再不能扶保乾坤。(\textless{}\!{\bfseries\akai 仓}\!\textgreater{}) }

杨延昭\hspace{20pt}~

\setlength{\hangindent}{60pt}{ 【{\akai 二黄散板}】霎时间(\textless{}\!{\bfseries\akai 仓}\!\textgreater{}) }

杨延昭\hspace{20pt}~

\setlength{\hangindent}{60pt}{ 【{\footnotesize 接}\akai 二黄散板}】心内痛啊({\akai 或}: 腹内痛啊) }

(\textless{}\!{\bfseries\akai 顷仓}\!\textgreater{})

杨延昭

\setlength{\hangindent}{60pt}{ 【{\footnotesize 接}\akai 二黄散板}】鲜血({\akai 或}: 心血)上(\textless{}\!{\bfseries\akai 仓仓仓仓仓才仓}\!\textgreater{}) }

(众起身,四鬼魂两边上)

杨延昭\hspace{20pt}~

\setlength{\hangindent}{60pt}{ 【{\footnotesize 接}\akai 二黄散板}】涌啊, }

(杨延昭吐介)

{呜$\cdots{}\cdots{}$(}\textless{}\!{\bfseries\akai 乱锤}\!\textgreater{}{)}

{(\textless{}凤点头\textgreater{})}

杨延昭\hspace{20pt}~

\setlength{\hangindent}{60pt}{ 【{\akai 二黄散板}】我面前站定了许多鬼魂:  }

(焦赞外、岳胜里小边,孟良外、老令公里大边)

杨延昭

\setlength{\hangindent}{60pt}{ 【{\akai 二黄散板}】焦克明气昂昂他的心、心怀不忿,那(、那)孟、孟佩苍他那里拱手相迎。 }

(焦赞、岳胜换位,{\textless{}凤点头\textgreater{}})

杨延昭

\setlength{\hangindent}{60pt}{ 【{\akai 二黄散板}】这一旁站定了勇将岳胜,(\textless{}\!{\bfseries\akai 仓}\!\textgreater{}) }

(孟良、老令公换位)

杨延昭\hspace{20pt}~

唉呀!

(\textless{}\!{\bfseries\akai 乱锤}\!\textgreater{},杨延昭跪老令公前,{\textless{}凤点头\textgreater{}})

杨延昭

\setlength{\hangindent}{60pt}{ 【{\akai 二黄散板}】抬头只见老严亲呐。(\textless{}\!{\bfseries\akai 仓}\!\textgreater{}) }

杨延昭\hspace{20pt}~

\setlength{\hangindent}{60pt}{ 【{\akai 二黄散板}】哭一声(\textless{}\!{\bfseries\akai 仓}\!\textgreater{}) }

杨延昭\hspace{20pt}~

\setlength{\hangindent}{60pt}{ 【{\footnotesize 接}\akai 二黄散板}】老爹尊({\akai 或}: 老爹爹) }

(\textless{}\!{\bfseries\akai 顷仓}\!\textgreater{})

杨延昭

\setlength{\hangindent}{60pt}{ 【{\footnotesize 接}\akai 二黄散板}】黄泉路呃(\textless{}\!{\bfseries\akai 仓仓仓仓仓才仓}\!\textgreater{}) }

杨延昭\hspace{20pt}~

\setlength{\hangindent}{60pt}{ 【{\footnotesize 接}\akai 二黄散板}】等啊, }

(\textless{}\!{\bfseries\akai 乱锤}\!\textgreater{}杨延昭洒,杨延昭众站台口正面,杨延昭吐介)

{呜$\cdots{}\cdots{}$(}\textless{}\!{\bfseries\akai 顷仓}\!\textgreater{}{)呜$\cdots{}\cdots{}$(}\textless{}\!{\bfseries\akai 顷仓}\!\textgreater{}\textless{}\!{\bfseries\akai 叭嗒仓}\!\textgreater{}{)呜$\cdots{}\cdots{}$}

{(}\textless{}\!{\bfseries\akai 凤点头}\!\textgreater{})

杨延昭\hspace{20pt}~

\setlength{\hangindent}{60pt}{ 【{\akai 二黄散板}】无常到万事休去见先人。 }

(\textless{}\!{\bfseries\akai 嘟仓}\!\textgreater{},杨延昭死介,倒坐在台口椅上,{唢呐}{[}{吹打}{]}

\textless{}\!{\bfseries\akai 牌子}\!\textgreater{},四鬼魂合拢挡介,杨延昭解头上绸条和腰包,腰包搭椅背上、绸条折成两条搭在腰包上,四鬼魂领杨延昭下,孟良、老令公、杨延昭、岳胜、焦赞``搭轿''下)

({唢呐}停,赵德芳上,站大边,众哭,\textless{}\!{\bfseries\akai 凤点头}\!\textgreater{})

(佘太君\hspace{20pt}~

\setlength{\hangindent}{60pt}{ 【{\akai 二黄散板}】见此情不由人心中酸痛,) }

(柴夫人\hspace{20pt}~

\setlength{\hangindent}{60pt}{ 【{\akai 二黄散板}】撇下了母子们好不伤情。) }

({\textless{}凤点头\textgreater{}})

(赵德芳\hspace{20pt}~

\setlength{\hangindent}{60pt}{ 【{\akai 二黄散板}】劝太君和御妹须要珍重,宗保儿随为王金殿面君。) }

(赵德芳出门,杨宗保出门,赵德芳下场门下,杨宗保跟下,同时佘太君上场门下,柴夫人托腰包和绸条随下)

(柴夫人取腰包、绸条时检场撤台口椅,在\textless{}\!{\bfseries\akai 尾声}\!\textgreater{}中下场)

{附}: 关于《洪羊洞》阴魂人物扮相: 

{四鬼卒}: 龙套扮演,本脸,小鬼发、青袍、卒坎、黑风旗。

{老令公}: 金大镫、白满、白蟒、玉带;

{岳胜}: 忠纱、黑三、绿蟒、玉带;

{孟良}: 硬扎巾、黪红紥、红蟒、玉带;

{焦赞}: 硬扎巾、黪紥、黑蟒、玉带。

老令公、岳胜、孟良、焦赞{四人都拿云帚};岳胜、孟良、焦赞{戴黑纱},老令公{不戴黑纱}。

{杨延昭}: 

死后解去绸条、腰包,剩下员外巾、古铜褶子,不戴黑纱,空手,左手拉老令公云帚尾下。

\newpage

\hypertarget{ux5929ux96f7ux62a5-ux4e4b-ux5f20ux5143ux79c0ux8d3aux6c0f}{%

\subsection{天雷报 之

张元秀、贺氏}\label{ux5929ux96f7ux62a5-ux4e4b-ux5f20ux5143ux79c0ux8d3aux6c0f}}

{{[}第一场{]}}

贺氏

\textless{}\!{\bfseries\akai 哭相思}\!\textgreater{}{[}{\akai 引子}{]}娇儿一去无音信,倒教老身挂在心。

贺氏

({\akai 念})有子无钱终有靠,有钱无子枉徒劳。恩养一子防年老,数点\footnote{ 段公平君建议作``数年''。{481}}心血一旦抛。

贺氏

老身贺氏,配夫张元秀为妻,夫妻二人在这永寿街前开了一座豆腐坊,无非是糊口而已。只因那年在周梁桥下拾得一子,取名继保,多亏我抚养了他一十三载,才得长大成人。可恨我那个老天杀的,今日也打,明日也骂,还要赶出门外({\akai 或}: 还要赶奔在外)。(也不知是他的亲娘不是他的亲娘,无凭无据地将儿认了去了啊,呃$\cdots{}\cdots{}$(哭介))是我朝思暮想,就想出了一场病,呃$\cdots{}\cdots{}$(哭介)

贺氏

唉------我今日略觉好了些。我不免将这个老天杀的({\akai 或}: 我不免将那个老天杀的),唤将出来,我说他几句,出出我这口恶气!

贺氏\hspace{30pt}~

老老,你这个老天杀的,你与我走了出来呦!

张元秀\hspace{20pt}~

来了!

张元秀

\textless{}\!{\bfseries\akai 哭相思}\!\textgreater{}{[}{\akai 引子}{]}年纪迈,血气衰,年迈无儿绝后代。

贺氏\hspace{30pt}~

你这个老天杀的!(你与我走了出来呀。)

张元秀\hspace{20pt}~

唉!

张元秀

\textless{}\!{\bfseries\akai 哭相思}\!\textgreater{}听妈妈哭声唤悲哀,莫不是为娇儿失却恩爱。

张元秀\hspace{20pt}~

妈妈,今日为何起来甚早啊?({\akai 或}: 妈妈,你今日起床甚早啊。)

贺氏\hspace{30pt}~

我今日略觉好了些,(你)怎么不教我起床甚早啊。

张元秀\hspace{20pt}~

不是哟,你乃是久病之人({\akai 或}: 有病之人)呐。

贺氏\hspace{30pt}~

不错,我是久病之人({\akai 或}: 有病之人),你可晓得,我这病是从何而起的呢?

张元秀\hspace{20pt}~

不过是打从那不孝的奴才身上所起(的)罢。

贺氏\hspace{30pt}~

不错,是打从娇儿身上所起。也是你这个老天杀的你气得我啊!

张元秀

(吾怎么会气得你的,\footnote{ 此处据《余叔岩与余派艺术》\textsuperscript{{[}21{]}.}收录的余叔岩的手抄谭派《天雷报》本加。{482}})

贺氏

(我好端端的一个儿子,你终朝打骂,还要赶奔在外,哼,这不是你气得我?!)

张元秀\hspace{20pt}~

养儿子哪有不教训的?

贺氏\hspace{30pt}~

教训?谁像你({\akai 或}: 有像你)那样地教训(的吗)?

今日也打,明日也骂,还要赶出门外。是我朝思暮想,才想出这场病(来)哟,呃$\cdots{}\cdots{}$(哭介)

贺氏

\setlength{\hangindent}{60pt}{ 【四平调】思前事不由我恼心怀({\akai 或}: 思前事不由人恼心怀;未开言不由人恼心怀),出言埋怨老无才。好好一子你无福载({\akai 或}: 好好一子你无福待),一心将呃他赶出了门外。 }

张元秀

唉!那日我赶到了青风亭\footnote{ ``青风亭''从《余叔岩与余派艺术》收录的余叔岩的手抄谭派《天雷报》本。{483}},偏偏就遇着他亲娘到来,说得字字不差,而况又有血书为证呐,所以教她才认了去了。

贺氏

慢说是一个人呐,就是一只鸡犬,难道说白白地就让她认了去吗({\akai 或}: 教她认了去吗)?

张元秀

\setlength{\hangindent}{60pt}{ 【四平调】青风亭遇着他的亲娘到来,教我无计可奈。纵然教她认了呃去,并非是妈妈十月怀胎。 }

贺氏\hspace{30pt}~

虽不是我十月怀胎,也亏我抚养了他一十三载。

慢说是一个人呐,就是一块石头,被我今日磨,明日磨,也磨啊,也光了哦。({\akai 或}: 被我今日磨,明日磨,这磨啊,也磨光了哦。)

贺氏

\setlength{\hangindent}{60pt}{ 【四平调】虽不是我十月怀胎,也亏我抚养他一十三呐载。眼前若有娇儿在,万事全休无有话来。 }

张元秀

啊,我不埋怨你,你倒埋怨起我来了。(旁人娶妻,为的是生子。)我自从娶了你这个(唉,)老乞婆,你又不生男,(你)又不养女。你呀,(你)绝了我张门的宗嗣了!

贺氏

哦哦哦$\cdots{}\cdots{}$我、我不生不养,我、我不生不养,也是你张门祖上的阴功德行呐。

贺氏

\setlength{\hangindent}{60pt}{ 【四平调】我不生不养(你)前世债,无有儿子你埋怨何来({\akai 或}: 你埋怨我来)。 }

张元秀

\setlength{\hangindent}{60pt}{ 【四平调】这才是年高呃迈、血气衰,前世欠下了儿女债。你苦苦与我来撒赖,活活地逼我赴哇泉台。 }

贺氏

赴泉台$\cdots{}\cdots{}$哦,这是要死啊?!你死,呃,我也死。要死,我们大家一起来死啊({\akai 或}: 我们大家死在一块哟)。

张元秀

\setlength{\hangindent}{60pt}{ 【四平调】老无才大不该,说你几呃句你反撒赖。你也该死我也该埋,倒不如一死两撒开! }

张元秀\hspace{20pt}~

(哦,)两撒开,好啊(,就两撒开),难道说我这条老命还拚不过你吗?

贺氏\hspace{30pt}~

我这条老命,(哼,)还拚不过你吗?

张元秀\hspace{20pt}~

我把你这个老乞婆!

贺氏\hspace{30pt}~

我把你这个老天杀的!

张元秀\hspace{20pt}~

嘿嘿,

你这是气我啊!

贺氏\hspace{30pt}~

你这是呕我啊!

张元秀

我,我要打你了,我要打你呀。\footnote{ 陈超老师介绍: 老生念完``我要打你呀'',拿棍打三下,正好转到大边,双进门,三碰头,三甩髯,再扶老旦。{谭派没有摔倒后与老旦屁股对碰},{又倒下的表演}。{484}}

贺氏\hspace{30pt}~

你要打我?

张元秀、贺氏 打哟!

(张元秀、贺氏打介)

贺氏

呃$\cdots{}\cdots{}$老老!(老老!你、)你回来呀,呃$\cdots{}\cdots{}$(哭介)你死了({\akai 或}: 你走了),(可)就苦了我了,呃$\cdots{}\cdots{}$(哭介)

张元秀\hspace{20pt}~

呃$\cdots{}\cdots{}$(哭介)

贺氏

老老你回来,呃$\cdots{}\cdots{}$(哭介)你不要气,你屈了我了,呃$\cdots{}\cdots{}$(哭介)

张元秀

唉,(我)不要你思念那个奴才,你偏偏要思念那个奴才。那个奴才天良丧尽,终究地养不得家的呀。({\akai 或}: 那奴才人大心大,终究是养不得家的呀。)

贺氏\hspace{30pt}~

呃,是是是,呃,(我不想他,)我不想他了。

张元秀\hspace{20pt}~

呃,不要想他了。

贺氏

呃呃呃$\cdots{}\cdots{}$啊,老老,我要出去望望啊({\akai 或}: 我要到外面望望)。

张元秀\hspace{20pt}~

诶,你乃是久病之人,外面的风大呀!

贺氏\hspace{30pt}~

你又来气我。

张元秀\hspace{20pt}~

(哦哦哦,)好好好,(待)我与你开了门。(你望望就进来。)

贺氏\hspace{30pt}~

(呃,这$\cdots{}\cdots{}$)我望望就进来。

张元秀\hspace{20pt}~

望望就是了。({\akai 或}: 好好好,望望就回来。)

贺氏\hspace{30pt}~

诶哟嚯$\cdots{}\cdots{}$

张元秀\hspace{20pt}~

如何,如何,我的话是不错的吧({\akai 或}: 我这话是不错的吧)?

贺氏\hspace{30pt}~

嗯,我还要望望。({\akai 或}: 哦,我望望就进来。)

张元秀\hspace{20pt}~

还要望望。

(贺氏\hspace{30pt}~

望望就进来。)

(张元秀\hspace{20pt}~

好。)

(张元秀、贺氏望介)

贺氏\hspace{30pt}~

啊老老,这一条大路是往哪里去的?

张元秀\hspace{20pt}~

呃------呃,这是往东京去的。({\akai 或}: 这条,东京去的。)

贺氏\hspace{30pt}~

哦往东京去的------(呃,老老,)这一条呢?

张元秀\hspace{20pt}~

这是往荆州去的。({\akai 或}: 这是往荆襄去的大路哇。)

贺氏\hspace{30pt}~

哦这是往荆襄去的------(啊)老老,中间这条大路呢({\akai 或}: 中间这条道路呢)?

张元秀\hspace{20pt}~

中间这条道路么,嗯,这就是往青风亭去的大路呃。

贺氏\hspace{30pt}~

娇儿({\akai 或}: 你我的儿子,)可是打(从)此道而去?

张元秀\hspace{20pt}~

正是打(从)此道而去。

贺氏\hspace{30pt}~

我们要叫哇!

张元秀\hspace{20pt}~

叫哇!

张元秀\hspace{20pt}~

\textless{}\!{\bfseries\akai 叫头}\!\textgreater{}张继保!

贺氏\hspace{30pt}~

\textless{}\!{\bfseries\akai 叫头}\!\textgreater{}小娇儿!

张元秀\hspace{20pt}~

儿打此道而去,({\akai 或}: 儿打此路而去)

贺氏\hspace{30pt}~

不打此道而回。({\akai 或}: 儿打此路而回)

张元秀\hspace{20pt}~

为父的在此盼你呀。

贺氏\hspace{30pt}~

为娘的在此想你,你$\cdots{}\cdots{}$(哭介)

张元秀\hspace{20pt}~

\textless{}\!{\bfseries\akai 叫头}\!\textgreater{}张继保!

贺氏

\textless{}\!{\bfseries\akai 叫头}\!\textgreater{}小娇儿!喂呀,儿$\cdots{}\cdots{}$(哭介)

张元秀、贺氏 儿、儿啊$\cdots{}\cdots{}$(哭介)

张元秀\hspace{20pt}~

\setlength{\hangindent}{60pt}{ 【{\akai 二黄摇板}】到如今呐路在人不在, }

贺氏\hspace{30pt}~

\setlength{\hangindent}{60pt}{ 【{\akai 二黄摇板}】狠心的娇儿(你)不回来。 }

张元秀\hspace{20pt}~

\setlength{\hangindent}{60pt}{ 【{\akai 二黄摇板}】儿再不能随为父打草鞋, }

贺氏\hspace{30pt}~

\setlength{\hangindent}{60pt}{ 【{\akai 二黄摇板}】儿再不能随为娘把磨来捱。 }

张元秀、贺氏

\setlength{\hangindent}{60pt}{ 【{\akai 二黄摇板}】哭一声娇儿今何\textless{}\!{\bfseries\akai 哭头}\!\textgreater{}在, }

贺氏\hspace{30pt}~

呜呜呜$\cdots{}\cdots{}$

(张元秀\hspace{20pt}~

啊------妈妈,妈妈,妈妈------)

张元秀\hspace{20pt}~

\setlength{\hangindent}{60pt}{ 【{\akai 二黄摇板}】可怜你气堵咽喉倒在怀。 }

张元秀\hspace{20pt}~

妈妈,妈妈,妈妈!

贺氏\hspace{30pt}~

呃$\cdots{}\cdots{}$老老。

张元秀\hspace{20pt}~

妈妈。

贺氏\hspace{30pt}~

呃$\cdots{}\cdots{}$儿啊$\cdots{}\cdots{}$(哭介)

张元秀

不要你想念那个({\akai 或}: 这个)奴才,你偏偏要想念那个({\akai 或}: 这个)奴才。({\akai 或}: 你不要想他了,那个奴才天良丧尽,终究是养不得家的呀。)

贺氏\hspace{30pt}~

是,是,是,我不想他了,呃,我们回去罢!

张元秀\hspace{20pt}~

啊,回去罢!

张元秀\hspace{20pt}~

唉!

张元秀\hspace{20pt}~

({\akai 念})周梁桥下一婴孩,

贺氏\hspace{30pt}~

({\akai 念})夫妻将养十三载。

张元秀\hspace{20pt}~

({\akai 念})早知奴才良心坏呀,

贺氏\hspace{30pt}~

({\akai 念})当初不该捡回来。

张元秀\hspace{20pt}~

是我错了,呃,错了,回去罢。

贺氏\hspace{30pt}~

呃,老老,你回来。({\akai 或}: 啊,老老。)

张元秀\hspace{20pt}~

做什么?({\akai 或}: 哦,看什么?)

贺氏\hspace{30pt}~

呃,你来看------你我的儿子(他)回来了!

张元秀\hspace{20pt}~

哦?(在哪里?)

贺氏\hspace{30pt}~

你来看。

张元秀\hspace{20pt}~

诶,在哪里?

贺氏\hspace{30pt}~

大树底下乘凉的,那不是你我的儿子么?

张元秀\hspace{20pt}~

诶,那不是你我的儿子啊。

贺氏\hspace{30pt}~

呃,他是哪个?

张元秀\hspace{20pt}~

那是放牛的牧童啊。

贺氏\hspace{30pt}~

呃,那你我的儿子呢?

张元秀\hspace{20pt}~

唉,在这里呀。({\akai 或}: 唉,在这厢呢!)

张元秀\hspace{20pt}~

\textless{}\!{\bfseries\akai 叫头}\!\textgreater{}张继保!

贺氏\hspace{30pt}~

\textless{}\!{\bfseries\akai 叫头}\!\textgreater{}小娇儿!

张元秀、贺氏 喂呀儿,儿啊,呃$\cdots{}\cdots{}$(哭介)

{{[}第二场{]}}

张继保中状元,见生身父母,生父建议他认下养父、养母,张继保不愿意,生母向他展示血书。

{(略)}

{{[}第三场{]}}

(周小哥

({\akai 念})\textless{}\!{\bfseries\akai 水底鱼}\!\textgreater{}结彩悬旗------结彩悬旗是鲜明要整齐,两旁站立休要笑嘻嘻,是休要笑嘻嘻。\footnote{ 刘曾复先生介绍此处用的\textless{}\!{\bfseries\akai 水底鱼}\!\textgreater{}也是《麒麟阁》贺芳追秦琼过场用的,原词是``急急前行,紧走莫稍停,将他拿住管教一命倾,管教一命倾''。《双背凳》里的\textless{}\!{\bfseries\akai 水底鱼}\!\textgreater{}词句是``自幼癫狂,管着媳妇叫亲娘,如若不叫,嘴巴脸上量,嘴巴脸上量'';``自幼赖呆,管着媳妇叫奶奶,如若不叫,嘴巴脸上掴,嘴巴脸上掴''。{485}})

{{[}第四场{]}}

张元秀\hspace{20pt}~

走哇!({\akai 或}: 唉------)

(张元秀搀扶贺氏上)

张元秀\hspace{20pt}~

\setlength{\hangindent}{60pt}{ 【{\akai 二黄摇板}】老来无嗣绝后传({\akai 或}: 老来无子接后传)呐, }

贺氏\hspace{30pt}~

\setlength{\hangindent}{60pt}{ 【{\akai 二黄摇板}】行也难来坐也难。 }

贺氏\hspace{30pt}~

喂呀,呃$\cdots{}\cdots{}$(哭介)

张元秀\hspace{20pt}~

啊,妈妈,你(歇得是)怎么样了?

贺氏\hspace{30pt}~

(我)腹中饥饿,难以行走啊。

张元秀\hspace{20pt}~

唉,倒也可怜!

张元秀

妈妈,看前面有一大户人家,待我前去讨些汤水来吃。({\akai 或}: 妈妈,你我到前面大户人家,讨些汤水吃。)

贺氏\hspace{30pt}~

我走不动呃。

张元秀\hspace{20pt}~

(哦,)你走不动呃。

贺氏\hspace{30pt}~

你来搀我$\cdots{}\cdots{}$

张元秀\hspace{20pt}~

是是是,唉,待我来搀你呀。

张元秀\hspace{20pt}~

\setlength{\hangindent}{60pt}{ 【{\akai 二黄摇板}】屋漏偏遭连阴雨呀, }

贺氏\hspace{30pt}~

\setlength{\hangindent}{60pt}{ 【{\akai 二黄摇板}】破船又遇当头风。 }

张元秀

唉!我当大户人家({\akai 或}: 我道是什么大户人家),偏偏又来到这个讨厌的亭子上了。

贺氏\hspace{30pt}~

老老,(你)到了无有?

张元秀\hspace{20pt}~

(呃,)还未曾到啊。

贺氏\hspace{30pt}~

我们歇息歇息。

张元秀\hspace{20pt}~

哦,歇息歇息再走哇。

贺氏\hspace{30pt}~

啊老老,这是什么所在呀?!

张元秀\hspace{20pt}~

这就是青风亭!

贺氏\hspace{30pt}~

哦,这就是青风亭------依我看来,不叫作青风亭。

张元秀\hspace{20pt}~

叫什么?

贺氏\hspace{30pt}~

要叫(、叫坐)``望儿亭''!

张元秀\hspace{20pt}~

诶,``断肠亭''啊!

张元秀\hspace{20pt}~

张继保!

贺氏\hspace{30pt}~

小娇儿!

张元秀、贺氏 喂呀,儿啊,呃$\cdots{}\cdots{}$(哭介)

张元秀\hspace{20pt}~

\setlength{\hangindent}{60pt}{ 【{\akai 二黄散板}】到如今呐亭在人不在, }

贺氏\hspace{30pt}~

\setlength{\hangindent}{60pt}{ 【{\akai 二黄散板}】水流千遭不回来。 }

张元秀

\setlength{\hangindent}{60pt}{ 【{\akai 二黄散板}】哭一声娇儿啊今何\textless{}\!{\bfseries\akai 哭头}\!\textgreater{}在, }

贺氏\hspace{30pt}~

\setlength{\hangindent}{60pt}{ 【{\akai 二黄散板}】我二老过世呀({\akai 或}: 下世)无有人埋。 }

(周小哥

诶,这儿$\cdots{}\cdots{}$打扫亭子。哦哟,这儿怎么来两位,这儿歇着了。)

(周小哥\hspace{20pt}~

诶哟,这不是张家伯伯么?待我叫他。)

(周小哥\hspace{20pt}~

啊,张伯伯------)

贺氏\hspace{30pt}~

啊老老,外面有人唤你去。({\akai 或}: 啊老老,外厢有人唤你呢。)

张元秀\hspace{20pt}~

啊,外面有人唤我?({\akai 或}: 怎么,有人唤我?)不错,待我看来。

贺氏\hspace{30pt}~

呃,老老,外面的风大,快些进来。

张元秀\hspace{20pt}~

晓得------是哪一位呀?

(周小哥\hspace{20pt}~

嘿,张伯伯,是我。)

张元秀\hspace{20pt}~

哦,原来是周小哥。

(周小哥\hspace{20pt}~

是我呀。)

张元秀\hspace{20pt}~

(惊介)你哪里来得这身荣耀哇?

(周小哥\hspace{20pt}~

嗨,这不是状元老爷让我当了此地的地方。)

张元秀\hspace{20pt}~

哦,当了此处(的)地方,可喜可贺!

(周小哥\hspace{20pt}~

诶,这就是$\cdots{}\cdots{}$哎,我说您二位怎么着?)

张元秀

唉!再休提起。只因你那继保兄弟逃奔\footnote{ 姜骏按: ``逃奔在外''亦可作``逃门在外'',``逃门''即因为某种缘故逃出家门,下同。{486}}在外,我二老双双染病在床,生意难做,故而落在这乞讨之途了哇。

(周小哥\hspace{20pt}~

诶呀,真是$\cdots{}\cdots{}$)

(周小哥

诶,我说张伯伯,我倒想------我怎么瞅着咱们这个状元老爷,像、像$\cdots{}\cdots{}$像我这继保哥$\cdots{}\cdots{}$)

贺氏\hspace{30pt}~

哎呀儿啊!

(周小哥\hspace{20pt}~

您,您,您$\cdots{}\cdots{}$您这是做什么?)

贺氏\hspace{30pt}~

儿啊!

(周小哥\hspace{20pt}~

您,您,您$\cdots{}\cdots{}$您这是说什么呀。)

张元秀\hspace{20pt}~

呃------他不是继保啊。({\akai 或}: 他不是你我的儿子。)

贺氏\hspace{30pt}~

是哪个?({\akai 或}: 他是哪个。)

张元秀\hspace{20pt}~

他是周小哥啊。

贺氏\hspace{30pt}~

呃,我说不像呀。

(周小哥\hspace{20pt}~

嗨,您坐下。您坐下歇会儿吧,歇会儿吧,歇会儿吧。)

贺氏\hspace{30pt}~

他说什么继保?

张元秀

他说新科状元像你我的儿子啊。呃,我说他不像。那奴才天良丧尽,无有这样的福气({\akai 或}: 无有这样的造化)。

(贺氏\hspace{30pt}~

是啊。)

(张元秀\hspace{20pt}~

呃,无有这样福气!)

张元秀\hspace{20pt}~

是呀。

(周小哥

我瞅着这个状元老爷像咱们的继保兄弟,明儿他在这儿打坐,诶,您去认认看,看看是不是。)

张元秀\hspace{20pt}~

呃,他们的人多,我二老挨挤不上啊。

(周小哥\hspace{20pt}~

我这不是当了地方么,我给您轰散闲人,我把您就给让进去了。)

张元秀

呃,好好好,少时状元老爷打坐,你来叫我一声({\akai 或}: 你要叫我一声),呃,叫我一声,叫我一声就是。

张元秀\hspace{20pt}~

呵呵,好好好,(如此说来,)有劳你了,有劳你了!

(周小哥

您、您、您也甭什么$\cdots{}\cdots{}$得了,得了,得了$\cdots{}\cdots{}$咱们明儿见吧!)

张元秀\hspace{20pt}~

哈哈哈$\cdots{}\cdots{}$(笑介)

贺氏\hspace{30pt}~

呵,老老,你笑的什么?

张元秀\hspace{20pt}~

呵,你不曾听见呐?

贺氏\hspace{30pt}~

听见什么?

张元秀\hspace{20pt}~

你我的儿子中了(新科)状元。少不得我就是老太爷。

贺氏\hspace{30pt}~

呃,(那)我呢?

张元秀\hspace{20pt}~

太夫人呐。

贺氏\hspace{30pt}~

诶呀,哪有这样的太夫人呐。

(张元秀\hspace{20pt}~

如此说来,你我要称唤称唤。)

(贺氏\hspace{30pt}~

我们要称唤称唤。)

张元秀\hspace{20pt}~

啊------那旁来的敢是太夫人么?

贺氏\hspace{30pt}~

那旁来的敢是太老爷呀?

贺氏\hspace{30pt}~

如此说来,呃,太老爷请------

张元秀\hspace{20pt}~

呃呃呃,儿子是你养的,还是太夫人请。

贺氏\hspace{30pt}~

儿子是你捡来的,还是太老爷请------

张元秀\hspace{20pt}~

你我是恩爱夫妻,挽手而行!

贺氏\hspace{30pt}~

走走走!

张元秀\hspace{20pt}~

慢来慢来,你的病怎么样了({\akai 或}: 你的病呢)?

贺氏\hspace{30pt}~

呵,除了根了。

张元秀\hspace{20pt}~

呵呵哈哈哈$\cdots{}\cdots{}$(笑介)

{{[}第三场{]}}

(青袍、红旗站门,门子上)

(周小哥\hspace{20pt}~

这两位还还不来。)

(周小哥\hspace{20pt}~

这两位怎么还不见?)

张元秀\hspace{20pt}~

快些走!

快些走!

贺氏\hspace{30pt}~

呃,你慢些。

张元秀\hspace{20pt}~

呃,快些走。

贺氏\hspace{30pt}~

慢些个。

(周小哥\hspace{20pt}~

您来了------)

张元秀\hspace{20pt}~

呃,来了(,来了)$\cdots{}\cdots{}$状元老爷,可曾打坐?

(周小哥\hspace{20pt}~

打坐多时了,您呐,去看看吧。)

贺氏\hspace{30pt}~

好好好。

(周小哥\hspace{20pt}~

您瞧瞧是不是他。)

张元秀\hspace{20pt}~

妈妈,你在此等候。待我看来。

贺氏\hspace{30pt}~

你要看仔细。

张元秀\hspace{20pt}~

晓得。

张元秀\hspace{20pt}~

是的,呃,是(的)。不错,是他!是他。

(周小哥\hspace{20pt}~

怎么着?是他------伙计们,来两乘大轿!)

张元秀\hspace{20pt}~

呃,慢来慢来,两乘小轿罢,两乘小轿罢。

(周小哥\hspace{20pt}~

您这个,就别啬刻了!)

张元秀\hspace{20pt}~

啊呵呵哈哈哈$\cdots{}\cdots{}$(笑介)

贺氏\hspace{30pt}~

呃,可是你我的儿子?

张元秀\hspace{20pt}~

是的,是的。

张元秀\hspace{20pt}~

妈妈,你在此等候,待我前去({\akai 或}: 待我进去),他就认下了。

贺氏\hspace{30pt}~

(呃呃呃,)转来。({\akai 或}: 啊,老老,你回来。)

张元秀\hspace{20pt}~

做什么?

贺氏\hspace{30pt}~

儿子若是将你认下,你不要忘了我啊!

张元秀

诶------常言说得好: 少时夫妻老是伴\footnote{ 夏行涛君建议作``少时夫妻老时伴''。{487}},我岂肯丢下你这个老伴呐({\akai 或}: 我焉能忘得了你这个老伴呐)?!

贺氏\hspace{30pt}~

哦呵,好好好。

张元秀\hspace{20pt}~

哈哈哈$\cdots{}\cdots{}$(笑介)

张元秀\hspace{20pt}~

儿啊,为父的来了,就该快快下位迎接才是呀。

(张继保\hspace{20pt}~

哦,恩父来了,待我下位迎接------)

张元秀\hspace{20pt}~

着哇({\akai 或}: 是呀------)少年登科,可喜可贺。

(张继保\hspace{20pt}~

诶------那一老乞丐$\cdots{}\cdots{}$)

张元秀\hspace{20pt}~

啊?为父的姓张啊。

(张继保\hspace{20pt}~

这就不对了------)

张元秀\hspace{20pt}~

怎么不对了?

(张继保\hspace{20pt}~

(所略念白大意为,既是父子为何你姓张,我姓薛。))

张元秀\hspace{20pt}~

诶------这恩父义子原是不同姓的呀。

(张继保\hspace{20pt}~

$\cdots{}\cdots{}$有何为证?)

张元秀\hspace{20pt}~

有血书为证呐。

(张继保\hspace{20pt}~

$\cdots{}\cdots{}$血书今在何处$\cdots{}\cdots{}$)

张元秀

(呃,)这个$\cdots{}\cdots{}$哎呀儿啊,前些年也是在这个亭子上,被我儿抢了去了,难道说(就)忘怀了吗?

张继保\hspace{20pt}~

与我轰了下去!

(众\hspace{40pt}~

哦!)

张元秀\hspace{20pt}~

({\akai 念})吾儿暂熄雷霆之怒。

(众\hspace{40pt}~

哦!)

张元秀

({\akai 念})两旁撤去虎狼之威。容我这老乞丐------一言诉禀呐,呃$\cdots{}\cdots{}$(哭介)

张元秀

\setlength{\hangindent}{60pt}{ 【{\akai 二黄摇板}】未开言不由人泪汪汪,儿子老爷听端详啊: 儿怎不学丁郎刻木把双亲奉养,儿怎不学卧冰小王祥。哭一声娇儿啊将父\textless{}\!{\bfseries\akai 哭头}\!\textgreater{}认,喂呀我的儿啊------ }

张元秀\hspace{20pt}~

\setlength{\hangindent}{60pt}{ 【{\akai 二黄散板}】这奴才一旦丧天良。 }

张元秀\hspace{20pt}~

不认就罢!

贺氏\hspace{30pt}~

呃,怎么样了?({\akai 或}: 儿子可曾将你认下?)

张元秀\hspace{20pt}~

他不认了!

贺氏\hspace{30pt}~

他不认?

张元秀\hspace{20pt}~

他不认呐!

贺氏

(唉,这也难怪呀,)你终日打骂于他,我去,(他)就认下了。({\akai 或}: 唉,这也难怪,你当初尽打骂于他,待我进去,他必然就认下了。)

张元秀

着啊,你捧的是他({\akai 或}: 你疼的是他),你进去,他必定认下了({\akai 或}: 是啊,你疼的是他,你进去,诶,他就认下了啊。)

贺氏\hspace{30pt}~

我去!

张元秀\hspace{20pt}~

转来。({\akai 或}: 呃,呃,呃妈妈,你回来。)

贺氏\hspace{30pt}~

做什么?

张元秀

儿子要是将你认下,你不要忘了我啊!({\akai 或}: 倘若儿子将你认下,不要忘怀了我啊!)

贺氏\hspace{30pt}~

诶------你我是恩爱夫妻,我焉能忘了你这个老头子呀({\akai 或}: 老东西呀)。

张元秀\hspace{20pt}~

诶。

贺氏

儿啊,为娘的来了,快快下位迎接的才是呀。({\akai 或}: 啊儿啊,为娘的来了,就该下位迎接与我啊。)

张继保\hspace{20pt}~

(略)

贺氏\hspace{30pt}~

着哇。

张继保\hspace{20pt}~

与我轰了下去!

贺氏

哎呀儿啊,自从我儿离家呵。({\akai 或}: 哎呀儿啊,自从儿逃奔在外,为娘的朝思暮想,就想出这场病呐!)

贺氏

\setlength{\hangindent}{60pt}{ 【{\akai 二黄摇板}】哪一天不哭儿三两遍({\akai 或}: 三五遍),哪一夜不哭到五更寒。哭一声娇儿将娘\textless{}\!{\bfseries\akai 哭头}\!\textgreater{}认,我的儿啊, }

贺氏\hspace{30pt}~

\setlength{\hangindent}{60pt}{ 【{\akai 二黄散板}】这奴才一旦忘恩欺了天。 }

贺氏\hspace{30pt}~

喂呀$\cdots{}\cdots{}$(哭介)

张元秀\hspace{20pt}~

怎么样了?

贺氏\hspace{30pt}~

奴才他也是不认呐。

张元秀

哦,他也是不认。不认就拉倒,回去!({\akai 或}: 不认?好,就罢------我们回去。)

贺氏\hspace{30pt}~

回去?

张元秀\hspace{20pt}~

走啊!

贺氏\hspace{30pt}~

呃$\cdots{}\cdots{}$转来。({\akai 或}: 呃$\cdots{}\cdots{}$老老,你回来。)

张元秀\hspace{20pt}~

做什么?

贺氏

你我二老一同进去,苦苦哀求,倘若认下,也未可知。({\akai 或}: 你我二人一同进去,苦苦哀求于他,也许就认、认下了。)

张元秀\hspace{20pt}~

再若不认呢?

贺氏\hspace{30pt}~

唉,``若要好,大作小'',你我就屈他一膝({\akai 或}: 你我就跪他一跪)。

张元秀\hspace{20pt}~

怎么讲,还要屈他一膝?!

贺氏\hspace{30pt}~

跪跪何妨啊?

张元秀\hspace{20pt}~

天呐,这是我二老({\akai 或}: 这就是我们)无有儿子的下场头啊!

张元秀\hspace{20pt}~

儿子老爷,

贺氏\hspace{30pt}~

儿子太爷!

张元秀\hspace{20pt}~

把我二老,不要当作恩父、义母,({\akai 或}: 休把我二老当作恩父、义母,)

贺氏\hspace{30pt}~

就当作侍女、丫鬟。({\akai 或}: 就把我当作侍女、丫鬟。)

张元秀\hspace{20pt}~

儿有吃不了的残茶剩饭,

贺氏\hspace{30pt}~

赏与我二老一碗半碗。

张元秀\hspace{20pt}~

儿有穿不了的破衣破衫,

贺氏\hspace{30pt}~

赏与我二老遮寒。

张元秀\hspace{20pt}~

儿子老爷,({\akai 或}: 张继保!)

贺氏\hspace{30pt}~

儿子太爷!

({\akai 或}: 小娇儿!)

张元秀\hspace{20pt}~

你可曾听见呐?

贺氏\hspace{30pt}~

你可曾跪呀?(唉,儿啊------)你跪呀!

张元秀\hspace{20pt}~

诶,跪下呀!

张元秀、贺氏 喂呀儿,儿啊,呃$\cdots{}\cdots{}$(哭介)

(雷神上,摘张继保帽花,换插``雷''字,下)

张元秀\hspace{20pt}~

\setlength{\hangindent}{60pt}{ 【{\akai 二黄摇板}】泪汪汪跪在亭台上, }

贺氏\hspace{30pt}~

\setlength{\hangindent}{60pt}{ 【{\akai 二黄摇板}】儿子老爷听端详:  }

张元秀\hspace{20pt}~

\setlength{\hangindent}{60pt}{ 【{\akai 二黄摇板}】但愿你辈辈为宰相, }

贺氏\hspace{30pt}~

\setlength{\hangindent}{60pt}{ 【{\akai 二黄摇板}】但愿你子子孙孙入庙廊。 }

张元秀、贺氏

\setlength{\hangindent}{60pt}{ 【{\akai 二黄摇板}】哭一声娇儿将父、母\textless{}\!{\bfseries\akai 哭头}\!\textgreater{}认,我的儿呀, }

{贺氏\hspace{30pt}~

\setlength{\hangindent}{60pt}{ 【{\akai 二黄摇板}】把我二老当一对老仆人收在身旁。} }

{(张继保 看他二老哭得如此可怜,赏他二百铜钱。)}

{(门子\hspace{30pt}~

您起来,您起来吧。)}

{张元秀\hspace{20pt}~

做什么?}

{(门子\hspace{30pt}~

状元老爷赏下来了。)}

{张元秀\hspace{20pt}~

哦,有了赏了?赏些什么啊?}({\akai 或}: 怎么,有了赏了,赏的什么?)

{(门子\hspace{30pt}~

您拿眼瞧吧------二百铜钱。)}

{张元秀\hspace{20pt}~

哦?好好好!拿来------}

{张元秀\hspace{20pt}~

妈妈。起来,起来------}

{贺氏\hspace{30pt}~

呃,我是不(肯)起来的了。儿子不认我}$\cdots{}\cdots{}$

{张元秀\hspace{20pt}~

起来起来,儿子老爷有了赏了。}

{贺氏

哦,有了赏了,赏的什么啊?}({\akai 或}: 哦,怎么,有了赏了,赏些什么?)

{张元秀\hspace{20pt}~

赏你我二百铜钱!}

{贺氏\hspace{30pt}~

啊?二百铜钱?!唉呀老老啊!看将起来他不是你我的儿子啊,}

{张元秀\hspace{20pt}~

你我的儿子呢?}

{贺氏\hspace{30pt}~

在亭外呢。}

{张元秀\hspace{20pt}~

但凭于你({\akai 或}: 任凭于你),我不管了。}

{贺氏}

\textless{}\!{\bfseries\akai 叫头}\!\textgreater{}{张继保啊,小奴才!}

{贺氏

曾记得({\akai 或}: 想当年)儿不满三日,在周梁桥下啼哭,多亏我二老将你救下,抱回家去,抚养儿一十三载,才得长大成人。不想儿天良丧尽,逃奔在外,如今做官回来,就该好好将我二老认下的才是;你不认还则罢了,怎么你还反赏与我二老这二百铜钱?!这二百铜钱为娘的我不要啊,赏与你这个奴才买烧纸啊!}

{(贺氏蹉步,双盖头,碰死介)}

{张元秀

妈妈,不认就罢,我们回去。({\akai 或}: 妈妈,,妈妈,不认就拉倒,我们回去。)}

{张元秀\hspace{20pt}~

妈妈,我们回去!}

{张元秀\hspace{20pt}~

啊?!}

{(张元秀甩帽,扔棍)}

{张元秀

({\akai 念})世人不可手无钱,有钱无子亦枉然。我今无子又无钱,抚养义子接香烟。身荣不把义父认呐,逼死恩母在亭前。({\akai 或}: 身荣不把恩父认呐,逼死义母在亭前。)抱男抱女世间有,愚者愚来贤者贤。奉劝世人休继子,报恩只得这二百钱!}

{张元秀}

\textless{}\!{\bfseries\akai 叫头}\!\textgreater{}{张继保啊,小奴才!}

{张元秀

想当年({\akai 或}: 曾记得)儿不满三日,在周梁桥下啼哭,堪堪就冻饿而死啊,(那时)是为父(的)打从此那里经过,将儿抱回家来,抚养儿长大成人。一十三载({\akai 或}: 一十三岁)送在学中攻书,指望我二老终生有靠啊。不想({\akai 或}: 谁想)你这个奴才人大心大,不听教训,逃奔在外。如今做官回来,就该好好将我二老认下才是啊({\akai 或}: 就该将我二老好好认下才是呀)!常言说得好: ``生身父母在一边,养身父母大如天''。你不认还则罢了,怎么反赏与为父的({\akai 或}: 怎么还赏与我夫妻)这二百铜钱?!这二百铜钱,一十三载,是够儿吃,够儿穿,还是够儿在学中的纸笔墨砚钱呐?}

{张元秀}

\textless{}\!{\bfseries\akai 叫头}\!\textgreater{}{张继保啊,小奴才!}

{张元秀\hspace{20pt}~

这二百铜钱为父的不要啊,赏与你这个奴才打棺材板罢!}

{(张元秀磕头,指介,碰死介)}

(门子

$\cdots{}\cdots{}$双双碰死,$\cdots{}\cdots{}$赏下棺木,盛殓起来$\cdots{}\cdots{}$)

{(张继保 哪有什么棺木,芦席两卷,搭至荒郊!)}

{(门子\hspace{30pt}~

好良心呐!)}

{(张继保 前面什么所在?)}

{(门子\hspace{30pt}~

速报寺。)}

{(张继保 打道速报寺。)}

{{[}第四场{]}}

{(风、雨、雷、电上,雷祖上)}

{(雷祖}

\textless{}\!{\bfseries\akai 点绛唇}\!\textgreater{}{祥光灿烂,紫雾盘旋;电光闪,风云翩翩,除暴诛谗奸。)}

{(风、雨、雷、电上,雷祖上高台)}

{(雷祖

({\akai 念})布雨兴云祝太平,支配万物育祥云。九天雷部承天数,诛恶安良达圣明。)}

{(雷祖\hspace{30pt}~

吾乃九天因缘雷神普化天尊是也。)}

{(雷神\hspace{30pt}~

今有凡间张继保,逼死恩父、义母,理当诛之。众神将: )}

{(众\hspace{40pt}~

有!)}

{(雷神\hspace{30pt}~

将逆子五雷殛顶!)}

{(众\hspace{40pt}~

啊。)}

{(张继保穿青褶子,甩发,扑跌,跪,左手上挂二百铜钱)}

{(众\hspace{40pt}~

凡身已灭。)}

{(雷神\hspace{30pt}~

收回威严者!)}

{(}\textless{}\!{\bfseries\akai 尾声}\!\textgreater{}{前段},{周小哥拿雨伞上)}

{(周小哥 嘿哟这雨可真大呀!哎呀------)}

{(周小哥 嘿,这位怎么跑这儿跪这死了?)}

{(周小哥

哎哟------这不是状元老爷吗?哎哟,是让雷劈了------嘿,也该啊!)}

{(周小哥 诶,这二百铜钱还在这儿呢,得了,这归我得了!)}

{(起雷声)}

{(周小哥 不义之财,我也不要!)}

{(}\textless{}\!{\bfseries\akai 尾声合头}\!\textgreater{},{周小哥带张继保下)}

{陈超老师按: }

{《天雷报》谭鑫培的老头脚步很讲究,分前后脚,走的时候重视棍儿的角度、还要转腰、松腰,重心始终在后脚。不是岔着腿,重心平均在两脚上。}

\newpage

\hypertarget{ux4e4cux76c6ux8bb0-ux4e4b-ux5218ux4e16ux660c}{%

\subsection{乌盆记 之

刘世昌}\label{ux4e4cux76c6ux8bb0-ux4e4b-ux5218ux4e16ux660c}}

{{[}第一场{]}}

{刘升,带路------}

\setlength{\hangindent}{60pt} {【{\akai 西皮摇板}】一路美景观不尽,人宿旅店鸟宿林。}

{卑人刘世昌,南阳人氏,贩卖绸缎为生。只因我离家日久,尤恐双亲在家悬念,为此算清账目,带领家人刘升,回转家园,以奉}甘旨{。刘升,前面什么所在?}

{何县所管?}

{你看天色不好哇,我们急急趱行。}

\setlength{\hangindent}{60pt} {【{\akai 西皮原板}】叹人生世间名利牵,抛父母撇妻子}\footnote{ 刘曾复先生录音作``抛父母别妻子'';吴焕老师整理的剧本(经刘曾复先生审订)记作``别父母抛妻子'';此处从吴小如先生的建议修改。{488}}{离故园。道旁美景懒得看,披星戴月奔家园。行程之间把天变,}

\setlength{\hangindent}{60pt} {【{\akai 西皮散板}】狂风大雨遮满天。}

\setlength{\hangindent}{60pt} {【{\akai 西皮散板}】刘升带路往前趱。}

{[}第二场{]}

\setlength{\hangindent}{60pt} {【{\akai 西皮散板}】大雨顷至风雷紧}\footnote{ 李元皓君建议作``风雷劲''。{489}}{,浑身上下水淋淋。}

{看前面有一人家,上前借宿。}

{好话多讲。}

{诶------下站!({\akai 或}: 放肆,下站!)}\footnote{ 此处及以下括号中的词句据吴焕老师整理的剧本添加。{490}}

{这位大哥,卑人这厢有礼。({\akai 或}: 兄台请了。)}

{我们是远方来的,行至此处,天降大雨,堪堪黄昏。前不着村,后不着店。万般无奈,借宿一宵。望求大哥({\akai 或}: 望求兄台),多行方便,明日早行,自当重谢。}

{哦,有劳了。多谢大哥。}

{哦,是是是。}

{哦,有劳了,有劳了。}

{哦,是是是。}

{哦,有座。}

{在下姓刘名世昌,南阳人氏,贩卖绸缎为生。}

{呃,小买卖。}

{小本钱呐。}

{啊,呵呵哈哈哈$\cdots{}\cdots{}$(笑介)}

{请问大哥上姓({\akai 或}: 请问兄台上姓)。}

{哦,原来是赵大哥。}

{做何生意?}

{哦,乃是大生意呀。}

{大本钱。啊,呵呵哈哈哈$\cdots{}\cdots{}$(笑介)}

{呃,前途用过,不必费心呐。}

{打搅不当了!({\akai 或}: 如此打搅,不当了!)}

\setlength{\hangindent}{60pt} {【{\akai 西皮原板}】好一位赵大哥人慷慨,顷刻间酒饭有安排。行至在中途呃大雨盖,萍水相逢理不该。到明天自当多谢拜,昏昏沉沉倒卧土台。}

\setlength{\hangindent}{60pt} {【{\akai 西皮导板}】霎时一阵呐肝肠断,}

{(唉呀,唉呀!)}

\setlength{\hangindent}{60pt} {【{\akai 西皮散板}】刀绞柔肠为哪般?}

\setlength{\hangindent}{60pt} {【{\akai 西皮散板}】回头忙把刘升唤呐,}

{刘升!刘升!}

{唉呀!}

\setlength{\hangindent}{60pt} {【{\akai 西皮散板}】奴才早已丧黄泉。}

\setlength{\hangindent}{60pt} {【{\akai 西皮散板}】是是是来明白了,中了赵大巧机关。眼望着南阳高声\textless{}哭头\textgreater{}喊,爹娘啊,}

\setlength{\hangindent}{60pt} {【{\akai 西皮散板}】阴曹地府啊走一番。}

{[}第三场{]}

{参见判爷。}

{本当前去,奈无见证之人。}

{多谢判爷。}

{[}第四场{]}

{张别古!}

{老丈------}

\setlength{\hangindent}{60pt} {【{\akai 二黄原板}】老丈不必胆怕惊,我有言来你试听: 休把我当作了妖魔论,我本屈死一鬼魂。我忙将树枝摆摇动,}

\setlength{\hangindent}{60pt} {【{\akai 二黄原板}】抓一呀把沙土扬灰尘。}

\setlength{\hangindent}{60pt} {【{\akai 二黄原板}】我和你远无冤,近无有仇恨,望求老丈把冤申。}

{[}第五场{]}

{有。}

{张别古。}

{老丈啊,呃$\cdots{}\cdots{}$(哭介)}

\setlength{\hangindent}{60pt} {【{\akai 反二黄慢板}】未曾开言泪满腮,尊呃一声老丈细听开怀: 家住在南阳城关外,离城十里太平街。}

\setlength{\hangindent}{60pt} {【{\akai 反二黄慢板}】刘世昌祖居有数代,务农为本颇有家财。奉(母)命上京做买卖,贩卖绸缎倒生财。前三年也曾把货卖,算清账目转回家来。行至在定远县地界,忽然间老天爷降下雨来。路过赵大的窑门以外,借宿一宵惹祸灾。赵大夫妻将我谋害,把我的尸骨何曾葬埋。烧作了乌盆窑中卖,幸遇老丈讨债来。可怜我冤仇有三载,有三载,老丈呐!}

\setlength{\hangindent}{60pt} {【{\akai 反二黄原板}】因此上随老丈转回家来。}

\setlength{\hangindent}{60pt} {【{\akai 反二黄原板}】劈头盖脸洒下来,奇臭难闻口难开。可怜我哇命丧他乡以外,可怜我魂在望乡台。父母盼儿,儿不能奉拜}\footnote{ 吴焕老师整理的剧本记作``奉待''。{491}}{;妻子盼夫,夫不能回来。望求老丈将我带,你带我去见包县台。倘若是把我的冤仇来解,但愿你福寿康宁永无灾。}

{正是。}

{(你告我诉。)}

{老丈多行方便。}

{方便方便罢。}

{我拿你头疼。}

{你告我诉就是。}

{多谢老丈。}

{有!}

{有!}

{[}第六场{]}

{啊老丈,这就是太爷的衙门。}

{是。}

{有。}

{正待进入,门神老爷阻拦。求太爷赏下一陌纸钱焚化,也好入内。}

{有劳了。}

{有。}

{本当进去,因念当初遇害之日({\akai 或}: 是我被害之时),被赵大夫妻将衣帽剥去,赤身露体。太爷日后有三公之位,尤恐冲撞,望求太爷赏下青衣一件,遮盖乌盆,方好进入。}

{老丈多行方便。}

{方便方便罢。}

{我拿你头疼。}

{有。}

{有。}

{有。}

{有。}

{有哇------}

{太爷容禀: }

\setlength{\hangindent}{60pt} {【{\akai 西皮快板}】未曾开言泪汪汪,尊一声太爷听端详: 家住南阳太平庄,姓刘名安字世昌。贩卖绸缎把京上,算清账目啊转还乡。赵大夫妻图财害命、主仆双双命丧({\akai 或}: }赵大夫妻图财害命把身丧{),望求太爷与我做主张啊。}

钟馗爷爷!
