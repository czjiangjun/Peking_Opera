\newpage
\renewcommand{\thefootnote}{\arabic{footnote}}
\setcounter{footnote}{0}
\phantomsection %实现目录的正确跳转
%\hypertarget{ux6e2dux6c34ux6cb3}{%
\section*{\hei\large 渭水河}%\label{ux6e2dux6c34ux6cb3}}
\addcontentsline{toc}{section}{\hei 渭水河}

\hangafter=1                   %2. 设置从第1⾏之后开始悬挂缩进  %
\setlength{\parindent}{0pt}{
{\centerline{\textrm{{[}\hei 第一场{]}}}}
\vspace{5pt}
姬昌\hspace{30pt}{[}{\akai 引子}{]}为建帝基,一路平安,到西岐。

姬昌\hspace{30pt}({\akai 念})纣王无道宠妲己,苦害忠良受凌逼。孤王回转西岐地,重整山河统华夷。

\setlength{\hangindent}{52pt}   %3. 设置悬挂缩进量                %
{姬昌\hspace{30pt}孤,西伯侯姬昌。只因纣王无道,信宠妲己,苦害忠良,是孤回转西岐,自立基业。那日有一樵夫,名叫武吉,将孤门军\footnote{刘曾复先生说戏录音作``军门'',似非,此处从《京剧汇编》~第十三集陈少武、苏连汉~口述本。}%\protect\hyperlink{fn3}{\textsuperscript{3}}
打死,拿他问罪。他言道:~家有八旬老母,无人侍奉。孤王念他是一孝子,赐他斗米贯钱,限定七日前来抵罪,去了数十余天不见到来。我不免在八卦之中,查看吉凶。}

姬昌\hspace{30pt}内侍,

内侍\hspace{30pt}有。

姬昌\hspace{30pt}香案伺候。

内侍\hspace{30pt}香案伺候哇。

姬昌\hspace{30pt}哎呀!

\setlength{\hangindent}{52pt}   %3. 设置悬挂缩进量                %
{姬昌\hspace{30pt}【{\akai 二黄原板}】摇动了金钱告上苍,八卦之中显示明详\footnote{《京剧汇编》第十三集~陈少武、苏连汉~口述本~作``明祥''。}%\protect\hyperlink{fn4}{\textsuperscript{4}}
。单见单来仄见仄,查不出小武吉落于何方。}

姬昌\hspace{30pt}({\akai 念})春有寅萌芽出土,夏有寅火炼金身。秋有寅黄叶落地,冬有寅滴水成冰。

\setlength{\hangindent}{52pt}   %3. 设置悬挂缩进量                %
{姬昌\hspace{30pt}呜哙呀,我道此人还在,原来入土而亡。他今一死不大紧要,可叹他八旬老母,无人侍奉。唉,可叹呐可叹。}

姬昌\hspace{30pt}回避了。

\setlength{\hangindent}{52pt}   %3. 设置悬挂缩进量                %
{姬昌\hspace{30pt}【{\akai 二黄原板}】孤王建业在西方,只为江山昼夜忙。东路反了姜文焕,南路鄂广反陈塘。他两家俱有那书信来往,叫孤王领人马去反商王。臣反君来小犯上,倒不如稳坐西岐乐安康。移步儿\footnote{``移步儿''吴小如先生建议作``一步儿'',此处从《京剧汇编》第十三集~陈少武、苏连汉~口述本。}%\protect\hyperlink{fn5}{\textsuperscript{5}}
来至在灵台上,且做南柯梦一场。}

姬昌\hspace{30pt}【{\akai 二黄导板}】孤王正在睡朦胧,

	姬昌\hspace{30pt}【{\akai 二黄摇板}】只见飞熊扑帐中。手执宝剑将你斩,化阵清风无影踪。

内侍\hspace{30pt}千岁醒来。

姬昌\hspace{30pt}~【{\akai 二黄导板}】适才朦胧见一怪,

	姬昌\hspace{30pt}【{\akai 二黄摇板}】醒来依然在灵台。

姬昌\hspace{30pt}内侍。

内侍\hspace{30pt}有。

姬昌\hspace{30pt}宣散宜生上殿。

内侍\hspace{30pt}领旨。散宜生灵台见驾。

散宜生\hspace{20pt}领旨。

散宜生\hspace{20pt}({\akai 念})袖里乾坤大,怀揣日月明。

散宜生 \hspace{20pt}~散宜生见驾,主公千岁。

姬昌\hspace{30pt}平身。

散宜生\hspace{20pt}千千岁。

姬昌\hspace{30pt}赐坐。

散宜生\hspace{20pt}谢坐。宣臣来见,有何圣谕?

姬昌\hspace{30pt}孤王三更时分,梦一飞熊入帐,抓伤孤的左膀,不知主何吉凶?

散宜生\hspace{20pt}这$\cdots{}\cdots{}$此乃大吉之兆。

姬昌\hspace{30pt}怎见得?

散宜生\hspace{20pt}主公传旨,郊外射猎,不得虎臣,必得良将。

姬昌\hspace{30pt}先生替孤传旨。

散宜生\hspace{20pt}领旨。

姬昌\hspace{30pt}正是:~({\akai 念})夜梦飞熊入帐来,

	散宜生\hspace{20pt}({\akai 念})郊外射猎访贤才。

\vspace{3pt}{\centerline{\textrm{{[}{\hei 第二场}{]}}}}\vspace{5pt}

武吉\hspace{30pt}嗯哼!

武吉\hspace{30pt}({\akai 念})胆小天下去得,刚强寸步难行。

\setlength{\hangindent}{52pt}   %3. 设置悬挂缩进量                %
{武吉\hspace{30pt}小子武吉。自从那日上山砍樵,进城去卖,偶遇姬千岁门军,被我失手打死。那姬千岁拿我问罪。我曾言道:~家有八旬老母,无人侍奉。那姬千岁念我是一孝子,赏我斗米贯钱,回家见母一面,限定七日前来抵罪。不想行至渭水河边,见一老者,呃,在那里垂钓,他见我面带煞气,必有凶事。是我将打死门军之事,对他实言。他教我一个法术:~回到家中,老母床前,挖一土井,宽要七尺,深要丈二,口含灯芯、糯米,睡在井内。躲过七七四十九日,方保无事。今当四十八天,老母腹中饥饿,只得将我唤醒,命我上山砍樵,卖了钱文,买米度日。}

武吉\hspace{30pt}正是:~({\akai 念})上山擒虎易,开口告人难。

\vspace{3pt}{\centerline{\textrm{{[}{\hei 第三场}{]}}}}\vspace{5pt}

南宫适\hspace{20pt} \textless{}\!{\bfseries\akai 点绛唇}\!\textgreater{}扶保西岐,

北宫高\hspace{20pt} \textless{}\!{\bfseries\akai 点绛唇}\!\textgreater{}同心协力,

辛甲\hspace{30pt} \textless{}\!{\bfseries\akai 点绛唇}\!\textgreater{}立帝基,

辛免\hspace{30pt} \textless{}\!{\bfseries\akai 点绛唇}\!\textgreater{}四海归一,

众\hspace{41pt} \textless{}\!{\bfseries\akai 点绛唇}\!\textgreater{}方显英雄气。

南宫适\hspace{20pt}俺,南宫适,

北宫高\hspace{20pt}北宫高,

辛甲\hspace{30pt}辛甲,

辛免\hspace{30pt}辛免。

南宫适\hspace{20pt}众位将军请了。

众\hspace{41pt}请了。

南宫适\hspace{20pt}主公郊外射猎,两厢伺候。

众\hspace{41pt}请。

姬昌\hspace{30pt}({\akai 念})旌旗遮日月,郊外访贤臣。

众\hspace{41pt}参见主公。

姬昌\hspace{30pt}人马可齐?

众\hspace{41pt}俱已齐备。

姬昌\hspace{30pt}郊外去者!

众\hspace{41pt}得令。

众\hspace{41pt}众将官,郊外去者。带马!

杂\hspace{41pt}啊。

众\hspace{41pt}前道为何不行?

杂\hspace{41pt}樵夫挡道。

众\hspace{41pt}人马列开!

杂\hspace{41pt}樵夫当面。

武吉\hspace{30pt}樵夫叩头。

姬昌\hspace{30pt}下跪可是武吉?

武吉\hspace{30pt}正是。

姬昌\hspace{30pt}见了孤王为何不抬起头来?

武吉\hspace{30pt}有罪不敢抬头。

姬昌\hspace{30pt}恕你无罪。

武吉\hspace{30pt}谢千岁。

姬昌\hspace{30pt}呃嗯------限定七日前来抵罪。为何今日才来见孤,该当何罪?

\setlength{\hangindent}{52pt}   %3. 设置悬挂缩进量                %
{武吉\hspace{30pt}千岁容禀:~那日多蒙千岁天恩,放小子回家,见母一面,不想行至渭水河边,见一老者,在那厢垂钓,他见我面带煞气,必有凶事。我将打死门军之事对他言明。他教我小小法术:~回到家去,老母床前,挖一土井,宽要七尺,深要丈二,口含灯芯、糯米,睡在其内。躲过七七四十九日,方保无事。今乃四十八日,我母不解其意,将我唤醒,命我上山砍柴,进城去卖,不想撞了\footnote{《京剧汇编》第十三集~陈少武、苏连汉~口述本~作``闯了''。}%\protect\hyperlink{fn6}{\textsuperscript{6}}
千岁御驾。唉,也是小人命该如此,情愿抵罪。}

姬昌\hspace{30pt}可曾问过那渔人的名姓?

武吉\hspace{30pt}这,不曾问得。

姬昌\hspace{30pt}垂钓所在?

武吉\hspace{30pt}渭水河边。

姬昌\hspace{30pt}将柴担放下,引孤前往。

{\akai 二}太子\hspace{20pt}且慢。启奏父王:~为国访贤,必须换了便服。

姬昌\hspace{30pt}看衣改换。

众\hspace{41pt}撒下围场。

姬昌\hspace{30pt}武吉。

武吉\hspace{30pt}有。

姬昌\hspace{30pt}带路。

武吉\hspace{30pt}领旨。

\setlength{\hangindent}{52pt}   %3. 设置悬挂缩进量                %
{姬昌\hspace{30pt}【{\akai 二黄原板}】樵夫道那渔人颇有奥妙,因此上带皇儿亲走一遭。叫武吉与孤向前引道,青的山绿是水难画难描。}

\vspace{3pt}{\centerline{\textrm{{[}{\hei 第四场}{]}}}}\vspace{5pt}

\setlength{\hangindent}{52pt}   %3. 设置悬挂缩进量                %
{姜尚\hspace{30pt}【{\akai 二黄原板}】纣王无道贪色酒,午门外修一座摘星楼。比干丞相遭毒手,贾氏夫人坠高楼。(我不愿在朝为官侯,隐居磻溪垂钓钩。\footnote{据段公平{\scriptsize 君}告知,原有此二句,可不唱。}%\protect\hyperlink{fn7}{\textsuperscript{7}}
) 昨夜晚在土台观看星斗,算就了西伯侯灭纣兴周。移步儿来至在渭水河口,那一边来的是西伯王侯。}

姬昌\hspace{30pt}({\akai 内})武吉带路。

\setlength{\hangindent}{52pt}   %3. 设置悬挂缩进量                %
{姬昌\hspace{30pt}【{\akai 二黄原板}】太平鸟不住当头叫,叫得孤王喜眉梢。土台上一道长稳坐垂钓,手执鱼竿自在逍遥。我观他倒有那仙家奥妙,想必是他腹中藏有略韬。叫武吉你与我暂且退了,二皇儿向前去细问根苗。}

{\akai 二}太子\hspace{20pt}啊,那渔人请了。

姜尚\hspace{30pt}({\akai 念})钓、钓、钓,大鱼不到小鱼到。用你不着,去吧。

{\akai 二}太子\hspace{20pt}好生与你见礼,为何佯装不睬?

姜尚\hspace{30pt}({\akai 念})从来未识金龙面,要见明君便开言。

{\akai 二}太子\hspace{20pt}好大的口气。

{\akai 二}太子\hspace{20pt}啊,父王,那渔翁言道:~从来未识金龙面,要见明君便开言。

姬昌\hspace{30pt}呃嗯------敢是尔等轻慢于他?

{\akai 二}太子\hspace{20pt}儿臣不敢。

姬昌\hspace{30pt}待我向前。

姬昌\hspace{30pt}啊,渔人请了。

姜尚\hspace{30pt}贫道稽首。

姬昌\hspace{30pt}口称``稽首'',在哪座名山修炼?

姜尚\hspace{30pt}昆仑学道。稽首为尊。

姬昌\hspace{30pt}在此作甚?

姜尚\hspace{30pt}在此垂钓。

姬昌\hspace{30pt}钓得何鱼?

姜尚\hspace{30pt}西海鳌鱼。

姬昌\hspace{30pt}小小竹竿,焉能钓得鳌鱼?

姜尚\hspace{30pt}竹竿虽小,能坠千斤。

姬昌\hspace{30pt}借来一观。

姜尚\hspace{30pt}请看。

姬昌\hspace{30pt}呜哙呀!为何用直钩垂钓?

姜尚\hspace{30pt}贫道心直性直,故而用直钩垂钓,有道是``愿者上钩''。

姬昌\hspace{30pt}但不知谁愿谁不愿?

姜尚\hspace{30pt}({\akai 念})耐烦等到群鱼到,自有鱼儿来吞钩。

姬昌\hspace{30pt}请问渔人尊姓大名?

姜尚\hspace{30pt}贫道姓姜名尚字子牙。

姬昌\hspace{30pt}道号?

姜尚\hspace{30pt}飞熊。

姬昌\hspace{30pt}``飞熊''!应孤梦兆也。

姜尚\hspace{30pt}来者上姓?

姬昌\hspace{30pt}在下西伯侯姬昌。

姜尚\hspace{30pt}哦,原来是姬千岁,失敬了。

姬昌\hspace{30pt}岂敢。

姜尚\hspace{30pt}不在西岐,来在渭水则甚?

姬昌\hspace{30pt}特请先生,扶保姬氏河山。

姜尚\hspace{30pt}贫道才疏学浅,不敢当此重任。

姬昌\hspace{30pt}先生不必推辞。

姬昌\hspace{30pt}武吉。

武吉\hspace{30pt}在。

姬昌\hspace{30pt}封你开路先锋,准备车辇伺候。

武吉\hspace{30pt}遵命。

武吉\hspace{30pt}({\akai 念})不是姜太公,怎能作先锋。

姬昌\hspace{30pt}请先生转至大营。

姜尚\hspace{30pt}请。

姬昌\hspace{30pt}【{\akai 二黄原板}】久闻先生今相见,

姜尚\hspace{30pt}【{\akai 二黄原板}】姜子牙八十二才遇名贤。

姬昌\hspace{30pt}【{\akai 二黄原板}】孤王江山全仗你,

姜尚\hspace{30pt}【{\akai 二黄原板}】要保江山万万年。

\vspace{3pt}{\centerline{\textrm{{[}{\hei 第五场}{]}}}}\vspace{5pt}

\textrm{武吉\hspace{30pt}启千岁:~车辇齐备。}

\textrm{姬昌\hspace{30pt}请先生登辇。}

\textrm{姜尚\hspace{30pt}贫道有僭了。}

\textrm{姜尚\hspace{30pt}【{\akai 西皮导板}】听号炮三声响旌旗招展,}

\setlength{\hangindent}{52pt}   %3. 设置悬挂缩进量                %
\textrm{姜尚\hspace{30pt}【{\akai 西皮原板}】满营中众将官齐跨雕鞍。对主公施一礼忙登车辇,姜子牙坐车辇细把君观:~前三皇后五帝年深日远,有尧、舜并禹、汤四大名贤。西伯侯夜得兆飞熊扑面,散宜生奏一本【{\footnotesize 转}{\akai 西皮快板}】郊外访贤。打柴汉名武吉引君来见,姜子牙八十二才遇名贤。行八百单八步住了车辇,}

\textrm{姜尚\hspace{30pt}【{\akai 西皮摇板}】到后来保周室八百八年。}

\textrm{武吉\hspace{30pt}启千岁:~西岐众百姓,头顶香盘,迎接千岁、先生进城。}

\textrm{姬昌\hspace{30pt}香盘撤去,摆队进城。}

\textrm{武吉\hspace{30pt}香盘撤去,摆队进城。}
}

