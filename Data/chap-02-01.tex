\newpage
\phantomsection %实现目录的正确跳转
\section*{\large\hei {捉放曹~{\small 之}~陈宫}}
\addcontentsline{toc}{section}{\hei 捉放曹~{\small 之}~陈宫}

\hangafter=1                   %2. 设置从第1⾏之后开始悬挂缩进  %
\setlength{\parindent}{0pt}{
{\centerline{{[}{\hei 第一场}{]}}}\vspace{5pt}

{[}{\akai {\akai 引}子}{]}官居县令,与黎民,判断冤情。

({\akai 念})头戴乌纱奉行先\footnote{``奉行''是``遵照执行''之意;``头戴乌纱奉行先''陈宫身为县令,为一县表率;},四乡开可\footnote{``开可''是``许可''的意思;}万民欢。家邑有语呼循吏\footnote{``家邑''本意为``采地'',这里表示陈宫管辖之中牟县;  ``循吏''见于《史记》的《循吏列传》,一般指实施、推行善政、口碑声好的州、县级地方官;},德配汪洋水地天\footnote{``水地天''乃``尧天、舜地、禹水''之意。  吴小如先生学的定场诗作``{头戴乌纱奉孝先},{慈祥恺悌万民欢}。{嘉言犹如湖中地},{得配汪洋水底天}。''后两句源自《后汉书·黄宪传》:~郭林宗评黄宪(叔度)``汪汪若千顷之陂,澄之不清,淆之不浊,不可量也。''}。

下官姓陈名宫,字公台。蒙圣恩,身授中牟县令({\akai 或}:~职授中牟县正堂)。(自到任以来,地方宁靖。)前者董太师有钧谕到来,命各府州县,画影图形,捉拿刺客曹操。也曾命班头王申\footnote{段公平{\scriptsize 君}建议作``王升'',此处从``中国京剧戏考''网站《戏考》第一册本。}(、李顺)等四门严拿({\akai 或}:~严查),未见交差回报({\akai 或}:~未见交签)。今当三、六、九日放告之期({\akai 或}:~今当三、六、九日升堂理事)。

左右,伺候了。

罢了。

(捉拿刺客之事如何?)

喜从何来?

有何为证?

呈上来。

左右,看赏。({\akai 或}:~来,看赏。)

愿者何来?

呵呵哈哈哈$\cdots{}\cdots{}$({\hwfs 笑}{\hwfs 介})

官升吏赏,理所当然。国家法度自无不行。

吩咐下去,将刺客带上堂来。

\setlength{\hangindent}{56pt}{【{\akai 西皮摇板}】曹孟德进衙来齐声威吓,胥吏们列两旁虎立山坡。观此人面貌上带定凶恶,见本县不下跪却是为何。 }

下站可是曹操?

见了本县大胆不跪({\akai 或}:~为何不跪)。

你可知({\akai 或}:~岂不知)王子犯法,与民同罪。

刺杀太师,还说无罪。

虽非亲眼得见,现有董太师钧谕到来,(你)还敢强辩?

帘外之官({\akai 或}:~我在帘外为官),不问朝阁之事。

住口。

\setlength{\hangindent}{56pt}{【{\akai 西皮二六}】曹孟德休得要谤毁朝阁,董太师他自有治国韬略。扶灵帝虽无功却也无过({\akai 或}:~扶灵帝为都尉并无过错),十常侍乱宫闱扫荡妖魔。收下了吕奉先威震海角,传一令好一似地动山挪。我将你解进京是我份责({\akai 或}:~是我所责),千金赏万户侯加官进爵({\akai 或}:~并非为千金赏加官晋爵)。你好比扑灯蛾自来投火,又好比抢食鱼自投在网罗。你好比平阳虎把路走错,擒虎易放虎难自己揣摩。 }

\setlength{\hangindent}{56pt}{【{\akai 西皮快板}】听他言不由我双眉皱锁,这件事好教我无计奈何。既擒住若放他罪归于我,若不放又恐怕惹出风波。左思量右辗转({\akai 或}:~前思量后辗转)无计定夺\footnote{夏行涛{\scriptsize 君}建议作``定妥''。}, }

有了!

\setlength{\hangindent}{56pt}{【{\akai 西皮快板}】学苏秦仿张仪计上心窝。既擒住放不放全凭于我,就是放也说个情理相合。 }

\setlength{\hangindent}{56pt}{【{\akai 西皮摇板}】曹孟德说此话如梦方觉,七品官焉能得辅相朝阁。(倒不如弃县令随他入伙,走天涯奔海角重整山河。)下位去与明公忙松绳索, }

\setlength{\hangindent}{56pt}{【{\akai 西皮摇板}】胥吏们且回避爷有发落。 }

\vspace{3pt}{\centerline{{[}{\hei 第二场}{]}}}\vspace{5pt}

\setlength{\hangindent}{56pt}{【{\akai 西皮摇板}】手挽手与明公二堂里坐,驾光临少奉迎望乞恕过。 }

适才关前、堂上多有冒犯,明公海涵({\akai 或}:~望乞恕罪。)。

明公今欲({\akai 或}:~明公意欲)何往?

我意欲随同明公,奔走({\akai 或}:~海走)天涯,(会合诸侯,)共图大事(,幸勿见却)。

不妨,老母妻室,现在原郡,料然无事。

明公请至书房(待茶),容我安排({\akai 或}:~待我吩咐他们)。

(曹操\hspace{30pt}暂时别。)

少刻奉(陪)。来,

吩咐下去,老爷下乡查旱,多则半月,少则十天({\akai 或}:~多则十日,少则七天)。将信印付与佐堂执掌({\akai 或}:~掌管),不可({\akai 或}:~休得)罗唣。

附耳上来。

记下了。

小心把守。

\vspace{3pt}{\centerline{{[}{\hei 第三场}{]}}}\vspace{5pt}

\setlength{\hangindent}{56pt}{【{\akai 西皮散板}】路上行人马蹄忙。\hspace{10pt}~ }

\setlength{\hangindent}{56pt}{【{\akai 西皮散板}】见一老丈坐道旁。\hspace{10pt}~ }

明公,(你我)还是趱路要紧呐。

明公,去得的?

这就不敢。

嗯哼。

冒造宝庄,老丈海涵。

有座。

老丈,

\setlength{\hangindent}{56pt}{【{\akai 西皮快板}】多蒙老丈美言奖,释放皇家一栋梁。七品的郎官何足讲,同奔原为汉家邦。 }

前途用过,不必费心({\akai 或}:~不用费心呐)。

(啊)明公,适才({\akai 或}:~方才)老丈提起令尊大人,明公双目落泪,真乃忠孝双全。

明公啊。

\setlength{\hangindent}{56pt}{【{\akai 西皮快板}】休流泪来免悲伤,忠孝二字天下扬。同心协力除奸党,凌烟阁上把名扬。 }

这般时候,哪道而去?

家常随便,万勿劳心。({\akai 或}:~前途用过,万勿费心呐。)

呵呵哈哈哈$\cdots{}\cdots{}$({\hwfs 笑}{\hwfs 介})

\setlength{\hangindent}{56pt}{【{\akai 西皮摇板}】老丈亲自沽佳酿,待人礼仪似孟尝。 }

听见什么?

诶,老丈与令尊大人(有)八拜之交,焉有此事,你何必多疑呀?({\akai 或}:~焉有此意,你不要多疑呀。)

呃,这倒使得。

\setlength{\hangindent}{56pt}{【{\akai 西皮散板}】言语恍惚实难详。\hspace{10pt}~ }

又听见什么?

老丈临行言道,家中菜蔬俱有,只是缺少美酒({\akai 或}:~缺少好酒)。亲自去往前村沽酒回来,还要把敬你我。你不要见差了。

明白何来?

我看老丈面带厚道,断非贪赏之辈。

(依你之见?)

\textless{}\!{\bfseries\akai 叫头}\!\textgreater{}明公!

待等老丈沽酒回来,问上他三言两语({\akai 或}:~三言五语),他若无言,那时节再动手也还,不,不,不迟呀。

依你之见?

使不得。

唉!

\setlength{\hangindent}{56pt}{【{\akai 西皮散板}】未必他有此心肠。\hspace{10pt}~ }

\setlength{\hangindent}{56pt}{【{\akai 西皮散板}】求赏焉有此风光。\hspace{10pt}~ }

(使不得!)

(唉!)

\setlength{\hangindent}{56pt}{【{\akai 西皮散板}】他一家大小要遭祸殃。 }

\setlength{\hangindent}{56pt}{【{\akai 西皮散板}】吓得我三魂七魄茫啊。 }

\setlength{\hangindent}{56pt}{【{\akai 西皮散板}】陈宫上前扯衣裳。\hspace{10pt}~ }

明公(,将老丈一家杀死,)你意欲何往?

杀人放火不是你我所为。

\setlength{\hangindent}{56pt}{【{\akai 西皮散板}】杀人还要火焚房。\hspace{10pt}~ }

哎呀!

\setlength{\hangindent}{56pt}{【{\akai 西皮散板}】见一捆猪在厨房。\hspace{10pt}~ }

明公,你到底将他一家错杀了。

老丈一片好心,杀猪宰羊,款待你我,岂不是错杀了?

你去看呐。

嘿嘿。

\textless{}\!{\bfseries\akai 叫头}\!\textgreater{}明公!

你将老丈一家杀死,待等老丈沽酒回来,问起情由,你有({\akai 或}:~你是)何言答对?

事到如今,也只好是一走哇。

走哇,

走哇,

走走走!

\vspace{3pt}{\centerline{{[}{\hei 第四场}{]}}}\vspace{5pt}

啊?!

\setlength{\hangindent}{56pt}{【{\akai 西皮快板}】背转身来自参详。指望他是定国安邦将,却原来贼是个人面兽心肠。 }

啊,老丈,不必强留,回家自然明白。

你我后会有期,就此谢谢了。

\setlength{\hangindent}{56pt}{【{\akai 反西皮散板}】陈宫心中似刀扎。 }

老丈!

\setlength{\hangindent}{56pt}{【{\akai 反西皮散板}】多蒙老丈你的美意大,好意反成恶冤家。急忙里难说你我的知心话, }

老丈!

\setlength{\hangindent}{56pt}{【{\akai 反西皮散板}】休怨我陈宫你怨他。 }

\setlength{\hangindent}{56pt}{【{\akai 西皮摇板}】他人不走事有差。\hspace{10pt}~ }

有什么言语,(你)饶他一条老命吧。

不要回来。

哎呀!

\setlength{\hangindent}{56pt}{【{\akai 西皮散板}】陈宫一见咽喉哑,白发老丈染黄沙。一家大小丧剑\textless{}\!{\bfseries\akai 哭头}\!\textgreater{}下,老丈啊, }

呀呸!

\setlength{\hangindent}{56pt}{【{\akai 西皮散板}】再与孟德把话答。\hspace{10pt}~ }

\textless{}\!{\bfseries\akai 叫头}\!\textgreater{}明公!

你将老丈一家杀死,尚且追悔不及,出庄又将老丈剑劈道旁,是何理也?

似你这等({\akai 或}:~这般)行事,岂不怕天下人咒骂于你?

哦!

\setlength{\hangindent}{56pt}{【{\akai 西皮慢板}】听他言吓得我心惊胆怕,背转身自埋怨我自己做差。我先前指望他宽宏量大,却原来贼是个无义的冤家。马行在夹道内我难以回马,这才是花随水水不能恋花。这时候我只得暂且忍耐在心下,既同行共大事必须要劝解于他。 }

\setlength{\hangindent}{56pt}{【{\akai 西皮二六}】休道我言语多必有奸诈,你本是大义人把事作差。吕伯奢与你父相交不假,为什么起疑心杀他的全家。一家人被你杀也就该罢,出庄来你为何把老丈来杀,是何根芽。 }

\setlength{\hangindent}{56pt}{【{\akai 西皮摇板}】好言语劝不醒蠢牛木马,把此贼好一比井底之蛙。 }

但凭于你。

\vspace{3pt}{\centerline{{[}{\hei 第五场}{]}\protect\footnote{陈超老师介绍:~``宿店''一场,陈宫坐小边,【{\akai 二黄慢板}】也坐小边虎头椅。}}\vspace{5pt}

马不要下鞍。

明灯一盏。

鞍马劳顿,吞吃不下呀。

既已同行,何言({\akai 或}:~说什么)不服?

你的疑心呐,也忒大了。({\akai 或}:~你那疑心,也忒大了。)

明公。

睡着了。

我陈宫好悔也!

\setlength{\hangindent}{56pt}{【{\akai 二黄慢板}】一轮明月照窗下,陈宫心中乱如麻。悔不该心猿并意马,悔不该随他人到吕家。吕伯奢可算得义气大,杀猪沽酒款待于他。又谁知此贼的疑心太大,拔出剑将他的满门杀。一家人俱丧在宝剑之下,年迈老丈命染黄沙。屈死的冤鬼魂休来怨咱,自有那神灵天理鉴察。 }

\setlength{\hangindent}{56pt}{【{\akai 二黄原板}】听谯楼打罢了二更鼓下\footnote{陈超老师介绍:~起二更时陈宫有个身段:~搭左腿,左手水袖搭在左膝上,右手扶座。},越思越想把事来做差。悔不该把家属一旦撇下,悔不该弃县令抛却了乌纱。我只说贼是个宽宏量大,汉室后来贼是惹祸的根芽。 }

明公。

睡着了。

\setlength{\hangindent}{56pt}{【{\akai 二黄原板}】观此贼睡卧真潇洒,安眠好似井底之蛙。贼好比蛟龙未生鳞甲,贼好比猛虎未曾长牙。虎在笼中我不打,我岂肯放虎归山又把人抓。 }

\setlength{\hangindent}{56pt}{【{\akai 二黄散板}】执宝剑将贼的头割下, }

\setlength{\hangindent}{56pt}{【{\akai 二黄散板}】险些儿把事又做差。 }

我若将他(一剑)杀死,旁人岂不道我与董卓同党?

(这、这、这$\cdots{}\cdots{}$)

看桌案之上现有笔砚,我不免题诗一首,打动于他。

但不知以何为题?

就以四更为题。

({\akai 念})鼓打四更月正浓,心猿意马归旧踪。杀死吕家人数口,方知曹------

(啊,明公------)

(明公,睡着了。)

({\akai 念})方知曹操是奸雄!

陈宫题。

趁此天色朦胧,我不免寻找马匹逃走了罢。

唉!

\setlength{\hangindent}{56pt}{【{\akai 二黄散板}】也是我陈宫做事差,不该随贼走天涯({\akai 或}:~悔不该随贼奔天涯)。落花有意随流水,流水无情恋落花。 }

我好悔也!
}
