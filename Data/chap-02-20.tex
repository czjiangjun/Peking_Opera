\newpage
\phantomsection %实现目录的正确跳转
\section*{\large\hei {空城计~{\small 之}~诸葛亮}}
\addcontentsline{toc}{section}{\hei 空城计~{\small 之}~诸葛亮}

\hangafter=1                   %2. 设置从第1⾏之后开始悬挂缩进  %
\setlength{\parindent}{0pt}{

\vspace{3pt}{\centerline{{[}{\hei 第一场}{]}}}\vspace{5pt}

{[}{\akai {\akai 引}子}{]}羽扇纶巾,四轮车,快似风云。阴阳反掌定乾坤,保汉家两代贤臣。

列位({\akai 或}:~众位)将军少礼。

(众\hspace{40pt}啊!)

({\akai 念})忆昔当年居卧龙,万里乾坤掌握中。扫尽狼烟扶汉统,人曰男儿大英雄。

老夫,复姓诸葛名亮,字孔明,道号卧龙。先帝爷白帝城托孤遗言,扫荡中原,保留汉室。闻得司马懿兵至岐山,必然夺取街亭。必须派一能将,前去防守。啊,众位将军,

(众\hspace{40pt}啊!)

哪位将军带领人马,镇守街亭,甘当此任。

(马谡\hspace{30pt}$\cdots{}\cdots{}$镇守街亭。)

那司马(懿)虽则年迈,用兵如神,将军不可轻敌。

(马谡\hspace{30pt}$\cdots{}\cdots{}$攻无不取$\cdots{}\cdots{}$何况小小街亭。)

(呃嗯------)

(马谡{\hwfs 躬揖})

街亭虽小,干系甚重啊。

军无戏言。

(马谡\hspace{30pt}愿立军状。)

好,当帐立来。

帐外候令。

(马谡{\hwfs 下})

众位将军,

(众\hspace{40pt}丞相。)

哪位将军愿协同马谡,镇守街亭,当帐请令。

王将军素来谨慎;此番到了({\akai 或}:~去至)街亭,必须靠山近水,安营扎寨。扎寨已毕,画一四至八道地理图,速报我知。

(王平\hspace{30pt}得令。)

赵老将军听令:~带领三千人马,镇守列柳城。

马岱听令:~押解粮草,军中({\akai 或}:~军前)需用。

马谡进帐。

(马谡{\hwfs 上})

一旁坐下。

(马谡\hspace{30pt}$\cdots{}\cdots{}$有何密令?)

今逢大敌,非比寻常。我有一言,将军听了:~

\setlength{\hangindent}{56pt}{【{\akai 西皮原板}】两国交锋龙虎斗,各为其主统貔貅。管带三军要宽厚,赏罚中公平莫要自由。此一番领兵去镇守,靠山近水把营守({\akai 或}:~把营收;把陉\footnote{``陉''是山脉中断的地方,这样的地方往往是重要的关隘。这里特指街亭。}守)。 }

\setlength{\hangindent}{52pt}{(马谡\hspace{30pt}【{\akai 西皮摇板}】$\cdots{}\cdots{}$辞别丞相出帐口,$\cdots{}\cdots{}$顺水去推舟。) }

\setlength{\hangindent}{56pt}{【{\akai 西皮摇板}】先帝爷白帝城叮咛就,汉诸葛扶幼主岂能无忧。但愿得此一去扫平贼寇,免得我亲自去把贼收。 }

\vspace{3pt}{\centerline{{[}{\hei 第二场}{]}}}\vspace{5pt}

({\akai 念})兵扎祁山地,要擒司马懿。

(旗牌{\hwfs 上})

(旗牌\hspace{30pt}门上哪位在?)

传。

罢了。

奉何人所差?

(旗牌\hspace{30pt}王平王将军所差。)

手捧何物?

(旗牌\hspace{30pt}地理图。)

展开。

命你去到列柳城,速速将赵老将军调回营来!快去快去!

(旗牌{\hwfs 下})

啊------?!好大胆的马谡哇。临行怎样吩咐({\akai 或}:~嘱咐)与你?靠山近水,安营扎寨。怎么,你偏偏要在山顶扎营?!哎呀,大略街亭难保哇。

(探子{\hwfs 上})

(探子\hspace{30pt}报!)

(探子\hspace{30pt}$\cdots{}\cdots{}$失守街亭!)

再探!

(探子{\hwfs 下})

如何,果然把街亭失守了。唉------呀!虽然马谡失守街亭,乃诸葛(亮)之罪也。

(探子{\hwfs 上})

(探子\hspace{30pt}报!)

(探子\hspace{30pt}$\cdots{}\cdots{}$带兵夺取西城!)

再探!

(探子{\hwfs 下})

呜哙呀!司马懿居然带兵夺取西城来了。唉------

(诸葛亮{\hwfs 站起})

当初先帝爷白帝城托孤之时言过:~马谡言过其实,不可大用。悔不听先帝遗言,今日错差马谡,失守街亭,悔之晚矣呀!

(探子{\hwfs 上})

(探子\hspace{30pt}报!)

再,再探!

(探子{\hwfs 下})

啊?!,司马懿的兵,他来得好快呀!嗯------人言司马,用兵如神,今日一见,令人可敬呐,令人可服!

哎呀且住!

(诸葛亮{\hwfs 站起})

想这西城的将官,俱被老夫调遣在外,所剩下尽是些个老弱残兵。倘若司马兵到,难道说教我束手被擒,这束手------被擒$\cdots{}\cdots{}$哎呀!\textless{}\!{\bfseries\akai 乱锤}\!\textgreater{}

老军们进见。

(老军甲\hspace{20pt}~ 司马兵到,)

(老军乙\hspace{20pt}~ 心惊肉跳。)

(老军甲\hspace{20pt}~ 见了丞相,)

(老军乙\hspace{20pt}~ 急忙跪倒。)

(二老军\hspace{20pt}~ 有何吩咐?)

命尔等将四门大开,每门上二十名老军,洒扫街道。司马兵到,不可惊慌浮躁,违令者斩。

(老军甲\hspace{20pt}~ 丞相吩咐我,)

(老军乙\hspace{20pt}~ 准死不能活。)

天呐,天------

汉室兴败就在这空城一计也!

\setlength{\hangindent}{56pt}{【{\akai 西皮摇板}】我用兵数十年从来谨慎,错用了小马谡无用之人。无奈何定空城计我的心神不定,望空中求先帝大显威灵。 }

\vspace{3pt}{\centerline{{[}{\hei 第三场}{]}}}\vspace{5pt}

\setlength{\hangindent}{56pt}{【{\akai 西皮摇板}】恨马谡失街亭令人可恨,这时候倒教我难以调停。 }

呃------

\setlength{\hangindent}{56pt}{【{\akai 西皮摇板}】老军们因何故纷纷议论, }

\setlength{\hangindent}{56pt}{【{\akai 西皮摇板}】国家事用不着尔等劳心。 }

\setlength{\hangindent}{56pt}{【{\akai 西皮摇板}】这西城地原本是咽喉路径, }

\setlength{\hangindent}{56pt}{【{\akai 西皮摇板}】我城内早埋伏有十万神兵。 }

(老军甲\hspace{20pt}~ 我再到里头瞧瞧去,)

(老军乙\hspace{20pt}~ 你瞧见什么没有?)

(老军甲\hspace{20pt}~ 什么我也没瞧见,瞧见李佩卿在那拉胡琴呢!\footnote{刘曾复先生示范说戏时介绍,这是慈瑞泉临场抓的哏。})

\setlength{\hangindent}{56pt}{【{\akai 西皮摇板}】叫老军扫街道把宽心放稳({\akai 或}:~把宽心拿稳), }

\setlength{\hangindent}{56pt}{【{\akai 西皮摇板}】退司马保空城全仗此琴。 }

(司马懿{\hwfs 上})

\setlength{\hangindent}{52pt}{(司马懿\hspace{20pt}~ 【{\akai 西皮原板}】大队人马往前进,$\cdots{}\cdots{}$) }

\setlength{\hangindent}{56pt}{【{\akai 西皮慢板}】我本是卧龙岗散淡的人,评阴阳如反掌保定乾坤。先帝爷下南阳御驾三请,算就了汉家的业鼎足三分。官封到武乡侯执掌帅印,东西战南北剿博古通今。周文王访姜尚周室大振,汉诸葛怎比得前辈的先生。闲无事在敌楼我亮一亮琴音, }

(诸葛亮{\hwfs 抚琴介})

呵呵哈哈哈$\cdots{}\cdots{}$({\hwfs 笑介})

\setlength{\hangindent}{56pt}{【{\akai 西皮原板}】我眼前缺少个知音的人。 }

\setlength{\hangindent}{56pt}{【{\akai 西皮二六}】我正在城楼观山景,又听得城外乱纷纷。旌旗招展空翻影,原来是司马发来的兵。我也曾差人去打听,打听得司马领兵往西行。一来是马谡无谋少才能,二来是将帅不和({\akai 或}:~二将不和;两将不和)失街亭。连得三城多侥幸,贪而无厌又夺我西城。诸葛亮在敌楼把驾等,等候你到此谈、谈、谈谈心。西城的街道({\akai 或}:~城外的街道)打扫净,准备司马好屯兵。到此并无有别的敬,早备下羊羔美酒犒赏你的三军(临)。既到此就该把城进,为什么你犹豫不定、进退两难为的是何情。我只有琴童人两个,我是又无有埋伏又无有兵({\akai 或}:~我是又没有埋伏又没有兵)。你不要胡思乱想心不定,来来来,请上城来听我抚琴。 }

(探子\hspace{30pt}$\cdots{}\cdots{}$兵退四十里呐!)

(险呐~!)

\setlength{\hangindent}{56pt}{【{\akai 西皮散板}】人言司马善用兵,到此不敢进空城呐。诸葛从来永不弄险,险中又险显才能。 }

哎呀老将军呐!方才司马懿兵临城下,被我({\akai 或}:~被山人)用空城计将他哄走。必然复返,老将军速速抵挡一阵。

(赵云\hspace{30pt}得令!)

正是:~({\akai 念})虎在深山人咸远,蛟龙得水又复还。

险呐!

\vspace{3pt}{\centerline{{[}{\hei 第四场}{]}}}\vspace{5pt}

\setlength{\hangindent}{56pt}{【{\akai 西皮摇板}】算就汉家三分鼎,险些一旦化灰尘呐。 }

(探子\hspace{30pt}报!)

(探子\hspace{30pt}马谡、王平回营请罪!)

升帐。

有请。

带王平!

\setlength{\hangindent}{56pt}{【{\akai 西皮摇板}】怒上心头难消恨, }

(王平\hspace{30pt}丞相。)

\setlength{\hangindent}{56pt}{【{\akai 西皮快板}】抬头只见小王平。临行再三嘱咐你,靠山近水扎大营。大胆不听我的令,失守街亭你的罪不轻。 }

(王平\hspace{30pt}丞相!)

\setlength{\hangindent}{52pt}{(王平\hspace{30pt}【{\akai 西皮快板}】丞相不必怒气生,王平言来听分明:~马谡不听丞相令,他在山顶扎大营。丞相若是不肯信,现有画图作证凭。) }

\setlength{\hangindent}{56pt}{【{\akai 西皮快板}】若不是画图来得紧,定与马谡同罪名。将王平责打【{\footnotesize 转}{\akai 西皮摇板}】四十棍\footnote{刘曾复先生示范说戏时介绍,此句原作``叉出帐去免责问''。},}

\setlength{\hangindent}{56pt}{【{\akai 西皮摇板}】快带马谡这无用的人呐。 }

(马谡\hspace{30pt}唉呀!)

\setlength{\hangindent}{56pt}{【{\akai 西皮快板}】见马谡跪帐下,不由老夫怒气发。大胆不听我的话,失守街亭差不差。 }

\setlength{\hangindent}{56pt}{【{\akai 西皮散板}】吩咐两旁刀斧手,快斩马谡正军法。 }

\setlength{\hangindent}{56pt}{【{\akai 西皮摇板}】见马谡只哭得珠泪\textless{}\!{\bfseries\akai 哭头}\!\textgreater{}洒, }

\setlength{\hangindent}{56pt}{【{\akai 西皮摇板}】我心中好似乱刀扎。 }

\textless{}\!{\bfseries\akai 三叫头}\!\textgreater{}马谡!幼常!唉------参军呐!({\hwfs 哭介})

马谡,你临行之时,(当着满营的将官,)先立下军令状啊。如今若不将你正法,何以服众?

\textless{}\!{\bfseries\akai 三叫头}\!\textgreater{}马谡!幼常!唉------参军呐!

(众\hspace{40pt}哦$\cdots{}\cdots{}$)

来!斩!

招回来!

马谡哇,方才言道({\akai 或}:~方才言过):~家有八旬老母,无人侍奉。你死之后,将你兵马钱粮,拨与你老母,以为养老之费。

\textless{}\!{\bfseries\akai 三叫头}\!\textgreater{}马谡!幼常!唉------参军呐!({\hwfs 哭介})

来!~斩,斩,斩,斩$\cdots{}\cdots{}$

\setlength{\hangindent}{56pt}{【{\akai 西皮散板}】我哭、哭一声马参军,叫、叫、叫一声马幼常啊。未出兵先立下军令状,可叹你为国家刀下身亡。 }

\textless{}\!{\bfseries\akai 哭头}\!\textgreater{}马谡哇!参谋啊!啊!马幼常啊!

(赵云\hspace{30pt}$\cdots{}\cdots{}$为何落泪?)

唉!老将军呐!(我哪里哭的是马谡啊!)当初先帝爷(白帝城)托孤之时言过:~马谡言过其实,不可大用。悔不听先帝遗言,至有今日之过。我哪里哭的是马谡哇,乃深恨己之不明,追思先帝遗言呐,呃呃呃$\cdots{}\cdots{}$({\hwfs 哭介})

也罢,待山人拜本进京,奏明幼主,贬去武乡侯。整顿人马。再与司马决战。

后帐有宴,与老将军贺功。

}
