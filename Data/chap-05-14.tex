\newpage\hspace{30pt}~

{%

\subsubsection{\large\hei {二进宫~{\small 之}~

杨波}}

{千岁请!}\hspace{30pt}~

\setlength{\hangindent}{56pt}{【}二黄摇板{】宫门上锁贼李良。}

\setlength{\hangindent}{56pt}{【}二黄摇板{】四郎我儿击宫呃墙。}

{我儿休要动手,此乃徐家小姐,上前见过。}

{领旨。}\hspace{40pt}~

{千岁。}\hspace{40pt}~

{全仗千岁。}\hspace{30pt}~

{千岁------}\hspace{20pt}~

\setlength{\hangindent}{56pt}{【}二黄慢板{】千岁爷进寒宫休要慌忙,站宫门听学生细说比方:~昔日里楚汉两争强,鸿门设宴要害汉王。张子房背宝剑把韩信来访,九里山前摆下战场。只逼得楚项羽乌江命丧,到后来封韩信三齐王。他朝中有一个萧何丞相,后宫院有一位吕后皇娘。君臣们设下了天罗地网,三宣韩信斩首在未央。九月十三严霜降,盖世忠良不得久长。千岁爷进寒宫学生不往,}

\setlength{\hangindent}{56pt}{【}二黄慢板{】怕的是辜负了十载寒窗、九州遨游、八月科场、七篇的文章}\footnote{柴俊为老师出示湘剧《二进宫》老唱片,``七篇文章''一段有``怕的是效韩信,辜负我十载寒窗、九载遨游、八月科场、七篇锦绣、鹿鸣筵欢、五经魁首、四杆彩旗、三杯御酒、两朵金花、鳌头独占、独占鳌头,好容易个兵部侍郎。''近于谑而虐了。}{,才落得个兵部侍郎,无有下场。}

\setlength{\hangindent}{56pt}{【}二黄原板{】我好比鱼闯过千层罗网,受了些惊恐着些慌忙。}

\setlength{\hangindent}{56pt}{【}二黄原板{】千岁爷保学生满门无恙,拚着一死闯进昭阳。}

\setlength{\hangindent}{56pt}{【}二黄原板{】后面跟随兵部杨侍郎。}

\setlength{\hangindent}{56pt}{【}二黄原板{】观则见龙国太怀抱太子两泪汪汪,口口声声哭的是先皇。}

\setlength{\hangindent}{56pt}{【}二黄原板{】摆一摆手儿切莫要承当。}

\setlength{\hangindent}{56pt}{【}二黄原板{】学一个文站东,}

\setlength{\hangindent}{56pt}{【}二黄原板{】各自分班站立在两厢。}

\setlength{\hangindent}{56pt}{【}二黄原板{】为什么恨天怨地、颊带惆怅、所为哪桩。}

\setlength{\hangindent}{56pt}{【}二黄原板{】有什么大祸从天降,}

{(徐延昭\hspace{40pt}~

\setlength{\hangindent}{56pt}{【}二黄原板{】你就该请太师父女商量。)} }

\setlength{\hangindent}{56pt}{【}二黄原板{】他未必一旦无情起下了谋位的心肠,太师爷忠良呃。}

\setlength{\hangindent}{56pt}{【}二黄原板{】臣七月十三日三本奏上,龙国太偏偏要让啊。}

{(徐延昭\hspace{40pt}~

\setlength{\hangindent}{56pt}{【}二黄原板{】你言道大明朝$\cdots{}\cdots{}$)} }

{(徐延昭\hspace{40pt}~

\setlength{\hangindent}{56pt}{【}二黄原板{】龙国太慢把懿旨降,老臣言来听端详:~臣难学赵廉颇列国老将,臣难学汉马援大战昆阳。臣难学尉迟恭八寨来抢,臣难学老吴祯保驾百凉。臣年迈难把疆场上,臣年迈难挽马丝缰。臣年迈听不见金鼓声响,臣年迈眼昏花观不见阵头枪。老臣我年迈如霜降,要保国还有那兵部侍郎。)} }

\setlength{\hangindent}{56pt}{【}二黄原板{】吓得臣低头不敢望,战战兢兢启奏皇娘:~臣愿学严子陵垂钓矶上,臣愿学钟子期砍樵山岗。臣愿学诸葛亮躬耕垄上,臣愿学吕蒙正苦读寒窗。春来桃李齐开放,夏至荷花满呃池塘。到秋来菊桂花开金钱样,冬至腊梅带雪霜。弹一曲高山流水琴音亮,下一局残棋消遣闷愁肠。书几幅法书精神爽,巧笔丹青悬挂草堂。}\footnote{{关于这段唱词的  ``渔樵耕读''、``琴棋书画''、``四季花名''词句,各有所本,自成体系。}兹再举三例:~ \begin{quote}  吴小如先生记录的张伯驹先生的余派``渔樵耕读''、``四季花名''词\textsuperscript{{[}17{]}.}为:~ 臣要学姜子牙钓鱼渭上,臣要学钟子期采樵山岗,臣要学诸葛亮躬耕陇上,臣要学吕蒙正苦读寒窗。春来百花齐开放,夏至荷花满池塘,秋来菊花金钱样,冬至腊梅戴雪霜。  王庾生先生的``渔樵耕读''词\textsuperscript{{[}17{]}.}为:~ 臣不学兴周灭纣姜吕望,臣不学管仲相齐邦,臣不学三国中诸葛丞相,臣要学隐居山林的张子房。  宋湛清先生转述言派的``渔樵耕读''、``琴棋书画''、``四季花名''词\textsuperscript{{[}25{]}.}为:~ 臣不学兴周的姜公吕望,臣愿学钟子期砍樵山岗。臣不学尉迟恭种田在庄上,臣愿学吕蒙正苦读文章。抚一曲高山流水声嘹亮,闲无事对棋盘散心肠。看一本古书精神爽,巧笔丹青挂在两旁。春兰发花王者之相,夏时莲花满池塘。秋后的菊花高士样,冬日的红梅雪上添香。  \end{quote}  }{臣昨晚修下了辞王表章,今日里带进宫叩别皇娘。望国太开恩将臣放,落一个清闲自在、散淡逍遥、无忧无虑、无是无非,做什么兵部侍郎,臣告职还乡。}

{(徐延昭\hspace{40pt}~

\setlength{\hangindent}{56pt}{【{\akai 二黄原板}】吓坏了定国王,)} }

\setlength{\hangindent}{56pt}{【{\akai 二黄原板}】兵部的侍郎。}\hspace{10pt}~

{(徐延昭\hspace{40pt}~

\setlength{\hangindent}{56pt}{【{\akai 二黄原板}】自从盘古立帝邦,)} }

\setlength{\hangindent}{56pt}{【{\akai 二黄原板}】君跪臣来臣怎敢当。}

{(徐延昭\hspace{40pt}~

\setlength{\hangindent}{56pt}{【{\akai 二黄原板}】锦家邦来锦呢家邦,)} }

\setlength{\hangindent}{56pt}{【{\akai 二黄原板}】臣有一本启奏皇娘。}

{(徐延昭\hspace{40pt}~

\setlength{\hangindent}{56pt}{【{\akai 二黄原板}】昔日里有一个李文、李广,)} }

\setlength{\hangindent}{56pt}{【{\akai 二黄原板}】弟兄双双扶保朝廊。}

{(徐延昭\hspace{40pt}~

\setlength{\hangindent}{56pt}{【{\akai 二黄原板}】李文北门带箭丧,)} }

\setlength{\hangindent}{56pt}{【{\akai 二黄原板}】万家山前又收李刚。}

{(徐延昭\hspace{40pt}~

\setlength{\hangindent}{56pt}{【{\akai 二黄原板}】收了一将损伤一将,)} }

\setlength{\hangindent}{56pt}{【{\akai 二黄原板}】一将倒比一将强。}

{(徐延昭\hspace{40pt}~

\setlength{\hangindent}{56pt}{【{\akai 二黄原板}】到后来保太子登龙位上,)} }

\setlength{\hangindent}{56pt}{【{\akai 二黄原板}】反把那李广斩首庆阳。}

{(徐延昭\hspace{40pt}~

\setlength{\hangindent}{56pt}{【{\akai 二黄原板}】这都是前朝的忠臣良将,)} }

\setlength{\hangindent}{56pt}{【{\akai 二黄原板}】哪个忠良又有下场。}

{(徐延昭\hspace{40pt}~

\setlength{\hangindent}{56pt}{【{\akai 二黄原板}】困龙思想长呃江浪,)} }

\setlength{\hangindent}{56pt}{【{\akai 二黄原板}】虎落平阳想奔山岗。}

{(徐延昭\hspace{40pt}~

\setlength{\hangindent}{56pt}{【{\akai 二黄原板}】事到头来想一想,)} }

\setlength{\hangindent}{56pt}{【{\akai 二黄原板}】谁是忠良哪个是奸党。}

{(徐延昭\hspace{40pt}~

\setlength{\hangindent}{56pt}{【{\akai 二黄摇板}】铜锤一举王请上,)} }

\setlength{\hangindent}{56pt}{【{\akai 二黄摇板}】老杨波搀扶起定国王。}

\setlength{\hangindent}{56pt}{【{\akai 二黄摇板}】用手接过龙一条,二目睁睁把臣瞧。趁此机会生计呃巧}\footnote{吴小如先生告知,此句夏山楼主改为``莫不是嫌为臣官卑职小''。}{,}

{哎呀!}\hspace{40pt}~

\setlength{\hangindent}{56pt}{【{\akai 二黄摇板}】浑身上下似水浇,难以保朝。}

{臣!}\hspace{40pt}~

\setlength{\hangindent}{56pt}{【{\akai 二黄摇板}】叩罢头来谢罢恩呐,}

{(徐延昭\hspace{40pt}~

\setlength{\hangindent}{56pt}{【{\akai 二黄摇板}】徐延昭代驾}\footnote{吴小如先生早年曾有专文\textsuperscript{{[}17{]}.}指出,此处徐延昭原来唱的是``杨波带驾'',从文意上亦更通顺。}{且平身。)} }

\setlength{\hangindent}{56pt}{【{\akai 二黄摇板}】一文、}

{(徐延昭\hspace{40pt}~

\setlength{\hangindent}{56pt}{【{\akai 二黄摇板}】一武,)} }

\setlength{\hangindent}{56pt}{【{\akai 二黄摇板}】出宫门,}\hspace{10pt}~

\setlength{\hangindent}{56pt}{【{\akai 二黄摇板}】仗着太子叫皇兄呃。}

\setlength{\hangindent}{56pt}{【{\akai 二黄摇板}】大明江山全在呃你呀,}

{(徐延昭\hspace{40pt}~

\setlength{\hangindent}{56pt}{【{\akai 二黄摇板}】保国家全仗你杨家父子兵。)} }
