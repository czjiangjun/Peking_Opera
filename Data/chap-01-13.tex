\newpage
\subsubsection{\large\hei {武昭关~{\small 之}~伍员}}
\addcontentsline{toc}{subsection}{\hei 武昭关~\small{之}~伍员}

\hangafter=1                   %2. 设置从第1⾏之后开始悬挂缩进  %
\setlength{\parindent}{0pt}{
{\centerline{{[}{\hei 第一场}{]}}}\vspace{5pt}

{({\akai 念})拜相封侯印,盖世掌乾坤。扶保楚王主,四路扫烟尘。}

{俺,伍员,楚国监利人也。可恨楚王无道,纲常败坏,父纳子媳。是我心头不忿,保定皇家四口}\footnote{ ``皇家四口''一般指蔡国母、皇姑(一说为公子建)、马昭仪和王孙(芈胜);国母和皇姑(或公子建)在逃亡出京城时丧命,所以后面马昭仪有唱词为``保皇家四口丧两口,不怨将军怨着谁。''}{,我就反------出了朝堂。}

{俺伍员好比淤泥\textcolor{red}{污}住车}\footnote{李元皓{\scriptsize 君}建议此处作``误驻车''。此处从樊百乐{\scriptsize 君}转述的刘曾复先生确认文字,污(\textrm{w\`u})表示泥泞的东西将车轮胶着住了。姜骏按:~此处可能用``淤''字文意最准确,这个字在南方客家方言中仍保留有念\textrm{w\=u}音的。}{,沙滩------困蛟龙。}

{天呐,天!俺伍员好比一轮明月,却被乌云}\footnote{《京剧汇编》第六集~赵桐珊~藏本作``浮云''。}{遮盖也。}

{且住!耳旁听得金声呐喊,想是卞庄追兵到来,不免请出国太凤驾启程。}

{国太有请!}

{臣,伍员见驾。}

{国太、幼主,千岁,千千岁!}

{是臣听得金鼓呐喊,想是卞庄追兵到来,特请国太凤驾启程。}

{臣。}

{领{\footnotesize 呐}------旨!}

\setlength{\hangindent}{56pt}{【{\akai 二黄散板}】耳旁听得金鼓震,必是卞庄发来兵。远望松林一寺院,请国太下龙驹臣挡贼兵。}

{(马昭仪\hspace{20pt}【{\akai 二黄导板}】兵困禅宇)}

\setlength{\hangindent}{56pt}{【{\footnotesize 接}{\akai 二黄导板}】马后悲呀,}

\setlength{\hangindent}{56pt}{【{\akai 回龙}】那卞庄追兵似虎威。}

\setlength{\hangindent}{56pt}{【{\akai 二黄原板}】在葵花井({\akai 或}:~葵花柱)边拴战马,}

{虎头,}

\setlength{\hangindent}{56pt}{【{\akai 二黄原板}】虎头金枪插在丹墀。将身儿来至在西廊下,且听国太降玉音。}

\setlength{\hangindent}{56pt}{【{\akai 二黄原板}】未曾开言先落泪,臣有一本奏得知:~那卞庄贼好一比得胜狸猫,他欢似虎,为臣我虎落平阳反被犬欺。幼主爷似蛟龙,龙离了沧海反遭虾戏,龙国太似凤凰难腾飞。臣要保保定小千岁,难保国太杀出重围。}

\setlength{\hangindent}{56pt}{【{\akai 二黄原板}】子胥闻言屈膝跪,臣有二本奏端的:~国太年轻花容粉,为臣年纪正青春。知者说是臣保主,}

{(马昭仪\hspace{20pt}不知者~?)}

\setlength{\hangindent}{56pt}{【{\akai 二黄散板}】是一对少年人。}

\setlength{\hangindent}{56pt}{【{\akai 二黄散板}】一句话儿错出唇,国太一旁怒气生。屈膝跪在尘埃地,过往神灵听分明:~我若保主有假意,}

{罢!}

\setlength{\hangindent}{56pt}{【{\akai 二黄散板}】死在千军万马营。}

\setlength{\hangindent}{56pt}{【{\akai 二黄散板}】西廊下又来了保驾臣。}

\setlength{\hangindent}{56pt}{【{\akai 二黄散板}】撩铠甲且把佛殿进{\footnotesize 呐},见了国太愧煞人。}

\setlength{\hangindent}{56pt}{【{\akai 二黄散板}】用手接过小储君。}

\setlength{\hangindent}{56pt}{【{\akai 二黄散板}】打开甲胄藏幼主{\footnotesize 呃},拜别国太忙登程。}

\setlength{\hangindent}{56pt}{【{\akai 二黄散板}】葵花井边马牵定,国太有话快些云。}

\setlength{\hangindent}{56pt}{【{\akai 二黄散板}】有朝幼主登龙位,君是君来臣是臣。}

\setlength{\hangindent}{56pt}{【{\akai 二黄散板}】国太快把心拿稳,放我君臣早逃生。}

\setlength{\hangindent}{56pt}{【{\akai 二黄散板}】一见国太寻自尽,子胥倒做不忠臣。}

\setlength{\hangindent}{56pt}{【{\akai 二黄散板}】推倒土墙忙掩井{\footnotesize 呐}。}

{\vspace{3pt}{\centerline{{[}{\hei 第二场}{]}}}\vspace{5pt}}

{杀败了啊,杀败了。}\footnote{ 段公平{\scriptsize 君}注:~刘曾复先生曾介绍,此处面向里念白。}

{({\akai 念})走国离家日如年,不杀楚王心不甘。}

{({\akai 念})打开甲胄观幼主,}

{哈哈、哈哈、啊,呵哈哈哈$\cdots{}\cdots{}$({\hwfs 笑介})}

{({\akai 念})一夜须白过昭关。}

{呔!卞庄贼子休赶,伍将军去也。}
}
