\newpage
\phantomsection %实现目录的正确跳转
\section*{\large\hei {乾坤带~{\small 之}~李世民}}
\addcontentsline{toc}{section}{\hei 乾坤带~{\small 之}~李世民}

\hangafter=1                   %2. 设置从第1⾏之后开始悬挂缩进  %
\setlength{\parindent}{0pt}{

({\akai 内})摆驾!

\setlength{\hangindent}{56pt}{【{\akai 二黄慢板}】想当年老王爷带兵出征,下江南十余载才得回程。得了胜回朝来交旨复命,麒麟阁摆筵宴犒赏功臣。小杨广在席前言语不正,紫金杯打奸王惹下祸根。因此上修下了辞王表本,连夜里带家属转回故林。行至在临潼山被贼围困,多亏了秦恩公搭救满门。隋炀帝坐山河天心不顺,下扬州观琼花涂炭黎民。天降下五花棒奸王丧命,众公卿保父皇驾坐乾坤。遭不幸老王爷龙归海境,众老臣一个个辅孤王驾坐九重。恨只恨摩里沙兴兵犯境,命驸马秦怀玉前去剿平。但愿得此一去旗开得胜,但愿得此一去马到功成。内侍臣摆御驾九龙口进,又听得殿角下大放悲声。 }

\setlength{\hangindent}{81pt}{({\akai 或}:~【{\akai 二黄慢板}】想当年老王爷带兵出征,下江南十余载得胜回程,得胜归回朝来交旨复命,麒麟阁摆筵宴犒赏功臣。小杨广在席前言语不正,紫金杯打奸王惹下祸根。因此上修下了辞王表本,连夜里带家属转回故林。行至在临潼山前被贼围困,多亏了秦恩公搭救满门。隋炀帝坐江山天心不顺,下扬州观琼花涂炭黎民。天降下五花棒奸王丧命,众公卿保父皇驾坐龙庭。遭不幸老王爷龙归海境,窦太后望儿楼凤驭上宾。众老臣一个个忠心耿耿,一个个辅孤王驾坐金龙。恨只恨摩里沙打来奏本,他要夺孤王的锦绣乾坤。为王的在金殿传下旨意,命驸马秦怀玉去把贼平。但愿得此一去旗开得胜,但愿得此一去马到功成。侍内臣摆御驾九龙口进,又听得后宫院大放悲声。)}

梓童为何这等模样?

呜哙呀,有这等事?

梓童平身。

赐座。

内侍({\akai 或}:~来),宣银屏公主带子上殿。

(公主\hspace{30pt}万岁~!)

皇儿,你可知罪?

这才是皇儿的道理。

平身。

殿前武士,将秦英绑上殿来。

唗!胆大秦英,前番将程雄打死,孤不降罪于你,也就是了。怎么,今日又将詹老太师打死(金水桥前),二罪归一。

殿前武士,将秦英推出午门斩首({\akai 或}:~斩了)。

(长孙皇后\hspace{10pt}吾皇万岁。)

御妻平身。

赐座。

(梓童上殿有和本奏?)

(长孙皇后\hspace{10pt}$\cdots{}\cdots{}$是哪位大臣?)

小将秦英。

(长孙皇后\hspace{10pt}$\cdots{}\cdots{}$所犯何罪?)

前番将程雄打死,不降罪于他。今日又将詹老太师打死金水桥前,故而推出斩首。

(这个$\cdots{}\cdots{}$)

寡人龙心已定了,御妻不必多奏。

你又来多事了。

\setlength{\hangindent}{56pt}{【{\akai 西皮慢板}】劝御妻休得要把本奏上,孤怎比开河运无道隋炀。孤岂肯听信那谗言毁谤,孤岂肯斩忠良绝了那秦门后香。慢说是打死了詹老丞相,就是那庶民人也要抵偿。 }

是啊,你母女在金殿奏得本,爱梓童,哎,连一句话都讲不得吗?

梓童你有本?当殿奏来,寡人与你作主。

梓童平身。

梓童赐座。

哼,这还了得~!

呃,梓童奏来。

\setlength{\hangindent}{56pt}{【{\akai 西皮导板}】这桩事教孤王难以发放, }

(长孙皇后\hspace{10pt}儿啊$\cdots{}\cdots{}$({\hwfs 哭介}))

\setlength{\hangindent}{56pt}{【{\akai 西皮原板}】娘哭儿、女哭父好不惨伤。孤传旨斩了那秦英小将, }

唉!

\setlength{\hangindent}{56pt}{【{\akai 西皮原板}】孤皇儿在一旁两泪汪洋。孤传旨赦了那秦英小将, }

\textless{}\!{\bfseries\akai 哭头}\!\textgreater{}教孤好为难呐,(老皇妻呐,)

\setlength{\hangindent}{56pt}{【{\akai 西皮原板}】爱梓童殿角下哭断肝肠。唐贞观在龙书案前思后想, }

\setlength{\hangindent}{56pt}{【{\akai 西皮原板}】爱梓童近前来【{\footnotesize 转}{\akai 西皮二六}】细听端详:~你的父并不曾欺君罔上,可怜他金水桥一命身亡。孤劝你把此事休挂心上,哪有个人死后又能还阳。孤传旨挑选那能工巧匠,孤传旨修一座忠义祠堂。孤传旨赐你父金井玉葬,孤传旨文武臣送至在山岗,王去拈香,孤的爱梓童,你那里且免愁肠。}

\setlength{\hangindent}{52pt}{(詹妃\hspace{30pt}【{\akai 西皮摇板}】$\cdots{}\cdots{}$母女二人。) }

呃------

\setlength{\hangindent}{56pt}{【{\akai 西皮摇板}】唐贞观亦非是懦弱(的)皇上,为的是安黎民整顿朝纲。哪一个大胆人敢来违抗? }

\setlength{\hangindent}{56pt}{【{\akai 西皮二六}】叫皇儿近前来父女商量。小秦英打死了皇亲国丈,论国法就应该叫他抵偿。念秦门昔年间东杀西挡,念秦门只有这一脉后香。金銮殿父赐儿玉液琼浆,殿角下去哀求詹妃娘娘。 }

\setlength{\hangindent}{56pt}{【{\akai 西皮摇板}】好一个爱梓童宽宏大量,不由孤心内喜({\akai 或}:~不由孤龙心喜)暗称贤良。 }

\setlength{\hangindent}{56pt}{【{\akai 西皮摇板}】为王的在金殿把旨来降,午门外快赦回秦门儿郎。 }

非是寡人不斩于你,詹娘娘讲情,将你饶恕。一旁谢过詹娘娘({\akai 或}:~上前谢过詹娘娘)。

(秦英\hspace{30pt}谢过姨姥哦!)

御妻、梓童、皇儿回避。

内侍({\akai 或}:~内臣),宣徐勣上殿。

平身。

赐座。

卿家上殿,有何本奏?

呈上来。

待孤({\akai 或}:~待寡人)看来。

哎呀!原来驸马被困,卿家计将安在({\akai 或}:~卿家有何良策)?

小将秦英,打死皇亲国戚({\akai 或}:~皇亲国丈)。

已然赦却({\akai 或}:~已然赦回)。

依卿所奏。

内侍({\akai 或}:~内臣),宣秦英上殿。

秦英,今有你父被困摩里沙。命你带领人马前去征剿,得胜还朝({\akai 或}:~得胜回朝),将功折罪。外赐乾坤宝带,以振军威。

(秦英\hspace{30pt}谢万岁!)

见过儿徐祖父。

退班。

}
