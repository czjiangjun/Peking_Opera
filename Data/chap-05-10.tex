\newpage
\phantomsection %实现目录的正确跳转
\section*{\large\hei {打严嵩}~\protect\footnote{此戏的文字整理是在刘曾复先生两次为吴小如先生说《打严嵩》的邹应龙的基础上,结合刘先生为樊百乐{\scriptsize 君}说戏总讲的实况录音整理完成的,因为两个本子略有出入,文本尽量兼顾两方面。刘曾复先生说戏时专门说明,《打严嵩》是本戏《玉夔龙》的头一出。}}
\addcontentsline{toc}{section}{\hei 打严嵩}

\hangafter=1                   %2. 设置从第1⾏之后开始悬挂缩进  %
\setlength{\parindent}{0pt}{

\vspace{3pt}{\centerline{{[}{\hei 第一场}{]}}}\vspace{5pt}

(\textless{}\!{\bfseries\akai 小锣打上}\!\textgreater{})

\setlength{\hangindent}{52pt}{邹应龙\hspace{20pt}{[}{\akai 引子}{]}奸贼不参反成仇,不觉白了少年头。({\akai 或}:~奸佞当道,何日里,才把贼平。)}

\setlength{\hangindent}{52pt}{邹应龙\hspace{20pt}({\akai 念})空怀忠义胆,枉自伴君前。为参奸佞贼,只落口怨天。}

\setlength{\hangindent}{52pt}{邹应龙\hspace{20pt}下官邹应龙。嘉靖驾前为臣,官居外帘巡城御史之职。只因严嵩老贼,在朝专横({\akai 或}:~专权),苦害贤臣({\akai 或}:~忠良),是我等同年三十六人,在双塔寺中定下({\akai 或}:~盟下;~许下)心愿,以我为首,定要参倒老贼({\akai 或}:~以我为首,参奏老贼)。是我与开山王府常小国公定下一计,救下邱、马二将\footnote{一般戏中都作``邱、马二匠'',邱、马二人是两位匠人;~据陈超老师告知,刘曾复先生传授他的是``邱、马二将'',此处从之,下同。},去至严府,假献殷勤,暗中察勘老贼过错。今当三六九日,老贼议事之期,我不免去至相府,见机而行。({\akai 或}:~我不免去至严府,假献殷勤,暗中察勘。)({\akai 或}:~下官邹应龙。嘉靖驾前为臣,官居外帘巡丞御史。只因严嵩老贼,在朝专横,苦害贤臣,我等同年三十六人,在双塔寺中盟下誓愿,定要参奏老贼。以我举首,为国除害,日前闻知,老贼要害邱、马两将,是我与开山王府常小国公定下一计,将他二人救下。今去至严府,假献殷勤,暗中察勘老贼动静。今当三六九日,老贼议事之期,我不免前去见机行事。)}

\setlength{\hangindent}{52pt}{邹应龙\hspace{20pt}正是:~({\akai 念})胆大扳龙角,心雄拔虎牙。({\akai 或}:~胆壮攀龙角,心雄拔虎毛。{\akai 或}:~参倒贼奸佞,方为栋梁臣。)}

\setlength{\hangindent}{52pt}{邹应龙\hspace{20pt}【{\akai 西皮原板}】嘉靖爷坐山河风调雨顺,信宠那严嵩贼苦害贤臣。行奸计杀夏言丧了性命,进谗言害曾铣阖家满门。({\akai 或}:~他不该害死了杨继盛,杀夏言、害曾铣阖家满门。)众位年兄俱议论,不灭老贼枉为人。}

{(\textless{}\!{\bfseries\akai 小锣打下}\!\textgreater{})}

\vspace{3pt}{\centerline{{[}{\hei 第二场}{]}}}\vspace{5pt}

{(严侠{\hwfs 上},\textless{}\!{\bfseries\akai 小锣打上}\!\textgreater{})}

\setlength{\hangindent}{52pt}{严侠\hspace{30pt}啊哈------}

\setlength{\hangindent}{52pt}{严侠\hspace{30pt}({\akai 念})}相府门前七品官,见他容易见我难。

\setlength{\hangindent}{52pt}{严侠\hspace{30pt}我严侠是也,蒙太师信任,身当相府门官,小有威势。今当三六九日,太师议事、放官之期,恐有帘外官员前来拜谒,不免府门伺候便了。}

{(邹应龙{\hwfs 上},\textless{}\!{\bfseries\akai 小锣打上}\!\textgreater{})}

\setlength{\hangindent}{52pt}{邹应龙\hspace{20pt}({\akai 念})深山虎狼震,要做打猎人。}

\setlength{\hangindent}{52pt}{邹应龙\hspace{20pt}来此(已是)严府({\akai 或}:~相府),(看)那旁({\akai 或}:~厢)打坐严侠,待我向前({\akai 或}:~近前)------}

\setlength{\hangindent}{52pt}{邹应龙\hspace{20pt}尊官\footnote{``尊官''也可称``亲翁''。}请来搭话({\akai 或}:~啊,尊官请过来)。}

\setlength{\hangindent}{52pt}{(严侠\hspace{30pt}是。)}

\setlength{\hangindent}{52pt}{邹应龙\hspace{20pt}下官({\akai 或}:~弟)这厢有礼------}

\setlength{\hangindent}{52pt}{严侠\hspace{30pt}施礼为何,请问尊姓?({\akai 或}:~呃,诶,呃$\cdots{}\cdots{}$您是哪一位?)}

\setlength{\hangindent}{52pt}{邹应龙\hspace{20pt}怎么(你)连下官你都不认识了?}

\setlength{\hangindent}{52pt}{严侠\hspace{30pt}呃,瞧您面熟,不敢下笊篱。({\akai 或}:~(惊{\hwfs 介})眼熟称呼不上来。)}

\setlength{\hangindent}{52pt}{邹应龙\hspace{20pt}外帘御史邹应龙就是我哇。}

\setlength{\hangindent}{52pt}{严侠\hspace{30pt}哦,就是外帘御史邹应龙哇。({\akai 或}:~你就是外帘御史邹应龙。)}

\setlength{\hangindent}{52pt}{邹应龙\hspace{20pt}嗯。({\akai 或}:~正是。弟这厢有礼。)}

\setlength{\hangindent}{52pt}{严侠\hspace{30pt}到此何事?~({\akai 或}:~呃,施(/此)礼为何?)}

\setlength{\hangindent}{52pt}{(邹应龙\hspace{20pt}请问亲翁,太师可曾升堂议事?)}

\setlength{\hangindent}{52pt}{(严侠\hspace{30pt}已经升堂。)}

\setlength{\hangindent}{52pt}{邹应龙\hspace{20pt}求见太师。({\akai 或}:~小官求见)}

\setlength{\hangindent}{52pt}{严侠\hspace{30pt}为了何事?}

\setlength{\hangindent}{52pt}{邹应龙\hspace{20pt}有好心献上。}

\setlength{\hangindent}{52pt}{严侠\hspace{30pt}太师已经升堂。}

\setlength{\hangindent}{52pt}{邹应龙\hspace{20pt}烦劳通禀。}

\setlength{\hangindent}{52pt}{严侠\hspace{30pt}拿来。}

\setlength{\hangindent}{52pt}{邹应龙\hspace{20pt}什么,哦哦哦,想是手本。尊官(请看),小官\footnote{也可自称``弟''。}有失打点了。(烦劳通禀。)}

\setlength{\hangindent}{52pt}{严侠\hspace{30pt}哎,我说邹老爷,八成啊,您没来过这儿罢?~({\akai 或}:~诶,邹老爷我记得您是来过呀?)}

\setlength{\hangindent}{52pt}{邹应龙\hspace{20pt}不错,是初次。({\akai 或}:~乃是初次。)}

\setlength{\hangindent}{52pt}{严侠\hspace{30pt}这就难怪喽,你要见我家太师爷,可知这严府门口儿的规矩?({\akai 或}:~哎呀,这------那可就------难怪了,求见太师,可有个规矩。)}

\setlength{\hangindent}{52pt}{邹应龙\hspace{20pt}怎么,还有规矩?({\akai 或}:~府上啊$\cdots{}\cdots{}$府上------还有规矩?,{\akai 或}:~哦,严府上还有什么规矩?)}

\setlength{\hangindent}{52pt}{严侠\hspace{30pt}呃,(是了,)不以规矩,是不能成方圆呐。}

\setlength{\hangindent}{52pt}{邹应龙\hspace{20pt}哦,倒要请教。}

\setlength{\hangindent}{52pt}{严侠\hspace{30pt}诶,(求见太师,)大礼三百二,小礼二百四。}

\setlength{\hangindent}{52pt}{邹应龙\hspace{20pt}有礼------}

\setlength{\hangindent}{52pt}{严侠\hspace{30pt}就见!}

\setlength{\hangindent}{52pt}{邹应龙\hspace{20pt}无礼呢?}

\setlength{\hangindent}{52pt}{严侠\hspace{30pt}就免见呗。}

\setlength{\hangindent}{52pt}{邹应龙\hspace{20pt}啊尊官,下官({\akai 或}:~小官)今日来得忙迫,未带大礼,改日再补。}

\setlength{\hangindent}{52pt}{严侠\hspace{30pt}那就改日再见呗!}

\setlength{\hangindent}{52pt}{邹应龙\hspace{20pt}呜哙呀,难怪({\akai 或}:~怪道)严嵩老贼在朝专横({\akai 或}:~专权),就是他门下之人({\akai 或}:~府下之人),都是({\akai 或}:~也是)这般行事。改日再见$\cdots{}\cdots{}$我就改日再见罢------(哎呀!)想我邹应龙若说({\akai 或}:~要是说)他门下之人不过,还与老贼作的什么对啊?!}

\setlength{\hangindent}{52pt}{邹应龙\hspace{20pt}这这这$\cdots{}\cdots{}$嗯,我自有道理。}

\setlength{\hangindent}{52pt}{邹应龙\hspace{20pt}(亲翁请)过来!}

\setlength{\hangindent}{52pt}{严侠\hspace{30pt}诶,长调门了。}

\setlength{\hangindent}{52pt}{严侠\hspace{30pt}嘿,过来就过来。干什么?}

\setlength{\hangindent}{52pt}{邹应龙\hspace{20pt}(我来问你,)太师爷({\akai 或}:~老太师)每日可要朝王见驾?}

\setlength{\hangindent}{52pt}{严侠\hspace{30pt}自必朝王见驾。}

\setlength{\hangindent}{52pt}{邹应龙\hspace{20pt}朔望之日可于太庙降香?}

\setlength{\hangindent}{52pt}{严侠\hspace{30pt}怎不太庙降香?}

\setlength{\hangindent}{52pt}{邹应龙\hspace{20pt}着哇,待等老太师朝王见驾、太庙降香的时节({\akai 或}:~倘若老太师朝王见驾、太庙降香),(下官)我就走上前去({\akai 或}:~我就向前),(一把)拦住了轿杆,禀告老太师就说道:~那一日小官求见太师,有好心献上,(府上)有那么一位把门的官儿与我({\akai 或}:~小官)要什么``大礼三百二,小礼二百四'',有礼就见({\akai 或}:~有礼则见),无礼免见。今日呀!呵呵,我也不见了,(我们)改日再见罢!}

\setlength{\hangindent}{52pt}{严侠\hspace{30pt}哎诶诶,回来回来。}

\setlength{\hangindent}{52pt}{邹应龙\hspace{20pt}呃,改日再见罢!}

\setlength{\hangindent}{52pt}{严侠\hspace{30pt}你拿过来吧。嘿嘿嘿嘿$\cdots{}\cdots{}$({\hwfs 陪笑介})}

\setlength{\hangindent}{52pt}{严侠\hspace{30pt}我说邹老爷哟,我跟您(说句游戏之言,)闹着玩儿,怎么你就帘子脸儿------叭嗒,诶,就掉下来了?}

\setlength{\hangindent}{52pt}{邹应龙\hspace{20pt}哼!(这是邹老爷的脾气。)({\akai 或}:~那你不去通禀啊?!)}

\setlength{\hangindent}{52pt}{严侠\hspace{30pt}呃,通报可是通报哇。({\akai 或}:~得得得得了,我啊给你通禀报)这儿可是有尺寸的地方,诶,你得------往下站。}

\setlength{\hangindent}{52pt}{邹应龙\hspace{20pt}哦,我往下站。}

\setlength{\hangindent}{52pt}{严侠\hspace{30pt}诶,再往下站!}

\setlength{\hangindent}{52pt}{(邹应龙\hspace{20pt}嗯。)}

\setlength{\hangindent}{52pt}{严侠\hspace{30pt}还得再往下站!}

\setlength{\hangindent}{52pt}{邹应龙\hspace{20pt}啊?!你叫你邹老爷站在何处啊?啊?!}

\setlength{\hangindent}{52pt}{严侠\hspace{30pt}诶,得得得,您爱站哪儿就哪儿站得了。}

\setlength{\hangindent}{52pt}{邹应龙\hspace{20pt}哼,势利的小人!}

\setlength{\hangindent}{52pt}{(严侠\hspace{30pt}嘿,诚然。唉,打一早啊就觉得不大对劲儿,得,还得$\cdots{}\cdots{}$没法子给他通禀吧。)}

\setlength{\hangindent}{52pt}{严侠\hspace{30pt}有请太师爷。}

\setlength{\hangindent}{52pt}{严嵩\hspace{30pt}({\akai 内})【{\akai 西皮导板}】昔日有个王莽臣,}

\setlength{\hangindent}{52pt}{严嵩\hspace{30pt}【{\akai 西皮快板}】起下谋朝篡位心。巧计设下松棚会,药酒毒死亲帝君。老夫压定文共({\akai 或}:~和)武,要夺大明锦乾坤。三六九日放官任,}

\setlength{\hangindent}{52pt}{严嵩\hspace{30pt}【{\akai 西皮摇板}】威风好似五阎君。}

\setlength{\hangindent}{52pt}{严嵩\hspace{30pt}({\akai 念})}君不君来臣不臣,朝事不问嘉靖君。私造九龙冠一顶,要夺大明锦乾坤。

\setlength{\hangindent}{52pt}{严嵩\hspace{30pt}老夫,严嵩,嘉靖皇帝驾前为臣。我儿世蕃与嘉靖皇帝同年同月同日出生,嘉靖皇帝有天子之位,我儿难道就无有九五之尊?今当三、六、九日,恐有外帘官员前来拜谒,严侠,}

\setlength{\hangindent}{52pt}{严侠\hspace{30pt}有。}

\setlength{\hangindent}{52pt}{严嵩\hspace{30pt}大事通报({\akai 或}:~大事通禀),小事任你去办。}

\setlength{\hangindent}{52pt}{严侠\hspace{30pt}启禀太师:~外帘御史邹应龙求见。}

\setlength{\hangindent}{52pt}{严嵩\hspace{30pt}什么邹应龙,老夫这道衙门,见也罢,不见也罢。}

\setlength{\hangindent}{52pt}{严侠\hspace{30pt}他言道:~有好心进献({\akai 或}:~好心来献)。}

\setlength{\hangindent}{52pt}{严嵩\hspace{30pt}哦,有好心献上。吩咐站堂伺候。}

{({\hwfs 四}站堂{\hwfs 上},严嵩{\hwfs 坐内场椅})}

\setlength{\hangindent}{52pt}{严嵩\hspace{30pt}严侠,}

\setlength{\hangindent}{52pt}{严侠\hspace{30pt}有。}

\setlength{\hangindent}{52pt}{严嵩\hspace{30pt}传话出去:~教那邹应龙东角门施礼,西角门打躬,低头合目,报门而进。}

\setlength{\hangindent}{52pt}{严侠\hspace{30pt}遵命!}

\setlength{\hangindent}{52pt}{严侠\hspace{30pt}下面听者:~}

\setlength{\hangindent}{52pt}{邹应龙\hspace{20pt}在!}

\setlength{\hangindent}{52pt}{严侠\hspace{30pt}太师传话:~命邹应龙东角门施礼,}

\setlength{\hangindent}{52pt}{邹应龙\hspace{20pt}是!({\akai 或}:~有!)}

\setlength{\hangindent}{52pt}{严侠\hspace{30pt}西角门打躬。}

\setlength{\hangindent}{52pt}{邹应龙\hspace{20pt}是!}

\setlength{\hangindent}{52pt}{严侠\hspace{30pt}低头合目,报门而进呐!}

\setlength{\hangindent}{52pt}{邹应龙\hspace{20pt}是是是!}

\setlength{\hangindent}{52pt}{严侠\hspace{30pt}你要仔细,你要打点!}

\setlength{\hangindent}{52pt}{邹应龙\hspace{20pt}知、知、知、知道了!}

\setlength{\hangindent}{52pt}{邹应龙\hspace{20pt}【{\akai 西皮快板}】忽听严侠传我进,狐假虎威乱胡行。({\akai 或}:~严侠教我报门进,不由应龙怒气生。)怠慢本官多({\akai 或}:~莫)侥幸,老爷是尔对头人。东角门首礼施定,西角门首打一躬。整装敛容相府进({\akai 或}:~回廊进),}

\setlength{\hangindent}{52pt}{邹应龙\hspace{20pt}【{\akai 西皮摇板}】参拜皇王宠信臣。}

\setlength{\hangindent}{52pt}{邹应龙\hspace{20pt}参见太师。}

\setlength{\hangindent}{52pt}{严侠\hspace{30pt}禀太师:~邹应龙到。}

\setlength{\hangindent}{52pt}{严侠\hspace{30pt}邹应龙到。}

\setlength{\hangindent}{52pt}{严侠\hspace{30pt}邹应龙到啊------}

\setlength{\hangindent}{52pt}{严嵩\hspace{30pt}邹应龙!}

\setlength{\hangindent}{52pt}{邹应龙\hspace{20pt}小官在!({\akai 或}:~有)}

\setlength{\hangindent}{52pt}{严嵩\hspace{30pt}({\akai 念})老夫门深似海。}

\setlength{\hangindent}{52pt}{邹应龙\hspace{20pt}({\akai 念})小官好心献上。({\akai 或}:~启禀老太师:~小官有好心献上。)}

\setlength{\hangindent}{52pt}{严嵩\hspace{30pt}怎么有好心献上。({\akai 或}:~哦,你有好心献上么?)}

\setlength{\hangindent}{52pt}{邹应龙\hspace{20pt}正是。}

\setlength{\hangindent}{52pt}{严嵩\hspace{30pt}起来。}

\setlength{\hangindent}{52pt}{邹应龙\hspace{20pt}谢太师。}

\setlength{\hangindent}{52pt}{严嵩\hspace{30pt}严侠,与邹老爷({\akai 或}:~大人)看坐。}

\setlength{\hangindent}{52pt}{严侠\hspace{30pt}是,有坐。}

\setlength{\hangindent}{52pt}{邹应龙\hspace{20pt}且慢,太师虎威在此,哪有小官的座位?}

\setlength{\hangindent}{52pt}{严嵩\hspace{30pt}有话叙谈,焉有不坐之理?坐下。}

\setlength{\hangindent}{52pt}{邹应龙\hspace{20pt}小官告坐。}

\setlength{\hangindent}{52pt}{严嵩\hspace{30pt}哦------}

\setlength{\hangindent}{52pt}{严侠\hspace{30pt}诶,启禀太师爷,邹应龙他不敢坐呀。}

\setlength{\hangindent}{52pt}{严嵩\hspace{30pt}教他大胆坐下。}

\setlength{\hangindent}{52pt}{严侠\hspace{30pt}太师爷啊,教你坐下。}

\setlength{\hangindent}{52pt}{邹应龙\hspace{20pt}谢太师。}

\setlength{\hangindent}{52pt}{严侠\hspace{30pt}邹老爷您请坐请坐。}

\setlength{\hangindent}{52pt}{邹应龙\hspace{20pt}尊官请坐。}

\setlength{\hangindent}{52pt}{严侠\hspace{30pt}嘿嘿,我站惯了。}

\setlength{\hangindent}{52pt}{严嵩\hspace{30pt}邹应龙!}

\setlength{\hangindent}{52pt}{邹应龙\hspace{20pt}在!}

\setlength{\hangindent}{52pt}{(严嵩\hspace{30pt}你有何好心献上?}

\setlength{\hangindent}{52pt}{邹应龙\hspace{20pt}前番太师下朝命锦衣卫陆唐追赶何人?({\akai 或}:~启禀老太师,小官前日巡城打从开山王府经过,有那锦衣卫人员({\akai 或}:~有一位长官),他叫什么陆$\cdots{}\cdots{}$)}

\setlength{\hangindent}{52pt}{(严嵩\hspace{30pt}敢是陆唐?)}

\setlength{\hangindent}{52pt}{(邹应龙\hspace{20pt}正是。太师命他追赶何人?({\akai 或}:~不错,正是锦衣卫陆唐,押解邱、马二将。))}

\setlength{\hangindent}{52pt}{严嵩\hspace{30pt}追赶邱、马二将。}

\setlength{\hangindent}{52pt}{邹应龙\hspace{20pt}可曾追获?}

\setlength{\hangindent}{52pt}{严嵩\hspace{30pt}未见回报。}

\setlength{\hangindent}{52pt}{邹应龙\hspace{20pt}追赶不上了。}

\setlength{\hangindent}{52pt}{严嵩\hspace{30pt}怎见得?}

\setlength{\hangindent}{52pt}{邹应龙\hspace{20pt}那日小官巡查御街,路过开山王府,偶遇常小国公拦下邱、马二将,窝藏府中。并将陆唐抓进府去,吊在头门以里,仪门以外,日间打至犬吠,晚来打到鸡鸣。(打来打去,)还有两句歹话。({\akai 或}:~那陆唐解压邱、马两将路过开山王府,正遇常小国公拦下邱、马两将,窝藏府中。并将陆唐抓进府去,吊在头门以里,仪门以外,日间打至犬吠,晚来打至鸡鸣。还有两句言语。)}

\setlength{\hangindent}{52pt}{严嵩\hspace{30pt}哪两句歹话?({\akai 或}:~呃呃,有什么言语?)}

\setlength{\hangindent}{52pt}{邹应龙\hspace{20pt}老太师台前({\akai 或}:~在此),小官不敢言讲。}

\setlength{\hangindent}{52pt}{严嵩\hspace{30pt}大胆讲来!}

\setlength{\hangindent}{52pt}{邹应龙\hspace{20pt}打在他人腿上,羞在老太师脸上({\akai 或}:~犹如记在太师爷的脸上)。}

\setlength{\hangindent}{52pt}{严嵩\hspace{30pt}怎么讲?}

\setlength{\hangindent}{52pt}{邹应龙\hspace{20pt}老太师({\akai 或}:~太师爷的)脸上。}

\setlength{\hangindent}{52pt}{严嵩\hspace{30pt}好奴才!({\akai 或}:~可恼!)}

\setlength{\hangindent}{52pt}{严嵩\hspace{30pt}【{\akai 西皮摇板}】老夫闻言怒气冲,开言大骂常宝童。自古常言道得好,打犬还看主人翁。}

\setlength{\hangindent}{52pt}{严嵩\hspace{30pt}邹应龙!}

\setlength{\hangindent}{52pt}{邹应龙\hspace{20pt}在。}

\setlength{\hangindent}{52pt}{严嵩\hspace{30pt}是你亲眼得见({\akai 或}:~还是耳闻还是亲见)?}

\setlength{\hangindent}{52pt}{邹应龙\hspace{20pt}是小官亲眼得见。}

\setlength{\hangindent}{52pt}{(邹应龙\hspace{20pt}且慢,太师意欲何往?)}

\setlength{\hangindent}{52pt}{严嵩\hspace{30pt}老夫有意上殿奏本({\akai 或}:~参奏)。}

\setlength{\hangindent}{52pt}{(邹应龙\hspace{20pt}倘若圣上问起,何人见证?)}

\setlength{\hangindent}{52pt}{严嵩\hspace{30pt}你可与老夫做一见证?}

\setlength{\hangindent}{52pt}{邹应龙\hspace{20pt}小官愿与(老)太师做一见证。({\akai 或}:~当得效劳。)}

\setlength{\hangindent}{52pt}{邹应龙\hspace{20pt}呃呃,只怕不便。}

\setlength{\hangindent}{52pt}{严嵩\hspace{30pt}为何?}

\setlength{\hangindent}{52pt}{邹应龙\hspace{20pt}(怎奈)小官官卑职小,不能上(皇王金)殿见驾({\akai 或}:~见君),也是枉然。}

\setlength{\hangindent}{52pt}{严嵩\hspace{30pt}在朝官居何职?}

\setlength{\hangindent}{52pt}{邹应龙\hspace{20pt}外帘巡城御史。}

\setlength{\hangindent}{52pt}{严嵩\hspace{30pt}嗯,老夫不通圣命,升你为内帘御史。}

\setlength{\hangindent}{52pt}{(邹应龙\hspace{20pt}但不知几时领凭上任?)}

\setlength{\hangindent}{52pt}{严嵩\hspace{30pt}嘉靖封官,少不得周年半载;~老夫放官,即刻上任。({\akai 或}:~即时领凭上任。)}

\setlength{\hangindent}{52pt}{邹应龙\hspace{20pt}(多)谢太师。}

\setlength{\hangindent}{52pt}{严嵩\hspace{30pt}随同(或:随定)老夫道后。}

\setlength{\hangindent}{52pt}{邹应龙\hspace{20pt}遵命。}

\setlength{\hangindent}{52pt}{严嵩\hspace{30pt}顺轿上朝!}

{({\hwfs 一番两番},严嵩{\hwfs 跪台中间},{\hwfs 台中间摆一张桌},{\hwfs 上摆香炉})}

\setlength{\hangindent}{52pt}{严嵩\hspace{30pt}臣,严嵩见驾,吾皇万岁!}

\setlength{\hangindent}{52pt}{嘉靖\hspace{30pt}({\akai 内})老卿家上殿,有何本奏?}

\setlength{\hangindent}{52pt}{严嵩\hspace{30pt}启奏万岁:~今有常小国公窝藏邱、马两将,请旨定夺。}

\setlength{\hangindent}{52pt}{嘉靖\hspace{30pt}({\akai 内})老卿家亲眼还是耳闻?}

\setlength{\hangindent}{52pt}{严嵩\hspace{30pt}乃是外帘御史邹应龙亲眼所见,万岁圣鉴。}

\setlength{\hangindent}{52pt}{嘉靖\hspace{30pt}({\akai 内})邹应龙官卑职小,焉能上殿?}

\setlength{\hangindent}{52pt}{严嵩\hspace{30pt}老臣有一行大罪。}

\setlength{\hangindent}{52pt}{嘉靖\hspace{30pt}({\akai 内})卿家何罪之有?}

\setlength{\hangindent}{52pt}{严嵩\hspace{30pt}老臣未通圣命,放他为内帘御史。}

\setlength{\hangindent}{52pt}{嘉靖\hspace{30pt}({\akai 内})卿家放官,与朕一样。何罪之有?殿角赐座。}

\setlength{\hangindent}{52pt}{严嵩\hspace{30pt}老臣谢座。}

\setlength{\hangindent}{52pt}{嘉靖\hspace{30pt}({\akai 内})内侍,宣邹应龙上殿。}

\setlength{\hangindent}{52pt}{内侍\hspace{40pt}({\akai 内})邹应龙上殿呐。}

\setlength{\hangindent}{52pt}{邹应龙\hspace{20pt}({\akai 内})领旨!}

\setlength{\hangindent}{52pt}{邹应龙\hspace{20pt}({\akai 念})袖内藏文本,}

\setlength{\hangindent}{52pt}{严嵩\hspace{30pt}嗯------}

\setlength{\hangindent}{52pt}{邹应龙\hspace{20pt}({\akai 念})假意顺谗臣。}

\setlength{\hangindent}{52pt}{邹应龙\hspace{20pt}臣邹应龙见驾,吾皇万岁。}

\setlength{\hangindent}{52pt}{嘉靖\hspace{30pt}({\akai 内})邹应龙。}

\setlength{\hangindent}{52pt}{邹应龙\hspace{20pt}臣。}

\setlength{\hangindent}{52pt}{嘉靖\hspace{30pt}({\akai 内})常小国公窝藏邱、马两将,可是你({\akai 或}:~卿)亲眼得见?}

\setlength{\hangindent}{52pt}{邹应龙\hspace{20pt}是臣亲眼得见。}

\setlength{\hangindent}{52pt}{嘉靖\hspace{30pt}({\akai 内})与卿无干,下殿。}

\setlength{\hangindent}{52pt}{邹应龙\hspace{20pt}谢万岁。}

\setlength{\hangindent}{52pt}{邹应龙\hspace{20pt}正是:~({\akai 念})点起灯芯火,要烧万重山。}

{(邹应龙{\hwfs 一指},{\hwfs 下})}

\setlength{\hangindent}{52pt}{嘉靖\hspace{30pt}({\akai 内})太师接旨。}

\setlength{\hangindent}{52pt}{严嵩\hspace{30pt}臣。}

\setlength{\hangindent}{52pt}{嘉靖\hspace{30pt}({\akai 内})赐卿圣旨一道,校尉四十名,去至开山王府,押解常小国公上殿辩理。}

{(太监{\hwfs 递圣旨},严嵩{\hwfs 接旨})}

\setlength{\hangindent}{52pt}{严嵩\hspace{30pt}领旨。}

{({\hwfs 一翻两翻}\footnote{陈超老师介绍:~刘曾复先生传授此戏时说过,严嵩接旨后,``一翻两翻''是现在的演法,应该是唱四句``金钟三响王退殿,文武有怒不敢言,别驾离朝回府转,见了应龙说根源。''})}

{(严嵩{\hwfs 下轿},{\hwfs 进门坐中间},严侠{\hwfs 上场门上},邹应龙{\hwfs 下场门上},邹应龙{\hwfs 跟过来},严侠{\hwfs 站旁边})}

\setlength{\hangindent}{52pt}{邹应龙\hspace{20pt}老太师爷回府来了?}

\setlength{\hangindent}{52pt}{严嵩\hspace{30pt}回府来了。}

\setlength{\hangindent}{52pt}{邹应龙\hspace{20pt}圣上怎样传旨(下来)?}

\setlength{\hangindent}{52pt}{严嵩\hspace{30pt}圣上赐校尉四十名,去到开山王府,押解常小国公上殿辩理。}

\setlength{\hangindent}{52pt}{邹应龙\hspace{20pt}有道明君。}

\setlength{\hangindent}{52pt}{严嵩\hspace{30pt}嗯,真是有道的明君。}

\setlength{\hangindent}{52pt}{严嵩\hspace{30pt}顺轿。({\akai 或}:~来,外厢开道。)}

\setlength{\hangindent}{52pt}{邹应龙\hspace{20pt}且慢!小官阻道。({\akai 或}:~且慢,不是小官在此,老太师险些把事办错了。)}

\setlength{\hangindent}{52pt}{严嵩\hspace{30pt}为何阻道?}

\setlength{\hangindent}{52pt}{邹应龙\hspace{20pt}太师爷意欲何往?}

\setlength{\hangindent}{52pt}{严嵩\hspace{30pt}捉拿常宝童,上殿辩理。}

\setlength{\hangindent}{52pt}{邹应龙\hspace{20pt}启禀老太师:~万岁({\akai 或}:~圣上)虽赐有校尉(四十名),去到开山王府捉拿常小国公上殿,他有金杈银档,倘若打草惊蛇,反而大事难成。({\akai 或}:~倘若打草惊蛇,而况他又有金杈银档,倘若他抗旨不遵,反而大事难成。)}

\setlength{\hangindent}{52pt}{严嵩\hspace{30pt}依你之见?}

\setlength{\hangindent}{52pt}{邹应龙\hspace{20pt}依小官拙见:~呃,就在府内({\akai 或}:~相府)百里挑十,十里选一({\akai 或}:~百中选十,十中选一),挑选上四十名精壮的家丁,扮作校尉模样,随定({\akai 或}:~随同)太师爷去到({\akai 或}:~去往)开山王府,(前去宣读圣旨,)捉拿常小国公上殿辩理。他若上殿,也就罢了({\akai 或}:~常小国公遵旨上殿,倒还不讲;~{\akai 或}:~倘若他上殿,那还不讲;~)倘若他不肯上殿,这四十名校尉,({\akai 或}:~倘若他抗旨不遵,老太师命这些家人,{\akai 或}:~若是他抗旨不遵,老太师命随从们将他)抬么,也就把他抬上了(皇王的)金殿呐。}

\setlength{\hangindent}{52pt}{严嵩\hspace{30pt}哦,这是何计?}

\setlength{\hangindent}{52pt}{邹应龙\hspace{20pt}此乃万全之计。}

\setlength{\hangindent}{52pt}{严嵩\hspace{30pt}怎么,万全之计。呃------呵呵哈哈哈$\cdots{}\cdots{}$({\hwfs 笑介})}

\setlength{\hangindent}{52pt}{严嵩\hspace{30pt}(啊,邹大人,)看将起来,你是老夫心腹之人了。}

\setlength{\hangindent}{52pt}{邹应龙\hspace{20pt}着啊,本来是太师爷的心腹人呐。({\akai 或}:~太师爷夸奖了。{\akai 或}:~正是老太师的心腹之人。)呃呃呃,心腹之人好做,就是太师爷的金面难见呐。}

\setlength{\hangindent}{52pt}{严嵩\hspace{30pt}啊,你早来早见,晚了晚见。何言难见?}

\setlength{\hangindent}{52pt}{严侠\hspace{30pt}(低声)呃,是啊。}

\setlength{\hangindent}{52pt}{邹应龙\hspace{20pt}启禀太师爷:~小官求见太师爷({\akai 或}:~小官那日有好心献上),({\akai 或}:~不是哟,那日小官求见太师有好心献上。)府上有一位门官,(他)问我要什么大礼三百二,小礼二百四,有礼就见({\akai 或}:~有礼通禀),无礼免见。想小官为了太师爷之事,难道拿银子打点不成么?~({\akai 或}:~为了老太师之事,难道还要一个穷御史借({\akai 或}:~小官拿)银子(来)打点不成么?)}

\setlength{\hangindent}{52pt}{严嵩\hspace{30pt}啊,竟有此事?!({\akai 或}:~哦,有这等事?!)你可认识此人?}

\setlength{\hangindent}{52pt}{邹应龙\hspace{20pt}小官见面就认得。({\akai 或}:~呃,见面就认识了。)}

\setlength{\hangindent}{52pt}{严嵩\hspace{30pt}好,将他抓来见我!}

\setlength{\hangindent}{52pt}{严侠\hspace{30pt}糟糕,来了,要坏。({\akai 或}:~呃,坏了,不好,闹到我这儿来了。)}

\setlength{\hangindent}{52pt}{邹应龙\hspace{20pt}尊官,我看你往哪里去?({\akai 或}:~呃呃呃,亲翁({\akai 或}:~尊官)我看见你了。)}

\setlength{\hangindent}{52pt}{严侠\hspace{30pt}诶诶,邹老爷,我给您倒茶去。}

\setlength{\hangindent}{52pt}{邹应龙\hspace{20pt}不用。你当的好差呀!({\akai 或}:~诶,多谢多谢,你的差事当得好哇,太师爷传你呀。({\akai 或}:~啊,尊官,老太师唤你呀。))}

\setlength{\hangindent}{52pt}{严侠\hspace{30pt}诶,全仗您栽培。}

\setlength{\hangindent}{52pt}{邹应龙\hspace{20pt}我在太师爷台前讲了你的好话。({\akai 或}:~诶,一定要重重有赏,随我见太师。)}

\setlength{\hangindent}{52pt}{严侠\hspace{30pt}谢谢您的提拔。({\akai 或}:~呵呵,我的铺盖卷儿早就打好喽$\cdots{}\cdots{}$)}

\setlength{\hangindent}{52pt}{邹应龙\hspace{20pt}太师爷传。}

\setlength{\hangindent}{52pt}{严侠\hspace{30pt}不是我,不是我。}

\setlength{\hangindent}{52pt}{邹应龙\hspace{20pt}随我来!}

\setlength{\hangindent}{52pt}{严侠\hspace{30pt}报应到喽!}

\setlength{\hangindent}{52pt}{邹应龙\hspace{20pt}呃,就是此人。}

\setlength{\hangindent}{52pt}{严嵩\hspace{30pt}唗------}

\setlength{\hangindent}{52pt}{严嵩\hspace{30pt}你还要多少$\cdots{}\cdots{}$,斩了!({\akai 或}:~胆大严侠,帘外官员你讹诈了多少,扯下去打!)}

\setlength{\hangindent}{52pt}{严侠\hspace{30pt}留头讲话哟!~({\akai 或}:~太师爷,恩典喏!)}

\setlength{\hangindent}{52pt}{严侠\hspace{30pt}嘿嘿,邹老爷。}

\setlength{\hangindent}{52pt}{邹应龙\hspace{20pt}嗯哼!({\akai 或}:~是哪一位呀?)}

\setlength{\hangindent}{52pt}{严侠\hspace{30pt}哼!端起来了。邹老爷,邹老爷!}

\setlength{\hangindent}{52pt}{严侠\hspace{30pt}邹老爷您往下瞧,我在这儿呢。}

\setlength{\hangindent}{52pt}{邹应龙\hspace{20pt}哦,原来是尊官。}

\setlength{\hangindent}{52pt}{严侠\hspace{30pt}是我,诶,是,是我。({\akai 或}:~我在这儿呢。)}

\setlength{\hangindent}{52pt}{邹应龙\hspace{20pt}一时不见,你怎么矮了啊?}

\setlength{\hangindent}{52pt}{严侠\hspace{30pt}邹老爷我这儿给您跪着喽!}

\setlength{\hangindent}{52pt}{邹应龙\hspace{20pt}你亲翁你跪着则甚呐?({\akai 或}:~哦,不错,你是跪着呢。跪在你邹老爷面前则甚呐?)}

\setlength{\hangindent}{52pt}{严侠\hspace{30pt}邹老爷,您不知道,我在门口外头不是跟您------说笑话来着么。唉,太师爷知道了,要杀我({\akai 或}:~老太师,他要罚我)。}

\setlength{\hangindent}{52pt}{邹应龙\hspace{20pt}哦,要杀你?}

\setlength{\hangindent}{52pt}{严侠\hspace{30pt}诶。是要杀要杀。}

\setlength{\hangindent}{52pt}{邹应龙\hspace{20pt}你就让他杀啵!~ ({\akai 或}:~那就让他杀罢。)}

\setlength{\hangindent}{52pt}{严侠\hspace{30pt}哎$\cdots{}\cdots{}$得了得了,您给我说个情儿啵!}

\setlength{\hangindent}{52pt}{邹应龙\hspace{20pt}哪个?}

\setlength{\hangindent}{52pt}{严侠\hspace{30pt}诶,求您喽。}

\setlength{\hangindent}{52pt}{邹应龙\hspace{20pt}要我与你讲个人情?}

\setlength{\hangindent}{52pt}{严侠\hspace{30pt}是喽,非您不可!}

\setlength{\hangindent}{52pt}{邹应龙\hspace{20pt}你可晓得严府的规矩呀?({\akai 或}:~你可知邹老爷的规矩啊?)}

\setlength{\hangindent}{52pt}{严侠\hspace{30pt}邹老爷,咱们讲人情就甭讲规矩喽。({\akai 或}:~得了得了,您别提这规矩了。)}

\setlength{\hangindent}{52pt}{邹应龙\hspace{20pt}尊官,你讲过啊:~``没有规矩,不能成方圆''呐。({\akai 或}:~``大礼六百四,小礼四百八''。)}

\setlength{\hangindent}{52pt}{严侠\hspace{30pt}哎呦,常言道得好啊``大人不计小人过'',邹老爷,我可真服了您了。}

\setlength{\hangindent}{52pt}{邹应龙\hspace{20pt}我讲个人情不难,不知你的造化如何。({\akai 或}:~好,看你的造化。)}

\setlength{\hangindent}{52pt}{严侠\hspace{30pt}我就得瞧您的了。({\akai 或}:~我的造化就看您了。)}

\setlength{\hangindent}{52pt}{邹应龙\hspace{20pt}朝上跪!}

{(严侠{\hwfs 转身面朝里跪})}

\setlength{\hangindent}{52pt}{邹应龙\hspace{20pt}啊太师爷({\akai 或}:~启禀老太师),若责罚此人,小官出入多有不便呐。({\akai 或}:~启禀太师爷,若斩此人,于小官出入不便。)}

\setlength{\hangindent}{52pt}{严嵩\hspace{30pt}敢是与他讲情?({\akai 或}:~敢是与这奴才讲情?)}

\setlength{\hangindent}{52pt}{邹应龙\hspace{20pt}老太师开恩。({\akai 或}:~太师爷恩德。{\akai 或}:~太师爷开恩。)}

\setlength{\hangindent}{52pt}{严嵩\hspace{30pt}老夫准情。严侠,谢过邹老爷。}

\setlength{\hangindent}{52pt}{严侠\hspace{30pt}是,谢过邹老爷。}

\setlength{\hangindent}{52pt}{邹应龙\hspace{20pt}你要谢过老太师。}

\setlength{\hangindent}{52pt}{严侠\hspace{30pt}是,多谢老太师。}

\setlength{\hangindent}{52pt}{严嵩\hspace{30pt}严侠,还不谢过邹老爷。}

\setlength{\hangindent}{52pt}{严侠\hspace{30pt}多谢,多谢邹老爷。}

\setlength{\hangindent}{52pt}{邹应龙\hspace{20pt}谢过太师爷。({\akai 或}:~谢过老太师。)}

\setlength{\hangindent}{52pt}{严侠\hspace{30pt}多谢太师爷。}

\setlength{\hangindent}{52pt}{严嵩\hspace{30pt}谢过邹老爷。}

\setlength{\hangindent}{52pt}{邹应龙\hspace{20pt}谢过太师爷。}

\setlength{\hangindent}{52pt}{严嵩\hspace{30pt}谢过邹老爷。}

\setlength{\hangindent}{52pt}{邹应龙\hspace{20pt}谢过太师爷。}

\setlength{\hangindent}{52pt}{严侠\hspace{30pt}谢过太师爷。}

\setlength{\hangindent}{52pt}{严嵩\hspace{30pt}嗯------还不下去?!({\akai 或}:~教你谢过邹老爷,还不跪下?)}

{(严侠{\hwfs 跪})}

\setlength{\hangindent}{52pt}{邹应龙\hspace{20pt}诶,起来起来。}

\setlength{\hangindent}{52pt}{严侠\hspace{30pt}我是两边受着``夹板气''呢。}

\setlength{\hangindent}{52pt}{严嵩\hspace{30pt}心腹人。你为了老夫之事,还是乘马而来,还是坐轿而来?}

\setlength{\hangindent}{52pt}{邹应龙\hspace{20pt}小官乃是步行。({\akai 或}:~与太师爷办事,自然是步行而来。)}

\setlength{\hangindent}{52pt}{严嵩\hspace{30pt}岂不跑坏心腹人的两腿?({\akai 或}:~特以地辛苦了!)}

\setlength{\hangindent}{52pt}{邹应龙\hspace{20pt}当得效劳。}

\setlength{\hangindent}{52pt}{严嵩\hspace{30pt}圣上赐有白龙御马,老夫相赠。赐老夫穿朝御马,我赠与你乘骑了吧。}

\setlength{\hangindent}{52pt}{邹应龙\hspace{20pt}小官不敢乘骑。}

\setlength{\hangindent}{52pt}{严嵩\hspace{30pt}也罢,严侠,将万岁所赐老夫穿朝白龙御马,卸去金鞍玉辔,另备鞍韂。命与邹老爷乘骑。}

\setlength{\hangindent}{52pt}{严侠\hspace{30pt}没落到银子,落了个带马。}

\setlength{\hangindent}{52pt}{严嵩\hspace{30pt}你得罪了邹老爷,与邹老爷带马赔------礼。}

\setlength{\hangindent}{52pt}{严侠\hspace{30pt}喳!得,邹老爷您上马。}

\setlength{\hangindent}{52pt}{邹应龙\hspace{20pt}太师爷虎威在此,将马往下带。}

\setlength{\hangindent}{52pt}{严侠\hspace{30pt}哦,往下带,是。哨,哨,哨$\cdots{}\cdots{}$}

\setlength{\hangindent}{52pt}{严嵩\hspace{30pt}严侠,往上带。}

\setlength{\hangindent}{52pt}{严侠\hspace{30pt}喳!是。嗯、嗯$\cdots{}\cdots{}$}

\setlength{\hangindent}{52pt}{严侠\hspace{30pt}邹老爷。}

\setlength{\hangindent}{52pt}{邹应龙\hspace{20pt}(这是)有尺寸的地方,往下带。}

\setlength{\hangindent}{52pt}{严侠\hspace{30pt}哦,往下带,哨,哨,哨$\cdots{}\cdots{}$}

\setlength{\hangindent}{52pt}{严嵩\hspace{30pt}无用的奴才,往上带。}

\setlength{\hangindent}{52pt}{严侠\hspace{30pt}喳!邹老爷。}

\setlength{\hangindent}{52pt}{邹应龙\hspace{20pt}还要往下带。}

\setlength{\hangindent}{52pt}{严嵩\hspace{30pt}往$\cdots{}\cdots{}$}

\setlength{\hangindent}{52pt}{邹应龙\hspace{20pt}往$\cdots{}\cdots{}$}

\setlength{\hangindent}{52pt}{严侠\hspace{30pt}邹老爷!够瞧老大半天的喽!}

\setlength{\hangindent}{52pt}{严嵩\hspace{30pt}上马去罢!}

\setlength{\hangindent}{52pt}{邹应龙\hspace{20pt}谢太师!}

\setlength{\hangindent}{52pt}{邹应龙\hspace{20pt}【{\akai 西皮散板}】月台下辞别了严太尊,}

\setlength{\hangindent}{52pt}{严侠\hspace{30pt}送邹老爷!}

\setlength{\hangindent}{52pt}{邹应龙\hspace{20pt}【{\akai 西皮散板}】叫声尊官你试听:~}

\setlength{\hangindent}{52pt}{严侠\hspace{30pt}邹老爷您有话请讲。}

\setlength{\hangindent}{52pt}{邹应龙\hspace{20pt}【{\akai 西皮散板}】三百两银子有多少,}

\setlength{\hangindent}{52pt}{严侠\hspace{30pt}您没给我可也没要。}

\setlength{\hangindent}{52pt}{邹应龙\hspace{20pt}【{\akai 西皮散板}】有道是脸面值千金。}

\setlength{\hangindent}{52pt}{严侠\hspace{30pt}您可真有点面子。({\akai 或}:~您好大面子哦)}

\setlength{\hangindent}{52pt}{邹应龙\hspace{20pt}【{\akai 西皮散板}】适才太师对我论,}

\setlength{\hangindent}{52pt}{严侠\hspace{30pt}您可听得清楚。({\akai 或}:~太师爷就与您说得来。)}

\setlength{\hangindent}{52pt}{邹应龙\hspace{20pt}【{\akai 西皮散板}】我就是太师爷心腹的人。}

\setlength{\hangindent}{52pt}{严侠\hspace{30pt}一棵树的枣儿,就红了您这么一个。}

\setlength{\hangindent}{52pt}{邹应龙\hspace{20pt}【{\akai 西皮散板}】从今后不把你当尊官敬,}

\setlength{\hangindent}{52pt}{严侠\hspace{30pt}呃,您把我当什么呢?}

\setlength{\hangindent}{52pt}{邹应龙\hspace{20pt}哦------}

\setlength{\hangindent}{52pt}{严侠\hspace{30pt}我可没这小名儿。}

\setlength{\hangindent}{52pt}{邹应龙\hspace{20pt}【{\akai 西皮散板}】你就是邹老爷牵马坠镫------}

\setlength{\hangindent}{52pt}{严侠\hspace{30pt}手都酸喽!({\akai 或}:~来回不要盘缠喏。)}

\setlength{\hangindent}{52pt}{邹应龙\hspace{20pt}【{\akai 西皮散板}】一个势利的小人。}

\setlength{\hangindent}{52pt}{严侠\hspace{30pt}这回他可出了气喽!({\akai 或}:~嘿我也值得这么一骂)}

\setlength{\hangindent}{52pt}{严嵩\hspace{30pt}【{\akai 西皮散板}】咬牙切齿把宝童恨,窝藏邱、马罪不轻。人来与爷把道引,捉拿宝童面圣君。}

{(\textless{}\!{\bfseries\akai 抽头}\!\textgreater{}{\hwfs 下})}

\vspace{3pt}{\centerline{{[}{\hei 第三场}{]}}}\vspace{5pt}

{(\textless{}\!{\bfseries\akai 长锤}\!\textgreater{}{\hwfs 四}太监{\hwfs 上},常宝童{\hwfs 上})}

\setlength{\hangindent}{52pt}{常宝童\hspace{20pt}【{\akai 西皮摇板}】先祖在朝功劳大,保定太祖定邦家。钦赐银挡与金杈,}

{(常宝童{\hwfs 坐外场椅})}

\setlength{\hangindent}{52pt}{常宝童\hspace{20pt}【{\akai 西皮摇板}】官居王位第一家。}

{(\textless{}快长锤\textgreater{}邹应龙{\hwfs 上})}

\setlength{\hangindent}{52pt}{邹应龙\hspace{20pt}【{\akai 西皮快板}】严府假意献殷勤,老贼把我当心腹人。暗藏金钩来拿定\footnote{``暗藏金钩来拿定,千岁驾前说分明。''两句,陈超老师从刘曾复先生学的是``暗放金钩海鳌引,千岁驾前问安宁。''}({\akai 或}:~暗藏金钩探鳌鱼),}

\setlength{\hangindent}{52pt}{(邹应龙\hspace{20pt}有劳了!)}

\setlength{\hangindent}{52pt}{邹应龙\hspace{20pt}【{\akai 西皮摇板}】千岁驾前({\akai 或}:~台前)说分明。({\akai 或}:~见了千岁说分明。)}

\setlength{\hangindent}{52pt}{邹应龙\hspace{20pt}参见千岁。}

\setlength{\hangindent}{52pt}{常宝童\hspace{20pt}邹官儿平身呐。}

\setlength{\hangindent}{52pt}{邹应龙\hspace{20pt}谢千岁。}

\setlength{\hangindent}{52pt}{常宝童\hspace{20pt}孩子们给邹官儿看坐。}

\setlength{\hangindent}{52pt}{邹应龙\hspace{20pt}谢坐。}

\setlength{\hangindent}{52pt}{常宝童\hspace{20pt}邹官儿({\akai 或}:~卿家)你发了财了?}

\setlength{\hangindent}{52pt}{邹应龙\hspace{20pt}怎见得是臣发了财了呢?}

\setlength{\hangindent}{52pt}{常宝童\hspace{20pt}你身穿大红,岂不是发了财了吗?}

\setlength{\hangindent}{52pt}{邹应龙\hspace{20pt}不错,是臣升了官了啊。}

\setlength{\hangindent}{52pt}{常宝童\hspace{20pt}是啊,升的什么官啊?}

\setlength{\hangindent}{52pt}{邹应龙\hspace{20pt}(外帘御史升为)内帘御史。}

\setlength{\hangindent}{52pt}{常宝童\hspace{20pt}可喜可贺,谁({\akai 或}:~何人)的保举?}

\setlength{\hangindent}{52pt}{邹应龙\hspace{20pt}(乃)严嵩(的)保举({\akai 或}:~保荐)。({\akai 或}:~严太师的保举)}

\setlength{\hangindent}{52pt}{常宝童\hspace{20pt}怎么着,变了奸臣了啊?!}

\setlength{\hangindent}{52pt}{常宝童\hspace{20pt}嘿!孩子们,撤座!({\akai 或}:~撤奸臣的座儿)}

\setlength{\hangindent}{52pt}{邹应龙\hspace{20pt}啊,慢来慢来,千岁,虽则是严嵩的保举({\akai 或}:~保荐),(臣)还是为开山王府办事啊。}

\setlength{\hangindent}{52pt}{常宝童\hspace{20pt}哦,你还是给本御办事?}

\setlength{\hangindent}{52pt}{邹应龙\hspace{20pt}是。}

\setlength{\hangindent}{52pt}{常宝童\hspace{20pt}那你就再坐下。}

\setlength{\hangindent}{52pt}{邹应龙\hspace{20pt}谢千岁!({\akai 或}:~谢坐。)}

\setlength{\hangindent}{52pt}{常宝童\hspace{20pt}邹官儿你可知罪?}

\setlength{\hangindent}{52pt}{邹应龙\hspace{20pt}臣知何罪?}

\setlength{\hangindent}{52pt}{常宝童\hspace{20pt}昨儿个约定与本御围棋玩耍,今儿个才到,岂不是有罪吗?}

\setlength{\hangindent}{52pt}{邹应龙\hspace{20pt}今日开山王府围不得棋了啊!~({\akai 或}:~哎呀,吓得微臣吃了一惊。千岁,如今这开山王府是围不得棋了啊!)}

\setlength{\hangindent}{52pt}{常宝童\hspace{20pt}怎么围不得了?}

\setlength{\hangindent}{52pt}{邹应龙\hspace{20pt}眼前就有一场大祸!}

\setlength{\hangindent}{52pt}{常宝童\hspace{20pt}哎呦,你可别吓唬我啊。}

\setlength{\hangindent}{52pt}{常宝童\hspace{20pt}我开山府欠粮?}

\setlength{\hangindent}{52pt}{邹应龙\hspace{20pt}不欠粮。}

\setlength{\hangindent}{52pt}{常宝童\hspace{20pt}缺饷?}

\setlength{\hangindent}{52pt}{邹应龙\hspace{20pt}也不缺饷。}

\setlength{\hangindent}{52pt}{常宝童\hspace{20pt}一不欠粮,二不缺饷,没什么大祸,开山府怎么围不得了棋了呢?}

\setlength{\hangindent}{52pt}{邹应龙\hspace{20pt}可恨({\akai 或}:~只因)严嵩老贼金殿参奏一本,要千岁与他上殿辩理。({\akai 或}:~道千岁隐藏邱、马二将,要千岁上朝与他辩理。)}

\setlength{\hangindent}{52pt}{常宝童\hspace{20pt}辩的何理?}

\setlength{\hangindent}{52pt}{邹应龙\hspace{20pt}隐藏邱、马二将。({\akai 或}:~乃是邱、马两将之事。)}

\setlength{\hangindent}{52pt}{常宝童\hspace{20pt}哎,本御将他二人献出就是嘛!}

\setlength{\hangindent}{52pt}{邹应龙\hspace{20pt}千岁此言差矣。({\akai 或}:~哎呀献不得,献不得。)}

\setlength{\hangindent}{52pt}{常宝童\hspace{20pt}何差?({\akai 或}:~为何?)}

\setlength{\hangindent}{52pt}{邹应龙\hspace{20pt}此时献了邱、马二将({\akai 或}:~若是将他二人献出),岂不(是)弄假成真?({\akai 或}:~既要献,当初就不该救啊。)}

\setlength{\hangindent}{52pt}{常宝童\hspace{20pt}那依你之见呢?~({\akai 或}:~卿家的高见?)}

\setlength{\hangindent}{52pt}{邹应龙\hspace{20pt}这$\cdots{}\cdots{}$({\hwfs 思介})}

\setlength{\hangindent}{52pt}{邹应龙\hspace{20pt}依臣拙见:~少时老贼捧旨到此,千岁用金锏挡住他的校尉,只放他一人进府。({\akai 或}:~也罢,少时老贼捧旨前来,千岁吩咐抬起皇挡将他一人放进府来,千岁用金锏挡住他的校尉。{\akai 或}:~少时老贼捧旨到此,吩咐人役将皇挡抬出,将老贼一人挡进府来。)}

\setlength{\hangindent}{52pt}{常宝童\hspace{20pt}孩子们听见了没有?}

\setlength{\hangindent}{52pt}{众\hspace{40pt}听见了。}

\setlength{\hangindent}{52pt}{邹应龙\hspace{20pt}彼时他必然在银安殿上开读({\akai 或}:~宣读)圣旨,千岁言道:~老太师不必开读,本御知罪。({\akai 或}:~他彼时必然宣读圣旨,千岁不要他宣读,就说``本御知罪''。)}

\setlength{\hangindent}{52pt}{常宝童\hspace{20pt}本御何罪之有?}

\setlength{\hangindent}{52pt}{邹应龙\hspace{20pt}愿将邱、马二将献上。}

\setlength{\hangindent}{52pt}{常宝童\hspace{20pt}诏罢之后?}

\setlength{\hangindent}{52pt}{邹应龙\hspace{20pt}请过圣旨。({\akai 或}:~将圣旨请过。)}

\setlength{\hangindent}{52pt}{常宝童\hspace{20pt}请过了圣旨?}

\setlength{\hangindent}{52pt}{邹应龙\hspace{20pt}赐他一个座位。}

\setlength{\hangindent}{52pt}{常宝童\hspace{20pt}不成!哼哼,开山王府哪有老贼的座位?}

\setlength{\hangindent}{52pt}{邹应龙\hspace{20pt}看在微臣的份上。({\akai 或}:~皇王金殿,二十四把金交椅,尚有老贼的座位呀。)}

\setlength{\hangindent}{52pt}{常宝童\hspace{20pt}哼,这可是瞧了你的。}

\setlength{\hangindent}{52pt}{邹应龙\hspace{20pt}多谢千岁。}

\setlength{\hangindent}{52pt}{常宝童\hspace{20pt}坐下之后?}

\setlength{\hangindent}{52pt}{邹应龙\hspace{20pt}问他是忠是奸呐。({\akai 或}:~千岁问他是忠臣还是奸臣。)}

\setlength{\hangindent}{52pt}{常宝童\hspace{20pt}他准得说是大大的忠臣呐。}

\setlength{\hangindent}{52pt}{邹应龙\hspace{20pt}(既然是大大的忠臣,)教他抬头观看!}

\setlength{\hangindent}{52pt}{常宝童\hspace{20pt}呃,看,看什么呀?}

\setlength{\hangindent}{52pt}{邹应龙\hspace{20pt}千岁将笼帘卷起,教他观看老皇御容、伴驾王\footnote{明太祖和常遇春(录音中作徐达)的画像。}真像({\akai 或}:~千岁将老皇御容、伴驾王的真像悬挂中堂),他身为大臣,见君不参,就是({\akai 或}:~就有)一项大罪({\akai 或}:~一行大罪)。}

\setlength{\hangindent}{52pt}{常宝童\hspace{20pt}他必然有辩呐。}

\setlength{\hangindent}{52pt}{邹应龙\hspace{20pt}由他({\akai 或}:~容他)去辩。}

\setlength{\hangindent}{52pt}{常宝童\hspace{20pt}辩罢之后?}

\setlength{\hangindent}{52pt}{(邹应龙\hspace{20pt}再赐他一个座位。)}

\setlength{\hangindent}{52pt}{邹应龙\hspace{20pt}(千岁)问他:~开山府欠粮?({\akai 或}:~再教他坐下,问道:~开山府欠粮?)}

\setlength{\hangindent}{52pt}{常宝童\hspace{20pt}不欠粮。}

\setlength{\hangindent}{52pt}{邹应龙\hspace{20pt}缺饷?}

\setlength{\hangindent}{52pt}{常宝童\hspace{20pt}不缺饷。}

\setlength{\hangindent}{52pt}{邹应龙\hspace{20pt}一不欠粮,二不缺饷,(太师爷你)来到开山王府有何贵干呐({\akai 或}:~太师到此何干呐)?}

\setlength{\hangindent}{52pt}{常宝童\hspace{20pt}呃,请本御上殿辩理呐?}

\setlength{\hangindent}{52pt}{邹应龙\hspace{20pt}辩得什么理啊?拿来。}

\setlength{\hangindent}{52pt}{常宝童\hspace{20pt}什么?}

\setlength{\hangindent}{52pt}{邹应龙\hspace{20pt}圣旨啊。}

\setlength{\hangindent}{52pt}{常宝童\hspace{20pt}本御方才请过去了?}

\setlength{\hangindent}{52pt}{邹应龙\hspace{20pt}原是请过了啊,千岁(你)与他个不认账啊!}

\setlength{\hangindent}{52pt}{常宝童\hspace{20pt}诶!这我成!({\akai 或}:~诶,我这回说回瞎话。)}

\setlength{\hangindent}{52pt}{邹应龙\hspace{20pt}千岁(就动起怒来,)言道:~唗!胆大严嵩,今日在朝害文,明日在朝害武,(害来害去,害到小王({\akai 或}:~本御)的头上来了。你在金殿奉了圣旨,在哪厢失落,)今天竟然来到开山王府,前来讹诈本御,这回不打你两下惯坏了你的下次。来呀!脱袍打严嵩。吩咐左右。乒乒乓乓,糊里糊涂,打他一顿轰了出去,千岁的气也出了,你看此计如何?({\akai 或}:~就动起怒来,骂一声:~胆大的严嵩,今日在朝害文,明日在朝害武,害来害去,害到本御的头上来了!今日若不打你,尤恐惯坏了你的下次。吩咐人役脱袍解带,乒乒乓乓,将他一顿暴打,糊里糊涂,就轰了出去。)}

\setlength{\hangindent}{52pt}{常宝童\hspace{20pt}打出祸来呢!}

\setlength{\hangindent}{52pt}{邹应龙\hspace{20pt}由(微)臣担待。}

\setlength{\hangindent}{52pt}{常宝童\hspace{20pt}那可就瞧你的了。}

\setlength{\hangindent}{52pt}{邹应龙\hspace{20pt}臣告便。}

\setlength{\hangindent}{52pt}{邹应龙\hspace{20pt}啊尊侍,少时老贼到此,要打在他的身上,不可打在他的脸上。}

\setlength{\hangindent}{52pt}{侍卫\hspace{30pt}却是为何?}

\setlength{\hangindent}{52pt}{邹应龙\hspace{20pt}自有妙用。}

\setlength{\hangindent}{52pt}{侍卫\hspace{30pt}哦,我等记下。}

\setlength{\hangindent}{52pt}{侍卫\hspace{30pt}圣旨下。}

\setlength{\hangindent}{52pt}{常宝童\hspace{20pt}诶,来了来了。}

\setlength{\hangindent}{52pt}{常宝童\hspace{20pt}你在哪儿?({\akai 或}:~那你哪里藏躲呢?)}

\setlength{\hangindent}{52pt}{邹应龙\hspace{20pt}屏风后面。}

\setlength{\hangindent}{52pt}{(邹应龙\hspace{20pt}啊列位,少时老贼到此,浑身都容你们打,面貌不要打坏。我自有用处,记下了。拜托拜托。)}

\setlength{\hangindent}{52pt}{邹应龙\hspace{20pt}老贼来了。}

\setlength{\hangindent}{52pt}{常宝童\hspace{20pt}快去回避。}

\setlength{\hangindent}{52pt}{常宝童\hspace{20pt}香案接旨。}

{(严嵩{\hwfs 上},常宝童{\hwfs 用锏指},严校尉{\hwfs 下},严嵩{\hwfs 进门})}

\setlength{\hangindent}{52pt}{严嵩\hspace{30pt}圣旨下。}

\setlength{\hangindent}{52pt}{常宝童\hspace{20pt}老太师不用宣读,小王知罪。}

\setlength{\hangindent}{52pt}{严嵩\hspace{30pt}小千岁何罪之有?}

\setlength{\hangindent}{52pt}{常宝童\hspace{20pt}愿将邱、马二将献上当今。}

\setlength{\hangindent}{52pt}{严嵩\hspace{30pt}圣旨转过。}

\setlength{\hangindent}{52pt}{常宝童\hspace{20pt}香案供奉。}

{(常宝童{\hwfs 中间坐})}

\setlength{\hangindent}{52pt}{严嵩\hspace{30pt}小千岁请上,老臣大礼参拜。}

\setlength{\hangindent}{52pt}{常宝童\hspace{20pt}老太师您年高有德,不拜也罢。}

\setlength{\hangindent}{52pt}{严嵩\hspace{30pt}见了千岁,哪有不拜之理?}

\setlength{\hangindent}{52pt}{常宝童\hspace{20pt}哦,那就是小王我受你一拜}

\setlength{\hangindent}{52pt}{众\hspace{40pt}跪。}

\setlength{\hangindent}{52pt}{众\hspace{40pt}一个。}

\setlength{\hangindent}{52pt}{众\hspace{40pt}一个。}

\setlength{\hangindent}{52pt}{众\hspace{40pt}一个。}

\setlength{\hangindent}{52pt}{严嵩\hspace{30pt}小千岁,老臣我磕了三个响头了。}

\setlength{\hangindent}{52pt}{常宝童\hspace{20pt}老太师请起。}

\setlength{\hangindent}{52pt}{严嵩\hspace{30pt}谢千岁。}

\setlength{\hangindent}{52pt}{常宝童\hspace{20pt}孩子们,与老太师看座。}

\setlength{\hangindent}{52pt}{严嵩\hspace{30pt}千岁在此,哪有老臣的座位。}

\setlength{\hangindent}{52pt}{常宝童\hspace{20pt}哎呀,我的老太师啊,金殿之上,二十四把金交椅,都有您的座位,何况我小小的开山王府呢。您请坐请坐。}

\setlength{\hangindent}{52pt}{严嵩\hspace{30pt}谢千岁。}

\setlength{\hangindent}{52pt}{常宝童\hspace{20pt}别谢了,您请坐吧。老太师,您好啊。}

\setlength{\hangindent}{52pt}{严嵩\hspace{30pt}老臣有何德能,敢劳千岁动问。}

\setlength{\hangindent}{52pt}{常宝童\hspace{20pt}闲谈。}

\setlength{\hangindent}{52pt}{严嵩\hspace{30pt}谢千岁。}

\setlength{\hangindent}{52pt}{常宝童\hspace{20pt}他又谢。你在朝是忠臣,还是奸臣?}

\setlength{\hangindent}{52pt}{严嵩\hspace{30pt}为臣是大大的忠臣。}

\setlength{\hangindent}{52pt}{常宝童\hspace{20pt}瞧您这样,您这个扮相就是个忠臣。}

\setlength{\hangindent}{52pt}{常宝童\hspace{20pt}孩子们,将笼帘卷起!}

\setlength{\hangindent}{52pt}{严嵩\hspace{30pt}({\hwfs 惊介})哎呀呀!这个娃娃,领了哪个高明先生的指教,将老王的御容、伴驾王的真相悬挂中堂,老夫身为大臣者见君不参,就有一行大罪。哎呀,这这这$\cdots{}\cdots{}$嗯,我自有道理。}

\setlength{\hangindent}{52pt}{严嵩\hspace{30pt}啊小千岁,可容老臣一辩。}

\setlength{\hangindent}{52pt}{常宝童\hspace{20pt}孩子们!金盆打水。}

\setlength{\hangindent}{52pt}{严嵩\hspace{30pt}呃,呃$\cdots{}\cdots{}$打水何用呐?}

\setlength{\hangindent}{52pt}{常宝童\hspace{20pt}老太师你会变呐,呃,变个乌龟,与小王玩耍玩耍。}

\setlength{\hangindent}{52pt}{严嵩\hspace{30pt}呃,老臣乃是舌辩之辩。}

\setlength{\hangindent}{52pt}{常宝童\hspace{20pt}那你就辩!}

\setlength{\hangindent}{52pt}{严嵩\hspace{30pt}千岁,``今非朔望,闲不参君''。}

\setlength{\hangindent}{52pt}{常宝童\hspace{20pt}好,那您就再坐一坐!}

\setlength{\hangindent}{52pt}{严嵩\hspace{30pt}谢千岁。}

\setlength{\hangindent}{52pt}{常宝童\hspace{20pt}老太师,我这开山王府欠粮?}

\setlength{\hangindent}{52pt}{严嵩\hspace{30pt}不欠粮。}

\setlength{\hangindent}{52pt}{常宝童\hspace{20pt}缺饷?}

\setlength{\hangindent}{52pt}{严嵩\hspace{30pt}不缺饷。}

\setlength{\hangindent}{52pt}{常宝童\hspace{20pt}一不欠粮,二不缺饷,您来到我开山王府有何贵干呐?}

\setlength{\hangindent}{52pt}{严嵩\hspace{30pt}呃,请千岁上殿辩理。}

\setlength{\hangindent}{52pt}{常宝童\hspace{20pt}拿来。}

\setlength{\hangindent}{52pt}{严嵩\hspace{30pt}什么?}

\setlength{\hangindent}{52pt}{常宝童\hspace{20pt}圣旨啊。}

\setlength{\hangindent}{52pt}{严嵩\hspace{30pt}啊!方才千岁请过了。}

\setlength{\hangindent}{52pt}{常宝童\hspace{20pt}孩子们,你们谁请过圣旨了?}

\setlength{\hangindent}{52pt}{众人\hspace{30pt}没有。}

\setlength{\hangindent}{52pt}{严嵩\hspace{30pt}请过去了。}

\setlength{\hangindent}{52pt}{众人\hspace{30pt}没有。}

\setlength{\hangindent}{52pt}{常宝童\hspace{20pt}唗!胆大严嵩,今日在朝害文,明日在朝害武,害来害去,害在小王的头上来了!你失落圣旨,也是有的,不该前来讹诈本御。今儿不打你几下,惯了你的下次。孩子们,脱袍打严嵩!}

\setlength{\hangindent}{52pt}{严嵩\hspace{30pt}哎呀,小千岁,老臣挨不起啊。}

\setlength{\hangindent}{52pt}{常宝童\hspace{20pt}【{\akai 西皮摇板}】矫传皇命遭戏弄\footnote{此句从陈超老师的建议当作``矫造皇命来戏弄''。},开山王府岂能容。手持金锏将贼打,打死奸贼方称心。}

{(邹应龙{\hwfs 踹}严嵩,严嵩{\hwfs 躺地上})}

\setlength{\hangindent}{52pt}{邹应龙\hspace{20pt}唗!唗!(你们)擅打大臣,该当何罪?!}

\setlength{\hangindent}{52pt}{众\hspace{40pt}有邹老爷在内。}

\setlength{\hangindent}{52pt}{邹应龙\hspace{20pt}哦,什么,有我(在内,呵呵哈哈)?诶,那就无有事了哇。}

\setlength{\hangindent}{52pt}{(常宝童\hspace{20pt}是你的主意。)}

\setlength{\hangindent}{52pt}{邹应龙\hspace{20pt}是我?那就无有事了哇。}

\setlength{\hangindent}{52pt}{常宝童\hspace{20pt}你可真坏啊!({\akai 或}:~好家伙,吓了我一跳。)}

\setlength{\hangindent}{52pt}{邹应龙\hspace{20pt}千岁,({\akai 或}:~怎样臣是一个坏人。坏人也罢,好人也罢,启禀千岁,)打了老贼,他必然({\akai 或}:~必定要)上殿奏本。千岁将老皇的御容,伴驾王的真像请入华车,抬上金殿,为臣邀请满朝文武,三十六名进士({\akai 或}:~再请满朝文武、众位进士)共参老贼。有何难哉?}

\setlength{\hangindent}{52pt}{常宝童\hspace{20pt}本御记下。({\akai 或}:~小王记下。)}

\setlength{\hangindent}{52pt}{邹应龙\hspace{20pt}啊千岁,你打了半日,可有名堂来?({\akai 或}:~列位,你们打了半日,可有名堂来?\footnote{邹应龙问话时不能直接问常宝童。})}

\setlength{\hangindent}{52pt}{常宝童\hspace{20pt}嘿,没什么名堂,乱打一锅粥哦。({\akai 或}:~别说他们,连我都是乱打一锅粥了。)}

\setlength{\hangindent}{52pt}{邹应龙\hspace{20pt}(微)臣(要)赶至御街,要打出({\akai 或}:~打他)一个名堂。}

\setlength{\hangindent}{52pt}{常宝童\hspace{20pt}什么名堂?}

\setlength{\hangindent}{52pt}{邹应龙\hspace{20pt}这就是({\akai 或}:~这叫作):~({\akai 念})文武齐喝彩,应龙闹京街。}

\setlength{\hangindent}{52pt}{常宝童\hspace{20pt}我却不信。}

\setlength{\hangindent}{52pt}{邹应龙\hspace{20pt}【{\akai 西皮摇板}】千岁但把心放稳,管叫老贼({\akai 或}:~打老贼)领人情。有劳皇挡来抬定({\akai 或}:~千岁休得不肯信,少时便知假和真。有劳皇挡来抬定,)}

\setlength{\hangindent}{52pt}{邹应龙\hspace{20pt}有劳了!}

\setlength{\hangindent}{52pt}{邹应龙\hspace{20pt}【{\akai 西皮摇板}】赶至御街打谗臣({\akai 或}:~奸臣)。}

\setlength{\hangindent}{52pt}{常宝童\hspace{20pt}【{\akai 西皮摇板}】御容请在华车上,抬上金殿奏君王({\akai 或}:~抬上金殿奏叔王)。}

\vspace{3pt}{\centerline{{[}{\hei 第四场}{]}}}\vspace{5pt}

\setlength{\hangindent}{52pt}{众\hspace{40pt}咱们找太师罢。}

\setlength{\hangindent}{52pt}{严嵩\hspace{30pt}唗!唗!你们都往哪里去了?}

\setlength{\hangindent}{52pt}{众\hspace{40pt}皇挡挡住,不敢进入。}

\setlength{\hangindent}{52pt}{严嵩\hspace{30pt}哎呀,难怪你们。起来起来,搭轿!}

\setlength{\hangindent}{52pt}{众\hspace{40pt}轿子被他们打坏了。}

\setlength{\hangindent}{52pt}{严嵩\hspace{30pt}带马,带马!}

\setlength{\hangindent}{52pt}{众\hspace{40pt}马被邹老爷骑了去了。}

\setlength{\hangindent}{52pt}{严嵩\hspace{30pt}哎呀,这样罢,你们挑选一个得力之人,背了老夫回去,重重有赏。}

\setlength{\hangindent}{52pt}{众\hspace{40pt}我们商议商议。这么大的大胖子,谁背得动?!}

\setlength{\hangindent}{52pt}{众\hspace{40pt}哎呀!常宝童赶来了。}

\setlength{\hangindent}{52pt}{严嵩\hspace{30pt}哎呀,千岁!老臣挨不起了。}

\setlength{\hangindent}{52pt}{邹应龙\hspace{20pt}参见太师(爷)!}

\setlength{\hangindent}{52pt}{严嵩\hspace{30pt}你是哪个?({\akai 或}:~你是何人?)}

\setlength{\hangindent}{52pt}{邹应龙\hspace{20pt}(小官)邹应龙,在此。}

\setlength{\hangindent}{52pt}{严嵩\hspace{30pt}心腹人,你来了?老夫被他们打坏了哇!}

\setlength{\hangindent}{52pt}{邹应龙\hspace{20pt}太师(爷)为何这等模样?}

\setlength{\hangindent}{52pt}{严嵩\hspace{30pt}哎呀,好打好打!}

\setlength{\hangindent}{52pt}{邹应龙\hspace{20pt}哪个敢打太师?}

\setlength{\hangindent}{52pt}{严嵩\hspace{30pt}哎呀你哪里知道------}

\setlength{\hangindent}{52pt}{邹应龙\hspace{20pt}小官怎能知晓?}

\setlength{\hangindent}{52pt}{严嵩\hspace{30pt}待老夫慢慢地对你言讲{\footnotesize 呃}。({\hwfs 喘介})}

\setlength{\hangindent}{52pt}{邹应龙\hspace{20pt}太师爷慢慢讲。}

\setlength{\hangindent}{52pt}{严嵩\hspace{30pt}老夫领了圣旨到开山王府,去捉拿常宝童上殿辩理。我到了王府,正要开读圣旨,那娃娃言道:~``老太师不必开读,小王知罪。''}

\setlength{\hangindent}{52pt}{邹应龙\hspace{20pt}他知何罪?}

\setlength{\hangindent}{52pt}{严嵩\hspace{30pt}他言道:~``愿将邱、马两将献与当今。''}

\setlength{\hangindent}{52pt}{邹应龙\hspace{20pt}说罢之后?}

\setlength{\hangindent}{52pt}{严嵩\hspace{30pt}他就请过了圣旨。}

\setlength{\hangindent}{52pt}{邹应龙\hspace{20pt}请过之后?}

\setlength{\hangindent}{52pt}{严嵩\hspace{30pt}那娃娃他赐了老夫一个座位。}

\setlength{\hangindent}{52pt}{邹应龙\hspace{20pt}着哇,金銮殿上({\akai 或}:~金殿之上)二十四把金交椅都有({\akai 或}:~尚有)太师爷的座位,何况他小小的开山王府啊。坐下后呢?}

\setlength{\hangindent}{52pt}{严嵩\hspace{30pt}呵!这一坐啊,可就坐出祸来了呃!}

\setlength{\hangindent}{52pt}{邹应龙\hspace{20pt}什么祸事?({\akai 或}:~怎见得?)}

\setlength{\hangindent}{52pt}{严嵩\hspace{30pt}那娃娃言道:~``老太师,你在我朝是个忠臣,还是个奸臣呢?''}

\setlength{\hangindent}{52pt}{邹应龙\hspace{20pt}老太师乃是大大的忠臣。}

\setlength{\hangindent}{52pt}{严嵩\hspace{30pt}着哇。原是忠臣。}

\setlength{\hangindent}{52pt}{严嵩\hspace{30pt}那娃娃言道:~``既是忠臣,孩子们,珠帘卷起,太师爷抬头观看。''}

\setlength{\hangindent}{52pt}{邹应龙\hspace{20pt}看些什么?}

\setlength{\hangindent}{52pt}{严嵩\hspace{30pt}那娃娃也不知听了哪个高明的坏种------({\akai 或}:~领了哪位先生------)}

\setlength{\hangindent}{52pt}{邹应龙\hspace{20pt}诶。}

\setlength{\hangindent}{52pt}{严嵩\hspace{30pt}的高教,将老皇御容、伴驾王真像,悬挂中堂,老夫身为大臣,见君不参,就有一行大罪。}

\setlength{\hangindent}{52pt}{邹应龙\hspace{20pt}太师这便怎处啊?}

\setlength{\hangindent}{52pt}{严嵩\hspace{30pt}我说,老臣有辩。}

\setlength{\hangindent}{52pt}{邹应龙\hspace{20pt}哦,有辩?}

\setlength{\hangindent}{52pt}{严嵩\hspace{30pt}那娃娃言道:~``太师爷还会变呐?''}

\setlength{\hangindent}{52pt}{邹应龙\hspace{20pt}会辩呐。}

\setlength{\hangindent}{52pt}{严嵩\hspace{30pt}那娃娃言道:~``孩子们,金盆打水!''}

\setlength{\hangindent}{52pt}{邹应龙\hspace{20pt}打水则甚{\footnotesize 呐}?~({\akai 或}:~打水何用{\footnotesize 呐}?)}

\setlength{\hangindent}{52pt}{严嵩\hspace{30pt}他说:~``老太师,你变个乌龟,与小王玩耍玩耍?''}

\setlength{\hangindent}{52pt}{邹应龙\hspace{20pt}老太师你变了没有?~({\akai 或}:~太师你如何变法?)}

\setlength{\hangindent}{52pt}{严嵩\hspace{30pt}呃!老夫焉能如此,乃是舌辩之辩。}

\setlength{\hangindent}{52pt}{邹应龙\hspace{20pt}我也是问的舌辩之辩呐。}

\setlength{\hangindent}{52pt}{严嵩\hspace{30pt}着啊。}

\setlength{\hangindent}{52pt}{邹应龙\hspace{20pt}太师怎样辩法?}

\setlength{\hangindent}{52pt}{严嵩\hspace{30pt}是我言道:~``亦非朔望,闲不参君。''}

\setlength{\hangindent}{52pt}{邹应龙\hspace{20pt}辩得好,辩得好------辩倒之后?({\akai 或}:~老太师高才!{\akai 或}:~坐罢之后?)}

\setlength{\hangindent}{52pt}{严嵩\hspace{30pt}那娃娃言道:~``老太师,我开山王府欠粮?''}

\setlength{\hangindent}{52pt}{邹应龙\hspace{20pt}不欠粮。}

\setlength{\hangindent}{52pt}{严嵩\hspace{30pt}``缺饷?''}

\setlength{\hangindent}{52pt}{邹应龙\hspace{20pt}不缺饷。}

\setlength{\hangindent}{52pt}{严嵩\hspace{30pt}``一不欠粮,二不缺饷,到这儿来,干嘛来了?''}

\setlength{\hangindent}{52pt}{邹应龙\hspace{20pt}请千岁上殿辩理呀------窝藏邱、马二将。}

\setlength{\hangindent}{52pt}{严嵩\hspace{30pt}``拿来。''}

\setlength{\hangindent}{52pt}{邹应龙\hspace{20pt}(拿)什么?}

\setlength{\hangindent}{52pt}{严嵩\hspace{30pt}``圣旨啊。''}

\setlength{\hangindent}{52pt}{邹应龙\hspace{20pt}呃,(方才)他请过去了哇。}

\setlength{\hangindent}{52pt}{严嵩\hspace{30pt}哎哟,老夫还不知道他请过去了吗?\footnote{此处严嵩念京白。}这个娃娃他与我来了个不认账啊。}

\setlength{\hangindent}{52pt}{邹应龙\hspace{20pt}那还了得?!}

\setlength{\hangindent}{52pt}{严嵩\hspace{30pt}那娃娃言道:~``唗!胆大严嵩,今日在朝害文,明日在朝害武,害来害去,害到小王的头上。今天不打你几下,惯了你的下次,来啊,吩咐脱袍打严嵩!''就是这样乒乒乓乓,一顿暴打。哎哟,可打坏了,打坏了!闪开,闪开!({\akai 或}:~搀扶了。)}

\setlength{\hangindent}{52pt}{邹应龙\hspace{20pt}哪里去?({\akai 或}:~太师你欲何往?{\akai 或}:~慢来慢来,太师往哪里去?)}

\setlength{\hangindent}{52pt}{严嵩\hspace{30pt}上殿参他一本呐。}

\setlength{\hangindent}{52pt}{邹应龙\hspace{20pt}参他一本?}

\setlength{\hangindent}{52pt}{严嵩\hspace{30pt}参他一本。}

\setlength{\hangindent}{52pt}{邹应龙\hspace{20pt}(万岁若问)有何伤痕?}

\setlength{\hangindent}{52pt}{严嵩\hspace{30pt}浑身是伤。}

\setlength{\hangindent}{52pt}{邹应龙\hspace{20pt}怎样验伤?({\akai 或}:~怎样见君?)}

\setlength{\hangindent}{52pt}{严嵩\hspace{30pt}脱袍验伤。}

\setlength{\hangindent}{52pt}{邹应龙\hspace{20pt}脱袍验伤------嗯------险呐!({\akai 或}:~哎呀太师,险呐!若不是小官在此,你把事可又办错了。)}

\setlength{\hangindent}{52pt}{严嵩\hspace{30pt}怎么?}

\setlength{\hangindent}{52pt}{邹应龙\hspace{20pt}太师爷身为大臣,脱袍见君岂不有欺君之罪!({\akai 或}:~你身为大臣,脱袍见君论律当斩。)}

\setlength{\hangindent}{52pt}{严嵩\hspace{30pt}哎呀是啊,``要交部严加议处''。哎呀呀心腹人,这便如何是好啊!({\akai 或}:~这便怎是呢?)}

\setlength{\hangindent}{52pt}{邹应龙\hspace{20pt}这$\cdots{}\cdots{}$依小官拙见,在这文武两班,寻一心粗胆壮之人,在脸面之上做一伤痕,上殿奏本,参倒那常宝童。({\akai 或}:~这$\cdots{}\cdots{}$依小官拙见,必须在脸面之上做一、两处伤痕,上殿奏本,一本就准。{\akai 或}:~必须要打一面伤,上殿奏本。)}

{(邹应龙{\hwfs 踢起地面一块金砖})\footnote{{\hei 刘曾复先生在两次说《打严嵩》一戏的时候,严嵩拿砖打脚面的情节安排得不完全一样,为吴小如先生介绍的时候此情节在最后。}具体可参阅录音。}}

\setlength{\hangindent}{52pt}{邹应龙\hspace{20pt}这有一块金砖,太师拿在手中,自己打自己,一定成功。}

\setlength{\hangindent}{52pt}{严嵩\hspace{30pt}心腹人,漫说是你与老夫凑趣,这块金砖也与老夫凑趣来了哇。}

\setlength{\hangindent}{52pt}{邹应龙\hspace{20pt}太师爷打呀。}

\setlength{\hangindent}{52pt}{严嵩\hspace{30pt}哎,自己打自己,要直着打,狠着打!}

\setlength{\hangindent}{52pt}{严嵩\hspace{30pt}常宝童,小奴才!老夫打一面伤,上殿奏本,一本就准!}

\setlength{\hangindent}{52pt}{严嵩\hspace{30pt}嘿,一本就准。唉哟哟$\cdots{}\cdots{}$,走走走$\cdots{}\cdots{}$}

{(严嵩{\hwfs 将砖头扔地上})}

\setlength{\hangindent}{52pt}{邹应龙\hspace{20pt}哪里去?}

\setlength{\hangindent}{52pt}{严嵩\hspace{30pt}上殿奏本{\footnotesize 呐}。}

\setlength{\hangindent}{52pt}{邹应龙\hspace{20pt}有何伤痕呐?}

\setlength{\hangindent}{52pt}{严嵩\hspace{30pt}脚面上面了。}

\setlength{\hangindent}{52pt}{邹应龙\hspace{20pt}那怎样见君呢?}

\setlength{\hangindent}{52pt}{严嵩\hspace{30pt}脱靴子。哎呀哎呀,这不中用了啊。}

\setlength{\hangindent}{52pt}{严嵩\hspace{30pt}心腹人,自己打自己,下不了手啊。}

\setlength{\hangindent}{52pt}{邹应龙\hspace{20pt}不如去请人打。({\akai 或}:~就该请人来打。)}

\setlength{\hangindent}{52pt}{严嵩\hspace{30pt}好,哪里去问?({\akai 或}:~哦,叫老夫去请何人来打。)}

\setlength{\hangindent}{52pt}{邹应龙\hspace{20pt}文班中去问。({\akai 或}:~到文班去请。)}

\setlength{\hangindent}{52pt}{严嵩\hspace{30pt}好好好,文班中去问。}

{(严嵩{\hwfs 朝下场门})}

\setlength{\hangindent}{52pt}{严嵩\hspace{30pt}列位大人,这有金砖一块,哪一个在老夫脸上做一面伤,上殿参倒常宝童,老夫重礼相谢!}

\setlength{\hangindent}{52pt}{众\hspace{40pt}我们不敢。}

\setlength{\hangindent}{52pt}{严嵩\hspace{30pt}哎呀,他们都走了!}

\setlength{\hangindent}{52pt}{邹应龙\hspace{20pt}({\hwfs 佯惊介})武班中去请。}

\setlength{\hangindent}{52pt}{严嵩\hspace{30pt}呃呃,武班中去问。({\akai 或}:~武班去请。)}

\setlength{\hangindent}{52pt}{邹应龙\hspace{20pt}他们有胆量。}

\setlength{\hangindent}{52pt}{严嵩\hspace{30pt}他们有胆量。}

\setlength{\hangindent}{52pt}{邹应龙\hspace{20pt}有力气。}

\setlength{\hangindent}{52pt}{严嵩\hspace{30pt}有力气。)}

\setlength{\hangindent}{52pt}{邹应龙\hspace{20pt}看得清,打得准呃。}

\setlength{\hangindent}{52pt}{严嵩\hspace{30pt}看得清,打得准。}

{(严嵩{\hwfs 朝上场门})}

\setlength{\hangindent}{52pt}{严嵩\hspace{30pt}列位大人,请了。这有金砖一块,哪一位将军在老夫脸上做一面伤,上殿参奏常宝童,老夫重礼相谢!}

\setlength{\hangindent}{52pt}{众\hspace{40pt}哎呀,我们不敢,不敢呐。}

\setlength{\hangindent}{52pt}{严嵩\hspace{30pt}哎呀$\cdots{}\cdots{}$他们都散了({\akai 或}:~溜了)。}

\setlength{\hangindent}{52pt}{邹应龙\hspace{20pt}唉!教我好恨呐!}

\setlength{\hangindent}{52pt}{严嵩\hspace{30pt}难道恨着老夫不成?}

\setlength{\hangindent}{52pt}{邹应龙\hspace{20pt}小官焉敢恨着老太师,我恨只恨这两班文武,有哪个不是老太师的保举,今日用着他们,一个个袖手旁观,怎不令人好恨呐!({\akai 或}:~唉,常宝童啊常宝童,幸喜你是打了老太师,若是打了我邹应龙,我一定打一面伤,上殿奏本,嗯,一本就准!{\akai 或}:~不是哟,想这文武官员,哪个不是太师的提拔,今日太师有事,一个个袖手旁观,怎不令人好恨呐!)}

\setlength{\hangindent}{52pt}{严嵩\hspace{30pt}嘿嘿!原来打老夫的人在这里({\akai 或}:~眼前)呢!}

\setlength{\hangindent}{52pt}{严嵩\hspace{30pt}心腹人,请上受老夫一礼。}

\setlength{\hangindent}{52pt}{邹应龙\hspace{20pt}太师,这是何意?({\akai 或}:~哎呀呀,折煞小官!{\akai 或}:~太师此礼为何?)}

\setlength{\hangindent}{52pt}{严嵩\hspace{30pt}心腹人,你看满朝文武他溜的溜了,跑的跑了,只有你这心腹人在此,有劳你的贵手,与老夫打一面伤,上殿参奏常宝童。老夫是重礼相谢。}

\setlength{\hangindent}{52pt}{邹应龙\hspace{20pt}哎呀呀,(启禀太师:~)小官多蒙老太师升官之恩尚未报达,焉敢下此毒手,诶,使不得,诶,使不得。({\akai 或}:~小官不敢,小官不敢。{\akai 或}:~小官多蒙升官之恩尚未报得,怎么还敢打老太师,不敢呐,不敢呐。)}

\setlength{\hangindent}{52pt}{严嵩\hspace{30pt}诶,只要在老夫面上做一伤痕,上殿参倒常宝童,比报那升官之恩胜强十倍,犹如报恩一样。}

\setlength{\hangindent}{52pt}{邹应龙\hspace{20pt}报恩一样?敢是教小官报恩么?({\akai 或}:~怎么。犹如报了升官之恩一般?)。}

\setlength{\hangindent}{52pt}{严嵩\hspace{30pt}呃,是教你报恩,教你报恩。}

\setlength{\hangindent}{52pt}{邹应龙\hspace{20pt}如此小官报恩了。({\akai 或}:~如此小官(当得)效劳。)}

\setlength{\hangindent}{52pt}{严嵩\hspace{30pt}来来来,报恩呐。({\akai 或}:~你要用力打)}

\setlength{\hangindent}{52pt}{邹应龙\hspace{20pt}小官(这里)报恩了。}

\setlength{\hangindent}{52pt}{邹应龙\hspace{20pt}【{\akai 西皮导板}】手指({\akai 或}:~指着)严嵩贼奸佞,}

\setlength{\hangindent}{52pt}{严嵩\hspace{30pt}唗,唗!老夫教你打,胆大邹应龙,你怎么骂起老夫来了?!嘿,真真的岂有此理呀!}

\setlength{\hangindent}{52pt}{邹应龙\hspace{20pt}唉呀呀,老太师,你把小官可错怪了哇。}

\setlength{\hangindent}{52pt}{严嵩\hspace{30pt}怎么错怪了你呀?}

\setlength{\hangindent}{52pt}{邹应龙\hspace{20pt}常言说得好:~举手难打笑脸人,小官与老太师夙无怨恨({\akai 或}:~小官与老太师远日无仇,近日无冤),况且升官之恩尚未报答,焉能下得手打太师?没奈何我指东骂西,指桑骂槐。指的是老太师,骂的是常宝童,骂起气来才好用力打呀。小官将将骂了一句,老太师就动起怒来。若是打着太师,那还了得?老太师,呃,小官得罪了,得罪了老太师,小官这厢赔礼了,小官这里请罪了。太师爷你另请高明罢。({\akai 或}:~常言说得好:~举拳不打笑脸人,想小官与太师爷远日无仇,近日无冤,况且又有升官之恩尚未报得,焉能下得手来打太师?没奈何只得是指东骂西,指桑骂槐。指的是老太师,骂的是常宝童,骂起气来才好使力气来打。小官刚刚的骂了一句,太师爷就降下罪来。小官就不敢,不敢。你另请高明罢。)}

\setlength{\hangindent}{52pt}{严嵩\hspace{30pt}回来回来,心腹人,老夫我明白了。你是指东骂西、指桑骂槐。指的是老夫,骂的是常宝童。}

\setlength{\hangindent}{52pt}{邹应龙\hspace{20pt}是啊。({\akai 或}:~如此说来,太师爷你不恼?)}

\setlength{\hangindent}{52pt}{(邹应龙\hspace{20pt}你要我骂?)}

\setlength{\hangindent}{52pt}{(邹应龙\hspace{20pt}骂你这个奸贼?)}

\setlength{\hangindent}{52pt}{严嵩\hspace{30pt}骂上气来才好下手打,如此说来,心腹人!~你就连打带骂\footnote{此处严嵩念京白。}。}

\setlength{\hangindent}{52pt}{邹应龙\hspace{20pt}如此严嵩!}

\setlength{\hangindent}{52pt}{严嵩\hspace{30pt}嗳!}

\setlength{\hangindent}{52pt}{邹应龙\hspace{20pt}我把你这(误国的)老贼!}

\setlength{\hangindent}{52pt}{严嵩\hspace{30pt}骂得好啊!}

\setlength{\hangindent}{52pt}{邹应龙\hspace{20pt}【{\akai 西皮快板}】骂一声老贼听分明:~你在当朝官一品,上欺天子下压臣。满朝文武来观定({\akai 或}:~齐来看),我与忠良把冤申。}

\setlength{\hangindent}{52pt}{严嵩\hspace{30pt}哎呀!邹应龙打坏了!}

\setlength{\hangindent}{52pt}{邹应龙\hspace{20pt}呃------常宝童打的。}

\setlength{\hangindent}{52pt}{严嵩\hspace{30pt}呃是是是,常宝童打坏人了。走走走,上殿参本({\akai 或}:~参本)。}

\setlength{\hangindent}{52pt}{邹应龙\hspace{20pt}慢来慢来,看看伤痕呐。({\akai 或}:~待我来看看伤。)}

\setlength{\hangindent}{52pt}{严嵩\hspace{30pt}对对对对,验验伤,验验伤。}

\setlength{\hangindent}{52pt}{邹应龙\hspace{20pt}哎呀太师爷,还是不中用啊。({\akai 或}:~还是不成功。)}

\setlength{\hangindent}{52pt}{严嵩\hspace{30pt}怎见得?({\akai 或}:~怎么不成功?)}

\setlength{\hangindent}{52pt}{邹应龙\hspace{20pt}只有一伤呐。}

\setlength{\hangindent}{52pt}{严嵩\hspace{30pt}一块可也就够了。}

\setlength{\hangindent}{52pt}{邹应龙\hspace{20pt}诶,一伤乃是误伤,要两伤才算是打伤呢!还要做上一块。}

\setlength{\hangindent}{52pt}{严嵩\hspace{30pt}哎呀,老夫我挨受不起了哇。}

\setlength{\hangindent}{52pt}{邹应龙\hspace{20pt}呃,倒有个方法({\akai 或}:~太师,小官倒有个办法)。}

\setlength{\hangindent}{52pt}{严嵩\hspace{30pt}什么方法,快快说来。}

\setlength{\hangindent}{52pt}{邹应龙\hspace{20pt}来来来------这有金砖一块,({\akai 或}:~这有砖头一块)}

\setlength{\hangindent}{52pt}{严嵩\hspace{30pt}哦,一块砖头呃。}

\setlength{\hangindent}{52pt}{邹应龙\hspace{20pt}(太师)拿在手中。}

\setlength{\hangindent}{52pt}{严嵩\hspace{30pt}呃,拿在手中,}

\setlength{\hangindent}{52pt}{邹应龙\hspace{20pt}将袍角({\akai 或}:~袍襟)衔在口内,}

\setlength{\hangindent}{52pt}{严嵩\hspace{30pt}衔在口内。}

\setlength{\hangindent}{52pt}{邹应龙\hspace{20pt}自己打自己,}

\setlength{\hangindent}{52pt}{严嵩\hspace{30pt}自己打自己。}

\setlength{\hangindent}{52pt}{邹应龙\hspace{20pt}打一下,哼一声,({\akai 或}:~小官用力打,太师用力哼。)}

\setlength{\hangindent}{52pt}{严嵩\hspace{30pt}打一下,哼一声。}

\setlength{\hangindent}{52pt}{邹应龙\hspace{20pt}这还有个名堂。}

\setlength{\hangindent}{52pt}{严嵩\hspace{30pt}嘿,还有个名堂?}

\setlength{\hangindent}{52pt}{邹应龙\hspace{20pt}这叫作恨病吃药。({\akai 或}:~恨病------;~欠债------)}

\setlength{\hangindent}{52pt}{严嵩\hspace{30pt}欠债还钱。({\akai 或}:~吃药;~换钱。)}

\setlength{\hangindent}{52pt}{邹应龙\hspace{20pt}着啊,试上一试。}

\setlength{\hangindent}{52pt}{严嵩\hspace{30pt}诶,砖头拿在手中,袍角衔在口内,自己打自己,打一下,哼一声,恨病吃药。}

\setlength{\hangindent}{52pt}{严嵩\hspace{30pt}砖头啊砖头,你不是砖头------}

\setlength{\hangindent}{52pt}{邹应龙\hspace{20pt}是什么?}

\setlength{\hangindent}{52pt}{严嵩\hspace{30pt}你是老夫的对头。}

\setlength{\hangindent}{52pt}{邹应龙\hspace{20pt}呃,打呀!}

\setlength{\hangindent}{52pt}{严嵩\hspace{30pt}往哪里打?}

\setlength{\hangindent}{52pt}{邹应龙\hspace{20pt}面上打。}

\setlength{\hangindent}{52pt}{严嵩\hspace{30pt}面上啊------噢,哎呦嚯,打着了打着了!}

\setlength{\hangindent}{52pt}{邹应龙\hspace{20pt}在哪里?({\akai 或}:~怎么样了?)}

\setlength{\hangindent}{52pt}{严嵩\hspace{30pt}诶,在脚面之上。}

\setlength{\hangindent}{52pt}{邹应龙\hspace{20pt}唉呀,还是不中用呃。}

\setlength{\hangindent}{52pt}{严嵩\hspace{30pt}嘿,白挨了一下打呀!}

\setlength{\hangindent}{52pt}{严嵩\hspace{30pt}得了,我啊,脱靴子见君罢。}

\setlength{\hangindent}{52pt}{邹应龙\hspace{20pt}那还了得。}

\setlength{\hangindent}{52pt}{严嵩\hspace{30pt}真真打不下手啊,唉!一客不烦二主啊,还是心腹人代劳罢。}

\setlength{\hangindent}{52pt}{邹应龙\hspace{20pt}还要我报恩?({\akai 或}:~怎么,还要心腹人代劳?)}

\setlength{\hangindent}{52pt}{严嵩\hspace{30pt}嗯,一定我要你报恩呐。}

\setlength{\hangindent}{52pt}{邹应龙\hspace{20pt}(如此说来,)太师爷,你要忍呐!}

\setlength{\hangindent}{52pt}{严嵩\hspace{30pt}心腹人,你要狠呐!}

\setlength{\hangindent}{52pt}{邹应龙\hspace{20pt}打青了脸------({\akai 或}:~打伤了脸------)}

\setlength{\hangindent}{52pt}{严嵩\hspace{30pt}好奏本。}

\setlength{\hangindent}{52pt}{邹应龙\hspace{20pt}严嵩!}

\setlength{\hangindent}{52pt}{严嵩\hspace{30pt}嗳!}

\setlength{\hangindent}{52pt}{邹应龙\hspace{20pt}(我把你这卖国的)奸贼!({\akai 或}:~老贼------)}

\setlength{\hangindent}{52pt}{严嵩\hspace{30pt}骂得好啊!({\akai 或}:~嗯------)}

\setlength{\hangindent}{52pt}{邹应龙\hspace{20pt}【{\akai 西皮快板}】听罢言来喜气生,不由得应龙抖精神。开言大骂贼奸佞({\akai 或}:~开言便把老贼问),苦害忠良为何情。欺君误国心太狠,应龙心中恨难平。({\akai 或}:~欺君误国实可恨,管教老贼命归阴。{\akai 或}:~欺君误国心太狠,管教老贼两眼青。)}

\setlength{\hangindent}{52pt}{(严嵩\hspace{30pt}哎哟,哎哟!)}

\setlength{\hangindent}{52pt}{严嵩\hspace{30pt}邹应龙,你打坏了!}

\setlength{\hangindent}{52pt}{邹应龙\hspace{20pt}常宝童打伤人了。({\akai 或}:~呃,常宝童打的呢。)}

\setlength{\hangindent}{52pt}{严嵩\hspace{30pt}常宝童打伤人了。}

\setlength{\hangindent}{52pt}{严嵩\hspace{30pt}邹应龙,你来看看伤痕。}

\setlength{\hangindent}{52pt}{邹应龙\hspace{20pt}还要修补修补。({\akai 或}:~是浮伤啊,还轻啊。) ({\akai 或}:~待小官来看看------打得倒还好啊。当中还有一条缝儿。两伤中间,有一条缝------还要找补找补!)}

\setlength{\hangindent}{52pt}{严嵩\hspace{30pt}诶,眼睛都看不见了。}

\setlength{\hangindent}{52pt}{严嵩\hspace{30pt}唉呀!将就了罢,将就了罢!搀扶了------}

\setlength{\hangindent}{52pt}{邹应龙\hspace{20pt}遵命!}

\setlength{\hangindent}{52pt}{严嵩\hspace{30pt}唉,慢来慢来,心腹人,老夫我倒想起一桩心事来了。}

\setlength{\hangindent}{52pt}{邹应龙\hspace{20pt}什么大事!({\akai 或}:~什么心事?)}

\setlength{\hangindent}{52pt}{严嵩\hspace{30pt}老夫在开山王府挨打的时节,屏风后面有那么一位穿红袍的官儿闪将出来,呃,他还踢了老夫一靴尖,也不知他是何人,弄什么诡计!~({\akai 或}:~常宝童这个娃娃,不知领了何人的高见,将老夫弄得这副模样。)}

\setlength{\hangindent}{52pt}{邹应龙\hspace{20pt}这有何难?~(老)太师在朝内({\akai 或}:~京内)访,小官朝外({\akai 或}:~京外)访,访着此人,便知分晓({\akai 或}:~定不与他干休)。}

\setlength{\hangindent}{52pt}{严嵩\hspace{30pt}嗯,我定要灭他的九族!}

\setlength{\hangindent}{52pt}{邹应龙\hspace{20pt}要灭贼的满门!}

\setlength{\hangindent}{52pt}{严嵩\hspace{30pt}搀扶了!}

\setlength{\hangindent}{52pt}{(邹应龙\hspace{20pt}来了!)}

\setlength{\hangindent}{52pt}{邹应龙\hspace{20pt}哎呀,常宝童来了!}

\setlength{\hangindent}{52pt}{严嵩\hspace{30pt}嘿呀!}

\setlength{\hangindent}{52pt}{严嵩\hspace{30pt}心腹人!~劳您驾。}

\setlength{\hangindent}{52pt}{邹应龙\hspace{20pt}不成敬意!}

}
