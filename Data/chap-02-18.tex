\newpage
\subsubsection{\large\hei (安居)平五路~\protect\footnote{根据刘曾复先生钞本整理,刘曾复先生说戏``平五路''是本剧的``观鱼遣邓''部分。刘曾复先生的钞本与《清车王府藏曲本(全印本)》\upcite{CWFCQB}第二册所收录的``安五路(总讲)''基本一致,但``安五路(总讲)''没有最后的``邓芝扑油鼎''部分。}}
\addcontentsline{toc}{subsection}{\hei 平五路}

\hangafter=1                   %2. 设置从第1⾏之后开始悬挂缩进  %
\setlength{\parindent}{0pt}{
\vspace{3pt}{\centerline{{[}{\hei 第一场}{]}}}\vspace{5pt}

\setlength{\hangindent}{56pt}{(打朝,末扮贾诩、外扮辛毗、副扮曹真、净扮司马懿,上)}

\setlength{\hangindent}{56pt}{贾诩\hspace{30pt}({\akai 念})自古良禽择木栖,}

\setlength{\hangindent}{56pt}{辛毗\hspace{30pt}({\akai 念})而今喜得拜丹墀。}

\setlength{\hangindent}{56pt}{曹真\hspace{30pt}({\akai 念})男儿须当封侯印,}

\setlength{\hangindent}{56pt}{司马懿\hspace{30pt}({\akai 念})正是英雄得志时。}

\setlength{\hangindent}{56pt}{贾诩\hspace{30pt}请了。}

\setlength{\hangindent}{56pt}{辛毗、曹真、司马懿\hspace{20pt}请了。}

\setlength{\hangindent}{56pt}{贾诩\hspace{30pt}今日万岁升殿,必有军情议论。}

\setlength{\hangindent}{56pt}{辛毗、曹真、司马懿\hspace{20pt}大家分班伺候。}

\setlength{\hangindent}{56pt}{贾诩、辛毗、曹真、司马懿\hspace{10pt}请。}

\setlength{\hangindent}{56pt}{(四太监、一大太监,\textless{}\!{\bfseries\akai 小开门}\!\textgreater{},小生扮曹丕\footnote{刘曾复先生钞本注``小生戴髯'',即曹丕归小生行应工。}上)}

\setlength{\hangindent}{56pt}{曹丕\hspace{30pt}{[}{\akai 引子}{]}驾坐朝阁受三分,重整山河。}

\setlength{\hangindent}{56pt}{曹丕\hspace{30pt}({\akai 念})献帝无福民不安,人心归朕乐尧天。上苍若肯遂孤愿,扫平东吴灭西川。}

\setlength{\hangindent}{56pt}{曹丕\hspace{30pt}寡人曹丕,国号黄初在位。蒙众卿忠勇,扶孤禅位,更改国号。深感上天之福佑也。朕闻刘备兵伐东吴,中了陆逊火攻之计,败入白帝城,气忿身亡。朕闻此信心无忧矣。孤有心攻取西川,我想必获全胜。众贤卿。}

\setlength{\hangindent}{56pt}{贾诩、辛毗、曹真、司马懿\hspace{10pt}万岁。}

\setlength{\hangindent}{56pt}{曹丕\hspace{30pt}朕想刘备新亡,乘他国内无主,人心未定,攻取西川,蜀可得矣。卿等意下如何?}

\setlength{\hangindent}{56pt}{贾诩\hspace{30pt}臣贾诩奏闻陛下。}

\setlength{\hangindent}{56pt}{曹丕\hspace{30pt}当面奏来。}

\setlength{\hangindent}{56pt}{贾诩\hspace{30pt}臣想刘备虽亡,必托孤与诸葛亮,那孔明感刘备知遇之恩,必要倾心竭力扶持嗣主,陛下不可轻伐。}

\setlength{\hangindent}{56pt}{司马懿\hspace{30pt}臣司马懿有本启奏。}

\setlength{\hangindent}{56pt}{曹丕\hspace{30pt}卿有何良谋,奏与朕知。}

\setlength{\hangindent}{56pt}{司马懿\hspace{30pt}西蜀新败,休容他养成锐气,若不乘此发兵,待等何时?}

\setlength{\hangindent}{56pt}{曹丕\hspace{30pt}卿言正合孤意,当用何计?}

\setlength{\hangindent}{56pt}{司马懿\hspace{30pt}若用中原之兵,恐难取胜。须用五路大兵,四面攻打。那诸葛亮首尾不能相顾,西川之地,必然唾手可得。}

\setlength{\hangindent}{56pt}{曹丕\hspace{30pt}哪五路呢?}

\setlength{\hangindent}{56pt}{司马懿\hspace{30pt}可修国书遣使臣去到鲜卑国见那国王轲比能,贿以金帛,令他起羌兵十万\footnote{刘曾复先生钞本作``令起起羌兵十万'',似欠通;此处从``安五路(总讲)''。}攻打西平关,此一路也;}

\setlength{\hangindent}{56pt}{曹丕\hspace{30pt}二路呢?}

\setlength{\hangindent}{56pt}{司马懿\hspace{30pt}差人直入蛮洞买通蛮王孟获,领起蛮兵十万攻打益州、永昌四郡,此二路也;}

\setlength{\hangindent}{56pt}{曹丕\hspace{30pt}三路呢?}

\setlength{\hangindent}{56pt}{司马懿\hspace{30pt}再遣能言使臣入吴和好\footnote{刘曾复先生钞本作``合好''。},许以割地为约,领起吴兵十万入峡口取涪城,此三路也;}

\setlength{\hangindent}{56pt}{曹丕\hspace{30pt}四路呢?}

\setlength{\hangindent}{56pt}{司马懿\hspace{30pt}急调孟达起上庸兵十万攻打汉中,此四路也;}

\setlength{\hangindent}{56pt}{曹丕\hspace{30pt}那五路呢?}

\setlength{\hangindent}{56pt}{司马懿\hspace{30pt}就命大将军曹真起中原大兵十万攻打阳平关,此五路也。五路大兵共五十万,併力攻取西川,那孔明纵有吕望之才,难逃五路雄兵也。}

\setlength{\hangindent}{56pt}{曹丕\hspace{30pt}此本奏之有理,孤王依计而行。曹真进位\footnote{进位是进升爵位,封号的意思。}。}

\setlength{\hangindent}{56pt}{曹真\hspace{30pt}万岁。}

\setlength{\hangindent}{56pt}{曹丕\hspace{30pt}卿领大兵十万攻取阳平关,得胜回朝,另加升赏。}

\setlength{\hangindent}{56pt}{曹真\hspace{30pt}领旨。}

\setlength{\hangindent}{56pt}{曹真\hspace{30pt}({\akai 念})金殿领君命,校场排雄兵。}

\setlength{\hangindent}{56pt}{(曹真下)}

\setlength{\hangindent}{56pt}{曹丕\hspace{30pt}退班\footnote{刘曾复先生钞本作``还班'',此处从``安五路(总讲)''原文。}。}

\setlength{\hangindent}{56pt}{(众分下)}

\vspace{3pt}{\centerline{{[}{\hei 第二场}{]}}}\vspace{5pt}

\setlength{\hangindent}{56pt}{(二小军抬杠箱上,丑扮差官跳上)}

\setlength{\hangindent}{56pt}{差官\hspace{30pt}吾乃北魏国王驾下差官是也,今奉我主之命押解礼物去到鲜卑国,聘请国王轲比能,今起羌兵十万攻打西平关。身奉君命,不敢怠慢。军士们。}

\setlength{\hangindent}{56pt}{二丑小卒\hspace{30pt}有话说罢。}

\setlength{\hangindent}{56pt}{差官\hspace{30pt}快趱行。}

\setlength{\hangindent}{56pt}{(差官倒退跳走{干\textless{}度柳翠\textgreater{}}\footnote{刘曾复先生钞本与``安五路(总讲)''均作``差官倒退跳赶\textless{}\!{\bfseries\akai 度柳翠}\!\textgreater{}'',经何毅老师指教,\textless{}\!{\bfseries\akai 度柳翠}\!\textgreater{}是牌子名,``干''表示是``干牌子''。},\textless{}\!{\bfseries\akai 挑子}\!\textgreater{}下)}

\vspace{3pt}{\centerline{{[}{\hei 第三场}{]}}}\vspace{5pt}

\setlength{\hangindent}{56pt}{(报子上)}}

\setlength{\hangindent}{56pt}{报子\hspace{30pt}马来。}

\setlength{\hangindent}{56pt}{报子\hspace{30pt}({\akai 念})胆量天生就,应变广机谋。探访邻邦事,名称夜不收。}

\setlength{\hangindent}{56pt}{报子\hspace{30pt}吾乃西蜀远探是也,探得曹丕发兵五十万,五路进兵攻取西川。探得真实,不免连夜飞报丞相知道便了。}

\setlength{\hangindent}{56pt}{(报子下)}

\vspace{3pt}{\centerline{{[}{\hei 第四场}{]}}}\vspace{5pt}

\setlength{\hangindent}{56pt}{(生扮诸葛亮上)}

\setlength{\hangindent}{56pt}{诸葛亮\hspace{30pt}【{\akai 西皮{原}板}】天命归人心归天时地利,一朝君一朝臣争夺华夷。西川地到而今虽归我主,普天下皆王土汉室地基。}

\setlength{\hangindent}{56pt}{诸葛亮\hspace{30pt}({\akai 念})铺谋设计正朝纲,国事纷纷费心肠。}

\setlength{\hangindent}{56pt}{诸葛亮\hspace{30pt}山人诸葛亮,字孔明,道号卧龙。只因\footnote{刘曾复先生钞本作``自因''。}先皇伐吴失利败入白帝城,气忿成疾,晏了圣驾。蒙托孤之重,扶保幼主登了龙位,安稳民心。可恨曹丕篡了汉位,更改国号。本当发兵问罪,怎奈我兵新败,不敢轻举妄动,待等兵精粮足再去发兵问罪。正是:~({\akai 念})只为托孤恩义重,披肝沥胆报国恩。}

\setlength{\hangindent}{56pt}{(报子上)}

\setlength{\hangindent}{56pt}{报子\hspace{30pt}({\akai 念})探听北魏军情事,报与西蜀丞相知。}

\setlength{\hangindent}{56pt}{报子\hspace{30pt}来此已是府门,里面哪位在?}

\setlength{\hangindent}{56pt}{(副扮听事官上)}

\setlength{\hangindent}{56pt}{听事官\hspace{30pt}({\akai 念})侯门深似海,不许外人来。}

\setlength{\hangindent}{56pt}{听事官\hspace{30pt}什么人?}

\setlength{\hangindent}{56pt}{探子\footnote{刘曾复先生钞本以下``报子''均作``探子'',应可通。} 探子要见丞相。}

\setlength{\hangindent}{56pt}{听事官\hspace{30pt}候着。探子求见丞相。}

\setlength{\hangindent}{56pt}{诸葛亮\hspace{30pt}传。}

\setlength{\hangindent}{56pt}{听事官\hspace{30pt}探子,丞相传,小心了。}

\setlength{\hangindent}{56pt}{探子\hspace{30pt}啊。探子叩头。}

\setlength{\hangindent}{56pt}{诸葛亮\hspace{30pt}探子,你探听哪路军情,一一讲来。}

\setlength{\hangindent}{56pt}{探子\hspace{30pt}相爷容禀:~\textless{}\!{\bfseries\akai 五字赞}\!\textgreater{}探子禀军情,相爷在上听:~曹丕人马踴,五路起雄兵:~中原兵十万,曹真攻阳平;上庸发人马,孟达取汉中;孙权入峡口,大兵攻涪城;西平羌兵众,国王轲比能;南蛮名孟获,四郡恶交锋。五路貔貅猛,十万虎狼兵。声如地裂山摇动,要把西川一扫平。}

\setlength{\hangindent}{56pt}{诸葛亮\hspace{30pt}赏你银牌一面,休叫成都\footnote{刘曾复先生钞本和``安五路(总讲)''所有``成都''均作``城都''。}军民知觉,不可走漏我的消息。}

\setlength{\hangindent}{56pt}{探子\hspace{30pt}谢相爷。}

\setlength{\hangindent}{56pt}{(探子下)}

\setlength{\hangindent}{56pt}{诸葛亮\hspace{30pt}这厮好生\footnote{刘曾复先生钞本``好生''二字不确认,疑作``如此''或``着实'',此处从``安五路(总讲)''。}可恶,,明知我国老王驾崩,新君年幼,乘我国丧,兵发五路来克我国,如此猖狂,我自有道理。听事官。}

\setlength{\hangindent}{56pt}{听事官\hspace{30pt}在。}

\setlength{\hangindent}{56pt}{诸葛亮\hspace{30pt}传四路旗牌进府听令。}

\setlength{\hangindent}{56pt}{听事官\hspace{30pt}是。相爷有令:~传四路旗牌进府听令。}

\setlength{\hangindent}{56pt}{(外扮北路旗牌,副扮西路旗牌,末扮南路旗牌,净扮东路旗牌;四旗牌内应,上)}

\setlength{\hangindent}{56pt}{四旗牌\hspace{30pt}({\akai 念})丞相来呼唤,忙步到府堂。}

\setlength{\hangindent}{56pt}{四旗牌\hspace{30pt}四路旗牌参见丞相。}

\setlength{\hangindent}{56pt}{诸葛亮\hspace{30pt}你等免参,听我分派。}

\setlength{\hangindent}{56pt}{四旗牌\hspace{30pt}愿听丞相军令。}

\setlength{\hangindent}{56pt}{诸葛亮\hspace{30pt}今有曹丕兵发五路攻取西川,要你等飞递边关,不叫成都军民知晓,莫要泄漏我的机关。}

\setlength{\hangindent}{56pt}{四旗牌\hspace{30pt}谨遵丞相军令。}

\setlength{\hangindent}{56pt}{诸葛亮\hspace{30pt}北路旗牌听令。}

\setlength{\hangindent}{56pt}{北路旗牌\hspace{30pt}在。}

\setlength{\hangindent}{56pt}{诸葛亮\hspace{30pt}命你赶到阳平关通知赵云,叫他暗设人马,不可交锋,待贼粮尽自退,督兵追杀,不得违误。}

\setlength{\hangindent}{56pt}{北路旗牌\hspace{30pt}得令。}

\setlength{\hangindent}{56pt}{(北路旗牌下)}

\setlength{\hangindent}{56pt}{诸葛亮\hspace{30pt}西路旗牌听令。}

\setlength{\hangindent}{56pt}{西路旗牌\hspace{30pt}在。}

\setlength{\hangindent}{56pt}{诸葛亮\hspace{30pt}命你赶到西平关急报马超,叫他虚立自己旗号,羌兵必不敢战,容他自退,不得违令。}

\setlength{\hangindent}{56pt}{西路旗牌\hspace{30pt}得令。}

\setlength{\hangindent}{56pt}{(西路旗牌下)}

\setlength{\hangindent}{56pt}{诸葛亮\hspace{30pt}南路旗牌听令。}

\setlength{\hangindent}{56pt}{南路旗牌\hspace{30pt}在。}

\setlength{\hangindent}{56pt}{诸葛亮\hspace{30pt}我有令箭一支,内有柬贴封好,赶到南郡交付魏延,叫他依计而行,不可错误。}

\setlength{\hangindent}{56pt}{南路旗牌\hspace{30pt}得令。}

\setlength{\hangindent}{56pt}{(南路旗牌下)}

\setlength{\hangindent}{56pt}{诸葛亮\hspace{30pt}东路旗牌听令。}

\setlength{\hangindent}{56pt}{东路旗牌\hspace{30pt}在。}

\setlength{\hangindent}{56pt}{诸葛亮\hspace{30pt}命你前去急调关兴、张苞各带汉中人马三万,四面接应,不得违误。}

\setlength{\hangindent}{56pt}{西路旗牌\hspace{30pt}得令。}

\setlength{\hangindent}{56pt}{(西路旗牌下)}

\setlength{\hangindent}{56pt}{诸葛亮\hspace{30pt}那上庸兵乃孟达督帅,不用劳动军卒,管叫他不战自退。我料东吴孙权必要兵扎三江口,虚作人情,静观两家胜败,就中攻取,事虽如此。怎奈我先皇昭烈兵伐东吴,结下仇怨,并未和解。我若兴兵伐魏,吴必攻取西蜀;昼夜思想,不得其人入吴和好\footnote{刘曾复先生钞本作``合好''。}。若得吴、蜀和好,结为唇齿,然后兴兵伐魏,也免我忧虑东吴之患也。}

\setlength{\hangindent}{56pt}{诸葛亮\hspace{30pt}【{\akai 西皮{原}板}】平生恨篡国贼欺君万恶,心想要灭贼子枉自揣摩。我本该去问罪天不容我,一桩桩一件件国事阻隔。到如今曹丕贼心威赫赫,乘国丧兵五路侵佔我国。西蜀中现有我区区诸葛,岂肯容贼猖獗奏唱凯歌。参想想和东吴长久计策,缺少个能言士前去说说。叹先皇心愿事敕命于我,灭国贼尽人力天意如何。居相位守臣节日日思索,我怎能负先皇临危重托。}

\setlength{\hangindent}{56pt}{(诸葛亮下)}

\vspace{3pt}{\centerline{{[}{\hei 第五场\footnote{刘曾复先生钞本注``带`扑油鼎'可考虑不要此场。''}}{]}}\vspace{5pt}

\setlength{\hangindent}{56pt}{(四旗牌同上)

\setlength{\hangindent}{56pt}{北路旗牌\hspace{30pt}列位请了。}

\setlength{\hangindent}{56pt}{三路旗牌\hspace{30pt}请了。}

\setlength{\hangindent}{56pt}{北路旗牌\hspace{30pt}今有曹丕兵发五路,攻取西川,你我奉了丞相军令,通知各路依令而行。军情紧急,分路投递。正是:~({\akai 念})将军不下马,}

\setlength{\hangindent}{56pt}{三路旗牌\hspace{30pt}({\akai 念})各自奔前程。}

\vspace{3pt}{\centerline{{[}{\hei 第六场}{]}}}\vspace{5pt}

\setlength{\hangindent}{56pt}{(四文堂、一中军站门,孟达上)}

\setlength{\hangindent}{56pt}{孟达\hspace{30pt}{[}{\akai {\akai 引}子}{]}只为一着错,满盘棋势空。}

\setlength{\hangindent}{56pt}{孟达\hspace{30pt}({\akai 念})昔侍蜀君今侍魏({\akai 或}:~昔仕蜀君今仕魏),俱是三呼称万岁。叹想原郡故乡土,谁到坟前化纸灰。}

\setlength{\hangindent}{56pt}{孟达\hspace{30pt}俺,孟达,昔在汉中称臣,为事不平弃蜀投魏,命俺镇守上庸等处。日前圣旨到来,命俺起兵十万攻取汉中。我想永安宫乃李严镇守,我若攻打,有碍生死之交;如不攻打,又恐魏王识破疑我,好不两难也。}

\setlength{\hangindent}{56pt}{(副扮下书人上)}

\setlength{\hangindent}{56pt}{下书人\hspace{30pt}({\akai 念})四季关银饷,一年走慌忙。}

\setlength{\hangindent}{56pt}{下书人\hspace{30pt}来此已是,营门有人么?}

\setlength{\hangindent}{56pt}{中军\hspace{30pt}什么人?}

\setlength{\hangindent}{56pt}{下书人\hspace{30pt}永安宫李严差人下书。}

\setlength{\hangindent}{56pt}{中军\hspace{30pt}候着。启禀帅爷:~永安宫李严差人下书。}

\setlength{\hangindent}{56pt}{孟达\hspace{30pt}传他进帐。}

\setlength{\hangindent}{56pt}{中军\hspace{30pt}是。下书人里面传你,小心了。}

\setlength{\hangindent}{56pt}{下书人\hspace{30pt}是。下书人叩头。}

\setlength{\hangindent}{56pt}{孟达\hspace{30pt}你奉何人所差?}

\setlength{\hangindent}{56pt}{下书人\hspace{30pt}奉永安宫李老爷所差,有书呈上。}

\setlength{\hangindent}{56pt}{孟达\hspace{30pt}后营用饭。}

\setlength{\hangindent}{56pt}{下书人\hspace{30pt}领爷赏赐。}

\setlength{\hangindent}{56pt}{(下书人下)}

\setlength{\hangindent}{56pt}{孟达\hspace{30pt}待我看来。}

\setlength{\hangindent}{56pt}{(孟达看信,起\textless{}\!{\bfseries\akai 牌子}\!\textgreater{})}

\setlength{\hangindent}{56pt}{孟达\hspace{30pt}来,传下书人。}

\setlength{\hangindent}{56pt}{中军\hspace{30pt}下书人。}

\setlength{\hangindent}{56pt}{(下书人上)}

\setlength{\hangindent}{56pt}{下书人\hspace{30pt}({\akai 念})后营用罢饭,帐下听回音。}

\setlength{\hangindent}{56pt}{下书人\hspace{30pt}谢爷的酒饭。}

\setlength{\hangindent}{56pt}{孟达\hspace{30pt}回覆你爷:~我这里修书不及,照书行事。}

\setlength{\hangindent}{56pt}{下书人\hspace{30pt}是,小人记下了。}

\setlength{\hangindent}{56pt}{(下书人下)}

\setlength{\hangindent}{56pt}{孟达\hspace{30pt}我正忧疑之间,李严有书到来,我岂忘了生死之交。不免假装重病,中军传令:~帅爷偶得重病,暂将人马撤回,再听调用。}

\setlength{\hangindent}{56pt}{中军\hspace{30pt}得令。下面听者:~}

\setlength{\hangindent}{56pt}{({\akai 内}应{\hwfs 介})}

\setlength{\hangindent}{56pt}{中军\hspace{30pt}元帅偶得重病,暂将人马撤回,再听调用。}

\setlength{\hangindent}{56pt}{众\hspace{30pt}({\akai 内})传令。}

\setlength{\hangindent}{56pt}{孟达\hspace{30pt}({\akai 念})谁人不思故乡土,洛阳虽好不如家。}

\setlength{\hangindent}{56pt}{孟达\hspace{30pt}唉哟哟,好不痛死人也。}

\setlength{\hangindent}{56pt}{(众搀扶孟达领下)\hspace{20pt}}

\vspace{3pt}{\centerline{{[}{\hei 第七场}{]}}}\vspace{5pt}

\setlength{\hangindent}{56pt}{(四文堂、四水军众站门上,小生扮陆逊\textless{}\!{\bfseries\akai 牌子}\!\textgreater{}上)}

\setlength{\hangindent}{56pt}{陆逊\hspace{30pt}吾乃东吴水军都督陆逊,今有北魏曹丕五路攻川,许以割地为约,令起大兵十万出峡口,攻打涪城。吾想吴、魏两国皆非诸葛之敌手,万难取胜。是我奏明主公,用两全之计,虚作人情,兵扎三江口,坐观胜败,就中取事。众将官------}

\setlength{\hangindent}{56pt}{(众应{\hwfs 介})}}

\setlength{\hangindent}{56pt}{陆逊\hspace{30pt}兵发三江口去者。}

\setlength{\hangindent}{56pt}{(\textless{}\!{\bfseries\akai 牌子}\!\textgreater{}众原场)}

\setlength{\hangindent}{56pt}{众\hspace{30pt}前面已到三江口。}

\setlength{\hangindent}{56pt}{陆逊\hspace{30pt}安营下寨。}

\setlength{\hangindent}{56pt}{(众应介,同下)}

\vspace{3pt}{\centerline{{[}{\hei 第八场}{]}}}\vspace{5pt}

\setlength{\hangindent}{56pt}{(四朝官上,末扮许靖,白髯;外扮董允,黪髯;副扮杜琼,黑三;生扮邓芝,黑三)}

\setlength{\hangindent}{56pt}{许靖\hspace{30pt}({\akai 念})金钟响罢禁门开,}

\setlength{\hangindent}{56pt}{董允\hspace{30pt}({\akai 念})雨露恩深拜龙台。}

\setlength{\hangindent}{56pt}{杜琼\hspace{30pt}({\akai 念})常思汉鼎三分在,}

\setlength{\hangindent}{56pt}{邓芝\hspace{30pt}({\akai 念})灭魏伐吴待时来。}

\setlength{\hangindent}{56pt}{许靖\hspace{30pt}下官,司徒许靖。}

\setlength{\hangindent}{56pt}{董允\hspace{30pt}下官,黄门侍郎董允。}

\setlength{\hangindent}{56pt}{杜琼\hspace{30pt}下官,谏议大夫杜琼。}

\setlength{\hangindent}{56pt}{邓芝\hspace{30pt}下官,户部尚书邓芝。}

\setlength{\hangindent}{56pt}{许靖\hspace{30pt}请了。}

\setlength{\hangindent}{56pt}{董允、杜琼、邓芝} 请了。

\setlength{\hangindent}{56pt}{许靖\hspace{30pt}今日早朝圣驾登殿,必有国政议论。}

\setlength{\hangindent}{56pt}{董允、杜琼、邓芝  金钟三响,想是圣驾临朝。}

\setlength{\hangindent}{56pt}{许靖、董允、杜琼、邓芝 请。}

\setlength{\hangindent}{56pt}{(许靖、董允、杜琼、邓芝分班站介,四太监、一大太监站门,小生扮刘禅上)}

\setlength{\hangindent}{56pt}{刘禅\hspace{30pt}{[}{\akai {\akai 引}子}{]}诏书赐孤王,驾坐成都称帝邦。}

\setlength{\hangindent}{56pt}{刘禅\hspace{30pt}({\akai 念})父皇白帝驾殡天,众卿扶保坐江山。但得吴、魏干戈定,永守西蜀心也安。}

\setlength{\hangindent}{56pt}{刘禅\hspace{30pt}孤刘禅,国号建兴,只因父皇兵伐东吴失利,兵退白帝城;圣驾殡天,托孤与诸葛丞相。扶孤登基,内理国政,外治民情,皆赖丞相之奇才也。今日早朝,内侍,展放龙帘。}

\setlength{\hangindent}{56pt}{大太监\hspace{30pt}领旨。}

\setlength{\hangindent}{56pt}{(黄门官上)}

\setlength{\hangindent}{56pt}{黄门官\hspace{30pt}({\akai 念})忙将动地惊天事,奏与君王御驾知。}

\setlength{\hangindent}{56pt}{黄门官\hspace{30pt}臣黄门官见驾,吾皇万岁。}

\setlength{\hangindent}{56pt}{刘禅\hspace{30pt}卿有何本奏?}

\setlength{\hangindent}{56pt}{黄门官\hspace{30pt}今有北魏曹丕兵发五路啊!~(\textless{}\!{\bfseries\akai 牌子}\!\textgreater{})}

\setlength{\hangindent}{56pt}{刘禅\hspace{30pt}既有此事,就命卿相府召丞相入朝理事。}

\setlength{\hangindent}{56pt}{黄门官\hspace{30pt}领旨。}

\setlength{\hangindent}{56pt}{黄门官\hspace{30pt}({\akai 念})五路雄兵起,三国战不息。}

\setlength{\hangindent}{56pt}{(黄门官下)}

\setlength{\hangindent}{56pt}{刘禅\hspace{30pt}适才黄门奏道:~曹丕五路进兵,攻取我国。众卿,}

\setlength{\hangindent}{56pt}{众\hspace{30pt}万岁。}

\setlength{\hangindent}{56pt}{刘禅\hspace{30pt}有何良谋可退贼兵?}

\setlength{\hangindent}{56pt}{众\hspace{30pt}万岁圣意宽怀,暂请放心。\footnote{刘曾复先生钞本作``万岁圣宽怀,暂且放心。'',此处从``安五路(总讲)''。}待丞相入朝,必有良谋妙策。}

\setlength{\hangindent}{56pt}{刘禅\hspace{30pt}孤亦想到如此。}

\setlength{\hangindent}{56pt}{黄门官\hspace{30pt}({\akai 内})走啊!}

\setlength{\hangindent}{56pt}{(黄门官上)}

\setlength{\hangindent}{56pt}{黄门官\hspace{30pt}启奏万岁:~丞相有病在府,不容进见,特来交旨。}

\setlength{\hangindent}{56pt}{刘禅\hspace{30pt}卿家暂退。}

\setlength{\hangindent}{56pt}{黄门官\hspace{30pt}领旨。}

\setlength{\hangindent}{56pt}{(黄门官下)}

\setlength{\hangindent}{56pt}{董允、杜琼\hspace{30pt}臣董允、杜琼同到相府求计,看有何说。}

\setlength{\hangindent}{56pt}{刘禅\hspace{30pt}二卿愿去,速来回奏。}

\setlength{\hangindent}{56pt}{董允、杜琼\hspace{30pt}领旨。}

\setlength{\hangindent}{56pt}{(董允、杜琼下)}

\setlength{\hangindent}{56pt}{众\hspace{30pt}诸事(已)毕\footnote{刘曾复先生钞本作``诸事毕''``安五路(总讲)''此处补``已'',此处从之。},请驾回宫。}

\setlength{\hangindent}{56pt}{刘禅\hspace{30pt}退班。}

\setlength{\hangindent}{56pt}{(众分班下)}

\vspace{3pt}{\centerline{{[}{\hei 第九场}{]}}}\vspace{5pt}\footnote{这一场是``安五路(总讲)''没有的。}

\setlength{\hangindent}{56pt}{(丑扮门官上,\textless{}\!{\bfseries\akai 普贤歌}\!\textgreater{}干牌子)}

\setlength{\hangindent}{56pt}{门官\hspace{30pt}({\akai 念})剑戟峥嵘将相门,谁敢杂踏与高声。常闻细柳营,天子按辔行,何况文官与武臣。}

\setlength{\hangindent}{56pt}{门官\hspace{30pt}咱家诸葛丞相府下门吏便是。可怪我家相爷向来真正霄晓勿遑\footnote{``霄晓勿遑''即不分昼夜之意。},日理万机。近日何故,终朝不出内阁,一切政务不理。慢说羽书雪片,多官请事。就是方才圣明来召,也是推病不起,你想这样身价,可是亘古罕有的。闲言少说,只恐又有人来请见,我且坐守等候。}

\setlength{\hangindent}{56pt}{(董允、杜琼、二青衣扮随侍上)}

\setlength{\hangindent}{56pt}{董允\hspace{30pt}({\akai 念})宰正百官才独称,}

\setlength{\hangindent}{56pt}{杜琼\hspace{30pt}({\akai 念})仪型四海圣王尊。}

\setlength{\hangindent}{56pt}{董允\hspace{30pt}({\akai 念})仪门台下下了马,}

\setlength{\hangindent}{56pt}{杜琼\hspace{30pt}({\akai 念})好向阁内问安宁。}

\setlength{\hangindent}{56pt}{董允、杜琼\hspace{30pt}回避。}

\setlength{\hangindent}{56pt}{(二随侍下)}

\setlength{\hangindent}{56pt}{董允\hspace{30pt}看仪门肃静,人寂无声,同到门上。}

\setlength{\hangindent}{56pt}{杜琼\hspace{30pt}请。}

\setlength{\hangindent}{56pt}{董允\hspace{30pt}({\akai 念})月照牙旂肃,}

\setlength{\hangindent}{56pt}{杜琼\hspace{30pt}({\akai 念})风吹画角寒。}

\setlength{\hangindent}{56pt}{董允、杜琼\hspace{30pt}门官。}

\setlength{\hangindent}{56pt}{门官\hspace{30pt}原来二位大人,请坐。}

\setlength{\hangindent}{56pt}{董允、杜琼\hspace{30pt}请。}

\setlength{\hangindent}{56pt}{董允\hspace{30pt}我来问你,丞相往常勤于政事,近日不见升堂,是何缘故?}

\setlength{\hangindent}{56pt}{门官\hspace{30pt}小官不知(何)故\footnote{刘曾复先生钞本作``小官不知故''。}。({\akai 念})宅门高挂止步牌,一切杂物何禀来。}

\setlength{\hangindent}{56pt}{门官\hspace{30pt}二位大人不曾见么:~({\akai 念})门外朝事与边事,谁敢进内去相催。}

\setlength{\hangindent}{56pt}{董允\hspace{30pt}这却(为)何也?\footnote{刘曾复先生钞本作``这却何地'',文意欠通。}}

\setlength{\hangindent}{56pt}{杜琼\hspace{30pt}丞相连日起居如何,饮食可曾加减?}

\setlength{\hangindent}{56pt}{门官\hspace{30pt}这小官也曾打听:~({\akai 念})起居倒觉不甚衰,肥肉三餐不吃斋。}

\setlength{\hangindent}{56pt}{董允、杜琼\hspace{30pt}既然身健食壮,缘何不出堂理事?}

\setlength{\hangindent}{56pt}{门官\hspace{30pt}据小官想来,相爷多管害心病\footnote{段公平{\scriptsize 君}注:~``多管'',即多半,大概之意。多见于元明小说、话本等。。}。}

\setlength{\hangindent}{56pt}{董允、杜琼\hspace{30pt}甚么心病?}

\setlength{\hangindent}{56pt}{门官\hspace{30pt}二位大人,从来出将入相之家,不言歌童舞女成群,即便那娇妻美妾也却无数。可怜我们相爷就是一位黄夫人,心性却有姜嫄之德,其颜却如嫫母之陋。\footnote{姜嫄是传说中上古农神``后稷''之母,非常贤德,后世尊为``圣母'';嫫母是传说中的丑女,是黄帝的次妃。}恁教相爷耐得住呢。}

\setlength{\hangindent}{56pt}{董允、杜琼\hspace{30pt}胡说。}

\setlength{\hangindent}{56pt}{门官\hspace{30pt}世情如此,不是小官妄说。}

\setlength{\hangindent}{56pt}{董允、杜琼\hspace{30pt}你可进去禀知说董允、杜琼特来问候金安,还有大事面启。}

\setlength{\hangindent}{56pt}{门官\hspace{30pt}哎呀呀$\cdots{}\cdots{}$相爷数日传谕:~所有一应大小官员,勿得擅入禀事。方才圣命来召,尚尔\footnote{段公平{\scriptsize 君}注:~``尚尔'':~即尚且之意。如纪昀《阅微草堂笔记·滦阳消夏录五》:~``对神尚尔,对人可知''。}辞去,何况大人。}

\setlength{\hangindent}{56pt}{董允、杜琼\hspace{30pt}我等今奉圣命而来,定要请见。}

\setlength{\hangindent}{56pt}{门官\hspace{30pt}既然如此,小官传禀便了。}

\setlength{\hangindent}{56pt}{(门官下)}

\setlength{\hangindent}{56pt}{董允\hspace{30pt}({\akai 念})似此葫芦闷\footnote{刘曾复先生钞本作``胡卢闷''。``闷葫芦''比喻极难猜透或令人纳闷的事或话。}难审,}

\setlength{\hangindent}{56pt}{杜琼\hspace{30pt}({\akai 念})只须等待听好音。}

\setlength{\hangindent}{56pt}{(门官上)}

\setlength{\hangindent}{56pt}{门官\hspace{30pt}回禀二位大人,丞相说:~知道了,请二位大人不必进见,有甚军国大事,等待病体稍可,改日自出都堂会议。请回罢。}

\setlength{\hangindent}{56pt}{(门官下)}

\setlength{\hangindent}{56pt}{董允\hspace{30pt}哎呀,军情至急,哪还等得改日会议。}

\setlength{\hangindent}{56pt}{杜琼\hspace{30pt}量来难以进见,且回奏圣上再处。}

\setlength{\hangindent}{56pt}{董允、杜琼\hspace{30pt}请。}

\setlength{\hangindent}{56pt}{董允\hspace{30pt}({\akai 念})召命尚安难通问,}

\setlength{\hangindent}{56pt}{董允\hspace{30pt}带马。}

\setlength{\hangindent}{56pt}{(二随侍左右上,应)\hspace{20pt}}

\setlength{\hangindent}{56pt}{杜琼\hspace{30pt}({\akai 念})何况区区僚佐臣。}

\setlength{\hangindent}{56pt}{(董允、杜琼急下)\hspace{20pt}}

\vspace{3pt}{\centerline{{[}{\hei 第十场}{]}}}\vspace{5pt}

\setlength{\hangindent}{56pt}{(二宫女、一大太监{\akai 引}正旦扮吴后上)}

\setlength{\hangindent}{56pt}{(吴后\hspace{30pt}【{\akai 二黄慢板}】老王爷祖居在大树楼桑,在桃园三结义万世名扬。伐东吴中奸计全军俱丧,梦魂里白帝城痛断肝肠。)}

\setlength{\hangindent}{56pt}{吴后\hspace{30pt}{[}{\akai 引子}{]}珠帘高卷似蓬莱,追思先帝心痛哀。}

\setlength{\hangindent}{56pt}{吴后\hspace{30pt}({\akai 念}) 老王祖居在楼桑,桃园结义万古扬。兵伐东吴全军丧,梦魂白帝断肝肠。}

\setlength{\hangindent}{56pt}{吴后\hspace{30pt}哀家吴后,先皇昭烈帝与二君侯报仇心切,兵伐东吴,连营失利,败入白帝,恸想二弟,思念桃园,气忿成疾,晏了圣驾,托孤与诸葛丞相。扶保皇儿,登了龙位,内修国政,外治民情,依赖丞相之贤也。正是:~({\akai 念})谋猷\footnote{谋猷为计谋,谋略之意。}人钦敬,调和鼎鼐臣。}

\setlength{\hangindent}{56pt}{刘禅\hspace{30pt}({\akai 内})摆驾。}

\setlength{\hangindent}{56pt}{(四太监一字引刘禅上)}

\setlength{\hangindent}{56pt}{刘禅\hspace{30pt}【{\akai 西皮摇板/散板}\footnote{刘曾复先生钞本未注明板式,下同。}】内臣宰无计策孤心烦躁,老丞相不出府所为哪条。}

\setlength{\hangindent}{56pt}{内监\hspace{30pt}万岁朝罢回宫,与国太请安。}

\setlength{\hangindent}{56pt}{吴后\hspace{30pt}请。}

\setlength{\hangindent}{56pt}{内监\hspace{30pt}请驾进宫。}

\setlength{\hangindent}{56pt}{刘禅\hspace{30pt}参见母后千岁。}

\setlength{\hangindent}{56pt}{吴后\hspace{30pt}哀家{平善如常}\footnote{刘曾复先生钞本作``加常'',系误;此处从``安五路(总讲)''。},坐了讲话。}

\setlength{\hangindent}{56pt}{刘禅\hspace{30pt}谢母后。唉!}

\setlength{\hangindent}{56pt}{吴后\hspace{30pt}王驾叹息为了何事?}

\setlength{\hangindent}{56pt}{刘禅\hspace{30pt}启母后:~大事不好了!}

\setlength{\hangindent}{56pt}{吴后\hspace{30pt}有何大事如此惊慌?}

\setlength{\hangindent}{56pt}{刘禅\hspace{30pt}北魏曹丕今发大兵五十万,攻打西川,怎不惊怕?}

\setlength{\hangindent}{56pt}{吴后\hspace{30pt}众文武岂无退兵之策?}

\setlength{\hangindent}{56pt}{刘禅\hspace{30pt}文武虽多,惶惶无策\footnote{刘曾复先生钞本与``安五路(总讲)''均作``慌慌无策'',此处从《三国演义》原文。}。}

\setlength{\hangindent}{56pt}{吴后\hspace{30pt}诸葛丞相必有退兵之策,召来一问。}

\setlength{\hangindent}{56pt}{刘禅\hspace{30pt}母后有所不知,儿也曾诏他上殿,怎奈\footnote{刘曾复先生钞本作``岂不等'',此处从``安五路(总讲)''。}推病在府,不容使臣入见;又命董允、杜琼同到相府问计,未见回奏如何。}

\setlength{\hangindent}{56pt}{(董允、杜琼上)}

\setlength{\hangindent}{56pt}{董允\hspace{30pt}【{\akai 西皮摇板/散板}】忙步踉跄走御道,}

\setlength{\hangindent}{56pt}{杜琼\hspace{30pt}【{\akai 西皮摇板/散板}】气喘嘘嘘滚油浇。}

\setlength{\hangindent}{56pt}{董允\hspace{30pt}【{\akai 西皮摇板/散板}】丞相忠心改变了,}

\setlength{\hangindent}{56pt}{杜琼\hspace{30pt}【{\akai 西皮摇板/散板}】他把托孤火化消。}

\setlength{\hangindent}{56pt}{董允\hspace{30pt}【{\akai 西皮摇板/散板}】你我何言回奏好,}

\setlength{\hangindent}{56pt}{杜琼\hspace{30pt}【{\akai 西皮摇板/散板}】须将实言奏当朝。}

\setlength{\hangindent}{56pt}{董允、杜琼\hspace{30pt}原来老承奉在此,烦劳转奏:~董允、杜琼回奏交旨。}

\setlength{\hangindent}{56pt}{内监\hspace{30pt}二位老大人回来了。}

\setlength{\hangindent}{56pt}{董允、杜琼\hspace{30pt}回来了。}

\setlength{\hangindent}{56pt}{内监\hspace{30pt}万岁在延寿宫与国太等候回奏,待咱家与你二人请驾。}

\setlength{\hangindent}{56pt}{董允、杜琼\hspace{30pt}有劳老承奉。}

\setlength{\hangindent}{56pt}{内监\hspace{30pt}是咧,交给咱家。启奏万岁:~董允、杜琼宫外候旨。}

\setlength{\hangindent}{56pt}{刘禅\hspace{30pt}儿启母后:~董允、杜琼回奏,待儿问明,回奏母后。}

\setlength{\hangindent}{56pt}{吴后\hspace{30pt}董允、杜琼乃旧日老臣,国事紧急,暂止肃避之条,宣进延寿宫,哀家面前回奏。}

\setlength{\hangindent}{56pt}{刘禅\hspace{30pt}母后之言极是。内侍,宣董允、杜琼进宫,在国太驾前回奏。}

\setlength{\hangindent}{56pt}{内监\hspace{30pt}董允、杜琼进宫,在国太驾前回奏。}

\setlength{\hangindent}{56pt}{董允、杜琼\hspace{30pt}领旨。臣董允、杜琼愿国太千岁。}

\setlength{\hangindent}{56pt}{吴后\hspace{30pt}二卿平身。}

\setlength{\hangindent}{56pt}{董允、杜琼\hspace{30pt}千千岁。}

\setlength{\hangindent}{56pt}{吴后\hspace{30pt}二卿同到相府求计,丞相有何良策回奏?}

\setlength{\hangindent}{56pt}{董允、杜琼\hspace{30pt}臣启国太:~丞相推病在府,不容臣等轻入,特来回奏。}

\setlength{\hangindent}{56pt}{刘禅\hspace{30pt}丞相推病为辞\footnote{刘曾复先生钞本与``安五路(总讲)''均作``推病为词''。},并不入朝理事,又无良谋回奏,不如小王趁早死了罢。}

\setlength{\hangindent}{56pt}{吴后\hspace{30pt}王驾休得如此,我想老王曾将大事托孤与丞相,皇儿拜他以为相父\footnote{段公平{\scriptsize 君}注:~刘曾复先生钞本此句作``皇儿拜他以为相父称之'',文意欠通,疑是``皇儿拜他以为相父''和``皇儿以相父称之''两句错杂而成。考``安五路(总讲)''原亦作``皇儿拜他以为相父称之'',后删去``称之'',作``皇儿拜他以为相父'',此处从``安五路(总讲)''。}。人臣之中位至极矣。今曹丕明知老王殡天,皇儿年幼,乘我国新丧,人心未定,五路进兵夺取西川,社稷危急之际,假以推病为辞,一谋不设。哀家亲到相府求计,看有何说。}

\setlength{\hangindent}{56pt}{刘禅\hspace{30pt}二卿意下如何?}

\setlength{\hangindent}{56pt}{董允、杜琼\hspace{30pt}万岁,依臣等愚昧之见,国太不可轻往。料丞相不出府门,必有奇谋妙策。暂请主公御驾亲往求计\footnote{``安五路(总讲)''此处原作``求计'',改为``问计''。},如有怠慢,再请国太召丞相入太庙对老王御影问之可也。}

\setlength{\hangindent}{56pt}{吴后\hspace{30pt}二卿所奏有理,皇儿速往,哀家立听回奏。}

\setlength{\hangindent}{56pt}{吴后\hspace{30pt}【{\akai 西皮摇板}】西川地到如今我蜀帝基,恨曹贼兴人马五路告急。却为何老相爷坐视不理,这内中必有那妙算神机。}

\setlength{\hangindent}{56pt}{(吴后\textless{}\!{\bfseries\akai 小锣打下}\!\textgreater{})\footnote{刘曾复先生钞本注``以下`观鱼遣邓'''。}}

\setlength{\hangindent}{56pt}{刘禅\hspace{30pt}摆驾。}

\setlength{\hangindent}{56pt}{(四大铠两边上,四太监{\akai 引}刘禅上辇\textless{}\!{\bfseries\akai 牌子}\!\textgreater{},当场见门官)}

\setlength{\hangindent}{56pt}{众\hspace{30pt}圣驾到。}

\setlength{\hangindent}{56pt}{门官\hspace{30pt}小臣接驾。}

\setlength{\hangindent}{56pt}{(刘禅下辇介,\textless{}\!{\bfseries\akai 牌子}\!\textgreater{}停)}

\setlength{\hangindent}{56pt}{门官\hspace{30pt}小臣叩见万岁。}

\setlength{\hangindent}{56pt}{刘禅\hspace{30pt}起来。}

\setlength{\hangindent}{56pt}{门官\hspace{30pt}万万岁。}

\setlength{\hangindent}{56pt}{刘禅\hspace{30pt}前去通禀。}

\setlength{\hangindent}{56pt}{门官\hspace{30pt}丞相令出森严,不准小官通禀。}

\setlength{\hangindent}{56pt}{刘禅\hspace{30pt}丞相今在何处?}

\setlength{\hangindent}{56pt}{门官\hspace{30pt}臣亦不知。只有丞相钧谕:~挡住百官,勿得擅入。}

\setlength{\hangindent}{56pt}{刘禅\hspace{30pt}起来。}

\setlength{\hangindent}{56pt}{门官\hspace{30pt}谢万岁!}

\setlength{\hangindent}{56pt}{刘禅\hspace{30pt}相父既有此谕,众臣府外等候,毋得喧哗。}

\setlength{\hangindent}{56pt}{众\hspace{30pt}领旨。}

\setlength{\hangindent}{56pt}{(众分下)}

\setlength{\hangindent}{56pt}{刘禅\hspace{30pt}门官引起。}

\setlength{\hangindent}{56pt}{门官\hspace{30pt}领旨。}

\setlength{\hangindent}{56pt}{刘禅\hspace{30pt}【{\akai 二黄摇板}】龙离潭凤离巢论礼不雅\footnote{``安五路(总讲)''本作``礼论不雅'',旁注``也是无法'';李元皓{\scriptsize 君}注``不雅'',犹言``君不登臣门''之意。},为的是平五路君到臣家。}

\vspace{3pt}{\centerline{{[}{\hei 第十一场}{]}}}\vspace{5pt}

\setlength{\hangindent}{56pt}{(诸葛亮持竹杖上)\hspace{20pt}}

\setlength{\hangindent}{56pt}{诸葛亮\hspace{30pt}【{\akai 二黄三眼}】报国家报不过黎元为大,扭人心扭不过事理无差。恨曹丕受禅台惨行强霸,叛逆贼终有日报应相加。我本当去问罪发动人马,怎奈我兵新败难以去杀。哭献帝恸先皇淋漓泪洒,好教人肝胆碎心乱如麻。}

\setlength{\hangindent}{56pt}{(诸葛亮大边外场观鱼,门官{\akai 引}刘禅上)}

\setlength{\hangindent}{56pt}{刘禅\hspace{30pt}【{\akai 二黄原板}】入相府({\akai 或}:~进相府)穿廊厦肃静幽雅,过几层曲湾处倒也可夸。进花园见相父闲坐潇洒,}

\setlength{\hangindent}{56pt}{(刘禅指门官退下,门官作揖退下)}

\setlength{\hangindent}{56pt}{刘禅\hspace{30pt}【{\akai 二黄原板}】孤这里走近前侧耳听他。}

\setlength{\hangindent}{56pt}{诸葛亮\hspace{30pt}【{\akai 二黄原板}】汉高皇创基业治平天下,至孝平方五载丧了邦家。}

\setlength{\hangindent}{56pt}{(诸葛亮站)}

\setlength{\hangindent}{56pt}{诸葛亮\hspace{30pt}【{\akai 二黄原板}】光武兴白水村【{\footnotesize 转}{\akai 二黄三眼}】重整人马,访邓禹、收岑彭到处征伐。剐王莽、诛苏献神惊鬼怕,洛阳城修宫殿一统中华。四百载东西汉六元七甲,传至在献帝朝国乱如麻({\akai 或}:~群寇如麻)。十常侍乱宫闱董卓强霸,许田射猎曹孟德把主欺压。曹丕贼篡汉位万民叫骂({\akai 或}:~万民怒发),吾主爷恨贼子咬碎齿牙。白帝城受血诏遗言留下,承受那托孤重({\akai 或}:~托孤情)怎敢有差。哭一声先帝爷在九泉之下,保佑臣增寿算扶保汉家。}

\setlength{\hangindent}{56pt}{(诸葛亮{\hwfs 看鱼指介})\hspace{20pt}}

\setlength{\hangindent}{56pt}{诸葛亮\hspace{30pt}【{\akai 二黄摇板}】这鱼儿\footnote{刘曾复先生说戏录音作``这鱼你'',此处从``安五路(总讲)''。}比陆逊行兵狡诈,有此计无此人怎能退他。猛回头站身边当今圣驾,}

\setlength{\hangindent}{56pt}{(诸葛亮{\hwfs 跪介},刘禅{\hwfs 搀介})\hspace{10pt}}

\setlength{\hangindent}{56pt}{诸葛亮\hspace{30pt}【{\akai 二黄摇板}】老孤臣轻慢君罪当重加。}

\setlength{\hangindent}{56pt}{刘禅\hspace{30pt}相父。}

\setlength{\hangindent}{56pt}{刘禅\hspace{30pt}【{\akai 二黄摇板}】相父病孤王我放心不下,因此上孤亲自来看卿家。见相父观鱼跃闲情潇洒,这几天小王我心乱如麻({\akai 或}:~这几天相父病我心乱如麻)。\textless{}\!{\bfseries\akai 行弦}\!\textgreater{}}

\setlength{\hangindent}{56pt}{诸葛亮\hspace{30pt}陛下何事忧心?}

\setlength{\hangindent}{56pt}{刘禅\hspace{30pt}【{\akai 二黄摇板}】曹丕无故兴人马,五路大兵来战杀({\akai 或}:~来征杀)。文武百官心惊怕,相父有病又在家。西蜀倾危在眼下,求取良谋\footnote{刘曾复先生钞本与  ``安五路(总讲)''均作``求条良谋''。}去退他。}

\setlength{\hangindent}{56pt}{诸葛亮\hspace{30pt}哦!}

\setlength{\hangindent}{56pt}{诸葛亮\hspace{30pt}【{\akai 西皮导板}】尊我主请正坐容臣参驾,}

\setlength{\hangindent}{56pt}{(刘禅正坐,诸葛亮拜,刘禅扶)}

\setlength{\hangindent}{56pt}{刘禅\hspace{30pt}相父免礼,请坐。}

\setlength{\hangindent}{56pt}{诸葛亮\hspace{30pt}谢座。}

\setlength{\hangindent}{56pt}{诸葛亮\hspace{30pt}【{\akai 西皮原板}】容忍老臣奏根芽:~曹丕国贼多奸诈,贿买羌、蛮辅助他。乱臣贼子人叫骂,谁肯真心死战杀。先皇在日({\akai 或}:~先皇在时)常怒发,本当问罪去征伐。乘人之丧毒手下,就是百万何惧他。臣非妄奏言虚假,望我主稳听捷报奏国家。\textless{}\!{\bfseries\akai 小拉子}\!\textgreater{}}

\setlength{\hangindent}{56pt}{刘禅\hspace{30pt}听相父之言,曹兵五路,如此容易退却?}

\setlength{\hangindent}{56pt}{诸葛亮\hspace{30pt}陛下只请放心({\akai 或}:~陛下只管放心,呃),且免忧虑。}

\setlength{\hangindent}{56pt}{刘禅\hspace{30pt}(呃,)望相父明言与孤,所调都是哪路人马?}

\setlength{\hangindent}{56pt}{诸葛亮\hspace{30pt}陛下。}

\setlength{\hangindent}{56pt}{诸葛亮\hspace{30pt}【{\akai 西皮原板}】臣不奏为的是行兵密法,怕的是成都民惊走天涯。非是臣瞒陛下事有虚假,都只为安人心保国保家。马孟起守西平威名颇大,魏文长疑兵计俱按兵法。赵子龙阳平关督理人马,一封书差人去赚走孟达。退吴兵臣已把良谋想下({\akai 或}:~东吴兵臣已罢良谋想下),缺少个能言士每日详查。}

\setlength{\hangindent}{56pt}{刘禅\hspace{30pt}呀。}

\setlength{\hangindent}{56pt}{刘禅\hspace{30pt}【{\akai 西皮原板}】孤王亲入相府地,君臣二人论军机。欺君篡位贼曹丕,兵发五路\footnote{刘曾复先生钞本作``兵伐五路''。}取川西。派将三员贼退去,孤王心内自猜疑。彼众我寡非容易,片纸岂退({\akai 或}:~片纸怎退)上庸敌?丞相在府观鱼戏,东吴怎肯卷旌旗。越思越想心忧虑,(收腿)孤必得拔树搜根仔细提。}

\setlength{\hangindent}{56pt}{诸葛亮\hspace{30pt}(啊,)万岁思索何事?}

\setlength{\hangindent}{56pt}{刘禅\hspace{30pt}孤王所虑,彼众我寡,孤闻贼兵五十万,五路攻川,相父所派蜀将三员,能否退敌。孟达、孙权何人抵挡?}

\setlength{\hangindent}{56pt}{诸葛亮\hspace{30pt}陛下!老王将大事托与老臣,臣怎敢不竭力?报答老王知遇之恩。况成都臣宰,不知兵法之妙论。若用成都人马,人民震动\footnote{刘曾复先生钞本作``人民振动''。},不能安稳,机关泄漏,大事去矣!臣身居相府之中,心在边关之外\footnote{刘曾复先生说戏录音中似作``神在边关之外'';段公平{\scriptsize 君}认为说戏录音误作``身在边关之外''。}(呀)。知己知彼,略韬\footnote{刘曾复先生说戏录音作``韬略''。}因人而使,量才择用。那马超祖居西土,声名远振。羌人称他为``神威将军''。羌人见是马超,必自退去。此西路之兵不必忧矣。}

\setlength{\hangindent}{56pt}{诸葛亮\hspace{30pt}【{\akai 西皮二六}】那羌王柯比能兴兵犯境,那马超继祖先世居西平。他父祖自从来声名远振,那羌人称马超神威将军。臣命他用奇兵四下伏定,那羌王见而丧胆决不敢前来相争。}

\setlength{\hangindent}{56pt}{刘禅\hspace{30pt}哦(哦),这一路退得妙!那蛮王孟获,闻他骁勇无比,谁可敌他?}

\setlength{\hangindent}{56pt}{诸葛亮\hspace{30pt}那南蛮孟获兵犯益州四郡,臣使魏延用疑兵之计。孟获虽勇,多生疑心,必要自退,我兵随后追杀,必然大获全胜。此南路之兵------陛下------心勿忧矣!}

\setlength{\hangindent}{56pt}{诸葛亮\hspace{30pt}【{\akai 西皮快板}】南蛮贼他夙习智量浅近,哪知晓孙、吴法机谋实深。忽见我左右军无数隐隐,管教他神魂不定不敢交兵。}

\setlength{\hangindent}{56pt}{刘禅\hspace{30pt}【{\akai 西皮摇板}】我相父素昔来谋猷谨慎,知己知彼着着胜人。还有那第三路汉中要紧,第四路恐难退贼将曹真。}

\setlength{\hangindent}{56pt}{诸葛亮\hspace{30pt}汉中是李严把守,孟达与李严有生死之交。臣已暗作一书,如严亲笔,着人潜递\footnote{刘曾复先生说戏录音作``潜送''。}孟达,孟达见之,必然推病不出。况孟达并非李严对手,臣回成都,留严镇守永安宫,正为此也。东路之兵,万岁,不足忧也。}

\setlength{\hangindent}{56pt}{刘禅\hspace{30pt}哦哦哦$\cdots{}\cdots{}$}

\setlength{\hangindent}{56pt}{诸葛亮\hspace{30pt}那曹真领中原大兵十万,攻取阳平。那阳平本非用武之地,山岭险峻\footnote{刘曾复先生说戏录音作``山岭峻险''。},道路崎岖。行运粮草不便,臣命赵云暗设人马,坚守勿战。待彼粮尽,一战成功。曹真必败于赵云之手,此北路人马不足忧矣。呃,臣尚恐不能全保,又秘调关兴、张苞呵------}

\setlength{\hangindent}{56pt}{诸葛亮\hspace{30pt}【{\akai 西皮摇板}】恐四路有哪处力不胜任,故命他分要口结寨安营。有不虞急提兵前往救应,{此密事故未便先使人闻}。}

\setlength{\hangindent}{56pt}{刘禅\hspace{30pt}如此说来,五路贼兵,(呃,)已退四路了。}

\setlength{\hangindent}{56pt}{诸葛亮\hspace{30pt}正是。}

\setlength{\hangindent}{56pt}{刘禅\hspace{30pt}孤恐孙权必怀伐吴之恨,借此而入,当之如何?}

\setlength{\hangindent}{56pt}{诸葛亮\hspace{30pt}(唉,)陛下呀!}

\setlength{\hangindent}{56pt}{诸葛亮\hspace{30pt}【{\akai 西皮摇板}】量孙权必观望兵未出境,止需用一能士说彼连横。臣观鱼非潇洒寻思线{\akai 引},看机变好得个这舌辩的苏秦呐。}

\setlength{\hangindent}{56pt}{刘禅\hspace{30pt}如相父之言,那五路大兵不日全退了?}

\setlength{\hangindent}{56pt}{诸葛亮\hspace{30pt}然。}

\setlength{\hangindent}{56pt}{刘禅\hspace{30pt}呵哈哈$\cdots{}\cdots{}$哎呀呀相父,你真有移星换斗之智,神鬼不测之机。使寡人万虑皆释矣。}

\setlength{\hangindent}{56pt}{诸葛亮\hspace{30pt}陛下既释怀疑,可请驾快快回宫,奏知太后要紧呐。}

\setlength{\hangindent}{56pt}{刘禅\hspace{30pt}既有此万全之策({\akai 或}:~万全之计),何必这等着忙,定要赶奏太后。}

\setlength{\hangindent}{56pt}{诸葛亮\hspace{30pt}不是呵,陛下若不及早回奏\footnote{刘曾复先生钞本作``急早回奏''。},诚恐太后已向太庙召臣,那时教老臣何以担当得起?}

\setlength{\hangindent}{56pt}{刘禅\hspace{30pt}(呃,)此话($\cdots{}\cdots{}$呃,)相父何以知之?}

\setlength{\hangindent}{56pt}{诸葛亮\hspace{30pt}(呃,)不过是推情度理(而已)。}

\setlength{\hangindent}{56pt}{刘禅\hspace{30pt}(唉,)真正羞煞群僚也({\akai 或}:~真正愧煞群僚也)。}

\setlength{\hangindent}{56pt}{刘禅\hspace{30pt}【{\akai 西皮摇板}】你谋猷真使人钦敬,不愧调和鼎鼐臣。}

\setlength{\hangindent}{56pt}{诸葛亮\hspace{30pt}【{\akai 西皮摇板}】鞠躬尽瘁臣之分,敢忘先帝委托恩({\akai 或}:~怎敢忘却先帝恩)。请主回宫心安定,}

\setlength{\hangindent}{56pt}{(\textless{}\!{\bfseries\akai 牌子}\!\textgreater{}刘禅、诸葛亮出门,众上,刘禅上辇{\hwfs 介})}

\setlength{\hangindent}{56pt}{诸葛亮\hspace{30pt}老臣送驾。}

\setlength{\hangindent}{56pt}{刘禅\hspace{30pt}啊哈哈哈$\cdots{}\cdots{}$}

\setlength{\hangindent}{56pt}{(刘禅笑介,邓芝点头,诸葛亮望)}

\setlength{\hangindent}{56pt}{诸葛亮\hspace{30pt}邓大夫暂留一步。}

\setlength{\hangindent}{56pt}{(众、刘禅下,留邓芝,\textless{}\!{\bfseries\akai 牌子}\!\textgreater{}停)}

\setlength{\hangindent}{56pt}{诸葛亮\hspace{30pt}【{\akai 西皮摇板}】请留大夫说分明。}

\setlength{\hangindent}{56pt}{诸葛亮\hspace{30pt}请。}

\setlength{\hangindent}{56pt}{邓芝\hspace{30pt}请。}

\setlength{\hangindent}{56pt}{诸葛亮\hspace{30pt}大夫请坐。}

\setlength{\hangindent}{56pt}{邓芝\hspace{30pt}谢座。}

\setlength{\hangindent}{56pt}{(诸葛亮右上,邓芝左偏,坐)\hspace{10pt}}

\setlength{\hangindent}{56pt}{邓芝\hspace{30pt}不知丞相有何面谕?}

\setlength{\hangindent}{56pt}{诸葛亮\hspace{30pt}挽留大夫,非为别事,有一桩国事难心领教。}

\setlength{\hangindent}{56pt}{邓芝\hspace{30pt}何事难心?}

\setlength{\hangindent}{56pt}{诸葛亮\hspace{30pt}今有魏、蜀、吴,鼎分三国,欲讨二国,一统中兴,请问大夫,先讨哪国?}

\setlength{\hangindent}{56pt}{邓芝\hspace{30pt}(这$\cdots{}\cdots{}$)若以愚意论之,魏虽汉贼,占据中原,其势甚大,急难摇动(呵),合当徐徐缓图;今当主上新登宝位,民心未安。当与东吴连合\footnote{刘曾复先生钞本作``连和'',此处从《三国演义》原文。},结为唇齿之邦,永结盟好,一洗先帝之怨,此乃长久之计。未审丞相钧意若何?}

\setlength{\hangindent}{56pt}{诸葛亮\hspace{30pt}呵哈哈哈哈$\cdots{}\cdots{}$({\hwfs 笑}{\hwfs 介})吾亦思之久矣,无奈不得其人。}

\setlength{\hangindent}{56pt}{邓芝\hspace{30pt}其人何用?}

\setlength{\hangindent}{56pt}{诸葛亮\hspace{30pt}吾欲使人往结东吴,大夫既明此意,必能不辱君命。这一大任,非大夫不可。}

\setlength{\hangindent}{56pt}{邓芝\hspace{30pt}邓芝才疏智浅,诚恐有负丞相所托。}

\setlength{\hangindent}{56pt}{诸葛亮\hspace{30pt}明日我便奏知天子,大夫休要谦让,有负吾意。}

\setlength{\hangindent}{56pt}{(诸葛亮\hspace{30pt}【{\akai 西皮快板}\footnote{刘曾复先生钞本注,此段可不唱。}】休谦让莫推辞听我言讲,我和你作臣宰同侍君王。须念在先皇爷恩如海样,谈国政量人才非比寻常。同受过托孤重遗命曾降,也是你明此意劳苦应当。与东吴结唇齿好言讲上,灭汉贼({\akai 或}:~灭国贼)报国仇美名远扬。伯苗你若推辞【{\footnotesize 转}{\akai 西皮摇板}】(你)不肯前往,笑西蜀无能将{志}成风霜。况先皇待臣宰手足一样,秉赤胆方显你干国忠良。)}

\setlength{\hangindent}{56pt}{(邓芝\hspace{30pt}丞相。)}

\setlength{\hangindent}{56pt}{邓芝\hspace{30pt}【{\akai 西皮摇板}】先皇爷托孤情恩深海样,去东吴见孙权自有主张。}

\setlength{\hangindent}{56pt}{邓芝\hspace{30pt}谨遵台命,邓芝告退。}

\setlength{\hangindent}{56pt}{诸葛亮\hspace{30pt}(且慢,)书房小酌,聊佐行色。}

\setlength{\hangindent}{56pt}{邓芝\hspace{30pt}多谢丞相。}

\setlength{\hangindent}{56pt}{诸葛亮\hspace{30pt}请。}

\setlength{\hangindent}{56pt}{诸葛亮\hspace{30pt}【{\akai 西皮摇板}】我与你到书房饮酒欢畅,到明天同入朝启奏君王。}

\setlength{\hangindent}{56pt}{诸葛亮\hspace{30pt}大夫请。}

\setlength{\hangindent}{56pt}{邓芝\hspace{30pt}丞相请。}

\setlength{\hangindent}{56pt}{诸葛亮\hspace{30pt}正是:~({\akai 念})但愿仲谋纳君训,}

\setlength{\hangindent}{56pt}{邓芝\hspace{30pt}({\akai 念})得统中华贺升平。}

\setlength{\hangindent}{56pt}{诸葛亮\hspace{30pt}啊,}

\setlength{\hangindent}{56pt}{邓芝\hspace{30pt}啊,}

\setlength{\hangindent}{56pt}{诸葛亮、邓芝 哈哈哈哈$\cdots{}\cdots{}$({\hwfs 笑介})}

\setlength{\hangindent}{56pt}{诸葛亮\hspace{30pt}大夫请。}

\setlength{\hangindent}{56pt}{邓芝\hspace{30pt}丞相请。}

\setlength{\hangindent}{56pt}{诸葛亮、邓芝 呵呵哈哈哈$\cdots{}\cdots{}$({\hwfs 笑介})}

\setlength{\hangindent}{56pt}{(诸葛亮、邓芝下)\hspace{20pt}}

\vspace{3pt}{\centerline{{[}{\hei 第十二场}{]}}}\vspace{5pt}

\setlength{\hangindent}{56pt}{(\textless{}\!{\bfseries\akai 牌子}\!\textgreater{}旦扮祝融夫人\textless{}\!{\bfseries\akai 点绛唇}\footnote{刘曾复先生钞本此处径写``点将诗''。疑是\textless{}\!{\bfseries\akai 点绛唇}\!\textgreater{}牌子,后接\textless{}\!{\bfseries\akai 定场诗}\!\textgreater{}四句。}\!\textgreater{}上)}

\setlength{\hangindent}{56pt}{祝融夫人\hspace{30pt}({\akai 念})自幼生长在南方,喜读战策演刀枪。上阵能斩千员将,谁人敢犯我边疆。}

\setlength{\hangindent}{56pt}{祝融夫人\hspace{30pt}咱家乃洞府都蛮王孟获之妻祝融夫人是也。只因中原皇帝曹丕兵督五路,攻取西蜀。遣臣前来聘请咱家大王起蛮兵十万攻打川南四郡,去之日久,不见回来。是咱家放心不下,为此催办粮草,置买水牛、菜蟒,咱家亲身押赴军营。嘟,众蛮兵------}

\setlength{\hangindent}{56pt}{众\hspace{30pt}啊。(众应)}

\setlength{\hangindent}{56pt}{祝融夫人\hspace{30pt}咱家命你们所办粮草等物可曾齐备?}

\setlength{\hangindent}{56pt}{众\hspace{30pt}齐备多时。}

\setlength{\hangindent}{56pt}{祝融夫人\hspace{30pt}随咱家解送军营。}

\setlength{\hangindent}{56pt}{众\hspace{30pt}啊。(众应)}

\setlength{\hangindent}{56pt}{祝融夫人\hspace{30pt}【{\akai 西皮导板}】汉室三分争江山,}

\setlength{\hangindent}{56pt}{祝融夫人\hspace{30pt}【{\akai 西皮原板}】北魏使臣把兵搬。大王率领兵十万,攻打四郡夺西川。咱家算来日期远,不见大王转回还。解押粮草日夜赶,到军营花沾雨露续团圆。}

\vspace{3pt}{\centerline{{[}{\hei 第十三场}{]}}}\vspace{5pt}

\setlength{\hangindent}{56pt}{(四文堂,净扮魏延上)}

\setlength{\hangindent}{56pt}{魏延\hspace{30pt}{[}{\akai 引子}{]}奉命守边关,敌将心胆寒。}

\setlength{\hangindent}{56pt}{魏延\hspace{30pt}({\akai 念})少年英勇走天涯,杀死韩玄献长沙。弃暗保定刘先主,占据西川定邦家。}

\setlength{\hangindent}{56pt}{魏延\hspace{30pt}某乃西蜀大将魏延,奉了军师将令,命俺挡住蛮王孟获,不要临阵交锋;又道贼心性多疑,必要自退。我不免照书行事。}

\setlength{\hangindent}{56pt}{(报子上)}

\setlength{\hangindent}{56pt}{探子\hspace{30pt}报,蛮兵退去。}

\setlength{\hangindent}{56pt}{魏延\hspace{30pt}再探。}

\setlength{\hangindent}{56pt}{报子\hspace{30pt}得令。}

\setlength{\hangindent}{56pt}{(报子下)}

\setlength{\hangindent}{56pt}{魏延\hspace{30pt}且住,果然不出军师妙算,趁此追杀前去。众将官,杀!}

\setlength{\hangindent}{56pt}{(魏延下)}

\vspace{3pt}{\centerline{{[}{\hei 第十四场}{]}}}\vspace{5pt}

\setlength{\hangindent}{56pt}{(\textless{}\!{\bfseries\akai 牌子}\!\textgreater{}众、净扮孟获上)}

\setlength{\hangindent}{56pt}{孟获\hspace{30pt}孤,都蛮王孟获,今有北魏皇帝聘请孤家帮助,因此领了蛮兵十万攻打川南四郡。孤自安营以来,蜀将并不出马交锋,见他军马每日左出右入,右入左出,不知是何缘故。孤家素闻诸葛亮诡计多端,不要入他圈套,孤不免将人马撤回,暂归蛮洞,再作计较。}

\setlength{\hangindent}{56pt}{(报子上)}

\setlength{\hangindent}{56pt}{报子\hspace{30pt}报,蜀兵追杀前来。}

\setlength{\hangindent}{56pt}{孟获\hspace{30pt}再探。}

\setlength{\hangindent}{56pt}{报子\hspace{30pt}得令。}

\setlength{\hangindent}{56pt}{(报子下)}

\setlength{\hangindent}{56pt}{孟获\hspace{30pt}嘟,众蛮兵,迎上前去。}

\setlength{\hangindent}{56pt}{(魏延、众人会阵上)\hspace{20pt}}

\setlength{\hangindent}{56pt}{孟获\hspace{30pt}蜀将通名。}

\setlength{\hangindent}{56pt}{魏延\hspace{30pt}听者,某乃西蜀大将魏延,尔知道某家厉害\footnote{刘曾复先生钞本中``厉害''均作``利害''。},快些下马受死。}

\setlength{\hangindent}{56pt}{孟获\hspace{30pt}魏延,孤家开恩,饶尔不死,竟敢大胆追赶孤王,前来送死。}

\setlength{\hangindent}{56pt}{魏延\hspace{30pt}住了,蛮贼无故兴兵,助逆侵犯边界,占据疆土。要想逃走,留下尔的人头。}

\setlength{\hangindent}{56pt}{孟获\hspace{30pt}少要多言,看枪({\akai 或}:~看刀)。}

\setlength{\hangindent}{56pt}{(魏延、孟获二人开打,下)\hspace{10pt}}

\vspace{3pt}{\centerline{{[}{\hei 第十五场}{]}}}\vspace{5pt}

\setlength{\hangindent}{56pt}{(四手下站门上,关兴、张苞上)}

\setlength{\hangindent}{56pt}{张苞\hspace{30pt}俺,张苞。}

\setlength{\hangindent}{56pt}{关兴\hspace{30pt}俺,关兴。}

\setlength{\hangindent}{56pt}{张苞\hspace{30pt}贤弟请了。}

\setlength{\hangindent}{56pt}{关兴\hspace{30pt}请了。}

\setlength{\hangindent}{56pt}{张苞\hspace{30pt}你我奉了军师将令,带领人马四路接应。适才探马报道:~魏延追赶蛮王孟获,不知胜败如何。}

\setlength{\hangindent}{56pt}{关兴\hspace{30pt}你我前去接应。前去接应杀退那贼。}

\setlength{\hangindent}{56pt}{张苞\hspace{30pt}好,就此迎上前去。}

\setlength{\hangindent}{56pt}{张苞、关兴\hspace{30pt}众将官,杀上前去。}

\setlength{\hangindent}{56pt}{(众全下)}

\setlength{\hangindent}{56pt}{[}{连场}{]}

\setlength{\hangindent}{56pt}{(众围魏延,关兴、张苞上,挑开起打;众围孟获,孟获败下,关兴、张苞、魏延追下)}

\vspace{3pt}{\centerline{{[}{\hei 第十六场}{]}}}\vspace{5pt}

\setlength{\hangindent}{56pt}{(众、祝融夫人上,报子报上)\hspace{10pt}}

\setlength{\hangindent}{56pt}{报子\hspace{30pt}报,大王遭了围困。}

\setlength{\hangindent}{56pt}{祝融夫人\hspace{30pt}再探。}

\setlength{\hangindent}{56pt}{报子\hspace{30pt}得令。}

\setlength{\hangindent}{56pt}{(报子下)}

\setlength{\hangindent}{56pt}{祝融夫人\hspace{30pt}蛮兵们。}

\setlength{\hangindent}{56pt}{众\hspace{30pt}有。}

\setlength{\hangindent}{56pt}{祝融夫人\hspace{30pt}杀上前去。}

\setlength{\hangindent}{56pt}{(众全下)}

\vspace{3pt}{\centerline{{[}{\hei 第十七场}{]}}}\vspace{5pt}

\setlength{\hangindent}{56pt}{(众围孟获、孟优,旦上救下;关兴、张苞追下;祝融夫人与魏延架住,磕开)}

\setlength{\hangindent}{56pt}{魏延\hspace{30pt}杀来杀去,杀出一个蛮婆来了。呔,那蛮婆少要送死,老爷开恩饶你去罢。}

\setlength{\hangindent}{56pt}{祝融夫人\hspace{30pt}住着。我乃都蛮王孟获之妻祝融夫人是也。你们知道咱家厉害,就在马前磕头饶你们不死。}

\setlength{\hangindent}{56pt}{魏延\hspace{30pt}休得胡言,放马过来。}

\setlength{\hangindent}{56pt}{(魏延、祝融夫人开打,关兴、张苞上战介;蛮众败下,蜀众追下;祝融夫人上)}

\setlength{\hangindent}{56pt}{祝融夫人\hspace{30pt}这厮们果然厉害,待咱家飞刀伤他便了。}

\setlength{\hangindent}{56pt}{(张苞等追上)}

\setlength{\hangindent}{56pt}{祝融夫人\hspace{30pt}看咱家飞刀取你。}

\setlength{\hangindent}{56pt}{(张苞坠马介,众救下;祝融夫人拉孟获,蛮众随下;魏延、众上)}

\setlength{\hangindent}{56pt}{魏延\hspace{30pt}张小将军怎么样了?}

\setlength{\hangindent}{56pt}{张苞\hspace{30pt}末将身无伤损,可惜战马被她杀死。}

\setlength{\hangindent}{56pt}{魏延、关兴\hspace{30pt}此乃万幸。谢天谢地。}

\setlength{\hangindent}{56pt}{张苞\hspace{30pt}快快换马。待俺追上蛮妇,好报杀马之仇。}

\setlength{\hangindent}{56pt}{魏延\hspace{30pt}将军不必如此,天色已晚,道路不明,趁此收兵。}

\setlength{\hangindent}{56pt}{关兴\hspace{30pt}老将军之言甚是。}

\setlength{\hangindent}{56pt}{张苞\hspace{30pt}便宜了蛮婆。}

\setlength{\hangindent}{56pt}{魏延、关兴、张苞 众将官,收兵。}

\setlength{\hangindent}{56pt}{(\textless{}\!{\bfseries\akai 牌子}\!\textgreater{}众下)}

\vspace{3pt}{\centerline{{[}{\hei 第十八场}{]}}}\vspace{5pt}

\setlength{\hangindent}{56pt}{(祝融夫人搀孟获上)\hspace{20pt}}

\setlength{\hangindent}{56pt}{孟获\hspace{30pt}孤自兴兵以来,从无如此大败,似这等狼狈不堪,有何颜面回见各家洞主。我不免碰死了罢。}

\setlength{\hangindent}{56pt}{(祝融夫人拉孟获)\hspace{20pt}}

\setlength{\hangindent}{56pt}{祝融夫人\hspace{30pt}大王不要如此短见,自古道:~军家胜败乃古之常理。依咱主意,暂将人马撤回蛮洞,养足锐气,平整人马,再来报仇不迟。}

\setlength{\hangindent}{56pt}{孟获\hspace{30pt}\textless{}\!{\bfseries\akai 叫头}\!\textgreater{}诸葛亮,孔明!}

\setlength{\hangindent}{56pt}{孟获\hspace{30pt}孤家与你誓不两立也。}

\setlength{\hangindent}{56pt}{祝融夫人\hspace{30pt}走了的好。}

\setlength{\hangindent}{56pt}{孟获\hspace{30pt}悄悄地收兵。}

\setlength{\hangindent}{56pt}{(孟获、祝融夫人同下)}

\vspace{3pt}{\centerline{{[}{\hei 第十九场}{]}}}\vspace{5pt}

\setlength{\hangindent}{56pt}{(四太监、一大太监站门,净扮孙权上)}

\setlength{\hangindent}{56pt}{孙权\hspace{30pt}{[}{\akai 引子}{]}坐镇江东,三分鼎;半壁山河。}

\setlength{\hangindent}{56pt}{孙权\hspace{30pt}({\akai 念})陆逊年幼智超群,蜀兵百万尽皆焚。看来孤王有福分,刘备命丧白帝城。}

\setlength{\hangindent}{56pt}{孙权\hspace{30pt}孤,孙权,今有曹丕聘孤兵伐西蜀,孤王难作决策,也曾命人探听各路消息,未见回报。}

\setlength{\hangindent}{56pt}{虞翻\hspace{30pt}({\akai 内})走啊。}

\setlength{\hangindent}{56pt}{虞翻上)}

\setlength{\hangindent}{56pt}{虞翻\hspace{30pt}【{\akai 西皮摇板}\footnote{刘曾复先生钞本未注明板式。}】奉使连朝暗徵听,不道诸葛果然能。忙上银安将情禀,见了主公说分明。}

\setlength{\hangindent}{56pt}{虞翻\hspace{30pt}虞翻参见主公。}

\setlength{\hangindent}{56pt}{孙权\hspace{30pt}命你徵听曹兵各路进取西蜀消息怎么样了?}

\setlength{\hangindent}{56pt}{虞翻\hspace{30pt}主公容启:~({\akai 念})曹兵四路寇蜀,诸葛调军相迎:~马超西平退羌兵,疑兵孟获远遁;孟达推病不出,子龙拒走曹真。眼见四路尽解纷,谁与西蜀争胜。}

\setlength{\hangindent}{56pt}{孙权\hspace{30pt}啊,听你所言,那西蜀四路的曹兵,竟多被孔明暗调兵马,全皆逐退了。}

\setlength{\hangindent}{56pt}{虞翻\hspace{30pt}便是主公,幸喜听了陆逊之言,未曾动兵,不然今日也要羞归江东矣。}

\setlength{\hangindent}{56pt}{孙权\hspace{30pt}果然。咳,孔明啊$\cdots{}\cdots{}$你真有神通也。}

\setlength{\hangindent}{56pt}{孙权\hspace{30pt}【{\akai 西皮摇板}】似此神通令人敬,堪笑曹丕枉用心。兴衰此际难拟定,}

\setlength{\hangindent}{56pt}{(薛综上)}

\setlength{\hangindent}{56pt}{薛综\hspace{30pt}【{\akai 西皮摇板}】狂儒胆敢来批鳞。}

\setlength{\hangindent}{56pt}{薛综\hspace{30pt}臣薛综启事:~今有西蜀邓芝特来请见主公。}

\setlength{\hangindent}{56pt}{孙权\hspace{30pt}他来见孤何意?}

\setlength{\hangindent}{56pt}{薛综\hspace{30pt}此定是孔明遣他来作说客,退我第五路兵耳。}

\setlength{\hangindent}{56pt}{孙权\hspace{30pt}他来何以答之?}

\setlength{\hangindent}{56pt}{薛综\hspace{30pt}依臣鄙意,可于殿前\footnote{刘曾复先生钞本作``可于光殿前''(``光''字不确认,疑此字误衍或脱漏,如作``光明''),此处从《三国演义》原文。}设一大大油鼎,贮油数百斤,下用木炭烧得烈烈腾沸;再选身长面大\footnote{刘曾复先生钞本作``身长大面'',此处从《三国演义》原文作``身长面大''。}武士千人,执利刃从宫门直排至殿角,后唤邓芝入见。他自胆裂魂飞,彼若开言责以郦食其说齐故事,可效那田广旧例而烹之。看他怕也不怕。}

\setlength{\hangindent}{56pt}{孙权\hspace{30pt}甚好。你就去传旨安排者。}

\setlength{\hangindent}{56pt}{虞翻\hspace{30pt}领旨。}

\setlength{\hangindent}{56pt}{(虞翻下)}

\setlength{\hangindent}{56pt}{孙权\hspace{30pt}你去候殿廷排列齐整,然后{\akai 引}那邓芝来见。}

\setlength{\hangindent}{56pt}{薛综\hspace{30pt}领旨。}

\setlength{\hangindent}{56pt}{(薛综应下)}

\setlength{\hangindent}{56pt}{孙权\hspace{30pt}【{\akai 西皮原板}】建邺王气承天运,黄武称号顺人心。虽云蜀、魏峙如鼎,谁似孤江东群俊英。武略初仗周公瑾,他壮猷深谋谁比伦。曹兵百万犯吾境,势如压卵好惊人。天意存吴东风趁,火炬烧他无几存、若不是关公释曹命,魏家何能鼎足分。近日里全仗小陆逊,年幼智广独超群。刘备妄想报仇恨,连络七百结下营。猇亭用计火攻盛,蜀兵百万尽皆焚。看来孤王有福分,刘备命丧白帝城。眼前刘禅遣使命,孤岂做齐王烹郦生。銮杖此际排齐整,}

\setlength{\hangindent}{56pt}{(吹打,四值殿、四小太监执銮杖左右上,往内抄;孙权上高台,众在高台后;四校尉抬油鼎上,放前中间,虞翻、薛综上)}

\setlength{\hangindent}{56pt}{虞翻、薛综\hspace{30pt}万岁。}

\setlength{\hangindent}{56pt}{孙权\hspace{30pt}【{\akai 西皮快板}】器杖刀剑亮如银。殿角之下设油鼎,汤沸火烈焰腾腾。教那蜀使来认一认,方知东吴不虚名。}

\setlength{\hangindent}{56pt}{孙权\hspace{30pt}来,}

\setlength{\hangindent}{56pt}{孙权\hspace{30pt}【{\akai 西皮快板}】众卿替孤传一令,速宣邓芝来觐至尊。}

\setlength{\hangindent}{56pt}{众\hspace{30pt}宣邓芝上殿。}

\setlength{\hangindent}{56pt}{邓芝\hspace{30pt}({\akai 内})来也。}

\setlength{\hangindent}{56pt}{(邓芝上)}

\setlength{\hangindent}{56pt}{邓芝\hspace{30pt}啊。}

\setlength{\hangindent}{56pt}{邓芝\hspace{30pt}【{\akai 西皮摇板}】我自谓钦承皇王命,到此连合去灭魏人。只见他列杖设油鼎,这其间教我解不明。}

\setlength{\hangindent}{56pt}{邓芝\hspace{30pt}且住,我今奉命而来,原想与他连合伐魏。看他不以礼待,反而列杖设鼎召我。呜哈哈哈$\cdots{}\cdots{}$({\hwfs 笑}{\hwfs 介})}

\setlength{\hangindent}{56pt}{邓芝\hspace{30pt}({\akai 念})孤身到虎穴,气节足凌云。从来试威武,岂能屈儒生。}

\setlength{\hangindent}{56pt}{邓芝\hspace{30pt}【{\akai 西皮快板}】孤身入了虎穴境,志谋气节贯凌云。撩袍端带金殿进,看那吴王怎样行。}

\setlength{\hangindent}{56pt}{邓芝\hspace{30pt}啊大王,邓芝奉揖了。}

\setlength{\hangindent}{56pt}{孙权\hspace{30pt}呃嗯------你既奉使来朝,缘何见孤长揖不拜\footnote{刘曾复先生钞本作``常揖不拜'',此处从《三国演义》原文。}?}

\setlength{\hangindent}{56pt}{邓芝\hspace{30pt}吾乃上国天使,来此小邦,汝不倒履相迎,便为不恭,何必拜汝。}

\setlength{\hangindent}{56pt}{孙权\hspace{30pt}哦,想尔\footnote{刘曾复先生钞本作``尔想'',似欠通。}欲掉三寸之舌,意效郦生说齐王么?但孤非田广者比,汝却作了食其之惨。速赴鼎镬,毋得饶舌。}

\setlength{\hangindent}{56pt}{邓芝\hspace{30pt}哈哈哈$\cdots{}\cdots{}$({\hwfs 笑}{\hwfs 介})}

\setlength{\hangindent}{56pt}{邓芝\hspace{30pt}人人皆言东吴是个名贤之邦,今日一见,竟惧怕一儒生,何其鄙哉。}

\setlength{\hangindent}{56pt}{孙权\hspace{30pt}寡人富有半壁山河,强有兵将数百余万,何惧一匹夫乎?量汝不过为诸葛亮来做说客,欲孤绝魏向蜀,可是么?}

\setlength{\hangindent}{56pt}{邓芝\hspace{30pt}嗯。吾实奉大汉丞相诸葛孔明钧旨,特为汝来陈说利害。请问近日还是向蜀乎,还是向魏乎?}

\setlength{\hangindent}{56pt}{孙权\hspace{30pt}而今曹丕据有中原,磐石之坚,尔那西蜀刘禅焉能与魏抗衡乎?}

\setlength{\hangindent}{56pt}{邓芝\hspace{30pt}唉,鄙哉斯言也。夫魏虽窃据中原\footnote{刘曾复先生钞本作``窃𢫑中原'',``𢫑''同``據''。},实为篡贼苗裔;蜀虽暂处西川,实为大汉帝脉。自上古以来,几见篡贼之后而能长享之理。}

\setlength{\hangindent}{56pt}{邓芝\hspace{30pt}【{\akai 西皮原板}】何故出言直不慎,全无高下枉批评。试问那王莽移汉鼎,他能可\footnote{刘曾复先生钞本疑``为''或``不''字,段公平{\scriptsize 君}注:~``他能可'',系``他可能''颠倒。全句为反诘语气。}遗享与子孙?一朝事败身家尽,惨祸波及家满门。曹丕贼目前似侥幸,我量他不久祸临身。炎汉昭穆承天运,不日旧业重复兴。凡有疑惑不自省,则恐怕前车覆而后车跟。}

\setlength{\hangindent}{56pt}{孙权\hspace{30pt}啊。}

\setlength{\hangindent}{56pt}{孙权\hspace{30pt}【{\akai 西皮摇板}】他言出金石令人信,不似张、苏说连横\footnote{``连横''亦作``连衡''。}。}

\setlength{\hangindent}{56pt}{孙权\hspace{30pt}邓芝。}

\setlength{\hangindent}{56pt}{孙权\hspace{30pt}【{\akai 西皮摇板}】兴亡大势孤且不问,如今之势怎样行。}

\setlength{\hangindent}{56pt}{邓芝\hspace{30pt}【{\akai 西皮快板}】承王问,芝直禀,非敢虚谬论世情:~大王诚算命世主,吾丞相诚算佐命臣。一边仗有山川险,一边仗有江河奫\footnote{奫,水深广的样子。刘曾复先生钞本注``奫(音  氲)''。}。若能连合为唇齿,兼併山河可二分。如或弃蜀归魏贼,我川兵不久顺流征。那时节蜀、魏同临境呃,凭便是铁桶的山河也保不成。}

\setlength{\hangindent}{56pt}{孙权\hspace{30pt}哦$\cdots{}\cdots{}$罢!}

\setlength{\hangindent}{56pt}{孙权\hspace{30pt}【{\akai 西皮摇板}】孤便绝魏从伊请,谁为介绍两通情。}

\setlength{\hangindent}{56pt}{邓芝\hspace{30pt}喏。}

\setlength{\hangindent}{56pt}{邓芝\hspace{30pt}【{\akai 西皮摇板}】王欲使臣臣从命,若还疑臣便烹臣。}

\setlength{\hangindent}{56pt}{邓芝\hspace{30pt}大王。}

\setlength{\hangindent}{56pt}{邓芝\hspace{30pt}【{\akai 西皮摇板}】如不信臣即赴油鼎,}

\setlength{\hangindent}{56pt}{邓芝\hspace{30pt}罢!}

\setlength{\hangindent}{56pt}{(邓芝扑鼎{\hwfs 介})}

\setlength{\hangindent}{56pt}{孙权\hspace{30pt}快快拉住。}

\setlength{\hangindent}{56pt}{(众应)}

\setlength{\hangindent}{56pt}{孙权\hspace{30pt}【{\akai 西皮摇板}】先生自是有信人。}

\setlength{\hangindent}{56pt}{孙权\hspace{30pt}将油鼎抬下,请邓先生上殿相见。}

\setlength{\hangindent}{56pt}{(众应;吹打,孙权下位,吹打住;校尉抬鼎下)}

\setlength{\hangindent}{56pt}{邓芝\hspace{30pt}大王。}

\setlength{\hangindent}{56pt}{孙权\hspace{30pt}先生。}

\setlength{\hangindent}{56pt}{孙权\hspace{30pt}【{\akai 西皮摇板}】你浩然之气令人敬,莫哂小量待高人。}

\setlength{\hangindent}{56pt}{邓芝\hspace{30pt}哎呀,大王过谦了。}

\setlength{\hangindent}{56pt}{孙权\hspace{30pt}先生一番金石之言,使孤顿开茅塞,从此吴、蜀连合,协力伐魏。}

\setlength{\hangindent}{56pt}{邓芝\hspace{30pt}({\akai 念})承蒙金诺予,看太平有待。}

\setlength{\hangindent}{56pt}{孙权\hspace{30pt}哈哈哈$\cdots{}\cdots{}$({\hwfs 笑介})}

\setlength{\hangindent}{56pt}{孙权\hspace{30pt}吩咐摆酒,与先生接谈。}

\setlength{\hangindent}{56pt}{(众应)}

\setlength{\hangindent}{56pt}{邓芝\hspace{30pt}多谢大王。}

\setlength{\hangindent}{56pt}{孙权\hspace{30pt}先生:~({\akai 念})从此蜀、吴连合定,孤与先生叙主宾。}

\setlength{\hangindent}{56pt}{(孙权拉邓芝下,众下)}

}
