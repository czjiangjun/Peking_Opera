\newpage\hspace{30pt}~

{%

\subsubsection{\large\hei {法门寺~{\small 之}~

赵廉}}

{\vspace{3pt}{\centerline{{[}{\hei 第一场}{]}}}\vspace{5pt}}

{臣不敢,赵廉。}\hspace{20pt}~

{有罪不敢抬头。}\hspace{20pt}~

{谢千岁!}\hspace{30pt}~

\setlength{\hangindent}{56pt}{【{\akai 西皮散板}】小傅朋他本是啊杀人凶犯,}

{千岁!}\hspace{40pt}~

\setlength{\hangindent}{56pt}{【{\akai 西皮散板}】臣问他}\footnote{夏行涛君建议可作``臣按他''。}{口供词件件招全。在公堂未动刑他自己招认,因此上臣将他拿问在监。}

{公公!}\hspace{40pt}~

{惭愧。}\hspace{40pt}~

{是(是是),二甲进士出身,焉有不识字的道理?}

{是是是。}\hspace{30pt}~

{``告状民女宋氏巧娇$\cdots{}\cdots{}$''}

{啊?这``巧娇''二字,好像在哪里会过({\akai 或}:~见过),怎么一时想它不起$\cdots{}\cdots{}$}

{哦,你就是宋国士之女,名唤巧娇的么?}

{你为何告此刁状?}\hspace{20pt}~

{先前为何不告?}\hspace{20pt}~

{如今呢?}\hspace{30pt}~

{着啊!}\hspace{40pt}~

{哦------}\hspace{20pt}~

\setlength{\hangindent}{56pt}{【{\akai 西皮散板}】谁知道小刘彪是杀人的凶犯呐,却原来这内中有许多牵连。在庙堂恕为臣呐才学\textless{}\!{\bfseries\akai 哭头}\!\textgreater{}浅,千岁爷呀!}

\setlength{\hangindent}{56pt}{【{\akai 西皮散板}】望千岁开大恩限臣三天。}

{谢千岁!}\hspace{30pt}~

{\vspace{3pt}{\centerline{{[}{\hei 第二场}{]}}}\vspace{5pt}}

{你们都来了?}\hspace{30pt}~

{将刘媒婆带好,带马打道孙家庄去者。}

{\vspace{3pt}{\centerline{{[}{\hei 第三场}{]}}}\vspace{5pt}}

{锁了!}\hspace{40pt}~

{带了!}\hspace{40pt}~

{两厢搜来。}\hspace{30pt}~

{钢刀入库,绣鞋放下。}\hspace{10pt}~

{带刘媒婆!}\hspace{30pt}~

{勾奸卖奸,可是此物?}\hspace{10pt}~

{(带)下去。}\hspace{30pt}~

{带刘彪!}\hspace{30pt}~

{你在大街怎样讹诈傅朋,从实讲来。}

{下去。}\hspace{40pt}~

{带刘公道!}\hspace{30pt}~

{刘彪在大街讹诈傅朋,可有你解劝过来({\akai 或}:~你可曾解劝过来)?}

{下去。}\hspace{40pt}~

{带刘彪!}\hspace{30pt}~

{讹诈是实。孙家庄一刀连伤二命,定(然)是你这个奴才所做的了。}

{(来,)打!}\hspace{30pt}~

{怎么样?}\hspace{30pt}~

{抓了回来!}\hspace{30pt}~

{一刀连伤二命,还说初犯?!}

{我来问你,男尸有头,女尸无头,这人头往哪里去了?}

{带刘公道!}\hspace{30pt}~

{(唗,)身当乡约({\akai 或}:~身作乡约),隐藏人头不报,该当何罪?}

{来,打!}\hspace{30pt}~

{打!}\hspace{40pt}~

{打道硃砂井。}\hspace{30pt}~

{打捞人头。}\hspace{30pt}~

{人头有了,就好落案了。}\hspace{10pt}~

{哦------还有死尸一口!}

{快快地打捞上来!}\hspace{20pt}~

{上前验来。}\hspace{30pt}~

{带刘公道!}\hspace{30pt}~

{这井内的死尸是哪里来的?}\hspace{10pt}~

{来,打!}\hspace{30pt}~

{他叫什么名字?}\hspace{20pt}~

{何事喧哗?({\akai 或}:~什么喧哗?)}

{(这)死尸呢?}\hspace{20pt}~

{唉,本县的对头到了({\akai 或}:~唉,本县的对头来了)!}

{啊,宋先生------}\hspace{10pt}~

{抱尸痛哭,敢是相认?}\hspace{10pt}~

{啊?!既不相认,前来搅乱尸场?!}

{左右,轰了下去!}\hspace{20pt}~

{唤他回来。}\hspace{30pt}~

{无用的奴才!}\hspace{30pt}~

{宋先生(请转),宋先生请转$\cdots{}\cdots{}$}

{好奴才!\hspace{40pt}~

({\akai 或}:~好狗才!)}

\setlength{\hangindent}{56pt}{【{\akai 西皮散板}】骂声公道老禽兽,身作乡约隐人头。硃砂井边下毒手哇,活活打死呃你这老蠢牛。}

{怎样打不得?}\hspace{30pt}~

{依你之见?}\hspace{30pt}~

{好,回衙有赏!}\hspace{20pt}~

{将一干人犯带好({\akai 或}:~带妥),与爷带马!}

\setlength{\hangindent}{56pt}{【{\akai 西皮慢板}】郿坞县在马上心神不定,这几天为人犯哪得安宁。劝世人休为官务农为本,可怜我七品的官不如黎民。实指望做清官【{\footnotesize 转}{\akai 西皮二六}】高升一品,又谁知孙家庄起下祸根。孙玉姣习针黹(在)门前站定,有傅朋起下了爱慕之情。假意儿买雄鸡来把话论,就有个刘媒婆老不正经。他二人婚姻事自有媒证,何用你诓绣鞋在暗地里勾情?只骂得老乞婆羞愧难忍,转面来骂刘彪大胆畜生。黑夜里你一刀连伤二命,将人头胡乱丢移祸旁人。刘公道在衙门充当里正({\akai 或}:~在衙门身为里正),你为何见人头不打报呈?你三人}\footnote{段公平君注:~{吴小如先生曾撰文称``你三人''是余派准词,孟小冬、李少春皆如此。李舒先生遗作《涉艺所得》载刘老唱词,亦是``你三人''。}}{问典刑({\akai 或}:~你三人问清楚)休来怨恨,这就是自作自受、王法森严难以徇情。教衙役将人犯一齐带定,}

\setlength{\hangindent}{56pt}{【{\akai 西皮摇板}】放大胆闯虎穴去见上呃人。}

{\vspace{3pt}{\centerline{{[}{\hei 第四场}{]}}}\vspace{5pt}}

{何事?}\hspace{40pt}~

{与我打!}\hspace{30pt}~

{这$\cdots{}\cdots{}$}

{也罢,将老爷的马与他乘骑。}

{(也)只好是步行的了哇。}\hspace{10pt}~

{呃------}\hspace{20pt}~

\setlength{\hangindent}{56pt}{【{\akai 西皮快板}】刘公道做事真胆大,身作乡约犯王法。打死兴儿反}讹诈,绝了那宋国士后代根芽。此一番去见千岁爷的驾,老无才准备下钢{呃}刀把你的头来杀。

{\vspace{3pt}{\centerline{{[}{\hei 第五场}{]}}}\vspace{5pt}}

{何事?}\hspace{40pt}~

{可曾递上?}\hspace{30pt}~

{呃,无用的奴才!}\hspace{20pt}~

{公公!}\hspace{40pt}~

{(来了。)}\hspace{30pt}~

{哦,带齐了。({\akai 或}:~哦,多谢公公。)}

{哦,是是是,有劳公公。}\hspace{10pt}~

{呃,有劳公公!}\hspace{20pt}~

{不是这样的投法,(还)要怎样投法呢?}

{呃$\cdots{}\cdots{}$}

{这里的({\akai 或}:~这里边)的银票呢?}

{哼,无用的奴才,还不取了过来!({\akai 或}:~呃,拿了过来!)}

{啊,公公!}\hspace{30pt}~

{呃,公公收下。}\hspace{20pt}~

{莫非嫌轻。}\hspace{30pt}~

{呃$\cdots{}\cdots{}$不敢不敢。}

{呃,有劳公公。}\hspace{20pt}~

{多谢公公。}\hspace{30pt}~

{呵,有劳了,有劳了。}\hspace{10pt}~

{啊,公公$\cdots{}\cdots{}$}

{呃,有劳了,有劳了。}\hspace{10pt}~

{是是是。}\hspace{30pt}~

{参见千岁!}\hspace{30pt}~

{千岁容禀:~}\hspace{30pt}~

\setlength{\hangindent}{56pt}{【{\akai 西皮摇板}】一干人犯俱带妥,望求千岁作定呐夺。}

{谢千岁!}\hspace{30pt}~

{千岁在上({\akai 或}:~千岁在此),哪有为臣的座位?}

{哦,是是是,如此谢坐。}\hspace{10pt}~

{啊,公公请坐。}\hspace{20pt}~

{全仗千岁!}\hspace{30pt}~

{(遵命。)}\hspace{30pt}~

{带刘彪。}\hspace{30pt}~

{一刀连伤二命,按律凌迟。千岁开恩,问他斩罪。}

{千岁开恩。}\hspace{30pt}~

{带刘公道。}\hspace{30pt}~

{身当乡约,隐藏人头不报,打伤人命,按律当斩。千岁开恩,问他绞罪。}

{千岁恩德。}\hspace{30pt}~

{带刘媒婆。}\hspace{30pt}~

{且慢千岁,有道是``子大不由母''啊!}

{千岁开恩。}\hspace{30pt}~

{带傅朋。}\hspace{30pt}~

{带孙玉姣。}\hspace{30pt}~

{她乃黄花幼女,可以见得。}\hspace{10pt}~

{带宋巧娇。}\hspace{30pt}~

{呃,有来头。}\hspace{30pt}~

{大大的有来头。}\hspace{20pt}~

{啊!}\hspace{40pt}~

{多谢千岁。}\hspace{30pt}~

{她么$\cdots{}\cdots{}$呃,也见得。}

{见得,见得,见得$\cdots{}\cdots{}$}

{(比作何来?)}\hspace{20pt}~

{谢千岁!}\hspace{30pt}~

{(多谢公公。)}\hspace{20pt}~

{带马!}\hspace{40pt}~
