\newpage
\subsubsection{\large\hei {搜孤救孤~{\small 之}~程婴、公孙杵臼}}
\addcontentsline{toc}{subsection}{\hei 搜孤救孤~\small{之}~程婴、公孙杵臼}

\hangafter=1
\setlength{\parindent}{0pt}{
{\centerline{{[}{\hei 第一场}{]}}}\vspace{5pt}
公孙杵臼\hspace{10pt}~{[}{\akai 引子}{]}赵、屠结冤仇,恨奸贼,何日罢休。

公孙杵臼\hspace{10pt}~({\akai 念})屠贼专权乱朝纲,欺君藐法似虎狼。可叹忠良满门丧,铁石人儿也悲伤。
\setlength{\hangindent}{56pt}{
	公孙杵臼\hspace{10pt} 老汉公孙杵臼,昔年曾在赵家以为门客。可恨屠贼戮杀\footnote{夏行涛{\scriptsize 君}建议``戮杀''均作``诬杀''。}赵家三百余口,只剩庄姬一人逃进宫去,生下孤儿。屠贼闻知,带剑进宫,搜孤不出。如今出了赏格在外,十日之内,有人献出孤儿便罢,倘若无人献出,要将晋国中与孤儿同庚者,俱要斩尽杀绝。天呐,天!眼见孤儿无救了。}

\setlength{\hangindent}{56pt}{
公孙杵臼\hspace{10pt}【{\akai 二黄原板}】恼恨屠贼心太狠,戮杀赵家一满门。眼见得忠良无有救应,大事还要问程婴。 }

\setlength{\hangindent}{56pt}{
程婴\hspace{30pt}【{\akai 二黄散板}】屠贼做事心太狠,三百余口赴幽冥。 }

程婴\hspace{30pt}公孙兄在家么?

公孙杵臼\hspace{10pt}是哪一位?

程婴\hspace{30pt}小弟来了。

公孙杵臼\hspace{10pt}哦,贤弟来了。请到里面。

程婴\hspace{30pt}请。

公孙杵臼\hspace{10pt}请坐。

程婴\hspace{30pt}有座。

程婴\hspace{30pt}唉!

公孙杵臼\hspace{10pt}贤弟为何长叹?

程婴\hspace{30pt}这晋国之中又出了一桩奇事,你还不晓么?

公孙杵臼\hspace{10pt}哦,什么奇事,愚兄不知呀。

程婴\hspace{30pt}可恨屠贼带剑进宫,搜孤不出,如今又起了狠毒之心呐。

公孙杵臼\hspace{10pt}什么狠毒之心?

\setlength{\hangindent}{56pt}{
程婴\hspace{30pt}那贼出了赏格在外,十日之内,有人献出孤儿,赏赐千金。不然,要将晋国中与孤儿同庚者,俱要斩尽杀绝。}

公孙杵臼\hspace{10pt}呃,此事愚兄早已知晓。贤弟,你的来意如何?

程婴\hspace{30pt}弟今此来,与兄商议这救孤之策。

公孙杵臼\hspace{10pt}愚兄我是忙中无计。

程婴\hspace{30pt}你也无计------唉!弟倒有一两全之计。或能救得孤儿。

公孙杵臼\hspace{10pt}何谓两全之计?

程婴\hspace{30pt}若有一人舍得一命,一人舍得一子,或能救得孤儿。

公孙杵臼\hspace{10pt}贤弟,你来看,愚兄偌大年纪,情愿舍命。但不知何人舍子?

\setlength{\hangindent}{56pt}{
	程婴\hspace{30pt} 呃,弟新生一子,与孤儿诞期相近。就将我儿藏至你处,待弟前去出首。就说你隐藏孤儿不报,那屠贼闻知,必定带领人役前来搜寻。哎呀,那时只怕你的性命难保哇。}

公孙杵臼\hspace{10pt}呃,愚兄方才言过,偌大年纪,死何足惜。呃,你只管地抱来就是。

程婴\hspace{30pt}话虽如此,你那弟妇她还不曾知道呢。

公孙杵臼\hspace{10pt}哎呀,弟妹不允,可也是枉然呐。

程婴\hspace{30pt}不妨不妨,你那弟妇虽是女流,颇知大义,不能不允呐。

公孙杵臼\hspace{10pt}话虽如此,你且先行,愚兄随后就到。

程婴\hspace{30pt}告辞了。

\setlength{\hangindent}{56pt}{程婴\hspace{30pt}【{\akai 二黄散板}】你我二人把计定,立孤({\akai 或}:~抚孤)的事儿我担承。 }

\setlength{\hangindent}{56pt}{公孙杵臼\hspace{10pt}【{\akai 二黄散板}】但愿救得孤儿命,不绝赵家后代根。 }

\vspace{3pt}{\centerline{{[}{\hei 第二场}{]}}}\vspace{5pt}

(程妻\hspace{25pt}({\akai 念})仗义救孤身,妻随夫志行。)

程婴\hspace{30pt}({\akai 念})大事安排定,劝妻舍亲生。

程婴\hspace{30pt}娘子。

程婴\hspace{30pt}唉!事到如今还讲什么天理报应({\akai 或}:~上苍报应)。

程婴\hspace{30pt}屠贼进宫,搜孤不出,又起了狠毒之心呐。

\setlength{\hangindent}{56pt}{程婴\hspace{30pt}那贼如今出了赏格在外,十日之内,有人献出孤儿,赏赐千金。不然,要将晋国中与孤儿同庚者,俱要斩尽杀绝。}

程婴\hspace{30pt}我与公孙老爷定下两全之计,可以救得孤儿。

程婴\hspace{30pt}若有一人舍得一命,一人舍得一子,就可以救得孤儿。

程婴\hspace{30pt}就是那公孙老爷他情愿舍命呐。

\setlength{\hangindent}{56pt}{程婴\hspace{30pt}舍子么$\cdots{}\cdots{}$唉,娘子,想你我夫妻曾受赵相厚恩,焉能坐观成败。我意欲将你我的儿子与孤儿调换下来,抚养成人。一来接得赵家宗嗣\footnote{``宗嗣''为宗族继承人、子孙后代之意。有人建议用``宗祀''。``宗祀''是对祖宗的祭祀之意。本文稿中其他剧目中亦同。},二来日后也好报仇雪恨。}

程婴\hspace{30pt}啊,娘子,你看这一条计策({\akai 或}:~你看此计)可好啊?

程婴\hspace{30pt}唉,娘子啊!

\setlength{\hangindent}{56pt}{程婴\hspace{30pt}【{\akai 二黄原板}】娘子不必太烈性,卑人言来你试听: 赵、屠二家有仇恨,三百余口命赴幽冥。我与那公孙杵臼把计定,他舍命来你我舍亲生。舍子搭救忠良后,老天爷岂绝我的后代根。你今舍了亲生子,来年必定降麒麟。 }

\setlength{\hangindent}{56pt}{程婴\hspace{30pt}【{\akai 二黄原板}】千言万语她不肯,不舍娇儿难救孤身。无奈何我只得双膝跪, }

\setlength{\hangindent}{56pt}{程婴\hspace{30pt}【{\akai 二黄摇板}】哀求娘子舍亲生。 }

\setlength{\hangindent}{56pt}{(程妻\hspace{25pt}【{\akai 二黄摇板}】你要跪来只管跪,要我舍子万不能。) }

\setlength{\hangindent}{56pt}{程婴\hspace{30pt}【{\akai 二黄散板}】人道妇人心肠狠呐,狠毒毒不过妇人的心。 }

\setlength{\hangindent}{56pt}{(程妻\hspace{25pt}【{\akai 二黄摇板}】$\cdots{}\cdots{}$不食子,你比虎狼狠十分。) }

\setlength{\hangindent}{56pt}{程婴\hspace{30pt}【{\akai 二黄散板}】不如程婴死了罢, }

\setlength{\hangindent}{56pt}{(程妻\hspace{25pt}【{\akai 二黄摇板}】或生或死一同行。) }

\setlength{\hangindent}{56pt}{程婴\hspace{30pt}【{\akai 二黄散板}】手执钢刀要你命, }

\setlength{\hangindent}{56pt}{(程妻\hspace{25pt}【{\akai 二黄摇板}】用手关上绣房门。) }

\setlength{\hangindent}{56pt}{公孙杵臼\hspace{10pt}【{\akai 二黄摇板}】程婴与我把计定,未知他心似我心。 }

公孙杵臼\hspace{10pt}贤弟,愚兄来了。

程婴\hspace{30pt}请坐。

程婴\hspace{30pt}呃,呃,呃,这$\cdots{}\cdots{}$这,这边坐。

公孙杵臼\hspace{10pt}呃,俱是一样啊。

公孙杵臼\hspace{10pt}啊,贤弟,弟妹可曾应允呐?

程婴\hspace{30pt}那贱人她执意地不允呐。

公孙杵臼\hspace{10pt}呃,先前言过,弟妹十分贤德,颇知大义,呃,怎么如今她不允起来了?

程婴\hspace{30pt}呃呃呃,是,是她不允呐!

公孙杵臼\hspace{10pt}呃,不必如此,请将出来,愚兄良言相劝。

程婴\hspace{30pt}遵命。

程婴\hspace{30pt}贱人走出来!

程婴\hspace{30pt}公孙老爷有话讲啊。

公孙杵臼\hspace{10pt}弟妹少礼,请坐。

公孙杵臼\hspace{10pt}劝弟妹听了丈夫之言,舍了亲生之子。

公孙杵臼\hspace{10pt}({\akai 念})弟妹搭救孤儿命,留得美名万古存。

\setlength{\hangindent}{56pt}{
公孙杵臼\hspace{10pt}【{\akai 二黄原板}】人有善念天有应,莫把阴骘当浮云。弟妹搭救忠良后,赵家代代不忘恩。 }

\setlength{\hangindent}{56pt}{ 公孙杵臼\hspace{10pt}【{\akai 二黄摇板}】老朽薄面情要准, }

\setlength{\hangindent}{56pt}{ 程婴\hspace{30pt}【{\akai 二黄散板}】看起来你是个不贤妇哇, }

\setlength{\hangindent}{56pt}{ 程婴\hspace{30pt}【{\akai 二黄散板}】手持钢刀项上刎, }

\setlength{\hangindent}{56pt}{ 公孙杵臼\hspace{10pt}【{\akai 二黄散板}】贤弟息怒且消停。 }

\setlength{\hangindent}{56pt}{ 公孙杵臼\hspace{10pt}【{\akai 二黄散板}】走向前来良言劝,死了丈夫靠何人。 }

\setlength{\hangindent}{56pt}{ 公孙杵臼\hspace{10pt}【{\akai 二黄散板}】无奈何我只得屈膝跪, }

程婴\hspace{30pt}不要跪。

公孙杵臼\hspace{10pt}你也跪下。

\setlength{\hangindent}{56pt}{ 公孙杵臼\hspace{10pt}【{\akai 二黄散板}】哀求弟妹救孤生。 }

\setlength{\hangindent}{56pt}{ 公孙杵臼\hspace{10pt}【{\akai 二黄散板}】弟妹舍了亲生子,列国之中标美名。 }

\setlength{\hangindent}{56pt}{ 程婴\hspace{30pt}【{\akai 二黄散板}】多谢娘子开了恩,母子快快两离分。 }

\vspace{3pt}{\centerline{{[}{\hei 第三场}{]}}}\vspace{5pt}
(\textless{}\!{\bfseries\akai 水底鱼}\!\textgreater{},程婴{\hwfs 上})

程婴\hspace{30pt}唉!

程婴\hspace{30pt}击鼓人告进。

程婴\hspace{30pt}叩见大人。

程婴\hspace{30pt}大人前番搜孤可曾搜出?

程婴\hspace{30pt}孤儿在------

程婴\hspace{30pt}现在首阳山公孙杵臼的家中。

\setlength{\hangindent}{56pt}{程婴\hspace{30pt}小人与他昔年皆为赵相门客,又有八拜之交。只因他隐藏孤儿不报,是小人劝他献出,不想他是执意地不肯,反将小人辱骂。小人本不愿出首,因见大人有言在先,知情不举是罪加一等,为此小人不敢隐瞒,特地前来禀明大人。}

程婴\hspace{30pt}小人名叫程婴。

程婴\hspace{30pt}有。

程婴\hspace{30pt}谢大人。

公孙杵臼\hspace{10pt}小人隐藏孤儿,何人得见?

公孙杵臼\hspace{10pt}哎呀大人呐,此人名叫程婴,与小人旧有仇恨,乃是诬告小人!

公孙杵臼\hspace{10pt}诬告小人!

\setlength{\hangindent}{56pt}{公孙杵臼\hspace{10pt}【{\akai 二黄散板}】白虎大堂一声禀,大人息怒听详情。程婴与我有仇恨,把什么孤儿予大人。 }

\setlength{\hangindent}{56pt}{公孙杵臼\hspace{10pt}【{\akai 二黄散板}】纵然打死我难招承。 }

(屠岸贾\hspace{20pt}程婴。)

\setlength{\hangindent}{56pt}{(屠岸贾\hspace{20pt}【{\akai 二黄散板}】$\cdots{}\cdots{}$赐你鞭一根。一边打来一边问,看他招承不招承。) }

\setlength{\hangindent}{56pt}{程婴\hspace{30pt}【{\akai 二黄导板}】白虎大堂奉了命, }

(屠岸贾\hspace{20pt}程婴!)

程婴\hspace{30pt}有。

(屠岸贾\hspace{20pt}与我着实地打!)

\setlength{\hangindent}{56pt}{程婴\hspace{30pt}【{\akai 回龙}】都只为救孤儿舍亲生,连累了年迈苍苍受苦刑,眼见得两离分。 }

\setlength{\hangindent}{56pt}{程婴\hspace{30pt}【{\akai 二黄原板}】我与他人定巧计,到如今连累他受苦刑。开言便把公孙兄问,小弟言来你试听: 你若是再三地不肯招认\footnote{据柴俊为老师介绍,余叔岩的词句是``你若是在丹墀不肯招认''。},大人的王法不徇情。手执皮鞭将你打, }

\setlength{\hangindent}{56pt}{程婴\hspace{30pt}【{\akai 二黄散板}】你$\cdots{}\cdots{}$你,你切莫要胡言攀扯({\akai 或}:~连累)我好人。 }

公孙杵臼\hspace{10pt}贼。

\setlength{\hangindent}{56pt}{公孙杵臼\hspace{10pt}【{\akai 二黄散板}】指着程婴骂高声,苦苦害我为何情。我今一死何足论,你留得骂名列国闻。 }

\setlength{\hangindent}{56pt}{程婴\hspace{30pt}【{\akai 二黄散板}】老儿执意不招认,急往首阳去搜寻({\akai 或}:~大人首阳去搜寻)。 }

公孙杵臼\hspace{10pt}屠贼。

\setlength{\hangindent}{56pt}{公孙杵臼\hspace{10pt}【{\akai 二黄散板}】奸贼做事心太狠,苦害忠良为何情。我今与你拼性命, }

\setlength{\hangindent}{56pt}{ 程婴\hspace{30pt}【{\akai 二黄散板}】(这是你)飞蛾投火自烧身。 }

程婴\hspace{30pt}小人讨祭。

(屠岸贾\hspace{15pt}你为何祭他?)

程婴\hspace{30pt}小人与他有八拜之交,若不祭奠于他,旁人道小人不义了。

程婴\hspace{30pt}谢大人。

\setlength{\hangindent}{56pt}{程婴\hspace{30pt}【{\akai 二黄散板}】虽然杯酒寻常饮,略表当年结拜情。 }

\vspace{3pt}{\centerline{{[}{\hei 第四场}{]}}}\vspace{5pt}
公孙杵臼\hspace{10pt}~({\akai 内})【{\akai 二黄导板}】一片好心反成恨,

\setlength{\hangindent}{56pt}{公孙杵臼\hspace{10pt}【{\akai 回龙}】年迈苍苍血染身。 }

\setlength{\hangindent}{56pt}{公孙杵臼\hspace{10pt}【{\akai 二黄原板}】我与他人把计定,一人舍命一人舍亲生。含悲忍泪法场进, }

\setlength{\hangindent}{56pt}{公孙杵臼\hspace{10pt}【{\akai 二黄散板}】咬定牙关等时辰。 }

\setlength{\hangindent}{56pt}{程婴\hspace{30pt} \textless{}\!{\bfseries\akai 撞金钟}\!\textgreater{}【{\akai 二黄摇板}】迈步儿来在法场中,只见孤儿与公孙。}

\setlength{\hangindent}{56pt}{程婴\hspace{30pt} \textless{}\!{\bfseries\akai 叫头}\!\textgreater{}公孙兄,赵公子,你二人死在九泉(之下),休怨我程婴。({\hwfs 哭介})
}

\setlength{\hangindent}{56pt}{程婴\hspace{30pt}【{\akai 二黄碰板原板}】躬身下拜礼恭敬,眼望孤儿泪淋淋。法场上看的人({\akai 或}:~法场上人人)都来叫骂,一个个骂的是我程婴,是一个无义的人。贪享荣华受富贵,断送了忠良人的后代根。这是我好意反成恶意,满怀心腹事向谁云。 }

\setlength{\hangindent}{56pt}{公孙杵臼\hspace{10pt}【{\akai 二黄原板}】法场上绑得我昏迷不醒,抬头只见小程婴。去掉好言换恶语,高声叫骂小程婴。我今一死不要紧,留得美名万古闻。 }

\setlength{\hangindent}{56pt}{程婴\hspace{30pt}【{\akai 二黄原板}】公孙兄说话需谨慎,泄漏了机关大事难成。先前抚孤是你我,到如今知心({\akai 或}:~同心)还有谁人。你为忠良舍了性命,可叹我程婴绝了后根。无奈何烧钱把酒奠,我那亲------\textless{}\!{\bfseries\akai 哭头}\!\textgreater{}我$\cdots{}\cdots{}$我,我的儿啊! }


程婴\hspace{30pt}公孙兄啊。

\setlength{\hangindent}{56pt}{程婴\hspace{30pt}【{\akai 二黄散板}】但愿你灵魂早超生。 }

程婴\hspace{30pt}祭奠已毕。

程婴\hspace{30pt}啊,大$\cdots{}\cdots{}$大人。

程婴\hspace{30pt} 小人先前言过,与那公孙杵臼有八拜之交,如今见他身首异处,思想前情,故而落泪呀。

程婴\hspace{30pt}且慢,小人不愿领赏。

\setlength{\hangindent}{56pt}{程婴\hspace{30pt} 小人新生一子,与孤儿诞期相近。今将孤儿出首,又恐旁人加害我父子,还望大人另外相照。}

程婴\hspace{30pt}谢大人。

\setlength{\hangindent}{56pt}{程婴\hspace{30pt}【{\akai 二黄散板}】两全之计全孤命,再把立孤的巧计生。({\akai 或}:~背转身来笑吟吟,奸贼中了我的巧计生。) }

(程婴{\hwfs 大边下},{\hwfs 小边上})

\setlength{\hangindent}{56pt}{
程婴\hspace{30pt}【{\akai 二黄散板}】心中大事安排定,孤儿长大杀仇人。({\akai 或}:~怀抱孤儿法场进,但愿你长大杀仇人。) }

程婴\hspace{30pt}有。

程婴\hspace{30pt}多谢大人!
}
