\newpage
\phantomsection %实现目录的正确跳转
\section*{\large\hei 桑园会~{\small 之}~秋胡~\protect\footnote{整理过程中参考了徐芃{\scriptsize 君}的硕士研究生学位论文\upcite{Xu-MasterThesis}载的刘曾复先生《桑园会》本词句和吴焕老师整理的剧本(经刘曾复先生审订)。}}
\addcontentsline{toc}{section}{\hei 桑园会~{\small 之}~秋胡}

\hangafter=1                   %2. 设置从第1⾏之后开始悬挂缩进  %
\setlength{\parindent}{0pt}{
{\centerline{{[}{\hei 第一场}{]}}}\vspace{5pt}

({\hwfs 四}青袍{\hwfs 引}秋胡{\hwfs 上})

{[}{\akai {\akai 引}子}{]}归心似箭,辞王驾,转回家园。

({\akai 念})抛妻别母二十春,千里迢迢无信音。人生若得全忠孝,臣报君恩子奉亲。

下官秋胡,鲁国人也。在楚为官,官居上大夫之职。只因我离家日久,不知老母妻室怎生度日。为此辞王别驾,回家探望。多蒙楚王赐我黄金人役,来此离家不远,我不免改换行装,以免惊动乡邻。

左右,看衣改换。

尔等远远跟随。

与爷带马。

\setlength{\hangindent}{56pt}{【{\akai 西皮原板}】想当年悲切切楚国投奔,今日里笑吟吟荣转家门。带黄金喜孜孜萱堂侍奉,安慰了娇滴滴我那少年夫人。 }

{\vspace{3pt}{\centerline{{[}{\hei 第二场}{]}}}\vspace{5pt}}

({\akai 内})马来!

\setlength{\hangindent}{56pt}{【{\akai 西皮快板}】秋胡打马奔家乡,路上行人马蹄忙。勒住丝缰【{\footnotesize 转}{\akai 西皮摇板}】来观望,}

\setlength{\hangindent}{56pt}{【{\akai 西皮快板}】见一个妇人手攀桑。前影好似罗敷女,后影儿又像我的妻房。本当上前把妻认,错认了民妻罪非常。\footnote{刘曾复先生曾介绍梆子剧中``秋胡打马''一段的唱法词句,兹照录如下:~\\{\footnotesize ``秋胡打马离山岗,那山上有个刘大王。那大王不把乡里念,他将我连人带马赶下山。回朝难见楚王面,因此上抛官弃印回家转。正行走,用目观,见一位大嫂在桑园。前影看也看不见,后影儿好似妻银环。有心上前把妻认,错认民妻礼不端。''}} }

哎呀且住!看那旁采桑妇人,好像我妻罗敷模样,怎奈我离家日久,不敢冒认。

待我下马问来。

啊,大嫂请了!

并非失迷路途,我乃找名问姓的。

请问大嫂,此处有一秋胡,大嫂可晓得?

那秋胡与我同朝为官,又有八拜之交,托我带来万金家书,故而动问。

我那秋兄言道,这书信么,要面交本人。

原书带回。

大嫂听了!

\setlength{\hangindent}{56pt}{【{\akai 西皮快板}】站立在桑田把话讲,尊一声大嫂听端详:~家住鲁国古田桑,姓秋名胡字高翔。他父名叫秋楚望,二十年前早已亡。老母柯氏六旬上,白发苍苍年迈的娘。娶妻本是罗敷女,苦持贞节守空房。临行送在了阳关上,叮咛的言辞记在心旁。但愿得早登龙虎榜,即刻里修书转还乡。此乃是秋兄对我讲,并无有虚言哄娘行。 }

大嫂要看书信,但不知你是他家何人?

哦,原来是秋大嫂,失敬了。

哦。

\setlength{\hangindent}{56pt}{【{\akai 西皮摇板}】听罢言来喜心上,果然是罗敷来采桑。 }

哎呀且住。想我秋胡离家二十余载,也不知她的光景如何?呃$\cdots{}\cdots{}$我自有道理呀。

啊大嫂,卑人有几句言语,要对大嫂言讲啊。

大嫂听了!

\setlength{\hangindent}{56pt}{【{\akai 西皮导板}】秋胡他把良心丧, }

唉!大嫂哇!

\setlength{\hangindent}{56pt}{【{\akai 西皮原板}】他在那楚国配了鸾凰。我劝他回家他不往,丢下了大嫂守空房。你好比鲜花【{\footnotesize 转}{\akai 西皮二六}】空开放,又好比明珠土内藏。你好比皓月当空明朗朗,}

\setlength{\hangindent}{56pt}{【{\akai 西皮快板}】晴天无日少阴阳。趁此\footnote{徐芃{\scriptsize 君}硕士研究生学位论文载刘曾复先生《桑园会》本作``乘此''。}桑田无人往,学一个巫山神女会襄王。 }

\setlength{\hangindent}{56pt}{【{\akai 西皮快板}】大嫂不是这样讲,细听卑人说比方:~枯木逢春花又放,人生几度好时光。千里的姻缘从天降,陌上相逢非寻常。阳关大道无人往,学一个织女配牛郎。 }

哦!

\setlength{\hangindent}{56pt}{【{\akai 西皮摇板}】我二人俱是空言讲,平地怎能托空梁。取出黄金有数两,假意求作凤鸾凰。 }

啊,大嫂。常言说得好({\akai 或}:~道得好):~``力田辜负青年少,采桑何如嫁富郎''。我这里有马蹄金一锭,赠与大嫂,来来来,看金呐。

哦,在哪里?哦$\cdots{}\cdots{}$

呵哈哈哈$\cdots{}\cdots{}$({\hwfs 笑介})

\setlength{\hangindent}{56pt}{【{\akai 西皮摇板}】黄金不顾回家往,贞节烈女世无双。拾起黄金忙赶上({\akai 或}:~把马上),回到家去奉高堂。 }

{\vspace{3pt}{\centerline{{[}{\hei 第三场}{]}}}\vspace{5pt}}

\setlength{\hangindent}{56pt}{【{\akai 西皮快板}】村庄以外下了马,杨柳深处是我家\footnote{徐芃{\scriptsize 君}硕士研究生学位论文载刘曾复先生《桑园会》本作``吾家''。}。来在门前用目洒, }

(秋母\hspace{30pt}哦$\cdots{}\cdots{}$)

\setlength{\hangindent}{56pt}{【{\akai 西皮快板}】老娘亲缘何怒声哗\footnote{徐芃{\scriptsize 君}硕士研究生学位论文载刘曾复先生《桑园会》本作``怒气哗''。}。 }

\setlength{\hangindent}{56pt}{【{\akai 西皮快板}】进得门,忙跪下,\footnote{徐芃{\scriptsize 君}硕士研究生学位论文载刘曾复先生《桑园会》本作``进得门来忙跪下''。}儿是秋胡转还家。 }

\setlength{\hangindent}{56pt}{【{\akai 西皮快板}】打罢春,又转夏,日月轮回催岁华。少年子弟江湖老,娘的青丝也转白发。 }

母亲请上,待孩儿大礼参拜。

久离膝下,少奉甘旨。母亲恕罪。

谢母亲。

谢座。

乃上大夫之职。

当谢天地。

谢座。

啊母亲,孩儿回家半日,为何不见娘的儿媳,她往哪里去了?

呃,母亲,不要唤她,呃,不要唤她$\cdots{}\cdots{}$

哎呀,糟了。

在。

在这里。

哎呀,母亲呐!孩儿一时难以分辩,恐她后面自、自、自$\cdots{}\cdots{}$自尽去了!

\setlength{\hangindent}{56pt}{【{\akai 西皮散板}】大不该在桑田调戏她呀。 }

{\vspace{3pt}{\centerline{{[}{\hei 第四场}{]}}}\vspace{5pt}}

啊娘子,千不是,万不是,俱是卑人的不是,我这厢赔礼了。

呵呵哈哈哈$\cdots{}\cdots{}$({\hwfs 笑介})

老娘!

\setlength{\hangindent}{56pt}{【{\akai 西皮快板}】老娘亲息怒容儿禀,水有源流树有根:~孩儿打马桑田进,夫妻们见面我认、是认也认不清({\akai 或}:~认不真)。千不是来万不正,情愿上前我赔个小心。 }

\setlength{\hangindent}{56pt}{【{\akai 西皮摇板}】走向前来把礼敬, }

啊娘子,方才在桑田是卑人的不是,喏喏喏,我这厢赔礼了。

啊娘子,卑人这厢就又赔礼了。

诶。

\setlength{\hangindent}{56pt}{【{\akai 西皮快板}】{佯偢不睬}\footnote{``偢''同``瞅'',``睬''同``倸'';《京剧汇编》第八十七集~臧岚光~藏本作``佯愀不睬''。}人上人。是是是,明白了,想是道我礼貌轻。我这里上前忙跪定, }

诶。

\setlength{\hangindent}{56pt}{【{\akai 西皮快板}】背转身来自思忖。常言道男儿膝下有黄金,岂肯低头跪妇人。 }

是。

\setlength{\hangindent}{56pt}{【{\akai 西皮快板}】母亲教训儿遵命,秋胡岂是不孝的人。二次向前屈膝跪, }

我这一条腿么,一路之上,受尽风霜,不跪也罢。

哎,不错不错,是得了风寒之症了。

哦,母亲会治。

哎哟!

\setlength{\hangindent}{56pt}{【{\akai 西皮快板}】尊一声娘子听分明:~老母多亏你孝敬,一来赔罪二谢恩。 }

母亲不要跪。

呵呵哈哈哈$\cdots{}\cdots{}$({\hwfs 笑}{\hwfs 介})

\setlength{\hangindent}{56pt}{【{\akai 西皮摇板}】多谢娘子开了恩,再谢老娘讲人情。 }

\setlength{\hangindent}{56pt}{【{\akai 西皮摇板}】一家骨肉团圆庆,只享荣华不受贫。 }

啊母亲,孩儿挣来官诰,母亲请来穿戴。

是。

正是:~({\akai 念})秋胡离家二十春,

(秋母\hspace{30pt}({\akai 念})盼得为娘两眼穿\footnote{夏行涛{\scriptsize 君}建议作``两眼昏''。}。)

(罗敷\hspace{30pt}({\akai 念})今日夫妻重相见,)

({\akai 念})只享荣华不受贫。

母亲。

来了。

回来。

我来问你,方才在桑田,呃,是我的不是呀。你不该,在母亲面前搬动是非,教我罚跪在堂前。幸喜无人前来呀,倘若有人看见,呃,成何体统啊?

呃,若是不看在母亲的份上呢\footnote{刘曾复先生所有的``呢''念``嗫(nie)''音。}?

哼哼,还是这样的生气呀。

我若是不看在母亲的份上啊,我早就------跪下了。

呵呵哈哈哈$\cdots{}\cdots{}$({\hwfs 笑介})

哈哈哈$\cdots{}\cdots{}$({\hwfs 笑介})

哎呀,列位呀,不要见笑哇,这是我秋胡的家规哟。
}

\vspace{10pt}
{\hei 附注}:~

文献\cite{Xu-MasterThesis}和吴焕老师整理的剧本都曾提到,刘曾复先生本剧的开场的唱词来源于石君宝的元杂剧《鲁大夫秋胡戏妻》第三折:~

``(秋胡{\hwfs 冠带上},{\akai 云})小官秋胡是也。自当军去,见了元帅,道我通文达武,甚是见喜,在他麾下,累立奇功,官加中大夫之职。小官诉说,离家十年,有老母在堂,欠缺侍养,乞赐给假还家。谢得鲁昭公可怜,赐小官黄金一饼,以充膳母之资。如今衣锦荣归,见母亲走一遭去。

({\akai 诗云})想当日哭啼啼远去从军,今日个笑吟吟荣转家门。捧着这赤资资黄金奉母,安慰了我那娇滴滴年少夫人。({\hwfs 下})''
