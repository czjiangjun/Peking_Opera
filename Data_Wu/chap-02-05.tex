\newpage
\phantomsection %实现目录的正确跳转
\section*{\large\hei {困曹府}~\protect\footnote{根据刘曾复先生为吴小如先生说戏总讲录音底本整理。  陈超老师注:~刘曾复先生学的《困曹府》路子是程长庚弟子、三庆班沈三元的路子,沈是谭鑫培的把兄弟,因此《困曹府》的传承不是谭派。而另外一路是李顺亭的,再往下传就是贯大元。}}  
\addcontentsline{toc}{section}{\hei 困曹府}

\hangafter=1                   %2. 设置从第1⾏之后开始悬挂缩进  %}
\setlength{\parindent}{0pt}{

{\vspace{3pt}{\centerline{{[}{\hei 第一场}{]}}}\vspace{5pt}}

\setlength{\hangindent}{52pt}{曹彬\hspace{30pt}({\akai 念})跳过虎穴龙潭,({\akai 或}:~$\cdots{}\cdots{}$虎穴,)}

\setlength{\hangindent}{52pt}{赵匡胤\hspace{20pt}({\akai 念})好似凤鹤腾空。({\akai 或}:~逃出天罗地网。)}

\setlength{\hangindent}{52pt}{曹彬\hspace{30pt}仁兄请。}

\setlength{\hangindent}{52pt}{赵匡胤\hspace{20pt}请。}

\setlength{\hangindent}{52pt}{曹彬\hspace{30pt}请坐。}

\setlength{\hangindent}{52pt}{赵匡胤\hspace{20pt}有座。}\footnote{陈超老师注:~此处台上摆骑马桌。}

\setlength{\hangindent}{52pt}{赵匡胤\hspace{20pt}适才关前多蒙贤弟阖家({\akai 或}:~全家)搭救,愚兄当面谢过。}

\setlength{\hangindent}{52pt}{曹彬\hspace{30pt}岂敢,小弟当报前恩。}

\setlength{\hangindent}{52pt}{赵匡胤\hspace{20pt}好说。}

\setlength{\hangindent}{52pt}{曹彬\hspace{30pt}桌案现有酒食,仁兄请来压惊。}

\setlength{\hangindent}{52pt}{赵匡胤\hspace{20pt}讨扰了。}

\setlength{\hangindent}{52pt}{曹彬\hspace{30pt}仁兄请。}

\setlength{\hangindent}{52pt}{(赵匡胤\hspace{20pt}请。)}

\setlength{\hangindent}{52pt}{曹彬\hspace{30pt}【{\akai 二黄摇板}】知恩不报非君子,忘恩负义是小人。({\akai 或}:~酒不醉人人自醉,色不迷人人自迷。)}

\setlength{\hangindent}{52pt}{赵匡胤\hspace{20pt}二相公,贤弟。({\akai 或}:~贤弟,二相公。)}

\setlength{\hangindent}{52pt}{(赵匡胤\hspace{20pt}再饮几杯。)}

\setlength{\hangindent}{52pt}{赵匡胤\hspace{20pt}睡着了。}

\setlength{\hangindent}{52pt}{({\hwfs 起初更})}

\setlength{\hangindent}{52pt}{赵匡胤\hspace{20pt}唉!想俺玄郎,今晚被困曹府({\akai 或}:~夜困曹府),好不焦虑人也------}

\setlength{\hangindent}{52pt}{赵匡胤\hspace{20pt}【{\akai 二黄慢板}】有豪杰在书房心神不爽,二相公心无事睡卧一旁。无奈何推吊窗观看月亮,真乃是中秋节({\akai 或}:~真乃是中秋夜}\footnote{段公平{\scriptsize 君}建议作``中秋月''。}{),星明月朗、轮月皎皎分外风光。}

\setlength{\hangindent}{52pt}{赵匡胤\hspace{20pt}【{\akai 二黄原板}】我本是宦门后娇生惯养,闯关东、走关西自逞豪强。思爹娘、想妹弟终朝悬望,山又高、水又深阻隔两厢。洒金桥遇苗顺曾把命讲,他算我到后来南面称王。周文王坐江山全凭姜尚,保周朝八百载国祚绵长。汉光武仗云台二十八将,文邓禹、武姚期、马武子张。小秦王收下了瓦岗诸将,有罗成锁五龙图霸称强。俺玄郎逃灾祸东西游荡,孤一身({\akai 或}:~独一身)并无有架海金梁。到如今坐江山全然不想,全然不想,登九五如南柯大梦一场。恨金鸡不报晓天光未亮,谯楼上睡着了打更儿郎。恨不得抛长枪刺落了天边月亮,用金钩钩出了红日轮光。}

\setlength{\hangindent}{52pt}{({\hwfs 起二更},张氏{\hwfs 上})}

\setlength{\hangindent}{52pt}{张氏\hspace{30pt}【{\akai 二黄摇板}】轻移莲步出房门,窗外且听他人云。}

\setlength{\hangindent}{52pt}{张氏\hspace{30pt}奴家张氏,适才关前搭救恩人,不知他有何言语,待奴细听一番。}

\setlength{\hangindent}{52pt}{赵匡胤\hspace{20pt}且住,适才关前,多蒙张氏嫂嫂,叫了我一声``丈夫'',真乃难得呀难得!({\akai 或}:~多蒙张氏嫂嫂搭救,思想起来,真是难得呀难得!)}

\setlength{\hangindent}{52pt}{赵匡胤\hspace{20pt}哎------(呀)!说什么难得------倘若(是)我那贺氏妻子,叫(道旁)人一声``丈夫'',我就是这一刀------}

\setlength{\hangindent}{52pt}{张氏\hspace{30pt}唉呀!}

\setlength{\hangindent}{52pt}{赵匡胤\hspace{20pt}唉呀,醉了!(呃,)醉了!}

\setlength{\hangindent}{52pt}{赵匡胤\hspace{20pt}【{\akai 二黄散板}】窗里窗外隔窗棂,窗里说话窗外听。窗里之人吃酒醉,}

\setlength{\hangindent}{52pt}{赵匡胤\hspace{20pt}醉了哇!醉了!(呜呜呜$\cdots{}\cdots{}$({\hwfs 吐介}))}

\setlength{\hangindent}{52pt}{赵匡胤\hspace{20pt}【{\akai 二黄散板}】窗外休听醉汉云。}

\setlength{\hangindent}{52pt}{赵匡胤\hspace{20pt}呜呜呜$\cdots{}\cdots{}$(吐{\hwfs 介})}

\setlength{\hangindent}{52pt}{张氏\hspace{30pt}【{\akai 二黄散板}】听一言来吃一惊,羞得奴家脸带红。腰间解下丝鸾带,不如一命丧残生。}

\setlength{\hangindent}{52pt}{({\hwfs 起三更},华佗{\hwfs 上}}\footnote{陈超老师注:~华佗{\hwfs 抱宝剑上}。}{)}

\setlength{\hangindent}{52pt}{华佗\hspace{30pt}【{\akai 二黄摇板}】灵霄领了玉帝命,曹府搭救赤须龙。}

\setlength{\hangindent}{52pt}{华佗\hspace{30pt}({\akai 念})闷坐松林下,修道数百年。三国我为首,自称华佗仙。}

\setlength{\hangindent}{52pt}{华佗\hspace{30pt}今有赤须龙有难,奉了玉帝敕旨,下凡搭救。来此已是曹府,不免进府寻找。}

\setlength{\hangindent}{52pt}{华佗\hspace{30pt}原来星君在此,待我用起功来。}

\setlength{\hangindent}{52pt}{华佗\hspace{30pt}({\akai 念})不用急来不用愁,真龙天子百灵佑。宝剑一挥龙瘤落,丢入长江顺水流。}

\setlength{\hangindent}{52pt}{华佗\hspace{30pt}且喜大功成就,我不免趁此机会,讨一封号。}

\setlength{\hangindent}{52pt}{华佗\hspace{30pt}参见圣上。}

\setlength{\hangindent}{52pt}{赵匡胤\hspace{20pt}(呃------)何处妖道,在此摆来摆去?}

\setlength{\hangindent}{52pt}{华佗\hspace{30pt}小道三国华佗。}

\setlength{\hangindent}{52pt}{赵匡胤\hspace{20pt}前来则甚?}

\setlength{\hangindent}{52pt}{华佗\hspace{30pt}前来讨封。}

\setlength{\hangindent}{52pt}{赵匡胤\hspace{20pt}修炼多少年了?}

\setlength{\hangindent}{52pt}{华佗\hspace{30pt}千年有余。}

\setlength{\hangindent}{52pt}{赵匡胤\hspace{20pt}嗯,可算得一洞老神仙了。}

\setlength{\hangindent}{52pt}{华佗\hspace{30pt}谢主隆恩。正是:~({\akai 念})不是天子隆恩}\footnote{段公平{\scriptsize 君}建议作``天赐隆恩''。}{重,焉得一洞老神仙。}

\setlength{\hangindent}{52pt}{({\hwfs 起四更},张氏{\hwfs 魂上})}

\setlength{\hangindent}{52pt}{张氏\hspace{30pt}({\akai 念})人死如灯灭,犹如汤浇雪。若得回阳转,水底捞明月。}

\setlength{\hangindent}{52pt}{张氏\hspace{30pt}奴家张氏阴魂是也,是奴一时不明,悬梁自尽,死后方知恩人乃是当今真龙天子,不免趁此机会,前去讨一封号。}

\setlength{\hangindent}{52pt}{张氏\hspace{30pt}参见圣上。}

\setlength{\hangindent}{52pt}{赵匡胤\hspace{20pt}啊------({\akai 或}:~呃------)何处冤鬼,在此摆来摆去?}

\setlength{\hangindent}{52pt}{张氏\hspace{30pt}冤鬼张氏。}

\setlength{\hangindent}{52pt}{赵匡胤\hspace{20pt}前来则甚?}

\setlength{\hangindent}{52pt}{张氏\hspace{30pt}前来讨封。}

\setlength{\hangindent}{52pt}{赵匡胤\hspace{20pt}呃,前者({\akai 或}:~前番)路过华山,少一圣母({\akai 或}:~缺一圣母),封你以为华山圣母之位({\akai 或}:~封你为插花圣母}\footnote{段公平{\scriptsize 君}注:~汉调二黄《水西门》剧本~(``割瘤讨封''是其中{[}第十场{]})~作``插花圣母'',豫剧似乎也是这个名字。插花圣母,不供香火,插花为供。今浙江云和龙门村瓯江边有插花殿,据传有``敕封护国插花圣母娘娘''匾。})。

\setlength{\hangindent}{52pt}{赵匡胤\hspace{20pt}金童、玉女何在?}

%金童\\玉女\footnote{陈超老师注:~金童、玉女上时,金童打幡,玉女捧托盘凤冠。}\raisebox{5pt}{\hspace{30pt}有何旨意?}
\raisebox{0pt}[22pt][16pt]{\raisebox{8pt}{金童}\raisebox{-8pt}{\hspace{-22pt}{玉女}}\raisebox{0pt}{\hspace{30pt}有何旨意?\footnote{陈超老师注:~金童、玉女上时,金童打幡,玉女捧托盘凤冠。}}}

\setlength{\hangindent}{52pt}{赵匡胤\hspace{20pt}护送圣母归位去者。({\akai 或}:~护送圣母归位,去罢。)}

%金童\\玉女\raisebox{5pt}{\hspace{30pt}领法旨。}
\raisebox{0pt}[22pt][16pt]{\raisebox{8pt}{金童}\raisebox{-8pt}{\hspace{-22pt}{玉女}}\raisebox{0pt}{\hspace{30pt}领法旨。}}

\setlength{\hangindent}{52pt}{张氏\hspace{30pt}谢主隆恩。}

\vspace{3pt}{\centerline{{[}{\hei 第二场}{]}}}\vspace{5pt}

\setlength{\hangindent}{52pt}{({\hwfs 起五更},曹小姐{\hwfs 上})}

\setlength{\hangindent}{52pt}{曹小姐\hspace{20pt}({\akai 念})忙将嫂嫂事,报与兄长知。}

\setlength{\hangindent}{52pt}{曹小姐\hspace{20pt}兄长快些醒来,大事不好了!}

\setlength{\hangindent}{52pt}{曹彬\hspace{30pt}何事惊慌?}

\setlength{\hangindent}{52pt}{曹小姐\hspace{20pt}嫂嫂悬梁自尽了!}

\setlength{\hangindent}{52pt}{曹彬\hspace{30pt}哦,待我观看。}

\setlength{\hangindent}{52pt}{曹彬\hspace{30pt}唉,嫂嫂啊$\cdots{}\cdots{}$({\hwfs 哭介})}

\setlength{\hangindent}{52pt}{曹彬\hspace{30pt}兄长醒来!}

\setlength{\hangindent}{52pt}{赵匡胤\hspace{20pt}【{\akai 二黄散板}】插花圣母归了位,三国华佗讨封回。猛然睁开丹凤眼,}

\setlength{\hangindent}{52pt}{曹彬\hspace{30pt}嫂嫂哇啊$\cdots{}\cdots{}$({\hwfs 哭介})}

\setlength{\hangindent}{52pt}{赵匡胤\hspace{20pt}【{\akai 二黄摇板}】贤弟缘何两泪垂?}

\setlength{\hangindent}{52pt}{曹彬\hspace{30pt}唉呀兄长,嫂嫂悬梁自尽了哇$\cdots{}\cdots{}$({\hwfs 哭介})}

\setlength{\hangindent}{52pt}{(赵匡胤\hspace{20pt}愚兄不信。)}

\setlength{\hangindent}{52pt}{赵匡胤\hspace{20pt}今在何处?}

\setlength{\hangindent}{52pt}{曹彬\hspace{30pt}随我来!}

\setlength{\hangindent}{52pt}{赵匡胤\hspace{20pt}唉!嫂嫂啊$\cdots{}\cdots{}$({\hwfs 哭介})}

\setlength{\hangindent}{52pt}{赵匡胤\hspace{20pt}啊贤弟,不必悲泪({\akai 或}:~休得悲恸),嫂嫂成仙去了。}

\setlength{\hangindent}{52pt}{曹彬\hspace{30pt}但愿如此。}

\setlength{\hangindent}{52pt}{家人\hspace{30pt}报!}

\setlength{\hangindent}{52pt}{家人\hspace{30pt}崔龙带兵围困府门,请二老爷答话。}

\setlength{\hangindent}{52pt}{曹彬\hspace{30pt}起过。}

\setlength{\hangindent}{52pt}{曹彬\hspace{30pt}兄长暂且回避。}

\setlength{\hangindent}{52pt}{赵匡胤\hspace{20pt}是。}

\setlength{\hangindent}{52pt}{曹彬\hspace{30pt}带路。}

\setlength{\hangindent}{52pt}{曹彬\hspace{30pt}请了。崔将军有何见谕?}

\setlength{\hangindent}{52pt}{崔龙\hspace{30pt}圣上有旨:~命大人将过关人犯,带上金殿审问。}

\setlength{\hangindent}{52pt}{曹彬\hspace{30pt}将军人马暂退一箭之地,待弟将人犯戴上刑具,一同上殿交旨。}

\setlength{\hangindent}{52pt}{崔龙\hspace{30pt}大人休得迟慢,请!}

\setlength{\hangindent}{52pt}{曹彬\hspace{30pt}有请仁兄。}

\setlength{\hangindent}{52pt}{赵匡胤\hspace{20pt}贤弟何事?}

\setlength{\hangindent}{52pt}{曹彬\hspace{30pt}今有崔龙,带领人马,要将仁兄押上金殿见驾。}

\setlength{\hangindent}{52pt}{赵匡胤\hspace{20pt}就依贤弟。}

\setlength{\hangindent}{52pt}{曹彬\hspace{30pt}仁兄受屈了。}

\setlength{\hangindent}{52pt}{赵匡胤\hspace{20pt}啊贤弟,今日何日?}

\setlength{\hangindent}{52pt}{曹彬\hspace{30pt}中秋佳节。}

\setlength{\hangindent}{52pt}{赵匡胤\hspace{20pt}明日呢?}

\setlength{\hangindent}{52pt}{曹彬\hspace{30pt}乃是十六日。}

\setlength{\hangindent}{52pt}{赵匡胤\hspace{20pt}明日,就是我光棍家出头之日了。({\akai 或}:~明日中秋,就是我光棍家出头之日了。)}

\setlength{\hangindent}{52pt}{曹彬\hspace{30pt}想你这光棍家,还有什么根本不成?}

\setlength{\hangindent}{52pt}{赵匡胤\hspace{20pt}贤弟------}

\setlength{\hangindent}{52pt}{赵匡胤\hspace{20pt}【{\akai 二黄碰板}】休道我光棍家根本不讲,请台座听玄郎细说端详:~}

\setlength{\hangindent}{52pt}{赵匡胤\hspace{20pt}【{\akai 二黄原板}】家住在西罗县\footnote{据史载,赵匡胤出生于河南洛阳夹马营。但很多地方戏曲都传说赵匡胤是``西罗县''(约在今山西洪洞县境内)。刘曾复先生有一版说戏录音中音近似``西蒙县'',可能是``西罗县''的讹误。}双龙街上,本姓赵名匡胤字表玄郎。头辈祖名赵暠家财颇广,二辈祖名赵霸({\akai 或}:~二辈祖名赵强;~二辈祖名淮庆)四海名扬。三辈祖名赵强({\akai 或}:~三辈祖名赵霸)隋唐为将,子不言父的名四品黄堂\footnote{``黄堂''是古代太守衙门中的正堂,可借指太守职位。  关于赵匡胤的家世,《残唐五代史演义》中称赵匡胤的祖父为赵霸,霸生弘殷,弘殷生匡胤;~据《宋史·太祖本纪》载``太祖启运立极英武睿文神德圣功至明大孝皇帝,讳匡胤,姓赵氏,涿郡人也。高祖朓,是为僖祖,仕唐历永清、文安、幽都令。朓生珽,是为顺祖,历藩镇从事,累官兼御史中丞。珽生敬,是为翼祖,历营、蓟、涿三州刺史。敬生弘殷,是为宣祖。周显德中,宣祖贵,赠敬左骁骑卫上将军$\cdots{}\cdots{}$太祖,宣祖仲子也,母杜氏。后唐天成二年,生于洛阳夹马营,赤光绕室,异香经宿不散。体有金色,三日不变。既长,容貌雄伟,器度豁如,识者知其非常人。''}。生下了俺玄郎面带奇相,酒醉后杀御乐\footnote{段公平{\scriptsize 君}注:~{据{[}清{]}吴}璿{《飞龙全传》,赵闹御院(勾栏),打女乐,后离家出逃。``杀御乐''一句即谓此。汉调二黄有``}悔不该将女乐满门杀坏{''。}}惹下祸殃。二爹娘修书信四路探望,遇柴荣和郑恩关西道旁({\akai 或}:~关西路旁)。我三人尧王庙同把香上,要学那三国中刘备、关、张。董家桥打五虎弟兄各往,柴大哥到怀庆受爵封王。好一个柴子耀不把友忘,差旗牌带书信迎接玄郎。弟兄们同饮酒花亭以上,久分手又相会畅叙衷肠。一霎时({\akai 或}:~顷刻间)鱼池内陡起风浪,吓坏了王府人俱各({\akai 或}:~吓坏了王府中个个)惊慌。大哥说鱼戏水常来常往,俺玄郎见妖魔甚是张狂。左挽弓来右搭箭照妖发放({\akai 或}:~对妖撒放;~照妖撒放),谯楼上打三筹鼓角凄凉。次日里兄带我朝见皇上,郭王爷想起了梦中箭伤。顷刻间({\akai 或}:~一霎时)传旨意将我捆绑,柴大哥奏一本({\akai 或}:~保一本)刺杀刘王。悔不该({\akai 或}:~最不该)在金殿海口夸讲,不用兵不用将独下燕邦({\akai 或}:~独上燕邦)。走西门({\akai 或}:~在西门)遇见了崔龙老将,多亏你阖家搭救玄郎。昨夜晚同饮酒桌案之上({\akai 或}:~桌案以上),俺玄郎得二梦\footnote{刘曾复先生有一版录音作``得一梦'',似误。}牢记心旁:~头一梦见华佗三国道长,用宝剑割龙瘤丢入长江。二贤弟你不信观看左膀,}

\setlength{\hangindent}{52pt}{曹彬\hspace{30pt}【{\footnotesize 接}{\akai 二黄原板}】果然是无肉瘤毫无损伤。}

\setlength{\hangindent}{52pt}{赵匡胤\hspace{20pt}【{\akai 二黄原板}】第二梦见张氏嫂嫂形象,项带锁披着发珠泪汪洋。我封她为圣母在华山以上,}

\setlength{\hangindent}{52pt}{赵匡胤\hspace{20pt}【{\akai 二黄散板}】有金童和玉女送至山岗({\akai 或}:~送上山岗)。}

\setlength{\hangindent}{52pt}{赵匡胤\hspace{20pt}【{\akai 二黄散板}】劝贤弟在此间休要来往,搬至在怀庆府可作家乡。柴大哥是玄郎结义兄长,大小事他必然念在玄郎。天牢内二双亲劳你探望({\akai 或}:~劳你看望),说玄郎成了功即刻还乡。}

\setlength{\hangindent}{52pt}{赵匡胤\hspace{20pt}【{\akai 二黄散板}】二贤弟将手杻与我戴上,}

(\textless{}\!{\bfseries\akai 阴锣}\!\textgreater{}崔龙{\hwfs 人马两边上},{\hwfs 巡查};~赵匡胤{\hwfs 换衣}、{\hwfs 戴手杻},{\hwfs 上})

\setlength{\hangindent}{52pt}{赵匡胤\hspace{20pt}【{\akai 二黄散板}】此一番上金殿我杀一个倒海翻江。}

\setlength{\hangindent}{52pt}{赵匡胤\hspace{20pt}【{\akai 二黄散板}】施一礼辞贤弟出门观望({\akai 或}:~入门观望),见崔龙人和马个自逞强({\akai 或}:~个个逞强)。}

\setlength{\hangindent}{52pt}{赵匡胤\hspace{20pt}【{\akai 二黄散板}】真和假是与非见尔主上,俺曹仁不犯法又有何妨?}

}

\newpage
\hangafter=1                   %2. 设置从第1⾏之后开始悬挂缩进  %}
\setlength{\parindent}{0pt}{
附录:~\\

汉剧《困曹府》据~何鸣峰~1981年~吊嗓录音(廖占魁~操琴;~杨小楼~司鼓)

\setlength{\hangindent}{52pt}{赵匡胤 \hspace{20pt}【{\akai 二黄三眼}】休道我官宦家根本无有,待愚兄把从前事细说根由:~({\akai 接唱})【{\akai 二流}】二爹娘生玄郎肉丘本有,抽着签卜着课相带冕旒。吃醉酒杀御乐一十七口,二双亲命豪杰四路奔走。棋盘坡绿叶岭遇着二友。柴大哥贩雨伞鲁郑恩卖油。我三人秦王庙捻香罚咒,胜似那亲父母同胞的骨肉。董家桥打五虎弟兄分手,柴大哥上怀庆官封王侯。好一个仁义兄不忘旧友,修书信接玄郎满门去投。弟兄们相会在花厅饮酒,鱼池内怪风起令人担忧,兄讲道鱼儿戏水世上本有。俺玄郎怕的是妖魔起头。左拉弓右搭箭随风射走,耳听得谯楼上鼓打三筹。次日里随柴兄上殿面奏,郭王爷见了兄忆起梦中之仇。兄一言并未发推出斩首。多亏了柴大哥再三保留。悔不该在金殿夸下了海口,不用兵不用将俺独下燕州。走西门被拉住亏你搭救,兄就有劳你搭救。

一家人把玄郎当作骨肉,昨夜晚与贤弟书房饮酒。睡梦间得二梦牢记心头:~兄第一梦梦见了那华佗老祖,用神剑割去了兄的朱砂肉丘。二相公你不信——来来来,兄解衣你来照透,

\setlength{\hangindent}{52pt}{曹杰}({\akai 接唱})果然是割去了朱砂肉丘。

\setlength{\hangindent}{52pt}{ 赵匡胤}({\akai 接唱})兄第二梦又只见张氏嫂嫂,披着发项带绳苦苦哀奏。兄封她插花圣母到幽冥地走,有金童和玉女送上瀛洲。好二弟兄劝你在此地休得要久守,倒不如一家人搬上怀州。在朝里柴王爷是兄好友,提起了俺玄郎将你收留。天牢内二爹娘托你问候,兄就托你问候,你就说俺成了功不久回头。

\setlength{\hangindent}{52pt}{ 赵匡胤}({\akai 接唱})耳旁内又听得人声马吼,想必是崔龙贼二把你的府搜。好贤弟看过了铁链手杻。常言道祸到临头我不能自由。
}
