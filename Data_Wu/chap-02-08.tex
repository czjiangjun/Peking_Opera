\newpage
\phantomsection %实现目录的正确跳转
\section*{\large\hei {甘露寺~{\small 之}~乔玄}}
\addcontentsline{toc}{section}{\hei 甘露寺~{\small 之}~乔玄}

\hangafter=1                   %2. 设置从第1⾏之后开始悬挂缩进  %}
\setlength{\parindent}{0pt}{

{\centerline{{[}{\hei 第一场}{]}}}\vspace{5pt}

(刘备\hspace{30pt}看------江水波涛,水天一色,好一派江景也!)

\setlength{\hangindent}{52pt}{(刘备\hspace{30pt}【{\akai 西皮原板}】看长江白茫茫银蛇滚滚,水与天共一色白浪纷纷。回头来再对四弟论,此一番到东吴见机而行。) }

$\cdots{}\cdots{}$

(刘备\hspace{30pt}四弟,准备厚礼。明日你我君臣前去拜访。)

(刘备\hspace{30pt}正是:~({\akai 念})来到东吴地,)

(赵云\hspace{30pt}({\akai 念})先去见乔玄。)

\vspace{3pt}{\centerline{{[}{\hei 第二场}{]}}}\vspace{5pt}

{呃------哼!~({\hwfs 内嗽介})}

{[}{\akai {\akai 引}子}{]}丹心镇国,辅君王,社稷安康。

({\akai 念})天子渊源重老臣,为子孝亲臣奉君。皇图永固民安乐,但愿东吴万万春。

老夫------乔玄,字嵩山,乃江东人氏。吴侯驾前为臣,官居首相,执掌江东十二内阁。夫人姜氏,膝下无儿,所生二女,长女大乔,许配孙策;次女小乔,配许周郎。适才朝罢归来,见街市之上,悬灯结彩;府下人等,一个个交头接耳,也不知他们说些甚么。

啊------家院。

老夫问你话呀。

老夫问你话呀。

适才老夫朝罢而归,见街市之上,悬灯结彩;府下人等,一个个交头接耳,不知他们说些么?

孙、刘两家结亲?呃,怎么老夫一些儿也不晓得呀!

哦,有这等事?老夫当朝首相,怎么一些儿也不知呢。

({\hwfs 思忖介})既是刘皇叔过江,也该前来拜拜老夫啊。

好好好,你且门上伺候!

(刘备\hspace{30pt}({\akai 念})身在东吴地,)

(赵云\hspace{30pt}({\akai 念})昼夜费心机。)

哦,果然来了。动乐有请。

啊------皇叔!

过江来了。

啊------呵呵呵哈哈哈$\cdots{}\cdots{}$({\hwfs 笑介})

请------

请坐。

皇叔驾到,蓬荜生辉。老朽有失远迎,望祈恕罪。

岂敢。

哦,罢了。

皇叔,此位是------

哦?!这就是在长坂坡前救幼主的子龙将军么?

真乃是虎将也。

哎呀呀,老朽怎敢受礼,万难从命。

呃,不、不,不敢收啊。

哎,老夫还未曾吩咐,你怎么就收下了?

好不中用!\\

呃呃呃,皇叔,如此我愧领了!

为何去心太急?

是啊,他们那里也该走走啊,只是老朽未得领教。

另日奉迎。

好,送客。

呃------嗯,皇叔到此,乃是贵客,我不肯收他的礼物,怎么你就大胆地收下了啊?

有道是:~无功不受禄哇。

呃,我功在哪里?

呵呵呵,你这老狗才的话,倒也中听。

呵!

哎呀且住!刘备既已过江,孙、刘两家若能结亲,一同出兵,共敌曹操,与我东吴大大有利。我不免进宫,与太后贺喜。

来,吩咐外厢打道进宫!

\vspace{3pt}{\centerline{{[}{\hei 第三场}{]}}}\vspace{5pt}

{呃------哼!~({\hwfs 内嗽介})}

({\akai 念})天上生瑞彩,人间配鸾凰。

来此已是,待我叩环!

乔玄求见。

领旨。

臣------乔玄见驾,国太千岁!

千千岁。

谢座。

呃,恭喜太后,贺喜太后!

太后将郡主招赘刘备,岂不是一喜?

这样的大事,太后不知,谁敢作主?

({\hwfs 思介})呃------莫非二千岁$\cdots{}\cdots{}$主意。

领旨!

太后有旨,二千岁进宫!

老臣参驾。

谢座。

太后醒来!

啊,千岁,若用此计,岂不被旁人耻笑么?

怎么?!又是周郎?

唉!他明明是害你呀。

呵呵$\cdots{}\cdots{}$我多口,多口哇$\cdots{}\cdots{}$

(孙权   【{\akai 西皮原板}】$\cdots{}\cdots{}$誓不休!)

千岁!

\setlength{\hangindent}{56pt}{【{\akai 西皮原板}】劝千岁杀字休出口,细听老臣说根由:~那刘备他本是------【{\footnotesize 转}{\akai 西皮二六}】靖王后,【{\akai 西皮快板}】汉帝玄孙一脉流。他有个二弟关羽汉寿亭侯,青龙偃月神鬼皆愁。他斩颜良、诛文丑,古城又斩蔡阳的头。他三弟翼德性情有\footnote{``性情有''也有唱``性情拗''的。},丈八蛇矛惯取咽喉。虎牢关前来争斗,枪挑金冠战温侯。当阳桥前一声吼,喝断桥梁水倒流。他四弟赵云常山将,盖世英名贯九州。长坂坡,救阿斗,杀得曹兵个个愁。这班武将哪国有?还有诸葛运计谋。杀了刘备不要紧,荆州岂肯来罢休?若是兴兵来争斗,曹操坐把渔利收。扭转回身启太后,老臣言来听从头:~龙凤{\footnotesize 呃}呈祥天造就,将计就计结鸾俦。

太后,(孙、刘若能结亲,\footnote{此处刘曾复先生录音疑似有缺失,根据《马连良演出剧本选集》\upcite{Ma_Opera-Collect}添加。})一同出兵,共灭曹操,与我东吴大大有利,不可失此机会也。

可以配得。

国太若相得上?

啊,太后,那刘备乃英雄之相,不相也罢。

领旨!

\vspace{3pt}{\centerline{{[}{\hei 第四场}{]}}}\vspace{5pt}

唉!明日太后在甘露寺面相刘备,我想刘备须发苍白,太后若相他不上,必被周郎所害,唉呀,这这这这$\cdots{}\cdots{}$

唉!他人闲事,不管也罢呀。

呃------都是你这个老狗才,我不肯收他礼物,你就大胆地收下了,如今岂不叫老夫作难么?

是啊,总要想个计策才是啊------

哦,有了!

乔福过来。这有乌须药一匣,命你送到馆驿,面交刘皇叔,教他连夜将须发染黑,明日在甘露寺中一相就相上了,快去快去!

啊$\cdots{}\cdots{}$转来!

对刘皇叔去说:~明日席前,恐其有诈,命保驾将军内穿铠甲,外罩袍服,作一个``防而不备,备而不防''!

记下了。

快去快去。

这个老狗才。

唉$\cdots{}\cdots{}$从今以后,老夫再也不贪人家的小利了。

唉,这才是``不经一事,不长一智''哦!

\vspace{3pt}{\centerline{{[}{\hei 第五场}{]}}}\vspace{5pt}

刘皇叔到。

是。

啊------皇叔。

上面就是太后,见了就拜呀。

啊,太后,新姑老爷,总是要拜的。

皇叔,你要多拜几拜。

领旨。

太后有旨,二千岁上佛殿呐------

啊太后,可知皇叔的根基呀?

皇叔乃中山靖王之后,汉景帝陛下之玄孙,荆襄王刘表之堂弟,当今天子之皇叔。喏喏喏,太后请看------生得是龙眉凤目,两耳垂肩,双手过膝,真不愧是帝王的根本呐!

呃,帝王的根本!

说说也无妨啊!

哦,是,是,是。

啊太后,关美髯太后可晓得?

此人姓关名羽字云长,乃蒲州解良人也。弟兄桃园结义以来,在徐州失散,万般无奈,暂归曹营。那曹操待他十分恩厚,三日一小宴,五日一大宴,上马金、下马银,美女十名,俱不肯受哇。闻得皇叔有了下落,彼时挂印封金,在灞桥挑袍,过五关、斩六将,弟兄在古城相会。这位将军的义气------哼,不小哇!

好义气!

呃,虽不是我亲眼得见,谁人不知,呃呃,哪个不晓哇!

哦哦,好、好、好。

啊太后,张翼德太后可知?

此人姓张名飞字翼德,乃涿郡范阳人也。这位将军,在当阳桥前大吼一声,吓得曹操收去青龙伞,惊死夏侯杰。这位将军好威风,好煞气呀!

呃好威风,好煞气!

啊太后,赵子龙太后可晓得?

这位将军姓赵名云字子龙,乃真定常山人也。在长坂坡前与曹兵交战,杀入曹营,是七进七出!

呃不不不,七进七出!

七出七进,是七进七出啊!

呃本来是七进七出啊!

诸葛亮太后可知啊?这位先生,复姓诸葛名亮字孔明,道号卧龙,乃阳都人也。皇叔三顾茅庐,他是才得出山。这位先生在我东吴南屏山,高设一台,名曰七星祭风坛,借来三日三夜东风,烧退曹兵八十三万,好烧哇好烧!

领旨。

哪个的主意?

莫非二千岁?

太后有旨,二千岁上佛殿。

啊太后,我东吴有员大将,名叫贾化。

呃太后,新姑老爷讲情,总是要准的呀!

下去!

是。

太后,老臣眼力如何?

遵旨。

太后回宫。

带马------

}
