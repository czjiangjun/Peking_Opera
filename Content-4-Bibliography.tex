%----%%%%%%%%%%%%%%%%%%%%%--------The Biblography of The Paper--------%%%%%%%%%%%%%%%%%%%-----%
%\newpage										      %
%\phantomsection\addcontentsline{toc}{section}{Bibliography} %直接调用\addcontentsline命令可能导致超链指向不准确,一般需要在之前调用一次\phantomsection命令加以修正%
%\phantomsection\addcontentsline{toc}{subsection}{\CJKfamily{hei} 主要参考资料}               % 

%---------------------------------------------------------------------------------------------%
%
\newpage																				%
%-----------------------------------------------------------------------------------------------------------------------------------------------------------------------%
\thispagestyle{plain}
\bibliographystyle{/home/jun-jiang/Documents/Peking_Opera/Bibliography/mybib} %% 接近ieeert样式
%\bibliographystyle{/home/jun_jiang/Documents/Latex_art_beamer/ref/mybib} %% 接近ieeert样式
%\bibliographystyle{../ref/mybib} %% 接近ieeert样式
%--------------------------------------------------------------------------The Biblography of The Paper-----------------------------------------------------------------%
%\begin{thebibliography}{99}																		%
%%\bibitem{PRL58-65_1987}H.Feil, C. Haas, {\it Phys. Rev. Lett.} {\bf 58}, 65 (1987).											%
%	\bibitem{kp-method} \textrm{Zhenxi Pan, Yong Pan, Jun Jiang$^{\ast}$, Liutao Zhao}, \textrm{High-Throughput Electronic Band Structure Calculations for Hexaborides}, \textit{Intelligent Computing}, \textbf{Springer}, \textbf{P.386-395}, (2019).%
%	\bibitem{PAW-dataset} \textrm{姜骏},\textrm{PAW原子数据集的构造与检验}, \textit{中国化学会第十二届全国量子化学会议论文摘要集},\textbf{太原},(2014).
%\end{thebibliography}																			%
%-----------------------------------------------------------------------------------------------------------------------------------------------------------------------%
%\phantomsection\addcontentsline{toc}{section}{Bibliography} %直接调用\addcontentsline命令可能导致超链指向不准确,一般需要在之前调用一次\phantomsection命令加以修正%
\phantomsection
\addcontentsline{toc}{section}{\CJKfamily{hei} 主要参考资料} %直接调用\addcontentsline命令可能导致超链指向不准确,一般需要在之前调用一次\phantomsection命令加以修正%
%\bibliography{../ref/Myref_from_2013}   %
%\bibliography{Bibliography/Peking_Opera}   %
\bibliography{Bibliography/Peking_Opera}   %
\pagestyle{fancy}    %与文献引用超链接style有冲突
\chead{主~要~参~考~资~料} % 页眉中间位置内容
%\bibliography{/home/jun_jiang/Documents/Reports_Book/thuthesis-master/Peking_Opera/Peking_Opera}   %
%\bibliography{/home/jun-jiang/Documents/ref/Myref} %% 接近ieeert样式
%-----------------------------------------------------------------------------------------------------------------------------------------------------------------------%
%
%
%---------------------------------------------------------------------------------------------%
%\begin{thebibliography}{99}								      %
%											      %
%											      %
%--------%%%%%%%%%%%%%%%%%%%%%%%%%%%%%%%%%%%%%%%%%%%%%--------%
%%\bibitem{PRL58-65_1987}H.Feil, C. Haas, {\it Phys. Rev. Lett.} {\bf 58}, 65 (1987).         %
%\bibitem{kp-method} \textrm{Zhenxi Pan, Yong Pan, Jun Jiang$^{\ast}$, Liutao Zhao}, \textrm{High-Throughput Electronic Band Structure Calculations for Hexaborides}, \textit{Intelligent Computing}, \textbf{Springer}, \textbf{P.386-395}, (2019).              %
%\bibitem{QCQC_2014} \textrm{姜骏},\textrm{PAW原子数据集的构造与检验}, \textit{中国化学会第十二届全国量子化学会议论文摘要集},\textbf{太原},(2014).
%											      %
%\end{thebibliography}								              %

%
%\bibliography{../../ref/Myref}                         %
%%\bibliography{../../ref/Myref, ../ref/Myref_from_2013}        %% 多个参考文献 .bib
%\bibliographystyle{../../ref/mybib}                   %%   接近ieeert样式

%%%%%%%%%%%%%%%%%%%%%%%%%%%%      \bibliographystyle         %%%%%%%%%%%%%%%%%%%%%%%%%%%%%%%%%%
%%%%%%      LaTeX 参考文献标准选项及其样式共有以下8种:                                %%%%%%%%
% plain,按字母的顺序排列,比较次序为作者、年度和标题.                                        %
% unsrt,样式同plain,只是按照引用的先后排序.                                                 %
% alpha,用作者名首字母+年份后两位作标号,以字母顺序排序.                                     %
% abbrv,类似plain,将月份全拼改为缩写,更显紧凑.                                             %
% ieeetr,国际电气电子工程师协会期刊样式.                                                     %
% acm,美国计算机学会期刊样式.                                                                %
% siam,美国工业和应用数学学会期刊样式.                                                       %
% apalike,美国心理学学会期刊样式.                                                            %
%%%%%%%%%%%%%%%%%%%%%%%%%%%%%%%%%%%%%%%%%%%%%%%%%%%%%%%%%%%%%%%%%%%%%%%%%%%%%%%%%%%%%%%%%%%%%%%

%\nocite{*}                                                                                   %
%---%%%%%%%%%%%%%%%%%%%%%-------%%%%%%%%%%%%%%%%%%%%%%%%%%%%%--------%%%%%%%%%%%%%%%%%%%------%

